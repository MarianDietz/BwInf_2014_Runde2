\documentclass[a4paper,12pt,table]{article}
\usepackage[ngerman]{babel}
\usepackage{ucs}
\usepackage[utf8]{inputenc}
\usepackage{graphicx}
\usepackage{amsmath}
\usepackage{amssymb}
\usepackage{fancyhdr}
\usepackage{listings}
\usepackage{multicol}
%\usepackage[toc]{multitoc}
%\renewcommand*{\multicolumntoc}{2}
\usepackage{hyperref}
\hypersetup{
    colorlinks,
    citecolor=black,
    filecolor=black,
    linkcolor=Mahogany,
    urlcolor=black
}


\usepackage{setspace}
\setlength{\textwidth}{16cm}
\setlength{\textheight}{25.7cm}
\setlength{\topmargin}{-1.7cm}
\setlength{\evensidemargin}{-8mm}
\setlength{\oddsidemargin}{-8mm}
\parskip 6pt plus 1pt minus 1pt
\parindent0pt
\pagestyle{fancy}

\lhead{\bfseries BWINF '13/'14}
\chead{Philip Wellnitz}
\rhead{Einsendungsnummer (850.01)}
\lfoot{\leftmark}
\cfoot{}
\rfoot{\thepage}
\renewcommand{\headrulewidth}{0.5pt}
\renewcommand{\footrulewidth}{0.5pt}

\usepackage{csquotes}
\usepackage[usenames,dvipsnames,svgnames]{xcolor}
\definecolor{bluekeywords}{rgb}{0.13,0.13,1}
\definecolor{greencomments}{rgb}{0,0.5,0}
\definecolor{redstrings}{rgb}{0.9,0,0}
\lstset{language=C++,
numbers=left,
stepnumber=1,
showspaces=false,
showtabs=false,
breaklines=true,
showstringspaces=false,
breakatwhitespace=true,
escapeinside={(*@}{@*)},
commentstyle=\color{greencomments},
keywordstyle=\color{bluekeywords}\bfseries,
stringstyle=\color{redstrings},
basicstyle=\ttfamily,
moredelim=**[is][\color{brown}]{@}{@},
moredelim=**[is][\color{purple}]{§}{§},
moredelim=**[is][\color{Mahogany}]{§§}{§§},
moredelim=**[is][\color{greencomments}]{§§§}{§§§}
}
\lstset{literate=%
{Ö}{{\"O}}1
{Ä}{{\"A}}1
{Ü}{{\"U}}1
{ß}{{\ss}}2
{ü}{{\"u}}1
{ä}{{\"a}}1
{ö}{{\"o}}1
}



\usepackage{tikz}
\usetikzlibrary{matrix}
\usepackage{tkz-berge}

\newcommand{\emphpar}[1]{\emph{#1}\marginpar[#1]{#1}\index{#1}}
\newcommand{\emphparr}[1]{\emph{#1}\marginpar[#1]{#1}}

\title{32. Bundeswettbewerb Informatik, Runde 2\\ \small{Ausarbeitungen zu den Aufgaben \enquote{Buschfeuer} und \enquote{Lebenslinien} }}
\author{Philip Wellnitz}
\date{}

\begin{document}
\thispagestyle{empty}
\maketitle
\newpage
\section*{Vorwort}

Diese Welt ist voller Rätsel und Probleme.

Zwei von ihnen habe ich auf den folgenden Seiten versucht zu lösen,\\ wobei sich mir durchaus neue Probleme in den Weg stellten.

Keine sollten dagegen beim Starten meiner Programme aus einem Linux-artigen Terminal auftreten. \\
Es sollte jedoch speziell bei meinen Programm zur Lösung von Aufgabe 1 darauf geachtet werden, dass das Terminal ASCII-Escape-Sequenzen unterstützt, damit die ausgegebenen Kunstwerke aka. Lösungen auch richtig dargestellt werden können.

Diese Welt ist voller Rätsel und Probleme. \\
Viele\footnote{Von Zeit zu Zeit könnte ich an dieser Stelle auch ein \emph{Wenige} vertreten...} sind lösbar.


%Armer Korrektor!\footnote{Dieses Vorwort existiert nur, weil es scheinbar zur Struktur gehören muss. Außerdem wollte ich sowas auch mal schreiben...}\footnote{Und es ist essenziell, um meine ständige Präsenz in den \emph{Perlen der Informatik} zu wahren...}\\
%Auf den folgenden Seiten sind die vor meiner unendlichen Genialität strotzenden Lösungen der diesjährigen Zweitrundenaufgaben niedergeschrieben und zu bewundern. Die glorreichen Programme zu den einzelnen Aufgaben sollten sich auf einem normalen Rechner\footnote{Das sei im Folgenden ein Rechner mit einem Linux-artigen OS} aus einem Terminal problemlos starten lassen; speziell für Aufgabe 1 eignet sich besonders eines, welches die göttlichen, also mir gleichen, ASCII-Escape-Sequenzen unterstützt.\\
%Weiterhin habe ich (noch) darauf verzichtet, meine überragenden Programmierfähigkeiten in wunderschönen  Programmen im gut lesbaren \texttt{ASM} auszudrücken; auch habe ich kein \texttt{C++} mit \texttt{inline-ASM} verwendet\footnote{Schade eigentlich.}.\\
%Nach diesem mit übermäßiger Bescheidenheit glänzenden Vorwort möchte ich Ihnen nun viel Freude bei der Korrektur meiner Lösungen wünschen...\\
%und eine erholsame Zeit danach.
\tableofcontents
\newpage
\section{Aufgabe 1 - Buschfeuer}
\subsection{Lösungsidee}

Ein \emphpar{Feld} ist ein quadratisches Stück Land, welches genau einen folgender Zustände inne haben kann:
\begin{itemize}
\item[BRENNBAR] Das Stück Land ist ind der Lage, zu brennen.
\item[BRENNEND] Ein brennendes Stück Land.
\item[GELÖSCHT] Ein Stück Land, welches nie wieder brennen  wird.
\item[LEER] Ein leeres Stück Land.
\end{itemize}

Alle Felder haben die selbe Fläche.

Ein \emphpar{Wald} ist nun die rechteckig-gitterförmige Anordnung von $n\times m$ Feldern. Die \emphpar{Umgebung} $U(f)$ eines Feldes $f$ in einem Wald $W$ ist dabei die Menge an Feldern, welche in $W$ eine gemeinsame Kante mit $f$ haben.

Der Wald wird nun diskret beobachtet. Es ist dabei sichergestellt, dass nur sofern ein Feld bei einer Beobachtung brennend ist, dieses und jedes brennbare Feld seiner Umgebung bei der nächsten Beobachtung brennen werden, sofern diese nicht schon brennen. Diese Eigenschaft des Waldes sei mit \emph{Feuerausbreitung} bezeichnet.

Ab der 2. Beobachtung kann pro Beobachtung genau 1 (brennendes) Feld gelöscht werden. Wird ein brennendes Feld gelöscht, so fängt seine Umgebung bis zur nächsten Beobachtung nicht an zu brennen.\\
Die erste Beobachtung, ab der ein Feld $f$ brennt, heiße \emph{Entflammung} von $f$.


Ziel ist es nun, eine Folge von zu löschenden Feldern anzugeben, sodass bei deren Einhaltung die Anzahl der brennenden Felder minimiert wird.\\


Im Folgenden seinen diejenigen Felder, welche bei mindestens 2 Beobachtungen brennend waren, als \emph{verkohlt} bezeichnet.\\
Nach der Feuerausbreiteung muss jedes Feld der Umgebung eines verkohlten Feldes $c$ brennend sein oder gewesen sein oder seit der Entflammung von $c$ nicht brennbar gewesen sein.

Sei nun zunächst der Fall betrachtet, dass nur brennende Feler gelöscht werden können.

Es ist leicht zu erkennen, dass es die Lösung nicht verschlechtert, wenn ab der 2. Beobachtung bei jeder Beobachtung 1 brennendes Feld gelöscht wird. Daher wird im Folgenden davon ausgegangen, dass bei jeder Beobachtung (ab der 2.) 1 brennendes Feld gelöscht wird. Es gilt nun also für jede dieser Beobachtungen dasjenige brennende Feld zu finden, durch dessen Löschung die Anzahl der im Folgenden (nicht unbedingt umittelbar folgend) zu brennen anfangenden Felder minimiert.

Sei nun eine Beobachtung fixiert.\\
Nun soll für ein brennendes Feld $F$ ein Maß $\mu(F)$ dafür gefunden werden, mit dem bestimmt werden kann, welches Feld zum Löschen in obigem Sinne am Besten ist. Sei $\mu(F)$ daher die Anzahl der brennbaren Felder, zu denen $F$ das brennende Feld mit dem \emph{kleinster Abstand} ist. Dieser kürzeste Abstand ist dabei die minimale Anzahl an Beobachtungen, bis das Feld anfängt zu brennen. (Unter der Annahme, dass keine weiteren Felder gelöscht werden.)\\
Löscht man nun $F$, so wird der kleinste Abstand aller Felder höchstens größer; bei allen Feldern, bei deren kürzestem Abstand $F$ jedoch keine Rolle spielte (bei denen der Abstand zu einem anderen brennenden Feld also kleiner oder gleich dem Abstand zu $F$ ist), tritt keine Veränderung auf.\\
Für 2 Werte $\mu(F_1)$ und $\mu(F_2)$ gilt nun: ist $\mu(F_1) < \mu(F_2)$, so erzeugte $F_2$ bei mehr Feldern eine Vergrößerung des kleinster Abstands als $F_1$.\\
Die \emph{minimale Lebenszeit} eines Feldes sei nun eben der kleinste Abstand zu einem brennenden Feld. Es ist leicht zu erkennen, dass nach mindesten so vielen Beobachtungen, wie die minimale Lebenszeit eines Feldes ist, das Feld zu brennen beginnt.\\
$\mu(F)$ gibt also auch die Anzahl der Felder an, deren minimale Lebenszeit allein durch $F$ besitmmt ist. Löscht man $F$, so wird, wie schon gesehen, die minimale Lebenszeit all dieser Felder höchstens größer, es ist also am Besten, dasjenige Feld $F^\star$ zum Löschen auszuwählen, welches $\mu(\cdot)$ für alle aktuell brennenden Felder maximiert.

Es gilt nun noch $\mu$ effizient zu bestimmen. Da ein Wald eine rechteckige Gitterform besitzt, ist der kürzeste Abstand zwischen 2 Feldern 1, ganau dann, wenn diese Felder eine gemeinsame Kante haben.\\
Fasse man das Gitter nun als Graphen auf, wobei die Felder die Knoten sind und zwischen 2 Knoten eine Kante ist, genau dann, wenn zwischen diesen Feldern eine Kante ist. Es nun offensichtlich, dass dieser Graph ungewichtet und ungerichtet  ist. Somit ist das Finden von kleinsten Abständen mittels einer \emph{Breitensuche} möglich.\\
Dabei sind die Startfelder der Breitensuche die brennenden Felder. Dabei muss für jedes dieser brennenden Felder eine eigene Breitensuche gestartet werden; wobei für alle Breitensuchen gemeinsam die ermittelten kleinsten Abstände gespeichert werden müssen. Zusätzlich zu den kleinsten Abständen müssen auch die dazugehörigen brennenden Felder gespeichert werden, von denen pro Feld eventuell mehr als 1 existiert. Weiterhin muss die Breitensuche nur brennbare Felder besuchen.\\
Sind die kleinsten Abstände gefunden, so kann $\mu$ ermittelt werden, mithilfe simplem durchiterieren über alle Felder und gleichzeitigem Zählen der Felder, für die nur 1 brennendes Feld gespeichert wurde.

In Pseudocode:
{\small
\begin{lstlisting}
Wald	; //Der Wald; ein 2D-Container

AnfangsBrennendeFelder()	{ //Ermittelt die von Anfang brennenden Felder
  brennendeFelder := null; //1D-Container für Positionen brennender Felder
  for (i = 0..Wald.Höhe())
    for (j = 0..Wald.Breite())
       if (Wald[i,j] == BRENNEND)
         brennendeFelder.Add((i;j)); //Gefundene Position hinzufügen
         
  return brennendeFelder; //Alle gefundenen Positionen zurückgeben
}

NächsteBeobachtung(aktBrennendeFelder) { //Ermittelt die bei der nächsten Beonachtung brennenden Felder, aus den Feldern, die aktuell brennen
  neuBrennendeFelder := null;
  for all((x;y) from aktBrennendeFelder)
    if(Wald[x,y] == GELÖSCHT)
      continue; //Feld kann kein Feuer verteilen
      
    Wald[x,y] := VERKOHLT; //2 mal brennende Felder sind verkohlt
    for all((x';y') from Umgebung((x;y)))
      if(Wald[x',y'] == BRENNBAR)
        neuBrennendeFelder.Add((x';y')); //Gefundene Position hinzufügen
        Wald[x',y'] := BRENNEND; //Wald beginnt zu brennen
        
  return neuBrennendeFelder;
}

GetOptBewässerungspunkt(aktBrennendeFelder) { //Ermittelt den besten Bewässerungspunkt
  kleinsterAbstand := null; //Speichert für alle Felder des Waldes den kleinsten Abstand zu jedem Feld aus aktBrennendeFelder
  
  for(i = 0..kleinsterAbstand.Size())
  	Fülle kleinsterAbstand[i] mithilfe einer Breitensuche

  anzEindeutigKleinstAbstände := null;
  
  for (i = 0..Wald.Höhe())
    for (j = 0..Wald.Breite())
      if(Es ex. k mit kleinsterAbstand[k][i,j] eindeutiges Minimum für alle mögliche k)
        anzEindeutigKleinstAbstände[k]++;
  
  return aktBrennendeFelder[k, sodass anzEindeutigKleinstAbstände[k] maximal];
}

SimuliereFeuer() { //Die eigentliche Berechnung
  aktBrennendeFelder := AnfangsBrennendeFelder(); //Anfangs interessante Felder; Kann brennende, von Feuer umschlossene Felder beinhalten
  while(!aktBrennendeFelder.Empty()) //Solange es brennende Felder gibt
    aktBrennendeFelder := NächsteBeobachtung( aktBrennendeFelder) //Ermittle die bei nächster Beobachtung brennenden Felder
    	if(aktBrennendeFelder.Empty())
    	  break;	 //Keine Felder brennen mehr
    	
  Wald[GetOptBewässerungspunkt(aktBrennendeFelder)] := GELÖSCHT; //Lösche das aktuell beste Feld
}
\end{lstlisting}
}

\subsubsection{Korrektheit}
Wie schon beschrieben, wird bei jeder Beobachtung das für diese Beobachtung nach $\mu$ beste Feld zum Löschen ausgewählt.\\
Es gilt also zu zeigen, dass insgesamt nicht weniger Felder abbrennen, sollte bei einer Beobachtung nicht das für diese Beobachtung nach $\mu$ optimalste Feld gelöscht werden. Verallgemeinernd muss gezeigt werden, dass kein $\mu'$ existiert, welches bei midestens 1 Beobachtung 1 anderes Feld als $\mu$ vorschlägt und bei der insgesamt weniger Felder abbrennen als bei $\mu$; dass $\mu$ also \emph{optimal} ist.\\
Außerdem muss gezeigt werden, dass der Algorithmus terminiert. Da der Algorithmus jedoch nur brennende und nicht verkohlte Felder betrachtet und jedes brennendes Feld nach endlicher Zeit ist den Zustand verkohlt übergeht, gibt es einen Zeitpunkt, ab dem alle einst brennenden Felder vekohlt sind. Dann gibt es jedoch keine Felder, auf denen der Algorithmus operieren kann, der Algorithmus terminiert dann, und somit immer.

Nach der Definition von $\mu$ wird dasjenige, beliebige Feld $F_i$ aus allen möglichen Feldern $F_1..F_n$ zum Löschen ausgewählt, welches die minimale Lebenszeit von den meisten Feldern erhöht. Wählte man ein beliebiges Feld $F_i^<$ aus $\{F_1,...,F_n\}$, mit $\mu(F_i^<) < \mu(F_i)$ so erhöht sich nach Definition der minimalen Lebenszeit diese bei $\mu(F_i) - \mu(F_i^<) > 0$ Feldern weniger, als wenn man $F_i$ wählte. Erhalten diese Felder in den nächsten $z_o$ Beobachtungen keine Lebenszeitvelängerung, so brennen sie ab, wobei $z_o$ der kleinste Abstand des Feldes $o$ zu $F_i$ ist.\\
Es verbleibt also zu zeigen, dass es keine Situation geben kann, bei der die Wahl von $F_i^<$ zu einer insgesamt geringeren Anzahl an verbrannten Feldern führt.\\
Angenommen es gäbe solch eine Situation.\\
Dies heißt jedoch, dass es eine oder mehrere Löschungen von Feldern gibt, welche insgesamt dazu führen, dass die Lebenszeit von $\mu(F_i) - \mu(F_i^<) + 1$ Feldern verlängert wird. Außerdem dürfen diese Löschungen nicht möglich sein, wenn $F_i$ anstatt $F_i^<$ gelöscht wird. Dies im Speziellen heißt jedoch, dass Felder gelöscht werden, welche sonst durch die Löschung von $F_i$ eine Lebenszeitverlängerung erhielten. Somit wäre es aber besser gewesen, $F_i$ zu löschen, da bei den Beobachtungen danach auch andere Felder gelöscht werden könnten und die insgesamte Anzahl an verbrannten Feldern so insgesamt gesunken wäre.\footnote{Es ist theoretisch möglich, dass die Wahl zwischen $F_i^<$ und $F_i$ keinen Unterschied macht, beispielsweise, wenn alle Felder innerhalb der nächsten $o$ Beobachtungen verkohlen oder gelöscht werden. Dabei sei $o$ der maximale Abstand, der in $\mu(F_i^<)$ Berücksichtigung fand. Dies stellt jedoch keinen Widerspruch zur Behauptung dar.}
Es ist also optimal, ein Feld mit maximalem $\mu(F_i)$ auszuwählen. Bleibt zu zeigen, dass die Wahl eines speziellen $F_i$ mit maximalem $\mu(F_i)$ an der insgesamten Anzahl an verbrannten Feldern nichts ändert.\\
Sei $F_i^=$ ein beliebiges Feld aus $\{F_1,...,F_n\}$, mit $\mu(F_i^=) = \mu(F_i)$ und $F_i^= \not= F_i$.\footnote{Sei vorrausgestzt, dass ein solches $F_i^=$ existiert. Andernfalls existiert dieser Fall nicht, der Beweis ist dann hier beendet.}\\
Es genügt zu zeigen, dass das Wählen von $\mu(F_i^=)$ keine Verringerung der am Ende insgesamt brennenden Felder gegenüber  $\mu(F_i)$ darstellt, da $\mu(F_i)$ und $\mu(F_i^=)$ beliebig gewählt sind.\\
Angenommen dies sei der Fall.\\


Der Algorithmus ist also korrekt und optimal.

\subsubsection{Laufzeitanalyse}
Eine Breitensuche hat eine Laufzeit von $\mathcal{O}(V + E)$ in einem Graphen mit $E$ Kanten und $V$ Knoten. Speziell hat der Graph bei dieser Aufgabe $n\cdot m$ Knoten und $(n-1)\cdot (m-1)$ Kanten.\\
Eine Breitensuche wird nach obigem Algorithmus bei jeder der insgesamt $b$ Beobachtungen $f(b_i)$-mal benötigt, wobei $f(b_i)$ die Anzahl der zu betrachtenden brennenden Felder bei Beobachtung $b_i$ sei.\\
Eine Breitensuche besucht nach obigem Algorithmus höchstens $n\cdot m - f(b_i)$ Felder; die Breitensuchen haben also eine Laufzeit von $\mathcal{O}(f(b_i)\cdot (2\cdot n\cdot m - f(b_i)))$. Es ist leicht zu erkennen, dass die Funktion $F(x) = x(a-x)$ das Maximum an der Stelle $x_{max} = \frac{a}{2}$ hat. Somit gilt $\mathcal{O}(f(b_i)\cdot (2\cdot n\cdot m - f(b_i))) = \mathcal{O}(\frac{nm}{2}(2nm - \frac{nm}{2}) = \mathcal{O}(\frac{3n²m²}{4}) = \mathcal{O}(n²m²)$
 Es ergibt sich eine Gesamtlaufzeit von $\mathcal{O}(n²\cdot m² \cdot b)$. Mit $b = \mathcal{O}(n\cdot m)$ ergibt sich eine (wohl sehr grobe) obere Schranke der Laufzeit von $\mathcal{O}(n^3 \cdot m^3)$.\\
Mit diesem Algorithmus lassen sich also Lösungen für Wälder gut berechnen, deren Dimensionen 200 nicht überschreiten, bei denen also $\max{n,m} \leq 200$.

\subsection{Umsetzung}
Für die Umsetzung habe ich die Sprache \texttt{C++} verwendet.\\
Zunächst habe ich für ein Feld \texttt{FIELDSTATE} als \texttt{char} definiert.\footnote{Das Wort \enquote{definiert} ist durchaus ernst zu nehmen, da es hier beschreiben soll, dass etwas mittels \texttt{\# define} \enquote{gelöst} wurde.} Dabei kann ein \texttt{FIELDSTATE} einen oder mehrere, ebenfalls definierter, Zustände annehmen. Dabei handelt es sich um die in der Lösungsidee beschriebenen Zustände eines Feldes, \texttt{EMPTY}, \texttt{BURNABLE}, \texttt{BURNED}, \texttt{WATERED} und \texttt{COAL}.\\
Für Wälder habe ich eine Klasse \texttt{Woods} geschrieben. Jeder Wald hat dabei eine Breite (\texttt{Width}) und eine Höhe (\texttt{Height}) und hält sich ein 2-dimensional Variables Feld an \texttt{FIELDSTATE}s \texttt{Fields}.\\
Durch geschickte Operatorenüberladung und geeigntete Akzessormethoden können diese Attribute vollständig gekapselt werden.

Das Lesen der Eingabe übernimmt die Prozedur \texttt{parseInput}, welche die Daten in eine globale Instanz der Klasse \texttt{Woods} \texttt{Forest} einliest.\\
Ist die Eingabe gelesen, werden aus dieser die zu Beginn brennenden Felder mithilfe der Funktion \texttt{getInitialBurningFields} ermittelt und dann gleich an die Prozedur \texttt{simulateFire} weitergereicht. Diese Prozedur \texttt{simulateFire} simuliert nun das Feuer und ermittelt die zu löschenden Felder unter Zuhilfenahme der Funktion \texttt{getOptimalWaterSpot}.\\
Dabei wird nach jedem Löschvorgang eine Ausgabe getätigt, welche die zu löschende Position (oben links mit (0|0) beginnend) ausgibt. Auch wird unter Verwendung von ASCII-Escape-Sequenzen ein Bild in der Konsole angezeigt, welches den Wald darstellt.\\
Ist das Feuer gelöscht (kann es sich also nicht weiter ausbreiten), wird dem NUtzer eine Meldung ausgegeben, wie viele Felder verbrannten und wie viele Felder verbrannt und gelöscht wurden. (Diese beiden Zahlen beschreiben disjunkte Mengen.) Auch hier wird wieder ein Bild erzeugt und ausgegeben.
\newpage
\subsection{Beispiele}
\subsubsection{Beispiel 0}
Die ist das Beispiel aus der Aufgabenstellung. Umgewandelt für mein Programm sieht diese Eingabe folgendermaßen aus\footnote{Diese Eingabe finden Sie auch in der Datei \texttt{0.in}}:
{\small
\lstinputlisting{../Aufgabe_1/0.in}
}
Mein Programm produziert folgende Ausgabe\footnote{Diese Ausgabe finden Sie auch in der Datei \texttt{0.out.tex}; Eine Datei \texttt{0.out} mit den ASCII-Escape-Sequenzen exisitert ebenfalls.}\footnote{Um die ASCII-Escape-Sequenzen in \TeX\, korrekt darzustellen, habe ich spezielle Ausgabemethoden geschrieben. Diese produzieren anstatt der ASCII-Sequenzen \TeX -Befehle, welche optisch zu ähnlichen Ergebnissen führen.}:\\
{\ttfamily \small
\\
\begin{tikzpicture}
\tikzset{square matrix/.style={
matrix of nodes,
column sep=-\pgflinewidth, row sep=-\pgflinewidth,
nodes={draw,
minimum height=#1,
anchor=center,
text width=#1,
align=center,
inner sep=0pt
},
},
square matrix/.default=1.2cm
}
\matrix[square matrix=1.4em] {
|[fill=green]|\color[gray]{0.75} FO%
 &|[fill=green]|\color[gray]{0.75} FO%
 &|[fill=white]|\color[gray]{0.5}WA%
 &|[fill=green]|\color[gray]{0.75} FO%
 &|[fill=green]|\color[gray]{0.75} FO%
 &|[fill=green]|\color[gray]{0.75} FO%
 &|[fill=green]|\color[gray]{0.75} FO%
 &|[fill=green]|\color[gray]{0.75} FO%
 &|[fill=white]|\color[gray]{0.5}WA%
 &|[fill=green]|\color[gray]{0.75} FO%
\\
|[fill=green]|\color[gray]{0.75} FO%
 &|[fill=white]|\color[gray]{0.5}WA%
 &|[fill=white]|\color[gray]{0.5}WA%
 &|[fill=green]|\color[gray]{0.75} FO%
 &|[fill=green]|\color[gray]{0.75} FO%
 &|[fill=green]|\color[gray]{0.75} FO%
 &|[fill=green]|\color[gray]{0.75} FO%
 &|[fill=green]|\color[gray]{0.75} FO%
 &|[fill=green]|\color[gray]{0.75} FO%
 &|[fill=white]|\color[gray]{0.5}WA%
\\
|[fill=green]|\color[gray]{0.75} FO%
 &|[fill=green]|\color[gray]{0.75} FO%
 &|[fill=green]|\color[gray]{0.75} FO%
 &|[fill=green]|\color[gray]{0.75} FO%
 &|[fill=green]|\color[gray]{0.75} FO%
 &|[fill=green]|\color[gray]{0.75} FO%
 &|[fill=green]|\color[gray]{0.75} FO%
 &|[fill=green]|\color[gray]{0.75} FO%
 &|[fill=green]|\color[gray]{0.75} FO%
 &|[fill=green]|\color[gray]{0.75} FO%
\\
|[fill=green]|\color[gray]{0.75} FO%
 &|[fill=green]|\color[gray]{0.75} FO%
 &|[fill=white]|\color[gray]{0.5}WA%
 &|[fill=white]|\color[gray]{0.5}WA%
 &|[fill=white]|\color[gray]{0.5}WA%
 &|[fill=green]|\color[gray]{0.75} FO%
 &|[fill=white]|\color[gray]{0.5}WA%
 &|[fill=white]|\color[gray]{0.5}WA%
 &|[fill=white]|\color[gray]{0.5}WA%
 &|[fill=green]|\color[gray]{0.75} FO%
\\
|[fill=green]|\color[gray]{0.75} FO%
 &|[fill=green]|\color[gray]{0.75} FO%
 &|[fill=green]|\color[gray]{0.75} FO%
 &|[fill=green]|\color[gray]{0.75} FO%
 &|[fill=green]|\color[gray]{0.75} FO%
 &|[fill=green]|\color[rgb]{1,0,0}\textbf{BU}%
 &|[fill=green]|\color[gray]{0.75} FO%
 &|[fill=green]|\color[gray]{0.75} FO%
 &|[fill=green]|\color[gray]{0.75} FO%
 &|[fill=green]|\color[gray]{0.75} FO%
\\
|[fill=green]|\color[gray]{0.75} FO%
 &|[fill=green]|\color[gray]{0.75} FO%
 &|[fill=white]|\color[gray]{0.5}WA%
 &|[fill=white]|\color[gray]{0.5}WA%
 &|[fill=green]|\color[gray]{0.75} FO%
 &|[fill=green]|\color[gray]{0.75} FO%
 &|[fill=green]|\color[gray]{0.75} FO%
 &|[fill=green]|\color[gray]{0.75} FO%
 &|[fill=green]|\color[gray]{0.75} FO%
 &|[fill=green]|\color[gray]{0.75} FO%
\\
|[fill=green]|\color[gray]{0.75} FO%
 &|[fill=green]|\color[gray]{0.75} FO%
 &|[fill=green]|\color[gray]{0.75} FO%
 &|[fill=green]|\color[gray]{0.75} FO%
 &|[fill=white]|\color[gray]{0.5}WA%
 &|[fill=green]|\color[gray]{0.75} FO%
 &|[fill=green]|\color[gray]{0.75} FO%
 &|[fill=white]|\color[gray]{0.5}WA%
 &|[fill=green]|\color[gray]{0.75} FO%
 &|[fill=green]|\color[gray]{0.75} FO%
\\
|[fill=white]|\color[gray]{0.5}WA%
 &|[fill=green]|\color[gray]{0.75} FO%
 &|[fill=green]|\color[gray]{0.75} FO%
 &|[fill=green]|\color[gray]{0.75} FO%
 &|[fill=white]|\color[gray]{0.5}WA%
 &|[fill=green]|\color[gray]{0.75} FO%
 &|[fill=green]|\color[gray]{0.75} FO%
 &|[fill=white]|\color[gray]{0.5}WA%
 &|[fill=green]|\color[gray]{0.75} FO%
 &|[fill=white]|\color[gray]{0.5}WA%
\\
|[fill=green]|\color[gray]{0.75} FO%
 &|[fill=white]|\color[gray]{0.5}WA%
 &|[fill=green]|\color[gray]{0.75} FO%
 &|[fill=green]|\color[gray]{0.75} FO%
 &|[fill=white]|\color[gray]{0.5}WA%
 &|[fill=green]|\color[gray]{0.75} FO%
 &|[fill=green]|\color[gray]{0.75} FO%
 &|[fill=white]|\color[gray]{0.5}WA%
 &|[fill=green]|\color[gray]{0.75} FO%
 &|[fill=green]|\color[gray]{0.75} FO%
\\
|[fill=green]|\color[gray]{0.75} FO%
 &|[fill=green]|\color[gray]{0.75} FO%
 &|[fill=green]|\color[gray]{0.75} FO%
 &|[fill=green]|\color[gray]{0.75} FO%
 &|[fill=green]|\color[gray]{0.75} FO%
 &|[fill=green]|\color[gray]{0.75} FO%
 &|[fill=green]|\color[gray]{0.75} FO%
 &|[fill=green]|\color[gray]{0.75} FO%
 &|[fill=green]|\color[gray]{0.75} FO%
 &|[fill=green]|\color[gray]{0.75} FO%
\\
};
\end{tikzpicture}\\
---
At time 1: Water spot (5|3)
\\
\begin{tikzpicture}
\tikzset{square matrix/.style={
matrix of nodes,
column sep=-\pgflinewidth, row sep=-\pgflinewidth,
nodes={draw,
minimum height=#1,
anchor=center,
text width=#1,
align=center,
inner sep=0pt
},
},
square matrix/.default=1.2cm
}
\matrix[square matrix=1.4em] {
|[fill=green]|\color[gray]{0.75} FO%
 &|[fill=green]|\color[gray]{0.75} FO%
 &|[fill=white]|\color[gray]{0.5}WA%
 &|[fill=green]|\color[gray]{0.75} FO%
 &|[fill=green]|\color[gray]{0.75} FO%
 &|[fill=green]|\color[gray]{0.75} FO%
 &|[fill=green]|\color[gray]{0.75} FO%
 &|[fill=green]|\color[gray]{0.75} FO%
 &|[fill=white]|\color[gray]{0.5}WA%
 &|[fill=green]|\color[gray]{0.75} FO%
\\
|[fill=green]|\color[gray]{0.75} FO%
 &|[fill=white]|\color[gray]{0.5}WA%
 &|[fill=white]|\color[gray]{0.5}WA%
 &|[fill=green]|\color[gray]{0.75} FO%
 &|[fill=green]|\color[gray]{0.75} FO%
 &|[fill=green]|\color[gray]{0.75} FO%
 &|[fill=green]|\color[gray]{0.75} FO%
 &|[fill=green]|\color[gray]{0.75} FO%
 &|[fill=green]|\color[gray]{0.75} FO%
 &|[fill=white]|\color[gray]{0.5}WA%
\\
|[fill=green]|\color[gray]{0.75} FO%
 &|[fill=green]|\color[gray]{0.75} FO%
 &|[fill=green]|\color[gray]{0.75} FO%
 &|[fill=green]|\color[gray]{0.75} FO%
 &|[fill=green]|\color[gray]{0.75} FO%
 &|[fill=green]|\color[gray]{0.75} FO%
 &|[fill=green]|\color[gray]{0.75} FO%
 &|[fill=green]|\color[gray]{0.75} FO%
 &|[fill=green]|\color[gray]{0.75} FO%
 &|[fill=green]|\color[gray]{0.75} FO%
\\
|[fill=green]|\color[gray]{0.75} FO%
 &|[fill=green]|\color[gray]{0.75} FO%
 &|[fill=white]|\color[gray]{0.5}WA%
 &|[fill=white]|\color[gray]{0.5}WA%
 &|[fill=white]|\color[gray]{0.5}WA%
 &|[fill=cyan]|\color[rgb]{1,0,0}\textbf{01}%
 &|[fill=white]|\color[gray]{0.5}WA%
 &|[fill=white]|\color[gray]{0.5}WA%
 &|[fill=white]|\color[gray]{0.5}WA%
 &|[fill=green]|\color[gray]{0.75} FO%
\\
|[fill=green]|\color[gray]{0.75} FO%
 &|[fill=green]|\color[gray]{0.75} FO%
 &|[fill=green]|\color[gray]{0.75} FO%
 &|[fill=green]|\color[gray]{0.75} FO%
 &|[fill=green]|\color[rgb]{1,0,0}\textbf{BU}%
 &|[fill=green]|\color[rgb]{0,0,0}\textbf{CO}%
 &|[fill=green]|\color[rgb]{1,0,0}\textbf{BU}%
 &|[fill=green]|\color[gray]{0.75} FO%
 &|[fill=green]|\color[gray]{0.75} FO%
 &|[fill=green]|\color[gray]{0.75} FO%
\\
|[fill=green]|\color[gray]{0.75} FO%
 &|[fill=green]|\color[gray]{0.75} FO%
 &|[fill=white]|\color[gray]{0.5}WA%
 &|[fill=white]|\color[gray]{0.5}WA%
 &|[fill=green]|\color[gray]{0.75} FO%
 &|[fill=green]|\color[rgb]{1,0,0}\textbf{BU}%
 &|[fill=green]|\color[gray]{0.75} FO%
 &|[fill=green]|\color[gray]{0.75} FO%
 &|[fill=green]|\color[gray]{0.75} FO%
 &|[fill=green]|\color[gray]{0.75} FO%
\\
|[fill=green]|\color[gray]{0.75} FO%
 &|[fill=green]|\color[gray]{0.75} FO%
 &|[fill=green]|\color[gray]{0.75} FO%
 &|[fill=green]|\color[gray]{0.75} FO%
 &|[fill=white]|\color[gray]{0.5}WA%
 &|[fill=green]|\color[gray]{0.75} FO%
 &|[fill=green]|\color[gray]{0.75} FO%
 &|[fill=white]|\color[gray]{0.5}WA%
 &|[fill=green]|\color[gray]{0.75} FO%
 &|[fill=green]|\color[gray]{0.75} FO%
\\
|[fill=white]|\color[gray]{0.5}WA%
 &|[fill=green]|\color[gray]{0.75} FO%
 &|[fill=green]|\color[gray]{0.75} FO%
 &|[fill=green]|\color[gray]{0.75} FO%
 &|[fill=white]|\color[gray]{0.5}WA%
 &|[fill=green]|\color[gray]{0.75} FO%
 &|[fill=green]|\color[gray]{0.75} FO%
 &|[fill=white]|\color[gray]{0.5}WA%
 &|[fill=green]|\color[gray]{0.75} FO%
 &|[fill=white]|\color[gray]{0.5}WA%
\\
|[fill=green]|\color[gray]{0.75} FO%
 &|[fill=white]|\color[gray]{0.5}WA%
 &|[fill=green]|\color[gray]{0.75} FO%
 &|[fill=green]|\color[gray]{0.75} FO%
 &|[fill=white]|\color[gray]{0.5}WA%
 &|[fill=green]|\color[gray]{0.75} FO%
 &|[fill=green]|\color[gray]{0.75} FO%
 &|[fill=white]|\color[gray]{0.5}WA%
 &|[fill=green]|\color[gray]{0.75} FO%
 &|[fill=green]|\color[gray]{0.75} FO%
\\
|[fill=green]|\color[gray]{0.75} FO%
 &|[fill=green]|\color[gray]{0.75} FO%
 &|[fill=green]|\color[gray]{0.75} FO%
 &|[fill=green]|\color[gray]{0.75} FO%
 &|[fill=green]|\color[gray]{0.75} FO%
 &|[fill=green]|\color[gray]{0.75} FO%
 &|[fill=green]|\color[gray]{0.75} FO%
 &|[fill=green]|\color[gray]{0.75} FO%
 &|[fill=green]|\color[gray]{0.75} FO%
 &|[fill=green]|\color[gray]{0.75} FO%
\\
};
\end{tikzpicture}\\
---
At time 2: Water spot (3|4)
\\
\begin{tikzpicture}
\tikzset{square matrix/.style={
matrix of nodes,
column sep=-\pgflinewidth, row sep=-\pgflinewidth,
nodes={draw,
minimum height=#1,
anchor=center,
text width=#1,
align=center,
inner sep=0pt
},
},
square matrix/.default=1.2cm
}
\matrix[square matrix=1.4em] {
|[fill=green]|\color[gray]{0.75} FO%
 &|[fill=green]|\color[gray]{0.75} FO%
 &|[fill=white]|\color[gray]{0.5}WA%
 &|[fill=green]|\color[gray]{0.75} FO%
 &|[fill=green]|\color[gray]{0.75} FO%
 &|[fill=green]|\color[gray]{0.75} FO%
 &|[fill=green]|\color[gray]{0.75} FO%
 &|[fill=green]|\color[gray]{0.75} FO%
 &|[fill=white]|\color[gray]{0.5}WA%
 &|[fill=green]|\color[gray]{0.75} FO%
\\
|[fill=green]|\color[gray]{0.75} FO%
 &|[fill=white]|\color[gray]{0.5}WA%
 &|[fill=white]|\color[gray]{0.5}WA%
 &|[fill=green]|\color[gray]{0.75} FO%
 &|[fill=green]|\color[gray]{0.75} FO%
 &|[fill=green]|\color[gray]{0.75} FO%
 &|[fill=green]|\color[gray]{0.75} FO%
 &|[fill=green]|\color[gray]{0.75} FO%
 &|[fill=green]|\color[gray]{0.75} FO%
 &|[fill=white]|\color[gray]{0.5}WA%
\\
|[fill=green]|\color[gray]{0.75} FO%
 &|[fill=green]|\color[gray]{0.75} FO%
 &|[fill=green]|\color[gray]{0.75} FO%
 &|[fill=green]|\color[gray]{0.75} FO%
 &|[fill=green]|\color[gray]{0.75} FO%
 &|[fill=green]|\color[gray]{0.75} FO%
 &|[fill=green]|\color[gray]{0.75} FO%
 &|[fill=green]|\color[gray]{0.75} FO%
 &|[fill=green]|\color[gray]{0.75} FO%
 &|[fill=green]|\color[gray]{0.75} FO%
\\
|[fill=green]|\color[gray]{0.75} FO%
 &|[fill=green]|\color[gray]{0.75} FO%
 &|[fill=white]|\color[gray]{0.5}WA%
 &|[fill=white]|\color[gray]{0.5}WA%
 &|[fill=white]|\color[gray]{0.5}WA%
 &|[fill=cyan]|\color[rgb]{1,0,0}\textbf{01}%
 &|[fill=white]|\color[gray]{0.5}WA%
 &|[fill=white]|\color[gray]{0.5}WA%
 &|[fill=white]|\color[gray]{0.5}WA%
 &|[fill=green]|\color[gray]{0.75} FO%
\\
|[fill=green]|\color[gray]{0.75} FO%
 &|[fill=green]|\color[gray]{0.75} FO%
 &|[fill=green]|\color[gray]{0.75} FO%
 &|[fill=cyan]|\color[rgb]{1,0,0}\textbf{02}%
 &|[fill=green]|\color[rgb]{0,0,0}\textbf{CO}%
 &|[fill=green]|\color[rgb]{0,0,0}\textbf{CO}%
 &|[fill=green]|\color[rgb]{0,0,0}\textbf{CO}%
 &|[fill=green]|\color[rgb]{1,0,0}\textbf{BU}%
 &|[fill=green]|\color[gray]{0.75} FO%
 &|[fill=green]|\color[gray]{0.75} FO%
\\
|[fill=green]|\color[gray]{0.75} FO%
 &|[fill=green]|\color[gray]{0.75} FO%
 &|[fill=white]|\color[gray]{0.5}WA%
 &|[fill=white]|\color[gray]{0.5}WA%
 &|[fill=green]|\color[rgb]{1,0,0}\textbf{BU}%
 &|[fill=green]|\color[rgb]{0,0,0}\textbf{CO}%
 &|[fill=green]|\color[rgb]{1,0,0}\textbf{BU}%
 &|[fill=green]|\color[gray]{0.75} FO%
 &|[fill=green]|\color[gray]{0.75} FO%
 &|[fill=green]|\color[gray]{0.75} FO%
\\
|[fill=green]|\color[gray]{0.75} FO%
 &|[fill=green]|\color[gray]{0.75} FO%
 &|[fill=green]|\color[gray]{0.75} FO%
 &|[fill=green]|\color[gray]{0.75} FO%
 &|[fill=white]|\color[gray]{0.5}WA%
 &|[fill=green]|\color[rgb]{1,0,0}\textbf{BU}%
 &|[fill=green]|\color[gray]{0.75} FO%
 &|[fill=white]|\color[gray]{0.5}WA%
 &|[fill=green]|\color[gray]{0.75} FO%
 &|[fill=green]|\color[gray]{0.75} FO%
\\
|[fill=white]|\color[gray]{0.5}WA%
 &|[fill=green]|\color[gray]{0.75} FO%
 &|[fill=green]|\color[gray]{0.75} FO%
 &|[fill=green]|\color[gray]{0.75} FO%
 &|[fill=white]|\color[gray]{0.5}WA%
 &|[fill=green]|\color[gray]{0.75} FO%
 &|[fill=green]|\color[gray]{0.75} FO%
 &|[fill=white]|\color[gray]{0.5}WA%
 &|[fill=green]|\color[gray]{0.75} FO%
 &|[fill=white]|\color[gray]{0.5}WA%
\\
|[fill=green]|\color[gray]{0.75} FO%
 &|[fill=white]|\color[gray]{0.5}WA%
 &|[fill=green]|\color[gray]{0.75} FO%
 &|[fill=green]|\color[gray]{0.75} FO%
 &|[fill=white]|\color[gray]{0.5}WA%
 &|[fill=green]|\color[gray]{0.75} FO%
 &|[fill=green]|\color[gray]{0.75} FO%
 &|[fill=white]|\color[gray]{0.5}WA%
 &|[fill=green]|\color[gray]{0.75} FO%
 &|[fill=green]|\color[gray]{0.75} FO%
\\
|[fill=green]|\color[gray]{0.75} FO%
 &|[fill=green]|\color[gray]{0.75} FO%
 &|[fill=green]|\color[gray]{0.75} FO%
 &|[fill=green]|\color[gray]{0.75} FO%
 &|[fill=green]|\color[gray]{0.75} FO%
 &|[fill=green]|\color[gray]{0.75} FO%
 &|[fill=green]|\color[gray]{0.75} FO%
 &|[fill=green]|\color[gray]{0.75} FO%
 &|[fill=green]|\color[gray]{0.75} FO%
 &|[fill=green]|\color[gray]{0.75} FO%
\\
};
\end{tikzpicture}\\
---
At time 3: Water spot (8|4)
\\
\begin{tikzpicture}
\tikzset{square matrix/.style={
matrix of nodes,
column sep=-\pgflinewidth, row sep=-\pgflinewidth,
nodes={draw,
minimum height=#1,
anchor=center,
text width=#1,
align=center,
inner sep=0pt
},
},
square matrix/.default=1.2cm
}
\matrix[square matrix=1.4em] {
|[fill=green]|\color[gray]{0.75} FO%
 &|[fill=green]|\color[gray]{0.75} FO%
 &|[fill=white]|\color[gray]{0.5}WA%
 &|[fill=green]|\color[gray]{0.75} FO%
 &|[fill=green]|\color[gray]{0.75} FO%
 &|[fill=green]|\color[gray]{0.75} FO%
 &|[fill=green]|\color[gray]{0.75} FO%
 &|[fill=green]|\color[gray]{0.75} FO%
 &|[fill=white]|\color[gray]{0.5}WA%
 &|[fill=green]|\color[gray]{0.75} FO%
\\
|[fill=green]|\color[gray]{0.75} FO%
 &|[fill=white]|\color[gray]{0.5}WA%
 &|[fill=white]|\color[gray]{0.5}WA%
 &|[fill=green]|\color[gray]{0.75} FO%
 &|[fill=green]|\color[gray]{0.75} FO%
 &|[fill=green]|\color[gray]{0.75} FO%
 &|[fill=green]|\color[gray]{0.75} FO%
 &|[fill=green]|\color[gray]{0.75} FO%
 &|[fill=green]|\color[gray]{0.75} FO%
 &|[fill=white]|\color[gray]{0.5}WA%
\\
|[fill=green]|\color[gray]{0.75} FO%
 &|[fill=green]|\color[gray]{0.75} FO%
 &|[fill=green]|\color[gray]{0.75} FO%
 &|[fill=green]|\color[gray]{0.75} FO%
 &|[fill=green]|\color[gray]{0.75} FO%
 &|[fill=green]|\color[gray]{0.75} FO%
 &|[fill=green]|\color[gray]{0.75} FO%
 &|[fill=green]|\color[gray]{0.75} FO%
 &|[fill=green]|\color[gray]{0.75} FO%
 &|[fill=green]|\color[gray]{0.75} FO%
\\
|[fill=green]|\color[gray]{0.75} FO%
 &|[fill=green]|\color[gray]{0.75} FO%
 &|[fill=white]|\color[gray]{0.5}WA%
 &|[fill=white]|\color[gray]{0.5}WA%
 &|[fill=white]|\color[gray]{0.5}WA%
 &|[fill=cyan]|\color[rgb]{1,0,0}\textbf{01}%
 &|[fill=white]|\color[gray]{0.5}WA%
 &|[fill=white]|\color[gray]{0.5}WA%
 &|[fill=white]|\color[gray]{0.5}WA%
 &|[fill=green]|\color[gray]{0.75} FO%
\\
|[fill=green]|\color[gray]{0.75} FO%
 &|[fill=green]|\color[gray]{0.75} FO%
 &|[fill=green]|\color[gray]{0.75} FO%
 &|[fill=cyan]|\color[rgb]{1,0,0}\textbf{02}%
 &|[fill=green]|\color[rgb]{0,0,0}\textbf{CO}%
 &|[fill=green]|\color[rgb]{0,0,0}\textbf{CO}%
 &|[fill=green]|\color[rgb]{0,0,0}\textbf{CO}%
 &|[fill=green]|\color[rgb]{0,0,0}\textbf{CO}%
 &|[fill=cyan]|\color[rgb]{1,0,0}\textbf{03}%
 &|[fill=green]|\color[gray]{0.75} FO%
\\
|[fill=green]|\color[gray]{0.75} FO%
 &|[fill=green]|\color[gray]{0.75} FO%
 &|[fill=white]|\color[gray]{0.5}WA%
 &|[fill=white]|\color[gray]{0.5}WA%
 &|[fill=green]|\color[rgb]{0,0,0}\textbf{CO}%
 &|[fill=green]|\color[rgb]{0,0,0}\textbf{CO}%
 &|[fill=green]|\color[rgb]{0,0,0}\textbf{CO}%
 &|[fill=green]|\color[rgb]{1,0,0}\textbf{BU}%
 &|[fill=green]|\color[gray]{0.75} FO%
 &|[fill=green]|\color[gray]{0.75} FO%
\\
|[fill=green]|\color[gray]{0.75} FO%
 &|[fill=green]|\color[gray]{0.75} FO%
 &|[fill=green]|\color[gray]{0.75} FO%
 &|[fill=green]|\color[gray]{0.75} FO%
 &|[fill=white]|\color[gray]{0.5}WA%
 &|[fill=green]|\color[rgb]{0,0,0}\textbf{CO}%
 &|[fill=green]|\color[rgb]{1,0,0}\textbf{BU}%
 &|[fill=white]|\color[gray]{0.5}WA%
 &|[fill=green]|\color[gray]{0.75} FO%
 &|[fill=green]|\color[gray]{0.75} FO%
\\
|[fill=white]|\color[gray]{0.5}WA%
 &|[fill=green]|\color[gray]{0.75} FO%
 &|[fill=green]|\color[gray]{0.75} FO%
 &|[fill=green]|\color[gray]{0.75} FO%
 &|[fill=white]|\color[gray]{0.5}WA%
 &|[fill=green]|\color[rgb]{1,0,0}\textbf{BU}%
 &|[fill=green]|\color[gray]{0.75} FO%
 &|[fill=white]|\color[gray]{0.5}WA%
 &|[fill=green]|\color[gray]{0.75} FO%
 &|[fill=white]|\color[gray]{0.5}WA%
\\
|[fill=green]|\color[gray]{0.75} FO%
 &|[fill=white]|\color[gray]{0.5}WA%
 &|[fill=green]|\color[gray]{0.75} FO%
 &|[fill=green]|\color[gray]{0.75} FO%
 &|[fill=white]|\color[gray]{0.5}WA%
 &|[fill=green]|\color[gray]{0.75} FO%
 &|[fill=green]|\color[gray]{0.75} FO%
 &|[fill=white]|\color[gray]{0.5}WA%
 &|[fill=green]|\color[gray]{0.75} FO%
 &|[fill=green]|\color[gray]{0.75} FO%
\\
|[fill=green]|\color[gray]{0.75} FO%
 &|[fill=green]|\color[gray]{0.75} FO%
 &|[fill=green]|\color[gray]{0.75} FO%
 &|[fill=green]|\color[gray]{0.75} FO%
 &|[fill=green]|\color[gray]{0.75} FO%
 &|[fill=green]|\color[gray]{0.75} FO%
 &|[fill=green]|\color[gray]{0.75} FO%
 &|[fill=green]|\color[gray]{0.75} FO%
 &|[fill=green]|\color[gray]{0.75} FO%
 &|[fill=green]|\color[gray]{0.75} FO%
\\
};
\end{tikzpicture}\\
---
At time 4: Water spot (8|5)
\\
\begin{tikzpicture}
\tikzset{square matrix/.style={
matrix of nodes,
column sep=-\pgflinewidth, row sep=-\pgflinewidth,
nodes={draw,
minimum height=#1,
anchor=center,
text width=#1,
align=center,
inner sep=0pt
},
},
square matrix/.default=1.2cm
}
\matrix[square matrix=1.4em] {
|[fill=green]|\color[gray]{0.75} FO%
 &|[fill=green]|\color[gray]{0.75} FO%
 &|[fill=white]|\color[gray]{0.5}WA%
 &|[fill=green]|\color[gray]{0.75} FO%
 &|[fill=green]|\color[gray]{0.75} FO%
 &|[fill=green]|\color[gray]{0.75} FO%
 &|[fill=green]|\color[gray]{0.75} FO%
 &|[fill=green]|\color[gray]{0.75} FO%
 &|[fill=white]|\color[gray]{0.5}WA%
 &|[fill=green]|\color[gray]{0.75} FO%
\\
|[fill=green]|\color[gray]{0.75} FO%
 &|[fill=white]|\color[gray]{0.5}WA%
 &|[fill=white]|\color[gray]{0.5}WA%
 &|[fill=green]|\color[gray]{0.75} FO%
 &|[fill=green]|\color[gray]{0.75} FO%
 &|[fill=green]|\color[gray]{0.75} FO%
 &|[fill=green]|\color[gray]{0.75} FO%
 &|[fill=green]|\color[gray]{0.75} FO%
 &|[fill=green]|\color[gray]{0.75} FO%
 &|[fill=white]|\color[gray]{0.5}WA%
\\
|[fill=green]|\color[gray]{0.75} FO%
 &|[fill=green]|\color[gray]{0.75} FO%
 &|[fill=green]|\color[gray]{0.75} FO%
 &|[fill=green]|\color[gray]{0.75} FO%
 &|[fill=green]|\color[gray]{0.75} FO%
 &|[fill=green]|\color[gray]{0.75} FO%
 &|[fill=green]|\color[gray]{0.75} FO%
 &|[fill=green]|\color[gray]{0.75} FO%
 &|[fill=green]|\color[gray]{0.75} FO%
 &|[fill=green]|\color[gray]{0.75} FO%
\\
|[fill=green]|\color[gray]{0.75} FO%
 &|[fill=green]|\color[gray]{0.75} FO%
 &|[fill=white]|\color[gray]{0.5}WA%
 &|[fill=white]|\color[gray]{0.5}WA%
 &|[fill=white]|\color[gray]{0.5}WA%
 &|[fill=cyan]|\color[rgb]{1,0,0}\textbf{01}%
 &|[fill=white]|\color[gray]{0.5}WA%
 &|[fill=white]|\color[gray]{0.5}WA%
 &|[fill=white]|\color[gray]{0.5}WA%
 &|[fill=green]|\color[gray]{0.75} FO%
\\
|[fill=green]|\color[gray]{0.75} FO%
 &|[fill=green]|\color[gray]{0.75} FO%
 &|[fill=green]|\color[gray]{0.75} FO%
 &|[fill=cyan]|\color[rgb]{1,0,0}\textbf{02}%
 &|[fill=green]|\color[rgb]{0,0,0}\textbf{CO}%
 &|[fill=green]|\color[rgb]{0,0,0}\textbf{CO}%
 &|[fill=green]|\color[rgb]{0,0,0}\textbf{CO}%
 &|[fill=green]|\color[rgb]{0,0,0}\textbf{CO}%
 &|[fill=cyan]|\color[rgb]{1,0,0}\textbf{03}%
 &|[fill=green]|\color[gray]{0.75} FO%
\\
|[fill=green]|\color[gray]{0.75} FO%
 &|[fill=green]|\color[gray]{0.75} FO%
 &|[fill=white]|\color[gray]{0.5}WA%
 &|[fill=white]|\color[gray]{0.5}WA%
 &|[fill=green]|\color[rgb]{0,0,0}\textbf{CO}%
 &|[fill=green]|\color[rgb]{0,0,0}\textbf{CO}%
 &|[fill=green]|\color[rgb]{0,0,0}\textbf{CO}%
 &|[fill=green]|\color[rgb]{0,0,0}\textbf{CO}%
 &|[fill=cyan]|\color[rgb]{1,0,0}\textbf{04}%
 &|[fill=green]|\color[gray]{0.75} FO%
\\
|[fill=green]|\color[gray]{0.75} FO%
 &|[fill=green]|\color[gray]{0.75} FO%
 &|[fill=green]|\color[gray]{0.75} FO%
 &|[fill=green]|\color[gray]{0.75} FO%
 &|[fill=white]|\color[gray]{0.5}WA%
 &|[fill=green]|\color[rgb]{0,0,0}\textbf{CO}%
 &|[fill=green]|\color[rgb]{0,0,0}\textbf{CO}%
 &|[fill=white]|\color[gray]{0.5}WA%
 &|[fill=green]|\color[gray]{0.75} FO%
 &|[fill=green]|\color[gray]{0.75} FO%
\\
|[fill=white]|\color[gray]{0.5}WA%
 &|[fill=green]|\color[gray]{0.75} FO%
 &|[fill=green]|\color[gray]{0.75} FO%
 &|[fill=green]|\color[gray]{0.75} FO%
 &|[fill=white]|\color[gray]{0.5}WA%
 &|[fill=green]|\color[rgb]{0,0,0}\textbf{CO}%
 &|[fill=green]|\color[rgb]{1,0,0}\textbf{BU}%
 &|[fill=white]|\color[gray]{0.5}WA%
 &|[fill=green]|\color[gray]{0.75} FO%
 &|[fill=white]|\color[gray]{0.5}WA%
\\
|[fill=green]|\color[gray]{0.75} FO%
 &|[fill=white]|\color[gray]{0.5}WA%
 &|[fill=green]|\color[gray]{0.75} FO%
 &|[fill=green]|\color[gray]{0.75} FO%
 &|[fill=white]|\color[gray]{0.5}WA%
 &|[fill=green]|\color[rgb]{1,0,0}\textbf{BU}%
 &|[fill=green]|\color[gray]{0.75} FO%
 &|[fill=white]|\color[gray]{0.5}WA%
 &|[fill=green]|\color[gray]{0.75} FO%
 &|[fill=green]|\color[gray]{0.75} FO%
\\
|[fill=green]|\color[gray]{0.75} FO%
 &|[fill=green]|\color[gray]{0.75} FO%
 &|[fill=green]|\color[gray]{0.75} FO%
 &|[fill=green]|\color[gray]{0.75} FO%
 &|[fill=green]|\color[gray]{0.75} FO%
 &|[fill=green]|\color[gray]{0.75} FO%
 &|[fill=green]|\color[gray]{0.75} FO%
 &|[fill=green]|\color[gray]{0.75} FO%
 &|[fill=green]|\color[gray]{0.75} FO%
 &|[fill=green]|\color[gray]{0.75} FO%
\\
};
\end{tikzpicture}\\
---
At time 5: Water spot (5|9)
\\
\begin{tikzpicture}
\tikzset{square matrix/.style={
matrix of nodes,
column sep=-\pgflinewidth, row sep=-\pgflinewidth,
nodes={draw,
minimum height=#1,
anchor=center,
text width=#1,
align=center,
inner sep=0pt
},
},
square matrix/.default=1.2cm
}
\matrix[square matrix=1.4em] {
|[fill=green]|\color[gray]{0.75} FO%
 &|[fill=green]|\color[gray]{0.75} FO%
 &|[fill=white]|\color[gray]{0.5}WA%
 &|[fill=green]|\color[gray]{0.75} FO%
 &|[fill=green]|\color[gray]{0.75} FO%
 &|[fill=green]|\color[gray]{0.75} FO%
 &|[fill=green]|\color[gray]{0.75} FO%
 &|[fill=green]|\color[gray]{0.75} FO%
 &|[fill=white]|\color[gray]{0.5}WA%
 &|[fill=green]|\color[gray]{0.75} FO%
\\
|[fill=green]|\color[gray]{0.75} FO%
 &|[fill=white]|\color[gray]{0.5}WA%
 &|[fill=white]|\color[gray]{0.5}WA%
 &|[fill=green]|\color[gray]{0.75} FO%
 &|[fill=green]|\color[gray]{0.75} FO%
 &|[fill=green]|\color[gray]{0.75} FO%
 &|[fill=green]|\color[gray]{0.75} FO%
 &|[fill=green]|\color[gray]{0.75} FO%
 &|[fill=green]|\color[gray]{0.75} FO%
 &|[fill=white]|\color[gray]{0.5}WA%
\\
|[fill=green]|\color[gray]{0.75} FO%
 &|[fill=green]|\color[gray]{0.75} FO%
 &|[fill=green]|\color[gray]{0.75} FO%
 &|[fill=green]|\color[gray]{0.75} FO%
 &|[fill=green]|\color[gray]{0.75} FO%
 &|[fill=green]|\color[gray]{0.75} FO%
 &|[fill=green]|\color[gray]{0.75} FO%
 &|[fill=green]|\color[gray]{0.75} FO%
 &|[fill=green]|\color[gray]{0.75} FO%
 &|[fill=green]|\color[gray]{0.75} FO%
\\
|[fill=green]|\color[gray]{0.75} FO%
 &|[fill=green]|\color[gray]{0.75} FO%
 &|[fill=white]|\color[gray]{0.5}WA%
 &|[fill=white]|\color[gray]{0.5}WA%
 &|[fill=white]|\color[gray]{0.5}WA%
 &|[fill=cyan]|\color[rgb]{1,0,0}\textbf{01}%
 &|[fill=white]|\color[gray]{0.5}WA%
 &|[fill=white]|\color[gray]{0.5}WA%
 &|[fill=white]|\color[gray]{0.5}WA%
 &|[fill=green]|\color[gray]{0.75} FO%
\\
|[fill=green]|\color[gray]{0.75} FO%
 &|[fill=green]|\color[gray]{0.75} FO%
 &|[fill=green]|\color[gray]{0.75} FO%
 &|[fill=cyan]|\color[rgb]{1,0,0}\textbf{02}%
 &|[fill=green]|\color[rgb]{0,0,0}\textbf{CO}%
 &|[fill=green]|\color[rgb]{0,0,0}\textbf{CO}%
 &|[fill=green]|\color[rgb]{0,0,0}\textbf{CO}%
 &|[fill=green]|\color[rgb]{0,0,0}\textbf{CO}%
 &|[fill=cyan]|\color[rgb]{1,0,0}\textbf{03}%
 &|[fill=green]|\color[gray]{0.75} FO%
\\
|[fill=green]|\color[gray]{0.75} FO%
 &|[fill=green]|\color[gray]{0.75} FO%
 &|[fill=white]|\color[gray]{0.5}WA%
 &|[fill=white]|\color[gray]{0.5}WA%
 &|[fill=green]|\color[rgb]{0,0,0}\textbf{CO}%
 &|[fill=green]|\color[rgb]{0,0,0}\textbf{CO}%
 &|[fill=green]|\color[rgb]{0,0,0}\textbf{CO}%
 &|[fill=green]|\color[rgb]{0,0,0}\textbf{CO}%
 &|[fill=cyan]|\color[rgb]{1,0,0}\textbf{04}%
 &|[fill=green]|\color[gray]{0.75} FO%
\\
|[fill=green]|\color[gray]{0.75} FO%
 &|[fill=green]|\color[gray]{0.75} FO%
 &|[fill=green]|\color[gray]{0.75} FO%
 &|[fill=green]|\color[gray]{0.75} FO%
 &|[fill=white]|\color[gray]{0.5}WA%
 &|[fill=green]|\color[rgb]{0,0,0}\textbf{CO}%
 &|[fill=green]|\color[rgb]{0,0,0}\textbf{CO}%
 &|[fill=white]|\color[gray]{0.5}WA%
 &|[fill=green]|\color[gray]{0.75} FO%
 &|[fill=green]|\color[gray]{0.75} FO%
\\
|[fill=white]|\color[gray]{0.5}WA%
 &|[fill=green]|\color[gray]{0.75} FO%
 &|[fill=green]|\color[gray]{0.75} FO%
 &|[fill=green]|\color[gray]{0.75} FO%
 &|[fill=white]|\color[gray]{0.5}WA%
 &|[fill=green]|\color[rgb]{0,0,0}\textbf{CO}%
 &|[fill=green]|\color[rgb]{0,0,0}\textbf{CO}%
 &|[fill=white]|\color[gray]{0.5}WA%
 &|[fill=green]|\color[gray]{0.75} FO%
 &|[fill=white]|\color[gray]{0.5}WA%
\\
|[fill=green]|\color[gray]{0.75} FO%
 &|[fill=white]|\color[gray]{0.5}WA%
 &|[fill=green]|\color[gray]{0.75} FO%
 &|[fill=green]|\color[gray]{0.75} FO%
 &|[fill=white]|\color[gray]{0.5}WA%
 &|[fill=green]|\color[rgb]{0,0,0}\textbf{CO}%
 &|[fill=green]|\color[rgb]{1,0,0}\textbf{BU}%
 &|[fill=white]|\color[gray]{0.5}WA%
 &|[fill=green]|\color[gray]{0.75} FO%
 &|[fill=green]|\color[gray]{0.75} FO%
\\
|[fill=green]|\color[gray]{0.75} FO%
 &|[fill=green]|\color[gray]{0.75} FO%
 &|[fill=green]|\color[gray]{0.75} FO%
 &|[fill=green]|\color[gray]{0.75} FO%
 &|[fill=green]|\color[gray]{0.75} FO%
 &|[fill=cyan]|\color[rgb]{1,0,0}\textbf{05}%
 &|[fill=green]|\color[gray]{0.75} FO%
 &|[fill=green]|\color[gray]{0.75} FO%
 &|[fill=green]|\color[gray]{0.75} FO%
 &|[fill=green]|\color[gray]{0.75} FO%
\\
};
\end{tikzpicture}\\
---
At time 6: Water spot (6|9)
\\
\begin{tikzpicture}
\tikzset{square matrix/.style={
matrix of nodes,
column sep=-\pgflinewidth, row sep=-\pgflinewidth,
nodes={draw,
minimum height=#1,
anchor=center,
text width=#1,
align=center,
inner sep=0pt
},
},
square matrix/.default=1.2cm
}
\matrix[square matrix=1.4em] {
|[fill=green]|\color[gray]{0.75} FO%
 &|[fill=green]|\color[gray]{0.75} FO%
 &|[fill=white]|\color[gray]{0.5}WA%
 &|[fill=green]|\color[gray]{0.75} FO%
 &|[fill=green]|\color[gray]{0.75} FO%
 &|[fill=green]|\color[gray]{0.75} FO%
 &|[fill=green]|\color[gray]{0.75} FO%
 &|[fill=green]|\color[gray]{0.75} FO%
 &|[fill=white]|\color[gray]{0.5}WA%
 &|[fill=green]|\color[gray]{0.75} FO%
\\
|[fill=green]|\color[gray]{0.75} FO%
 &|[fill=white]|\color[gray]{0.5}WA%
 &|[fill=white]|\color[gray]{0.5}WA%
 &|[fill=green]|\color[gray]{0.75} FO%
 &|[fill=green]|\color[gray]{0.75} FO%
 &|[fill=green]|\color[gray]{0.75} FO%
 &|[fill=green]|\color[gray]{0.75} FO%
 &|[fill=green]|\color[gray]{0.75} FO%
 &|[fill=green]|\color[gray]{0.75} FO%
 &|[fill=white]|\color[gray]{0.5}WA%
\\
|[fill=green]|\color[gray]{0.75} FO%
 &|[fill=green]|\color[gray]{0.75} FO%
 &|[fill=green]|\color[gray]{0.75} FO%
 &|[fill=green]|\color[gray]{0.75} FO%
 &|[fill=green]|\color[gray]{0.75} FO%
 &|[fill=green]|\color[gray]{0.75} FO%
 &|[fill=green]|\color[gray]{0.75} FO%
 &|[fill=green]|\color[gray]{0.75} FO%
 &|[fill=green]|\color[gray]{0.75} FO%
 &|[fill=green]|\color[gray]{0.75} FO%
\\
|[fill=green]|\color[gray]{0.75} FO%
 &|[fill=green]|\color[gray]{0.75} FO%
 &|[fill=white]|\color[gray]{0.5}WA%
 &|[fill=white]|\color[gray]{0.5}WA%
 &|[fill=white]|\color[gray]{0.5}WA%
 &|[fill=cyan]|\color[rgb]{1,0,0}\textbf{01}%
 &|[fill=white]|\color[gray]{0.5}WA%
 &|[fill=white]|\color[gray]{0.5}WA%
 &|[fill=white]|\color[gray]{0.5}WA%
 &|[fill=green]|\color[gray]{0.75} FO%
\\
|[fill=green]|\color[gray]{0.75} FO%
 &|[fill=green]|\color[gray]{0.75} FO%
 &|[fill=green]|\color[gray]{0.75} FO%
 &|[fill=cyan]|\color[rgb]{1,0,0}\textbf{02}%
 &|[fill=green]|\color[rgb]{0,0,0}\textbf{CO}%
 &|[fill=green]|\color[rgb]{0,0,0}\textbf{CO}%
 &|[fill=green]|\color[rgb]{0,0,0}\textbf{CO}%
 &|[fill=green]|\color[rgb]{0,0,0}\textbf{CO}%
 &|[fill=cyan]|\color[rgb]{1,0,0}\textbf{03}%
 &|[fill=green]|\color[gray]{0.75} FO%
\\
|[fill=green]|\color[gray]{0.75} FO%
 &|[fill=green]|\color[gray]{0.75} FO%
 &|[fill=white]|\color[gray]{0.5}WA%
 &|[fill=white]|\color[gray]{0.5}WA%
 &|[fill=green]|\color[rgb]{0,0,0}\textbf{CO}%
 &|[fill=green]|\color[rgb]{0,0,0}\textbf{CO}%
 &|[fill=green]|\color[rgb]{0,0,0}\textbf{CO}%
 &|[fill=green]|\color[rgb]{0,0,0}\textbf{CO}%
 &|[fill=cyan]|\color[rgb]{1,0,0}\textbf{04}%
 &|[fill=green]|\color[gray]{0.75} FO%
\\
|[fill=green]|\color[gray]{0.75} FO%
 &|[fill=green]|\color[gray]{0.75} FO%
 &|[fill=green]|\color[gray]{0.75} FO%
 &|[fill=green]|\color[gray]{0.75} FO%
 &|[fill=white]|\color[gray]{0.5}WA%
 &|[fill=green]|\color[rgb]{0,0,0}\textbf{CO}%
 &|[fill=green]|\color[rgb]{0,0,0}\textbf{CO}%
 &|[fill=white]|\color[gray]{0.5}WA%
 &|[fill=green]|\color[gray]{0.75} FO%
 &|[fill=green]|\color[gray]{0.75} FO%
\\
|[fill=white]|\color[gray]{0.5}WA%
 &|[fill=green]|\color[gray]{0.75} FO%
 &|[fill=green]|\color[gray]{0.75} FO%
 &|[fill=green]|\color[gray]{0.75} FO%
 &|[fill=white]|\color[gray]{0.5}WA%
 &|[fill=green]|\color[rgb]{0,0,0}\textbf{CO}%
 &|[fill=green]|\color[rgb]{0,0,0}\textbf{CO}%
 &|[fill=white]|\color[gray]{0.5}WA%
 &|[fill=green]|\color[gray]{0.75} FO%
 &|[fill=white]|\color[gray]{0.5}WA%
\\
|[fill=green]|\color[gray]{0.75} FO%
 &|[fill=white]|\color[gray]{0.5}WA%
 &|[fill=green]|\color[gray]{0.75} FO%
 &|[fill=green]|\color[gray]{0.75} FO%
 &|[fill=white]|\color[gray]{0.5}WA%
 &|[fill=green]|\color[rgb]{0,0,0}\textbf{CO}%
 &|[fill=green]|\color[rgb]{0,0,0}\textbf{CO}%
 &|[fill=white]|\color[gray]{0.5}WA%
 &|[fill=green]|\color[gray]{0.75} FO%
 &|[fill=green]|\color[gray]{0.75} FO%
\\
|[fill=green]|\color[gray]{0.75} FO%
 &|[fill=green]|\color[gray]{0.75} FO%
 &|[fill=green]|\color[gray]{0.75} FO%
 &|[fill=green]|\color[gray]{0.75} FO%
 &|[fill=green]|\color[gray]{0.75} FO%
 &|[fill=cyan]|\color[rgb]{1,0,0}\textbf{05}%
 &|[fill=cyan]|\color[rgb]{1,0,0}\textbf{06}%
 &|[fill=green]|\color[gray]{0.75} FO%
 &|[fill=green]|\color[gray]{0.75} FO%
 &|[fill=green]|\color[gray]{0.75} FO%
\\
};
\end{tikzpicture}\\
---
And you'll find 14 pieces of coal and 6 pieces of watered coal
\\
\begin{tikzpicture}
\tikzset{square matrix/.style={
matrix of nodes,
column sep=-\pgflinewidth, row sep=-\pgflinewidth,
nodes={draw,
minimum height=#1,
anchor=center,
text width=#1,
align=center,
inner sep=0pt
},
},
square matrix/.default=1.2cm
}
\matrix[square matrix=1.4em] {
|[fill=green]|\color[gray]{0.75} FO%
 &|[fill=green]|\color[gray]{0.75} FO%
 &|[fill=white]|\color[gray]{0.5}WA%
 &|[fill=green]|\color[gray]{0.75} FO%
 &|[fill=green]|\color[gray]{0.75} FO%
 &|[fill=green]|\color[gray]{0.75} FO%
 &|[fill=green]|\color[gray]{0.75} FO%
 &|[fill=green]|\color[gray]{0.75} FO%
 &|[fill=white]|\color[gray]{0.5}WA%
 &|[fill=green]|\color[gray]{0.75} FO%
\\
|[fill=green]|\color[gray]{0.75} FO%
 &|[fill=white]|\color[gray]{0.5}WA%
 &|[fill=white]|\color[gray]{0.5}WA%
 &|[fill=green]|\color[gray]{0.75} FO%
 &|[fill=green]|\color[gray]{0.75} FO%
 &|[fill=green]|\color[gray]{0.75} FO%
 &|[fill=green]|\color[gray]{0.75} FO%
 &|[fill=green]|\color[gray]{0.75} FO%
 &|[fill=green]|\color[gray]{0.75} FO%
 &|[fill=white]|\color[gray]{0.5}WA%
\\
|[fill=green]|\color[gray]{0.75} FO%
 &|[fill=green]|\color[gray]{0.75} FO%
 &|[fill=green]|\color[gray]{0.75} FO%
 &|[fill=green]|\color[gray]{0.75} FO%
 &|[fill=green]|\color[gray]{0.75} FO%
 &|[fill=green]|\color[gray]{0.75} FO%
 &|[fill=green]|\color[gray]{0.75} FO%
 &|[fill=green]|\color[gray]{0.75} FO%
 &|[fill=green]|\color[gray]{0.75} FO%
 &|[fill=green]|\color[gray]{0.75} FO%
\\
|[fill=green]|\color[gray]{0.75} FO%
 &|[fill=green]|\color[gray]{0.75} FO%
 &|[fill=white]|\color[gray]{0.5}WA%
 &|[fill=white]|\color[gray]{0.5}WA%
 &|[fill=white]|\color[gray]{0.5}WA%
 &|[fill=cyan]|\color[rgb]{1,0,0}\textbf{01}%
 &|[fill=white]|\color[gray]{0.5}WA%
 &|[fill=white]|\color[gray]{0.5}WA%
 &|[fill=white]|\color[gray]{0.5}WA%
 &|[fill=green]|\color[gray]{0.75} FO%
\\
|[fill=green]|\color[gray]{0.75} FO%
 &|[fill=green]|\color[gray]{0.75} FO%
 &|[fill=green]|\color[gray]{0.75} FO%
 &|[fill=cyan]|\color[rgb]{1,0,0}\textbf{02}%
 &|[fill=green]|\color[rgb]{0,0,0}\textbf{CO}%
 &|[fill=green]|\color[rgb]{0,0,0}\textbf{CO}%
 &|[fill=green]|\color[rgb]{0,0,0}\textbf{CO}%
 &|[fill=green]|\color[rgb]{0,0,0}\textbf{CO}%
 &|[fill=cyan]|\color[rgb]{1,0,0}\textbf{03}%
 &|[fill=green]|\color[gray]{0.75} FO%
\\
|[fill=green]|\color[gray]{0.75} FO%
 &|[fill=green]|\color[gray]{0.75} FO%
 &|[fill=white]|\color[gray]{0.5}WA%
 &|[fill=white]|\color[gray]{0.5}WA%
 &|[fill=green]|\color[rgb]{0,0,0}\textbf{CO}%
 &|[fill=green]|\color[rgb]{0,0,0}\textbf{CO}%
 &|[fill=green]|\color[rgb]{0,0,0}\textbf{CO}%
 &|[fill=green]|\color[rgb]{0,0,0}\textbf{CO}%
 &|[fill=cyan]|\color[rgb]{1,0,0}\textbf{04}%
 &|[fill=green]|\color[gray]{0.75} FO%
\\
|[fill=green]|\color[gray]{0.75} FO%
 &|[fill=green]|\color[gray]{0.75} FO%
 &|[fill=green]|\color[gray]{0.75} FO%
 &|[fill=green]|\color[gray]{0.75} FO%
 &|[fill=white]|\color[gray]{0.5}WA%
 &|[fill=green]|\color[rgb]{0,0,0}\textbf{CO}%
 &|[fill=green]|\color[rgb]{0,0,0}\textbf{CO}%
 &|[fill=white]|\color[gray]{0.5}WA%
 &|[fill=green]|\color[gray]{0.75} FO%
 &|[fill=green]|\color[gray]{0.75} FO%
\\
|[fill=white]|\color[gray]{0.5}WA%
 &|[fill=green]|\color[gray]{0.75} FO%
 &|[fill=green]|\color[gray]{0.75} FO%
 &|[fill=green]|\color[gray]{0.75} FO%
 &|[fill=white]|\color[gray]{0.5}WA%
 &|[fill=green]|\color[rgb]{0,0,0}\textbf{CO}%
 &|[fill=green]|\color[rgb]{0,0,0}\textbf{CO}%
 &|[fill=white]|\color[gray]{0.5}WA%
 &|[fill=green]|\color[gray]{0.75} FO%
 &|[fill=white]|\color[gray]{0.5}WA%
\\
|[fill=green]|\color[gray]{0.75} FO%
 &|[fill=white]|\color[gray]{0.5}WA%
 &|[fill=green]|\color[gray]{0.75} FO%
 &|[fill=green]|\color[gray]{0.75} FO%
 &|[fill=white]|\color[gray]{0.5}WA%
 &|[fill=green]|\color[rgb]{0,0,0}\textbf{CO}%
 &|[fill=green]|\color[rgb]{0,0,0}\textbf{CO}%
 &|[fill=white]|\color[gray]{0.5}WA%
 &|[fill=green]|\color[gray]{0.75} FO%
 &|[fill=green]|\color[gray]{0.75} FO%
\\
|[fill=green]|\color[gray]{0.75} FO%
 &|[fill=green]|\color[gray]{0.75} FO%
 &|[fill=green]|\color[gray]{0.75} FO%
 &|[fill=green]|\color[gray]{0.75} FO%
 &|[fill=green]|\color[gray]{0.75} FO%
 &|[fill=cyan]|\color[rgb]{1,0,0}\textbf{05}%
 &|[fill=cyan]|\color[rgb]{1,0,0}\textbf{06}%
 &|[fill=green]|\color[gray]{0.75} FO%
 &|[fill=green]|\color[gray]{0.75} FO%
 &|[fill=green]|\color[gray]{0.75} FO%
\\
};
\end{tikzpicture}\\
\\
Explanation:\\
\colorbox{white}{\color[gray]{0.5}WA}  ---  EMPTY\\
\colorbox{green}{\color[gray]{0.5}FO}  ---  BURNABLE\\
\colorbox{white}{\color[rgb]{1,0,0}\textbf{BU}}  ---  BURNED\\
\colorbox{white}{\color[rgb]{0,0,0}\textbf{CO}}  ---  COAL (doubly burned)\\
\colorbox{cyan}{\#\#}  ---  WATERED at time \#\#\\
Fields can have more than 1 state.
}
\subsubsection{Beispiel 1}
Eine Situation mit mehr als einem Feuer bei der ersten Beobachtung\footnote{Diese Eingabe finden Sie auch in der Datei \texttt{1.in}}:
{\small
\lstinputlisting{../Aufgabe_1/1.in}
}
Mein Programm produziert folgende Ausgabe\footnote{Diese Ausgabe finden Sie auch in der Datei \texttt{1.out.tex}; Eine Datei \texttt{1.out} mit den ASCII-Escape-Sequenzen exisitert ebenfalls.}:\\
{\ttfamily \small
\\
\begin{tikzpicture}
\tikzset{square matrix/.style={
matrix of nodes,
column sep=-\pgflinewidth, row sep=-\pgflinewidth,
nodes={draw,
minimum height=#1,
anchor=center,
text width=#1,
align=center,
inner sep=0pt
},
},
square matrix/.default=1.2cm
}
\matrix[square matrix=1.4em] {
|[fill=green]|\color[rgb]{1,0,0}\textbf{BU}%
 &|[fill=green]|\color[gray]{0.75} FO%
 &|[fill=white]|\color[gray]{0.5}WA%
 &|[fill=green]|\color[gray]{0.75} FO%
 &|[fill=green]|\color[gray]{0.75} FO%
 &|[fill=green]|\color[gray]{0.75} FO%
 &|[fill=green]|\color[gray]{0.75} FO%
 &|[fill=green]|\color[gray]{0.75} FO%
 &|[fill=white]|\color[gray]{0.5}WA%
 &|[fill=green]|\color[gray]{0.75} FO%
\\
|[fill=green]|\color[gray]{0.75} FO%
 &|[fill=white]|\color[gray]{0.5}WA%
 &|[fill=white]|\color[gray]{0.5}WA%
 &|[fill=green]|\color[gray]{0.75} FO%
 &|[fill=green]|\color[gray]{0.75} FO%
 &|[fill=green]|\color[gray]{0.75} FO%
 &|[fill=green]|\color[gray]{0.75} FO%
 &|[fill=green]|\color[gray]{0.75} FO%
 &|[fill=green]|\color[gray]{0.75} FO%
 &|[fill=white]|\color[gray]{0.5}WA%
\\
|[fill=green]|\color[gray]{0.75} FO%
 &|[fill=green]|\color[gray]{0.75} FO%
 &|[fill=green]|\color[gray]{0.75} FO%
 &|[fill=green]|\color[gray]{0.75} FO%
 &|[fill=green]|\color[gray]{0.75} FO%
 &|[fill=green]|\color[gray]{0.75} FO%
 &|[fill=green]|\color[gray]{0.75} FO%
 &|[fill=green]|\color[gray]{0.75} FO%
 &|[fill=green]|\color[gray]{0.75} FO%
 &|[fill=green]|\color[gray]{0.75} FO%
\\
|[fill=green]|\color[gray]{0.75} FO%
 &|[fill=green]|\color[gray]{0.75} FO%
 &|[fill=white]|\color[gray]{0.5}WA%
 &|[fill=white]|\color[gray]{0.5}WA%
 &|[fill=white]|\color[gray]{0.5}WA%
 &|[fill=green]|\color[gray]{0.75} FO%
 &|[fill=white]|\color[gray]{0.5}WA%
 &|[fill=white]|\color[gray]{0.5}WA%
 &|[fill=white]|\color[gray]{0.5}WA%
 &|[fill=green]|\color[gray]{0.75} FO%
\\
|[fill=green]|\color[gray]{0.75} FO%
 &|[fill=green]|\color[gray]{0.75} FO%
 &|[fill=green]|\color[gray]{0.75} FO%
 &|[fill=green]|\color[gray]{0.75} FO%
 &|[fill=green]|\color[gray]{0.75} FO%
 &|[fill=green]|\color[rgb]{1,0,0}\textbf{BU}%
 &|[fill=green]|\color[gray]{0.75} FO%
 &|[fill=green]|\color[gray]{0.75} FO%
 &|[fill=green]|\color[gray]{0.75} FO%
 &|[fill=green]|\color[gray]{0.75} FO%
\\
|[fill=green]|\color[gray]{0.75} FO%
 &|[fill=green]|\color[gray]{0.75} FO%
 &|[fill=white]|\color[gray]{0.5}WA%
 &|[fill=white]|\color[gray]{0.5}WA%
 &|[fill=green]|\color[gray]{0.75} FO%
 &|[fill=green]|\color[gray]{0.75} FO%
 &|[fill=green]|\color[gray]{0.75} FO%
 &|[fill=green]|\color[gray]{0.75} FO%
 &|[fill=green]|\color[gray]{0.75} FO%
 &|[fill=green]|\color[gray]{0.75} FO%
\\
|[fill=green]|\color[gray]{0.75} FO%
 &|[fill=green]|\color[gray]{0.75} FO%
 &|[fill=green]|\color[gray]{0.75} FO%
 &|[fill=green]|\color[gray]{0.75} FO%
 &|[fill=white]|\color[gray]{0.5}WA%
 &|[fill=green]|\color[gray]{0.75} FO%
 &|[fill=green]|\color[gray]{0.75} FO%
 &|[fill=white]|\color[gray]{0.5}WA%
 &|[fill=green]|\color[gray]{0.75} FO%
 &|[fill=green]|\color[gray]{0.75} FO%
\\
|[fill=white]|\color[gray]{0.5}WA%
 &|[fill=green]|\color[gray]{0.75} FO%
 &|[fill=green]|\color[gray]{0.75} FO%
 &|[fill=green]|\color[gray]{0.75} FO%
 &|[fill=white]|\color[gray]{0.5}WA%
 &|[fill=green]|\color[gray]{0.75} FO%
 &|[fill=green]|\color[gray]{0.75} FO%
 &|[fill=white]|\color[gray]{0.5}WA%
 &|[fill=green]|\color[gray]{0.75} FO%
 &|[fill=white]|\color[gray]{0.5}WA%
\\
|[fill=green]|\color[gray]{0.75} FO%
 &|[fill=white]|\color[gray]{0.5}WA%
 &|[fill=green]|\color[gray]{0.75} FO%
 &|[fill=green]|\color[gray]{0.75} FO%
 &|[fill=white]|\color[gray]{0.5}WA%
 &|[fill=green]|\color[gray]{0.75} FO%
 &|[fill=green]|\color[gray]{0.75} FO%
 &|[fill=white]|\color[gray]{0.5}WA%
 &|[fill=green]|\color[gray]{0.75} FO%
 &|[fill=green]|\color[gray]{0.75} FO%
\\
|[fill=green]|\color[gray]{0.75} FO%
 &|[fill=green]|\color[gray]{0.75} FO%
 &|[fill=green]|\color[gray]{0.75} FO%
 &|[fill=green]|\color[gray]{0.75} FO%
 &|[fill=green]|\color[gray]{0.75} FO%
 &|[fill=green]|\color[gray]{0.75} FO%
 &|[fill=green]|\color[rgb]{1,0,0}\textbf{BU}%
 &|[fill=green]|\color[gray]{0.75} FO%
 &|[fill=green]|\color[gray]{0.75} FO%
 &|[fill=green]|\color[gray]{0.75} FO%
\\
|[fill=green]|\color[gray]{0.75} FO%
 &|[fill=green]|\color[gray]{0.75} FO%
 &|[fill=green]|\color[gray]{0.75} FO%
 &|[fill=green]|\color[gray]{0.75} FO%
 &|[fill=green]|\color[gray]{0.75} FO%
 &|[fill=green]|\color[gray]{0.75} FO%
 &|[fill=green]|\color[gray]{0.75} FO%
 &|[fill=green]|\color[gray]{0.75} FO%
 &|[fill=green]|\color[gray]{0.75} FO%
 &|[fill=green]|\color[gray]{0.75} FO%
\\
};
\end{tikzpicture}\\
At time 1: Water spot (5|3):
\\
\begin{tikzpicture}
\tikzset{square matrix/.style={
matrix of nodes,
column sep=-\pgflinewidth, row sep=-\pgflinewidth,
nodes={draw,
minimum height=#1,
anchor=center,
text width=#1,
align=center,
inner sep=0pt
},
},
square matrix/.default=1.2cm
}
\matrix[square matrix=1.4em] {
|[fill=green]|\color[rgb]{0,0,0}\textbf{CO}%
 &|[fill=green]|\color[rgb]{1,0,0}\textbf{BU}%
 &|[fill=white]|\color[gray]{0.5}WA%
 &|[fill=green]|\color[gray]{0.75} FO%
 &|[fill=green]|\color[gray]{0.75} FO%
 &|[fill=green]|\color[gray]{0.75} FO%
 &|[fill=green]|\color[gray]{0.75} FO%
 &|[fill=green]|\color[gray]{0.75} FO%
 &|[fill=white]|\color[gray]{0.5}WA%
 &|[fill=green]|\color[gray]{0.75} FO%
\\
|[fill=green]|\color[rgb]{1,0,0}\textbf{BU}%
 &|[fill=white]|\color[gray]{0.5}WA%
 &|[fill=white]|\color[gray]{0.5}WA%
 &|[fill=green]|\color[gray]{0.75} FO%
 &|[fill=green]|\color[gray]{0.75} FO%
 &|[fill=green]|\color[gray]{0.75} FO%
 &|[fill=green]|\color[gray]{0.75} FO%
 &|[fill=green]|\color[gray]{0.75} FO%
 &|[fill=green]|\color[gray]{0.75} FO%
 &|[fill=white]|\color[gray]{0.5}WA%
\\
|[fill=green]|\color[gray]{0.75} FO%
 &|[fill=green]|\color[gray]{0.75} FO%
 &|[fill=green]|\color[gray]{0.75} FO%
 &|[fill=green]|\color[gray]{0.75} FO%
 &|[fill=green]|\color[gray]{0.75} FO%
 &|[fill=green]|\color[gray]{0.75} FO%
 &|[fill=green]|\color[gray]{0.75} FO%
 &|[fill=green]|\color[gray]{0.75} FO%
 &|[fill=green]|\color[gray]{0.75} FO%
 &|[fill=green]|\color[gray]{0.75} FO%
\\
|[fill=green]|\color[gray]{0.75} FO%
 &|[fill=green]|\color[gray]{0.75} FO%
 &|[fill=white]|\color[gray]{0.5}WA%
 &|[fill=white]|\color[gray]{0.5}WA%
 &|[fill=white]|\color[gray]{0.5}WA%
 &|[fill=cyan]|\color[rgb]{1,0,0}\textbf{01}%
 &|[fill=white]|\color[gray]{0.5}WA%
 &|[fill=white]|\color[gray]{0.5}WA%
 &|[fill=white]|\color[gray]{0.5}WA%
 &|[fill=green]|\color[gray]{0.75} FO%
\\
|[fill=green]|\color[gray]{0.75} FO%
 &|[fill=green]|\color[gray]{0.75} FO%
 &|[fill=green]|\color[gray]{0.75} FO%
 &|[fill=green]|\color[gray]{0.75} FO%
 &|[fill=green]|\color[rgb]{1,0,0}\textbf{BU}%
 &|[fill=green]|\color[rgb]{0,0,0}\textbf{CO}%
 &|[fill=green]|\color[rgb]{1,0,0}\textbf{BU}%
 &|[fill=green]|\color[gray]{0.75} FO%
 &|[fill=green]|\color[gray]{0.75} FO%
 &|[fill=green]|\color[gray]{0.75} FO%
\\
|[fill=green]|\color[gray]{0.75} FO%
 &|[fill=green]|\color[gray]{0.75} FO%
 &|[fill=white]|\color[gray]{0.5}WA%
 &|[fill=white]|\color[gray]{0.5}WA%
 &|[fill=green]|\color[gray]{0.75} FO%
 &|[fill=green]|\color[rgb]{1,0,0}\textbf{BU}%
 &|[fill=green]|\color[gray]{0.75} FO%
 &|[fill=green]|\color[gray]{0.75} FO%
 &|[fill=green]|\color[gray]{0.75} FO%
 &|[fill=green]|\color[gray]{0.75} FO%
\\
|[fill=green]|\color[gray]{0.75} FO%
 &|[fill=green]|\color[gray]{0.75} FO%
 &|[fill=green]|\color[gray]{0.75} FO%
 &|[fill=green]|\color[gray]{0.75} FO%
 &|[fill=white]|\color[gray]{0.5}WA%
 &|[fill=green]|\color[gray]{0.75} FO%
 &|[fill=green]|\color[gray]{0.75} FO%
 &|[fill=white]|\color[gray]{0.5}WA%
 &|[fill=green]|\color[gray]{0.75} FO%
 &|[fill=green]|\color[gray]{0.75} FO%
\\
|[fill=white]|\color[gray]{0.5}WA%
 &|[fill=green]|\color[gray]{0.75} FO%
 &|[fill=green]|\color[gray]{0.75} FO%
 &|[fill=green]|\color[gray]{0.75} FO%
 &|[fill=white]|\color[gray]{0.5}WA%
 &|[fill=green]|\color[gray]{0.75} FO%
 &|[fill=green]|\color[gray]{0.75} FO%
 &|[fill=white]|\color[gray]{0.5}WA%
 &|[fill=green]|\color[gray]{0.75} FO%
 &|[fill=white]|\color[gray]{0.5}WA%
\\
|[fill=green]|\color[gray]{0.75} FO%
 &|[fill=white]|\color[gray]{0.5}WA%
 &|[fill=green]|\color[gray]{0.75} FO%
 &|[fill=green]|\color[gray]{0.75} FO%
 &|[fill=white]|\color[gray]{0.5}WA%
 &|[fill=green]|\color[gray]{0.75} FO%
 &|[fill=green]|\color[rgb]{1,0,0}\textbf{BU}%
 &|[fill=white]|\color[gray]{0.5}WA%
 &|[fill=green]|\color[gray]{0.75} FO%
 &|[fill=green]|\color[gray]{0.75} FO%
\\
|[fill=green]|\color[gray]{0.75} FO%
 &|[fill=green]|\color[gray]{0.75} FO%
 &|[fill=green]|\color[gray]{0.75} FO%
 &|[fill=green]|\color[gray]{0.75} FO%
 &|[fill=green]|\color[gray]{0.75} FO%
 &|[fill=green]|\color[rgb]{1,0,0}\textbf{BU}%
 &|[fill=green]|\color[rgb]{0,0,0}\textbf{CO}%
 &|[fill=green]|\color[rgb]{1,0,0}\textbf{BU}%
 &|[fill=green]|\color[gray]{0.75} FO%
 &|[fill=green]|\color[gray]{0.75} FO%
\\
|[fill=green]|\color[gray]{0.75} FO%
 &|[fill=green]|\color[gray]{0.75} FO%
 &|[fill=green]|\color[gray]{0.75} FO%
 &|[fill=green]|\color[gray]{0.75} FO%
 &|[fill=green]|\color[gray]{0.75} FO%
 &|[fill=green]|\color[gray]{0.75} FO%
 &|[fill=green]|\color[rgb]{1,0,0}\textbf{BU}%
 &|[fill=green]|\color[gray]{0.75} FO%
 &|[fill=green]|\color[gray]{0.75} FO%
 &|[fill=green]|\color[gray]{0.75} FO%
\\
};
\end{tikzpicture}\\
At time 2: Water spot (0|2):
\\
\begin{tikzpicture}
\tikzset{square matrix/.style={
matrix of nodes,
column sep=-\pgflinewidth, row sep=-\pgflinewidth,
nodes={draw,
minimum height=#1,
anchor=center,
text width=#1,
align=center,
inner sep=0pt
},
},
square matrix/.default=1.2cm
}
\matrix[square matrix=1.4em] {
|[fill=green]|\color[rgb]{0,0,0}\textbf{CO}%
 &|[fill=green]|\color[rgb]{0,0,0}\textbf{CO}%
 &|[fill=white]|\color[gray]{0.5}WA%
 &|[fill=green]|\color[gray]{0.75} FO%
 &|[fill=green]|\color[gray]{0.75} FO%
 &|[fill=green]|\color[gray]{0.75} FO%
 &|[fill=green]|\color[gray]{0.75} FO%
 &|[fill=green]|\color[gray]{0.75} FO%
 &|[fill=white]|\color[gray]{0.5}WA%
 &|[fill=green]|\color[gray]{0.75} FO%
\\
|[fill=green]|\color[rgb]{0,0,0}\textbf{CO}%
 &|[fill=white]|\color[gray]{0.5}WA%
 &|[fill=white]|\color[gray]{0.5}WA%
 &|[fill=green]|\color[gray]{0.75} FO%
 &|[fill=green]|\color[gray]{0.75} FO%
 &|[fill=green]|\color[gray]{0.75} FO%
 &|[fill=green]|\color[gray]{0.75} FO%
 &|[fill=green]|\color[gray]{0.75} FO%
 &|[fill=green]|\color[gray]{0.75} FO%
 &|[fill=white]|\color[gray]{0.5}WA%
\\
|[fill=cyan]|\color[rgb]{1,0,0}\textbf{02}%
 &|[fill=green]|\color[gray]{0.75} FO%
 &|[fill=green]|\color[gray]{0.75} FO%
 &|[fill=green]|\color[gray]{0.75} FO%
 &|[fill=green]|\color[gray]{0.75} FO%
 &|[fill=green]|\color[gray]{0.75} FO%
 &|[fill=green]|\color[gray]{0.75} FO%
 &|[fill=green]|\color[gray]{0.75} FO%
 &|[fill=green]|\color[gray]{0.75} FO%
 &|[fill=green]|\color[gray]{0.75} FO%
\\
|[fill=green]|\color[gray]{0.75} FO%
 &|[fill=green]|\color[gray]{0.75} FO%
 &|[fill=white]|\color[gray]{0.5}WA%
 &|[fill=white]|\color[gray]{0.5}WA%
 &|[fill=white]|\color[gray]{0.5}WA%
 &|[fill=cyan]|\color[rgb]{1,0,0}\textbf{01}%
 &|[fill=white]|\color[gray]{0.5}WA%
 &|[fill=white]|\color[gray]{0.5}WA%
 &|[fill=white]|\color[gray]{0.5}WA%
 &|[fill=green]|\color[gray]{0.75} FO%
\\
|[fill=green]|\color[gray]{0.75} FO%
 &|[fill=green]|\color[gray]{0.75} FO%
 &|[fill=green]|\color[gray]{0.75} FO%
 &|[fill=green]|\color[rgb]{1,0,0}\textbf{BU}%
 &|[fill=green]|\color[rgb]{0,0,0}\textbf{CO}%
 &|[fill=green]|\color[rgb]{0,0,0}\textbf{CO}%
 &|[fill=green]|\color[rgb]{0,0,0}\textbf{CO}%
 &|[fill=green]|\color[rgb]{1,0,0}\textbf{BU}%
 &|[fill=green]|\color[gray]{0.75} FO%
 &|[fill=green]|\color[gray]{0.75} FO%
\\
|[fill=green]|\color[gray]{0.75} FO%
 &|[fill=green]|\color[gray]{0.75} FO%
 &|[fill=white]|\color[gray]{0.5}WA%
 &|[fill=white]|\color[gray]{0.5}WA%
 &|[fill=green]|\color[rgb]{1,0,0}\textbf{BU}%
 &|[fill=green]|\color[rgb]{0,0,0}\textbf{CO}%
 &|[fill=green]|\color[rgb]{1,0,0}\textbf{BU}%
 &|[fill=green]|\color[gray]{0.75} FO%
 &|[fill=green]|\color[gray]{0.75} FO%
 &|[fill=green]|\color[gray]{0.75} FO%
\\
|[fill=green]|\color[gray]{0.75} FO%
 &|[fill=green]|\color[gray]{0.75} FO%
 &|[fill=green]|\color[gray]{0.75} FO%
 &|[fill=green]|\color[gray]{0.75} FO%
 &|[fill=white]|\color[gray]{0.5}WA%
 &|[fill=green]|\color[rgb]{1,0,0}\textbf{BU}%
 &|[fill=green]|\color[gray]{0.75} FO%
 &|[fill=white]|\color[gray]{0.5}WA%
 &|[fill=green]|\color[gray]{0.75} FO%
 &|[fill=green]|\color[gray]{0.75} FO%
\\
|[fill=white]|\color[gray]{0.5}WA%
 &|[fill=green]|\color[gray]{0.75} FO%
 &|[fill=green]|\color[gray]{0.75} FO%
 &|[fill=green]|\color[gray]{0.75} FO%
 &|[fill=white]|\color[gray]{0.5}WA%
 &|[fill=green]|\color[gray]{0.75} FO%
 &|[fill=green]|\color[rgb]{1,0,0}\textbf{BU}%
 &|[fill=white]|\color[gray]{0.5}WA%
 &|[fill=green]|\color[gray]{0.75} FO%
 &|[fill=white]|\color[gray]{0.5}WA%
\\
|[fill=green]|\color[gray]{0.75} FO%
 &|[fill=white]|\color[gray]{0.5}WA%
 &|[fill=green]|\color[gray]{0.75} FO%
 &|[fill=green]|\color[gray]{0.75} FO%
 &|[fill=white]|\color[gray]{0.5}WA%
 &|[fill=green]|\color[rgb]{1,0,0}\textbf{BU}%
 &|[fill=green]|\color[rgb]{0,0,0}\textbf{CO}%
 &|[fill=white]|\color[gray]{0.5}WA%
 &|[fill=green]|\color[gray]{0.75} FO%
 &|[fill=green]|\color[gray]{0.75} FO%
\\
|[fill=green]|\color[gray]{0.75} FO%
 &|[fill=green]|\color[gray]{0.75} FO%
 &|[fill=green]|\color[gray]{0.75} FO%
 &|[fill=green]|\color[gray]{0.75} FO%
 &|[fill=green]|\color[rgb]{1,0,0}\textbf{BU}%
 &|[fill=green]|\color[rgb]{0,0,0}\textbf{CO}%
 &|[fill=green]|\color[rgb]{0,0,0}\textbf{CO}%
 &|[fill=green]|\color[rgb]{0,0,0}\textbf{CO}%
 &|[fill=green]|\color[rgb]{1,0,0}\textbf{BU}%
 &|[fill=green]|\color[gray]{0.75} FO%
\\
|[fill=green]|\color[gray]{0.75} FO%
 &|[fill=green]|\color[gray]{0.75} FO%
 &|[fill=green]|\color[gray]{0.75} FO%
 &|[fill=green]|\color[gray]{0.75} FO%
 &|[fill=green]|\color[gray]{0.75} FO%
 &|[fill=green]|\color[rgb]{1,0,0}\textbf{BU}%
 &|[fill=green]|\color[rgb]{0,0,0}\textbf{CO}%
 &|[fill=green]|\color[rgb]{1,0,0}\textbf{BU}%
 &|[fill=green]|\color[gray]{0.75} FO%
 &|[fill=green]|\color[gray]{0.75} FO%
\\
};
\end{tikzpicture}\\
At time 3: Water spot (2|4):
\\
\begin{tikzpicture}
\tikzset{square matrix/.style={
matrix of nodes,
column sep=-\pgflinewidth, row sep=-\pgflinewidth,
nodes={draw,
minimum height=#1,
anchor=center,
text width=#1,
align=center,
inner sep=0pt
},
},
square matrix/.default=1.2cm
}
\matrix[square matrix=1.4em] {
|[fill=green]|\color[rgb]{0,0,0}\textbf{CO}%
 &|[fill=green]|\color[rgb]{0,0,0}\textbf{CO}%
 &|[fill=white]|\color[gray]{0.5}WA%
 &|[fill=green]|\color[gray]{0.75} FO%
 &|[fill=green]|\color[gray]{0.75} FO%
 &|[fill=green]|\color[gray]{0.75} FO%
 &|[fill=green]|\color[gray]{0.75} FO%
 &|[fill=green]|\color[gray]{0.75} FO%
 &|[fill=white]|\color[gray]{0.5}WA%
 &|[fill=green]|\color[gray]{0.75} FO%
\\
|[fill=green]|\color[rgb]{0,0,0}\textbf{CO}%
 &|[fill=white]|\color[gray]{0.5}WA%
 &|[fill=white]|\color[gray]{0.5}WA%
 &|[fill=green]|\color[gray]{0.75} FO%
 &|[fill=green]|\color[gray]{0.75} FO%
 &|[fill=green]|\color[gray]{0.75} FO%
 &|[fill=green]|\color[gray]{0.75} FO%
 &|[fill=green]|\color[gray]{0.75} FO%
 &|[fill=green]|\color[gray]{0.75} FO%
 &|[fill=white]|\color[gray]{0.5}WA%
\\
|[fill=cyan]|\color[rgb]{1,0,0}\textbf{02}%
 &|[fill=green]|\color[gray]{0.75} FO%
 &|[fill=green]|\color[gray]{0.75} FO%
 &|[fill=green]|\color[gray]{0.75} FO%
 &|[fill=green]|\color[gray]{0.75} FO%
 &|[fill=green]|\color[gray]{0.75} FO%
 &|[fill=green]|\color[gray]{0.75} FO%
 &|[fill=green]|\color[gray]{0.75} FO%
 &|[fill=green]|\color[gray]{0.75} FO%
 &|[fill=green]|\color[gray]{0.75} FO%
\\
|[fill=green]|\color[gray]{0.75} FO%
 &|[fill=green]|\color[gray]{0.75} FO%
 &|[fill=white]|\color[gray]{0.5}WA%
 &|[fill=white]|\color[gray]{0.5}WA%
 &|[fill=white]|\color[gray]{0.5}WA%
 &|[fill=cyan]|\color[rgb]{1,0,0}\textbf{01}%
 &|[fill=white]|\color[gray]{0.5}WA%
 &|[fill=white]|\color[gray]{0.5}WA%
 &|[fill=white]|\color[gray]{0.5}WA%
 &|[fill=green]|\color[gray]{0.75} FO%
\\
|[fill=green]|\color[gray]{0.75} FO%
 &|[fill=green]|\color[gray]{0.75} FO%
 &|[fill=cyan]|\color[rgb]{1,0,0}\textbf{03}%
 &|[fill=green]|\color[rgb]{0,0,0}\textbf{CO}%
 &|[fill=green]|\color[rgb]{0,0,0}\textbf{CO}%
 &|[fill=green]|\color[rgb]{0,0,0}\textbf{CO}%
 &|[fill=green]|\color[rgb]{0,0,0}\textbf{CO}%
 &|[fill=green]|\color[rgb]{0,0,0}\textbf{CO}%
 &|[fill=green]|\color[rgb]{1,0,0}\textbf{BU}%
 &|[fill=green]|\color[gray]{0.75} FO%
\\
|[fill=green]|\color[gray]{0.75} FO%
 &|[fill=green]|\color[gray]{0.75} FO%
 &|[fill=white]|\color[gray]{0.5}WA%
 &|[fill=white]|\color[gray]{0.5}WA%
 &|[fill=green]|\color[rgb]{0,0,0}\textbf{CO}%
 &|[fill=green]|\color[rgb]{0,0,0}\textbf{CO}%
 &|[fill=green]|\color[rgb]{0,0,0}\textbf{CO}%
 &|[fill=green]|\color[rgb]{1,0,0}\textbf{BU}%
 &|[fill=green]|\color[gray]{0.75} FO%
 &|[fill=green]|\color[gray]{0.75} FO%
\\
|[fill=green]|\color[gray]{0.75} FO%
 &|[fill=green]|\color[gray]{0.75} FO%
 &|[fill=green]|\color[gray]{0.75} FO%
 &|[fill=green]|\color[gray]{0.75} FO%
 &|[fill=white]|\color[gray]{0.5}WA%
 &|[fill=green]|\color[rgb]{0,0,0}\textbf{CO}%
 &|[fill=green]|\color[rgb]{1,0,0}\textbf{BU}%
 &|[fill=white]|\color[gray]{0.5}WA%
 &|[fill=green]|\color[gray]{0.75} FO%
 &|[fill=green]|\color[gray]{0.75} FO%
\\
|[fill=white]|\color[gray]{0.5}WA%
 &|[fill=green]|\color[gray]{0.75} FO%
 &|[fill=green]|\color[gray]{0.75} FO%
 &|[fill=green]|\color[gray]{0.75} FO%
 &|[fill=white]|\color[gray]{0.5}WA%
 &|[fill=green]|\color[rgb]{1,0,0}\textbf{BU}%
 &|[fill=green]|\color[rgb]{0,0,0}\textbf{CO}%
 &|[fill=white]|\color[gray]{0.5}WA%
 &|[fill=green]|\color[gray]{0.75} FO%
 &|[fill=white]|\color[gray]{0.5}WA%
\\
|[fill=green]|\color[gray]{0.75} FO%
 &|[fill=white]|\color[gray]{0.5}WA%
 &|[fill=green]|\color[gray]{0.75} FO%
 &|[fill=green]|\color[gray]{0.75} FO%
 &|[fill=white]|\color[gray]{0.5}WA%
 &|[fill=green]|\color[rgb]{0,0,0}\textbf{CO}%
 &|[fill=green]|\color[rgb]{0,0,0}\textbf{CO}%
 &|[fill=white]|\color[gray]{0.5}WA%
 &|[fill=green]|\color[rgb]{1,0,0}\textbf{BU}%
 &|[fill=green]|\color[gray]{0.75} FO%
\\
|[fill=green]|\color[gray]{0.75} FO%
 &|[fill=green]|\color[gray]{0.75} FO%
 &|[fill=green]|\color[gray]{0.75} FO%
 &|[fill=green]|\color[rgb]{1,0,0}\textbf{BU}%
 &|[fill=green]|\color[rgb]{0,0,0}\textbf{CO}%
 &|[fill=green]|\color[rgb]{0,0,0}\textbf{CO}%
 &|[fill=green]|\color[rgb]{0,0,0}\textbf{CO}%
 &|[fill=green]|\color[rgb]{0,0,0}\textbf{CO}%
 &|[fill=green]|\color[rgb]{0,0,0}\textbf{CO}%
 &|[fill=green]|\color[rgb]{1,0,0}\textbf{BU}%
\\
|[fill=green]|\color[gray]{0.75} FO%
 &|[fill=green]|\color[gray]{0.75} FO%
 &|[fill=green]|\color[gray]{0.75} FO%
 &|[fill=green]|\color[gray]{0.75} FO%
 &|[fill=green]|\color[rgb]{1,0,0}\textbf{BU}%
 &|[fill=green]|\color[rgb]{0,0,0}\textbf{CO}%
 &|[fill=green]|\color[rgb]{0,0,0}\textbf{CO}%
 &|[fill=green]|\color[rgb]{0,0,0}\textbf{CO}%
 &|[fill=green]|\color[rgb]{1,0,0}\textbf{BU}%
 &|[fill=green]|\color[gray]{0.75} FO%
\\
};
\end{tikzpicture}\\
At time 4: Water spot (9|4):
\\
\begin{tikzpicture}
\tikzset{square matrix/.style={
matrix of nodes,
column sep=-\pgflinewidth, row sep=-\pgflinewidth,
nodes={draw,
minimum height=#1,
anchor=center,
text width=#1,
align=center,
inner sep=0pt
},
},
square matrix/.default=1.2cm
}
\matrix[square matrix=1.4em] {
|[fill=green]|\color[rgb]{0,0,0}\textbf{CO}%
 &|[fill=green]|\color[rgb]{0,0,0}\textbf{CO}%
 &|[fill=white]|\color[gray]{0.5}WA%
 &|[fill=green]|\color[gray]{0.75} FO%
 &|[fill=green]|\color[gray]{0.75} FO%
 &|[fill=green]|\color[gray]{0.75} FO%
 &|[fill=green]|\color[gray]{0.75} FO%
 &|[fill=green]|\color[gray]{0.75} FO%
 &|[fill=white]|\color[gray]{0.5}WA%
 &|[fill=green]|\color[gray]{0.75} FO%
\\
|[fill=green]|\color[rgb]{0,0,0}\textbf{CO}%
 &|[fill=white]|\color[gray]{0.5}WA%
 &|[fill=white]|\color[gray]{0.5}WA%
 &|[fill=green]|\color[gray]{0.75} FO%
 &|[fill=green]|\color[gray]{0.75} FO%
 &|[fill=green]|\color[gray]{0.75} FO%
 &|[fill=green]|\color[gray]{0.75} FO%
 &|[fill=green]|\color[gray]{0.75} FO%
 &|[fill=green]|\color[gray]{0.75} FO%
 &|[fill=white]|\color[gray]{0.5}WA%
\\
|[fill=cyan]|\color[rgb]{1,0,0}\textbf{02}%
 &|[fill=green]|\color[gray]{0.75} FO%
 &|[fill=green]|\color[gray]{0.75} FO%
 &|[fill=green]|\color[gray]{0.75} FO%
 &|[fill=green]|\color[gray]{0.75} FO%
 &|[fill=green]|\color[gray]{0.75} FO%
 &|[fill=green]|\color[gray]{0.75} FO%
 &|[fill=green]|\color[gray]{0.75} FO%
 &|[fill=green]|\color[gray]{0.75} FO%
 &|[fill=green]|\color[gray]{0.75} FO%
\\
|[fill=green]|\color[gray]{0.75} FO%
 &|[fill=green]|\color[gray]{0.75} FO%
 &|[fill=white]|\color[gray]{0.5}WA%
 &|[fill=white]|\color[gray]{0.5}WA%
 &|[fill=white]|\color[gray]{0.5}WA%
 &|[fill=cyan]|\color[rgb]{1,0,0}\textbf{01}%
 &|[fill=white]|\color[gray]{0.5}WA%
 &|[fill=white]|\color[gray]{0.5}WA%
 &|[fill=white]|\color[gray]{0.5}WA%
 &|[fill=green]|\color[gray]{0.75} FO%
\\
|[fill=green]|\color[gray]{0.75} FO%
 &|[fill=green]|\color[gray]{0.75} FO%
 &|[fill=cyan]|\color[rgb]{1,0,0}\textbf{03}%
 &|[fill=green]|\color[rgb]{0,0,0}\textbf{CO}%
 &|[fill=green]|\color[rgb]{0,0,0}\textbf{CO}%
 &|[fill=green]|\color[rgb]{0,0,0}\textbf{CO}%
 &|[fill=green]|\color[rgb]{0,0,0}\textbf{CO}%
 &|[fill=green]|\color[rgb]{0,0,0}\textbf{CO}%
 &|[fill=green]|\color[rgb]{0,0,0}\textbf{CO}%
 &|[fill=cyan]|\color[rgb]{1,0,0}\textbf{04}%
\\
|[fill=green]|\color[gray]{0.75} FO%
 &|[fill=green]|\color[gray]{0.75} FO%
 &|[fill=white]|\color[gray]{0.5}WA%
 &|[fill=white]|\color[gray]{0.5}WA%
 &|[fill=green]|\color[rgb]{0,0,0}\textbf{CO}%
 &|[fill=green]|\color[rgb]{0,0,0}\textbf{CO}%
 &|[fill=green]|\color[rgb]{0,0,0}\textbf{CO}%
 &|[fill=green]|\color[rgb]{0,0,0}\textbf{CO}%
 &|[fill=green]|\color[rgb]{1,0,0}\textbf{BU}%
 &|[fill=green]|\color[gray]{0.75} FO%
\\
|[fill=green]|\color[gray]{0.75} FO%
 &|[fill=green]|\color[gray]{0.75} FO%
 &|[fill=green]|\color[gray]{0.75} FO%
 &|[fill=green]|\color[gray]{0.75} FO%
 &|[fill=white]|\color[gray]{0.5}WA%
 &|[fill=green]|\color[rgb]{0,0,0}\textbf{CO}%
 &|[fill=green]|\color[rgb]{0,0,0}\textbf{CO}%
 &|[fill=white]|\color[gray]{0.5}WA%
 &|[fill=green]|\color[gray]{0.75} FO%
 &|[fill=green]|\color[gray]{0.75} FO%
\\
|[fill=white]|\color[gray]{0.5}WA%
 &|[fill=green]|\color[gray]{0.75} FO%
 &|[fill=green]|\color[gray]{0.75} FO%
 &|[fill=green]|\color[gray]{0.75} FO%
 &|[fill=white]|\color[gray]{0.5}WA%
 &|[fill=green]|\color[rgb]{0,0,0}\textbf{CO}%
 &|[fill=green]|\color[rgb]{0,0,0}\textbf{CO}%
 &|[fill=white]|\color[gray]{0.5}WA%
 &|[fill=green]|\color[rgb]{1,0,0}\textbf{BU}%
 &|[fill=white]|\color[gray]{0.5}WA%
\\
|[fill=green]|\color[gray]{0.75} FO%
 &|[fill=white]|\color[gray]{0.5}WA%
 &|[fill=green]|\color[gray]{0.75} FO%
 &|[fill=green]|\color[rgb]{1,0,0}\textbf{BU}%
 &|[fill=white]|\color[gray]{0.5}WA%
 &|[fill=green]|\color[rgb]{0,0,0}\textbf{CO}%
 &|[fill=green]|\color[rgb]{0,0,0}\textbf{CO}%
 &|[fill=white]|\color[gray]{0.5}WA%
 &|[fill=green]|\color[rgb]{0,0,0}\textbf{CO}%
 &|[fill=green]|\color[rgb]{1,0,0}\textbf{BU}%
\\
|[fill=green]|\color[gray]{0.75} FO%
 &|[fill=green]|\color[gray]{0.75} FO%
 &|[fill=green]|\color[rgb]{1,0,0}\textbf{BU}%
 &|[fill=green]|\color[rgb]{0,0,0}\textbf{CO}%
 &|[fill=green]|\color[rgb]{0,0,0}\textbf{CO}%
 &|[fill=green]|\color[rgb]{0,0,0}\textbf{CO}%
 &|[fill=green]|\color[rgb]{0,0,0}\textbf{CO}%
 &|[fill=green]|\color[rgb]{0,0,0}\textbf{CO}%
 &|[fill=green]|\color[rgb]{0,0,0}\textbf{CO}%
 &|[fill=green]|\color[rgb]{0,0,0}\textbf{CO}%
\\
|[fill=green]|\color[gray]{0.75} FO%
 &|[fill=green]|\color[gray]{0.75} FO%
 &|[fill=green]|\color[gray]{0.75} FO%
 &|[fill=green]|\color[rgb]{1,0,0}\textbf{BU}%
 &|[fill=green]|\color[rgb]{0,0,0}\textbf{CO}%
 &|[fill=green]|\color[rgb]{0,0,0}\textbf{CO}%
 &|[fill=green]|\color[rgb]{0,0,0}\textbf{CO}%
 &|[fill=green]|\color[rgb]{0,0,0}\textbf{CO}%
 &|[fill=green]|\color[rgb]{0,0,0}\textbf{CO}%
 &|[fill=green]|\color[rgb]{1,0,0}\textbf{BU}%
\\
};
\end{tikzpicture}\\
At time 5: Water spot (1|9):
\\
\begin{tikzpicture}
\tikzset{square matrix/.style={
matrix of nodes,
column sep=-\pgflinewidth, row sep=-\pgflinewidth,
nodes={draw,
minimum height=#1,
anchor=center,
text width=#1,
align=center,
inner sep=0pt
},
},
square matrix/.default=1.2cm
}
\matrix[square matrix=1.4em] {
|[fill=green]|\color[rgb]{0,0,0}\textbf{CO}%
 &|[fill=green]|\color[rgb]{0,0,0}\textbf{CO}%
 &|[fill=white]|\color[gray]{0.5}WA%
 &|[fill=green]|\color[gray]{0.75} FO%
 &|[fill=green]|\color[gray]{0.75} FO%
 &|[fill=green]|\color[gray]{0.75} FO%
 &|[fill=green]|\color[gray]{0.75} FO%
 &|[fill=green]|\color[gray]{0.75} FO%
 &|[fill=white]|\color[gray]{0.5}WA%
 &|[fill=green]|\color[gray]{0.75} FO%
\\
|[fill=green]|\color[rgb]{0,0,0}\textbf{CO}%
 &|[fill=white]|\color[gray]{0.5}WA%
 &|[fill=white]|\color[gray]{0.5}WA%
 &|[fill=green]|\color[gray]{0.75} FO%
 &|[fill=green]|\color[gray]{0.75} FO%
 &|[fill=green]|\color[gray]{0.75} FO%
 &|[fill=green]|\color[gray]{0.75} FO%
 &|[fill=green]|\color[gray]{0.75} FO%
 &|[fill=green]|\color[gray]{0.75} FO%
 &|[fill=white]|\color[gray]{0.5}WA%
\\
|[fill=cyan]|\color[rgb]{1,0,0}\textbf{02}%
 &|[fill=green]|\color[gray]{0.75} FO%
 &|[fill=green]|\color[gray]{0.75} FO%
 &|[fill=green]|\color[gray]{0.75} FO%
 &|[fill=green]|\color[gray]{0.75} FO%
 &|[fill=green]|\color[gray]{0.75} FO%
 &|[fill=green]|\color[gray]{0.75} FO%
 &|[fill=green]|\color[gray]{0.75} FO%
 &|[fill=green]|\color[gray]{0.75} FO%
 &|[fill=green]|\color[gray]{0.75} FO%
\\
|[fill=green]|\color[gray]{0.75} FO%
 &|[fill=green]|\color[gray]{0.75} FO%
 &|[fill=white]|\color[gray]{0.5}WA%
 &|[fill=white]|\color[gray]{0.5}WA%
 &|[fill=white]|\color[gray]{0.5}WA%
 &|[fill=cyan]|\color[rgb]{1,0,0}\textbf{01}%
 &|[fill=white]|\color[gray]{0.5}WA%
 &|[fill=white]|\color[gray]{0.5}WA%
 &|[fill=white]|\color[gray]{0.5}WA%
 &|[fill=green]|\color[gray]{0.75} FO%
\\
|[fill=green]|\color[gray]{0.75} FO%
 &|[fill=green]|\color[gray]{0.75} FO%
 &|[fill=cyan]|\color[rgb]{1,0,0}\textbf{03}%
 &|[fill=green]|\color[rgb]{0,0,0}\textbf{CO}%
 &|[fill=green]|\color[rgb]{0,0,0}\textbf{CO}%
 &|[fill=green]|\color[rgb]{0,0,0}\textbf{CO}%
 &|[fill=green]|\color[rgb]{0,0,0}\textbf{CO}%
 &|[fill=green]|\color[rgb]{0,0,0}\textbf{CO}%
 &|[fill=green]|\color[rgb]{0,0,0}\textbf{CO}%
 &|[fill=cyan]|\color[rgb]{1,0,0}\textbf{04}%
\\
|[fill=green]|\color[gray]{0.75} FO%
 &|[fill=green]|\color[gray]{0.75} FO%
 &|[fill=white]|\color[gray]{0.5}WA%
 &|[fill=white]|\color[gray]{0.5}WA%
 &|[fill=green]|\color[rgb]{0,0,0}\textbf{CO}%
 &|[fill=green]|\color[rgb]{0,0,0}\textbf{CO}%
 &|[fill=green]|\color[rgb]{0,0,0}\textbf{CO}%
 &|[fill=green]|\color[rgb]{0,0,0}\textbf{CO}%
 &|[fill=green]|\color[rgb]{0,0,0}\textbf{CO}%
 &|[fill=green]|\color[rgb]{1,0,0}\textbf{BU}%
\\
|[fill=green]|\color[gray]{0.75} FO%
 &|[fill=green]|\color[gray]{0.75} FO%
 &|[fill=green]|\color[gray]{0.75} FO%
 &|[fill=green]|\color[gray]{0.75} FO%
 &|[fill=white]|\color[gray]{0.5}WA%
 &|[fill=green]|\color[rgb]{0,0,0}\textbf{CO}%
 &|[fill=green]|\color[rgb]{0,0,0}\textbf{CO}%
 &|[fill=white]|\color[gray]{0.5}WA%
 &|[fill=green]|\color[rgb]{1,0,0}\textbf{BU}%
 &|[fill=green]|\color[gray]{0.75} FO%
\\
|[fill=white]|\color[gray]{0.5}WA%
 &|[fill=green]|\color[gray]{0.75} FO%
 &|[fill=green]|\color[gray]{0.75} FO%
 &|[fill=green]|\color[rgb]{1,0,0}\textbf{BU}%
 &|[fill=white]|\color[gray]{0.5}WA%
 &|[fill=green]|\color[rgb]{0,0,0}\textbf{CO}%
 &|[fill=green]|\color[rgb]{0,0,0}\textbf{CO}%
 &|[fill=white]|\color[gray]{0.5}WA%
 &|[fill=green]|\color[rgb]{0,0,0}\textbf{CO}%
 &|[fill=white]|\color[gray]{0.5}WA%
\\
|[fill=green]|\color[gray]{0.75} FO%
 &|[fill=white]|\color[gray]{0.5}WA%
 &|[fill=green]|\color[rgb]{1,0,0}\textbf{BU}%
 &|[fill=green]|\color[rgb]{0,0,0}\textbf{CO}%
 &|[fill=white]|\color[gray]{0.5}WA%
 &|[fill=green]|\color[rgb]{0,0,0}\textbf{CO}%
 &|[fill=green]|\color[rgb]{0,0,0}\textbf{CO}%
 &|[fill=white]|\color[gray]{0.5}WA%
 &|[fill=green]|\color[rgb]{0,0,0}\textbf{CO}%
 &|[fill=green]|\color[rgb]{0,0,0}\textbf{CO}%
\\
|[fill=green]|\color[gray]{0.75} FO%
 &|[fill=cyan]|\color[rgb]{1,0,0}\textbf{05}%
 &|[fill=green]|\color[rgb]{0,0,0}\textbf{CO}%
 &|[fill=green]|\color[rgb]{0,0,0}\textbf{CO}%
 &|[fill=green]|\color[rgb]{0,0,0}\textbf{CO}%
 &|[fill=green]|\color[rgb]{0,0,0}\textbf{CO}%
 &|[fill=green]|\color[rgb]{0,0,0}\textbf{CO}%
 &|[fill=green]|\color[rgb]{0,0,0}\textbf{CO}%
 &|[fill=green]|\color[rgb]{0,0,0}\textbf{CO}%
 &|[fill=green]|\color[rgb]{0,0,0}\textbf{CO}%
\\
|[fill=green]|\color[gray]{0.75} FO%
 &|[fill=green]|\color[gray]{0.75} FO%
 &|[fill=green]|\color[rgb]{1,0,0}\textbf{BU}%
 &|[fill=green]|\color[rgb]{0,0,0}\textbf{CO}%
 &|[fill=green]|\color[rgb]{0,0,0}\textbf{CO}%
 &|[fill=green]|\color[rgb]{0,0,0}\textbf{CO}%
 &|[fill=green]|\color[rgb]{0,0,0}\textbf{CO}%
 &|[fill=green]|\color[rgb]{0,0,0}\textbf{CO}%
 &|[fill=green]|\color[rgb]{0,0,0}\textbf{CO}%
 &|[fill=green]|\color[rgb]{0,0,0}\textbf{CO}%
\\
};
\end{tikzpicture}\\
At time 6: Water spot (1|10):
\\
\begin{tikzpicture}
\tikzset{square matrix/.style={
matrix of nodes,
column sep=-\pgflinewidth, row sep=-\pgflinewidth,
nodes={draw,
minimum height=#1,
anchor=center,
text width=#1,
align=center,
inner sep=0pt
},
},
square matrix/.default=1.2cm
}
\matrix[square matrix=1.4em] {
|[fill=green]|\color[rgb]{0,0,0}\textbf{CO}%
 &|[fill=green]|\color[rgb]{0,0,0}\textbf{CO}%
 &|[fill=white]|\color[gray]{0.5}WA%
 &|[fill=green]|\color[gray]{0.75} FO%
 &|[fill=green]|\color[gray]{0.75} FO%
 &|[fill=green]|\color[gray]{0.75} FO%
 &|[fill=green]|\color[gray]{0.75} FO%
 &|[fill=green]|\color[gray]{0.75} FO%
 &|[fill=white]|\color[gray]{0.5}WA%
 &|[fill=green]|\color[gray]{0.75} FO%
\\
|[fill=green]|\color[rgb]{0,0,0}\textbf{CO}%
 &|[fill=white]|\color[gray]{0.5}WA%
 &|[fill=white]|\color[gray]{0.5}WA%
 &|[fill=green]|\color[gray]{0.75} FO%
 &|[fill=green]|\color[gray]{0.75} FO%
 &|[fill=green]|\color[gray]{0.75} FO%
 &|[fill=green]|\color[gray]{0.75} FO%
 &|[fill=green]|\color[gray]{0.75} FO%
 &|[fill=green]|\color[gray]{0.75} FO%
 &|[fill=white]|\color[gray]{0.5}WA%
\\
|[fill=cyan]|\color[rgb]{1,0,0}\textbf{02}%
 &|[fill=green]|\color[gray]{0.75} FO%
 &|[fill=green]|\color[gray]{0.75} FO%
 &|[fill=green]|\color[gray]{0.75} FO%
 &|[fill=green]|\color[gray]{0.75} FO%
 &|[fill=green]|\color[gray]{0.75} FO%
 &|[fill=green]|\color[gray]{0.75} FO%
 &|[fill=green]|\color[gray]{0.75} FO%
 &|[fill=green]|\color[gray]{0.75} FO%
 &|[fill=green]|\color[gray]{0.75} FO%
\\
|[fill=green]|\color[gray]{0.75} FO%
 &|[fill=green]|\color[gray]{0.75} FO%
 &|[fill=white]|\color[gray]{0.5}WA%
 &|[fill=white]|\color[gray]{0.5}WA%
 &|[fill=white]|\color[gray]{0.5}WA%
 &|[fill=cyan]|\color[rgb]{1,0,0}\textbf{01}%
 &|[fill=white]|\color[gray]{0.5}WA%
 &|[fill=white]|\color[gray]{0.5}WA%
 &|[fill=white]|\color[gray]{0.5}WA%
 &|[fill=green]|\color[gray]{0.75} FO%
\\
|[fill=green]|\color[gray]{0.75} FO%
 &|[fill=green]|\color[gray]{0.75} FO%
 &|[fill=cyan]|\color[rgb]{1,0,0}\textbf{03}%
 &|[fill=green]|\color[rgb]{0,0,0}\textbf{CO}%
 &|[fill=green]|\color[rgb]{0,0,0}\textbf{CO}%
 &|[fill=green]|\color[rgb]{0,0,0}\textbf{CO}%
 &|[fill=green]|\color[rgb]{0,0,0}\textbf{CO}%
 &|[fill=green]|\color[rgb]{0,0,0}\textbf{CO}%
 &|[fill=green]|\color[rgb]{0,0,0}\textbf{CO}%
 &|[fill=cyan]|\color[rgb]{1,0,0}\textbf{04}%
\\
|[fill=green]|\color[gray]{0.75} FO%
 &|[fill=green]|\color[gray]{0.75} FO%
 &|[fill=white]|\color[gray]{0.5}WA%
 &|[fill=white]|\color[gray]{0.5}WA%
 &|[fill=green]|\color[rgb]{0,0,0}\textbf{CO}%
 &|[fill=green]|\color[rgb]{0,0,0}\textbf{CO}%
 &|[fill=green]|\color[rgb]{0,0,0}\textbf{CO}%
 &|[fill=green]|\color[rgb]{0,0,0}\textbf{CO}%
 &|[fill=green]|\color[rgb]{0,0,0}\textbf{CO}%
 &|[fill=green]|\color[rgb]{0,0,0}\textbf{CO}%
\\
|[fill=green]|\color[gray]{0.75} FO%
 &|[fill=green]|\color[gray]{0.75} FO%
 &|[fill=green]|\color[gray]{0.75} FO%
 &|[fill=green]|\color[rgb]{1,0,0}\textbf{BU}%
 &|[fill=white]|\color[gray]{0.5}WA%
 &|[fill=green]|\color[rgb]{0,0,0}\textbf{CO}%
 &|[fill=green]|\color[rgb]{0,0,0}\textbf{CO}%
 &|[fill=white]|\color[gray]{0.5}WA%
 &|[fill=green]|\color[rgb]{0,0,0}\textbf{CO}%
 &|[fill=green]|\color[rgb]{1,0,0}\textbf{BU}%
\\
|[fill=white]|\color[gray]{0.5}WA%
 &|[fill=green]|\color[gray]{0.75} FO%
 &|[fill=green]|\color[rgb]{1,0,0}\textbf{BU}%
 &|[fill=green]|\color[rgb]{0,0,0}\textbf{CO}%
 &|[fill=white]|\color[gray]{0.5}WA%
 &|[fill=green]|\color[rgb]{0,0,0}\textbf{CO}%
 &|[fill=green]|\color[rgb]{0,0,0}\textbf{CO}%
 &|[fill=white]|\color[gray]{0.5}WA%
 &|[fill=green]|\color[rgb]{0,0,0}\textbf{CO}%
 &|[fill=white]|\color[gray]{0.5}WA%
\\
|[fill=green]|\color[gray]{0.75} FO%
 &|[fill=white]|\color[gray]{0.5}WA%
 &|[fill=green]|\color[rgb]{0,0,0}\textbf{CO}%
 &|[fill=green]|\color[rgb]{0,0,0}\textbf{CO}%
 &|[fill=white]|\color[gray]{0.5}WA%
 &|[fill=green]|\color[rgb]{0,0,0}\textbf{CO}%
 &|[fill=green]|\color[rgb]{0,0,0}\textbf{CO}%
 &|[fill=white]|\color[gray]{0.5}WA%
 &|[fill=green]|\color[rgb]{0,0,0}\textbf{CO}%
 &|[fill=green]|\color[rgb]{0,0,0}\textbf{CO}%
\\
|[fill=green]|\color[gray]{0.75} FO%
 &|[fill=cyan]|\color[rgb]{1,0,0}\textbf{05}%
 &|[fill=green]|\color[rgb]{0,0,0}\textbf{CO}%
 &|[fill=green]|\color[rgb]{0,0,0}\textbf{CO}%
 &|[fill=green]|\color[rgb]{0,0,0}\textbf{CO}%
 &|[fill=green]|\color[rgb]{0,0,0}\textbf{CO}%
 &|[fill=green]|\color[rgb]{0,0,0}\textbf{CO}%
 &|[fill=green]|\color[rgb]{0,0,0}\textbf{CO}%
 &|[fill=green]|\color[rgb]{0,0,0}\textbf{CO}%
 &|[fill=green]|\color[rgb]{0,0,0}\textbf{CO}%
\\
|[fill=green]|\color[gray]{0.75} FO%
 &|[fill=cyan]|\color[rgb]{1,0,0}\textbf{06}%
 &|[fill=green]|\color[rgb]{0,0,0}\textbf{CO}%
 &|[fill=green]|\color[rgb]{0,0,0}\textbf{CO}%
 &|[fill=green]|\color[rgb]{0,0,0}\textbf{CO}%
 &|[fill=green]|\color[rgb]{0,0,0}\textbf{CO}%
 &|[fill=green]|\color[rgb]{0,0,0}\textbf{CO}%
 &|[fill=green]|\color[rgb]{0,0,0}\textbf{CO}%
 &|[fill=green]|\color[rgb]{0,0,0}\textbf{CO}%
 &|[fill=green]|\color[rgb]{0,0,0}\textbf{CO}%
\\
};
\end{tikzpicture}\\
At time 7: Water spot (1|7):
\\
\begin{tikzpicture}
\tikzset{square matrix/.style={
matrix of nodes,
column sep=-\pgflinewidth, row sep=-\pgflinewidth,
nodes={draw,
minimum height=#1,
anchor=center,
text width=#1,
align=center,
inner sep=0pt
},
},
square matrix/.default=1.2cm
}
\matrix[square matrix=1.4em] {
|[fill=green]|\color[rgb]{0,0,0}\textbf{CO}%
 &|[fill=green]|\color[rgb]{0,0,0}\textbf{CO}%
 &|[fill=white]|\color[gray]{0.5}WA%
 &|[fill=green]|\color[gray]{0.75} FO%
 &|[fill=green]|\color[gray]{0.75} FO%
 &|[fill=green]|\color[gray]{0.75} FO%
 &|[fill=green]|\color[gray]{0.75} FO%
 &|[fill=green]|\color[gray]{0.75} FO%
 &|[fill=white]|\color[gray]{0.5}WA%
 &|[fill=green]|\color[gray]{0.75} FO%
\\
|[fill=green]|\color[rgb]{0,0,0}\textbf{CO}%
 &|[fill=white]|\color[gray]{0.5}WA%
 &|[fill=white]|\color[gray]{0.5}WA%
 &|[fill=green]|\color[gray]{0.75} FO%
 &|[fill=green]|\color[gray]{0.75} FO%
 &|[fill=green]|\color[gray]{0.75} FO%
 &|[fill=green]|\color[gray]{0.75} FO%
 &|[fill=green]|\color[gray]{0.75} FO%
 &|[fill=green]|\color[gray]{0.75} FO%
 &|[fill=white]|\color[gray]{0.5}WA%
\\
|[fill=cyan]|\color[rgb]{1,0,0}\textbf{02}%
 &|[fill=green]|\color[gray]{0.75} FO%
 &|[fill=green]|\color[gray]{0.75} FO%
 &|[fill=green]|\color[gray]{0.75} FO%
 &|[fill=green]|\color[gray]{0.75} FO%
 &|[fill=green]|\color[gray]{0.75} FO%
 &|[fill=green]|\color[gray]{0.75} FO%
 &|[fill=green]|\color[gray]{0.75} FO%
 &|[fill=green]|\color[gray]{0.75} FO%
 &|[fill=green]|\color[gray]{0.75} FO%
\\
|[fill=green]|\color[gray]{0.75} FO%
 &|[fill=green]|\color[gray]{0.75} FO%
 &|[fill=white]|\color[gray]{0.5}WA%
 &|[fill=white]|\color[gray]{0.5}WA%
 &|[fill=white]|\color[gray]{0.5}WA%
 &|[fill=cyan]|\color[rgb]{1,0,0}\textbf{01}%
 &|[fill=white]|\color[gray]{0.5}WA%
 &|[fill=white]|\color[gray]{0.5}WA%
 &|[fill=white]|\color[gray]{0.5}WA%
 &|[fill=green]|\color[gray]{0.75} FO%
\\
|[fill=green]|\color[gray]{0.75} FO%
 &|[fill=green]|\color[gray]{0.75} FO%
 &|[fill=cyan]|\color[rgb]{1,0,0}\textbf{03}%
 &|[fill=green]|\color[rgb]{0,0,0}\textbf{CO}%
 &|[fill=green]|\color[rgb]{0,0,0}\textbf{CO}%
 &|[fill=green]|\color[rgb]{0,0,0}\textbf{CO}%
 &|[fill=green]|\color[rgb]{0,0,0}\textbf{CO}%
 &|[fill=green]|\color[rgb]{0,0,0}\textbf{CO}%
 &|[fill=green]|\color[rgb]{0,0,0}\textbf{CO}%
 &|[fill=cyan]|\color[rgb]{1,0,0}\textbf{04}%
\\
|[fill=green]|\color[gray]{0.75} FO%
 &|[fill=green]|\color[gray]{0.75} FO%
 &|[fill=white]|\color[gray]{0.5}WA%
 &|[fill=white]|\color[gray]{0.5}WA%
 &|[fill=green]|\color[rgb]{0,0,0}\textbf{CO}%
 &|[fill=green]|\color[rgb]{0,0,0}\textbf{CO}%
 &|[fill=green]|\color[rgb]{0,0,0}\textbf{CO}%
 &|[fill=green]|\color[rgb]{0,0,0}\textbf{CO}%
 &|[fill=green]|\color[rgb]{0,0,0}\textbf{CO}%
 &|[fill=green]|\color[rgb]{0,0,0}\textbf{CO}%
\\
|[fill=green]|\color[gray]{0.75} FO%
 &|[fill=green]|\color[gray]{0.75} FO%
 &|[fill=green]|\color[rgb]{1,0,0}\textbf{BU}%
 &|[fill=green]|\color[rgb]{0,0,0}\textbf{CO}%
 &|[fill=white]|\color[gray]{0.5}WA%
 &|[fill=green]|\color[rgb]{0,0,0}\textbf{CO}%
 &|[fill=green]|\color[rgb]{0,0,0}\textbf{CO}%
 &|[fill=white]|\color[gray]{0.5}WA%
 &|[fill=green]|\color[rgb]{0,0,0}\textbf{CO}%
 &|[fill=green]|\color[rgb]{0,0,0}\textbf{CO}%
\\
|[fill=white]|\color[gray]{0.5}WA%
 &|[fill=cyan]|\color[rgb]{1,0,0}\textbf{07}%
 &|[fill=green]|\color[rgb]{0,0,0}\textbf{CO}%
 &|[fill=green]|\color[rgb]{0,0,0}\textbf{CO}%
 &|[fill=white]|\color[gray]{0.5}WA%
 &|[fill=green]|\color[rgb]{0,0,0}\textbf{CO}%
 &|[fill=green]|\color[rgb]{0,0,0}\textbf{CO}%
 &|[fill=white]|\color[gray]{0.5}WA%
 &|[fill=green]|\color[rgb]{0,0,0}\textbf{CO}%
 &|[fill=white]|\color[gray]{0.5}WA%
\\
|[fill=green]|\color[gray]{0.75} FO%
 &|[fill=white]|\color[gray]{0.5}WA%
 &|[fill=green]|\color[rgb]{0,0,0}\textbf{CO}%
 &|[fill=green]|\color[rgb]{0,0,0}\textbf{CO}%
 &|[fill=white]|\color[gray]{0.5}WA%
 &|[fill=green]|\color[rgb]{0,0,0}\textbf{CO}%
 &|[fill=green]|\color[rgb]{0,0,0}\textbf{CO}%
 &|[fill=white]|\color[gray]{0.5}WA%
 &|[fill=green]|\color[rgb]{0,0,0}\textbf{CO}%
 &|[fill=green]|\color[rgb]{0,0,0}\textbf{CO}%
\\
|[fill=green]|\color[gray]{0.75} FO%
 &|[fill=cyan]|\color[rgb]{1,0,0}\textbf{05}%
 &|[fill=green]|\color[rgb]{0,0,0}\textbf{CO}%
 &|[fill=green]|\color[rgb]{0,0,0}\textbf{CO}%
 &|[fill=green]|\color[rgb]{0,0,0}\textbf{CO}%
 &|[fill=green]|\color[rgb]{0,0,0}\textbf{CO}%
 &|[fill=green]|\color[rgb]{0,0,0}\textbf{CO}%
 &|[fill=green]|\color[rgb]{0,0,0}\textbf{CO}%
 &|[fill=green]|\color[rgb]{0,0,0}\textbf{CO}%
 &|[fill=green]|\color[rgb]{0,0,0}\textbf{CO}%
\\
|[fill=green]|\color[gray]{0.75} FO%
 &|[fill=cyan]|\color[rgb]{1,0,0}\textbf{06}%
 &|[fill=green]|\color[rgb]{0,0,0}\textbf{CO}%
 &|[fill=green]|\color[rgb]{0,0,0}\textbf{CO}%
 &|[fill=green]|\color[rgb]{0,0,0}\textbf{CO}%
 &|[fill=green]|\color[rgb]{0,0,0}\textbf{CO}%
 &|[fill=green]|\color[rgb]{0,0,0}\textbf{CO}%
 &|[fill=green]|\color[rgb]{0,0,0}\textbf{CO}%
 &|[fill=green]|\color[rgb]{0,0,0}\textbf{CO}%
 &|[fill=green]|\color[rgb]{0,0,0}\textbf{CO}%
\\
};
\end{tikzpicture}\\
At time 8: Water spot (1|6):
\\
\begin{tikzpicture}
\tikzset{square matrix/.style={
matrix of nodes,
column sep=-\pgflinewidth, row sep=-\pgflinewidth,
nodes={draw,
minimum height=#1,
anchor=center,
text width=#1,
align=center,
inner sep=0pt
},
},
square matrix/.default=1.2cm
}
\matrix[square matrix=1.4em] {
|[fill=green]|\color[rgb]{0,0,0}\textbf{CO}%
 &|[fill=green]|\color[rgb]{0,0,0}\textbf{CO}%
 &|[fill=white]|\color[gray]{0.5}WA%
 &|[fill=green]|\color[gray]{0.75} FO%
 &|[fill=green]|\color[gray]{0.75} FO%
 &|[fill=green]|\color[gray]{0.75} FO%
 &|[fill=green]|\color[gray]{0.75} FO%
 &|[fill=green]|\color[gray]{0.75} FO%
 &|[fill=white]|\color[gray]{0.5}WA%
 &|[fill=green]|\color[gray]{0.75} FO%
\\
|[fill=green]|\color[rgb]{0,0,0}\textbf{CO}%
 &|[fill=white]|\color[gray]{0.5}WA%
 &|[fill=white]|\color[gray]{0.5}WA%
 &|[fill=green]|\color[gray]{0.75} FO%
 &|[fill=green]|\color[gray]{0.75} FO%
 &|[fill=green]|\color[gray]{0.75} FO%
 &|[fill=green]|\color[gray]{0.75} FO%
 &|[fill=green]|\color[gray]{0.75} FO%
 &|[fill=green]|\color[gray]{0.75} FO%
 &|[fill=white]|\color[gray]{0.5}WA%
\\
|[fill=cyan]|\color[rgb]{1,0,0}\textbf{02}%
 &|[fill=green]|\color[gray]{0.75} FO%
 &|[fill=green]|\color[gray]{0.75} FO%
 &|[fill=green]|\color[gray]{0.75} FO%
 &|[fill=green]|\color[gray]{0.75} FO%
 &|[fill=green]|\color[gray]{0.75} FO%
 &|[fill=green]|\color[gray]{0.75} FO%
 &|[fill=green]|\color[gray]{0.75} FO%
 &|[fill=green]|\color[gray]{0.75} FO%
 &|[fill=green]|\color[gray]{0.75} FO%
\\
|[fill=green]|\color[gray]{0.75} FO%
 &|[fill=green]|\color[gray]{0.75} FO%
 &|[fill=white]|\color[gray]{0.5}WA%
 &|[fill=white]|\color[gray]{0.5}WA%
 &|[fill=white]|\color[gray]{0.5}WA%
 &|[fill=cyan]|\color[rgb]{1,0,0}\textbf{01}%
 &|[fill=white]|\color[gray]{0.5}WA%
 &|[fill=white]|\color[gray]{0.5}WA%
 &|[fill=white]|\color[gray]{0.5}WA%
 &|[fill=green]|\color[gray]{0.75} FO%
\\
|[fill=green]|\color[gray]{0.75} FO%
 &|[fill=green]|\color[gray]{0.75} FO%
 &|[fill=cyan]|\color[rgb]{1,0,0}\textbf{03}%
 &|[fill=green]|\color[rgb]{0,0,0}\textbf{CO}%
 &|[fill=green]|\color[rgb]{0,0,0}\textbf{CO}%
 &|[fill=green]|\color[rgb]{0,0,0}\textbf{CO}%
 &|[fill=green]|\color[rgb]{0,0,0}\textbf{CO}%
 &|[fill=green]|\color[rgb]{0,0,0}\textbf{CO}%
 &|[fill=green]|\color[rgb]{0,0,0}\textbf{CO}%
 &|[fill=cyan]|\color[rgb]{1,0,0}\textbf{04}%
\\
|[fill=green]|\color[gray]{0.75} FO%
 &|[fill=green]|\color[gray]{0.75} FO%
 &|[fill=white]|\color[gray]{0.5}WA%
 &|[fill=white]|\color[gray]{0.5}WA%
 &|[fill=green]|\color[rgb]{0,0,0}\textbf{CO}%
 &|[fill=green]|\color[rgb]{0,0,0}\textbf{CO}%
 &|[fill=green]|\color[rgb]{0,0,0}\textbf{CO}%
 &|[fill=green]|\color[rgb]{0,0,0}\textbf{CO}%
 &|[fill=green]|\color[rgb]{0,0,0}\textbf{CO}%
 &|[fill=green]|\color[rgb]{0,0,0}\textbf{CO}%
\\
|[fill=green]|\color[gray]{0.75} FO%
 &|[fill=cyan]|\color[rgb]{1,0,0}\textbf{08}%
 &|[fill=green]|\color[rgb]{0,0,0}\textbf{CO}%
 &|[fill=green]|\color[rgb]{0,0,0}\textbf{CO}%
 &|[fill=white]|\color[gray]{0.5}WA%
 &|[fill=green]|\color[rgb]{0,0,0}\textbf{CO}%
 &|[fill=green]|\color[rgb]{0,0,0}\textbf{CO}%
 &|[fill=white]|\color[gray]{0.5}WA%
 &|[fill=green]|\color[rgb]{0,0,0}\textbf{CO}%
 &|[fill=green]|\color[rgb]{0,0,0}\textbf{CO}%
\\
|[fill=white]|\color[gray]{0.5}WA%
 &|[fill=cyan]|\color[rgb]{1,0,0}\textbf{07}%
 &|[fill=green]|\color[rgb]{0,0,0}\textbf{CO}%
 &|[fill=green]|\color[rgb]{0,0,0}\textbf{CO}%
 &|[fill=white]|\color[gray]{0.5}WA%
 &|[fill=green]|\color[rgb]{0,0,0}\textbf{CO}%
 &|[fill=green]|\color[rgb]{0,0,0}\textbf{CO}%
 &|[fill=white]|\color[gray]{0.5}WA%
 &|[fill=green]|\color[rgb]{0,0,0}\textbf{CO}%
 &|[fill=white]|\color[gray]{0.5}WA%
\\
|[fill=green]|\color[gray]{0.75} FO%
 &|[fill=white]|\color[gray]{0.5}WA%
 &|[fill=green]|\color[rgb]{0,0,0}\textbf{CO}%
 &|[fill=green]|\color[rgb]{0,0,0}\textbf{CO}%
 &|[fill=white]|\color[gray]{0.5}WA%
 &|[fill=green]|\color[rgb]{0,0,0}\textbf{CO}%
 &|[fill=green]|\color[rgb]{0,0,0}\textbf{CO}%
 &|[fill=white]|\color[gray]{0.5}WA%
 &|[fill=green]|\color[rgb]{0,0,0}\textbf{CO}%
 &|[fill=green]|\color[rgb]{0,0,0}\textbf{CO}%
\\
|[fill=green]|\color[gray]{0.75} FO%
 &|[fill=cyan]|\color[rgb]{1,0,0}\textbf{05}%
 &|[fill=green]|\color[rgb]{0,0,0}\textbf{CO}%
 &|[fill=green]|\color[rgb]{0,0,0}\textbf{CO}%
 &|[fill=green]|\color[rgb]{0,0,0}\textbf{CO}%
 &|[fill=green]|\color[rgb]{0,0,0}\textbf{CO}%
 &|[fill=green]|\color[rgb]{0,0,0}\textbf{CO}%
 &|[fill=green]|\color[rgb]{0,0,0}\textbf{CO}%
 &|[fill=green]|\color[rgb]{0,0,0}\textbf{CO}%
 &|[fill=green]|\color[rgb]{0,0,0}\textbf{CO}%
\\
|[fill=green]|\color[gray]{0.75} FO%
 &|[fill=cyan]|\color[rgb]{1,0,0}\textbf{06}%
 &|[fill=green]|\color[rgb]{0,0,0}\textbf{CO}%
 &|[fill=green]|\color[rgb]{0,0,0}\textbf{CO}%
 &|[fill=green]|\color[rgb]{0,0,0}\textbf{CO}%
 &|[fill=green]|\color[rgb]{0,0,0}\textbf{CO}%
 &|[fill=green]|\color[rgb]{0,0,0}\textbf{CO}%
 &|[fill=green]|\color[rgb]{0,0,0}\textbf{CO}%
 &|[fill=green]|\color[rgb]{0,0,0}\textbf{CO}%
 &|[fill=green]|\color[rgb]{0,0,0}\textbf{CO}%
\\
};
\end{tikzpicture}\\
And you'll find 48 pieces of coal and 8 pieces of watered coal:
\\
\begin{tikzpicture}
\tikzset{square matrix/.style={
matrix of nodes,
column sep=-\pgflinewidth, row sep=-\pgflinewidth,
nodes={draw,
minimum height=#1,
anchor=center,
text width=#1,
align=center,
inner sep=0pt
},
},
square matrix/.default=1.2cm
}
\matrix[square matrix=1.4em] {
|[fill=green]|\color[rgb]{0,0,0}\textbf{CO}%
 &|[fill=green]|\color[rgb]{0,0,0}\textbf{CO}%
 &|[fill=white]|\color[gray]{0.5}WA%
 &|[fill=green]|\color[gray]{0.75} FO%
 &|[fill=green]|\color[gray]{0.75} FO%
 &|[fill=green]|\color[gray]{0.75} FO%
 &|[fill=green]|\color[gray]{0.75} FO%
 &|[fill=green]|\color[gray]{0.75} FO%
 &|[fill=white]|\color[gray]{0.5}WA%
 &|[fill=green]|\color[gray]{0.75} FO%
\\
|[fill=green]|\color[rgb]{0,0,0}\textbf{CO}%
 &|[fill=white]|\color[gray]{0.5}WA%
 &|[fill=white]|\color[gray]{0.5}WA%
 &|[fill=green]|\color[gray]{0.75} FO%
 &|[fill=green]|\color[gray]{0.75} FO%
 &|[fill=green]|\color[gray]{0.75} FO%
 &|[fill=green]|\color[gray]{0.75} FO%
 &|[fill=green]|\color[gray]{0.75} FO%
 &|[fill=green]|\color[gray]{0.75} FO%
 &|[fill=white]|\color[gray]{0.5}WA%
\\
|[fill=cyan]|\color[rgb]{1,0,0}\textbf{02}%
 &|[fill=green]|\color[gray]{0.75} FO%
 &|[fill=green]|\color[gray]{0.75} FO%
 &|[fill=green]|\color[gray]{0.75} FO%
 &|[fill=green]|\color[gray]{0.75} FO%
 &|[fill=green]|\color[gray]{0.75} FO%
 &|[fill=green]|\color[gray]{0.75} FO%
 &|[fill=green]|\color[gray]{0.75} FO%
 &|[fill=green]|\color[gray]{0.75} FO%
 &|[fill=green]|\color[gray]{0.75} FO%
\\
|[fill=green]|\color[gray]{0.75} FO%
 &|[fill=green]|\color[gray]{0.75} FO%
 &|[fill=white]|\color[gray]{0.5}WA%
 &|[fill=white]|\color[gray]{0.5}WA%
 &|[fill=white]|\color[gray]{0.5}WA%
 &|[fill=cyan]|\color[rgb]{1,0,0}\textbf{01}%
 &|[fill=white]|\color[gray]{0.5}WA%
 &|[fill=white]|\color[gray]{0.5}WA%
 &|[fill=white]|\color[gray]{0.5}WA%
 &|[fill=green]|\color[gray]{0.75} FO%
\\
|[fill=green]|\color[gray]{0.75} FO%
 &|[fill=green]|\color[gray]{0.75} FO%
 &|[fill=cyan]|\color[rgb]{1,0,0}\textbf{03}%
 &|[fill=green]|\color[rgb]{0,0,0}\textbf{CO}%
 &|[fill=green]|\color[rgb]{0,0,0}\textbf{CO}%
 &|[fill=green]|\color[rgb]{0,0,0}\textbf{CO}%
 &|[fill=green]|\color[rgb]{0,0,0}\textbf{CO}%
 &|[fill=green]|\color[rgb]{0,0,0}\textbf{CO}%
 &|[fill=green]|\color[rgb]{0,0,0}\textbf{CO}%
 &|[fill=cyan]|\color[rgb]{1,0,0}\textbf{04}%
\\
|[fill=green]|\color[gray]{0.75} FO%
 &|[fill=green]|\color[gray]{0.75} FO%
 &|[fill=white]|\color[gray]{0.5}WA%
 &|[fill=white]|\color[gray]{0.5}WA%
 &|[fill=green]|\color[rgb]{0,0,0}\textbf{CO}%
 &|[fill=green]|\color[rgb]{0,0,0}\textbf{CO}%
 &|[fill=green]|\color[rgb]{0,0,0}\textbf{CO}%
 &|[fill=green]|\color[rgb]{0,0,0}\textbf{CO}%
 &|[fill=green]|\color[rgb]{0,0,0}\textbf{CO}%
 &|[fill=green]|\color[rgb]{0,0,0}\textbf{CO}%
\\
|[fill=green]|\color[gray]{0.75} FO%
 &|[fill=cyan]|\color[rgb]{1,0,0}\textbf{08}%
 &|[fill=green]|\color[rgb]{0,0,0}\textbf{CO}%
 &|[fill=green]|\color[rgb]{0,0,0}\textbf{CO}%
 &|[fill=white]|\color[gray]{0.5}WA%
 &|[fill=green]|\color[rgb]{0,0,0}\textbf{CO}%
 &|[fill=green]|\color[rgb]{0,0,0}\textbf{CO}%
 &|[fill=white]|\color[gray]{0.5}WA%
 &|[fill=green]|\color[rgb]{0,0,0}\textbf{CO}%
 &|[fill=green]|\color[rgb]{0,0,0}\textbf{CO}%
\\
|[fill=white]|\color[gray]{0.5}WA%
 &|[fill=cyan]|\color[rgb]{1,0,0}\textbf{07}%
 &|[fill=green]|\color[rgb]{0,0,0}\textbf{CO}%
 &|[fill=green]|\color[rgb]{0,0,0}\textbf{CO}%
 &|[fill=white]|\color[gray]{0.5}WA%
 &|[fill=green]|\color[rgb]{0,0,0}\textbf{CO}%
 &|[fill=green]|\color[rgb]{0,0,0}\textbf{CO}%
 &|[fill=white]|\color[gray]{0.5}WA%
 &|[fill=green]|\color[rgb]{0,0,0}\textbf{CO}%
 &|[fill=white]|\color[gray]{0.5}WA%
\\
|[fill=green]|\color[gray]{0.75} FO%
 &|[fill=white]|\color[gray]{0.5}WA%
 &|[fill=green]|\color[rgb]{0,0,0}\textbf{CO}%
 &|[fill=green]|\color[rgb]{0,0,0}\textbf{CO}%
 &|[fill=white]|\color[gray]{0.5}WA%
 &|[fill=green]|\color[rgb]{0,0,0}\textbf{CO}%
 &|[fill=green]|\color[rgb]{0,0,0}\textbf{CO}%
 &|[fill=white]|\color[gray]{0.5}WA%
 &|[fill=green]|\color[rgb]{0,0,0}\textbf{CO}%
 &|[fill=green]|\color[rgb]{0,0,0}\textbf{CO}%
\\
|[fill=green]|\color[gray]{0.75} FO%
 &|[fill=cyan]|\color[rgb]{1,0,0}\textbf{05}%
 &|[fill=green]|\color[rgb]{0,0,0}\textbf{CO}%
 &|[fill=green]|\color[rgb]{0,0,0}\textbf{CO}%
 &|[fill=green]|\color[rgb]{0,0,0}\textbf{CO}%
 &|[fill=green]|\color[rgb]{0,0,0}\textbf{CO}%
 &|[fill=green]|\color[rgb]{0,0,0}\textbf{CO}%
 &|[fill=green]|\color[rgb]{0,0,0}\textbf{CO}%
 &|[fill=green]|\color[rgb]{0,0,0}\textbf{CO}%
 &|[fill=green]|\color[rgb]{0,0,0}\textbf{CO}%
\\
|[fill=green]|\color[gray]{0.75} FO%
 &|[fill=cyan]|\color[rgb]{1,0,0}\textbf{06}%
 &|[fill=green]|\color[rgb]{0,0,0}\textbf{CO}%
 &|[fill=green]|\color[rgb]{0,0,0}\textbf{CO}%
 &|[fill=green]|\color[rgb]{0,0,0}\textbf{CO}%
 &|[fill=green]|\color[rgb]{0,0,0}\textbf{CO}%
 &|[fill=green]|\color[rgb]{0,0,0}\textbf{CO}%
 &|[fill=green]|\color[rgb]{0,0,0}\textbf{CO}%
 &|[fill=green]|\color[rgb]{0,0,0}\textbf{CO}%
 &|[fill=green]|\color[rgb]{0,0,0}\textbf{CO}%
\\
};
\end{tikzpicture}\\
\\
Explanation:\\
\colorbox{white}{\color[gray]{0.5}WA}  ---  WALL\\
\colorbox{green}{\color[gray]{0.5}FO}  ---  FOREST\\
{\color[rgb]{1,0,0}\textbf{BU}}  ---  BURNED\\
{\color[rgb]{0,0,0}\textbf{CO}}  ---  COAL (doubly burned)\\
\colorbox{cyan}{\#\#}  ---  WATERED at time \#\#\\
Fields can have more than 1 state.
}
\subsubsection{Beispiel 2}
\footnote{Diese Eingabe finden Sie auch in der Datei \texttt{2.in}}:
{\small
\lstinputlisting{../Aufgabe_1/2.in}
}
Mein Programm produziert folgende Ausgabe\footnote{Diese Ausgabe finden Sie auch in der Datei \texttt{2.out.tex}; Eine Datei \texttt{2.out} mit den ASCII-Escape-Sequenzen exisitert ebenfalls.}:\\
{\ttfamily \small
\\
\begin{tikzpicture}
\tikzset{square matrix/.style={
matrix of nodes,
column sep=-\pgflinewidth, row sep=-\pgflinewidth,
nodes={draw,
minimum height=#1,
anchor=center,
text width=#1,
align=center,
inner sep=0pt
},
},
square matrix/.default=1.2cm
}
\matrix[square matrix=1.4em] {
|[fill=green]|\color[gray]{0.75} FO%
 &|[fill=green]|\color[gray]{0.75} FO%
 &|[fill=green]|\color[gray]{0.75} FO%
 &|[fill=green]|\color[gray]{0.75} FO%
 &|[fill=green]|\color[gray]{0.75} FO%
 &|[fill=green]|\color[gray]{0.75} FO%
 &|[fill=green]|\color[gray]{0.75} FO%
 &|[fill=green]|\color[gray]{0.75} FO%
 &|[fill=green]|\color[gray]{0.75} FO%
 &|[fill=green]|\color[gray]{0.75} FO%
 &|[fill=green]|\color[gray]{0.75} FO%
 &|[fill=green]|\color[gray]{0.75} FO%
 &|[fill=green]|\color[gray]{0.75} FO%
\\
|[fill=green]|\color[gray]{0.75} FO%
 &|[fill=white]|\color[gray]{0.5}WA%
 &|[fill=white]|\color[gray]{0.5}WA%
 &|[fill=white]|\color[gray]{0.5}WA%
 &|[fill=white]|\color[gray]{0.5}WA%
 &|[fill=white]|\color[gray]{0.5}WA%
 &|[fill=green]|\color[gray]{0.75} FO%
 &|[fill=white]|\color[gray]{0.5}WA%
 &|[fill=white]|\color[gray]{0.5}WA%
 &|[fill=white]|\color[gray]{0.5}WA%
 &|[fill=white]|\color[gray]{0.5}WA%
 &|[fill=white]|\color[gray]{0.5}WA%
 &|[fill=green]|\color[gray]{0.75} FO%
\\
|[fill=green]|\color[gray]{0.75} FO%
 &|[fill=white]|\color[gray]{0.5}WA%
 &|[fill=green]|\color[gray]{0.75} FO%
 &|[fill=green]|\color[gray]{0.75} FO%
 &|[fill=green]|\color[gray]{0.75} FO%
 &|[fill=green]|\color[gray]{0.75} FO%
 &|[fill=green]|\color[gray]{0.75} FO%
 &|[fill=green]|\color[gray]{0.75} FO%
 &|[fill=green]|\color[gray]{0.75} FO%
 &|[fill=green]|\color[gray]{0.75} FO%
 &|[fill=green]|\color[gray]{0.75} FO%
 &|[fill=white]|\color[gray]{0.5}WA%
 &|[fill=green]|\color[gray]{0.75} FO%
\\
|[fill=green]|\color[gray]{0.75} FO%
 &|[fill=white]|\color[gray]{0.5}WA%
 &|[fill=green]|\color[gray]{0.75} FO%
 &|[fill=green]|\color[gray]{0.75} FO%
 &|[fill=green]|\color[gray]{0.75} FO%
 &|[fill=green]|\color[gray]{0.75} FO%
 &|[fill=green]|\color[gray]{0.75} FO%
 &|[fill=green]|\color[gray]{0.75} FO%
 &|[fill=green]|\color[gray]{0.75} FO%
 &|[fill=green]|\color[gray]{0.75} FO%
 &|[fill=green]|\color[gray]{0.75} FO%
 &|[fill=white]|\color[gray]{0.5}WA%
 &|[fill=green]|\color[gray]{0.75} FO%
\\
|[fill=green]|\color[gray]{0.75} FO%
 &|[fill=white]|\color[gray]{0.5}WA%
 &|[fill=green]|\color[gray]{0.75} FO%
 &|[fill=green]|\color[gray]{0.75} FO%
 &|[fill=green]|\color[gray]{0.75} FO%
 &|[fill=green]|\color[gray]{0.75} FO%
 &|[fill=green]|\color[gray]{0.75} FO%
 &|[fill=green]|\color[gray]{0.75} FO%
 &|[fill=green]|\color[gray]{0.75} FO%
 &|[fill=green]|\color[gray]{0.75} FO%
 &|[fill=green]|\color[gray]{0.75} FO%
 &|[fill=white]|\color[gray]{0.5}WA%
 &|[fill=green]|\color[gray]{0.75} FO%
\\
|[fill=green]|\color[gray]{0.75} FO%
 &|[fill=white]|\color[gray]{0.5}WA%
 &|[fill=green]|\color[gray]{0.75} FO%
 &|[fill=green]|\color[gray]{0.75} FO%
 &|[fill=green]|\color[gray]{0.75} FO%
 &|[fill=green]|\color[gray]{0.75} FO%
 &|[fill=green]|\color[gray]{0.75} FO%
 &|[fill=green]|\color[gray]{0.75} FO%
 &|[fill=green]|\color[gray]{0.75} FO%
 &|[fill=green]|\color[gray]{0.75} FO%
 &|[fill=green]|\color[gray]{0.75} FO%
 &|[fill=white]|\color[gray]{0.5}WA%
 &|[fill=green]|\color[gray]{0.75} FO%
\\
|[fill=green]|\color[gray]{0.75} FO%
 &|[fill=green]|\color[gray]{0.75} FO%
 &|[fill=green]|\color[gray]{0.75} FO%
 &|[fill=green]|\color[gray]{0.75} FO%
 &|[fill=green]|\color[gray]{0.75} FO%
 &|[fill=green]|\color[gray]{0.75} FO%
 &|[fill=green]|\color[rgb]{1,0,0}\textbf{BU}%
 &|[fill=green]|\color[gray]{0.75} FO%
 &|[fill=green]|\color[gray]{0.75} FO%
 &|[fill=green]|\color[gray]{0.75} FO%
 &|[fill=green]|\color[gray]{0.75} FO%
 &|[fill=green]|\color[gray]{0.75} FO%
 &|[fill=green]|\color[gray]{0.75} FO%
\\
|[fill=green]|\color[gray]{0.75} FO%
 &|[fill=white]|\color[gray]{0.5}WA%
 &|[fill=green]|\color[gray]{0.75} FO%
 &|[fill=green]|\color[gray]{0.75} FO%
 &|[fill=green]|\color[gray]{0.75} FO%
 &|[fill=green]|\color[gray]{0.75} FO%
 &|[fill=green]|\color[gray]{0.75} FO%
 &|[fill=green]|\color[gray]{0.75} FO%
 &|[fill=green]|\color[gray]{0.75} FO%
 &|[fill=green]|\color[gray]{0.75} FO%
 &|[fill=green]|\color[gray]{0.75} FO%
 &|[fill=white]|\color[gray]{0.5}WA%
 &|[fill=green]|\color[gray]{0.75} FO%
\\
|[fill=green]|\color[gray]{0.75} FO%
 &|[fill=white]|\color[gray]{0.5}WA%
 &|[fill=green]|\color[gray]{0.75} FO%
 &|[fill=green]|\color[gray]{0.75} FO%
 &|[fill=green]|\color[gray]{0.75} FO%
 &|[fill=green]|\color[gray]{0.75} FO%
 &|[fill=green]|\color[gray]{0.75} FO%
 &|[fill=green]|\color[gray]{0.75} FO%
 &|[fill=green]|\color[gray]{0.75} FO%
 &|[fill=green]|\color[gray]{0.75} FO%
 &|[fill=green]|\color[gray]{0.75} FO%
 &|[fill=white]|\color[gray]{0.5}WA%
 &|[fill=green]|\color[gray]{0.75} FO%
\\
|[fill=green]|\color[gray]{0.75} FO%
 &|[fill=white]|\color[gray]{0.5}WA%
 &|[fill=green]|\color[gray]{0.75} FO%
 &|[fill=green]|\color[gray]{0.75} FO%
 &|[fill=green]|\color[gray]{0.75} FO%
 &|[fill=green]|\color[gray]{0.75} FO%
 &|[fill=green]|\color[gray]{0.75} FO%
 &|[fill=green]|\color[gray]{0.75} FO%
 &|[fill=green]|\color[gray]{0.75} FO%
 &|[fill=green]|\color[gray]{0.75} FO%
 &|[fill=green]|\color[gray]{0.75} FO%
 &|[fill=white]|\color[gray]{0.5}WA%
 &|[fill=green]|\color[gray]{0.75} FO%
\\
|[fill=green]|\color[gray]{0.75} FO%
 &|[fill=white]|\color[gray]{0.5}WA%
 &|[fill=green]|\color[gray]{0.75} FO%
 &|[fill=green]|\color[gray]{0.75} FO%
 &|[fill=green]|\color[gray]{0.75} FO%
 &|[fill=green]|\color[gray]{0.75} FO%
 &|[fill=green]|\color[gray]{0.75} FO%
 &|[fill=green]|\color[gray]{0.75} FO%
 &|[fill=green]|\color[gray]{0.75} FO%
 &|[fill=green]|\color[gray]{0.75} FO%
 &|[fill=green]|\color[gray]{0.75} FO%
 &|[fill=white]|\color[gray]{0.5}WA%
 &|[fill=green]|\color[gray]{0.75} FO%
\\
|[fill=green]|\color[gray]{0.75} FO%
 &|[fill=white]|\color[gray]{0.5}WA%
 &|[fill=white]|\color[gray]{0.5}WA%
 &|[fill=white]|\color[gray]{0.5}WA%
 &|[fill=white]|\color[gray]{0.5}WA%
 &|[fill=white]|\color[gray]{0.5}WA%
 &|[fill=green]|\color[gray]{0.75} FO%
 &|[fill=white]|\color[gray]{0.5}WA%
 &|[fill=white]|\color[gray]{0.5}WA%
 &|[fill=white]|\color[gray]{0.5}WA%
 &|[fill=white]|\color[gray]{0.5}WA%
 &|[fill=white]|\color[gray]{0.5}WA%
 &|[fill=green]|\color[gray]{0.75} FO%
\\
|[fill=green]|\color[gray]{0.75} FO%
 &|[fill=green]|\color[gray]{0.75} FO%
 &|[fill=green]|\color[gray]{0.75} FO%
 &|[fill=green]|\color[gray]{0.75} FO%
 &|[fill=green]|\color[gray]{0.75} FO%
 &|[fill=green]|\color[gray]{0.75} FO%
 &|[fill=green]|\color[gray]{0.75} FO%
 &|[fill=green]|\color[gray]{0.75} FO%
 &|[fill=green]|\color[gray]{0.75} FO%
 &|[fill=green]|\color[gray]{0.75} FO%
 &|[fill=green]|\color[gray]{0.75} FO%
 &|[fill=green]|\color[gray]{0.75} FO%
 &|[fill=green]|\color[gray]{0.75} FO%
\\
};
\end{tikzpicture}\\
At time 1: Water spot (7|6):
\\
\begin{tikzpicture}
\tikzset{square matrix/.style={
matrix of nodes,
column sep=-\pgflinewidth, row sep=-\pgflinewidth,
nodes={draw,
minimum height=#1,
anchor=center,
text width=#1,
align=center,
inner sep=0pt
},
},
square matrix/.default=1.2cm
}
\matrix[square matrix=1.4em] {
|[fill=green]|\color[gray]{0.75} FO%
 &|[fill=green]|\color[gray]{0.75} FO%
 &|[fill=green]|\color[gray]{0.75} FO%
 &|[fill=green]|\color[gray]{0.75} FO%
 &|[fill=green]|\color[gray]{0.75} FO%
 &|[fill=green]|\color[gray]{0.75} FO%
 &|[fill=green]|\color[gray]{0.75} FO%
 &|[fill=green]|\color[gray]{0.75} FO%
 &|[fill=green]|\color[gray]{0.75} FO%
 &|[fill=green]|\color[gray]{0.75} FO%
 &|[fill=green]|\color[gray]{0.75} FO%
 &|[fill=green]|\color[gray]{0.75} FO%
 &|[fill=green]|\color[gray]{0.75} FO%
\\
|[fill=green]|\color[gray]{0.75} FO%
 &|[fill=white]|\color[gray]{0.5}WA%
 &|[fill=white]|\color[gray]{0.5}WA%
 &|[fill=white]|\color[gray]{0.5}WA%
 &|[fill=white]|\color[gray]{0.5}WA%
 &|[fill=white]|\color[gray]{0.5}WA%
 &|[fill=green]|\color[gray]{0.75} FO%
 &|[fill=white]|\color[gray]{0.5}WA%
 &|[fill=white]|\color[gray]{0.5}WA%
 &|[fill=white]|\color[gray]{0.5}WA%
 &|[fill=white]|\color[gray]{0.5}WA%
 &|[fill=white]|\color[gray]{0.5}WA%
 &|[fill=green]|\color[gray]{0.75} FO%
\\
|[fill=green]|\color[gray]{0.75} FO%
 &|[fill=white]|\color[gray]{0.5}WA%
 &|[fill=green]|\color[gray]{0.75} FO%
 &|[fill=green]|\color[gray]{0.75} FO%
 &|[fill=green]|\color[gray]{0.75} FO%
 &|[fill=green]|\color[gray]{0.75} FO%
 &|[fill=green]|\color[gray]{0.75} FO%
 &|[fill=green]|\color[gray]{0.75} FO%
 &|[fill=green]|\color[gray]{0.75} FO%
 &|[fill=green]|\color[gray]{0.75} FO%
 &|[fill=green]|\color[gray]{0.75} FO%
 &|[fill=white]|\color[gray]{0.5}WA%
 &|[fill=green]|\color[gray]{0.75} FO%
\\
|[fill=green]|\color[gray]{0.75} FO%
 &|[fill=white]|\color[gray]{0.5}WA%
 &|[fill=green]|\color[gray]{0.75} FO%
 &|[fill=green]|\color[gray]{0.75} FO%
 &|[fill=green]|\color[gray]{0.75} FO%
 &|[fill=green]|\color[gray]{0.75} FO%
 &|[fill=green]|\color[gray]{0.75} FO%
 &|[fill=green]|\color[gray]{0.75} FO%
 &|[fill=green]|\color[gray]{0.75} FO%
 &|[fill=green]|\color[gray]{0.75} FO%
 &|[fill=green]|\color[gray]{0.75} FO%
 &|[fill=white]|\color[gray]{0.5}WA%
 &|[fill=green]|\color[gray]{0.75} FO%
\\
|[fill=green]|\color[gray]{0.75} FO%
 &|[fill=white]|\color[gray]{0.5}WA%
 &|[fill=green]|\color[gray]{0.75} FO%
 &|[fill=green]|\color[gray]{0.75} FO%
 &|[fill=green]|\color[gray]{0.75} FO%
 &|[fill=green]|\color[gray]{0.75} FO%
 &|[fill=green]|\color[gray]{0.75} FO%
 &|[fill=green]|\color[gray]{0.75} FO%
 &|[fill=green]|\color[gray]{0.75} FO%
 &|[fill=green]|\color[gray]{0.75} FO%
 &|[fill=green]|\color[gray]{0.75} FO%
 &|[fill=white]|\color[gray]{0.5}WA%
 &|[fill=green]|\color[gray]{0.75} FO%
\\
|[fill=green]|\color[gray]{0.75} FO%
 &|[fill=white]|\color[gray]{0.5}WA%
 &|[fill=green]|\color[gray]{0.75} FO%
 &|[fill=green]|\color[gray]{0.75} FO%
 &|[fill=green]|\color[gray]{0.75} FO%
 &|[fill=green]|\color[gray]{0.75} FO%
 &|[fill=green]|\color[rgb]{1,0,0}\textbf{BU}%
 &|[fill=green]|\color[gray]{0.75} FO%
 &|[fill=green]|\color[gray]{0.75} FO%
 &|[fill=green]|\color[gray]{0.75} FO%
 &|[fill=green]|\color[gray]{0.75} FO%
 &|[fill=white]|\color[gray]{0.5}WA%
 &|[fill=green]|\color[gray]{0.75} FO%
\\
|[fill=green]|\color[gray]{0.75} FO%
 &|[fill=green]|\color[gray]{0.75} FO%
 &|[fill=green]|\color[gray]{0.75} FO%
 &|[fill=green]|\color[gray]{0.75} FO%
 &|[fill=green]|\color[gray]{0.75} FO%
 &|[fill=green]|\color[rgb]{1,0,0}\textbf{BU}%
 &|[fill=green]|\color[rgb]{0,0,0}\textbf{CO}%
 &|[fill=cyan]|\color[rgb]{1,0,0}\textbf{01}%
 &|[fill=green]|\color[gray]{0.75} FO%
 &|[fill=green]|\color[gray]{0.75} FO%
 &|[fill=green]|\color[gray]{0.75} FO%
 &|[fill=green]|\color[gray]{0.75} FO%
 &|[fill=green]|\color[gray]{0.75} FO%
\\
|[fill=green]|\color[gray]{0.75} FO%
 &|[fill=white]|\color[gray]{0.5}WA%
 &|[fill=green]|\color[gray]{0.75} FO%
 &|[fill=green]|\color[gray]{0.75} FO%
 &|[fill=green]|\color[gray]{0.75} FO%
 &|[fill=green]|\color[gray]{0.75} FO%
 &|[fill=green]|\color[rgb]{1,0,0}\textbf{BU}%
 &|[fill=green]|\color[gray]{0.75} FO%
 &|[fill=green]|\color[gray]{0.75} FO%
 &|[fill=green]|\color[gray]{0.75} FO%
 &|[fill=green]|\color[gray]{0.75} FO%
 &|[fill=white]|\color[gray]{0.5}WA%
 &|[fill=green]|\color[gray]{0.75} FO%
\\
|[fill=green]|\color[gray]{0.75} FO%
 &|[fill=white]|\color[gray]{0.5}WA%
 &|[fill=green]|\color[gray]{0.75} FO%
 &|[fill=green]|\color[gray]{0.75} FO%
 &|[fill=green]|\color[gray]{0.75} FO%
 &|[fill=green]|\color[gray]{0.75} FO%
 &|[fill=green]|\color[gray]{0.75} FO%
 &|[fill=green]|\color[gray]{0.75} FO%
 &|[fill=green]|\color[gray]{0.75} FO%
 &|[fill=green]|\color[gray]{0.75} FO%
 &|[fill=green]|\color[gray]{0.75} FO%
 &|[fill=white]|\color[gray]{0.5}WA%
 &|[fill=green]|\color[gray]{0.75} FO%
\\
|[fill=green]|\color[gray]{0.75} FO%
 &|[fill=white]|\color[gray]{0.5}WA%
 &|[fill=green]|\color[gray]{0.75} FO%
 &|[fill=green]|\color[gray]{0.75} FO%
 &|[fill=green]|\color[gray]{0.75} FO%
 &|[fill=green]|\color[gray]{0.75} FO%
 &|[fill=green]|\color[gray]{0.75} FO%
 &|[fill=green]|\color[gray]{0.75} FO%
 &|[fill=green]|\color[gray]{0.75} FO%
 &|[fill=green]|\color[gray]{0.75} FO%
 &|[fill=green]|\color[gray]{0.75} FO%
 &|[fill=white]|\color[gray]{0.5}WA%
 &|[fill=green]|\color[gray]{0.75} FO%
\\
|[fill=green]|\color[gray]{0.75} FO%
 &|[fill=white]|\color[gray]{0.5}WA%
 &|[fill=green]|\color[gray]{0.75} FO%
 &|[fill=green]|\color[gray]{0.75} FO%
 &|[fill=green]|\color[gray]{0.75} FO%
 &|[fill=green]|\color[gray]{0.75} FO%
 &|[fill=green]|\color[gray]{0.75} FO%
 &|[fill=green]|\color[gray]{0.75} FO%
 &|[fill=green]|\color[gray]{0.75} FO%
 &|[fill=green]|\color[gray]{0.75} FO%
 &|[fill=green]|\color[gray]{0.75} FO%
 &|[fill=white]|\color[gray]{0.5}WA%
 &|[fill=green]|\color[gray]{0.75} FO%
\\
|[fill=green]|\color[gray]{0.75} FO%
 &|[fill=white]|\color[gray]{0.5}WA%
 &|[fill=white]|\color[gray]{0.5}WA%
 &|[fill=white]|\color[gray]{0.5}WA%
 &|[fill=white]|\color[gray]{0.5}WA%
 &|[fill=white]|\color[gray]{0.5}WA%
 &|[fill=green]|\color[gray]{0.75} FO%
 &|[fill=white]|\color[gray]{0.5}WA%
 &|[fill=white]|\color[gray]{0.5}WA%
 &|[fill=white]|\color[gray]{0.5}WA%
 &|[fill=white]|\color[gray]{0.5}WA%
 &|[fill=white]|\color[gray]{0.5}WA%
 &|[fill=green]|\color[gray]{0.75} FO%
\\
|[fill=green]|\color[gray]{0.75} FO%
 &|[fill=green]|\color[gray]{0.75} FO%
 &|[fill=green]|\color[gray]{0.75} FO%
 &|[fill=green]|\color[gray]{0.75} FO%
 &|[fill=green]|\color[gray]{0.75} FO%
 &|[fill=green]|\color[gray]{0.75} FO%
 &|[fill=green]|\color[gray]{0.75} FO%
 &|[fill=green]|\color[gray]{0.75} FO%
 &|[fill=green]|\color[gray]{0.75} FO%
 &|[fill=green]|\color[gray]{0.75} FO%
 &|[fill=green]|\color[gray]{0.75} FO%
 &|[fill=green]|\color[gray]{0.75} FO%
 &|[fill=green]|\color[gray]{0.75} FO%
\\
};
\end{tikzpicture}\\
At time 2: Water spot (6|8):
\\
\begin{tikzpicture}
\tikzset{square matrix/.style={
matrix of nodes,
column sep=-\pgflinewidth, row sep=-\pgflinewidth,
nodes={draw,
minimum height=#1,
anchor=center,
text width=#1,
align=center,
inner sep=0pt
},
},
square matrix/.default=1.2cm
}
\matrix[square matrix=1.4em] {
|[fill=green]|\color[gray]{0.75} FO%
 &|[fill=green]|\color[gray]{0.75} FO%
 &|[fill=green]|\color[gray]{0.75} FO%
 &|[fill=green]|\color[gray]{0.75} FO%
 &|[fill=green]|\color[gray]{0.75} FO%
 &|[fill=green]|\color[gray]{0.75} FO%
 &|[fill=green]|\color[gray]{0.75} FO%
 &|[fill=green]|\color[gray]{0.75} FO%
 &|[fill=green]|\color[gray]{0.75} FO%
 &|[fill=green]|\color[gray]{0.75} FO%
 &|[fill=green]|\color[gray]{0.75} FO%
 &|[fill=green]|\color[gray]{0.75} FO%
 &|[fill=green]|\color[gray]{0.75} FO%
\\
|[fill=green]|\color[gray]{0.75} FO%
 &|[fill=white]|\color[gray]{0.5}WA%
 &|[fill=white]|\color[gray]{0.5}WA%
 &|[fill=white]|\color[gray]{0.5}WA%
 &|[fill=white]|\color[gray]{0.5}WA%
 &|[fill=white]|\color[gray]{0.5}WA%
 &|[fill=green]|\color[gray]{0.75} FO%
 &|[fill=white]|\color[gray]{0.5}WA%
 &|[fill=white]|\color[gray]{0.5}WA%
 &|[fill=white]|\color[gray]{0.5}WA%
 &|[fill=white]|\color[gray]{0.5}WA%
 &|[fill=white]|\color[gray]{0.5}WA%
 &|[fill=green]|\color[gray]{0.75} FO%
\\
|[fill=green]|\color[gray]{0.75} FO%
 &|[fill=white]|\color[gray]{0.5}WA%
 &|[fill=green]|\color[gray]{0.75} FO%
 &|[fill=green]|\color[gray]{0.75} FO%
 &|[fill=green]|\color[gray]{0.75} FO%
 &|[fill=green]|\color[gray]{0.75} FO%
 &|[fill=green]|\color[gray]{0.75} FO%
 &|[fill=green]|\color[gray]{0.75} FO%
 &|[fill=green]|\color[gray]{0.75} FO%
 &|[fill=green]|\color[gray]{0.75} FO%
 &|[fill=green]|\color[gray]{0.75} FO%
 &|[fill=white]|\color[gray]{0.5}WA%
 &|[fill=green]|\color[gray]{0.75} FO%
\\
|[fill=green]|\color[gray]{0.75} FO%
 &|[fill=white]|\color[gray]{0.5}WA%
 &|[fill=green]|\color[gray]{0.75} FO%
 &|[fill=green]|\color[gray]{0.75} FO%
 &|[fill=green]|\color[gray]{0.75} FO%
 &|[fill=green]|\color[gray]{0.75} FO%
 &|[fill=green]|\color[gray]{0.75} FO%
 &|[fill=green]|\color[gray]{0.75} FO%
 &|[fill=green]|\color[gray]{0.75} FO%
 &|[fill=green]|\color[gray]{0.75} FO%
 &|[fill=green]|\color[gray]{0.75} FO%
 &|[fill=white]|\color[gray]{0.5}WA%
 &|[fill=green]|\color[gray]{0.75} FO%
\\
|[fill=green]|\color[gray]{0.75} FO%
 &|[fill=white]|\color[gray]{0.5}WA%
 &|[fill=green]|\color[gray]{0.75} FO%
 &|[fill=green]|\color[gray]{0.75} FO%
 &|[fill=green]|\color[gray]{0.75} FO%
 &|[fill=green]|\color[gray]{0.75} FO%
 &|[fill=green]|\color[rgb]{1,0,0}\textbf{BU}%
 &|[fill=green]|\color[gray]{0.75} FO%
 &|[fill=green]|\color[gray]{0.75} FO%
 &|[fill=green]|\color[gray]{0.75} FO%
 &|[fill=green]|\color[gray]{0.75} FO%
 &|[fill=white]|\color[gray]{0.5}WA%
 &|[fill=green]|\color[gray]{0.75} FO%
\\
|[fill=green]|\color[gray]{0.75} FO%
 &|[fill=white]|\color[gray]{0.5}WA%
 &|[fill=green]|\color[gray]{0.75} FO%
 &|[fill=green]|\color[gray]{0.75} FO%
 &|[fill=green]|\color[gray]{0.75} FO%
 &|[fill=green]|\color[rgb]{1,0,0}\textbf{BU}%
 &|[fill=green]|\color[rgb]{0,0,0}\textbf{CO}%
 &|[fill=green]|\color[rgb]{1,0,0}\textbf{BU}%
 &|[fill=green]|\color[gray]{0.75} FO%
 &|[fill=green]|\color[gray]{0.75} FO%
 &|[fill=green]|\color[gray]{0.75} FO%
 &|[fill=white]|\color[gray]{0.5}WA%
 &|[fill=green]|\color[gray]{0.75} FO%
\\
|[fill=green]|\color[gray]{0.75} FO%
 &|[fill=green]|\color[gray]{0.75} FO%
 &|[fill=green]|\color[gray]{0.75} FO%
 &|[fill=green]|\color[gray]{0.75} FO%
 &|[fill=green]|\color[rgb]{1,0,0}\textbf{BU}%
 &|[fill=green]|\color[rgb]{0,0,0}\textbf{CO}%
 &|[fill=green]|\color[rgb]{0,0,0}\textbf{CO}%
 &|[fill=cyan]|\color[rgb]{1,0,0}\textbf{01}%
 &|[fill=green]|\color[gray]{0.75} FO%
 &|[fill=green]|\color[gray]{0.75} FO%
 &|[fill=green]|\color[gray]{0.75} FO%
 &|[fill=green]|\color[gray]{0.75} FO%
 &|[fill=green]|\color[gray]{0.75} FO%
\\
|[fill=green]|\color[gray]{0.75} FO%
 &|[fill=white]|\color[gray]{0.5}WA%
 &|[fill=green]|\color[gray]{0.75} FO%
 &|[fill=green]|\color[gray]{0.75} FO%
 &|[fill=green]|\color[gray]{0.75} FO%
 &|[fill=green]|\color[rgb]{1,0,0}\textbf{BU}%
 &|[fill=green]|\color[rgb]{0,0,0}\textbf{CO}%
 &|[fill=green]|\color[rgb]{1,0,0}\textbf{BU}%
 &|[fill=green]|\color[gray]{0.75} FO%
 &|[fill=green]|\color[gray]{0.75} FO%
 &|[fill=green]|\color[gray]{0.75} FO%
 &|[fill=white]|\color[gray]{0.5}WA%
 &|[fill=green]|\color[gray]{0.75} FO%
\\
|[fill=green]|\color[gray]{0.75} FO%
 &|[fill=white]|\color[gray]{0.5}WA%
 &|[fill=green]|\color[gray]{0.75} FO%
 &|[fill=green]|\color[gray]{0.75} FO%
 &|[fill=green]|\color[gray]{0.75} FO%
 &|[fill=green]|\color[gray]{0.75} FO%
 &|[fill=cyan]|\color[rgb]{1,0,0}\textbf{02}%
 &|[fill=green]|\color[gray]{0.75} FO%
 &|[fill=green]|\color[gray]{0.75} FO%
 &|[fill=green]|\color[gray]{0.75} FO%
 &|[fill=green]|\color[gray]{0.75} FO%
 &|[fill=white]|\color[gray]{0.5}WA%
 &|[fill=green]|\color[gray]{0.75} FO%
\\
|[fill=green]|\color[gray]{0.75} FO%
 &|[fill=white]|\color[gray]{0.5}WA%
 &|[fill=green]|\color[gray]{0.75} FO%
 &|[fill=green]|\color[gray]{0.75} FO%
 &|[fill=green]|\color[gray]{0.75} FO%
 &|[fill=green]|\color[gray]{0.75} FO%
 &|[fill=green]|\color[gray]{0.75} FO%
 &|[fill=green]|\color[gray]{0.75} FO%
 &|[fill=green]|\color[gray]{0.75} FO%
 &|[fill=green]|\color[gray]{0.75} FO%
 &|[fill=green]|\color[gray]{0.75} FO%
 &|[fill=white]|\color[gray]{0.5}WA%
 &|[fill=green]|\color[gray]{0.75} FO%
\\
|[fill=green]|\color[gray]{0.75} FO%
 &|[fill=white]|\color[gray]{0.5}WA%
 &|[fill=green]|\color[gray]{0.75} FO%
 &|[fill=green]|\color[gray]{0.75} FO%
 &|[fill=green]|\color[gray]{0.75} FO%
 &|[fill=green]|\color[gray]{0.75} FO%
 &|[fill=green]|\color[gray]{0.75} FO%
 &|[fill=green]|\color[gray]{0.75} FO%
 &|[fill=green]|\color[gray]{0.75} FO%
 &|[fill=green]|\color[gray]{0.75} FO%
 &|[fill=green]|\color[gray]{0.75} FO%
 &|[fill=white]|\color[gray]{0.5}WA%
 &|[fill=green]|\color[gray]{0.75} FO%
\\
|[fill=green]|\color[gray]{0.75} FO%
 &|[fill=white]|\color[gray]{0.5}WA%
 &|[fill=white]|\color[gray]{0.5}WA%
 &|[fill=white]|\color[gray]{0.5}WA%
 &|[fill=white]|\color[gray]{0.5}WA%
 &|[fill=white]|\color[gray]{0.5}WA%
 &|[fill=green]|\color[gray]{0.75} FO%
 &|[fill=white]|\color[gray]{0.5}WA%
 &|[fill=white]|\color[gray]{0.5}WA%
 &|[fill=white]|\color[gray]{0.5}WA%
 &|[fill=white]|\color[gray]{0.5}WA%
 &|[fill=white]|\color[gray]{0.5}WA%
 &|[fill=green]|\color[gray]{0.75} FO%
\\
|[fill=green]|\color[gray]{0.75} FO%
 &|[fill=green]|\color[gray]{0.75} FO%
 &|[fill=green]|\color[gray]{0.75} FO%
 &|[fill=green]|\color[gray]{0.75} FO%
 &|[fill=green]|\color[gray]{0.75} FO%
 &|[fill=green]|\color[gray]{0.75} FO%
 &|[fill=green]|\color[gray]{0.75} FO%
 &|[fill=green]|\color[gray]{0.75} FO%
 &|[fill=green]|\color[gray]{0.75} FO%
 &|[fill=green]|\color[gray]{0.75} FO%
 &|[fill=green]|\color[gray]{0.75} FO%
 &|[fill=green]|\color[gray]{0.75} FO%
 &|[fill=green]|\color[gray]{0.75} FO%
\\
};
\end{tikzpicture}\\
At time 3: Water spot (3|6):
\\
\begin{tikzpicture}
\tikzset{square matrix/.style={
matrix of nodes,
column sep=-\pgflinewidth, row sep=-\pgflinewidth,
nodes={draw,
minimum height=#1,
anchor=center,
text width=#1,
align=center,
inner sep=0pt
},
},
square matrix/.default=1.2cm
}
\matrix[square matrix=1.4em] {
|[fill=green]|\color[gray]{0.75} FO%
 &|[fill=green]|\color[gray]{0.75} FO%
 &|[fill=green]|\color[gray]{0.75} FO%
 &|[fill=green]|\color[gray]{0.75} FO%
 &|[fill=green]|\color[gray]{0.75} FO%
 &|[fill=green]|\color[gray]{0.75} FO%
 &|[fill=green]|\color[gray]{0.75} FO%
 &|[fill=green]|\color[gray]{0.75} FO%
 &|[fill=green]|\color[gray]{0.75} FO%
 &|[fill=green]|\color[gray]{0.75} FO%
 &|[fill=green]|\color[gray]{0.75} FO%
 &|[fill=green]|\color[gray]{0.75} FO%
 &|[fill=green]|\color[gray]{0.75} FO%
\\
|[fill=green]|\color[gray]{0.75} FO%
 &|[fill=white]|\color[gray]{0.5}WA%
 &|[fill=white]|\color[gray]{0.5}WA%
 &|[fill=white]|\color[gray]{0.5}WA%
 &|[fill=white]|\color[gray]{0.5}WA%
 &|[fill=white]|\color[gray]{0.5}WA%
 &|[fill=green]|\color[gray]{0.75} FO%
 &|[fill=white]|\color[gray]{0.5}WA%
 &|[fill=white]|\color[gray]{0.5}WA%
 &|[fill=white]|\color[gray]{0.5}WA%
 &|[fill=white]|\color[gray]{0.5}WA%
 &|[fill=white]|\color[gray]{0.5}WA%
 &|[fill=green]|\color[gray]{0.75} FO%
\\
|[fill=green]|\color[gray]{0.75} FO%
 &|[fill=white]|\color[gray]{0.5}WA%
 &|[fill=green]|\color[gray]{0.75} FO%
 &|[fill=green]|\color[gray]{0.75} FO%
 &|[fill=green]|\color[gray]{0.75} FO%
 &|[fill=green]|\color[gray]{0.75} FO%
 &|[fill=green]|\color[gray]{0.75} FO%
 &|[fill=green]|\color[gray]{0.75} FO%
 &|[fill=green]|\color[gray]{0.75} FO%
 &|[fill=green]|\color[gray]{0.75} FO%
 &|[fill=green]|\color[gray]{0.75} FO%
 &|[fill=white]|\color[gray]{0.5}WA%
 &|[fill=green]|\color[gray]{0.75} FO%
\\
|[fill=green]|\color[gray]{0.75} FO%
 &|[fill=white]|\color[gray]{0.5}WA%
 &|[fill=green]|\color[gray]{0.75} FO%
 &|[fill=green]|\color[gray]{0.75} FO%
 &|[fill=green]|\color[gray]{0.75} FO%
 &|[fill=green]|\color[gray]{0.75} FO%
 &|[fill=green]|\color[rgb]{1,0,0}\textbf{BU}%
 &|[fill=green]|\color[gray]{0.75} FO%
 &|[fill=green]|\color[gray]{0.75} FO%
 &|[fill=green]|\color[gray]{0.75} FO%
 &|[fill=green]|\color[gray]{0.75} FO%
 &|[fill=white]|\color[gray]{0.5}WA%
 &|[fill=green]|\color[gray]{0.75} FO%
\\
|[fill=green]|\color[gray]{0.75} FO%
 &|[fill=white]|\color[gray]{0.5}WA%
 &|[fill=green]|\color[gray]{0.75} FO%
 &|[fill=green]|\color[gray]{0.75} FO%
 &|[fill=green]|\color[gray]{0.75} FO%
 &|[fill=green]|\color[rgb]{1,0,0}\textbf{BU}%
 &|[fill=green]|\color[rgb]{0,0,0}\textbf{CO}%
 &|[fill=green]|\color[rgb]{1,0,0}\textbf{BU}%
 &|[fill=green]|\color[gray]{0.75} FO%
 &|[fill=green]|\color[gray]{0.75} FO%
 &|[fill=green]|\color[gray]{0.75} FO%
 &|[fill=white]|\color[gray]{0.5}WA%
 &|[fill=green]|\color[gray]{0.75} FO%
\\
|[fill=green]|\color[gray]{0.75} FO%
 &|[fill=white]|\color[gray]{0.5}WA%
 &|[fill=green]|\color[gray]{0.75} FO%
 &|[fill=green]|\color[gray]{0.75} FO%
 &|[fill=green]|\color[rgb]{1,0,0}\textbf{BU}%
 &|[fill=green]|\color[rgb]{0,0,0}\textbf{CO}%
 &|[fill=green]|\color[rgb]{0,0,0}\textbf{CO}%
 &|[fill=green]|\color[rgb]{0,0,0}\textbf{CO}%
 &|[fill=green]|\color[rgb]{1,0,0}\textbf{BU}%
 &|[fill=green]|\color[gray]{0.75} FO%
 &|[fill=green]|\color[gray]{0.75} FO%
 &|[fill=white]|\color[gray]{0.5}WA%
 &|[fill=green]|\color[gray]{0.75} FO%
\\
|[fill=green]|\color[gray]{0.75} FO%
 &|[fill=green]|\color[gray]{0.75} FO%
 &|[fill=green]|\color[gray]{0.75} FO%
 &|[fill=cyan]|\color[rgb]{1,0,0}\textbf{03}%
 &|[fill=green]|\color[rgb]{0,0,0}\textbf{CO}%
 &|[fill=green]|\color[rgb]{0,0,0}\textbf{CO}%
 &|[fill=green]|\color[rgb]{0,0,0}\textbf{CO}%
 &|[fill=cyan]|\color[rgb]{1,0,0}\textbf{01}%
 &|[fill=green]|\color[gray]{0.75} FO%
 &|[fill=green]|\color[gray]{0.75} FO%
 &|[fill=green]|\color[gray]{0.75} FO%
 &|[fill=green]|\color[gray]{0.75} FO%
 &|[fill=green]|\color[gray]{0.75} FO%
\\
|[fill=green]|\color[gray]{0.75} FO%
 &|[fill=white]|\color[gray]{0.5}WA%
 &|[fill=green]|\color[gray]{0.75} FO%
 &|[fill=green]|\color[gray]{0.75} FO%
 &|[fill=green]|\color[rgb]{1,0,0}\textbf{BU}%
 &|[fill=green]|\color[rgb]{0,0,0}\textbf{CO}%
 &|[fill=green]|\color[rgb]{0,0,0}\textbf{CO}%
 &|[fill=green]|\color[rgb]{0,0,0}\textbf{CO}%
 &|[fill=green]|\color[rgb]{1,0,0}\textbf{BU}%
 &|[fill=green]|\color[gray]{0.75} FO%
 &|[fill=green]|\color[gray]{0.75} FO%
 &|[fill=white]|\color[gray]{0.5}WA%
 &|[fill=green]|\color[gray]{0.75} FO%
\\
|[fill=green]|\color[gray]{0.75} FO%
 &|[fill=white]|\color[gray]{0.5}WA%
 &|[fill=green]|\color[gray]{0.75} FO%
 &|[fill=green]|\color[gray]{0.75} FO%
 &|[fill=green]|\color[gray]{0.75} FO%
 &|[fill=green]|\color[rgb]{1,0,0}\textbf{BU}%
 &|[fill=cyan]|\color[rgb]{1,0,0}\textbf{02}%
 &|[fill=green]|\color[rgb]{1,0,0}\textbf{BU}%
 &|[fill=green]|\color[gray]{0.75} FO%
 &|[fill=green]|\color[gray]{0.75} FO%
 &|[fill=green]|\color[gray]{0.75} FO%
 &|[fill=white]|\color[gray]{0.5}WA%
 &|[fill=green]|\color[gray]{0.75} FO%
\\
|[fill=green]|\color[gray]{0.75} FO%
 &|[fill=white]|\color[gray]{0.5}WA%
 &|[fill=green]|\color[gray]{0.75} FO%
 &|[fill=green]|\color[gray]{0.75} FO%
 &|[fill=green]|\color[gray]{0.75} FO%
 &|[fill=green]|\color[gray]{0.75} FO%
 &|[fill=green]|\color[gray]{0.75} FO%
 &|[fill=green]|\color[gray]{0.75} FO%
 &|[fill=green]|\color[gray]{0.75} FO%
 &|[fill=green]|\color[gray]{0.75} FO%
 &|[fill=green]|\color[gray]{0.75} FO%
 &|[fill=white]|\color[gray]{0.5}WA%
 &|[fill=green]|\color[gray]{0.75} FO%
\\
|[fill=green]|\color[gray]{0.75} FO%
 &|[fill=white]|\color[gray]{0.5}WA%
 &|[fill=green]|\color[gray]{0.75} FO%
 &|[fill=green]|\color[gray]{0.75} FO%
 &|[fill=green]|\color[gray]{0.75} FO%
 &|[fill=green]|\color[gray]{0.75} FO%
 &|[fill=green]|\color[gray]{0.75} FO%
 &|[fill=green]|\color[gray]{0.75} FO%
 &|[fill=green]|\color[gray]{0.75} FO%
 &|[fill=green]|\color[gray]{0.75} FO%
 &|[fill=green]|\color[gray]{0.75} FO%
 &|[fill=white]|\color[gray]{0.5}WA%
 &|[fill=green]|\color[gray]{0.75} FO%
\\
|[fill=green]|\color[gray]{0.75} FO%
 &|[fill=white]|\color[gray]{0.5}WA%
 &|[fill=white]|\color[gray]{0.5}WA%
 &|[fill=white]|\color[gray]{0.5}WA%
 &|[fill=white]|\color[gray]{0.5}WA%
 &|[fill=white]|\color[gray]{0.5}WA%
 &|[fill=green]|\color[gray]{0.75} FO%
 &|[fill=white]|\color[gray]{0.5}WA%
 &|[fill=white]|\color[gray]{0.5}WA%
 &|[fill=white]|\color[gray]{0.5}WA%
 &|[fill=white]|\color[gray]{0.5}WA%
 &|[fill=white]|\color[gray]{0.5}WA%
 &|[fill=green]|\color[gray]{0.75} FO%
\\
|[fill=green]|\color[gray]{0.75} FO%
 &|[fill=green]|\color[gray]{0.75} FO%
 &|[fill=green]|\color[gray]{0.75} FO%
 &|[fill=green]|\color[gray]{0.75} FO%
 &|[fill=green]|\color[gray]{0.75} FO%
 &|[fill=green]|\color[gray]{0.75} FO%
 &|[fill=green]|\color[gray]{0.75} FO%
 &|[fill=green]|\color[gray]{0.75} FO%
 &|[fill=green]|\color[gray]{0.75} FO%
 &|[fill=green]|\color[gray]{0.75} FO%
 &|[fill=green]|\color[gray]{0.75} FO%
 &|[fill=green]|\color[gray]{0.75} FO%
 &|[fill=green]|\color[gray]{0.75} FO%
\\
};
\end{tikzpicture}\\
At time 4: Water spot (6|2):
\\
\begin{tikzpicture}
\tikzset{square matrix/.style={
matrix of nodes,
column sep=-\pgflinewidth, row sep=-\pgflinewidth,
nodes={draw,
minimum height=#1,
anchor=center,
text width=#1,
align=center,
inner sep=0pt
},
},
square matrix/.default=1.2cm
}
\matrix[square matrix=1.4em] {
|[fill=green]|\color[gray]{0.75} FO%
 &|[fill=green]|\color[gray]{0.75} FO%
 &|[fill=green]|\color[gray]{0.75} FO%
 &|[fill=green]|\color[gray]{0.75} FO%
 &|[fill=green]|\color[gray]{0.75} FO%
 &|[fill=green]|\color[gray]{0.75} FO%
 &|[fill=green]|\color[gray]{0.75} FO%
 &|[fill=green]|\color[gray]{0.75} FO%
 &|[fill=green]|\color[gray]{0.75} FO%
 &|[fill=green]|\color[gray]{0.75} FO%
 &|[fill=green]|\color[gray]{0.75} FO%
 &|[fill=green]|\color[gray]{0.75} FO%
 &|[fill=green]|\color[gray]{0.75} FO%
\\
|[fill=green]|\color[gray]{0.75} FO%
 &|[fill=white]|\color[gray]{0.5}WA%
 &|[fill=white]|\color[gray]{0.5}WA%
 &|[fill=white]|\color[gray]{0.5}WA%
 &|[fill=white]|\color[gray]{0.5}WA%
 &|[fill=white]|\color[gray]{0.5}WA%
 &|[fill=green]|\color[gray]{0.75} FO%
 &|[fill=white]|\color[gray]{0.5}WA%
 &|[fill=white]|\color[gray]{0.5}WA%
 &|[fill=white]|\color[gray]{0.5}WA%
 &|[fill=white]|\color[gray]{0.5}WA%
 &|[fill=white]|\color[gray]{0.5}WA%
 &|[fill=green]|\color[gray]{0.75} FO%
\\
|[fill=green]|\color[gray]{0.75} FO%
 &|[fill=white]|\color[gray]{0.5}WA%
 &|[fill=green]|\color[gray]{0.75} FO%
 &|[fill=green]|\color[gray]{0.75} FO%
 &|[fill=green]|\color[gray]{0.75} FO%
 &|[fill=green]|\color[gray]{0.75} FO%
 &|[fill=cyan]|\color[rgb]{1,0,0}\textbf{04}%
 &|[fill=green]|\color[gray]{0.75} FO%
 &|[fill=green]|\color[gray]{0.75} FO%
 &|[fill=green]|\color[gray]{0.75} FO%
 &|[fill=green]|\color[gray]{0.75} FO%
 &|[fill=white]|\color[gray]{0.5}WA%
 &|[fill=green]|\color[gray]{0.75} FO%
\\
|[fill=green]|\color[gray]{0.75} FO%
 &|[fill=white]|\color[gray]{0.5}WA%
 &|[fill=green]|\color[gray]{0.75} FO%
 &|[fill=green]|\color[gray]{0.75} FO%
 &|[fill=green]|\color[gray]{0.75} FO%
 &|[fill=green]|\color[rgb]{1,0,0}\textbf{BU}%
 &|[fill=green]|\color[rgb]{0,0,0}\textbf{CO}%
 &|[fill=green]|\color[rgb]{1,0,0}\textbf{BU}%
 &|[fill=green]|\color[gray]{0.75} FO%
 &|[fill=green]|\color[gray]{0.75} FO%
 &|[fill=green]|\color[gray]{0.75} FO%
 &|[fill=white]|\color[gray]{0.5}WA%
 &|[fill=green]|\color[gray]{0.75} FO%
\\
|[fill=green]|\color[gray]{0.75} FO%
 &|[fill=white]|\color[gray]{0.5}WA%
 &|[fill=green]|\color[gray]{0.75} FO%
 &|[fill=green]|\color[gray]{0.75} FO%
 &|[fill=green]|\color[rgb]{1,0,0}\textbf{BU}%
 &|[fill=green]|\color[rgb]{0,0,0}\textbf{CO}%
 &|[fill=green]|\color[rgb]{0,0,0}\textbf{CO}%
 &|[fill=green]|\color[rgb]{0,0,0}\textbf{CO}%
 &|[fill=green]|\color[rgb]{1,0,0}\textbf{BU}%
 &|[fill=green]|\color[gray]{0.75} FO%
 &|[fill=green]|\color[gray]{0.75} FO%
 &|[fill=white]|\color[gray]{0.5}WA%
 &|[fill=green]|\color[gray]{0.75} FO%
\\
|[fill=green]|\color[gray]{0.75} FO%
 &|[fill=white]|\color[gray]{0.5}WA%
 &|[fill=green]|\color[gray]{0.75} FO%
 &|[fill=green]|\color[rgb]{1,0,0}\textbf{BU}%
 &|[fill=green]|\color[rgb]{0,0,0}\textbf{CO}%
 &|[fill=green]|\color[rgb]{0,0,0}\textbf{CO}%
 &|[fill=green]|\color[rgb]{0,0,0}\textbf{CO}%
 &|[fill=green]|\color[rgb]{0,0,0}\textbf{CO}%
 &|[fill=green]|\color[rgb]{0,0,0}\textbf{CO}%
 &|[fill=green]|\color[rgb]{1,0,0}\textbf{BU}%
 &|[fill=green]|\color[gray]{0.75} FO%
 &|[fill=white]|\color[gray]{0.5}WA%
 &|[fill=green]|\color[gray]{0.75} FO%
\\
|[fill=green]|\color[gray]{0.75} FO%
 &|[fill=green]|\color[gray]{0.75} FO%
 &|[fill=green]|\color[gray]{0.75} FO%
 &|[fill=cyan]|\color[rgb]{1,0,0}\textbf{03}%
 &|[fill=green]|\color[rgb]{0,0,0}\textbf{CO}%
 &|[fill=green]|\color[rgb]{0,0,0}\textbf{CO}%
 &|[fill=green]|\color[rgb]{0,0,0}\textbf{CO}%
 &|[fill=cyan]|\color[rgb]{1,0,0}\textbf{01}%
 &|[fill=green]|\color[rgb]{1,0,0}\textbf{BU}%
 &|[fill=green]|\color[gray]{0.75} FO%
 &|[fill=green]|\color[gray]{0.75} FO%
 &|[fill=green]|\color[gray]{0.75} FO%
 &|[fill=green]|\color[gray]{0.75} FO%
\\
|[fill=green]|\color[gray]{0.75} FO%
 &|[fill=white]|\color[gray]{0.5}WA%
 &|[fill=green]|\color[gray]{0.75} FO%
 &|[fill=green]|\color[rgb]{1,0,0}\textbf{BU}%
 &|[fill=green]|\color[rgb]{0,0,0}\textbf{CO}%
 &|[fill=green]|\color[rgb]{0,0,0}\textbf{CO}%
 &|[fill=green]|\color[rgb]{0,0,0}\textbf{CO}%
 &|[fill=green]|\color[rgb]{0,0,0}\textbf{CO}%
 &|[fill=green]|\color[rgb]{0,0,0}\textbf{CO}%
 &|[fill=green]|\color[rgb]{1,0,0}\textbf{BU}%
 &|[fill=green]|\color[gray]{0.75} FO%
 &|[fill=white]|\color[gray]{0.5}WA%
 &|[fill=green]|\color[gray]{0.75} FO%
\\
|[fill=green]|\color[gray]{0.75} FO%
 &|[fill=white]|\color[gray]{0.5}WA%
 &|[fill=green]|\color[gray]{0.75} FO%
 &|[fill=green]|\color[gray]{0.75} FO%
 &|[fill=green]|\color[rgb]{1,0,0}\textbf{BU}%
 &|[fill=green]|\color[rgb]{0,0,0}\textbf{CO}%
 &|[fill=cyan]|\color[rgb]{1,0,0}\textbf{02}%
 &|[fill=green]|\color[rgb]{0,0,0}\textbf{CO}%
 &|[fill=green]|\color[rgb]{1,0,0}\textbf{BU}%
 &|[fill=green]|\color[gray]{0.75} FO%
 &|[fill=green]|\color[gray]{0.75} FO%
 &|[fill=white]|\color[gray]{0.5}WA%
 &|[fill=green]|\color[gray]{0.75} FO%
\\
|[fill=green]|\color[gray]{0.75} FO%
 &|[fill=white]|\color[gray]{0.5}WA%
 &|[fill=green]|\color[gray]{0.75} FO%
 &|[fill=green]|\color[gray]{0.75} FO%
 &|[fill=green]|\color[gray]{0.75} FO%
 &|[fill=green]|\color[rgb]{1,0,0}\textbf{BU}%
 &|[fill=green]|\color[gray]{0.75} FO%
 &|[fill=green]|\color[rgb]{1,0,0}\textbf{BU}%
 &|[fill=green]|\color[gray]{0.75} FO%
 &|[fill=green]|\color[gray]{0.75} FO%
 &|[fill=green]|\color[gray]{0.75} FO%
 &|[fill=white]|\color[gray]{0.5}WA%
 &|[fill=green]|\color[gray]{0.75} FO%
\\
|[fill=green]|\color[gray]{0.75} FO%
 &|[fill=white]|\color[gray]{0.5}WA%
 &|[fill=green]|\color[gray]{0.75} FO%
 &|[fill=green]|\color[gray]{0.75} FO%
 &|[fill=green]|\color[gray]{0.75} FO%
 &|[fill=green]|\color[gray]{0.75} FO%
 &|[fill=green]|\color[gray]{0.75} FO%
 &|[fill=green]|\color[gray]{0.75} FO%
 &|[fill=green]|\color[gray]{0.75} FO%
 &|[fill=green]|\color[gray]{0.75} FO%
 &|[fill=green]|\color[gray]{0.75} FO%
 &|[fill=white]|\color[gray]{0.5}WA%
 &|[fill=green]|\color[gray]{0.75} FO%
\\
|[fill=green]|\color[gray]{0.75} FO%
 &|[fill=white]|\color[gray]{0.5}WA%
 &|[fill=white]|\color[gray]{0.5}WA%
 &|[fill=white]|\color[gray]{0.5}WA%
 &|[fill=white]|\color[gray]{0.5}WA%
 &|[fill=white]|\color[gray]{0.5}WA%
 &|[fill=green]|\color[gray]{0.75} FO%
 &|[fill=white]|\color[gray]{0.5}WA%
 &|[fill=white]|\color[gray]{0.5}WA%
 &|[fill=white]|\color[gray]{0.5}WA%
 &|[fill=white]|\color[gray]{0.5}WA%
 &|[fill=white]|\color[gray]{0.5}WA%
 &|[fill=green]|\color[gray]{0.75} FO%
\\
|[fill=green]|\color[gray]{0.75} FO%
 &|[fill=green]|\color[gray]{0.75} FO%
 &|[fill=green]|\color[gray]{0.75} FO%
 &|[fill=green]|\color[gray]{0.75} FO%
 &|[fill=green]|\color[gray]{0.75} FO%
 &|[fill=green]|\color[gray]{0.75} FO%
 &|[fill=green]|\color[gray]{0.75} FO%
 &|[fill=green]|\color[gray]{0.75} FO%
 &|[fill=green]|\color[gray]{0.75} FO%
 &|[fill=green]|\color[gray]{0.75} FO%
 &|[fill=green]|\color[gray]{0.75} FO%
 &|[fill=green]|\color[gray]{0.75} FO%
 &|[fill=green]|\color[gray]{0.75} FO%
\\
};
\end{tikzpicture}\\
At time 5: Water spot (10|7):
\\
\begin{tikzpicture}
\tikzset{square matrix/.style={
matrix of nodes,
column sep=-\pgflinewidth, row sep=-\pgflinewidth,
nodes={draw,
minimum height=#1,
anchor=center,
text width=#1,
align=center,
inner sep=0pt
},
},
square matrix/.default=1.2cm
}
\matrix[square matrix=1.4em] {
|[fill=green]|\color[gray]{0.75} FO%
 &|[fill=green]|\color[gray]{0.75} FO%
 &|[fill=green]|\color[gray]{0.75} FO%
 &|[fill=green]|\color[gray]{0.75} FO%
 &|[fill=green]|\color[gray]{0.75} FO%
 &|[fill=green]|\color[gray]{0.75} FO%
 &|[fill=green]|\color[gray]{0.75} FO%
 &|[fill=green]|\color[gray]{0.75} FO%
 &|[fill=green]|\color[gray]{0.75} FO%
 &|[fill=green]|\color[gray]{0.75} FO%
 &|[fill=green]|\color[gray]{0.75} FO%
 &|[fill=green]|\color[gray]{0.75} FO%
 &|[fill=green]|\color[gray]{0.75} FO%
\\
|[fill=green]|\color[gray]{0.75} FO%
 &|[fill=white]|\color[gray]{0.5}WA%
 &|[fill=white]|\color[gray]{0.5}WA%
 &|[fill=white]|\color[gray]{0.5}WA%
 &|[fill=white]|\color[gray]{0.5}WA%
 &|[fill=white]|\color[gray]{0.5}WA%
 &|[fill=green]|\color[gray]{0.75} FO%
 &|[fill=white]|\color[gray]{0.5}WA%
 &|[fill=white]|\color[gray]{0.5}WA%
 &|[fill=white]|\color[gray]{0.5}WA%
 &|[fill=white]|\color[gray]{0.5}WA%
 &|[fill=white]|\color[gray]{0.5}WA%
 &|[fill=green]|\color[gray]{0.75} FO%
\\
|[fill=green]|\color[gray]{0.75} FO%
 &|[fill=white]|\color[gray]{0.5}WA%
 &|[fill=green]|\color[gray]{0.75} FO%
 &|[fill=green]|\color[gray]{0.75} FO%
 &|[fill=green]|\color[gray]{0.75} FO%
 &|[fill=green]|\color[rgb]{1,0,0}\textbf{BU}%
 &|[fill=cyan]|\color[rgb]{1,0,0}\textbf{04}%
 &|[fill=green]|\color[rgb]{1,0,0}\textbf{BU}%
 &|[fill=green]|\color[gray]{0.75} FO%
 &|[fill=green]|\color[gray]{0.75} FO%
 &|[fill=green]|\color[gray]{0.75} FO%
 &|[fill=white]|\color[gray]{0.5}WA%
 &|[fill=green]|\color[gray]{0.75} FO%
\\
|[fill=green]|\color[gray]{0.75} FO%
 &|[fill=white]|\color[gray]{0.5}WA%
 &|[fill=green]|\color[gray]{0.75} FO%
 &|[fill=green]|\color[gray]{0.75} FO%
 &|[fill=green]|\color[rgb]{1,0,0}\textbf{BU}%
 &|[fill=green]|\color[rgb]{0,0,0}\textbf{CO}%
 &|[fill=green]|\color[rgb]{0,0,0}\textbf{CO}%
 &|[fill=green]|\color[rgb]{0,0,0}\textbf{CO}%
 &|[fill=green]|\color[rgb]{1,0,0}\textbf{BU}%
 &|[fill=green]|\color[gray]{0.75} FO%
 &|[fill=green]|\color[gray]{0.75} FO%
 &|[fill=white]|\color[gray]{0.5}WA%
 &|[fill=green]|\color[gray]{0.75} FO%
\\
|[fill=green]|\color[gray]{0.75} FO%
 &|[fill=white]|\color[gray]{0.5}WA%
 &|[fill=green]|\color[gray]{0.75} FO%
 &|[fill=green]|\color[rgb]{1,0,0}\textbf{BU}%
 &|[fill=green]|\color[rgb]{0,0,0}\textbf{CO}%
 &|[fill=green]|\color[rgb]{0,0,0}\textbf{CO}%
 &|[fill=green]|\color[rgb]{0,0,0}\textbf{CO}%
 &|[fill=green]|\color[rgb]{0,0,0}\textbf{CO}%
 &|[fill=green]|\color[rgb]{0,0,0}\textbf{CO}%
 &|[fill=green]|\color[rgb]{1,0,0}\textbf{BU}%
 &|[fill=green]|\color[gray]{0.75} FO%
 &|[fill=white]|\color[gray]{0.5}WA%
 &|[fill=green]|\color[gray]{0.75} FO%
\\
|[fill=green]|\color[gray]{0.75} FO%
 &|[fill=white]|\color[gray]{0.5}WA%
 &|[fill=green]|\color[rgb]{1,0,0}\textbf{BU}%
 &|[fill=green]|\color[rgb]{0,0,0}\textbf{CO}%
 &|[fill=green]|\color[rgb]{0,0,0}\textbf{CO}%
 &|[fill=green]|\color[rgb]{0,0,0}\textbf{CO}%
 &|[fill=green]|\color[rgb]{0,0,0}\textbf{CO}%
 &|[fill=green]|\color[rgb]{0,0,0}\textbf{CO}%
 &|[fill=green]|\color[rgb]{0,0,0}\textbf{CO}%
 &|[fill=green]|\color[rgb]{0,0,0}\textbf{CO}%
 &|[fill=green]|\color[rgb]{1,0,0}\textbf{BU}%
 &|[fill=white]|\color[gray]{0.5}WA%
 &|[fill=green]|\color[gray]{0.75} FO%
\\
|[fill=green]|\color[gray]{0.75} FO%
 &|[fill=green]|\color[gray]{0.75} FO%
 &|[fill=green]|\color[gray]{0.75} FO%
 &|[fill=cyan]|\color[rgb]{1,0,0}\textbf{03}%
 &|[fill=green]|\color[rgb]{0,0,0}\textbf{CO}%
 &|[fill=green]|\color[rgb]{0,0,0}\textbf{CO}%
 &|[fill=green]|\color[rgb]{0,0,0}\textbf{CO}%
 &|[fill=cyan]|\color[rgb]{1,0,0}\textbf{01}%
 &|[fill=green]|\color[rgb]{0,0,0}\textbf{CO}%
 &|[fill=green]|\color[rgb]{1,0,0}\textbf{BU}%
 &|[fill=green]|\color[gray]{0.75} FO%
 &|[fill=green]|\color[gray]{0.75} FO%
 &|[fill=green]|\color[gray]{0.75} FO%
\\
|[fill=green]|\color[gray]{0.75} FO%
 &|[fill=white]|\color[gray]{0.5}WA%
 &|[fill=green]|\color[rgb]{1,0,0}\textbf{BU}%
 &|[fill=green]|\color[rgb]{0,0,0}\textbf{CO}%
 &|[fill=green]|\color[rgb]{0,0,0}\textbf{CO}%
 &|[fill=green]|\color[rgb]{0,0,0}\textbf{CO}%
 &|[fill=green]|\color[rgb]{0,0,0}\textbf{CO}%
 &|[fill=green]|\color[rgb]{0,0,0}\textbf{CO}%
 &|[fill=green]|\color[rgb]{0,0,0}\textbf{CO}%
 &|[fill=green]|\color[rgb]{0,0,0}\textbf{CO}%
 &|[fill=cyan]|\color[rgb]{1,0,0}\textbf{05}%
 &|[fill=white]|\color[gray]{0.5}WA%
 &|[fill=green]|\color[gray]{0.75} FO%
\\
|[fill=green]|\color[gray]{0.75} FO%
 &|[fill=white]|\color[gray]{0.5}WA%
 &|[fill=green]|\color[gray]{0.75} FO%
 &|[fill=green]|\color[rgb]{1,0,0}\textbf{BU}%
 &|[fill=green]|\color[rgb]{0,0,0}\textbf{CO}%
 &|[fill=green]|\color[rgb]{0,0,0}\textbf{CO}%
 &|[fill=cyan]|\color[rgb]{1,0,0}\textbf{02}%
 &|[fill=green]|\color[rgb]{0,0,0}\textbf{CO}%
 &|[fill=green]|\color[rgb]{0,0,0}\textbf{CO}%
 &|[fill=green]|\color[rgb]{1,0,0}\textbf{BU}%
 &|[fill=green]|\color[gray]{0.75} FO%
 &|[fill=white]|\color[gray]{0.5}WA%
 &|[fill=green]|\color[gray]{0.75} FO%
\\
|[fill=green]|\color[gray]{0.75} FO%
 &|[fill=white]|\color[gray]{0.5}WA%
 &|[fill=green]|\color[gray]{0.75} FO%
 &|[fill=green]|\color[gray]{0.75} FO%
 &|[fill=green]|\color[rgb]{1,0,0}\textbf{BU}%
 &|[fill=green]|\color[rgb]{0,0,0}\textbf{CO}%
 &|[fill=green]|\color[rgb]{1,0,0}\textbf{BU}%
 &|[fill=green]|\color[rgb]{0,0,0}\textbf{CO}%
 &|[fill=green]|\color[rgb]{1,0,0}\textbf{BU}%
 &|[fill=green]|\color[gray]{0.75} FO%
 &|[fill=green]|\color[gray]{0.75} FO%
 &|[fill=white]|\color[gray]{0.5}WA%
 &|[fill=green]|\color[gray]{0.75} FO%
\\
|[fill=green]|\color[gray]{0.75} FO%
 &|[fill=white]|\color[gray]{0.5}WA%
 &|[fill=green]|\color[gray]{0.75} FO%
 &|[fill=green]|\color[gray]{0.75} FO%
 &|[fill=green]|\color[gray]{0.75} FO%
 &|[fill=green]|\color[rgb]{1,0,0}\textbf{BU}%
 &|[fill=green]|\color[gray]{0.75} FO%
 &|[fill=green]|\color[rgb]{1,0,0}\textbf{BU}%
 &|[fill=green]|\color[gray]{0.75} FO%
 &|[fill=green]|\color[gray]{0.75} FO%
 &|[fill=green]|\color[gray]{0.75} FO%
 &|[fill=white]|\color[gray]{0.5}WA%
 &|[fill=green]|\color[gray]{0.75} FO%
\\
|[fill=green]|\color[gray]{0.75} FO%
 &|[fill=white]|\color[gray]{0.5}WA%
 &|[fill=white]|\color[gray]{0.5}WA%
 &|[fill=white]|\color[gray]{0.5}WA%
 &|[fill=white]|\color[gray]{0.5}WA%
 &|[fill=white]|\color[gray]{0.5}WA%
 &|[fill=green]|\color[gray]{0.75} FO%
 &|[fill=white]|\color[gray]{0.5}WA%
 &|[fill=white]|\color[gray]{0.5}WA%
 &|[fill=white]|\color[gray]{0.5}WA%
 &|[fill=white]|\color[gray]{0.5}WA%
 &|[fill=white]|\color[gray]{0.5}WA%
 &|[fill=green]|\color[gray]{0.75} FO%
\\
|[fill=green]|\color[gray]{0.75} FO%
 &|[fill=green]|\color[gray]{0.75} FO%
 &|[fill=green]|\color[gray]{0.75} FO%
 &|[fill=green]|\color[gray]{0.75} FO%
 &|[fill=green]|\color[gray]{0.75} FO%
 &|[fill=green]|\color[gray]{0.75} FO%
 &|[fill=green]|\color[gray]{0.75} FO%
 &|[fill=green]|\color[gray]{0.75} FO%
 &|[fill=green]|\color[gray]{0.75} FO%
 &|[fill=green]|\color[gray]{0.75} FO%
 &|[fill=green]|\color[gray]{0.75} FO%
 &|[fill=green]|\color[gray]{0.75} FO%
 &|[fill=green]|\color[gray]{0.75} FO%
\\
};
\end{tikzpicture}\\
At time 6: Water spot (10|6):
\\
\begin{tikzpicture}
\tikzset{square matrix/.style={
matrix of nodes,
column sep=-\pgflinewidth, row sep=-\pgflinewidth,
nodes={draw,
minimum height=#1,
anchor=center,
text width=#1,
align=center,
inner sep=0pt
},
},
square matrix/.default=1.2cm
}
\matrix[square matrix=1.4em] {
|[fill=green]|\color[gray]{0.75} FO%
 &|[fill=green]|\color[gray]{0.75} FO%
 &|[fill=green]|\color[gray]{0.75} FO%
 &|[fill=green]|\color[gray]{0.75} FO%
 &|[fill=green]|\color[gray]{0.75} FO%
 &|[fill=green]|\color[gray]{0.75} FO%
 &|[fill=green]|\color[gray]{0.75} FO%
 &|[fill=green]|\color[gray]{0.75} FO%
 &|[fill=green]|\color[gray]{0.75} FO%
 &|[fill=green]|\color[gray]{0.75} FO%
 &|[fill=green]|\color[gray]{0.75} FO%
 &|[fill=green]|\color[gray]{0.75} FO%
 &|[fill=green]|\color[gray]{0.75} FO%
\\
|[fill=green]|\color[gray]{0.75} FO%
 &|[fill=white]|\color[gray]{0.5}WA%
 &|[fill=white]|\color[gray]{0.5}WA%
 &|[fill=white]|\color[gray]{0.5}WA%
 &|[fill=white]|\color[gray]{0.5}WA%
 &|[fill=white]|\color[gray]{0.5}WA%
 &|[fill=green]|\color[gray]{0.75} FO%
 &|[fill=white]|\color[gray]{0.5}WA%
 &|[fill=white]|\color[gray]{0.5}WA%
 &|[fill=white]|\color[gray]{0.5}WA%
 &|[fill=white]|\color[gray]{0.5}WA%
 &|[fill=white]|\color[gray]{0.5}WA%
 &|[fill=green]|\color[gray]{0.75} FO%
\\
|[fill=green]|\color[gray]{0.75} FO%
 &|[fill=white]|\color[gray]{0.5}WA%
 &|[fill=green]|\color[gray]{0.75} FO%
 &|[fill=green]|\color[gray]{0.75} FO%
 &|[fill=green]|\color[rgb]{1,0,0}\textbf{BU}%
 &|[fill=green]|\color[rgb]{0,0,0}\textbf{CO}%
 &|[fill=cyan]|\color[rgb]{1,0,0}\textbf{04}%
 &|[fill=green]|\color[rgb]{0,0,0}\textbf{CO}%
 &|[fill=green]|\color[rgb]{1,0,0}\textbf{BU}%
 &|[fill=green]|\color[gray]{0.75} FO%
 &|[fill=green]|\color[gray]{0.75} FO%
 &|[fill=white]|\color[gray]{0.5}WA%
 &|[fill=green]|\color[gray]{0.75} FO%
\\
|[fill=green]|\color[gray]{0.75} FO%
 &|[fill=white]|\color[gray]{0.5}WA%
 &|[fill=green]|\color[gray]{0.75} FO%
 &|[fill=green]|\color[rgb]{1,0,0}\textbf{BU}%
 &|[fill=green]|\color[rgb]{0,0,0}\textbf{CO}%
 &|[fill=green]|\color[rgb]{0,0,0}\textbf{CO}%
 &|[fill=green]|\color[rgb]{0,0,0}\textbf{CO}%
 &|[fill=green]|\color[rgb]{0,0,0}\textbf{CO}%
 &|[fill=green]|\color[rgb]{0,0,0}\textbf{CO}%
 &|[fill=green]|\color[rgb]{1,0,0}\textbf{BU}%
 &|[fill=green]|\color[gray]{0.75} FO%
 &|[fill=white]|\color[gray]{0.5}WA%
 &|[fill=green]|\color[gray]{0.75} FO%
\\
|[fill=green]|\color[gray]{0.75} FO%
 &|[fill=white]|\color[gray]{0.5}WA%
 &|[fill=green]|\color[rgb]{1,0,0}\textbf{BU}%
 &|[fill=green]|\color[rgb]{0,0,0}\textbf{CO}%
 &|[fill=green]|\color[rgb]{0,0,0}\textbf{CO}%
 &|[fill=green]|\color[rgb]{0,0,0}\textbf{CO}%
 &|[fill=green]|\color[rgb]{0,0,0}\textbf{CO}%
 &|[fill=green]|\color[rgb]{0,0,0}\textbf{CO}%
 &|[fill=green]|\color[rgb]{0,0,0}\textbf{CO}%
 &|[fill=green]|\color[rgb]{0,0,0}\textbf{CO}%
 &|[fill=green]|\color[rgb]{1,0,0}\textbf{BU}%
 &|[fill=white]|\color[gray]{0.5}WA%
 &|[fill=green]|\color[gray]{0.75} FO%
\\
|[fill=green]|\color[gray]{0.75} FO%
 &|[fill=white]|\color[gray]{0.5}WA%
 &|[fill=green]|\color[rgb]{0,0,0}\textbf{CO}%
 &|[fill=green]|\color[rgb]{0,0,0}\textbf{CO}%
 &|[fill=green]|\color[rgb]{0,0,0}\textbf{CO}%
 &|[fill=green]|\color[rgb]{0,0,0}\textbf{CO}%
 &|[fill=green]|\color[rgb]{0,0,0}\textbf{CO}%
 &|[fill=green]|\color[rgb]{0,0,0}\textbf{CO}%
 &|[fill=green]|\color[rgb]{0,0,0}\textbf{CO}%
 &|[fill=green]|\color[rgb]{0,0,0}\textbf{CO}%
 &|[fill=green]|\color[rgb]{0,0,0}\textbf{CO}%
 &|[fill=white]|\color[gray]{0.5}WA%
 &|[fill=green]|\color[gray]{0.75} FO%
\\
|[fill=green]|\color[gray]{0.75} FO%
 &|[fill=green]|\color[gray]{0.75} FO%
 &|[fill=green]|\color[rgb]{1,0,0}\textbf{BU}%
 &|[fill=cyan]|\color[rgb]{1,0,0}\textbf{03}%
 &|[fill=green]|\color[rgb]{0,0,0}\textbf{CO}%
 &|[fill=green]|\color[rgb]{0,0,0}\textbf{CO}%
 &|[fill=green]|\color[rgb]{0,0,0}\textbf{CO}%
 &|[fill=cyan]|\color[rgb]{1,0,0}\textbf{01}%
 &|[fill=green]|\color[rgb]{0,0,0}\textbf{CO}%
 &|[fill=green]|\color[rgb]{0,0,0}\textbf{CO}%
 &|[fill=cyan]|\color[rgb]{1,0,0}\textbf{06}%
 &|[fill=green]|\color[gray]{0.75} FO%
 &|[fill=green]|\color[gray]{0.75} FO%
\\
|[fill=green]|\color[gray]{0.75} FO%
 &|[fill=white]|\color[gray]{0.5}WA%
 &|[fill=green]|\color[rgb]{0,0,0}\textbf{CO}%
 &|[fill=green]|\color[rgb]{0,0,0}\textbf{CO}%
 &|[fill=green]|\color[rgb]{0,0,0}\textbf{CO}%
 &|[fill=green]|\color[rgb]{0,0,0}\textbf{CO}%
 &|[fill=green]|\color[rgb]{0,0,0}\textbf{CO}%
 &|[fill=green]|\color[rgb]{0,0,0}\textbf{CO}%
 &|[fill=green]|\color[rgb]{0,0,0}\textbf{CO}%
 &|[fill=green]|\color[rgb]{0,0,0}\textbf{CO}%
 &|[fill=cyan]|\color[rgb]{1,0,0}\textbf{05}%
 &|[fill=white]|\color[gray]{0.5}WA%
 &|[fill=green]|\color[gray]{0.75} FO%
\\
|[fill=green]|\color[gray]{0.75} FO%
 &|[fill=white]|\color[gray]{0.5}WA%
 &|[fill=green]|\color[rgb]{1,0,0}\textbf{BU}%
 &|[fill=green]|\color[rgb]{0,0,0}\textbf{CO}%
 &|[fill=green]|\color[rgb]{0,0,0}\textbf{CO}%
 &|[fill=green]|\color[rgb]{0,0,0}\textbf{CO}%
 &|[fill=cyan]|\color[rgb]{1,0,0}\textbf{02}%
 &|[fill=green]|\color[rgb]{0,0,0}\textbf{CO}%
 &|[fill=green]|\color[rgb]{0,0,0}\textbf{CO}%
 &|[fill=green]|\color[rgb]{0,0,0}\textbf{CO}%
 &|[fill=green]|\color[rgb]{1,0,0}\textbf{BU}%
 &|[fill=white]|\color[gray]{0.5}WA%
 &|[fill=green]|\color[gray]{0.75} FO%
\\
|[fill=green]|\color[gray]{0.75} FO%
 &|[fill=white]|\color[gray]{0.5}WA%
 &|[fill=green]|\color[gray]{0.75} FO%
 &|[fill=green]|\color[rgb]{1,0,0}\textbf{BU}%
 &|[fill=green]|\color[rgb]{0,0,0}\textbf{CO}%
 &|[fill=green]|\color[rgb]{0,0,0}\textbf{CO}%
 &|[fill=green]|\color[rgb]{0,0,0}\textbf{CO}%
 &|[fill=green]|\color[rgb]{0,0,0}\textbf{CO}%
 &|[fill=green]|\color[rgb]{0,0,0}\textbf{CO}%
 &|[fill=green]|\color[rgb]{1,0,0}\textbf{BU}%
 &|[fill=green]|\color[gray]{0.75} FO%
 &|[fill=white]|\color[gray]{0.5}WA%
 &|[fill=green]|\color[gray]{0.75} FO%
\\
|[fill=green]|\color[gray]{0.75} FO%
 &|[fill=white]|\color[gray]{0.5}WA%
 &|[fill=green]|\color[gray]{0.75} FO%
 &|[fill=green]|\color[gray]{0.75} FO%
 &|[fill=green]|\color[rgb]{1,0,0}\textbf{BU}%
 &|[fill=green]|\color[rgb]{0,0,0}\textbf{CO}%
 &|[fill=green]|\color[rgb]{1,0,0}\textbf{BU}%
 &|[fill=green]|\color[rgb]{0,0,0}\textbf{CO}%
 &|[fill=green]|\color[rgb]{1,0,0}\textbf{BU}%
 &|[fill=green]|\color[gray]{0.75} FO%
 &|[fill=green]|\color[gray]{0.75} FO%
 &|[fill=white]|\color[gray]{0.5}WA%
 &|[fill=green]|\color[gray]{0.75} FO%
\\
|[fill=green]|\color[gray]{0.75} FO%
 &|[fill=white]|\color[gray]{0.5}WA%
 &|[fill=white]|\color[gray]{0.5}WA%
 &|[fill=white]|\color[gray]{0.5}WA%
 &|[fill=white]|\color[gray]{0.5}WA%
 &|[fill=white]|\color[gray]{0.5}WA%
 &|[fill=green]|\color[gray]{0.75} FO%
 &|[fill=white]|\color[gray]{0.5}WA%
 &|[fill=white]|\color[gray]{0.5}WA%
 &|[fill=white]|\color[gray]{0.5}WA%
 &|[fill=white]|\color[gray]{0.5}WA%
 &|[fill=white]|\color[gray]{0.5}WA%
 &|[fill=green]|\color[gray]{0.75} FO%
\\
|[fill=green]|\color[gray]{0.75} FO%
 &|[fill=green]|\color[gray]{0.75} FO%
 &|[fill=green]|\color[gray]{0.75} FO%
 &|[fill=green]|\color[gray]{0.75} FO%
 &|[fill=green]|\color[gray]{0.75} FO%
 &|[fill=green]|\color[gray]{0.75} FO%
 &|[fill=green]|\color[gray]{0.75} FO%
 &|[fill=green]|\color[gray]{0.75} FO%
 &|[fill=green]|\color[gray]{0.75} FO%
 &|[fill=green]|\color[gray]{0.75} FO%
 &|[fill=green]|\color[gray]{0.75} FO%
 &|[fill=green]|\color[gray]{0.75} FO%
 &|[fill=green]|\color[gray]{0.75} FO%
\\
};
\end{tikzpicture}\\
At time 7: Water spot (6|11):
\\
\begin{tikzpicture}
\tikzset{square matrix/.style={
matrix of nodes,
column sep=-\pgflinewidth, row sep=-\pgflinewidth,
nodes={draw,
minimum height=#1,
anchor=center,
text width=#1,
align=center,
inner sep=0pt
},
},
square matrix/.default=1.2cm
}
\matrix[square matrix=1.4em] {
|[fill=green]|\color[gray]{0.75} FO%
 &|[fill=green]|\color[gray]{0.75} FO%
 &|[fill=green]|\color[gray]{0.75} FO%
 &|[fill=green]|\color[gray]{0.75} FO%
 &|[fill=green]|\color[gray]{0.75} FO%
 &|[fill=green]|\color[gray]{0.75} FO%
 &|[fill=green]|\color[gray]{0.75} FO%
 &|[fill=green]|\color[gray]{0.75} FO%
 &|[fill=green]|\color[gray]{0.75} FO%
 &|[fill=green]|\color[gray]{0.75} FO%
 &|[fill=green]|\color[gray]{0.75} FO%
 &|[fill=green]|\color[gray]{0.75} FO%
 &|[fill=green]|\color[gray]{0.75} FO%
\\
|[fill=green]|\color[gray]{0.75} FO%
 &|[fill=white]|\color[gray]{0.5}WA%
 &|[fill=white]|\color[gray]{0.5}WA%
 &|[fill=white]|\color[gray]{0.5}WA%
 &|[fill=white]|\color[gray]{0.5}WA%
 &|[fill=white]|\color[gray]{0.5}WA%
 &|[fill=green]|\color[gray]{0.75} FO%
 &|[fill=white]|\color[gray]{0.5}WA%
 &|[fill=white]|\color[gray]{0.5}WA%
 &|[fill=white]|\color[gray]{0.5}WA%
 &|[fill=white]|\color[gray]{0.5}WA%
 &|[fill=white]|\color[gray]{0.5}WA%
 &|[fill=green]|\color[gray]{0.75} FO%
\\
|[fill=green]|\color[gray]{0.75} FO%
 &|[fill=white]|\color[gray]{0.5}WA%
 &|[fill=green]|\color[gray]{0.75} FO%
 &|[fill=green]|\color[rgb]{1,0,0}\textbf{BU}%
 &|[fill=green]|\color[rgb]{0,0,0}\textbf{CO}%
 &|[fill=green]|\color[rgb]{0,0,0}\textbf{CO}%
 &|[fill=cyan]|\color[rgb]{1,0,0}\textbf{04}%
 &|[fill=green]|\color[rgb]{0,0,0}\textbf{CO}%
 &|[fill=green]|\color[rgb]{0,0,0}\textbf{CO}%
 &|[fill=green]|\color[rgb]{1,0,0}\textbf{BU}%
 &|[fill=green]|\color[gray]{0.75} FO%
 &|[fill=white]|\color[gray]{0.5}WA%
 &|[fill=green]|\color[gray]{0.75} FO%
\\
|[fill=green]|\color[gray]{0.75} FO%
 &|[fill=white]|\color[gray]{0.5}WA%
 &|[fill=green]|\color[rgb]{1,0,0}\textbf{BU}%
 &|[fill=green]|\color[rgb]{0,0,0}\textbf{CO}%
 &|[fill=green]|\color[rgb]{0,0,0}\textbf{CO}%
 &|[fill=green]|\color[rgb]{0,0,0}\textbf{CO}%
 &|[fill=green]|\color[rgb]{0,0,0}\textbf{CO}%
 &|[fill=green]|\color[rgb]{0,0,0}\textbf{CO}%
 &|[fill=green]|\color[rgb]{0,0,0}\textbf{CO}%
 &|[fill=green]|\color[rgb]{0,0,0}\textbf{CO}%
 &|[fill=green]|\color[rgb]{1,0,0}\textbf{BU}%
 &|[fill=white]|\color[gray]{0.5}WA%
 &|[fill=green]|\color[gray]{0.75} FO%
\\
|[fill=green]|\color[gray]{0.75} FO%
 &|[fill=white]|\color[gray]{0.5}WA%
 &|[fill=green]|\color[rgb]{0,0,0}\textbf{CO}%
 &|[fill=green]|\color[rgb]{0,0,0}\textbf{CO}%
 &|[fill=green]|\color[rgb]{0,0,0}\textbf{CO}%
 &|[fill=green]|\color[rgb]{0,0,0}\textbf{CO}%
 &|[fill=green]|\color[rgb]{0,0,0}\textbf{CO}%
 &|[fill=green]|\color[rgb]{0,0,0}\textbf{CO}%
 &|[fill=green]|\color[rgb]{0,0,0}\textbf{CO}%
 &|[fill=green]|\color[rgb]{0,0,0}\textbf{CO}%
 &|[fill=green]|\color[rgb]{0,0,0}\textbf{CO}%
 &|[fill=white]|\color[gray]{0.5}WA%
 &|[fill=green]|\color[gray]{0.75} FO%
\\
|[fill=green]|\color[gray]{0.75} FO%
 &|[fill=white]|\color[gray]{0.5}WA%
 &|[fill=green]|\color[rgb]{0,0,0}\textbf{CO}%
 &|[fill=green]|\color[rgb]{0,0,0}\textbf{CO}%
 &|[fill=green]|\color[rgb]{0,0,0}\textbf{CO}%
 &|[fill=green]|\color[rgb]{0,0,0}\textbf{CO}%
 &|[fill=green]|\color[rgb]{0,0,0}\textbf{CO}%
 &|[fill=green]|\color[rgb]{0,0,0}\textbf{CO}%
 &|[fill=green]|\color[rgb]{0,0,0}\textbf{CO}%
 &|[fill=green]|\color[rgb]{0,0,0}\textbf{CO}%
 &|[fill=green]|\color[rgb]{0,0,0}\textbf{CO}%
 &|[fill=white]|\color[gray]{0.5}WA%
 &|[fill=green]|\color[gray]{0.75} FO%
\\
|[fill=green]|\color[gray]{0.75} FO%
 &|[fill=green]|\color[rgb]{1,0,0}\textbf{BU}%
 &|[fill=green]|\color[rgb]{0,0,0}\textbf{CO}%
 &|[fill=cyan]|\color[rgb]{1,0,0}\textbf{03}%
 &|[fill=green]|\color[rgb]{0,0,0}\textbf{CO}%
 &|[fill=green]|\color[rgb]{0,0,0}\textbf{CO}%
 &|[fill=green]|\color[rgb]{0,0,0}\textbf{CO}%
 &|[fill=cyan]|\color[rgb]{1,0,0}\textbf{01}%
 &|[fill=green]|\color[rgb]{0,0,0}\textbf{CO}%
 &|[fill=green]|\color[rgb]{0,0,0}\textbf{CO}%
 &|[fill=cyan]|\color[rgb]{1,0,0}\textbf{06}%
 &|[fill=green]|\color[gray]{0.75} FO%
 &|[fill=green]|\color[gray]{0.75} FO%
\\
|[fill=green]|\color[gray]{0.75} FO%
 &|[fill=white]|\color[gray]{0.5}WA%
 &|[fill=green]|\color[rgb]{0,0,0}\textbf{CO}%
 &|[fill=green]|\color[rgb]{0,0,0}\textbf{CO}%
 &|[fill=green]|\color[rgb]{0,0,0}\textbf{CO}%
 &|[fill=green]|\color[rgb]{0,0,0}\textbf{CO}%
 &|[fill=green]|\color[rgb]{0,0,0}\textbf{CO}%
 &|[fill=green]|\color[rgb]{0,0,0}\textbf{CO}%
 &|[fill=green]|\color[rgb]{0,0,0}\textbf{CO}%
 &|[fill=green]|\color[rgb]{0,0,0}\textbf{CO}%
 &|[fill=cyan]|\color[rgb]{1,0,0}\textbf{05}%
 &|[fill=white]|\color[gray]{0.5}WA%
 &|[fill=green]|\color[gray]{0.75} FO%
\\
|[fill=green]|\color[gray]{0.75} FO%
 &|[fill=white]|\color[gray]{0.5}WA%
 &|[fill=green]|\color[rgb]{0,0,0}\textbf{CO}%
 &|[fill=green]|\color[rgb]{0,0,0}\textbf{CO}%
 &|[fill=green]|\color[rgb]{0,0,0}\textbf{CO}%
 &|[fill=green]|\color[rgb]{0,0,0}\textbf{CO}%
 &|[fill=cyan]|\color[rgb]{1,0,0}\textbf{02}%
 &|[fill=green]|\color[rgb]{0,0,0}\textbf{CO}%
 &|[fill=green]|\color[rgb]{0,0,0}\textbf{CO}%
 &|[fill=green]|\color[rgb]{0,0,0}\textbf{CO}%
 &|[fill=green]|\color[rgb]{0,0,0}\textbf{CO}%
 &|[fill=white]|\color[gray]{0.5}WA%
 &|[fill=green]|\color[gray]{0.75} FO%
\\
|[fill=green]|\color[gray]{0.75} FO%
 &|[fill=white]|\color[gray]{0.5}WA%
 &|[fill=green]|\color[rgb]{1,0,0}\textbf{BU}%
 &|[fill=green]|\color[rgb]{0,0,0}\textbf{CO}%
 &|[fill=green]|\color[rgb]{0,0,0}\textbf{CO}%
 &|[fill=green]|\color[rgb]{0,0,0}\textbf{CO}%
 &|[fill=green]|\color[rgb]{0,0,0}\textbf{CO}%
 &|[fill=green]|\color[rgb]{0,0,0}\textbf{CO}%
 &|[fill=green]|\color[rgb]{0,0,0}\textbf{CO}%
 &|[fill=green]|\color[rgb]{0,0,0}\textbf{CO}%
 &|[fill=green]|\color[rgb]{1,0,0}\textbf{BU}%
 &|[fill=white]|\color[gray]{0.5}WA%
 &|[fill=green]|\color[gray]{0.75} FO%
\\
|[fill=green]|\color[gray]{0.75} FO%
 &|[fill=white]|\color[gray]{0.5}WA%
 &|[fill=green]|\color[gray]{0.75} FO%
 &|[fill=green]|\color[rgb]{1,0,0}\textbf{BU}%
 &|[fill=green]|\color[rgb]{0,0,0}\textbf{CO}%
 &|[fill=green]|\color[rgb]{0,0,0}\textbf{CO}%
 &|[fill=green]|\color[rgb]{0,0,0}\textbf{CO}%
 &|[fill=green]|\color[rgb]{0,0,0}\textbf{CO}%
 &|[fill=green]|\color[rgb]{0,0,0}\textbf{CO}%
 &|[fill=green]|\color[rgb]{1,0,0}\textbf{BU}%
 &|[fill=green]|\color[gray]{0.75} FO%
 &|[fill=white]|\color[gray]{0.5}WA%
 &|[fill=green]|\color[gray]{0.75} FO%
\\
|[fill=green]|\color[gray]{0.75} FO%
 &|[fill=white]|\color[gray]{0.5}WA%
 &|[fill=white]|\color[gray]{0.5}WA%
 &|[fill=white]|\color[gray]{0.5}WA%
 &|[fill=white]|\color[gray]{0.5}WA%
 &|[fill=white]|\color[gray]{0.5}WA%
 &|[fill=cyan]|\color[rgb]{1,0,0}\textbf{07}%
 &|[fill=white]|\color[gray]{0.5}WA%
 &|[fill=white]|\color[gray]{0.5}WA%
 &|[fill=white]|\color[gray]{0.5}WA%
 &|[fill=white]|\color[gray]{0.5}WA%
 &|[fill=white]|\color[gray]{0.5}WA%
 &|[fill=green]|\color[gray]{0.75} FO%
\\
|[fill=green]|\color[gray]{0.75} FO%
 &|[fill=green]|\color[gray]{0.75} FO%
 &|[fill=green]|\color[gray]{0.75} FO%
 &|[fill=green]|\color[gray]{0.75} FO%
 &|[fill=green]|\color[gray]{0.75} FO%
 &|[fill=green]|\color[gray]{0.75} FO%
 &|[fill=green]|\color[gray]{0.75} FO%
 &|[fill=green]|\color[gray]{0.75} FO%
 &|[fill=green]|\color[gray]{0.75} FO%
 &|[fill=green]|\color[gray]{0.75} FO%
 &|[fill=green]|\color[gray]{0.75} FO%
 &|[fill=green]|\color[gray]{0.75} FO%
 &|[fill=green]|\color[gray]{0.75} FO%
\\
};
\end{tikzpicture}\\
At time 8: Water spot (0|6):
\\
\begin{tikzpicture}
\tikzset{square matrix/.style={
matrix of nodes,
column sep=-\pgflinewidth, row sep=-\pgflinewidth,
nodes={draw,
minimum height=#1,
anchor=center,
text width=#1,
align=center,
inner sep=0pt
},
},
square matrix/.default=1.2cm
}
\matrix[square matrix=1.4em] {
|[fill=green]|\color[gray]{0.75} FO%
 &|[fill=green]|\color[gray]{0.75} FO%
 &|[fill=green]|\color[gray]{0.75} FO%
 &|[fill=green]|\color[gray]{0.75} FO%
 &|[fill=green]|\color[gray]{0.75} FO%
 &|[fill=green]|\color[gray]{0.75} FO%
 &|[fill=green]|\color[gray]{0.75} FO%
 &|[fill=green]|\color[gray]{0.75} FO%
 &|[fill=green]|\color[gray]{0.75} FO%
 &|[fill=green]|\color[gray]{0.75} FO%
 &|[fill=green]|\color[gray]{0.75} FO%
 &|[fill=green]|\color[gray]{0.75} FO%
 &|[fill=green]|\color[gray]{0.75} FO%
\\
|[fill=green]|\color[gray]{0.75} FO%
 &|[fill=white]|\color[gray]{0.5}WA%
 &|[fill=white]|\color[gray]{0.5}WA%
 &|[fill=white]|\color[gray]{0.5}WA%
 &|[fill=white]|\color[gray]{0.5}WA%
 &|[fill=white]|\color[gray]{0.5}WA%
 &|[fill=green]|\color[gray]{0.75} FO%
 &|[fill=white]|\color[gray]{0.5}WA%
 &|[fill=white]|\color[gray]{0.5}WA%
 &|[fill=white]|\color[gray]{0.5}WA%
 &|[fill=white]|\color[gray]{0.5}WA%
 &|[fill=white]|\color[gray]{0.5}WA%
 &|[fill=green]|\color[gray]{0.75} FO%
\\
|[fill=green]|\color[gray]{0.75} FO%
 &|[fill=white]|\color[gray]{0.5}WA%
 &|[fill=green]|\color[rgb]{1,0,0}\textbf{BU}%
 &|[fill=green]|\color[rgb]{0,0,0}\textbf{CO}%
 &|[fill=green]|\color[rgb]{0,0,0}\textbf{CO}%
 &|[fill=green]|\color[rgb]{0,0,0}\textbf{CO}%
 &|[fill=cyan]|\color[rgb]{1,0,0}\textbf{04}%
 &|[fill=green]|\color[rgb]{0,0,0}\textbf{CO}%
 &|[fill=green]|\color[rgb]{0,0,0}\textbf{CO}%
 &|[fill=green]|\color[rgb]{0,0,0}\textbf{CO}%
 &|[fill=green]|\color[rgb]{1,0,0}\textbf{BU}%
 &|[fill=white]|\color[gray]{0.5}WA%
 &|[fill=green]|\color[gray]{0.75} FO%
\\
|[fill=green]|\color[gray]{0.75} FO%
 &|[fill=white]|\color[gray]{0.5}WA%
 &|[fill=green]|\color[rgb]{0,0,0}\textbf{CO}%
 &|[fill=green]|\color[rgb]{0,0,0}\textbf{CO}%
 &|[fill=green]|\color[rgb]{0,0,0}\textbf{CO}%
 &|[fill=green]|\color[rgb]{0,0,0}\textbf{CO}%
 &|[fill=green]|\color[rgb]{0,0,0}\textbf{CO}%
 &|[fill=green]|\color[rgb]{0,0,0}\textbf{CO}%
 &|[fill=green]|\color[rgb]{0,0,0}\textbf{CO}%
 &|[fill=green]|\color[rgb]{0,0,0}\textbf{CO}%
 &|[fill=green]|\color[rgb]{0,0,0}\textbf{CO}%
 &|[fill=white]|\color[gray]{0.5}WA%
 &|[fill=green]|\color[gray]{0.75} FO%
\\
|[fill=green]|\color[gray]{0.75} FO%
 &|[fill=white]|\color[gray]{0.5}WA%
 &|[fill=green]|\color[rgb]{0,0,0}\textbf{CO}%
 &|[fill=green]|\color[rgb]{0,0,0}\textbf{CO}%
 &|[fill=green]|\color[rgb]{0,0,0}\textbf{CO}%
 &|[fill=green]|\color[rgb]{0,0,0}\textbf{CO}%
 &|[fill=green]|\color[rgb]{0,0,0}\textbf{CO}%
 &|[fill=green]|\color[rgb]{0,0,0}\textbf{CO}%
 &|[fill=green]|\color[rgb]{0,0,0}\textbf{CO}%
 &|[fill=green]|\color[rgb]{0,0,0}\textbf{CO}%
 &|[fill=green]|\color[rgb]{0,0,0}\textbf{CO}%
 &|[fill=white]|\color[gray]{0.5}WA%
 &|[fill=green]|\color[gray]{0.75} FO%
\\
|[fill=green]|\color[gray]{0.75} FO%
 &|[fill=white]|\color[gray]{0.5}WA%
 &|[fill=green]|\color[rgb]{0,0,0}\textbf{CO}%
 &|[fill=green]|\color[rgb]{0,0,0}\textbf{CO}%
 &|[fill=green]|\color[rgb]{0,0,0}\textbf{CO}%
 &|[fill=green]|\color[rgb]{0,0,0}\textbf{CO}%
 &|[fill=green]|\color[rgb]{0,0,0}\textbf{CO}%
 &|[fill=green]|\color[rgb]{0,0,0}\textbf{CO}%
 &|[fill=green]|\color[rgb]{0,0,0}\textbf{CO}%
 &|[fill=green]|\color[rgb]{0,0,0}\textbf{CO}%
 &|[fill=green]|\color[rgb]{0,0,0}\textbf{CO}%
 &|[fill=white]|\color[gray]{0.5}WA%
 &|[fill=green]|\color[gray]{0.75} FO%
\\
|[fill=cyan]|\color[rgb]{1,0,0}\textbf{08}%
 &|[fill=green]|\color[rgb]{0,0,0}\textbf{CO}%
 &|[fill=green]|\color[rgb]{0,0,0}\textbf{CO}%
 &|[fill=cyan]|\color[rgb]{1,0,0}\textbf{03}%
 &|[fill=green]|\color[rgb]{0,0,0}\textbf{CO}%
 &|[fill=green]|\color[rgb]{0,0,0}\textbf{CO}%
 &|[fill=green]|\color[rgb]{0,0,0}\textbf{CO}%
 &|[fill=cyan]|\color[rgb]{1,0,0}\textbf{01}%
 &|[fill=green]|\color[rgb]{0,0,0}\textbf{CO}%
 &|[fill=green]|\color[rgb]{0,0,0}\textbf{CO}%
 &|[fill=cyan]|\color[rgb]{1,0,0}\textbf{06}%
 &|[fill=green]|\color[gray]{0.75} FO%
 &|[fill=green]|\color[gray]{0.75} FO%
\\
|[fill=green]|\color[gray]{0.75} FO%
 &|[fill=white]|\color[gray]{0.5}WA%
 &|[fill=green]|\color[rgb]{0,0,0}\textbf{CO}%
 &|[fill=green]|\color[rgb]{0,0,0}\textbf{CO}%
 &|[fill=green]|\color[rgb]{0,0,0}\textbf{CO}%
 &|[fill=green]|\color[rgb]{0,0,0}\textbf{CO}%
 &|[fill=green]|\color[rgb]{0,0,0}\textbf{CO}%
 &|[fill=green]|\color[rgb]{0,0,0}\textbf{CO}%
 &|[fill=green]|\color[rgb]{0,0,0}\textbf{CO}%
 &|[fill=green]|\color[rgb]{0,0,0}\textbf{CO}%
 &|[fill=cyan]|\color[rgb]{1,0,0}\textbf{05}%
 &|[fill=white]|\color[gray]{0.5}WA%
 &|[fill=green]|\color[gray]{0.75} FO%
\\
|[fill=green]|\color[gray]{0.75} FO%
 &|[fill=white]|\color[gray]{0.5}WA%
 &|[fill=green]|\color[rgb]{0,0,0}\textbf{CO}%
 &|[fill=green]|\color[rgb]{0,0,0}\textbf{CO}%
 &|[fill=green]|\color[rgb]{0,0,0}\textbf{CO}%
 &|[fill=green]|\color[rgb]{0,0,0}\textbf{CO}%
 &|[fill=cyan]|\color[rgb]{1,0,0}\textbf{02}%
 &|[fill=green]|\color[rgb]{0,0,0}\textbf{CO}%
 &|[fill=green]|\color[rgb]{0,0,0}\textbf{CO}%
 &|[fill=green]|\color[rgb]{0,0,0}\textbf{CO}%
 &|[fill=green]|\color[rgb]{0,0,0}\textbf{CO}%
 &|[fill=white]|\color[gray]{0.5}WA%
 &|[fill=green]|\color[gray]{0.75} FO%
\\
|[fill=green]|\color[gray]{0.75} FO%
 &|[fill=white]|\color[gray]{0.5}WA%
 &|[fill=green]|\color[rgb]{0,0,0}\textbf{CO}%
 &|[fill=green]|\color[rgb]{0,0,0}\textbf{CO}%
 &|[fill=green]|\color[rgb]{0,0,0}\textbf{CO}%
 &|[fill=green]|\color[rgb]{0,0,0}\textbf{CO}%
 &|[fill=green]|\color[rgb]{0,0,0}\textbf{CO}%
 &|[fill=green]|\color[rgb]{0,0,0}\textbf{CO}%
 &|[fill=green]|\color[rgb]{0,0,0}\textbf{CO}%
 &|[fill=green]|\color[rgb]{0,0,0}\textbf{CO}%
 &|[fill=green]|\color[rgb]{0,0,0}\textbf{CO}%
 &|[fill=white]|\color[gray]{0.5}WA%
 &|[fill=green]|\color[gray]{0.75} FO%
\\
|[fill=green]|\color[gray]{0.75} FO%
 &|[fill=white]|\color[gray]{0.5}WA%
 &|[fill=green]|\color[rgb]{1,0,0}\textbf{BU}%
 &|[fill=green]|\color[rgb]{0,0,0}\textbf{CO}%
 &|[fill=green]|\color[rgb]{0,0,0}\textbf{CO}%
 &|[fill=green]|\color[rgb]{0,0,0}\textbf{CO}%
 &|[fill=green]|\color[rgb]{0,0,0}\textbf{CO}%
 &|[fill=green]|\color[rgb]{0,0,0}\textbf{CO}%
 &|[fill=green]|\color[rgb]{0,0,0}\textbf{CO}%
 &|[fill=green]|\color[rgb]{0,0,0}\textbf{CO}%
 &|[fill=green]|\color[rgb]{1,0,0}\textbf{BU}%
 &|[fill=white]|\color[gray]{0.5}WA%
 &|[fill=green]|\color[gray]{0.75} FO%
\\
|[fill=green]|\color[gray]{0.75} FO%
 &|[fill=white]|\color[gray]{0.5}WA%
 &|[fill=white]|\color[gray]{0.5}WA%
 &|[fill=white]|\color[gray]{0.5}WA%
 &|[fill=white]|\color[gray]{0.5}WA%
 &|[fill=white]|\color[gray]{0.5}WA%
 &|[fill=cyan]|\color[rgb]{1,0,0}\textbf{07}%
 &|[fill=white]|\color[gray]{0.5}WA%
 &|[fill=white]|\color[gray]{0.5}WA%
 &|[fill=white]|\color[gray]{0.5}WA%
 &|[fill=white]|\color[gray]{0.5}WA%
 &|[fill=white]|\color[gray]{0.5}WA%
 &|[fill=green]|\color[gray]{0.75} FO%
\\
|[fill=green]|\color[gray]{0.75} FO%
 &|[fill=green]|\color[gray]{0.75} FO%
 &|[fill=green]|\color[gray]{0.75} FO%
 &|[fill=green]|\color[gray]{0.75} FO%
 &|[fill=green]|\color[gray]{0.75} FO%
 &|[fill=green]|\color[gray]{0.75} FO%
 &|[fill=green]|\color[gray]{0.75} FO%
 &|[fill=green]|\color[gray]{0.75} FO%
 &|[fill=green]|\color[gray]{0.75} FO%
 &|[fill=green]|\color[gray]{0.75} FO%
 &|[fill=green]|\color[gray]{0.75} FO%
 &|[fill=green]|\color[gray]{0.75} FO%
 &|[fill=green]|\color[gray]{0.75} FO%
\\
};
\end{tikzpicture}\\
And you'll find 76 pieces of coal and 8 pieces of watered coal:
\\
\begin{tikzpicture}
\tikzset{square matrix/.style={
matrix of nodes,
column sep=-\pgflinewidth, row sep=-\pgflinewidth,
nodes={draw,
minimum height=#1,
anchor=center,
text width=#1,
align=center,
inner sep=0pt
},
},
square matrix/.default=1.2cm
}
\matrix[square matrix=1.4em] {
|[fill=green]|\color[gray]{0.75} FO%
 &|[fill=green]|\color[gray]{0.75} FO%
 &|[fill=green]|\color[gray]{0.75} FO%
 &|[fill=green]|\color[gray]{0.75} FO%
 &|[fill=green]|\color[gray]{0.75} FO%
 &|[fill=green]|\color[gray]{0.75} FO%
 &|[fill=green]|\color[gray]{0.75} FO%
 &|[fill=green]|\color[gray]{0.75} FO%
 &|[fill=green]|\color[gray]{0.75} FO%
 &|[fill=green]|\color[gray]{0.75} FO%
 &|[fill=green]|\color[gray]{0.75} FO%
 &|[fill=green]|\color[gray]{0.75} FO%
 &|[fill=green]|\color[gray]{0.75} FO%
\\
|[fill=green]|\color[gray]{0.75} FO%
 &|[fill=white]|\color[gray]{0.5}WA%
 &|[fill=white]|\color[gray]{0.5}WA%
 &|[fill=white]|\color[gray]{0.5}WA%
 &|[fill=white]|\color[gray]{0.5}WA%
 &|[fill=white]|\color[gray]{0.5}WA%
 &|[fill=green]|\color[gray]{0.75} FO%
 &|[fill=white]|\color[gray]{0.5}WA%
 &|[fill=white]|\color[gray]{0.5}WA%
 &|[fill=white]|\color[gray]{0.5}WA%
 &|[fill=white]|\color[gray]{0.5}WA%
 &|[fill=white]|\color[gray]{0.5}WA%
 &|[fill=green]|\color[gray]{0.75} FO%
\\
|[fill=green]|\color[gray]{0.75} FO%
 &|[fill=white]|\color[gray]{0.5}WA%
 &|[fill=green]|\color[rgb]{0,0,0}\textbf{CO}%
 &|[fill=green]|\color[rgb]{0,0,0}\textbf{CO}%
 &|[fill=green]|\color[rgb]{0,0,0}\textbf{CO}%
 &|[fill=green]|\color[rgb]{0,0,0}\textbf{CO}%
 &|[fill=cyan]|\color[rgb]{1,0,0}\textbf{04}%
 &|[fill=green]|\color[rgb]{0,0,0}\textbf{CO}%
 &|[fill=green]|\color[rgb]{0,0,0}\textbf{CO}%
 &|[fill=green]|\color[rgb]{0,0,0}\textbf{CO}%
 &|[fill=green]|\color[rgb]{0,0,0}\textbf{CO}%
 &|[fill=white]|\color[gray]{0.5}WA%
 &|[fill=green]|\color[gray]{0.75} FO%
\\
|[fill=green]|\color[gray]{0.75} FO%
 &|[fill=white]|\color[gray]{0.5}WA%
 &|[fill=green]|\color[rgb]{0,0,0}\textbf{CO}%
 &|[fill=green]|\color[rgb]{0,0,0}\textbf{CO}%
 &|[fill=green]|\color[rgb]{0,0,0}\textbf{CO}%
 &|[fill=green]|\color[rgb]{0,0,0}\textbf{CO}%
 &|[fill=green]|\color[rgb]{0,0,0}\textbf{CO}%
 &|[fill=green]|\color[rgb]{0,0,0}\textbf{CO}%
 &|[fill=green]|\color[rgb]{0,0,0}\textbf{CO}%
 &|[fill=green]|\color[rgb]{0,0,0}\textbf{CO}%
 &|[fill=green]|\color[rgb]{0,0,0}\textbf{CO}%
 &|[fill=white]|\color[gray]{0.5}WA%
 &|[fill=green]|\color[gray]{0.75} FO%
\\
|[fill=green]|\color[gray]{0.75} FO%
 &|[fill=white]|\color[gray]{0.5}WA%
 &|[fill=green]|\color[rgb]{0,0,0}\textbf{CO}%
 &|[fill=green]|\color[rgb]{0,0,0}\textbf{CO}%
 &|[fill=green]|\color[rgb]{0,0,0}\textbf{CO}%
 &|[fill=green]|\color[rgb]{0,0,0}\textbf{CO}%
 &|[fill=green]|\color[rgb]{0,0,0}\textbf{CO}%
 &|[fill=green]|\color[rgb]{0,0,0}\textbf{CO}%
 &|[fill=green]|\color[rgb]{0,0,0}\textbf{CO}%
 &|[fill=green]|\color[rgb]{0,0,0}\textbf{CO}%
 &|[fill=green]|\color[rgb]{0,0,0}\textbf{CO}%
 &|[fill=white]|\color[gray]{0.5}WA%
 &|[fill=green]|\color[gray]{0.75} FO%
\\
|[fill=green]|\color[gray]{0.75} FO%
 &|[fill=white]|\color[gray]{0.5}WA%
 &|[fill=green]|\color[rgb]{0,0,0}\textbf{CO}%
 &|[fill=green]|\color[rgb]{0,0,0}\textbf{CO}%
 &|[fill=green]|\color[rgb]{0,0,0}\textbf{CO}%
 &|[fill=green]|\color[rgb]{0,0,0}\textbf{CO}%
 &|[fill=green]|\color[rgb]{0,0,0}\textbf{CO}%
 &|[fill=green]|\color[rgb]{0,0,0}\textbf{CO}%
 &|[fill=green]|\color[rgb]{0,0,0}\textbf{CO}%
 &|[fill=green]|\color[rgb]{0,0,0}\textbf{CO}%
 &|[fill=green]|\color[rgb]{0,0,0}\textbf{CO}%
 &|[fill=white]|\color[gray]{0.5}WA%
 &|[fill=green]|\color[gray]{0.75} FO%
\\
|[fill=cyan]|\color[rgb]{1,0,0}\textbf{08}%
 &|[fill=green]|\color[rgb]{0,0,0}\textbf{CO}%
 &|[fill=green]|\color[rgb]{0,0,0}\textbf{CO}%
 &|[fill=cyan]|\color[rgb]{1,0,0}\textbf{03}%
 &|[fill=green]|\color[rgb]{0,0,0}\textbf{CO}%
 &|[fill=green]|\color[rgb]{0,0,0}\textbf{CO}%
 &|[fill=green]|\color[rgb]{0,0,0}\textbf{CO}%
 &|[fill=cyan]|\color[rgb]{1,0,0}\textbf{01}%
 &|[fill=green]|\color[rgb]{0,0,0}\textbf{CO}%
 &|[fill=green]|\color[rgb]{0,0,0}\textbf{CO}%
 &|[fill=cyan]|\color[rgb]{1,0,0}\textbf{06}%
 &|[fill=green]|\color[gray]{0.75} FO%
 &|[fill=green]|\color[gray]{0.75} FO%
\\
|[fill=green]|\color[gray]{0.75} FO%
 &|[fill=white]|\color[gray]{0.5}WA%
 &|[fill=green]|\color[rgb]{0,0,0}\textbf{CO}%
 &|[fill=green]|\color[rgb]{0,0,0}\textbf{CO}%
 &|[fill=green]|\color[rgb]{0,0,0}\textbf{CO}%
 &|[fill=green]|\color[rgb]{0,0,0}\textbf{CO}%
 &|[fill=green]|\color[rgb]{0,0,0}\textbf{CO}%
 &|[fill=green]|\color[rgb]{0,0,0}\textbf{CO}%
 &|[fill=green]|\color[rgb]{0,0,0}\textbf{CO}%
 &|[fill=green]|\color[rgb]{0,0,0}\textbf{CO}%
 &|[fill=cyan]|\color[rgb]{1,0,0}\textbf{05}%
 &|[fill=white]|\color[gray]{0.5}WA%
 &|[fill=green]|\color[gray]{0.75} FO%
\\
|[fill=green]|\color[gray]{0.75} FO%
 &|[fill=white]|\color[gray]{0.5}WA%
 &|[fill=green]|\color[rgb]{0,0,0}\textbf{CO}%
 &|[fill=green]|\color[rgb]{0,0,0}\textbf{CO}%
 &|[fill=green]|\color[rgb]{0,0,0}\textbf{CO}%
 &|[fill=green]|\color[rgb]{0,0,0}\textbf{CO}%
 &|[fill=cyan]|\color[rgb]{1,0,0}\textbf{02}%
 &|[fill=green]|\color[rgb]{0,0,0}\textbf{CO}%
 &|[fill=green]|\color[rgb]{0,0,0}\textbf{CO}%
 &|[fill=green]|\color[rgb]{0,0,0}\textbf{CO}%
 &|[fill=green]|\color[rgb]{0,0,0}\textbf{CO}%
 &|[fill=white]|\color[gray]{0.5}WA%
 &|[fill=green]|\color[gray]{0.75} FO%
\\
|[fill=green]|\color[gray]{0.75} FO%
 &|[fill=white]|\color[gray]{0.5}WA%
 &|[fill=green]|\color[rgb]{0,0,0}\textbf{CO}%
 &|[fill=green]|\color[rgb]{0,0,0}\textbf{CO}%
 &|[fill=green]|\color[rgb]{0,0,0}\textbf{CO}%
 &|[fill=green]|\color[rgb]{0,0,0}\textbf{CO}%
 &|[fill=green]|\color[rgb]{0,0,0}\textbf{CO}%
 &|[fill=green]|\color[rgb]{0,0,0}\textbf{CO}%
 &|[fill=green]|\color[rgb]{0,0,0}\textbf{CO}%
 &|[fill=green]|\color[rgb]{0,0,0}\textbf{CO}%
 &|[fill=green]|\color[rgb]{0,0,0}\textbf{CO}%
 &|[fill=white]|\color[gray]{0.5}WA%
 &|[fill=green]|\color[gray]{0.75} FO%
\\
|[fill=green]|\color[gray]{0.75} FO%
 &|[fill=white]|\color[gray]{0.5}WA%
 &|[fill=green]|\color[rgb]{0,0,0}\textbf{CO}%
 &|[fill=green]|\color[rgb]{0,0,0}\textbf{CO}%
 &|[fill=green]|\color[rgb]{0,0,0}\textbf{CO}%
 &|[fill=green]|\color[rgb]{0,0,0}\textbf{CO}%
 &|[fill=green]|\color[rgb]{0,0,0}\textbf{CO}%
 &|[fill=green]|\color[rgb]{0,0,0}\textbf{CO}%
 &|[fill=green]|\color[rgb]{0,0,0}\textbf{CO}%
 &|[fill=green]|\color[rgb]{0,0,0}\textbf{CO}%
 &|[fill=green]|\color[rgb]{0,0,0}\textbf{CO}%
 &|[fill=white]|\color[gray]{0.5}WA%
 &|[fill=green]|\color[gray]{0.75} FO%
\\
|[fill=green]|\color[gray]{0.75} FO%
 &|[fill=white]|\color[gray]{0.5}WA%
 &|[fill=white]|\color[gray]{0.5}WA%
 &|[fill=white]|\color[gray]{0.5}WA%
 &|[fill=white]|\color[gray]{0.5}WA%
 &|[fill=white]|\color[gray]{0.5}WA%
 &|[fill=cyan]|\color[rgb]{1,0,0}\textbf{07}%
 &|[fill=white]|\color[gray]{0.5}WA%
 &|[fill=white]|\color[gray]{0.5}WA%
 &|[fill=white]|\color[gray]{0.5}WA%
 &|[fill=white]|\color[gray]{0.5}WA%
 &|[fill=white]|\color[gray]{0.5}WA%
 &|[fill=green]|\color[gray]{0.75} FO%
\\
|[fill=green]|\color[gray]{0.75} FO%
 &|[fill=green]|\color[gray]{0.75} FO%
 &|[fill=green]|\color[gray]{0.75} FO%
 &|[fill=green]|\color[gray]{0.75} FO%
 &|[fill=green]|\color[gray]{0.75} FO%
 &|[fill=green]|\color[gray]{0.75} FO%
 &|[fill=green]|\color[gray]{0.75} FO%
 &|[fill=green]|\color[gray]{0.75} FO%
 &|[fill=green]|\color[gray]{0.75} FO%
 &|[fill=green]|\color[gray]{0.75} FO%
 &|[fill=green]|\color[gray]{0.75} FO%
 &|[fill=green]|\color[gray]{0.75} FO%
 &|[fill=green]|\color[gray]{0.75} FO%
\\
};
\end{tikzpicture}\\
\\
Explanation:\\
\colorbox{white}{\color[gray]{0.5}WA}  ---  WALL\\
\colorbox{green}{\color[gray]{0.5}FO}  ---  FOREST\\
{\color[rgb]{1,0,0}\textbf{BU}}  ---  BURNED\\
{\color[rgb]{0,0,0}\textbf{CO}}  ---  COAL (doubly burned)\\
\colorbox{cyan}{\#\#}  ---  WATERED at time \#\#\\
Fields can have more than 1 state.
}
\subsubsection{Beispiel 3}
Ein (etwas) größeres Beispiel.\footnote{Diese Eingabe finden Sie auch in der Datei \texttt{3.in}}:
{\small
\lstinputlisting{../Aufgabe_1/3.in}
}
Mein Programm produziert folgende Ausgabe\footnote{Diese Ausgabe finden Sie auch in der Datei \texttt{3.out.tex2};} Dabei hat die Berechnung wenige Sekunden in Anspruch genommen, sofern nicht die Ausgabe der ASCII-Escape-Sequenzen gefordert wird. Dies erhöhte die Laufzeit auf ca. 30s.:\\
{\ttfamily \small
\\
\colorbox{green}{\color[gray]{0.75}FO}%
\colorbox{green}{\color[gray]{0.75}FO}%
\colorbox{green}{\color[gray]{0.75}FO}%
\colorbox{green}{\color[gray]{0.75}FO}%
\colorbox{green}{\color[gray]{0.75}FO}%
\colorbox{green}{\color[gray]{0.75}FO}%
\colorbox{green}{\color[gray]{0.75}FO}%
\colorbox{green}{\color[gray]{0.75}FO}%
\colorbox{green}{\color[gray]{0.75}FO}%
\colorbox{green}{\color[gray]{0.75}FO}%
\colorbox{green}{\color[gray]{0.75}FO}%
\colorbox{green}{\color[gray]{0.75}FO}%
\colorbox{green}{\color[gray]{0.75}FO}%
\colorbox{green}{\color[gray]{0.75}FO}%
\colorbox{green}{\color[gray]{0.75}FO}%
\colorbox{green}{\color[gray]{0.75}FO}%
\colorbox{green}{\color[gray]{0.75}FO}%
\colorbox{green}{\color[gray]{0.75}FO}%
\colorbox{green}{\color[gray]{0.75}FO}%
\colorbox{green}{\color[gray]{0.75}FO}%
\colorbox{green}{\color[gray]{0.75}FO}%
\colorbox{green}{\color[gray]{0.75}FO}%
\colorbox{green}{\color[gray]{0.75}FO}%
\colorbox{green}{\color[gray]{0.75}FO}%
\colorbox{green}{\color[gray]{0.75}FO}%
\colorbox{green}{\color[gray]{0.75}FO}%
\colorbox{green}{\color[gray]{0.75}FO}%
\colorbox{green}{\color[gray]{0.75}FO}%
\colorbox{green}{\color[gray]{0.75}FO}%
\colorbox{green}{\color[gray]{0.75}FO}%
\colorbox{green}{\color[gray]{0.75}FO}%
\colorbox{green}{\color[gray]{0.75}FO}%
\colorbox{green}{\color[gray]{0.75}FO}%
\colorbox{green}{\color[gray]{0.75}FO}%
\colorbox{green}{\color[gray]{0.75}FO}%
\colorbox{green}{\color[gray]{0.75}FO}%
\colorbox{green}{\color[gray]{0.75}FO}%
\colorbox{green}{\color[gray]{0.75}FO}%
\colorbox{green}{\color[gray]{0.75}FO}%
\colorbox{green}{\color[gray]{0.75}FO}%
\colorbox{green}{\color[gray]{0.75}FO}%
\colorbox{green}{\color[gray]{0.75}FO}%
\colorbox{green}{\color[gray]{0.75}FO}%
\colorbox{green}{\color[gray]{0.75}FO}%
\colorbox{green}{\color[gray]{0.75}FO}%
\colorbox{green}{\color[gray]{0.75}FO}%
\colorbox{green}{\color[gray]{0.75}FO}%
\colorbox{green}{\color[gray]{0.75}FO}%
\colorbox{green}{\color[gray]{0.75}FO}%
\colorbox{green}{\color[gray]{0.75}FO}%
\colorbox{green}{\color[gray]{0.75}FO}%
\colorbox{green}{\color[gray]{0.75}FO}%
\colorbox{green}{\color[gray]{0.75}FO}%
\colorbox{green}{\color[gray]{0.75}FO}%
\colorbox{green}{\color[gray]{0.75}FO}%
\colorbox{green}{\color[gray]{0.75}FO}%
\colorbox{green}{\color[gray]{0.75}FO}%
\colorbox{green}{\color[gray]{0.75}FO}%
\colorbox{green}{\color[gray]{0.75}FO}%
\colorbox{green}{\color[gray]{0.75}FO}%
\colorbox{green}{\color[gray]{0.75}FO}%
\colorbox{green}{\color[gray]{0.75}FO}%
\colorbox{green}{\color[gray]{0.75}FO}%
\colorbox{green}{\color[gray]{0.75}FO}%
\colorbox{green}{\color[gray]{0.75}FO}%
\colorbox{green}{\color[gray]{0.75}FO}%
\colorbox{green}{\color[gray]{0.75}FO}%
\colorbox{green}{\color[gray]{0.75}FO}%
\colorbox{green}{\color[gray]{0.75}FO}%
\colorbox{green}{\color[gray]{0.75}FO}%
\colorbox{green}{\color[gray]{0.75}FO}%
\colorbox{green}{\color[gray]{0.75}FO}%
\colorbox{green}{\color[gray]{0.75}FO}%
\colorbox{green}{\color[gray]{0.75}FO}%
\colorbox{green}{\color[gray]{0.75}FO}%
\colorbox{green}{\color[gray]{0.75}FO}%
\colorbox{green}{\color[gray]{0.75}FO}%
\colorbox{green}{\color[gray]{0.75}FO}%
\colorbox{green}{\color[gray]{0.75}FO}%
\colorbox{green}{\color[gray]{0.75}FO}%
\colorbox{green}{\color[gray]{0.75}FO}%
\colorbox{green}{\color[gray]{0.75}FO}%
\colorbox{green}{\color[gray]{0.75}FO}%
\colorbox{green}{\color[gray]{0.75}FO}%
\colorbox{green}{\color[gray]{0.75}FO}%
\colorbox{green}{\color[gray]{0.75}FO}%
\colorbox{green}{\color[gray]{0.75}FO}%
\colorbox{green}{\color[gray]{0.75}FO}%
\colorbox{green}{\color[gray]{0.75}FO}%
\colorbox{green}{\color[gray]{0.75}FO}%
\colorbox{green}{\color[gray]{0.75}FO}%
\colorbox{green}{\color[gray]{0.75}FO}%
\colorbox{green}{\color[gray]{0.75}FO}%
\colorbox{green}{\color[gray]{0.75}FO}%
\colorbox{green}{\color[gray]{0.75}FO}%
\colorbox{green}{\color[gray]{0.75}FO}%
\colorbox{green}{\color[gray]{0.75}FO}%
\colorbox{green}{\color[gray]{0.75}FO}%
\colorbox{green}{\color[gray]{0.75}FO}%
\colorbox{green}{\color[gray]{0.75}FO}%
\\
\colorbox{green}{\color[gray]{0.75}FO}%
\colorbox{green}{\color[gray]{0.75}FO}%
\colorbox{green}{\color[gray]{0.75}FO}%
\colorbox{green}{\color[gray]{0.75}FO}%
\colorbox{green}{\color[gray]{0.75}FO}%
\colorbox{green}{\color[gray]{0.75}FO}%
\colorbox{green}{\color[gray]{0.75}FO}%
\colorbox{green}{\color[gray]{0.75}FO}%
\colorbox{green}{\color[gray]{0.75}FO}%
\colorbox{green}{\color[gray]{0.75}FO}%
\colorbox{green}{\color[gray]{0.75}FO}%
\colorbox{green}{\color[gray]{0.75}FO}%
\colorbox{green}{\color[gray]{0.75}FO}%
\colorbox{green}{\color[gray]{0.75}FO}%
\colorbox{green}{\color[gray]{0.75}FO}%
\colorbox{green}{\color[gray]{0.75}FO}%
\colorbox{green}{\color[gray]{0.75}FO}%
\colorbox{green}{\color[gray]{0.75}FO}%
\colorbox{green}{\color[gray]{0.75}FO}%
\colorbox{green}{\color[gray]{0.75}FO}%
\colorbox{green}{\color[gray]{0.75}FO}%
\colorbox{green}{\color[gray]{0.75}FO}%
\colorbox{green}{\color[gray]{0.75}FO}%
\colorbox{green}{\color[gray]{0.75}FO}%
\colorbox{green}{\color[gray]{0.75}FO}%
\colorbox{green}{\color[gray]{0.75}FO}%
\colorbox{green}{\color[gray]{0.75}FO}%
\colorbox{green}{\color[gray]{0.75}FO}%
\colorbox{green}{\color[gray]{0.75}FO}%
\colorbox{green}{\color[gray]{0.75}FO}%
\colorbox{green}{\color[gray]{0.75}FO}%
\colorbox{green}{\color[gray]{0.75}FO}%
\colorbox{green}{\color[gray]{0.75}FO}%
\colorbox{green}{\color[gray]{0.75}FO}%
\colorbox{green}{\color[gray]{0.75}FO}%
\colorbox{green}{\color[gray]{0.75}FO}%
\colorbox{green}{\color[gray]{0.75}FO}%
\colorbox{green}{\color[gray]{0.75}FO}%
\colorbox{green}{\color[gray]{0.75}FO}%
\colorbox{green}{\color[gray]{0.75}FO}%
\colorbox{green}{\color[gray]{0.75}FO}%
\colorbox{green}{\color[gray]{0.75}FO}%
\colorbox{green}{\color[gray]{0.75}FO}%
\colorbox{green}{\color[gray]{0.75}FO}%
\colorbox{green}{\color[gray]{0.75}FO}%
\colorbox{green}{\color[gray]{0.75}FO}%
\colorbox{green}{\color[gray]{0.75}FO}%
\colorbox{green}{\color[gray]{0.75}FO}%
\colorbox{green}{\color[gray]{0.75}FO}%
\colorbox{green}{\color[gray]{0.75}FO}%
\colorbox{green}{\color[gray]{0.75}FO}%
\colorbox{green}{\color[gray]{0.75}FO}%
\colorbox{green}{\color[gray]{0.75}FO}%
\colorbox{green}{\color[gray]{0.75}FO}%
\colorbox{green}{\color[gray]{0.75}FO}%
\colorbox{green}{\color[gray]{0.75}FO}%
\colorbox{green}{\color[gray]{0.75}FO}%
\colorbox{green}{\color[gray]{0.75}FO}%
\colorbox{green}{\color[gray]{0.75}FO}%
\colorbox{green}{\color[gray]{0.75}FO}%
\colorbox{green}{\color[gray]{0.75}FO}%
\colorbox{green}{\color[gray]{0.75}FO}%
\colorbox{green}{\color[gray]{0.75}FO}%
\colorbox{green}{\color[gray]{0.75}FO}%
\colorbox{green}{\color[gray]{0.75}FO}%
\colorbox{green}{\color[gray]{0.75}FO}%
\colorbox{green}{\color[gray]{0.75}FO}%
\colorbox{green}{\color[gray]{0.75}FO}%
\colorbox{green}{\color[gray]{0.75}FO}%
\colorbox{green}{\color[gray]{0.75}FO}%
\colorbox{green}{\color[gray]{0.75}FO}%
\colorbox{green}{\color[gray]{0.75}FO}%
\colorbox{green}{\color[gray]{0.75}FO}%
\colorbox{green}{\color[gray]{0.75}FO}%
\colorbox{green}{\color[gray]{0.75}FO}%
\colorbox{green}{\color[gray]{0.75}FO}%
\colorbox{green}{\color[gray]{0.75}FO}%
\colorbox{green}{\color[gray]{0.75}FO}%
\colorbox{green}{\color[gray]{0.75}FO}%
\colorbox{green}{\color[gray]{0.75}FO}%
\colorbox{green}{\color[gray]{0.75}FO}%
\colorbox{green}{\color[gray]{0.75}FO}%
\colorbox{green}{\color[gray]{0.75}FO}%
\colorbox{green}{\color[gray]{0.75}FO}%
\colorbox{green}{\color[gray]{0.75}FO}%
\colorbox{green}{\color[gray]{0.75}FO}%
\colorbox{green}{\color[gray]{0.75}FO}%
\colorbox{green}{\color[gray]{0.75}FO}%
\colorbox{green}{\color[gray]{0.75}FO}%
\colorbox{green}{\color[gray]{0.75}FO}%
\colorbox{green}{\color[gray]{0.75}FO}%
\colorbox{green}{\color[gray]{0.75}FO}%
\colorbox{green}{\color[gray]{0.75}FO}%
\colorbox{green}{\color[gray]{0.75}FO}%
\colorbox{green}{\color[gray]{0.75}FO}%
\colorbox{green}{\color[gray]{0.75}FO}%
\colorbox{green}{\color[gray]{0.75}FO}%
\colorbox{green}{\color[gray]{0.75}FO}%
\colorbox{green}{\color[gray]{0.75}FO}%
\colorbox{green}{\color[gray]{0.75}FO}%
\\
\colorbox{green}{\color[gray]{0.75}FO}%
\colorbox{green}{\color[gray]{0.75}FO}%
\colorbox{green}{\color[gray]{0.75}FO}%
\colorbox{green}{\color[gray]{0.75}FO}%
\colorbox{green}{\color[gray]{0.75}FO}%
\colorbox{green}{\color[gray]{0.75}FO}%
\colorbox{green}{\color[gray]{0.75}FO}%
\colorbox{green}{\color[gray]{0.75}FO}%
\colorbox{green}{\color[gray]{0.75}FO}%
\colorbox{green}{\color[gray]{0.75}FO}%
\colorbox{green}{\color[gray]{0.75}FO}%
\colorbox{green}{\color[gray]{0.75}FO}%
\colorbox{green}{\color[gray]{0.75}FO}%
\colorbox{green}{\color[gray]{0.75}FO}%
\colorbox{green}{\color[gray]{0.75}FO}%
\colorbox{green}{\color[gray]{0.75}FO}%
\colorbox{green}{\color[gray]{0.75}FO}%
\colorbox{green}{\color[gray]{0.75}FO}%
\colorbox{green}{\color[gray]{0.75}FO}%
\colorbox{green}{\color[gray]{0.75}FO}%
\colorbox{green}{\color[gray]{0.75}FO}%
\colorbox{green}{\color[gray]{0.75}FO}%
\colorbox{green}{\color[gray]{0.75}FO}%
\colorbox{green}{\color[gray]{0.75}FO}%
\colorbox{green}{\color[gray]{0.75}FO}%
\colorbox{green}{\color[gray]{0.75}FO}%
\colorbox{green}{\color[gray]{0.75}FO}%
\colorbox{green}{\color[gray]{0.75}FO}%
\colorbox{green}{\color[gray]{0.75}FO}%
\colorbox{green}{\color[gray]{0.75}FO}%
\colorbox{green}{\color[gray]{0.75}FO}%
\colorbox{green}{\color[gray]{0.75}FO}%
\colorbox{green}{\color[gray]{0.75}FO}%
\colorbox{green}{\color[gray]{0.75}FO}%
\colorbox{green}{\color[gray]{0.75}FO}%
\colorbox{green}{\color[gray]{0.75}FO}%
\colorbox{green}{\color[gray]{0.75}FO}%
\colorbox{green}{\color[gray]{0.75}FO}%
\colorbox{green}{\color[gray]{0.75}FO}%
\colorbox{green}{\color[gray]{0.75}FO}%
\colorbox{green}{\color[gray]{0.75}FO}%
\colorbox{green}{\color[gray]{0.75}FO}%
\colorbox{green}{\color[gray]{0.75}FO}%
\colorbox{green}{\color[gray]{0.75}FO}%
\colorbox{green}{\color[gray]{0.75}FO}%
\colorbox{green}{\color[gray]{0.75}FO}%
\colorbox{green}{\color[gray]{0.75}FO}%
\colorbox{green}{\color[gray]{0.75}FO}%
\colorbox{green}{\color[gray]{0.75}FO}%
\colorbox{green}{\color[gray]{0.75}FO}%
\colorbox{green}{\color[gray]{0.75}FO}%
\colorbox{green}{\color[gray]{0.75}FO}%
\colorbox{green}{\color[gray]{0.75}FO}%
\colorbox{green}{\color[gray]{0.75}FO}%
\colorbox{green}{\color[gray]{0.75}FO}%
\colorbox{green}{\color[gray]{0.75}FO}%
\colorbox{green}{\color[gray]{0.75}FO}%
\colorbox{green}{\color[gray]{0.75}FO}%
\colorbox{green}{\color[gray]{0.75}FO}%
\colorbox{green}{\color[gray]{0.75}FO}%
\colorbox{green}{\color[gray]{0.75}FO}%
\colorbox{green}{\color[gray]{0.75}FO}%
\colorbox{green}{\color[gray]{0.75}FO}%
\colorbox{green}{\color[gray]{0.75}FO}%
\colorbox{green}{\color[gray]{0.75}FO}%
\colorbox{green}{\color[gray]{0.75}FO}%
\colorbox{green}{\color[gray]{0.75}FO}%
\colorbox{green}{\color[gray]{0.75}FO}%
\colorbox{green}{\color[gray]{0.75}FO}%
\colorbox{green}{\color[gray]{0.75}FO}%
\colorbox{green}{\color[gray]{0.75}FO}%
\colorbox{green}{\color[gray]{0.75}FO}%
\colorbox{green}{\color[gray]{0.75}FO}%
\colorbox{green}{\color[gray]{0.75}FO}%
\colorbox{green}{\color[gray]{0.75}FO}%
\colorbox{green}{\color[gray]{0.75}FO}%
\colorbox{green}{\color[gray]{0.75}FO}%
\colorbox{green}{\color[gray]{0.75}FO}%
\colorbox{green}{\color[gray]{0.75}FO}%
\colorbox{green}{\color[gray]{0.75}FO}%
\colorbox{green}{\color[gray]{0.75}FO}%
\colorbox{green}{\color[gray]{0.75}FO}%
\colorbox{green}{\color[gray]{0.75}FO}%
\colorbox{green}{\color[gray]{0.75}FO}%
\colorbox{green}{\color[gray]{0.75}FO}%
\colorbox{green}{\color[gray]{0.75}FO}%
\colorbox{green}{\color[gray]{0.75}FO}%
\colorbox{green}{\color[gray]{0.75}FO}%
\colorbox{green}{\color[gray]{0.75}FO}%
\colorbox{green}{\color[gray]{0.75}FO}%
\colorbox{green}{\color[gray]{0.75}FO}%
\colorbox{green}{\color[gray]{0.75}FO}%
\colorbox{green}{\color[gray]{0.75}FO}%
\colorbox{green}{\color[gray]{0.75}FO}%
\colorbox{green}{\color[gray]{0.75}FO}%
\colorbox{green}{\color[gray]{0.75}FO}%
\colorbox{green}{\color[gray]{0.75}FO}%
\colorbox{green}{\color[gray]{0.75}FO}%
\colorbox{green}{\color[gray]{0.75}FO}%
\colorbox{green}{\color[gray]{0.75}FO}%
\\
\colorbox{green}{\color[gray]{0.75}FO}%
\colorbox{green}{\color[gray]{0.75}FO}%
\colorbox{green}{\color[gray]{0.75}FO}%
\colorbox{green}{\color[gray]{0.75}FO}%
\colorbox{green}{\color[gray]{0.75}FO}%
\colorbox{green}{\color[gray]{0.75}FO}%
\colorbox{green}{\color[gray]{0.75}FO}%
\colorbox{green}{\color[gray]{0.75}FO}%
\colorbox{green}{\color[gray]{0.75}FO}%
\colorbox{green}{\color[gray]{0.75}FO}%
\colorbox{green}{\color[gray]{0.75}FO}%
\colorbox{green}{\color[gray]{0.75}FO}%
\colorbox{green}{\color[gray]{0.75}FO}%
\colorbox{green}{\color[gray]{0.75}FO}%
\colorbox{green}{\color[gray]{0.75}FO}%
\colorbox{green}{\color[gray]{0.75}FO}%
\colorbox{green}{\color[gray]{0.75}FO}%
\colorbox{green}{\color[gray]{0.75}FO}%
\colorbox{green}{\color[gray]{0.75}FO}%
\colorbox{green}{\color[gray]{0.75}FO}%
\colorbox{green}{\color[gray]{0.75}FO}%
\colorbox{green}{\color[gray]{0.75}FO}%
\colorbox{green}{\color[gray]{0.75}FO}%
\colorbox{green}{\color[gray]{0.75}FO}%
\colorbox{green}{\color[gray]{0.75}FO}%
\colorbox{green}{\color[gray]{0.75}FO}%
\colorbox{green}{\color[gray]{0.75}FO}%
\colorbox{green}{\color[gray]{0.75}FO}%
\colorbox{green}{\color[gray]{0.75}FO}%
\colorbox{green}{\color[gray]{0.75}FO}%
\colorbox{green}{\color[gray]{0.75}FO}%
\colorbox{green}{\color[gray]{0.75}FO}%
\colorbox{green}{\color[gray]{0.75}FO}%
\colorbox{green}{\color[gray]{0.75}FO}%
\colorbox{green}{\color[gray]{0.75}FO}%
\colorbox{green}{\color[gray]{0.75}FO}%
\colorbox{green}{\color[gray]{0.75}FO}%
\colorbox{green}{\color[gray]{0.75}FO}%
\colorbox{green}{\color[gray]{0.75}FO}%
\colorbox{green}{\color[gray]{0.75}FO}%
\colorbox{green}{\color[gray]{0.75}FO}%
\colorbox{green}{\color[gray]{0.75}FO}%
\colorbox{green}{\color[gray]{0.75}FO}%
\colorbox{green}{\color[gray]{0.75}FO}%
\colorbox{green}{\color[gray]{0.75}FO}%
\colorbox{green}{\color[gray]{0.75}FO}%
\colorbox{green}{\color[gray]{0.75}FO}%
\colorbox{green}{\color[gray]{0.75}FO}%
\colorbox{green}{\color[gray]{0.75}FO}%
\colorbox{green}{\color[gray]{0.75}FO}%
\colorbox{green}{\color[gray]{0.75}FO}%
\colorbox{green}{\color[gray]{0.75}FO}%
\colorbox{green}{\color[gray]{0.75}FO}%
\colorbox{green}{\color[gray]{0.75}FO}%
\colorbox{green}{\color[gray]{0.75}FO}%
\colorbox{green}{\color[gray]{0.75}FO}%
\colorbox{green}{\color[gray]{0.75}FO}%
\colorbox{green}{\color[gray]{0.75}FO}%
\colorbox{green}{\color[gray]{0.75}FO}%
\colorbox{green}{\color[gray]{0.75}FO}%
\colorbox{green}{\color[gray]{0.75}FO}%
\colorbox{green}{\color[gray]{0.75}FO}%
\colorbox{green}{\color[gray]{0.75}FO}%
\colorbox{green}{\color[gray]{0.75}FO}%
\colorbox{green}{\color[gray]{0.75}FO}%
\colorbox{green}{\color[gray]{0.75}FO}%
\colorbox{green}{\color[gray]{0.75}FO}%
\colorbox{green}{\color[gray]{0.75}FO}%
\colorbox{green}{\color[gray]{0.75}FO}%
\colorbox{green}{\color[gray]{0.75}FO}%
\colorbox{green}{\color[gray]{0.75}FO}%
\colorbox{green}{\color[gray]{0.75}FO}%
\colorbox{green}{\color[gray]{0.75}FO}%
\colorbox{green}{\color[gray]{0.75}FO}%
\colorbox{green}{\color[gray]{0.75}FO}%
\colorbox{green}{\color[gray]{0.75}FO}%
\colorbox{green}{\color[gray]{0.75}FO}%
\colorbox{green}{\color[gray]{0.75}FO}%
\colorbox{green}{\color[gray]{0.75}FO}%
\colorbox{green}{\color[gray]{0.75}FO}%
\colorbox{green}{\color[gray]{0.75}FO}%
\colorbox{green}{\color[gray]{0.75}FO}%
\colorbox{green}{\color[gray]{0.75}FO}%
\colorbox{green}{\color[gray]{0.75}FO}%
\colorbox{green}{\color[gray]{0.75}FO}%
\colorbox{green}{\color[gray]{0.75}FO}%
\colorbox{green}{\color[gray]{0.75}FO}%
\colorbox{green}{\color[gray]{0.75}FO}%
\colorbox{green}{\color[gray]{0.75}FO}%
\colorbox{green}{\color[gray]{0.75}FO}%
\colorbox{green}{\color[gray]{0.75}FO}%
\colorbox{green}{\color[gray]{0.75}FO}%
\colorbox{green}{\color[gray]{0.75}FO}%
\colorbox{green}{\color[gray]{0.75}FO}%
\colorbox{green}{\color[gray]{0.75}FO}%
\colorbox{green}{\color[gray]{0.75}FO}%
\colorbox{green}{\color[gray]{0.75}FO}%
\colorbox{green}{\color[gray]{0.75}FO}%
\colorbox{green}{\color[gray]{0.75}FO}%
\colorbox{green}{\color[gray]{0.75}FO}%
\\
\colorbox{green}{\color[gray]{0.75}FO}%
\colorbox{green}{\color[gray]{0.75}FO}%
\colorbox{green}{\color[gray]{0.75}FO}%
\colorbox{green}{\color[gray]{0.75}FO}%
\colorbox{green}{\color[gray]{0.75}FO}%
\colorbox{green}{\color[gray]{0.75}FO}%
\colorbox{green}{\color[gray]{0.75}FO}%
\colorbox{green}{\color[gray]{0.75}FO}%
\colorbox{green}{\color[gray]{0.75}FO}%
\colorbox{green}{\color[gray]{0.75}FO}%
\colorbox{green}{\color[gray]{0.75}FO}%
\colorbox{green}{\color[gray]{0.75}FO}%
\colorbox{green}{\color[gray]{0.75}FO}%
\colorbox{green}{\color[gray]{0.75}FO}%
\colorbox{green}{\color[gray]{0.75}FO}%
\colorbox{green}{\color[gray]{0.75}FO}%
\colorbox{green}{\color[gray]{0.75}FO}%
\colorbox{green}{\color[gray]{0.75}FO}%
\colorbox{green}{\color[gray]{0.75}FO}%
\colorbox{green}{\color[gray]{0.75}FO}%
\colorbox{green}{\color[gray]{0.75}FO}%
\colorbox{green}{\color[gray]{0.75}FO}%
\colorbox{green}{\color[gray]{0.75}FO}%
\colorbox{green}{\color[gray]{0.75}FO}%
\colorbox{green}{\color[gray]{0.75}FO}%
\colorbox{green}{\color[gray]{0.75}FO}%
\colorbox{green}{\color[gray]{0.75}FO}%
\colorbox{green}{\color[gray]{0.75}FO}%
\colorbox{green}{\color[gray]{0.75}FO}%
\colorbox{green}{\color[gray]{0.75}FO}%
\colorbox{green}{\color[gray]{0.75}FO}%
\colorbox{green}{\color[gray]{0.75}FO}%
\colorbox{green}{\color[gray]{0.75}FO}%
\colorbox{green}{\color[gray]{0.75}FO}%
\colorbox{green}{\color[gray]{0.75}FO}%
\colorbox{green}{\color[gray]{0.75}FO}%
\colorbox{green}{\color[gray]{0.75}FO}%
\colorbox{green}{\color[gray]{0.75}FO}%
\colorbox{green}{\color[gray]{0.75}FO}%
\colorbox{green}{\color[gray]{0.75}FO}%
\colorbox{green}{\color[gray]{0.75}FO}%
\colorbox{green}{\color[gray]{0.75}FO}%
\colorbox{green}{\color[gray]{0.75}FO}%
\colorbox{green}{\color[gray]{0.75}FO}%
\colorbox{green}{\color[gray]{0.75}FO}%
\colorbox{green}{\color[gray]{0.75}FO}%
\colorbox{green}{\color[gray]{0.75}FO}%
\colorbox{green}{\color[gray]{0.75}FO}%
\colorbox{green}{\color[gray]{0.75}FO}%
\colorbox{green}{\color[gray]{0.75}FO}%
\colorbox{green}{\color[gray]{0.75}FO}%
\colorbox{green}{\color[gray]{0.75}FO}%
\colorbox{green}{\color[gray]{0.75}FO}%
\colorbox{green}{\color[gray]{0.75}FO}%
\colorbox{green}{\color[gray]{0.75}FO}%
\colorbox{green}{\color[gray]{0.75}FO}%
\colorbox{green}{\color[gray]{0.75}FO}%
\colorbox{green}{\color[gray]{0.75}FO}%
\colorbox{green}{\color[gray]{0.75}FO}%
\colorbox{green}{\color[gray]{0.75}FO}%
\colorbox{green}{\color[gray]{0.75}FO}%
\colorbox{green}{\color[gray]{0.75}FO}%
\colorbox{green}{\color[gray]{0.75}FO}%
\colorbox{green}{\color[gray]{0.75}FO}%
\colorbox{green}{\color[gray]{0.75}FO}%
\colorbox{green}{\color[gray]{0.75}FO}%
\colorbox{green}{\color[gray]{0.75}FO}%
\colorbox{green}{\color[gray]{0.75}FO}%
\colorbox{green}{\color[gray]{0.75}FO}%
\colorbox{green}{\color[gray]{0.75}FO}%
\colorbox{green}{\color[gray]{0.75}FO}%
\colorbox{green}{\color[gray]{0.75}FO}%
\colorbox{green}{\color[gray]{0.75}FO}%
\colorbox{green}{\color[gray]{0.75}FO}%
\colorbox{green}{\color[gray]{0.75}FO}%
\colorbox{green}{\color[gray]{0.75}FO}%
\colorbox{green}{\color[gray]{0.75}FO}%
\colorbox{green}{\color[gray]{0.75}FO}%
\colorbox{green}{\color[gray]{0.75}FO}%
\colorbox{green}{\color[gray]{0.75}FO}%
\colorbox{green}{\color[gray]{0.75}FO}%
\colorbox{green}{\color[gray]{0.75}FO}%
\colorbox{green}{\color[gray]{0.75}FO}%
\colorbox{green}{\color[gray]{0.75}FO}%
\colorbox{green}{\color[gray]{0.75}FO}%
\colorbox{green}{\color[gray]{0.75}FO}%
\colorbox{green}{\color[gray]{0.75}FO}%
\colorbox{green}{\color[gray]{0.75}FO}%
\colorbox{green}{\color[gray]{0.75}FO}%
\colorbox{green}{\color[gray]{0.75}FO}%
\colorbox{green}{\color[gray]{0.75}FO}%
\colorbox{green}{\color[gray]{0.75}FO}%
\colorbox{green}{\color[gray]{0.75}FO}%
\colorbox{green}{\color[gray]{0.75}FO}%
\colorbox{green}{\color[gray]{0.75}FO}%
\colorbox{green}{\color[gray]{0.75}FO}%
\colorbox{green}{\color[gray]{0.75}FO}%
\colorbox{green}{\color[gray]{0.75}FO}%
\colorbox{green}{\color[gray]{0.75}FO}%
\colorbox{green}{\color[gray]{0.75}FO}%
\\
\colorbox{green}{\color[gray]{0.75}FO}%
\colorbox{green}{\color[gray]{0.75}FO}%
\colorbox{green}{\color[gray]{0.75}FO}%
\colorbox{green}{\color[gray]{0.75}FO}%
\colorbox{green}{\color[gray]{0.75}FO}%
\colorbox{green}{\color[gray]{0.75}FO}%
\colorbox{green}{\color[gray]{0.75}FO}%
\colorbox{green}{\color[gray]{0.75}FO}%
\colorbox{green}{\color[gray]{0.75}FO}%
\colorbox{green}{\color[gray]{0.75}FO}%
\colorbox{green}{\color[gray]{0.75}FO}%
\colorbox{green}{\color[gray]{0.75}FO}%
\colorbox{green}{\color[gray]{0.75}FO}%
\colorbox{green}{\color[gray]{0.75}FO}%
\colorbox{green}{\color[gray]{0.75}FO}%
\colorbox{green}{\color[gray]{0.75}FO}%
\colorbox{green}{\color[gray]{0.75}FO}%
\colorbox{green}{\color[gray]{0.75}FO}%
\colorbox{green}{\color[gray]{0.75}FO}%
\colorbox{green}{\color[gray]{0.75}FO}%
\colorbox{green}{\color[gray]{0.75}FO}%
\colorbox{green}{\color[gray]{0.75}FO}%
\colorbox{green}{\color[gray]{0.75}FO}%
\colorbox{green}{\color[gray]{0.75}FO}%
\colorbox{green}{\color[gray]{0.75}FO}%
\colorbox{green}{\color[gray]{0.75}FO}%
\colorbox{green}{\color[gray]{0.75}FO}%
\colorbox{green}{\color[gray]{0.75}FO}%
\colorbox{green}{\color[gray]{0.75}FO}%
\colorbox{green}{\color[gray]{0.75}FO}%
\colorbox{green}{\color[gray]{0.75}FO}%
\colorbox{green}{\color[gray]{0.75}FO}%
\colorbox{green}{\color[gray]{0.75}FO}%
\colorbox{green}{\color[gray]{0.75}FO}%
\colorbox{green}{\color[gray]{0.75}FO}%
\colorbox{green}{\color[gray]{0.75}FO}%
\colorbox{green}{\color[gray]{0.75}FO}%
\colorbox{green}{\color[gray]{0.75}FO}%
\colorbox{green}{\color[gray]{0.75}FO}%
\colorbox{green}{\color[gray]{0.75}FO}%
\colorbox{green}{\color[gray]{0.75}FO}%
\colorbox{green}{\color[gray]{0.75}FO}%
\colorbox{green}{\color[gray]{0.75}FO}%
\colorbox{green}{\color[gray]{0.75}FO}%
\colorbox{green}{\color[gray]{0.75}FO}%
\colorbox{green}{\color[gray]{0.75}FO}%
\colorbox{green}{\color[gray]{0.75}FO}%
\colorbox{green}{\color[gray]{0.75}FO}%
\colorbox{green}{\color[gray]{0.75}FO}%
\colorbox{green}{\color[gray]{0.75}FO}%
\colorbox{green}{\color[gray]{0.75}FO}%
\colorbox{green}{\color[gray]{0.75}FO}%
\colorbox{green}{\color[gray]{0.75}FO}%
\colorbox{green}{\color[gray]{0.75}FO}%
\colorbox{green}{\color[gray]{0.75}FO}%
\colorbox{green}{\color[gray]{0.75}FO}%
\colorbox{green}{\color[gray]{0.75}FO}%
\colorbox{green}{\color[gray]{0.75}FO}%
\colorbox{green}{\color[gray]{0.75}FO}%
\colorbox{green}{\color[gray]{0.75}FO}%
\colorbox{green}{\color[gray]{0.75}FO}%
\colorbox{green}{\color[gray]{0.75}FO}%
\colorbox{green}{\color[gray]{0.75}FO}%
\colorbox{green}{\color[gray]{0.75}FO}%
\colorbox{green}{\color[gray]{0.75}FO}%
\colorbox{green}{\color[gray]{0.75}FO}%
\colorbox{green}{\color[gray]{0.75}FO}%
\colorbox{green}{\color[gray]{0.75}FO}%
\colorbox{green}{\color[gray]{0.75}FO}%
\colorbox{green}{\color[gray]{0.75}FO}%
\colorbox{green}{\color[gray]{0.75}FO}%
\colorbox{green}{\color[gray]{0.75}FO}%
\colorbox{green}{\color[gray]{0.75}FO}%
\colorbox{green}{\color[gray]{0.75}FO}%
\colorbox{green}{\color[gray]{0.75}FO}%
\colorbox{green}{\color[gray]{0.75}FO}%
\colorbox{green}{\color[gray]{0.75}FO}%
\colorbox{green}{\color[gray]{0.75}FO}%
\colorbox{green}{\color[gray]{0.75}FO}%
\colorbox{green}{\color[gray]{0.75}FO}%
\colorbox{green}{\color[gray]{0.75}FO}%
\colorbox{green}{\color[gray]{0.75}FO}%
\colorbox{green}{\color[gray]{0.75}FO}%
\colorbox{green}{\color[gray]{0.75}FO}%
\colorbox{green}{\color[gray]{0.75}FO}%
\colorbox{green}{\color[gray]{0.75}FO}%
\colorbox{green}{\color[gray]{0.75}FO}%
\colorbox{green}{\color[gray]{0.75}FO}%
\colorbox{green}{\color[gray]{0.75}FO}%
\colorbox{green}{\color[gray]{0.75}FO}%
\colorbox{green}{\color[gray]{0.75}FO}%
\colorbox{green}{\color[gray]{0.75}FO}%
\colorbox{green}{\color[gray]{0.75}FO}%
\colorbox{green}{\color[gray]{0.75}FO}%
\colorbox{green}{\color[gray]{0.75}FO}%
\colorbox{green}{\color[gray]{0.75}FO}%
\colorbox{green}{\color[gray]{0.75}FO}%
\colorbox{green}{\color[gray]{0.75}FO}%
\colorbox{green}{\color[gray]{0.75}FO}%
\colorbox{green}{\color[gray]{0.75}FO}%
\\
\colorbox{green}{\color[gray]{0.75}FO}%
\colorbox{green}{\color[gray]{0.75}FO}%
\colorbox{green}{\color[gray]{0.75}FO}%
\colorbox{green}{\color[gray]{0.75}FO}%
\colorbox{green}{\color[gray]{0.75}FO}%
\colorbox{green}{\color[gray]{0.75}FO}%
\colorbox{green}{\color[gray]{0.75}FO}%
\colorbox{green}{\color[gray]{0.75}FO}%
\colorbox{green}{\color[gray]{0.75}FO}%
\colorbox{green}{\color[gray]{0.75}FO}%
\colorbox{green}{\color[gray]{0.75}FO}%
\colorbox{green}{\color[gray]{0.75}FO}%
\colorbox{green}{\color[gray]{0.75}FO}%
\colorbox{green}{\color[gray]{0.75}FO}%
\colorbox{green}{\color[gray]{0.75}FO}%
\colorbox{green}{\color[gray]{0.75}FO}%
\colorbox{green}{\color[gray]{0.75}FO}%
\colorbox{green}{\color[gray]{0.75}FO}%
\colorbox{green}{\color[gray]{0.75}FO}%
\colorbox{green}{\color[gray]{0.75}FO}%
\colorbox{green}{\color[gray]{0.75}FO}%
\colorbox{green}{\color[gray]{0.75}FO}%
\colorbox{green}{\color[gray]{0.75}FO}%
\colorbox{green}{\color[gray]{0.75}FO}%
\colorbox{green}{\color[gray]{0.75}FO}%
\colorbox{green}{\color[gray]{0.75}FO}%
\colorbox{green}{\color[gray]{0.75}FO}%
\colorbox{green}{\color[gray]{0.75}FO}%
\colorbox{green}{\color[gray]{0.75}FO}%
\colorbox{green}{\color[gray]{0.75}FO}%
\colorbox{green}{\color[gray]{0.75}FO}%
\colorbox{green}{\color[gray]{0.75}FO}%
\colorbox{green}{\color[gray]{0.75}FO}%
\colorbox{green}{\color[gray]{0.75}FO}%
\colorbox{green}{\color[gray]{0.75}FO}%
\colorbox{green}{\color[gray]{0.75}FO}%
\colorbox{green}{\color[gray]{0.75}FO}%
\colorbox{green}{\color[gray]{0.75}FO}%
\colorbox{green}{\color[gray]{0.75}FO}%
\colorbox{green}{\color[gray]{0.75}FO}%
\colorbox{green}{\color[gray]{0.75}FO}%
\colorbox{green}{\color[gray]{0.75}FO}%
\colorbox{green}{\color[gray]{0.75}FO}%
\colorbox{green}{\color[gray]{0.75}FO}%
\colorbox{green}{\color[gray]{0.75}FO}%
\colorbox{green}{\color[gray]{0.75}FO}%
\colorbox{green}{\color[gray]{0.75}FO}%
\colorbox{green}{\color[gray]{0.75}FO}%
\colorbox{green}{\color[gray]{0.75}FO}%
\colorbox{green}{\color[gray]{0.75}FO}%
\colorbox{green}{\color[gray]{0.75}FO}%
\colorbox{green}{\color[gray]{0.75}FO}%
\colorbox{green}{\color[gray]{0.75}FO}%
\colorbox{green}{\color[gray]{0.75}FO}%
\colorbox{green}{\color[gray]{0.75}FO}%
\colorbox{green}{\color[gray]{0.75}FO}%
\colorbox{green}{\color[gray]{0.75}FO}%
\colorbox{green}{\color[gray]{0.75}FO}%
\colorbox{green}{\color[gray]{0.75}FO}%
\colorbox{green}{\color[gray]{0.75}FO}%
\colorbox{green}{\color[gray]{0.75}FO}%
\colorbox{green}{\color[gray]{0.75}FO}%
\colorbox{green}{\color[gray]{0.75}FO}%
\colorbox{green}{\color[gray]{0.75}FO}%
\colorbox{green}{\color[gray]{0.75}FO}%
\colorbox{green}{\color[gray]{0.75}FO}%
\colorbox{green}{\color[gray]{0.75}FO}%
\colorbox{green}{\color[gray]{0.75}FO}%
\colorbox{green}{\color[gray]{0.75}FO}%
\colorbox{green}{\color[gray]{0.75}FO}%
\colorbox{green}{\color[gray]{0.75}FO}%
\colorbox{green}{\color[gray]{0.75}FO}%
\colorbox{green}{\color[gray]{0.75}FO}%
\colorbox{green}{\color[gray]{0.75}FO}%
\colorbox{green}{\color[gray]{0.75}FO}%
\colorbox{green}{\color[gray]{0.75}FO}%
\colorbox{green}{\color[gray]{0.75}FO}%
\colorbox{green}{\color[gray]{0.75}FO}%
\colorbox{green}{\color[gray]{0.75}FO}%
\colorbox{green}{\color[gray]{0.75}FO}%
\colorbox{green}{\color[gray]{0.75}FO}%
\colorbox{green}{\color[gray]{0.75}FO}%
\colorbox{green}{\color[gray]{0.75}FO}%
\colorbox{green}{\color[gray]{0.75}FO}%
\colorbox{green}{\color[gray]{0.75}FO}%
\colorbox{green}{\color[gray]{0.75}FO}%
\colorbox{green}{\color[gray]{0.75}FO}%
\colorbox{green}{\color[gray]{0.75}FO}%
\colorbox{green}{\color[gray]{0.75}FO}%
\colorbox{green}{\color[gray]{0.75}FO}%
\colorbox{green}{\color[gray]{0.75}FO}%
\colorbox{green}{\color[gray]{0.75}FO}%
\colorbox{green}{\color[gray]{0.75}FO}%
\colorbox{green}{\color[gray]{0.75}FO}%
\colorbox{green}{\color[gray]{0.75}FO}%
\colorbox{green}{\color[gray]{0.75}FO}%
\colorbox{green}{\color[gray]{0.75}FO}%
\colorbox{green}{\color[gray]{0.75}FO}%
\colorbox{green}{\color[gray]{0.75}FO}%
\colorbox{green}{\color[gray]{0.75}FO}%
\\
\colorbox{green}{\color[gray]{0.75}FO}%
\colorbox{green}{\color[gray]{0.75}FO}%
\colorbox{green}{\color[gray]{0.75}FO}%
\colorbox{green}{\color[gray]{0.75}FO}%
\colorbox{green}{\color[gray]{0.75}FO}%
\colorbox{green}{\color[gray]{0.75}FO}%
\colorbox{green}{\color[gray]{0.75}FO}%
\colorbox{green}{\color[gray]{0.75}FO}%
\colorbox{green}{\color[gray]{0.75}FO}%
\colorbox{green}{\color[gray]{0.75}FO}%
\colorbox{green}{\color[gray]{0.75}FO}%
\colorbox{green}{\color[gray]{0.75}FO}%
\colorbox{green}{\color[gray]{0.75}FO}%
\colorbox{green}{\color[gray]{0.75}FO}%
\colorbox{green}{\color[gray]{0.75}FO}%
\colorbox{green}{\color[gray]{0.75}FO}%
\colorbox{green}{\color[gray]{0.75}FO}%
\colorbox{green}{\color[gray]{0.75}FO}%
\colorbox{green}{\color[gray]{0.75}FO}%
\colorbox{green}{\color[gray]{0.75}FO}%
\colorbox{green}{\color[gray]{0.75}FO}%
\colorbox{green}{\color[gray]{0.75}FO}%
\colorbox{green}{\color[gray]{0.75}FO}%
\colorbox{green}{\color[gray]{0.75}FO}%
\colorbox{green}{\color[gray]{0.75}FO}%
\colorbox{green}{\color[gray]{0.75}FO}%
\colorbox{green}{\color[gray]{0.75}FO}%
\colorbox{green}{\color[gray]{0.75}FO}%
\colorbox{green}{\color[gray]{0.75}FO}%
\colorbox{green}{\color[gray]{0.75}FO}%
\colorbox{green}{\color[gray]{0.75}FO}%
\colorbox{green}{\color[gray]{0.75}FO}%
\colorbox{green}{\color[gray]{0.75}FO}%
\colorbox{green}{\color[gray]{0.75}FO}%
\colorbox{green}{\color[gray]{0.75}FO}%
\colorbox{green}{\color[gray]{0.75}FO}%
\colorbox{green}{\color[gray]{0.75}FO}%
\colorbox{green}{\color[gray]{0.75}FO}%
\colorbox{green}{\color[gray]{0.75}FO}%
\colorbox{green}{\color[gray]{0.75}FO}%
\colorbox{green}{\color[gray]{0.75}FO}%
\colorbox{green}{\color[gray]{0.75}FO}%
\colorbox{green}{\color[gray]{0.75}FO}%
\colorbox{green}{\color[gray]{0.75}FO}%
\colorbox{green}{\color[gray]{0.75}FO}%
\colorbox{green}{\color[gray]{0.75}FO}%
\colorbox{green}{\color[gray]{0.75}FO}%
\colorbox{green}{\color[gray]{0.75}FO}%
\colorbox{green}{\color[gray]{0.75}FO}%
\colorbox{green}{\color[gray]{0.75}FO}%
\colorbox{green}{\color[gray]{0.75}FO}%
\colorbox{green}{\color[gray]{0.75}FO}%
\colorbox{green}{\color[gray]{0.75}FO}%
\colorbox{green}{\color[gray]{0.75}FO}%
\colorbox{green}{\color[gray]{0.75}FO}%
\colorbox{green}{\color[gray]{0.75}FO}%
\colorbox{green}{\color[gray]{0.75}FO}%
\colorbox{green}{\color[gray]{0.75}FO}%
\colorbox{green}{\color[gray]{0.75}FO}%
\colorbox{green}{\color[gray]{0.75}FO}%
\colorbox{green}{\color[gray]{0.75}FO}%
\colorbox{green}{\color[gray]{0.75}FO}%
\colorbox{green}{\color[gray]{0.75}FO}%
\colorbox{green}{\color[gray]{0.75}FO}%
\colorbox{green}{\color[gray]{0.75}FO}%
\colorbox{green}{\color[gray]{0.75}FO}%
\colorbox{green}{\color[gray]{0.75}FO}%
\colorbox{green}{\color[gray]{0.75}FO}%
\colorbox{green}{\color[gray]{0.75}FO}%
\colorbox{green}{\color[gray]{0.75}FO}%
\colorbox{green}{\color[gray]{0.75}FO}%
\colorbox{green}{\color[gray]{0.75}FO}%
\colorbox{green}{\color[gray]{0.75}FO}%
\colorbox{green}{\color[gray]{0.75}FO}%
\colorbox{green}{\color[gray]{0.75}FO}%
\colorbox{green}{\color[gray]{0.75}FO}%
\colorbox{green}{\color[gray]{0.75}FO}%
\colorbox{green}{\color[gray]{0.75}FO}%
\colorbox{green}{\color[gray]{0.75}FO}%
\colorbox{green}{\color[gray]{0.75}FO}%
\colorbox{green}{\color[gray]{0.75}FO}%
\colorbox{green}{\color[gray]{0.75}FO}%
\colorbox{green}{\color[gray]{0.75}FO}%
\colorbox{green}{\color[gray]{0.75}FO}%
\colorbox{green}{\color[gray]{0.75}FO}%
\colorbox{green}{\color[gray]{0.75}FO}%
\colorbox{green}{\color[gray]{0.75}FO}%
\colorbox{green}{\color[gray]{0.75}FO}%
\colorbox{green}{\color[gray]{0.75}FO}%
\colorbox{green}{\color[gray]{0.75}FO}%
\colorbox{green}{\color[gray]{0.75}FO}%
\colorbox{green}{\color[gray]{0.75}FO}%
\colorbox{green}{\color[gray]{0.75}FO}%
\colorbox{green}{\color[gray]{0.75}FO}%
\colorbox{green}{\color[gray]{0.75}FO}%
\colorbox{green}{\color[gray]{0.75}FO}%
\colorbox{green}{\color[gray]{0.75}FO}%
\colorbox{green}{\color[gray]{0.75}FO}%
\colorbox{green}{\color[gray]{0.75}FO}%
\colorbox{green}{\color[gray]{0.75}FO}%
\\
\colorbox{green}{\color[gray]{0.75}FO}%
\colorbox{green}{\color[gray]{0.75}FO}%
\colorbox{green}{\color[gray]{0.75}FO}%
\colorbox{green}{\color[gray]{0.75}FO}%
\colorbox{green}{\color[gray]{0.75}FO}%
\colorbox{green}{\color[gray]{0.75}FO}%
\colorbox{green}{\color[gray]{0.75}FO}%
\colorbox{green}{\color[gray]{0.75}FO}%
\colorbox{green}{\color[gray]{0.75}FO}%
\colorbox{green}{\color[gray]{0.75}FO}%
\colorbox{green}{\color[gray]{0.75}FO}%
\colorbox{green}{\color[gray]{0.75}FO}%
\colorbox{green}{\color[gray]{0.75}FO}%
\colorbox{green}{\color[gray]{0.75}FO}%
\colorbox{green}{\color[gray]{0.75}FO}%
\colorbox{green}{\color[gray]{0.75}FO}%
\colorbox{green}{\color[gray]{0.75}FO}%
\colorbox{green}{\color[gray]{0.75}FO}%
\colorbox{green}{\color[gray]{0.75}FO}%
\colorbox{green}{\color[gray]{0.75}FO}%
\colorbox{green}{\color[gray]{0.75}FO}%
\colorbox{green}{\color[gray]{0.75}FO}%
\colorbox{green}{\color[gray]{0.75}FO}%
\colorbox{green}{\color[gray]{0.75}FO}%
\colorbox{green}{\color[gray]{0.75}FO}%
\colorbox{green}{\color[gray]{0.75}FO}%
\colorbox{green}{\color[gray]{0.75}FO}%
\colorbox{green}{\color[gray]{0.75}FO}%
\colorbox{green}{\color[gray]{0.75}FO}%
\colorbox{green}{\color[gray]{0.75}FO}%
\colorbox{green}{\color[gray]{0.75}FO}%
\colorbox{green}{\color[gray]{0.75}FO}%
\colorbox{green}{\color[gray]{0.75}FO}%
\colorbox{green}{\color[gray]{0.75}FO}%
\colorbox{green}{\color[gray]{0.75}FO}%
\colorbox{green}{\color[gray]{0.75}FO}%
\colorbox{green}{\color[gray]{0.75}FO}%
\colorbox{green}{\color[gray]{0.75}FO}%
\colorbox{green}{\color[gray]{0.75}FO}%
\colorbox{green}{\color[gray]{0.75}FO}%
\colorbox{green}{\color[gray]{0.75}FO}%
\colorbox{green}{\color[gray]{0.75}FO}%
\colorbox{green}{\color[gray]{0.75}FO}%
\colorbox{green}{\color[gray]{0.75}FO}%
\colorbox{green}{\color[gray]{0.75}FO}%
\colorbox{green}{\color[gray]{0.75}FO}%
\colorbox{green}{\color[gray]{0.75}FO}%
\colorbox{green}{\color[gray]{0.75}FO}%
\colorbox{green}{\color[gray]{0.75}FO}%
\colorbox{green}{\color[gray]{0.75}FO}%
\colorbox{green}{\color[gray]{0.75}FO}%
\colorbox{green}{\color[gray]{0.75}FO}%
\colorbox{green}{\color[gray]{0.75}FO}%
\colorbox{green}{\color[gray]{0.75}FO}%
\colorbox{green}{\color[gray]{0.75}FO}%
\colorbox{green}{\color[gray]{0.75}FO}%
\colorbox{green}{\color[gray]{0.75}FO}%
\colorbox{green}{\color[gray]{0.75}FO}%
\colorbox{green}{\color[gray]{0.75}FO}%
\colorbox{green}{\color[gray]{0.75}FO}%
\colorbox{green}{\color[gray]{0.75}FO}%
\colorbox{green}{\color[gray]{0.75}FO}%
\colorbox{green}{\color[gray]{0.75}FO}%
\colorbox{green}{\color[gray]{0.75}FO}%
\colorbox{green}{\color[gray]{0.75}FO}%
\colorbox{green}{\color[gray]{0.75}FO}%
\colorbox{green}{\color[gray]{0.75}FO}%
\colorbox{green}{\color[gray]{0.75}FO}%
\colorbox{green}{\color[gray]{0.75}FO}%
\colorbox{green}{\color[gray]{0.75}FO}%
\colorbox{green}{\color[gray]{0.75}FO}%
\colorbox{green}{\color[gray]{0.75}FO}%
\colorbox{green}{\color[gray]{0.75}FO}%
\colorbox{green}{\color[gray]{0.75}FO}%
\colorbox{green}{\color[gray]{0.75}FO}%
\colorbox{green}{\color[gray]{0.75}FO}%
\colorbox{green}{\color[gray]{0.75}FO}%
\colorbox{green}{\color[gray]{0.75}FO}%
\colorbox{green}{\color[gray]{0.75}FO}%
\colorbox{green}{\color[gray]{0.75}FO}%
\colorbox{green}{\color[gray]{0.75}FO}%
\colorbox{green}{\color[gray]{0.75}FO}%
\colorbox{green}{\color[gray]{0.75}FO}%
\colorbox{green}{\color[gray]{0.75}FO}%
\colorbox{green}{\color[gray]{0.75}FO}%
\colorbox{green}{\color[gray]{0.75}FO}%
\colorbox{green}{\color[gray]{0.75}FO}%
\colorbox{green}{\color[gray]{0.75}FO}%
\colorbox{green}{\color[gray]{0.75}FO}%
\colorbox{green}{\color[gray]{0.75}FO}%
\colorbox{green}{\color[gray]{0.75}FO}%
\colorbox{green}{\color[gray]{0.75}FO}%
\colorbox{green}{\color[gray]{0.75}FO}%
\colorbox{green}{\color[gray]{0.75}FO}%
\colorbox{green}{\color[gray]{0.75}FO}%
\colorbox{green}{\color[gray]{0.75}FO}%
\colorbox{green}{\color[gray]{0.75}FO}%
\colorbox{green}{\color[gray]{0.75}FO}%
\colorbox{green}{\color[gray]{0.75}FO}%
\colorbox{green}{\color[gray]{0.75}FO}%
\\
\colorbox{green}{\color[gray]{0.75}FO}%
\colorbox{green}{\color[gray]{0.75}FO}%
\colorbox{green}{\color[gray]{0.75}FO}%
\colorbox{green}{\color[gray]{0.75}FO}%
\colorbox{green}{\color[gray]{0.75}FO}%
\colorbox{green}{\color[gray]{0.75}FO}%
\colorbox{green}{\color[gray]{0.75}FO}%
\colorbox{green}{\color[gray]{0.75}FO}%
\colorbox{green}{\color[gray]{0.75}FO}%
\colorbox{green}{\color[gray]{0.75}FO}%
\colorbox{green}{\color[gray]{0.75}FO}%
\colorbox{green}{\color[gray]{0.75}FO}%
\colorbox{green}{\color[gray]{0.75}FO}%
\colorbox{green}{\color[gray]{0.75}FO}%
\colorbox{green}{\color[gray]{0.75}FO}%
\colorbox{green}{\color[gray]{0.75}FO}%
\colorbox{green}{\color[gray]{0.75}FO}%
\colorbox{green}{\color[gray]{0.75}FO}%
\colorbox{green}{\color[gray]{0.75}FO}%
\colorbox{green}{\color[gray]{0.75}FO}%
\colorbox{green}{\color[gray]{0.75}FO}%
\colorbox{green}{\color[gray]{0.75}FO}%
\colorbox{green}{\color[gray]{0.75}FO}%
\colorbox{green}{\color[gray]{0.75}FO}%
\colorbox{green}{\color[gray]{0.75}FO}%
\colorbox{green}{\color[gray]{0.75}FO}%
\colorbox{green}{\color[gray]{0.75}FO}%
\colorbox{green}{\color[gray]{0.75}FO}%
\colorbox{green}{\color[gray]{0.75}FO}%
\colorbox{green}{\color[gray]{0.75}FO}%
\colorbox{green}{\color[gray]{0.75}FO}%
\colorbox{green}{\color[gray]{0.75}FO}%
\colorbox{green}{\color[gray]{0.75}FO}%
\colorbox{green}{\color[gray]{0.75}FO}%
\colorbox{green}{\color[gray]{0.75}FO}%
\colorbox{green}{\color[gray]{0.75}FO}%
\colorbox{green}{\color[gray]{0.75}FO}%
\colorbox{green}{\color[gray]{0.75}FO}%
\colorbox{green}{\color[gray]{0.75}FO}%
\colorbox{green}{\color[gray]{0.75}FO}%
\colorbox{green}{\color[gray]{0.75}FO}%
\colorbox{green}{\color[gray]{0.75}FO}%
\colorbox{green}{\color[gray]{0.75}FO}%
\colorbox{green}{\color[gray]{0.75}FO}%
\colorbox{green}{\color[gray]{0.75}FO}%
\colorbox{green}{\color[gray]{0.75}FO}%
\colorbox{green}{\color[gray]{0.75}FO}%
\colorbox{green}{\color[gray]{0.75}FO}%
\colorbox{green}{\color[gray]{0.75}FO}%
\colorbox{green}{\color[gray]{0.75}FO}%
\colorbox{green}{\color[gray]{0.75}FO}%
\colorbox{green}{\color[gray]{0.75}FO}%
\colorbox{green}{\color[gray]{0.75}FO}%
\colorbox{green}{\color[gray]{0.75}FO}%
\colorbox{green}{\color[gray]{0.75}FO}%
\colorbox{green}{\color[gray]{0.75}FO}%
\colorbox{green}{\color[gray]{0.75}FO}%
\colorbox{green}{\color[gray]{0.75}FO}%
\colorbox{green}{\color[gray]{0.75}FO}%
\colorbox{green}{\color[gray]{0.75}FO}%
\colorbox{green}{\color[gray]{0.75}FO}%
\colorbox{green}{\color[gray]{0.75}FO}%
\colorbox{green}{\color[gray]{0.75}FO}%
\colorbox{green}{\color[gray]{0.75}FO}%
\colorbox{green}{\color[gray]{0.75}FO}%
\colorbox{green}{\color[gray]{0.75}FO}%
\colorbox{green}{\color[gray]{0.75}FO}%
\colorbox{green}{\color[gray]{0.75}FO}%
\colorbox{green}{\color[gray]{0.75}FO}%
\colorbox{green}{\color[gray]{0.75}FO}%
\colorbox{green}{\color[gray]{0.75}FO}%
\colorbox{green}{\color[gray]{0.75}FO}%
\colorbox{green}{\color[gray]{0.75}FO}%
\colorbox{green}{\color[gray]{0.75}FO}%
\colorbox{green}{\color[gray]{0.75}FO}%
\colorbox{green}{\color[gray]{0.75}FO}%
\colorbox{green}{\color[gray]{0.75}FO}%
\colorbox{green}{\color[gray]{0.75}FO}%
\colorbox{green}{\color[gray]{0.75}FO}%
\colorbox{green}{\color[gray]{0.75}FO}%
\colorbox{green}{\color[gray]{0.75}FO}%
\colorbox{green}{\color[gray]{0.75}FO}%
\colorbox{green}{\color[gray]{0.75}FO}%
\colorbox{green}{\color[gray]{0.75}FO}%
\colorbox{green}{\color[gray]{0.75}FO}%
\colorbox{green}{\color[gray]{0.75}FO}%
\colorbox{green}{\color[gray]{0.75}FO}%
\colorbox{green}{\color[gray]{0.75}FO}%
\colorbox{green}{\color[gray]{0.75}FO}%
\colorbox{green}{\color[gray]{0.75}FO}%
\colorbox{green}{\color[gray]{0.75}FO}%
\colorbox{green}{\color[gray]{0.75}FO}%
\colorbox{green}{\color[gray]{0.75}FO}%
\colorbox{green}{\color[gray]{0.75}FO}%
\colorbox{green}{\color[gray]{0.75}FO}%
\colorbox{green}{\color[gray]{0.75}FO}%
\colorbox{green}{\color[gray]{0.75}FO}%
\colorbox{green}{\color[gray]{0.75}FO}%
\colorbox{green}{\color[gray]{0.75}FO}%
\colorbox{green}{\color[gray]{0.75}FO}%
\\
\colorbox{green}{\color[gray]{0.75}FO}%
\colorbox{green}{\color[gray]{0.75}FO}%
\colorbox{green}{\color[gray]{0.75}FO}%
\colorbox{green}{\color[gray]{0.75}FO}%
\colorbox{green}{\color[gray]{0.75}FO}%
\colorbox{green}{\color[gray]{0.75}FO}%
\colorbox{green}{\color[gray]{0.75}FO}%
\colorbox{green}{\color[gray]{0.75}FO}%
\colorbox{green}{\color[gray]{0.75}FO}%
\colorbox{green}{\color[gray]{0.75}FO}%
\colorbox{green}{\color[gray]{0.75}FO}%
\colorbox{green}{\color[gray]{0.75}FO}%
\colorbox{green}{\color[gray]{0.75}FO}%
\colorbox{green}{\color[gray]{0.75}FO}%
\colorbox{green}{\color[gray]{0.75}FO}%
\colorbox{green}{\color[gray]{0.75}FO}%
\colorbox{green}{\color[gray]{0.75}FO}%
\colorbox{green}{\color[gray]{0.75}FO}%
\colorbox{green}{\color[gray]{0.75}FO}%
\colorbox{green}{\color[gray]{0.75}FO}%
\colorbox{green}{\color[gray]{0.75}FO}%
\colorbox{green}{\color[gray]{0.75}FO}%
\colorbox{green}{\color[gray]{0.75}FO}%
\colorbox{green}{\color[gray]{0.75}FO}%
\colorbox{green}{\color[gray]{0.75}FO}%
\colorbox{green}{\color[gray]{0.75}FO}%
\colorbox{green}{\color[gray]{0.75}FO}%
\colorbox{green}{\color[gray]{0.75}FO}%
\colorbox{green}{\color[gray]{0.75}FO}%
\colorbox{green}{\color[gray]{0.75}FO}%
\colorbox{green}{\color[gray]{0.75}FO}%
\colorbox{green}{\color[gray]{0.75}FO}%
\colorbox{green}{\color[gray]{0.75}FO}%
\colorbox{green}{\color[gray]{0.75}FO}%
\colorbox{green}{\color[gray]{0.75}FO}%
\colorbox{green}{\color[gray]{0.75}FO}%
\colorbox{green}{\color[gray]{0.75}FO}%
\colorbox{green}{\color[gray]{0.75}FO}%
\colorbox{green}{\color[gray]{0.75}FO}%
\colorbox{green}{\color[gray]{0.75}FO}%
\colorbox{green}{\color[gray]{0.75}FO}%
\colorbox{green}{\color[gray]{0.75}FO}%
\colorbox{green}{\color[gray]{0.75}FO}%
\colorbox{green}{\color[gray]{0.75}FO}%
\colorbox{green}{\color[gray]{0.75}FO}%
\colorbox{green}{\color[gray]{0.75}FO}%
\colorbox{green}{\color[gray]{0.75}FO}%
\colorbox{green}{\color[gray]{0.75}FO}%
\colorbox{green}{\color[gray]{0.75}FO}%
\colorbox{green}{\color[gray]{0.75}FO}%
\colorbox{green}{\color[gray]{0.75}FO}%
\colorbox{green}{\color[gray]{0.75}FO}%
\colorbox{green}{\color[gray]{0.75}FO}%
\colorbox{green}{\color[gray]{0.75}FO}%
\colorbox{green}{\color[gray]{0.75}FO}%
\colorbox{green}{\color[gray]{0.75}FO}%
\colorbox{green}{\color[gray]{0.75}FO}%
\colorbox{green}{\color[gray]{0.75}FO}%
\colorbox{green}{\color[gray]{0.75}FO}%
\colorbox{green}{\color[gray]{0.75}FO}%
\colorbox{green}{\color[gray]{0.75}FO}%
\colorbox{green}{\color[gray]{0.75}FO}%
\colorbox{green}{\color[gray]{0.75}FO}%
\colorbox{green}{\color[gray]{0.75}FO}%
\colorbox{green}{\color[gray]{0.75}FO}%
\colorbox{green}{\color[gray]{0.75}FO}%
\colorbox{green}{\color[gray]{0.75}FO}%
\colorbox{green}{\color[gray]{0.75}FO}%
\colorbox{green}{\color[gray]{0.75}FO}%
\colorbox{green}{\color[gray]{0.75}FO}%
\colorbox{green}{\color[gray]{0.75}FO}%
\colorbox{green}{\color[gray]{0.75}FO}%
\colorbox{green}{\color[gray]{0.75}FO}%
\colorbox{green}{\color[gray]{0.75}FO}%
\colorbox{green}{\color[gray]{0.75}FO}%
\colorbox{green}{\color[gray]{0.75}FO}%
\colorbox{green}{\color[gray]{0.75}FO}%
\colorbox{green}{\color[gray]{0.75}FO}%
\colorbox{green}{\color[gray]{0.75}FO}%
\colorbox{green}{\color[gray]{0.75}FO}%
\colorbox{green}{\color[gray]{0.75}FO}%
\colorbox{green}{\color[gray]{0.75}FO}%
\colorbox{green}{\color[gray]{0.75}FO}%
\colorbox{green}{\color[gray]{0.75}FO}%
\colorbox{green}{\color[gray]{0.75}FO}%
\colorbox{green}{\color[gray]{0.75}FO}%
\colorbox{green}{\color[gray]{0.75}FO}%
\colorbox{green}{\color[gray]{0.75}FO}%
\colorbox{green}{\color[gray]{0.75}FO}%
\colorbox{green}{\color[gray]{0.75}FO}%
\colorbox{green}{\color[gray]{0.75}FO}%
\colorbox{green}{\color[gray]{0.75}FO}%
\colorbox{green}{\color[gray]{0.75}FO}%
\colorbox{green}{\color[gray]{0.75}FO}%
\colorbox{green}{\color[gray]{0.75}FO}%
\colorbox{green}{\color[gray]{0.75}FO}%
\colorbox{green}{\color[gray]{0.75}FO}%
\colorbox{green}{\color[gray]{0.75}FO}%
\colorbox{green}{\color[gray]{0.75}FO}%
\colorbox{green}{\color[gray]{0.75}FO}%
\\
\colorbox{green}{\color[gray]{0.75}FO}%
\colorbox{green}{\color[gray]{0.75}FO}%
\colorbox{green}{\color[gray]{0.75}FO}%
\colorbox{green}{\color[gray]{0.75}FO}%
\colorbox{green}{\color[gray]{0.75}FO}%
\colorbox{green}{\color[gray]{0.75}FO}%
\colorbox{green}{\color[gray]{0.75}FO}%
\colorbox{green}{\color[gray]{0.75}FO}%
\colorbox{green}{\color[gray]{0.75}FO}%
\colorbox{green}{\color[gray]{0.75}FO}%
\colorbox{green}{\color[gray]{0.75}FO}%
\colorbox{green}{\color[gray]{0.75}FO}%
\colorbox{green}{\color[gray]{0.75}FO}%
\colorbox{green}{\color[gray]{0.75}FO}%
\colorbox{green}{\color[gray]{0.75}FO}%
\colorbox{green}{\color[gray]{0.75}FO}%
\colorbox{green}{\color[gray]{0.75}FO}%
\colorbox{green}{\color[gray]{0.75}FO}%
\colorbox{green}{\color[gray]{0.75}FO}%
\colorbox{green}{\color[gray]{0.75}FO}%
\colorbox{green}{\color[gray]{0.75}FO}%
\colorbox{green}{\color[gray]{0.75}FO}%
\colorbox{green}{\color[gray]{0.75}FO}%
\colorbox{green}{\color[gray]{0.75}FO}%
\colorbox{green}{\color[gray]{0.75}FO}%
\colorbox{green}{\color[gray]{0.75}FO}%
\colorbox{green}{\color[gray]{0.75}FO}%
\colorbox{green}{\color[gray]{0.75}FO}%
\colorbox{green}{\color[gray]{0.75}FO}%
\colorbox{green}{\color[gray]{0.75}FO}%
\colorbox{green}{\color[gray]{0.75}FO}%
\colorbox{green}{\color[gray]{0.75}FO}%
\colorbox{green}{\color[gray]{0.75}FO}%
\colorbox{green}{\color[gray]{0.75}FO}%
\colorbox{green}{\color[gray]{0.75}FO}%
\colorbox{green}{\color[gray]{0.75}FO}%
\colorbox{green}{\color[gray]{0.75}FO}%
\colorbox{green}{\color[gray]{0.75}FO}%
\colorbox{green}{\color[gray]{0.75}FO}%
\colorbox{green}{\color[gray]{0.75}FO}%
\colorbox{green}{\color[gray]{0.75}FO}%
\colorbox{green}{\color[gray]{0.75}FO}%
\colorbox{green}{\color[gray]{0.75}FO}%
\colorbox{green}{\color[gray]{0.75}FO}%
\colorbox{green}{\color[gray]{0.75}FO}%
\colorbox{green}{\color[gray]{0.75}FO}%
\colorbox{green}{\color[gray]{0.75}FO}%
\colorbox{green}{\color[gray]{0.75}FO}%
\colorbox{green}{\color[gray]{0.75}FO}%
\colorbox{green}{\color[gray]{0.75}FO}%
\colorbox{green}{\color[gray]{0.75}FO}%
\colorbox{green}{\color[gray]{0.75}FO}%
\colorbox{green}{\color[gray]{0.75}FO}%
\colorbox{green}{\color[gray]{0.75}FO}%
\colorbox{green}{\color[gray]{0.75}FO}%
\colorbox{green}{\color[gray]{0.75}FO}%
\colorbox{green}{\color[gray]{0.75}FO}%
\colorbox{green}{\color[gray]{0.75}FO}%
\colorbox{green}{\color[gray]{0.75}FO}%
\colorbox{green}{\color[gray]{0.75}FO}%
\colorbox{green}{\color[gray]{0.75}FO}%
\colorbox{green}{\color[gray]{0.75}FO}%
\colorbox{green}{\color[gray]{0.75}FO}%
\colorbox{green}{\color[gray]{0.75}FO}%
\colorbox{green}{\color[gray]{0.75}FO}%
\colorbox{green}{\color[gray]{0.75}FO}%
\colorbox{green}{\color[gray]{0.75}FO}%
\colorbox{green}{\color[gray]{0.75}FO}%
\colorbox{green}{\color[gray]{0.75}FO}%
\colorbox{green}{\color[gray]{0.75}FO}%
\colorbox{green}{\color[gray]{0.75}FO}%
\colorbox{green}{\color[gray]{0.75}FO}%
\colorbox{green}{\color[gray]{0.75}FO}%
\colorbox{green}{\color[gray]{0.75}FO}%
\colorbox{green}{\color[gray]{0.75}FO}%
\colorbox{green}{\color[gray]{0.75}FO}%
\colorbox{green}{\color[gray]{0.75}FO}%
\colorbox{green}{\color[gray]{0.75}FO}%
\colorbox{green}{\color[gray]{0.75}FO}%
\colorbox{green}{\color[gray]{0.75}FO}%
\colorbox{green}{\color[gray]{0.75}FO}%
\colorbox{green}{\color[gray]{0.75}FO}%
\colorbox{green}{\color[gray]{0.75}FO}%
\colorbox{green}{\color[gray]{0.75}FO}%
\colorbox{green}{\color[gray]{0.75}FO}%
\colorbox{green}{\color[gray]{0.75}FO}%
\colorbox{green}{\color[gray]{0.75}FO}%
\colorbox{green}{\color[gray]{0.75}FO}%
\colorbox{green}{\color[gray]{0.75}FO}%
\colorbox{green}{\color[gray]{0.75}FO}%
\colorbox{green}{\color[gray]{0.75}FO}%
\colorbox{green}{\color[gray]{0.75}FO}%
\colorbox{green}{\color[gray]{0.75}FO}%
\colorbox{green}{\color[gray]{0.75}FO}%
\colorbox{green}{\color[gray]{0.75}FO}%
\colorbox{green}{\color[gray]{0.75}FO}%
\colorbox{green}{\color[gray]{0.75}FO}%
\colorbox{green}{\color[gray]{0.75}FO}%
\colorbox{green}{\color[gray]{0.75}FO}%
\colorbox{green}{\color[gray]{0.75}FO}%
\\
\colorbox{green}{\color[gray]{0.75}FO}%
\colorbox{green}{\color[gray]{0.75}FO}%
\colorbox{green}{\color[gray]{0.75}FO}%
\colorbox{green}{\color[gray]{0.75}FO}%
\colorbox{green}{\color[gray]{0.75}FO}%
\colorbox{green}{\color[gray]{0.75}FO}%
\colorbox{green}{\color[gray]{0.75}FO}%
\colorbox{green}{\color[gray]{0.75}FO}%
\colorbox{green}{\color[gray]{0.75}FO}%
\colorbox{green}{\color[gray]{0.75}FO}%
\colorbox{green}{\color[gray]{0.75}FO}%
\colorbox{green}{\color[gray]{0.75}FO}%
\colorbox{green}{\color[gray]{0.75}FO}%
\colorbox{green}{\color[gray]{0.75}FO}%
\colorbox{green}{\color[gray]{0.75}FO}%
\colorbox{green}{\color[gray]{0.75}FO}%
\colorbox{green}{\color[gray]{0.75}FO}%
\colorbox{green}{\color[gray]{0.75}FO}%
\colorbox{green}{\color[gray]{0.75}FO}%
\colorbox{green}{\color[gray]{0.75}FO}%
\colorbox{green}{\color[gray]{0.75}FO}%
\colorbox{green}{\color[gray]{0.75}FO}%
\colorbox{green}{\color[gray]{0.75}FO}%
\colorbox{green}{\color[gray]{0.75}FO}%
\colorbox{green}{\color[gray]{0.75}FO}%
\colorbox{green}{\color[gray]{0.75}FO}%
\colorbox{green}{\color[gray]{0.75}FO}%
\colorbox{green}{\color[gray]{0.75}FO}%
\colorbox{green}{\color[gray]{0.75}FO}%
\colorbox{green}{\color[gray]{0.75}FO}%
\colorbox{green}{\color[gray]{0.75}FO}%
\colorbox{green}{\color[gray]{0.75}FO}%
\colorbox{green}{\color[gray]{0.75}FO}%
\colorbox{green}{\color[gray]{0.75}FO}%
\colorbox{green}{\color[gray]{0.75}FO}%
\colorbox{green}{\color[gray]{0.75}FO}%
\colorbox{green}{\color[gray]{0.75}FO}%
\colorbox{green}{\color[gray]{0.75}FO}%
\colorbox{green}{\color[gray]{0.75}FO}%
\colorbox{green}{\color[gray]{0.75}FO}%
\colorbox{green}{\color[gray]{0.75}FO}%
\colorbox{green}{\color[gray]{0.75}FO}%
\colorbox{green}{\color[gray]{0.75}FO}%
\colorbox{green}{\color[gray]{0.75}FO}%
\colorbox{green}{\color[gray]{0.75}FO}%
\colorbox{green}{\color[gray]{0.75}FO}%
\colorbox{green}{\color[gray]{0.75}FO}%
\colorbox{green}{\color[gray]{0.75}FO}%
\colorbox{green}{\color[gray]{0.75}FO}%
\colorbox{green}{\color[gray]{0.75}FO}%
\colorbox{green}{\color[gray]{0.75}FO}%
\colorbox{green}{\color[gray]{0.75}FO}%
\colorbox{green}{\color[gray]{0.75}FO}%
\colorbox{green}{\color[gray]{0.75}FO}%
\colorbox{green}{\color[gray]{0.75}FO}%
\colorbox{green}{\color[gray]{0.75}FO}%
\colorbox{green}{\color[gray]{0.75}FO}%
\colorbox{green}{\color[gray]{0.75}FO}%
\colorbox{green}{\color[gray]{0.75}FO}%
\colorbox{green}{\color[gray]{0.75}FO}%
\colorbox{green}{\color[gray]{0.75}FO}%
\colorbox{green}{\color[gray]{0.75}FO}%
\colorbox{green}{\color[gray]{0.75}FO}%
\colorbox{green}{\color[gray]{0.75}FO}%
\colorbox{green}{\color[gray]{0.75}FO}%
\colorbox{green}{\color[gray]{0.75}FO}%
\colorbox{green}{\color[gray]{0.75}FO}%
\colorbox{green}{\color[gray]{0.75}FO}%
\colorbox{green}{\color[gray]{0.75}FO}%
\colorbox{green}{\color[gray]{0.75}FO}%
\colorbox{green}{\color[gray]{0.75}FO}%
\colorbox{green}{\color[gray]{0.75}FO}%
\colorbox{green}{\color[gray]{0.75}FO}%
\colorbox{green}{\color[gray]{0.75}FO}%
\colorbox{green}{\color[gray]{0.75}FO}%
\colorbox{green}{\color[gray]{0.75}FO}%
\colorbox{green}{\color[gray]{0.75}FO}%
\colorbox{green}{\color[gray]{0.75}FO}%
\colorbox{green}{\color[gray]{0.75}FO}%
\colorbox{green}{\color[gray]{0.75}FO}%
\colorbox{green}{\color[gray]{0.75}FO}%
\colorbox{green}{\color[gray]{0.75}FO}%
\colorbox{green}{\color[gray]{0.75}FO}%
\colorbox{green}{\color[gray]{0.75}FO}%
\colorbox{green}{\color[gray]{0.75}FO}%
\colorbox{green}{\color[gray]{0.75}FO}%
\colorbox{green}{\color[gray]{0.75}FO}%
\colorbox{green}{\color[gray]{0.75}FO}%
\colorbox{green}{\color[gray]{0.75}FO}%
\colorbox{green}{\color[gray]{0.75}FO}%
\colorbox{green}{\color[gray]{0.75}FO}%
\colorbox{green}{\color[gray]{0.75}FO}%
\colorbox{green}{\color[gray]{0.75}FO}%
\colorbox{green}{\color[gray]{0.75}FO}%
\colorbox{green}{\color[gray]{0.75}FO}%
\colorbox{green}{\color[gray]{0.75}FO}%
\colorbox{green}{\color[gray]{0.75}FO}%
\colorbox{green}{\color[gray]{0.75}FO}%
\colorbox{green}{\color[gray]{0.75}FO}%
\colorbox{green}{\color[gray]{0.75}FO}%
\\
\colorbox{green}{\color[gray]{0.75}FO}%
\colorbox{green}{\color[gray]{0.75}FO}%
\colorbox{green}{\color[gray]{0.75}FO}%
\colorbox{green}{\color[gray]{0.75}FO}%
\colorbox{green}{\color[gray]{0.75}FO}%
\colorbox{green}{\color[gray]{0.75}FO}%
\colorbox{green}{\color[gray]{0.75}FO}%
\colorbox{green}{\color[gray]{0.75}FO}%
\colorbox{green}{\color[gray]{0.75}FO}%
\colorbox{green}{\color[gray]{0.75}FO}%
\colorbox{green}{\color[gray]{0.75}FO}%
\colorbox{green}{\color[gray]{0.75}FO}%
\colorbox{green}{\color[gray]{0.75}FO}%
\colorbox{green}{\color[gray]{0.75}FO}%
\colorbox{green}{\color[gray]{0.75}FO}%
\colorbox{green}{\color[gray]{0.75}FO}%
\colorbox{green}{\color[gray]{0.75}FO}%
\colorbox{green}{\color[gray]{0.75}FO}%
\colorbox{green}{\color[gray]{0.75}FO}%
\colorbox{green}{\color[gray]{0.75}FO}%
\colorbox{green}{\color[gray]{0.75}FO}%
\colorbox{green}{\color[gray]{0.75}FO}%
\colorbox{green}{\color[gray]{0.75}FO}%
\colorbox{green}{\color[gray]{0.75}FO}%
\colorbox{green}{\color[gray]{0.75}FO}%
\colorbox{green}{\color[gray]{0.75}FO}%
\colorbox{green}{\color[gray]{0.75}FO}%
\colorbox{green}{\color[gray]{0.75}FO}%
\colorbox{green}{\color[gray]{0.75}FO}%
\colorbox{green}{\color[gray]{0.75}FO}%
\colorbox{green}{\color[gray]{0.75}FO}%
\colorbox{green}{\color[gray]{0.75}FO}%
\colorbox{green}{\color[gray]{0.75}FO}%
\colorbox{green}{\color[gray]{0.75}FO}%
\colorbox{green}{\color[gray]{0.75}FO}%
\colorbox{green}{\color[gray]{0.75}FO}%
\colorbox{green}{\color[gray]{0.75}FO}%
\colorbox{green}{\color[gray]{0.75}FO}%
\colorbox{green}{\color[gray]{0.75}FO}%
\colorbox{green}{\color[gray]{0.75}FO}%
\colorbox{green}{\color[gray]{0.75}FO}%
\colorbox{green}{\color[gray]{0.75}FO}%
\colorbox{green}{\color[gray]{0.75}FO}%
\colorbox{green}{\color[gray]{0.75}FO}%
\colorbox{green}{\color[gray]{0.75}FO}%
\colorbox{green}{\color[gray]{0.75}FO}%
\colorbox{green}{\color[gray]{0.75}FO}%
\colorbox{green}{\color[gray]{0.75}FO}%
\colorbox{green}{\color[gray]{0.75}FO}%
\colorbox{green}{\color[gray]{0.75}FO}%
\colorbox{green}{\color[gray]{0.75}FO}%
\colorbox{green}{\color[gray]{0.75}FO}%
\colorbox{green}{\color[gray]{0.75}FO}%
\colorbox{green}{\color[gray]{0.75}FO}%
\colorbox{green}{\color[gray]{0.75}FO}%
\colorbox{green}{\color[gray]{0.75}FO}%
\colorbox{green}{\color[gray]{0.75}FO}%
\colorbox{green}{\color[gray]{0.75}FO}%
\colorbox{green}{\color[gray]{0.75}FO}%
\colorbox{green}{\color[gray]{0.75}FO}%
\colorbox{green}{\color[gray]{0.75}FO}%
\colorbox{green}{\color[gray]{0.75}FO}%
\colorbox{green}{\color[gray]{0.75}FO}%
\colorbox{green}{\color[gray]{0.75}FO}%
\colorbox{green}{\color[gray]{0.75}FO}%
\colorbox{green}{\color[gray]{0.75}FO}%
\colorbox{green}{\color[gray]{0.75}FO}%
\colorbox{green}{\color[gray]{0.75}FO}%
\colorbox{green}{\color[gray]{0.75}FO}%
\colorbox{green}{\color[gray]{0.75}FO}%
\colorbox{green}{\color[gray]{0.75}FO}%
\colorbox{green}{\color[gray]{0.75}FO}%
\colorbox{green}{\color[gray]{0.75}FO}%
\colorbox{green}{\color[gray]{0.75}FO}%
\colorbox{green}{\color[gray]{0.75}FO}%
\colorbox{green}{\color[gray]{0.75}FO}%
\colorbox{green}{\color[gray]{0.75}FO}%
\colorbox{green}{\color[gray]{0.75}FO}%
\colorbox{green}{\color[gray]{0.75}FO}%
\colorbox{green}{\color[gray]{0.75}FO}%
\colorbox{green}{\color[gray]{0.75}FO}%
\colorbox{green}{\color[gray]{0.75}FO}%
\colorbox{green}{\color[gray]{0.75}FO}%
\colorbox{green}{\color[gray]{0.75}FO}%
\colorbox{green}{\color[gray]{0.75}FO}%
\colorbox{green}{\color[gray]{0.75}FO}%
\colorbox{green}{\color[gray]{0.75}FO}%
\colorbox{green}{\color[gray]{0.75}FO}%
\colorbox{green}{\color[gray]{0.75}FO}%
\colorbox{green}{\color[gray]{0.75}FO}%
\colorbox{green}{\color[gray]{0.75}FO}%
\colorbox{green}{\color[gray]{0.75}FO}%
\colorbox{green}{\color[gray]{0.75}FO}%
\colorbox{green}{\color[gray]{0.75}FO}%
\colorbox{green}{\color[gray]{0.75}FO}%
\colorbox{green}{\color[gray]{0.75}FO}%
\colorbox{green}{\color[gray]{0.75}FO}%
\colorbox{green}{\color[gray]{0.75}FO}%
\colorbox{green}{\color[gray]{0.75}FO}%
\colorbox{green}{\color[gray]{0.75}FO}%
\\
\colorbox{green}{\color[gray]{0.75}FO}%
\colorbox{green}{\color[gray]{0.75}FO}%
\colorbox{green}{\color[gray]{0.75}FO}%
\colorbox{green}{\color[gray]{0.75}FO}%
\colorbox{green}{\color[gray]{0.75}FO}%
\colorbox{green}{\color[gray]{0.75}FO}%
\colorbox{green}{\color[gray]{0.75}FO}%
\colorbox{green}{\color[gray]{0.75}FO}%
\colorbox{green}{\color[gray]{0.75}FO}%
\colorbox{green}{\color[gray]{0.75}FO}%
\colorbox{green}{\color[gray]{0.75}FO}%
\colorbox{green}{\color[gray]{0.75}FO}%
\colorbox{green}{\color[gray]{0.75}FO}%
\colorbox{green}{\color[gray]{0.75}FO}%
\colorbox{green}{\color[gray]{0.75}FO}%
\colorbox{green}{\color[gray]{0.75}FO}%
\colorbox{green}{\color[gray]{0.75}FO}%
\colorbox{green}{\color[gray]{0.75}FO}%
\colorbox{green}{\color[gray]{0.75}FO}%
\colorbox{green}{\color[gray]{0.75}FO}%
\colorbox{green}{\color[gray]{0.75}FO}%
\colorbox{green}{\color[gray]{0.75}FO}%
\colorbox{green}{\color[gray]{0.75}FO}%
\colorbox{green}{\color[gray]{0.75}FO}%
\colorbox{green}{\color[gray]{0.75}FO}%
\colorbox{green}{\color[gray]{0.75}FO}%
\colorbox{green}{\color[gray]{0.75}FO}%
\colorbox{green}{\color[gray]{0.75}FO}%
\colorbox{green}{\color[gray]{0.75}FO}%
\colorbox{green}{\color[gray]{0.75}FO}%
\colorbox{green}{\color[gray]{0.75}FO}%
\colorbox{green}{\color[gray]{0.75}FO}%
\colorbox{green}{\color[gray]{0.75}FO}%
\colorbox{green}{\color[gray]{0.75}FO}%
\colorbox{green}{\color[gray]{0.75}FO}%
\colorbox{green}{\color[gray]{0.75}FO}%
\colorbox{green}{\color[gray]{0.75}FO}%
\colorbox{green}{\color[gray]{0.75}FO}%
\colorbox{green}{\color[gray]{0.75}FO}%
\colorbox{green}{\color[gray]{0.75}FO}%
\colorbox{green}{\color[gray]{0.75}FO}%
\colorbox{green}{\color[gray]{0.75}FO}%
\colorbox{green}{\color[gray]{0.75}FO}%
\colorbox{green}{\color[gray]{0.75}FO}%
\colorbox{green}{\color[gray]{0.75}FO}%
\colorbox{green}{\color[gray]{0.75}FO}%
\colorbox{green}{\color[gray]{0.75}FO}%
\colorbox{green}{\color[gray]{0.75}FO}%
\colorbox{green}{\color[gray]{0.75}FO}%
\colorbox{green}{\color[gray]{0.75}FO}%
\colorbox{green}{\color[gray]{0.75}FO}%
\colorbox{green}{\color[gray]{0.75}FO}%
\colorbox{green}{\color[gray]{0.75}FO}%
\colorbox{green}{\color[gray]{0.75}FO}%
\colorbox{green}{\color[gray]{0.75}FO}%
\colorbox{green}{\color[gray]{0.75}FO}%
\colorbox{green}{\color[gray]{0.75}FO}%
\colorbox{green}{\color[gray]{0.75}FO}%
\colorbox{green}{\color[gray]{0.75}FO}%
\colorbox{green}{\color[gray]{0.75}FO}%
\colorbox{green}{\color[gray]{0.75}FO}%
\colorbox{green}{\color[gray]{0.75}FO}%
\colorbox{green}{\color[gray]{0.75}FO}%
\colorbox{green}{\color[gray]{0.75}FO}%
\colorbox{green}{\color[gray]{0.75}FO}%
\colorbox{green}{\color[gray]{0.75}FO}%
\colorbox{green}{\color[gray]{0.75}FO}%
\colorbox{green}{\color[gray]{0.75}FO}%
\colorbox{green}{\color[gray]{0.75}FO}%
\colorbox{green}{\color[gray]{0.75}FO}%
\colorbox{green}{\color[gray]{0.75}FO}%
\colorbox{green}{\color[gray]{0.75}FO}%
\colorbox{green}{\color[gray]{0.75}FO}%
\colorbox{green}{\color[gray]{0.75}FO}%
\colorbox{green}{\color[gray]{0.75}FO}%
\colorbox{green}{\color[gray]{0.75}FO}%
\colorbox{green}{\color[gray]{0.75}FO}%
\colorbox{green}{\color[gray]{0.75}FO}%
\colorbox{green}{\color[gray]{0.75}FO}%
\colorbox{green}{\color[gray]{0.75}FO}%
\colorbox{green}{\color[gray]{0.75}FO}%
\colorbox{green}{\color[gray]{0.75}FO}%
\colorbox{green}{\color[gray]{0.75}FO}%
\colorbox{green}{\color[gray]{0.75}FO}%
\colorbox{green}{\color[gray]{0.75}FO}%
\colorbox{green}{\color[gray]{0.75}FO}%
\colorbox{green}{\color[gray]{0.75}FO}%
\colorbox{green}{\color[gray]{0.75}FO}%
\colorbox{green}{\color[gray]{0.75}FO}%
\colorbox{green}{\color[gray]{0.75}FO}%
\colorbox{green}{\color[gray]{0.75}FO}%
\colorbox{green}{\color[gray]{0.75}FO}%
\colorbox{green}{\color[gray]{0.75}FO}%
\colorbox{green}{\color[gray]{0.75}FO}%
\colorbox{green}{\color[gray]{0.75}FO}%
\colorbox{green}{\color[gray]{0.75}FO}%
\colorbox{green}{\color[gray]{0.75}FO}%
\colorbox{green}{\color[gray]{0.75}FO}%
\colorbox{green}{\color[gray]{0.75}FO}%
\colorbox{green}{\color[gray]{0.75}FO}%
\\
\colorbox{green}{\color[gray]{0.75}FO}%
\colorbox{green}{\color[gray]{0.75}FO}%
\colorbox{green}{\color[gray]{0.75}FO}%
\colorbox{green}{\color[gray]{0.75}FO}%
\colorbox{green}{\color[gray]{0.75}FO}%
\colorbox{green}{\color[gray]{0.75}FO}%
\colorbox{green}{\color[gray]{0.75}FO}%
\colorbox{green}{\color[gray]{0.75}FO}%
\colorbox{green}{\color[gray]{0.75}FO}%
\colorbox{green}{\color[gray]{0.75}FO}%
\colorbox{green}{\color[gray]{0.75}FO}%
\colorbox{green}{\color[gray]{0.75}FO}%
\colorbox{green}{\color[gray]{0.75}FO}%
\colorbox{green}{\color[gray]{0.75}FO}%
\colorbox{green}{\color[gray]{0.75}FO}%
\colorbox{green}{\color[gray]{0.75}FO}%
\colorbox{green}{\color[gray]{0.75}FO}%
\colorbox{green}{\color[gray]{0.75}FO}%
\colorbox{green}{\color[gray]{0.75}FO}%
\colorbox{green}{\color[gray]{0.75}FO}%
\colorbox{green}{\color[gray]{0.75}FO}%
\colorbox{green}{\color[gray]{0.75}FO}%
\colorbox{green}{\color[gray]{0.75}FO}%
\colorbox{green}{\color[gray]{0.75}FO}%
\colorbox{green}{\color[gray]{0.75}FO}%
\colorbox{green}{\color[gray]{0.75}FO}%
\colorbox{green}{\color[gray]{0.75}FO}%
\colorbox{green}{\color[gray]{0.75}FO}%
\colorbox{green}{\color[gray]{0.75}FO}%
\colorbox{green}{\color[gray]{0.75}FO}%
\colorbox{green}{\color[gray]{0.75}FO}%
\colorbox{green}{\color[gray]{0.75}FO}%
\colorbox{green}{\color[gray]{0.75}FO}%
\colorbox{green}{\color[gray]{0.75}FO}%
\colorbox{green}{\color[gray]{0.75}FO}%
\colorbox{green}{\color[gray]{0.75}FO}%
\colorbox{green}{\color[gray]{0.75}FO}%
\colorbox{green}{\color[gray]{0.75}FO}%
\colorbox{green}{\color[gray]{0.75}FO}%
\colorbox{green}{\color[gray]{0.75}FO}%
\colorbox{green}{\color[gray]{0.75}FO}%
\colorbox{green}{\color[gray]{0.75}FO}%
\colorbox{green}{\color[gray]{0.75}FO}%
\colorbox{green}{\color[gray]{0.75}FO}%
\colorbox{green}{\color[gray]{0.75}FO}%
\colorbox{green}{\color[gray]{0.75}FO}%
\colorbox{green}{\color[gray]{0.75}FO}%
\colorbox{green}{\color[gray]{0.75}FO}%
\colorbox{green}{\color[gray]{0.75}FO}%
\colorbox{green}{\color[gray]{0.75}FO}%
\colorbox{green}{\color[gray]{0.75}FO}%
\colorbox{green}{\color[gray]{0.75}FO}%
\colorbox{green}{\color[gray]{0.75}FO}%
\colorbox{green}{\color[gray]{0.75}FO}%
\colorbox{green}{\color[gray]{0.75}FO}%
\colorbox{green}{\color[gray]{0.75}FO}%
\colorbox{green}{\color[gray]{0.75}FO}%
\colorbox{green}{\color[gray]{0.75}FO}%
\colorbox{green}{\color[gray]{0.75}FO}%
\colorbox{green}{\color[gray]{0.75}FO}%
\colorbox{green}{\color[gray]{0.75}FO}%
\colorbox{green}{\color[gray]{0.75}FO}%
\colorbox{green}{\color[gray]{0.75}FO}%
\colorbox{green}{\color[gray]{0.75}FO}%
\colorbox{green}{\color[gray]{0.75}FO}%
\colorbox{green}{\color[gray]{0.75}FO}%
\colorbox{green}{\color[gray]{0.75}FO}%
\colorbox{green}{\color[gray]{0.75}FO}%
\colorbox{green}{\color[gray]{0.75}FO}%
\colorbox{green}{\color[gray]{0.75}FO}%
\colorbox{green}{\color[gray]{0.75}FO}%
\colorbox{green}{\color[gray]{0.75}FO}%
\colorbox{green}{\color[gray]{0.75}FO}%
\colorbox{green}{\color[gray]{0.75}FO}%
\colorbox{green}{\color[gray]{0.75}FO}%
\colorbox{green}{\color[gray]{0.75}FO}%
\colorbox{green}{\color[gray]{0.75}FO}%
\colorbox{green}{\color[gray]{0.75}FO}%
\colorbox{green}{\color[gray]{0.75}FO}%
\colorbox{green}{\color[gray]{0.75}FO}%
\colorbox{green}{\color[gray]{0.75}FO}%
\colorbox{green}{\color[gray]{0.75}FO}%
\colorbox{green}{\color[gray]{0.75}FO}%
\colorbox{green}{\color[gray]{0.75}FO}%
\colorbox{green}{\color[gray]{0.75}FO}%
\colorbox{green}{\color[gray]{0.75}FO}%
\colorbox{green}{\color[gray]{0.75}FO}%
\colorbox{green}{\color[gray]{0.75}FO}%
\colorbox{green}{\color[gray]{0.75}FO}%
\colorbox{green}{\color[gray]{0.75}FO}%
\colorbox{green}{\color[gray]{0.75}FO}%
\colorbox{green}{\color[gray]{0.75}FO}%
\colorbox{green}{\color[gray]{0.75}FO}%
\colorbox{green}{\color[gray]{0.75}FO}%
\colorbox{green}{\color[gray]{0.75}FO}%
\colorbox{green}{\color[gray]{0.75}FO}%
\colorbox{green}{\color[gray]{0.75}FO}%
\colorbox{green}{\color[gray]{0.75}FO}%
\colorbox{green}{\color[gray]{0.75}FO}%
\colorbox{green}{\color[gray]{0.75}FO}%
\\
\colorbox{green}{\color[gray]{0.75}FO}%
\colorbox{green}{\color[gray]{0.75}FO}%
\colorbox{green}{\color[gray]{0.75}FO}%
\colorbox{green}{\color[gray]{0.75}FO}%
\colorbox{green}{\color[gray]{0.75}FO}%
\colorbox{green}{\color[gray]{0.75}FO}%
\colorbox{green}{\color[gray]{0.75}FO}%
\colorbox{green}{\color[gray]{0.75}FO}%
\colorbox{green}{\color[gray]{0.75}FO}%
\colorbox{green}{\color[gray]{0.75}FO}%
\colorbox{green}{\color[gray]{0.75}FO}%
\colorbox{green}{\color[gray]{0.75}FO}%
\colorbox{green}{\color[gray]{0.75}FO}%
\colorbox{green}{\color[gray]{0.75}FO}%
\colorbox{green}{\color[gray]{0.75}FO}%
\colorbox{green}{\color[gray]{0.75}FO}%
\colorbox{green}{\color[gray]{0.75}FO}%
\colorbox{green}{\color[gray]{0.75}FO}%
\colorbox{green}{\color[gray]{0.75}FO}%
\colorbox{green}{\color[gray]{0.75}FO}%
\colorbox{green}{\color[gray]{0.75}FO}%
\colorbox{green}{\color[gray]{0.75}FO}%
\colorbox{green}{\color[gray]{0.75}FO}%
\colorbox{green}{\color[gray]{0.75}FO}%
\colorbox{green}{\color[gray]{0.75}FO}%
\colorbox{green}{\color[gray]{0.75}FO}%
\colorbox{green}{\color[gray]{0.75}FO}%
\colorbox{green}{\color[gray]{0.75}FO}%
\colorbox{green}{\color[gray]{0.75}FO}%
\colorbox{green}{\color[gray]{0.75}FO}%
\colorbox{green}{\color[gray]{0.75}FO}%
\colorbox{green}{\color[gray]{0.75}FO}%
\colorbox{green}{\color[gray]{0.75}FO}%
\colorbox{green}{\color[gray]{0.75}FO}%
\colorbox{green}{\color[gray]{0.75}FO}%
\colorbox{green}{\color[gray]{0.75}FO}%
\colorbox{green}{\color[gray]{0.75}FO}%
\colorbox{green}{\color[gray]{0.75}FO}%
\colorbox{green}{\color[gray]{0.75}FO}%
\colorbox{green}{\color[gray]{0.75}FO}%
\colorbox{green}{\color[gray]{0.75}FO}%
\colorbox{green}{\color[gray]{0.75}FO}%
\colorbox{green}{\color[gray]{0.75}FO}%
\colorbox{green}{\color[gray]{0.75}FO}%
\colorbox{green}{\color[gray]{0.75}FO}%
\colorbox{green}{\color[gray]{0.75}FO}%
\colorbox{green}{\color[gray]{0.75}FO}%
\colorbox{green}{\color[gray]{0.75}FO}%
\colorbox{green}{\color[gray]{0.75}FO}%
\colorbox{green}{\color[gray]{0.75}FO}%
\colorbox{green}{\color[gray]{0.75}FO}%
\colorbox{green}{\color[gray]{0.75}FO}%
\colorbox{green}{\color[gray]{0.75}FO}%
\colorbox{green}{\color[gray]{0.75}FO}%
\colorbox{green}{\color[gray]{0.75}FO}%
\colorbox{green}{\color[gray]{0.75}FO}%
\colorbox{green}{\color[gray]{0.75}FO}%
\colorbox{green}{\color[gray]{0.75}FO}%
\colorbox{green}{\color[gray]{0.75}FO}%
\colorbox{green}{\color[gray]{0.75}FO}%
\colorbox{green}{\color[gray]{0.75}FO}%
\colorbox{green}{\color[gray]{0.75}FO}%
\colorbox{green}{\color[gray]{0.75}FO}%
\colorbox{green}{\color[gray]{0.75}FO}%
\colorbox{green}{\color[gray]{0.75}FO}%
\colorbox{green}{\color[gray]{0.75}FO}%
\colorbox{green}{\color[gray]{0.75}FO}%
\colorbox{green}{\color[gray]{0.75}FO}%
\colorbox{green}{\color[gray]{0.75}FO}%
\colorbox{green}{\color[gray]{0.75}FO}%
\colorbox{green}{\color[gray]{0.75}FO}%
\colorbox{green}{\color[gray]{0.75}FO}%
\colorbox{green}{\color[gray]{0.75}FO}%
\colorbox{green}{\color[gray]{0.75}FO}%
\colorbox{green}{\color[gray]{0.75}FO}%
\colorbox{green}{\color[gray]{0.75}FO}%
\colorbox{green}{\color[gray]{0.75}FO}%
\colorbox{green}{\color[gray]{0.75}FO}%
\colorbox{green}{\color[gray]{0.75}FO}%
\colorbox{green}{\color[gray]{0.75}FO}%
\colorbox{green}{\color[gray]{0.75}FO}%
\colorbox{green}{\color[gray]{0.75}FO}%
\colorbox{green}{\color[gray]{0.75}FO}%
\colorbox{green}{\color[gray]{0.75}FO}%
\colorbox{green}{\color[gray]{0.75}FO}%
\colorbox{green}{\color[gray]{0.75}FO}%
\colorbox{green}{\color[gray]{0.75}FO}%
\colorbox{green}{\color[gray]{0.75}FO}%
\colorbox{green}{\color[gray]{0.75}FO}%
\colorbox{green}{\color[gray]{0.75}FO}%
\colorbox{green}{\color[gray]{0.75}FO}%
\colorbox{green}{\color[gray]{0.75}FO}%
\colorbox{green}{\color[gray]{0.75}FO}%
\colorbox{green}{\color[gray]{0.75}FO}%
\colorbox{green}{\color[gray]{0.75}FO}%
\colorbox{green}{\color[gray]{0.75}FO}%
\colorbox{green}{\color[gray]{0.75}FO}%
\colorbox{green}{\color[gray]{0.75}FO}%
\colorbox{green}{\color[gray]{0.75}FO}%
\colorbox{green}{\color[gray]{0.75}FO}%
\\
\colorbox{green}{\color[gray]{0.75}FO}%
\colorbox{green}{\color[gray]{0.75}FO}%
\colorbox{green}{\color[gray]{0.75}FO}%
\colorbox{green}{\color[gray]{0.75}FO}%
\colorbox{green}{\color[gray]{0.75}FO}%
\colorbox{green}{\color[gray]{0.75}FO}%
\colorbox{green}{\color[gray]{0.75}FO}%
\colorbox{green}{\color[gray]{0.75}FO}%
\colorbox{green}{\color[gray]{0.75}FO}%
\colorbox{green}{\color[gray]{0.75}FO}%
\colorbox{green}{\color[gray]{0.75}FO}%
\colorbox{green}{\color[gray]{0.75}FO}%
\colorbox{green}{\color[gray]{0.75}FO}%
\colorbox{green}{\color[gray]{0.75}FO}%
\colorbox{green}{\color[gray]{0.75}FO}%
\colorbox{green}{\color[gray]{0.75}FO}%
\colorbox{green}{\color[gray]{0.75}FO}%
\colorbox{green}{\color[gray]{0.75}FO}%
\colorbox{green}{\color[gray]{0.75}FO}%
\colorbox{green}{\color[gray]{0.75}FO}%
\colorbox{green}{\color[gray]{0.75}FO}%
\colorbox{green}{\color[gray]{0.75}FO}%
\colorbox{green}{\color[gray]{0.75}FO}%
\colorbox{green}{\color[gray]{0.75}FO}%
\colorbox{green}{\color[gray]{0.75}FO}%
\colorbox{green}{\color[gray]{0.75}FO}%
\colorbox{green}{\color[gray]{0.75}FO}%
\colorbox{green}{\color[gray]{0.75}FO}%
\colorbox{green}{\color[gray]{0.75}FO}%
\colorbox{green}{\color[gray]{0.75}FO}%
\colorbox{green}{\color[gray]{0.75}FO}%
\colorbox{green}{\color[gray]{0.75}FO}%
\colorbox{green}{\color[gray]{0.75}FO}%
\colorbox{green}{\color[gray]{0.75}FO}%
\colorbox{green}{\color[gray]{0.75}FO}%
\colorbox{green}{\color[gray]{0.75}FO}%
\colorbox{green}{\color[gray]{0.75}FO}%
\colorbox{green}{\color[gray]{0.75}FO}%
\colorbox{green}{\color[gray]{0.75}FO}%
\colorbox{green}{\color[gray]{0.75}FO}%
\colorbox{green}{\color[gray]{0.75}FO}%
\colorbox{green}{\color[gray]{0.75}FO}%
\colorbox{green}{\color[gray]{0.75}FO}%
\colorbox{green}{\color[gray]{0.75}FO}%
\colorbox{green}{\color[gray]{0.75}FO}%
\colorbox{green}{\color[gray]{0.75}FO}%
\colorbox{green}{\color[gray]{0.75}FO}%
\colorbox{green}{\color[gray]{0.75}FO}%
\colorbox{green}{\color[gray]{0.75}FO}%
\colorbox{green}{\color[gray]{0.75}FO}%
\colorbox{green}{\color[gray]{0.75}FO}%
\colorbox{green}{\color[gray]{0.75}FO}%
\colorbox{green}{\color[gray]{0.75}FO}%
\colorbox{green}{\color[gray]{0.75}FO}%
\colorbox{green}{\color[gray]{0.75}FO}%
\colorbox{green}{\color[gray]{0.75}FO}%
\colorbox{green}{\color[gray]{0.75}FO}%
\colorbox{green}{\color[gray]{0.75}FO}%
\colorbox{green}{\color[gray]{0.75}FO}%
\colorbox{green}{\color[gray]{0.75}FO}%
\colorbox{green}{\color[gray]{0.75}FO}%
\colorbox{green}{\color[gray]{0.75}FO}%
\colorbox{green}{\color[gray]{0.75}FO}%
\colorbox{green}{\color[gray]{0.75}FO}%
\colorbox{green}{\color[gray]{0.75}FO}%
\colorbox{green}{\color[gray]{0.75}FO}%
\colorbox{green}{\color[gray]{0.75}FO}%
\colorbox{green}{\color[gray]{0.75}FO}%
\colorbox{green}{\color[gray]{0.75}FO}%
\colorbox{green}{\color[gray]{0.75}FO}%
\colorbox{green}{\color[gray]{0.75}FO}%
\colorbox{green}{\color[gray]{0.75}FO}%
\colorbox{green}{\color[gray]{0.75}FO}%
\colorbox{green}{\color[gray]{0.75}FO}%
\colorbox{green}{\color[gray]{0.75}FO}%
\colorbox{green}{\color[gray]{0.75}FO}%
\colorbox{green}{\color[gray]{0.75}FO}%
\colorbox{green}{\color[gray]{0.75}FO}%
\colorbox{green}{\color[gray]{0.75}FO}%
\colorbox{green}{\color[gray]{0.75}FO}%
\colorbox{green}{\color[gray]{0.75}FO}%
\colorbox{green}{\color[gray]{0.75}FO}%
\colorbox{green}{\color[gray]{0.75}FO}%
\colorbox{green}{\color[gray]{0.75}FO}%
\colorbox{green}{\color[gray]{0.75}FO}%
\colorbox{green}{\color[gray]{0.75}FO}%
\colorbox{green}{\color[gray]{0.75}FO}%
\colorbox{green}{\color[gray]{0.75}FO}%
\colorbox{green}{\color[gray]{0.75}FO}%
\colorbox{green}{\color[gray]{0.75}FO}%
\colorbox{green}{\color[gray]{0.75}FO}%
\colorbox{green}{\color[gray]{0.75}FO}%
\colorbox{green}{\color[gray]{0.75}FO}%
\colorbox{green}{\color[gray]{0.75}FO}%
\colorbox{green}{\color[gray]{0.75}FO}%
\colorbox{green}{\color[gray]{0.75}FO}%
\colorbox{green}{\color[gray]{0.75}FO}%
\colorbox{green}{\color[gray]{0.75}FO}%
\colorbox{green}{\color[gray]{0.75}FO}%
\colorbox{green}{\color[gray]{0.75}FO}%
\\
\colorbox{green}{\color[gray]{0.75}FO}%
\colorbox{green}{\color[gray]{0.75}FO}%
\colorbox{green}{\color[gray]{0.75}FO}%
\colorbox{green}{\color[gray]{0.75}FO}%
\colorbox{green}{\color[gray]{0.75}FO}%
\colorbox{green}{\color[gray]{0.75}FO}%
\colorbox{green}{\color[gray]{0.75}FO}%
\colorbox{green}{\color[gray]{0.75}FO}%
\colorbox{green}{\color[gray]{0.75}FO}%
\colorbox{green}{\color[gray]{0.75}FO}%
\colorbox{green}{\color[gray]{0.75}FO}%
\colorbox{green}{\color[gray]{0.75}FO}%
\colorbox{green}{\color[gray]{0.75}FO}%
\colorbox{green}{\color[gray]{0.75}FO}%
\colorbox{green}{\color[gray]{0.75}FO}%
\colorbox{green}{\color[gray]{0.75}FO}%
\colorbox{green}{\color[gray]{0.75}FO}%
\colorbox{green}{\color[gray]{0.75}FO}%
\colorbox{green}{\color[gray]{0.75}FO}%
\colorbox{green}{\color[gray]{0.75}FO}%
\colorbox{green}{\color[gray]{0.75}FO}%
\colorbox{green}{\color[gray]{0.75}FO}%
\colorbox{green}{\color[gray]{0.75}FO}%
\colorbox{green}{\color[gray]{0.75}FO}%
\colorbox{green}{\color[gray]{0.75}FO}%
\colorbox{green}{\color[gray]{0.75}FO}%
\colorbox{green}{\color[gray]{0.75}FO}%
\colorbox{green}{\color[gray]{0.75}FO}%
\colorbox{green}{\color[gray]{0.75}FO}%
\colorbox{green}{\color[gray]{0.75}FO}%
\colorbox{green}{\color[gray]{0.75}FO}%
\colorbox{green}{\color[gray]{0.75}FO}%
\colorbox{green}{\color[gray]{0.75}FO}%
\colorbox{green}{\color[gray]{0.75}FO}%
\colorbox{green}{\color[gray]{0.75}FO}%
\colorbox{green}{\color[gray]{0.75}FO}%
\colorbox{green}{\color[gray]{0.75}FO}%
\colorbox{green}{\color[gray]{0.75}FO}%
\colorbox{green}{\color[gray]{0.75}FO}%
\colorbox{green}{\color[gray]{0.75}FO}%
\colorbox{green}{\color[gray]{0.75}FO}%
\colorbox{green}{\color[gray]{0.75}FO}%
\colorbox{green}{\color[gray]{0.75}FO}%
\colorbox{green}{\color[gray]{0.75}FO}%
\colorbox{green}{\color[gray]{0.75}FO}%
\colorbox{green}{\color[gray]{0.75}FO}%
\colorbox{green}{\color[gray]{0.75}FO}%
\colorbox{green}{\color[gray]{0.75}FO}%
\colorbox{green}{\color[gray]{0.75}FO}%
\colorbox{green}{\color[gray]{0.75}FO}%
\colorbox{green}{\color[gray]{0.75}FO}%
\colorbox{green}{\color[gray]{0.75}FO}%
\colorbox{green}{\color[gray]{0.75}FO}%
\colorbox{green}{\color[gray]{0.75}FO}%
\colorbox{green}{\color[gray]{0.75}FO}%
\colorbox{green}{\color[gray]{0.75}FO}%
\colorbox{green}{\color[gray]{0.75}FO}%
\colorbox{green}{\color[gray]{0.75}FO}%
\colorbox{green}{\color[gray]{0.75}FO}%
\colorbox{green}{\color[gray]{0.75}FO}%
\colorbox{green}{\color[gray]{0.75}FO}%
\colorbox{green}{\color[gray]{0.75}FO}%
\colorbox{green}{\color[gray]{0.75}FO}%
\colorbox{green}{\color[gray]{0.75}FO}%
\colorbox{green}{\color[gray]{0.75}FO}%
\colorbox{green}{\color[gray]{0.75}FO}%
\colorbox{green}{\color[gray]{0.75}FO}%
\colorbox{green}{\color[gray]{0.75}FO}%
\colorbox{green}{\color[gray]{0.75}FO}%
\colorbox{green}{\color[gray]{0.75}FO}%
\colorbox{green}{\color[gray]{0.75}FO}%
\colorbox{green}{\color[gray]{0.75}FO}%
\colorbox{green}{\color[gray]{0.75}FO}%
\colorbox{green}{\color[gray]{0.75}FO}%
\colorbox{green}{\color[gray]{0.75}FO}%
\colorbox{green}{\color[gray]{0.75}FO}%
\colorbox{green}{\color[gray]{0.75}FO}%
\colorbox{green}{\color[gray]{0.75}FO}%
\colorbox{green}{\color[gray]{0.75}FO}%
\colorbox{green}{\color[gray]{0.75}FO}%
\colorbox{green}{\color[gray]{0.75}FO}%
\colorbox{green}{\color[gray]{0.75}FO}%
\colorbox{green}{\color[gray]{0.75}FO}%
\colorbox{green}{\color[gray]{0.75}FO}%
\colorbox{green}{\color[gray]{0.75}FO}%
\colorbox{green}{\color[gray]{0.75}FO}%
\colorbox{green}{\color[gray]{0.75}FO}%
\colorbox{green}{\color[gray]{0.75}FO}%
\colorbox{green}{\color[gray]{0.75}FO}%
\colorbox{green}{\color[gray]{0.75}FO}%
\colorbox{green}{\color[gray]{0.75}FO}%
\colorbox{green}{\color[gray]{0.75}FO}%
\colorbox{green}{\color[gray]{0.75}FO}%
\colorbox{green}{\color[gray]{0.75}FO}%
\colorbox{green}{\color[gray]{0.75}FO}%
\colorbox{green}{\color[gray]{0.75}FO}%
\colorbox{green}{\color[gray]{0.75}FO}%
\colorbox{green}{\color[gray]{0.75}FO}%
\colorbox{green}{\color[gray]{0.75}FO}%
\colorbox{green}{\color[gray]{0.75}FO}%
\\
\colorbox{green}{\color[gray]{0.75}FO}%
\colorbox{green}{\color[gray]{0.75}FO}%
\colorbox{green}{\color[gray]{0.75}FO}%
\colorbox{green}{\color[gray]{0.75}FO}%
\colorbox{green}{\color[gray]{0.75}FO}%
\colorbox{green}{\color[gray]{0.75}FO}%
\colorbox{green}{\color[gray]{0.75}FO}%
\colorbox{green}{\color[gray]{0.75}FO}%
\colorbox{green}{\color[gray]{0.75}FO}%
\colorbox{green}{\color[gray]{0.75}FO}%
\colorbox{green}{\color[gray]{0.75}FO}%
\colorbox{green}{\color[gray]{0.75}FO}%
\colorbox{green}{\color[gray]{0.75}FO}%
\colorbox{green}{\color[gray]{0.75}FO}%
\colorbox{green}{\color[gray]{0.75}FO}%
\colorbox{green}{\color[gray]{0.75}FO}%
\colorbox{green}{\color[gray]{0.75}FO}%
\colorbox{green}{\color[gray]{0.75}FO}%
\colorbox{green}{\color[gray]{0.75}FO}%
\colorbox{green}{\color[gray]{0.75}FO}%
\colorbox{green}{\color[gray]{0.75}FO}%
\colorbox{green}{\color[gray]{0.75}FO}%
\colorbox{green}{\color[gray]{0.75}FO}%
\colorbox{green}{\color[gray]{0.75}FO}%
\colorbox{green}{\color[gray]{0.75}FO}%
\colorbox{green}{\color[gray]{0.75}FO}%
\colorbox{green}{\color[gray]{0.75}FO}%
\colorbox{green}{\color[gray]{0.75}FO}%
\colorbox{green}{\color[gray]{0.75}FO}%
\colorbox{green}{\color[gray]{0.75}FO}%
\colorbox{green}{\color[gray]{0.75}FO}%
\colorbox{green}{\color[gray]{0.75}FO}%
\colorbox{green}{\color[gray]{0.75}FO}%
\colorbox{green}{\color[gray]{0.75}FO}%
\colorbox{green}{\color[gray]{0.75}FO}%
\colorbox{green}{\color[gray]{0.75}FO}%
\colorbox{green}{\color[gray]{0.75}FO}%
\colorbox{green}{\color[gray]{0.75}FO}%
\colorbox{green}{\color[gray]{0.75}FO}%
\colorbox{green}{\color[gray]{0.75}FO}%
\colorbox{green}{\color[gray]{0.75}FO}%
\colorbox{green}{\color[gray]{0.75}FO}%
\colorbox{green}{\color[gray]{0.75}FO}%
\colorbox{green}{\color[gray]{0.75}FO}%
\colorbox{green}{\color[gray]{0.75}FO}%
\colorbox{green}{\color[gray]{0.75}FO}%
\colorbox{green}{\color[gray]{0.75}FO}%
\colorbox{green}{\color[gray]{0.75}FO}%
\colorbox{green}{\color[gray]{0.75}FO}%
\colorbox{green}{\color[gray]{0.75}FO}%
\colorbox{green}{\color[gray]{0.75}FO}%
\colorbox{green}{\color[gray]{0.75}FO}%
\colorbox{green}{\color[gray]{0.75}FO}%
\colorbox{green}{\color[gray]{0.75}FO}%
\colorbox{green}{\color[gray]{0.75}FO}%
\colorbox{green}{\color[gray]{0.75}FO}%
\colorbox{green}{\color[gray]{0.75}FO}%
\colorbox{green}{\color[gray]{0.75}FO}%
\colorbox{green}{\color[gray]{0.75}FO}%
\colorbox{green}{\color[gray]{0.75}FO}%
\colorbox{green}{\color[gray]{0.75}FO}%
\colorbox{green}{\color[gray]{0.75}FO}%
\colorbox{green}{\color[gray]{0.75}FO}%
\colorbox{green}{\color[gray]{0.75}FO}%
\colorbox{green}{\color[gray]{0.75}FO}%
\colorbox{green}{\color[gray]{0.75}FO}%
\colorbox{green}{\color[gray]{0.75}FO}%
\colorbox{green}{\color[gray]{0.75}FO}%
\colorbox{green}{\color[gray]{0.75}FO}%
\colorbox{green}{\color[gray]{0.75}FO}%
\colorbox{green}{\color[gray]{0.75}FO}%
\colorbox{green}{\color[gray]{0.75}FO}%
\colorbox{green}{\color[gray]{0.75}FO}%
\colorbox{green}{\color[gray]{0.75}FO}%
\colorbox{green}{\color[gray]{0.75}FO}%
\colorbox{green}{\color[gray]{0.75}FO}%
\colorbox{green}{\color[gray]{0.75}FO}%
\colorbox{green}{\color[gray]{0.75}FO}%
\colorbox{green}{\color[gray]{0.75}FO}%
\colorbox{green}{\color[gray]{0.75}FO}%
\colorbox{green}{\color[gray]{0.75}FO}%
\colorbox{green}{\color[gray]{0.75}FO}%
\colorbox{green}{\color[gray]{0.75}FO}%
\colorbox{green}{\color[gray]{0.75}FO}%
\colorbox{green}{\color[gray]{0.75}FO}%
\colorbox{green}{\color[gray]{0.75}FO}%
\colorbox{green}{\color[gray]{0.75}FO}%
\colorbox{green}{\color[gray]{0.75}FO}%
\colorbox{green}{\color[gray]{0.75}FO}%
\colorbox{green}{\color[gray]{0.75}FO}%
\colorbox{green}{\color[gray]{0.75}FO}%
\colorbox{green}{\color[gray]{0.75}FO}%
\colorbox{green}{\color[gray]{0.75}FO}%
\colorbox{green}{\color[gray]{0.75}FO}%
\colorbox{green}{\color[gray]{0.75}FO}%
\colorbox{green}{\color[gray]{0.75}FO}%
\colorbox{green}{\color[gray]{0.75}FO}%
\colorbox{green}{\color[gray]{0.75}FO}%
\colorbox{green}{\color[gray]{0.75}FO}%
\colorbox{green}{\color[gray]{0.75}FO}%
\\
\colorbox{green}{\color[gray]{0.75}FO}%
\colorbox{green}{\color[gray]{0.75}FO}%
\colorbox{green}{\color[gray]{0.75}FO}%
\colorbox{green}{\color[gray]{0.75}FO}%
\colorbox{green}{\color[gray]{0.75}FO}%
\colorbox{green}{\color[gray]{0.75}FO}%
\colorbox{green}{\color[gray]{0.75}FO}%
\colorbox{green}{\color[gray]{0.75}FO}%
\colorbox{green}{\color[gray]{0.75}FO}%
\colorbox{green}{\color[gray]{0.75}FO}%
\colorbox{green}{\color[gray]{0.75}FO}%
\colorbox{green}{\color[gray]{0.75}FO}%
\colorbox{green}{\color[gray]{0.75}FO}%
\colorbox{green}{\color[gray]{0.75}FO}%
\colorbox{green}{\color[gray]{0.75}FO}%
\colorbox{green}{\color[gray]{0.75}FO}%
\colorbox{green}{\color[gray]{0.75}FO}%
\colorbox{green}{\color[gray]{0.75}FO}%
\colorbox{green}{\color[gray]{0.75}FO}%
\colorbox{green}{\color[gray]{0.75}FO}%
\colorbox{green}{\color[gray]{0.75}FO}%
\colorbox{green}{\color[gray]{0.75}FO}%
\colorbox{green}{\color[gray]{0.75}FO}%
\colorbox{green}{\color[gray]{0.75}FO}%
\colorbox{green}{\color[gray]{0.75}FO}%
\colorbox{green}{\color[gray]{0.75}FO}%
\colorbox{green}{\color[gray]{0.75}FO}%
\colorbox{green}{\color[gray]{0.75}FO}%
\colorbox{green}{\color[gray]{0.75}FO}%
\colorbox{green}{\color[gray]{0.75}FO}%
\colorbox{green}{\color[gray]{0.75}FO}%
\colorbox{green}{\color[gray]{0.75}FO}%
\colorbox{green}{\color[gray]{0.75}FO}%
\colorbox{green}{\color[gray]{0.75}FO}%
\colorbox{green}{\color[gray]{0.75}FO}%
\colorbox{green}{\color[gray]{0.75}FO}%
\colorbox{green}{\color[gray]{0.75}FO}%
\colorbox{green}{\color[gray]{0.75}FO}%
\colorbox{green}{\color[gray]{0.75}FO}%
\colorbox{green}{\color[gray]{0.75}FO}%
\colorbox{green}{\color[gray]{0.75}FO}%
\colorbox{green}{\color[gray]{0.75}FO}%
\colorbox{green}{\color[gray]{0.75}FO}%
\colorbox{green}{\color[gray]{0.75}FO}%
\colorbox{green}{\color[gray]{0.75}FO}%
\colorbox{green}{\color[gray]{0.75}FO}%
\colorbox{green}{\color[gray]{0.75}FO}%
\colorbox{green}{\color[gray]{0.75}FO}%
\colorbox{green}{\color[gray]{0.75}FO}%
\colorbox{green}{\color[gray]{0.75}FO}%
\colorbox{green}{\color[gray]{0.75}FO}%
\colorbox{green}{\color[gray]{0.75}FO}%
\colorbox{green}{\color[gray]{0.75}FO}%
\colorbox{green}{\color[gray]{0.75}FO}%
\colorbox{green}{\color[gray]{0.75}FO}%
\colorbox{green}{\color[gray]{0.75}FO}%
\colorbox{green}{\color[gray]{0.75}FO}%
\colorbox{green}{\color[gray]{0.75}FO}%
\colorbox{green}{\color[gray]{0.75}FO}%
\colorbox{green}{\color[gray]{0.75}FO}%
\colorbox{green}{\color[gray]{0.75}FO}%
\colorbox{green}{\color[gray]{0.75}FO}%
\colorbox{green}{\color[gray]{0.75}FO}%
\colorbox{green}{\color[gray]{0.75}FO}%
\colorbox{green}{\color[gray]{0.75}FO}%
\colorbox{green}{\color[gray]{0.75}FO}%
\colorbox{green}{\color[gray]{0.75}FO}%
\colorbox{green}{\color[gray]{0.75}FO}%
\colorbox{green}{\color[gray]{0.75}FO}%
\colorbox{green}{\color[gray]{0.75}FO}%
\colorbox{green}{\color[gray]{0.75}FO}%
\colorbox{green}{\color[gray]{0.75}FO}%
\colorbox{green}{\color[gray]{0.75}FO}%
\colorbox{green}{\color[gray]{0.75}FO}%
\colorbox{green}{\color[gray]{0.75}FO}%
\colorbox{green}{\color[gray]{0.75}FO}%
\colorbox{green}{\color[gray]{0.75}FO}%
\colorbox{green}{\color[gray]{0.75}FO}%
\colorbox{green}{\color[gray]{0.75}FO}%
\colorbox{green}{\color[gray]{0.75}FO}%
\colorbox{green}{\color[gray]{0.75}FO}%
\colorbox{green}{\color[gray]{0.75}FO}%
\colorbox{green}{\color[gray]{0.75}FO}%
\colorbox{green}{\color[gray]{0.75}FO}%
\colorbox{green}{\color[gray]{0.75}FO}%
\colorbox{green}{\color[gray]{0.75}FO}%
\colorbox{green}{\color[gray]{0.75}FO}%
\colorbox{green}{\color[gray]{0.75}FO}%
\colorbox{green}{\color[gray]{0.75}FO}%
\colorbox{green}{\color[gray]{0.75}FO}%
\colorbox{green}{\color[gray]{0.75}FO}%
\colorbox{green}{\color[gray]{0.75}FO}%
\colorbox{green}{\color[gray]{0.75}FO}%
\colorbox{green}{\color[gray]{0.75}FO}%
\colorbox{green}{\color[gray]{0.75}FO}%
\colorbox{green}{\color[gray]{0.75}FO}%
\colorbox{green}{\color[gray]{0.75}FO}%
\colorbox{green}{\color[gray]{0.75}FO}%
\colorbox{green}{\color[gray]{0.75}FO}%
\colorbox{green}{\color[gray]{0.75}FO}%
\\
\colorbox{green}{\color[gray]{0.75}FO}%
\colorbox{green}{\color[gray]{0.75}FO}%
\colorbox{green}{\color[gray]{0.75}FO}%
\colorbox{green}{\color[gray]{0.75}FO}%
\colorbox{green}{\color[gray]{0.75}FO}%
\colorbox{green}{\color[gray]{0.75}FO}%
\colorbox{green}{\color[gray]{0.75}FO}%
\colorbox{green}{\color[gray]{0.75}FO}%
\colorbox{green}{\color[gray]{0.75}FO}%
\colorbox{green}{\color[gray]{0.75}FO}%
\colorbox{green}{\color[gray]{0.75}FO}%
\colorbox{green}{\color[gray]{0.75}FO}%
\colorbox{green}{\color[gray]{0.75}FO}%
\colorbox{green}{\color[gray]{0.75}FO}%
\colorbox{green}{\color[gray]{0.75}FO}%
\colorbox{green}{\color[gray]{0.75}FO}%
\colorbox{green}{\color[gray]{0.75}FO}%
\colorbox{green}{\color[gray]{0.75}FO}%
\colorbox{green}{\color[gray]{0.75}FO}%
\colorbox{green}{\color[gray]{0.75}FO}%
\colorbox{green}{\color[gray]{0.75}FO}%
\colorbox{green}{\color[gray]{0.75}FO}%
\colorbox{green}{\color[gray]{0.75}FO}%
\colorbox{green}{\color[gray]{0.75}FO}%
\colorbox{green}{\color[gray]{0.75}FO}%
\colorbox{green}{\color[gray]{0.75}FO}%
\colorbox{green}{\color[gray]{0.75}FO}%
\colorbox{green}{\color[gray]{0.75}FO}%
\colorbox{green}{\color[gray]{0.75}FO}%
\colorbox{green}{\color[gray]{0.75}FO}%
\colorbox{green}{\color[gray]{0.75}FO}%
\colorbox{green}{\color[gray]{0.75}FO}%
\colorbox{green}{\color[gray]{0.75}FO}%
\colorbox{green}{\color[gray]{0.75}FO}%
\colorbox{green}{\color[gray]{0.75}FO}%
\colorbox{green}{\color[gray]{0.75}FO}%
\colorbox{green}{\color[gray]{0.75}FO}%
\colorbox{green}{\color[gray]{0.75}FO}%
\colorbox{green}{\color[gray]{0.75}FO}%
\colorbox{green}{\color[gray]{0.75}FO}%
\colorbox{green}{\color[gray]{0.75}FO}%
\colorbox{green}{\color[gray]{0.75}FO}%
\colorbox{green}{\color[gray]{0.75}FO}%
\colorbox{green}{\color[gray]{0.75}FO}%
\colorbox{green}{\color[gray]{0.75}FO}%
\colorbox{green}{\color[gray]{0.75}FO}%
\colorbox{green}{\color[gray]{0.75}FO}%
\colorbox{green}{\color[gray]{0.75}FO}%
\colorbox{green}{\color[gray]{0.75}FO}%
\colorbox{green}{\color[gray]{0.75}FO}%
\colorbox{green}{\color[gray]{0.75}FO}%
\colorbox{green}{\color[gray]{0.75}FO}%
\colorbox{green}{\color[gray]{0.75}FO}%
\colorbox{green}{\color[gray]{0.75}FO}%
\colorbox{green}{\color[gray]{0.75}FO}%
\colorbox{green}{\color[gray]{0.75}FO}%
\colorbox{green}{\color[gray]{0.75}FO}%
\colorbox{green}{\color[gray]{0.75}FO}%
\colorbox{green}{\color[gray]{0.75}FO}%
\colorbox{green}{\color[gray]{0.75}FO}%
\colorbox{green}{\color[gray]{0.75}FO}%
\colorbox{green}{\color[gray]{0.75}FO}%
\colorbox{green}{\color[gray]{0.75}FO}%
\colorbox{green}{\color[gray]{0.75}FO}%
\colorbox{green}{\color[gray]{0.75}FO}%
\colorbox{green}{\color[gray]{0.75}FO}%
\colorbox{green}{\color[gray]{0.75}FO}%
\colorbox{green}{\color[gray]{0.75}FO}%
\colorbox{green}{\color[gray]{0.75}FO}%
\colorbox{green}{\color[gray]{0.75}FO}%
\colorbox{green}{\color[gray]{0.75}FO}%
\colorbox{green}{\color[gray]{0.75}FO}%
\colorbox{green}{\color[gray]{0.75}FO}%
\colorbox{green}{\color[gray]{0.75}FO}%
\colorbox{green}{\color[gray]{0.75}FO}%
\colorbox{green}{\color[gray]{0.75}FO}%
\colorbox{green}{\color[gray]{0.75}FO}%
\colorbox{green}{\color[gray]{0.75}FO}%
\colorbox{green}{\color[gray]{0.75}FO}%
\colorbox{green}{\color[gray]{0.75}FO}%
\colorbox{green}{\color[gray]{0.75}FO}%
\colorbox{green}{\color[gray]{0.75}FO}%
\colorbox{green}{\color[gray]{0.75}FO}%
\colorbox{green}{\color[gray]{0.75}FO}%
\colorbox{green}{\color[gray]{0.75}FO}%
\colorbox{green}{\color[gray]{0.75}FO}%
\colorbox{green}{\color[gray]{0.75}FO}%
\colorbox{green}{\color[gray]{0.75}FO}%
\colorbox{green}{\color[gray]{0.75}FO}%
\colorbox{green}{\color[gray]{0.75}FO}%
\colorbox{green}{\color[gray]{0.75}FO}%
\colorbox{green}{\color[gray]{0.75}FO}%
\colorbox{green}{\color[gray]{0.75}FO}%
\colorbox{green}{\color[gray]{0.75}FO}%
\colorbox{green}{\color[gray]{0.75}FO}%
\colorbox{green}{\color[gray]{0.75}FO}%
\colorbox{green}{\color[gray]{0.75}FO}%
\colorbox{green}{\color[gray]{0.75}FO}%
\colorbox{green}{\color[gray]{0.75}FO}%
\colorbox{green}{\color[gray]{0.75}FO}%
\\
\colorbox{green}{\color[gray]{0.75}FO}%
\colorbox{green}{\color[gray]{0.75}FO}%
\colorbox{green}{\color[gray]{0.75}FO}%
\colorbox{green}{\color[gray]{0.75}FO}%
\colorbox{green}{\color[gray]{0.75}FO}%
\colorbox{green}{\color[gray]{0.75}FO}%
\colorbox{green}{\color[gray]{0.75}FO}%
\colorbox{green}{\color[gray]{0.75}FO}%
\colorbox{green}{\color[gray]{0.75}FO}%
\colorbox{green}{\color[gray]{0.75}FO}%
\colorbox{green}{\color[gray]{0.75}FO}%
\colorbox{green}{\color[gray]{0.75}FO}%
\colorbox{green}{\color[gray]{0.75}FO}%
\colorbox{green}{\color[gray]{0.75}FO}%
\colorbox{green}{\color[gray]{0.75}FO}%
\colorbox{green}{\color[gray]{0.75}FO}%
\colorbox{green}{\color[gray]{0.75}FO}%
\colorbox{green}{\color[gray]{0.75}FO}%
\colorbox{green}{\color[gray]{0.75}FO}%
\colorbox{green}{\color[gray]{0.75}FO}%
\colorbox{green}{\color[gray]{0.75}FO}%
\colorbox{green}{\color[gray]{0.75}FO}%
\colorbox{green}{\color[gray]{0.75}FO}%
\colorbox{green}{\color[gray]{0.75}FO}%
\colorbox{green}{\color[gray]{0.75}FO}%
\colorbox{green}{\color[gray]{0.75}FO}%
\colorbox{green}{\color[gray]{0.75}FO}%
\colorbox{green}{\color[gray]{0.75}FO}%
\colorbox{green}{\color[gray]{0.75}FO}%
\colorbox{green}{\color[gray]{0.75}FO}%
\colorbox{green}{\color[gray]{0.75}FO}%
\colorbox{green}{\color[gray]{0.75}FO}%
\colorbox{green}{\color[gray]{0.75}FO}%
\colorbox{green}{\color[gray]{0.75}FO}%
\colorbox{green}{\color[gray]{0.75}FO}%
\colorbox{green}{\color[gray]{0.75}FO}%
\colorbox{green}{\color[gray]{0.75}FO}%
\colorbox{green}{\color[gray]{0.75}FO}%
\colorbox{green}{\color[gray]{0.75}FO}%
\colorbox{green}{\color[gray]{0.75}FO}%
\colorbox{green}{\color[gray]{0.75}FO}%
\colorbox{green}{\color[gray]{0.75}FO}%
\colorbox{green}{\color[gray]{0.75}FO}%
\colorbox{green}{\color[gray]{0.75}FO}%
\colorbox{green}{\color[gray]{0.75}FO}%
\colorbox{green}{\color[gray]{0.75}FO}%
\colorbox{green}{\color[gray]{0.75}FO}%
\colorbox{green}{\color[gray]{0.75}FO}%
\colorbox{green}{\color[gray]{0.75}FO}%
\colorbox{green}{\color[gray]{0.75}FO}%
\colorbox{green}{\color[gray]{0.75}FO}%
\colorbox{green}{\color[gray]{0.75}FO}%
\colorbox{green}{\color[gray]{0.75}FO}%
\colorbox{green}{\color[gray]{0.75}FO}%
\colorbox{green}{\color[gray]{0.75}FO}%
\colorbox{green}{\color[gray]{0.75}FO}%
\colorbox{green}{\color[gray]{0.75}FO}%
\colorbox{green}{\color[gray]{0.75}FO}%
\colorbox{green}{\color[gray]{0.75}FO}%
\colorbox{green}{\color[gray]{0.75}FO}%
\colorbox{green}{\color[gray]{0.75}FO}%
\colorbox{green}{\color[gray]{0.75}FO}%
\colorbox{green}{\color[gray]{0.75}FO}%
\colorbox{green}{\color[gray]{0.75}FO}%
\colorbox{green}{\color[gray]{0.75}FO}%
\colorbox{green}{\color[gray]{0.75}FO}%
\colorbox{green}{\color[gray]{0.75}FO}%
\colorbox{green}{\color[gray]{0.75}FO}%
\colorbox{green}{\color[gray]{0.75}FO}%
\colorbox{green}{\color[gray]{0.75}FO}%
\colorbox{green}{\color[gray]{0.75}FO}%
\colorbox{green}{\color[gray]{0.75}FO}%
\colorbox{green}{\color[gray]{0.75}FO}%
\colorbox{green}{\color[gray]{0.75}FO}%
\colorbox{green}{\color[gray]{0.75}FO}%
\colorbox{green}{\color[gray]{0.75}FO}%
\colorbox{green}{\color[gray]{0.75}FO}%
\colorbox{green}{\color[gray]{0.75}FO}%
\colorbox{green}{\color[gray]{0.75}FO}%
\colorbox{green}{\color[gray]{0.75}FO}%
\colorbox{green}{\color[gray]{0.75}FO}%
\colorbox{green}{\color[gray]{0.75}FO}%
\colorbox{green}{\color[gray]{0.75}FO}%
\colorbox{green}{\color[gray]{0.75}FO}%
\colorbox{green}{\color[gray]{0.75}FO}%
\colorbox{green}{\color[gray]{0.75}FO}%
\colorbox{green}{\color[gray]{0.75}FO}%
\colorbox{green}{\color[gray]{0.75}FO}%
\colorbox{green}{\color[gray]{0.75}FO}%
\colorbox{green}{\color[gray]{0.75}FO}%
\colorbox{green}{\color[gray]{0.75}FO}%
\colorbox{green}{\color[gray]{0.75}FO}%
\colorbox{green}{\color[gray]{0.75}FO}%
\colorbox{green}{\color[gray]{0.75}FO}%
\colorbox{green}{\color[gray]{0.75}FO}%
\colorbox{green}{\color[gray]{0.75}FO}%
\colorbox{green}{\color[gray]{0.75}FO}%
\colorbox{green}{\color[gray]{0.75}FO}%
\colorbox{green}{\color[gray]{0.75}FO}%
\colorbox{green}{\color[gray]{0.75}FO}%
\\
\colorbox{green}{\color[gray]{0.75}FO}%
\colorbox{green}{\color[gray]{0.75}FO}%
\colorbox{green}{\color[gray]{0.75}FO}%
\colorbox{green}{\color[gray]{0.75}FO}%
\colorbox{green}{\color[gray]{0.75}FO}%
\colorbox{green}{\color[gray]{0.75}FO}%
\colorbox{green}{\color[gray]{0.75}FO}%
\colorbox{green}{\color[gray]{0.75}FO}%
\colorbox{green}{\color[gray]{0.75}FO}%
\colorbox{green}{\color[gray]{0.75}FO}%
\colorbox{green}{\color[gray]{0.75}FO}%
\colorbox{green}{\color[gray]{0.75}FO}%
\colorbox{green}{\color[gray]{0.75}FO}%
\colorbox{green}{\color[gray]{0.75}FO}%
\colorbox{green}{\color[gray]{0.75}FO}%
\colorbox{green}{\color[gray]{0.75}FO}%
\colorbox{green}{\color[gray]{0.75}FO}%
\colorbox{green}{\color[gray]{0.75}FO}%
\colorbox{green}{\color[gray]{0.75}FO}%
\colorbox{green}{\color[gray]{0.75}FO}%
\colorbox{green}{\color[gray]{0.75}FO}%
\colorbox{green}{\color[gray]{0.75}FO}%
\colorbox{green}{\color[gray]{0.75}FO}%
\colorbox{green}{\color[gray]{0.75}FO}%
\colorbox{green}{\color[gray]{0.75}FO}%
\colorbox{green}{\color[gray]{0.75}FO}%
\colorbox{green}{\color[gray]{0.75}FO}%
\colorbox{green}{\color[gray]{0.75}FO}%
\colorbox{green}{\color[gray]{0.75}FO}%
\colorbox{green}{\color[gray]{0.75}FO}%
\colorbox{green}{\color[gray]{0.75}FO}%
\colorbox{green}{\color[gray]{0.75}FO}%
\colorbox{green}{\color[gray]{0.75}FO}%
\colorbox{green}{\color[gray]{0.75}FO}%
\colorbox{green}{\color[gray]{0.75}FO}%
\colorbox{green}{\color[gray]{0.75}FO}%
\colorbox{green}{\color[gray]{0.75}FO}%
\colorbox{green}{\color[gray]{0.75}FO}%
\colorbox{green}{\color[gray]{0.75}FO}%
\colorbox{green}{\color[gray]{0.75}FO}%
\colorbox{green}{\color[gray]{0.75}FO}%
\colorbox{green}{\color[gray]{0.75}FO}%
\colorbox{green}{\color[gray]{0.75}FO}%
\colorbox{green}{\color[gray]{0.75}FO}%
\colorbox{green}{\color[gray]{0.75}FO}%
\colorbox{green}{\color[gray]{0.75}FO}%
\colorbox{green}{\color[gray]{0.75}FO}%
\colorbox{green}{\color[gray]{0.75}FO}%
\colorbox{green}{\color[gray]{0.75}FO}%
\colorbox{green}{\color[gray]{0.75}FO}%
\colorbox{green}{\color[gray]{0.75}FO}%
\colorbox{green}{\color[gray]{0.75}FO}%
\colorbox{green}{\color[gray]{0.75}FO}%
\colorbox{green}{\color[gray]{0.75}FO}%
\colorbox{green}{\color[gray]{0.75}FO}%
\colorbox{green}{\color[gray]{0.75}FO}%
\colorbox{green}{\color[gray]{0.75}FO}%
\colorbox{green}{\color[gray]{0.75}FO}%
\colorbox{green}{\color[gray]{0.75}FO}%
\colorbox{green}{\color[gray]{0.75}FO}%
\colorbox{green}{\color[gray]{0.75}FO}%
\colorbox{green}{\color[gray]{0.75}FO}%
\colorbox{green}{\color[gray]{0.75}FO}%
\colorbox{green}{\color[gray]{0.75}FO}%
\colorbox{green}{\color[gray]{0.75}FO}%
\colorbox{green}{\color[gray]{0.75}FO}%
\colorbox{green}{\color[gray]{0.75}FO}%
\colorbox{green}{\color[gray]{0.75}FO}%
\colorbox{green}{\color[gray]{0.75}FO}%
\colorbox{green}{\color[gray]{0.75}FO}%
\colorbox{green}{\color[gray]{0.75}FO}%
\colorbox{green}{\color[gray]{0.75}FO}%
\colorbox{green}{\color[gray]{0.75}FO}%
\colorbox{green}{\color[gray]{0.75}FO}%
\colorbox{green}{\color[gray]{0.75}FO}%
\colorbox{green}{\color[gray]{0.75}FO}%
\colorbox{green}{\color[gray]{0.75}FO}%
\colorbox{green}{\color[gray]{0.75}FO}%
\colorbox{green}{\color[gray]{0.75}FO}%
\colorbox{green}{\color[gray]{0.75}FO}%
\colorbox{green}{\color[gray]{0.75}FO}%
\colorbox{green}{\color[gray]{0.75}FO}%
\colorbox{green}{\color[gray]{0.75}FO}%
\colorbox{green}{\color[gray]{0.75}FO}%
\colorbox{green}{\color[gray]{0.75}FO}%
\colorbox{green}{\color[gray]{0.75}FO}%
\colorbox{green}{\color[gray]{0.75}FO}%
\colorbox{green}{\color[gray]{0.75}FO}%
\colorbox{green}{\color[gray]{0.75}FO}%
\colorbox{green}{\color[gray]{0.75}FO}%
\colorbox{green}{\color[gray]{0.75}FO}%
\colorbox{green}{\color[gray]{0.75}FO}%
\colorbox{green}{\color[gray]{0.75}FO}%
\colorbox{green}{\color[gray]{0.75}FO}%
\colorbox{green}{\color[gray]{0.75}FO}%
\colorbox{green}{\color[gray]{0.75}FO}%
\colorbox{green}{\color[gray]{0.75}FO}%
\colorbox{green}{\color[gray]{0.75}FO}%
\colorbox{green}{\color[gray]{0.75}FO}%
\colorbox{green}{\color[gray]{0.75}FO}%
\\
\colorbox{green}{\color[gray]{0.75}FO}%
\colorbox{green}{\color[gray]{0.75}FO}%
\colorbox{green}{\color[gray]{0.75}FO}%
\colorbox{green}{\color[gray]{0.75}FO}%
\colorbox{green}{\color[gray]{0.75}FO}%
\colorbox{green}{\color[gray]{0.75}FO}%
\colorbox{green}{\color[gray]{0.75}FO}%
\colorbox{green}{\color[gray]{0.75}FO}%
\colorbox{green}{\color[gray]{0.75}FO}%
\colorbox{green}{\color[gray]{0.75}FO}%
\colorbox{green}{\color[gray]{0.75}FO}%
\colorbox{green}{\color[gray]{0.75}FO}%
\colorbox{green}{\color[gray]{0.75}FO}%
\colorbox{green}{\color[gray]{0.75}FO}%
\colorbox{green}{\color[gray]{0.75}FO}%
\colorbox{green}{\color[gray]{0.75}FO}%
\colorbox{green}{\color[gray]{0.75}FO}%
\colorbox{green}{\color[gray]{0.75}FO}%
\colorbox{green}{\color[gray]{0.75}FO}%
\colorbox{green}{\color[gray]{0.75}FO}%
\colorbox{green}{\color[gray]{0.75}FO}%
\colorbox{green}{\color[gray]{0.75}FO}%
\colorbox{green}{\color[gray]{0.75}FO}%
\colorbox{green}{\color[gray]{0.75}FO}%
\colorbox{green}{\color[gray]{0.75}FO}%
\colorbox{green}{\color[gray]{0.75}FO}%
\colorbox{green}{\color[gray]{0.75}FO}%
\colorbox{green}{\color[gray]{0.75}FO}%
\colorbox{green}{\color[gray]{0.75}FO}%
\colorbox{green}{\color[gray]{0.75}FO}%
\colorbox{green}{\color[gray]{0.75}FO}%
\colorbox{green}{\color[gray]{0.75}FO}%
\colorbox{green}{\color[gray]{0.75}FO}%
\colorbox{green}{\color[gray]{0.75}FO}%
\colorbox{green}{\color[gray]{0.75}FO}%
\colorbox{green}{\color[gray]{0.75}FO}%
\colorbox{green}{\color[gray]{0.75}FO}%
\colorbox{green}{\color[gray]{0.75}FO}%
\colorbox{green}{\color[gray]{0.75}FO}%
\colorbox{green}{\color[gray]{0.75}FO}%
\colorbox{green}{\color[gray]{0.75}FO}%
\colorbox{green}{\color[gray]{0.75}FO}%
\colorbox{green}{\color[gray]{0.75}FO}%
\colorbox{green}{\color[gray]{0.75}FO}%
\colorbox{green}{\color[gray]{0.75}FO}%
\colorbox{green}{\color[gray]{0.75}FO}%
\colorbox{green}{\color[gray]{0.75}FO}%
\colorbox{green}{\color[gray]{0.75}FO}%
\colorbox{green}{\color[gray]{0.75}FO}%
\colorbox{green}{\color[gray]{0.75}FO}%
\colorbox{green}{\color[gray]{0.75}FO}%
\colorbox{green}{\color[gray]{0.75}FO}%
\colorbox{green}{\color[gray]{0.75}FO}%
\colorbox{green}{\color[gray]{0.75}FO}%
\colorbox{green}{\color[gray]{0.75}FO}%
\colorbox{green}{\color[gray]{0.75}FO}%
\colorbox{green}{\color[gray]{0.75}FO}%
\colorbox{green}{\color[gray]{0.75}FO}%
\colorbox{green}{\color[gray]{0.75}FO}%
\colorbox{green}{\color[gray]{0.75}FO}%
\colorbox{green}{\color[gray]{0.75}FO}%
\colorbox{green}{\color[gray]{0.75}FO}%
\colorbox{green}{\color[gray]{0.75}FO}%
\colorbox{green}{\color[gray]{0.75}FO}%
\colorbox{green}{\color[gray]{0.75}FO}%
\colorbox{green}{\color[gray]{0.75}FO}%
\colorbox{green}{\color[gray]{0.75}FO}%
\colorbox{green}{\color[gray]{0.75}FO}%
\colorbox{green}{\color[gray]{0.75}FO}%
\colorbox{green}{\color[gray]{0.75}FO}%
\colorbox{green}{\color[gray]{0.75}FO}%
\colorbox{green}{\color[gray]{0.75}FO}%
\colorbox{green}{\color[gray]{0.75}FO}%
\colorbox{green}{\color[gray]{0.75}FO}%
\colorbox{green}{\color[gray]{0.75}FO}%
\colorbox{green}{\color[gray]{0.75}FO}%
\colorbox{green}{\color[gray]{0.75}FO}%
\colorbox{green}{\color[gray]{0.75}FO}%
\colorbox{green}{\color[gray]{0.75}FO}%
\colorbox{green}{\color[gray]{0.75}FO}%
\colorbox{green}{\color[gray]{0.75}FO}%
\colorbox{green}{\color[gray]{0.75}FO}%
\colorbox{green}{\color[gray]{0.75}FO}%
\colorbox{green}{\color[gray]{0.75}FO}%
\colorbox{green}{\color[gray]{0.75}FO}%
\colorbox{green}{\color[gray]{0.75}FO}%
\colorbox{green}{\color[gray]{0.75}FO}%
\colorbox{green}{\color[gray]{0.75}FO}%
\colorbox{green}{\color[gray]{0.75}FO}%
\colorbox{green}{\color[gray]{0.75}FO}%
\colorbox{green}{\color[gray]{0.75}FO}%
\colorbox{green}{\color[gray]{0.75}FO}%
\colorbox{green}{\color[gray]{0.75}FO}%
\colorbox{green}{\color[gray]{0.75}FO}%
\colorbox{green}{\color[gray]{0.75}FO}%
\colorbox{green}{\color[gray]{0.75}FO}%
\colorbox{green}{\color[gray]{0.75}FO}%
\colorbox{green}{\color[gray]{0.75}FO}%
\colorbox{green}{\color[gray]{0.75}FO}%
\colorbox{green}{\color[gray]{0.75}FO}%
\\
\colorbox{green}{\color[gray]{0.75}FO}%
\colorbox{green}{\color[gray]{0.75}FO}%
\colorbox{green}{\color[gray]{0.75}FO}%
\colorbox{green}{\color[gray]{0.75}FO}%
\colorbox{green}{\color[gray]{0.75}FO}%
\colorbox{green}{\color[gray]{0.75}FO}%
\colorbox{green}{\color[gray]{0.75}FO}%
\colorbox{green}{\color[gray]{0.75}FO}%
\colorbox{green}{\color[gray]{0.75}FO}%
\colorbox{green}{\color[gray]{0.75}FO}%
\colorbox{green}{\color[gray]{0.75}FO}%
\colorbox{green}{\color[gray]{0.75}FO}%
\colorbox{green}{\color[gray]{0.75}FO}%
\colorbox{green}{\color[gray]{0.75}FO}%
\colorbox{green}{\color[gray]{0.75}FO}%
\colorbox{green}{\color[gray]{0.75}FO}%
\colorbox{green}{\color[gray]{0.75}FO}%
\colorbox{green}{\color[gray]{0.75}FO}%
\colorbox{green}{\color[gray]{0.75}FO}%
\colorbox{green}{\color[gray]{0.75}FO}%
\colorbox{green}{\color[gray]{0.75}FO}%
\colorbox{green}{\color[gray]{0.75}FO}%
\colorbox{green}{\color[gray]{0.75}FO}%
\colorbox{green}{\color[gray]{0.75}FO}%
\colorbox{green}{\color[gray]{0.75}FO}%
\colorbox{green}{\color[gray]{0.75}FO}%
\colorbox{green}{\color[gray]{0.75}FO}%
\colorbox{green}{\color[gray]{0.75}FO}%
\colorbox{green}{\color[gray]{0.75}FO}%
\colorbox{green}{\color[gray]{0.75}FO}%
\colorbox{green}{\color[gray]{0.75}FO}%
\colorbox{green}{\color[gray]{0.75}FO}%
\colorbox{green}{\color[gray]{0.75}FO}%
\colorbox{green}{\color[gray]{0.75}FO}%
\colorbox{green}{\color[gray]{0.75}FO}%
\colorbox{green}{\color[gray]{0.75}FO}%
\colorbox{green}{\color[gray]{0.75}FO}%
\colorbox{green}{\color[gray]{0.75}FO}%
\colorbox{green}{\color[gray]{0.75}FO}%
\colorbox{green}{\color[gray]{0.75}FO}%
\colorbox{green}{\color[gray]{0.75}FO}%
\colorbox{green}{\color[gray]{0.75}FO}%
\colorbox{green}{\color[gray]{0.75}FO}%
\colorbox{green}{\color[gray]{0.75}FO}%
\colorbox{green}{\color[gray]{0.75}FO}%
\colorbox{green}{\color[gray]{0.75}FO}%
\colorbox{green}{\color[gray]{0.75}FO}%
\colorbox{green}{\color[gray]{0.75}FO}%
\colorbox{green}{\color[gray]{0.75}FO}%
\colorbox{green}{\color[gray]{0.75}FO}%
\colorbox{green}{\color[gray]{0.75}FO}%
\colorbox{green}{\color[gray]{0.75}FO}%
\colorbox{green}{\color[gray]{0.75}FO}%
\colorbox{green}{\color[gray]{0.75}FO}%
\colorbox{green}{\color[gray]{0.75}FO}%
\colorbox{green}{\color[gray]{0.75}FO}%
\colorbox{green}{\color[gray]{0.75}FO}%
\colorbox{green}{\color[gray]{0.75}FO}%
\colorbox{green}{\color[gray]{0.75}FO}%
\colorbox{green}{\color[gray]{0.75}FO}%
\colorbox{green}{\color[gray]{0.75}FO}%
\colorbox{green}{\color[gray]{0.75}FO}%
\colorbox{green}{\color[gray]{0.75}FO}%
\colorbox{green}{\color[gray]{0.75}FO}%
\colorbox{green}{\color[gray]{0.75}FO}%
\colorbox{green}{\color[gray]{0.75}FO}%
\colorbox{green}{\color[gray]{0.75}FO}%
\colorbox{green}{\color[gray]{0.75}FO}%
\colorbox{green}{\color[gray]{0.75}FO}%
\colorbox{green}{\color[gray]{0.75}FO}%
\colorbox{green}{\color[gray]{0.75}FO}%
\colorbox{green}{\color[gray]{0.75}FO}%
\colorbox{green}{\color[gray]{0.75}FO}%
\colorbox{green}{\color[gray]{0.75}FO}%
\colorbox{green}{\color[gray]{0.75}FO}%
\colorbox{green}{\color[gray]{0.75}FO}%
\colorbox{green}{\color[gray]{0.75}FO}%
\colorbox{green}{\color[gray]{0.75}FO}%
\colorbox{green}{\color[gray]{0.75}FO}%
\colorbox{green}{\color[gray]{0.75}FO}%
\colorbox{green}{\color[gray]{0.75}FO}%
\colorbox{green}{\color[gray]{0.75}FO}%
\colorbox{green}{\color[gray]{0.75}FO}%
\colorbox{green}{\color[gray]{0.75}FO}%
\colorbox{green}{\color[gray]{0.75}FO}%
\colorbox{green}{\color[gray]{0.75}FO}%
\colorbox{green}{\color[gray]{0.75}FO}%
\colorbox{green}{\color[gray]{0.75}FO}%
\colorbox{green}{\color[gray]{0.75}FO}%
\colorbox{green}{\color[gray]{0.75}FO}%
\colorbox{green}{\color[gray]{0.75}FO}%
\colorbox{green}{\color[gray]{0.75}FO}%
\colorbox{green}{\color[gray]{0.75}FO}%
\colorbox{green}{\color[gray]{0.75}FO}%
\colorbox{green}{\color[gray]{0.75}FO}%
\colorbox{green}{\color[gray]{0.75}FO}%
\colorbox{green}{\color[gray]{0.75}FO}%
\colorbox{green}{\color[gray]{0.75}FO}%
\colorbox{green}{\color[gray]{0.75}FO}%
\colorbox{green}{\color[gray]{0.75}FO}%
\\
\colorbox{green}{\color[gray]{0.75}FO}%
\colorbox{green}{\color[gray]{0.75}FO}%
\colorbox{green}{\color[gray]{0.75}FO}%
\colorbox{green}{\color[gray]{0.75}FO}%
\colorbox{green}{\color[gray]{0.75}FO}%
\colorbox{green}{\color[gray]{0.75}FO}%
\colorbox{green}{\color[gray]{0.75}FO}%
\colorbox{green}{\color[gray]{0.75}FO}%
\colorbox{green}{\color[gray]{0.75}FO}%
\colorbox{green}{\color[gray]{0.75}FO}%
\colorbox{green}{\color[gray]{0.75}FO}%
\colorbox{green}{\color[gray]{0.75}FO}%
\colorbox{green}{\color[gray]{0.75}FO}%
\colorbox{green}{\color[gray]{0.75}FO}%
\colorbox{green}{\color[gray]{0.75}FO}%
\colorbox{green}{\color[gray]{0.75}FO}%
\colorbox{green}{\color[gray]{0.75}FO}%
\colorbox{green}{\color[gray]{0.75}FO}%
\colorbox{green}{\color[gray]{0.75}FO}%
\colorbox{green}{\color[gray]{0.75}FO}%
\colorbox{green}{\color[gray]{0.75}FO}%
\colorbox{green}{\color[gray]{0.75}FO}%
\colorbox{green}{\color[gray]{0.75}FO}%
\colorbox{green}{\color[gray]{0.75}FO}%
\colorbox{green}{\color[gray]{0.75}FO}%
\colorbox{green}{\color[gray]{0.75}FO}%
\colorbox{green}{\color[gray]{0.75}FO}%
\colorbox{green}{\color[gray]{0.75}FO}%
\colorbox{green}{\color[gray]{0.75}FO}%
\colorbox{green}{\color[gray]{0.75}FO}%
\colorbox{green}{\color[gray]{0.75}FO}%
\colorbox{green}{\color[gray]{0.75}FO}%
\colorbox{green}{\color[gray]{0.75}FO}%
\colorbox{green}{\color[gray]{0.75}FO}%
\colorbox{green}{\color[gray]{0.75}FO}%
\colorbox{green}{\color[gray]{0.75}FO}%
\colorbox{green}{\color[gray]{0.75}FO}%
\colorbox{green}{\color[gray]{0.75}FO}%
\colorbox{green}{\color[gray]{0.75}FO}%
\colorbox{green}{\color[gray]{0.75}FO}%
\colorbox{green}{\color[gray]{0.75}FO}%
\colorbox{green}{\color[gray]{0.75}FO}%
\colorbox{green}{\color[gray]{0.75}FO}%
\colorbox{green}{\color[gray]{0.75}FO}%
\colorbox{green}{\color[gray]{0.75}FO}%
\colorbox{green}{\color[gray]{0.75}FO}%
\colorbox{green}{\color[gray]{0.75}FO}%
\colorbox{green}{\color[gray]{0.75}FO}%
\colorbox{green}{\color[gray]{0.75}FO}%
\colorbox{green}{\color[gray]{0.75}FO}%
\colorbox{green}{\color[gray]{0.75}FO}%
\colorbox{green}{\color[gray]{0.75}FO}%
\colorbox{green}{\color[gray]{0.75}FO}%
\colorbox{green}{\color[gray]{0.75}FO}%
\colorbox{green}{\color[gray]{0.75}FO}%
\colorbox{green}{\color[gray]{0.75}FO}%
\colorbox{green}{\color[gray]{0.75}FO}%
\colorbox{green}{\color[gray]{0.75}FO}%
\colorbox{green}{\color[gray]{0.75}FO}%
\colorbox{green}{\color[gray]{0.75}FO}%
\colorbox{green}{\color[gray]{0.75}FO}%
\colorbox{green}{\color[gray]{0.75}FO}%
\colorbox{green}{\color[gray]{0.75}FO}%
\colorbox{green}{\color[gray]{0.75}FO}%
\colorbox{green}{\color[gray]{0.75}FO}%
\colorbox{green}{\color[gray]{0.75}FO}%
\colorbox{green}{\color[gray]{0.75}FO}%
\colorbox{green}{\color[gray]{0.75}FO}%
\colorbox{green}{\color[gray]{0.75}FO}%
\colorbox{green}{\color[gray]{0.75}FO}%
\colorbox{green}{\color[gray]{0.75}FO}%
\colorbox{green}{\color[gray]{0.75}FO}%
\colorbox{green}{\color[gray]{0.75}FO}%
\colorbox{green}{\color[gray]{0.75}FO}%
\colorbox{green}{\color[gray]{0.75}FO}%
\colorbox{green}{\color[gray]{0.75}FO}%
\colorbox{green}{\color[gray]{0.75}FO}%
\colorbox{green}{\color[gray]{0.75}FO}%
\colorbox{green}{\color[gray]{0.75}FO}%
\colorbox{green}{\color[gray]{0.75}FO}%
\colorbox{green}{\color[gray]{0.75}FO}%
\colorbox{green}{\color[gray]{0.75}FO}%
\colorbox{green}{\color[gray]{0.75}FO}%
\colorbox{green}{\color[gray]{0.75}FO}%
\colorbox{green}{\color[gray]{0.75}FO}%
\colorbox{green}{\color[gray]{0.75}FO}%
\colorbox{green}{\color[gray]{0.75}FO}%
\colorbox{green}{\color[gray]{0.75}FO}%
\colorbox{green}{\color[gray]{0.75}FO}%
\colorbox{green}{\color[gray]{0.75}FO}%
\colorbox{green}{\color[gray]{0.75}FO}%
\colorbox{green}{\color[gray]{0.75}FO}%
\colorbox{green}{\color[gray]{0.75}FO}%
\colorbox{green}{\color[gray]{0.75}FO}%
\colorbox{green}{\color[gray]{0.75}FO}%
\colorbox{green}{\color[gray]{0.75}FO}%
\colorbox{green}{\color[gray]{0.75}FO}%
\colorbox{green}{\color[gray]{0.75}FO}%
\colorbox{green}{\color[gray]{0.75}FO}%
\colorbox{green}{\color[gray]{0.75}FO}%
\\
\colorbox{green}{\color[gray]{0.75}FO}%
\colorbox{green}{\color[gray]{0.75}FO}%
\colorbox{green}{\color[gray]{0.75}FO}%
\colorbox{green}{\color[gray]{0.75}FO}%
\colorbox{green}{\color[gray]{0.75}FO}%
\colorbox{green}{\color[gray]{0.75}FO}%
\colorbox{green}{\color[gray]{0.75}FO}%
\colorbox{green}{\color[gray]{0.75}FO}%
\colorbox{green}{\color[gray]{0.75}FO}%
\colorbox{green}{\color[gray]{0.75}FO}%
\colorbox{green}{\color[gray]{0.75}FO}%
\colorbox{green}{\color[gray]{0.75}FO}%
\colorbox{green}{\color[gray]{0.75}FO}%
\colorbox{green}{\color[gray]{0.75}FO}%
\colorbox{green}{\color[gray]{0.75}FO}%
\colorbox{green}{\color[gray]{0.75}FO}%
\colorbox{green}{\color[gray]{0.75}FO}%
\colorbox{green}{\color[gray]{0.75}FO}%
\colorbox{green}{\color[gray]{0.75}FO}%
\colorbox{green}{\color[gray]{0.75}FO}%
\colorbox{green}{\color[gray]{0.75}FO}%
\colorbox{green}{\color[gray]{0.75}FO}%
\colorbox{green}{\color[gray]{0.75}FO}%
\colorbox{green}{\color[gray]{0.75}FO}%
\colorbox{green}{\color[gray]{0.75}FO}%
\colorbox{green}{\color[gray]{0.75}FO}%
\colorbox{green}{\color[gray]{0.75}FO}%
\colorbox{green}{\color[gray]{0.75}FO}%
\colorbox{green}{\color[gray]{0.75}FO}%
\colorbox{green}{\color[gray]{0.75}FO}%
\colorbox{green}{\color[gray]{0.75}FO}%
\colorbox{green}{\color[gray]{0.75}FO}%
\colorbox{green}{\color[gray]{0.75}FO}%
\colorbox{green}{\color[gray]{0.75}FO}%
\colorbox{green}{\color[gray]{0.75}FO}%
\colorbox{green}{\color[gray]{0.75}FO}%
\colorbox{green}{\color[gray]{0.75}FO}%
\colorbox{green}{\color[gray]{0.75}FO}%
\colorbox{green}{\color[gray]{0.75}FO}%
\colorbox{green}{\color[gray]{0.75}FO}%
\colorbox{green}{\color[gray]{0.75}FO}%
\colorbox{green}{\color[gray]{0.75}FO}%
\colorbox{green}{\color[gray]{0.75}FO}%
\colorbox{green}{\color[gray]{0.75}FO}%
\colorbox{green}{\color[gray]{0.75}FO}%
\colorbox{green}{\color[gray]{0.75}FO}%
\colorbox{green}{\color[gray]{0.75}FO}%
\colorbox{green}{\color[gray]{0.75}FO}%
\colorbox{green}{\color[gray]{0.75}FO}%
\colorbox{green}{\color[gray]{0.75}FO}%
\colorbox{green}{\color[gray]{0.75}FO}%
\colorbox{green}{\color[gray]{0.75}FO}%
\colorbox{green}{\color[gray]{0.75}FO}%
\colorbox{green}{\color[gray]{0.75}FO}%
\colorbox{green}{\color[gray]{0.75}FO}%
\colorbox{green}{\color[gray]{0.75}FO}%
\colorbox{green}{\color[gray]{0.75}FO}%
\colorbox{green}{\color[gray]{0.75}FO}%
\colorbox{green}{\color[gray]{0.75}FO}%
\colorbox{green}{\color[gray]{0.75}FO}%
\colorbox{green}{\color[gray]{0.75}FO}%
\colorbox{green}{\color[gray]{0.75}FO}%
\colorbox{green}{\color[gray]{0.75}FO}%
\colorbox{green}{\color[gray]{0.75}FO}%
\colorbox{green}{\color[gray]{0.75}FO}%
\colorbox{green}{\color[gray]{0.75}FO}%
\colorbox{green}{\color[gray]{0.75}FO}%
\colorbox{green}{\color[gray]{0.75}FO}%
\colorbox{green}{\color[gray]{0.75}FO}%
\colorbox{green}{\color[gray]{0.75}FO}%
\colorbox{green}{\color[gray]{0.75}FO}%
\colorbox{green}{\color[gray]{0.75}FO}%
\colorbox{green}{\color[gray]{0.75}FO}%
\colorbox{green}{\color[gray]{0.75}FO}%
\colorbox{green}{\color[gray]{0.75}FO}%
\colorbox{green}{\color[gray]{0.75}FO}%
\colorbox{green}{\color[gray]{0.75}FO}%
\colorbox{green}{\color[gray]{0.75}FO}%
\colorbox{green}{\color[gray]{0.75}FO}%
\colorbox{green}{\color[gray]{0.75}FO}%
\colorbox{green}{\color[gray]{0.75}FO}%
\colorbox{green}{\color[gray]{0.75}FO}%
\colorbox{green}{\color[gray]{0.75}FO}%
\colorbox{green}{\color[gray]{0.75}FO}%
\colorbox{green}{\color[gray]{0.75}FO}%
\colorbox{green}{\color[gray]{0.75}FO}%
\colorbox{green}{\color[gray]{0.75}FO}%
\colorbox{green}{\color[gray]{0.75}FO}%
\colorbox{green}{\color[gray]{0.75}FO}%
\colorbox{green}{\color[gray]{0.75}FO}%
\colorbox{green}{\color[gray]{0.75}FO}%
\colorbox{green}{\color[gray]{0.75}FO}%
\colorbox{green}{\color[gray]{0.75}FO}%
\colorbox{green}{\color[gray]{0.75}FO}%
\colorbox{green}{\color[gray]{0.75}FO}%
\colorbox{green}{\color[gray]{0.75}FO}%
\colorbox{green}{\color[gray]{0.75}FO}%
\colorbox{green}{\color[gray]{0.75}FO}%
\colorbox{green}{\color[gray]{0.75}FO}%
\colorbox{green}{\color[gray]{0.75}FO}%
\\
\colorbox{green}{\color[gray]{0.75}FO}%
\colorbox{green}{\color[gray]{0.75}FO}%
\colorbox{green}{\color[gray]{0.75}FO}%
\colorbox{green}{\color[gray]{0.75}FO}%
\colorbox{green}{\color[gray]{0.75}FO}%
\colorbox{green}{\color[gray]{0.75}FO}%
\colorbox{green}{\color[gray]{0.75}FO}%
\colorbox{green}{\color[gray]{0.75}FO}%
\colorbox{green}{\color[gray]{0.75}FO}%
\colorbox{green}{\color[gray]{0.75}FO}%
\colorbox{green}{\color[gray]{0.75}FO}%
\colorbox{green}{\color[gray]{0.75}FO}%
\colorbox{green}{\color[gray]{0.75}FO}%
\colorbox{green}{\color[gray]{0.75}FO}%
\colorbox{green}{\color[gray]{0.75}FO}%
\colorbox{green}{\color[gray]{0.75}FO}%
\colorbox{green}{\color[gray]{0.75}FO}%
\colorbox{green}{\color[gray]{0.75}FO}%
\colorbox{green}{\color[gray]{0.75}FO}%
\colorbox{green}{\color[gray]{0.75}FO}%
\colorbox{green}{\color[gray]{0.75}FO}%
\colorbox{green}{\color[gray]{0.75}FO}%
\colorbox{green}{\color[gray]{0.75}FO}%
\colorbox{green}{\color[gray]{0.75}FO}%
\colorbox{green}{\color[gray]{0.75}FO}%
\colorbox{green}{\color[gray]{0.75}FO}%
\colorbox{green}{\color[gray]{0.75}FO}%
\colorbox{green}{\color[gray]{0.75}FO}%
\colorbox{green}{\color[gray]{0.75}FO}%
\colorbox{green}{\color[gray]{0.75}FO}%
\colorbox{green}{\color[gray]{0.75}FO}%
\colorbox{green}{\color[gray]{0.75}FO}%
\colorbox{green}{\color[gray]{0.75}FO}%
\colorbox{green}{\color[gray]{0.75}FO}%
\colorbox{green}{\color[gray]{0.75}FO}%
\colorbox{green}{\color[gray]{0.75}FO}%
\colorbox{green}{\color[gray]{0.75}FO}%
\colorbox{green}{\color[gray]{0.75}FO}%
\colorbox{green}{\color[gray]{0.75}FO}%
\colorbox{green}{\color[gray]{0.75}FO}%
\colorbox{green}{\color[gray]{0.75}FO}%
\colorbox{green}{\color[gray]{0.75}FO}%
\colorbox{green}{\color[gray]{0.75}FO}%
\colorbox{green}{\color[gray]{0.75}FO}%
\colorbox{green}{\color[gray]{0.75}FO}%
\colorbox{green}{\color[gray]{0.75}FO}%
\colorbox{green}{\color[gray]{0.75}FO}%
\colorbox{green}{\color[gray]{0.75}FO}%
\colorbox{green}{\color[gray]{0.75}FO}%
\colorbox{green}{\color[gray]{0.75}FO}%
\colorbox{green}{\color[gray]{0.75}FO}%
\colorbox{green}{\color[gray]{0.75}FO}%
\colorbox{green}{\color[gray]{0.75}FO}%
\colorbox{green}{\color[gray]{0.75}FO}%
\colorbox{green}{\color[gray]{0.75}FO}%
\colorbox{green}{\color[gray]{0.75}FO}%
\colorbox{green}{\color[gray]{0.75}FO}%
\colorbox{green}{\color[gray]{0.75}FO}%
\colorbox{green}{\color[gray]{0.75}FO}%
\colorbox{green}{\color[gray]{0.75}FO}%
\colorbox{green}{\color[gray]{0.75}FO}%
\colorbox{green}{\color[gray]{0.75}FO}%
\colorbox{green}{\color[gray]{0.75}FO}%
\colorbox{green}{\color[gray]{0.75}FO}%
\colorbox{green}{\color[gray]{0.75}FO}%
\colorbox{green}{\color[gray]{0.75}FO}%
\colorbox{green}{\color[gray]{0.75}FO}%
\colorbox{green}{\color[gray]{0.75}FO}%
\colorbox{green}{\color[gray]{0.75}FO}%
\colorbox{green}{\color[gray]{0.75}FO}%
\colorbox{green}{\color[gray]{0.75}FO}%
\colorbox{green}{\color[gray]{0.75}FO}%
\colorbox{green}{\color[gray]{0.75}FO}%
\colorbox{green}{\color[gray]{0.75}FO}%
\colorbox{green}{\color[gray]{0.75}FO}%
\colorbox{green}{\color[gray]{0.75}FO}%
\colorbox{green}{\color[gray]{0.75}FO}%
\colorbox{green}{\color[gray]{0.75}FO}%
\colorbox{green}{\color[gray]{0.75}FO}%
\colorbox{green}{\color[gray]{0.75}FO}%
\colorbox{green}{\color[gray]{0.75}FO}%
\colorbox{green}{\color[gray]{0.75}FO}%
\colorbox{green}{\color[gray]{0.75}FO}%
\colorbox{green}{\color[gray]{0.75}FO}%
\colorbox{green}{\color[gray]{0.75}FO}%
\colorbox{green}{\color[gray]{0.75}FO}%
\colorbox{green}{\color[gray]{0.75}FO}%
\colorbox{green}{\color[gray]{0.75}FO}%
\colorbox{green}{\color[gray]{0.75}FO}%
\colorbox{green}{\color[gray]{0.75}FO}%
\colorbox{green}{\color[gray]{0.75}FO}%
\colorbox{green}{\color[gray]{0.75}FO}%
\colorbox{green}{\color[gray]{0.75}FO}%
\colorbox{green}{\color[gray]{0.75}FO}%
\colorbox{green}{\color[gray]{0.75}FO}%
\colorbox{green}{\color[gray]{0.75}FO}%
\colorbox{green}{\color[gray]{0.75}FO}%
\colorbox{green}{\color[gray]{0.75}FO}%
\colorbox{green}{\color[gray]{0.75}FO}%
\colorbox{green}{\color[gray]{0.75}FO}%
\\
\colorbox{green}{\color[gray]{0.75}FO}%
\colorbox{green}{\color[gray]{0.75}FO}%
\colorbox{green}{\color[gray]{0.75}FO}%
\colorbox{green}{\color[gray]{0.75}FO}%
\colorbox{green}{\color[gray]{0.75}FO}%
\colorbox{green}{\color[gray]{0.75}FO}%
\colorbox{green}{\color[gray]{0.75}FO}%
\colorbox{green}{\color[gray]{0.75}FO}%
\colorbox{green}{\color[gray]{0.75}FO}%
\colorbox{green}{\color[gray]{0.75}FO}%
\colorbox{green}{\color[gray]{0.75}FO}%
\colorbox{green}{\color[gray]{0.75}FO}%
\colorbox{green}{\color[gray]{0.75}FO}%
\colorbox{green}{\color[gray]{0.75}FO}%
\colorbox{green}{\color[gray]{0.75}FO}%
\colorbox{green}{\color[gray]{0.75}FO}%
\colorbox{green}{\color[gray]{0.75}FO}%
\colorbox{green}{\color[gray]{0.75}FO}%
\colorbox{green}{\color[gray]{0.75}FO}%
\colorbox{green}{\color[gray]{0.75}FO}%
\colorbox{green}{\color[gray]{0.75}FO}%
\colorbox{green}{\color[gray]{0.75}FO}%
\colorbox{green}{\color[gray]{0.75}FO}%
\colorbox{green}{\color[gray]{0.75}FO}%
\colorbox{green}{\color[gray]{0.75}FO}%
\colorbox{green}{\color[gray]{0.75}FO}%
\colorbox{green}{\color[gray]{0.75}FO}%
\colorbox{green}{\color[gray]{0.75}FO}%
\colorbox{green}{\color[gray]{0.75}FO}%
\colorbox{green}{\color[gray]{0.75}FO}%
\colorbox{green}{\color[gray]{0.75}FO}%
\colorbox{green}{\color[gray]{0.75}FO}%
\colorbox{green}{\color[gray]{0.75}FO}%
\colorbox{green}{\color[gray]{0.75}FO}%
\colorbox{green}{\color[gray]{0.75}FO}%
\colorbox{green}{\color[gray]{0.75}FO}%
\colorbox{green}{\color[gray]{0.75}FO}%
\colorbox{green}{\color[gray]{0.75}FO}%
\colorbox{green}{\color[gray]{0.75}FO}%
\colorbox{green}{\color[gray]{0.75}FO}%
\colorbox{green}{\color[gray]{0.75}FO}%
\colorbox{green}{\color[gray]{0.75}FO}%
\colorbox{green}{\color[gray]{0.75}FO}%
\colorbox{green}{\color[gray]{0.75}FO}%
\colorbox{green}{\color[gray]{0.75}FO}%
\colorbox{green}{\color[gray]{0.75}FO}%
\colorbox{green}{\color[gray]{0.75}FO}%
\colorbox{green}{\color[gray]{0.75}FO}%
\colorbox{green}{\color[gray]{0.75}FO}%
\colorbox{green}{\color[gray]{0.75}FO}%
\colorbox{green}{\color[gray]{0.75}FO}%
\colorbox{green}{\color[gray]{0.75}FO}%
\colorbox{green}{\color[gray]{0.75}FO}%
\colorbox{green}{\color[gray]{0.75}FO}%
\colorbox{green}{\color[gray]{0.75}FO}%
\colorbox{green}{\color[gray]{0.75}FO}%
\colorbox{green}{\color[gray]{0.75}FO}%
\colorbox{green}{\color[gray]{0.75}FO}%
\colorbox{green}{\color[gray]{0.75}FO}%
\colorbox{green}{\color[gray]{0.75}FO}%
\colorbox{green}{\color[gray]{0.75}FO}%
\colorbox{green}{\color[gray]{0.75}FO}%
\colorbox{green}{\color[gray]{0.75}FO}%
\colorbox{green}{\color[gray]{0.75}FO}%
\colorbox{green}{\color[gray]{0.75}FO}%
\colorbox{green}{\color[gray]{0.75}FO}%
\colorbox{green}{\color[gray]{0.75}FO}%
\colorbox{green}{\color[gray]{0.75}FO}%
\colorbox{green}{\color[gray]{0.75}FO}%
\colorbox{green}{\color[gray]{0.75}FO}%
\colorbox{green}{\color[gray]{0.75}FO}%
\colorbox{green}{\color[gray]{0.75}FO}%
\colorbox{green}{\color[gray]{0.75}FO}%
\colorbox{green}{\color[gray]{0.75}FO}%
\colorbox{green}{\color[gray]{0.75}FO}%
\colorbox{green}{\color[gray]{0.75}FO}%
\colorbox{green}{\color[gray]{0.75}FO}%
\colorbox{green}{\color[gray]{0.75}FO}%
\colorbox{green}{\color[gray]{0.75}FO}%
\colorbox{green}{\color[gray]{0.75}FO}%
\colorbox{green}{\color[gray]{0.75}FO}%
\colorbox{green}{\color[gray]{0.75}FO}%
\colorbox{green}{\color[gray]{0.75}FO}%
\colorbox{green}{\color[gray]{0.75}FO}%
\colorbox{green}{\color[gray]{0.75}FO}%
\colorbox{green}{\color[gray]{0.75}FO}%
\colorbox{green}{\color[gray]{0.75}FO}%
\colorbox{green}{\color[gray]{0.75}FO}%
\colorbox{green}{\color[gray]{0.75}FO}%
\colorbox{green}{\color[gray]{0.75}FO}%
\colorbox{green}{\color[gray]{0.75}FO}%
\colorbox{green}{\color[gray]{0.75}FO}%
\colorbox{green}{\color[gray]{0.75}FO}%
\colorbox{green}{\color[gray]{0.75}FO}%
\colorbox{green}{\color[gray]{0.75}FO}%
\colorbox{green}{\color[gray]{0.75}FO}%
\colorbox{green}{\color[gray]{0.75}FO}%
\colorbox{green}{\color[gray]{0.75}FO}%
\colorbox{green}{\color[gray]{0.75}FO}%
\colorbox{green}{\color[gray]{0.75}FO}%
\\
\colorbox{green}{\color[gray]{0.75}FO}%
\colorbox{green}{\color[gray]{0.75}FO}%
\colorbox{green}{\color[gray]{0.75}FO}%
\colorbox{green}{\color[gray]{0.75}FO}%
\colorbox{green}{\color[gray]{0.75}FO}%
\colorbox{green}{\color[gray]{0.75}FO}%
\colorbox{green}{\color[gray]{0.75}FO}%
\colorbox{green}{\color[gray]{0.75}FO}%
\colorbox{green}{\color[gray]{0.75}FO}%
\colorbox{green}{\color[gray]{0.75}FO}%
\colorbox{green}{\color[gray]{0.75}FO}%
\colorbox{green}{\color[gray]{0.75}FO}%
\colorbox{green}{\color[gray]{0.75}FO}%
\colorbox{green}{\color[gray]{0.75}FO}%
\colorbox{green}{\color[gray]{0.75}FO}%
\colorbox{green}{\color[gray]{0.75}FO}%
\colorbox{green}{\color[gray]{0.75}FO}%
\colorbox{green}{\color[gray]{0.75}FO}%
\colorbox{green}{\color[gray]{0.75}FO}%
\colorbox{green}{\color[gray]{0.75}FO}%
\colorbox{green}{\color[gray]{0.75}FO}%
\colorbox{green}{\color[gray]{0.75}FO}%
\colorbox{green}{\color[gray]{0.75}FO}%
\colorbox{green}{\color[gray]{0.75}FO}%
\colorbox{green}{\color[gray]{0.75}FO}%
\colorbox{green}{\color[gray]{0.75}FO}%
\colorbox{green}{\color[gray]{0.75}FO}%
\colorbox{green}{\color[gray]{0.75}FO}%
\colorbox{green}{\color[gray]{0.75}FO}%
\colorbox{green}{\color[gray]{0.75}FO}%
\colorbox{green}{\color[gray]{0.75}FO}%
\colorbox{green}{\color[gray]{0.75}FO}%
\colorbox{green}{\color[gray]{0.75}FO}%
\colorbox{green}{\color[gray]{0.75}FO}%
\colorbox{green}{\color[gray]{0.75}FO}%
\colorbox{green}{\color[gray]{0.75}FO}%
\colorbox{green}{\color[gray]{0.75}FO}%
\colorbox{green}{\color[gray]{0.75}FO}%
\colorbox{green}{\color[gray]{0.75}FO}%
\colorbox{green}{\color[gray]{0.75}FO}%
\colorbox{green}{\color[gray]{0.75}FO}%
\colorbox{green}{\color[gray]{0.75}FO}%
\colorbox{green}{\color[gray]{0.75}FO}%
\colorbox{green}{\color[gray]{0.75}FO}%
\colorbox{green}{\color[gray]{0.75}FO}%
\colorbox{green}{\color[gray]{0.75}FO}%
\colorbox{green}{\color[gray]{0.75}FO}%
\colorbox{green}{\color[gray]{0.75}FO}%
\colorbox{green}{\color[gray]{0.75}FO}%
\colorbox{green}{\color[gray]{0.75}FO}%
\colorbox{green}{\color[gray]{0.75}FO}%
\colorbox{green}{\color[gray]{0.75}FO}%
\colorbox{green}{\color[gray]{0.75}FO}%
\colorbox{green}{\color[gray]{0.75}FO}%
\colorbox{green}{\color[gray]{0.75}FO}%
\colorbox{green}{\color[gray]{0.75}FO}%
\colorbox{green}{\color[gray]{0.75}FO}%
\colorbox{green}{\color[gray]{0.75}FO}%
\colorbox{green}{\color[gray]{0.75}FO}%
\colorbox{green}{\color[gray]{0.75}FO}%
\colorbox{green}{\color[gray]{0.75}FO}%
\colorbox{green}{\color[gray]{0.75}FO}%
\colorbox{green}{\color[gray]{0.75}FO}%
\colorbox{green}{\color[gray]{0.75}FO}%
\colorbox{green}{\color[gray]{0.75}FO}%
\colorbox{green}{\color[gray]{0.75}FO}%
\colorbox{green}{\color[gray]{0.75}FO}%
\colorbox{green}{\color[gray]{0.75}FO}%
\colorbox{green}{\color[gray]{0.75}FO}%
\colorbox{green}{\color[gray]{0.75}FO}%
\colorbox{green}{\color[gray]{0.75}FO}%
\colorbox{green}{\color[gray]{0.75}FO}%
\colorbox{green}{\color[gray]{0.75}FO}%
\colorbox{green}{\color[gray]{0.75}FO}%
\colorbox{green}{\color[gray]{0.75}FO}%
\colorbox{green}{\color[gray]{0.75}FO}%
\colorbox{green}{\color[gray]{0.75}FO}%
\colorbox{green}{\color[gray]{0.75}FO}%
\colorbox{green}{\color[gray]{0.75}FO}%
\colorbox{green}{\color[gray]{0.75}FO}%
\colorbox{green}{\color[gray]{0.75}FO}%
\colorbox{green}{\color[gray]{0.75}FO}%
\colorbox{green}{\color[gray]{0.75}FO}%
\colorbox{green}{\color[gray]{0.75}FO}%
\colorbox{green}{\color[gray]{0.75}FO}%
\colorbox{green}{\color[gray]{0.75}FO}%
\colorbox{green}{\color[gray]{0.75}FO}%
\colorbox{green}{\color[gray]{0.75}FO}%
\colorbox{green}{\color[gray]{0.75}FO}%
\colorbox{green}{\color[gray]{0.75}FO}%
\colorbox{green}{\color[gray]{0.75}FO}%
\colorbox{green}{\color[gray]{0.75}FO}%
\colorbox{green}{\color[gray]{0.75}FO}%
\colorbox{green}{\color[gray]{0.75}FO}%
\colorbox{green}{\color[gray]{0.75}FO}%
\colorbox{green}{\color[gray]{0.75}FO}%
\colorbox{green}{\color[gray]{0.75}FO}%
\colorbox{green}{\color[gray]{0.75}FO}%
\colorbox{green}{\color[gray]{0.75}FO}%
\colorbox{green}{\color[gray]{0.75}FO}%
\\
\colorbox{green}{\color[gray]{0.75}FO}%
\colorbox{green}{\color[gray]{0.75}FO}%
\colorbox{green}{\color[gray]{0.75}FO}%
\colorbox{green}{\color[gray]{0.75}FO}%
\colorbox{green}{\color[gray]{0.75}FO}%
\colorbox{green}{\color[gray]{0.75}FO}%
\colorbox{green}{\color[gray]{0.75}FO}%
\colorbox{green}{\color[gray]{0.75}FO}%
\colorbox{green}{\color[gray]{0.75}FO}%
\colorbox{green}{\color[gray]{0.75}FO}%
\colorbox{green}{\color[gray]{0.75}FO}%
\colorbox{green}{\color[gray]{0.75}FO}%
\colorbox{green}{\color[gray]{0.75}FO}%
\colorbox{green}{\color[gray]{0.75}FO}%
\colorbox{green}{\color[gray]{0.75}FO}%
\colorbox{green}{\color[gray]{0.75}FO}%
\colorbox{green}{\color[gray]{0.75}FO}%
\colorbox{green}{\color[gray]{0.75}FO}%
\colorbox{green}{\color[gray]{0.75}FO}%
\colorbox{green}{\color[gray]{0.75}FO}%
\colorbox{green}{\color[gray]{0.75}FO}%
\colorbox{green}{\color[gray]{0.75}FO}%
\colorbox{green}{\color[gray]{0.75}FO}%
\colorbox{green}{\color[gray]{0.75}FO}%
\colorbox{green}{\color[gray]{0.75}FO}%
\colorbox{green}{\color[gray]{0.75}FO}%
\colorbox{green}{\color[gray]{0.75}FO}%
\colorbox{green}{\color[gray]{0.75}FO}%
\colorbox{green}{\color[gray]{0.75}FO}%
\colorbox{green}{\color[gray]{0.75}FO}%
\colorbox{green}{\color[gray]{0.75}FO}%
\colorbox{green}{\color[gray]{0.75}FO}%
\colorbox{green}{\color[gray]{0.75}FO}%
\colorbox{green}{\color[gray]{0.75}FO}%
\colorbox{green}{\color[gray]{0.75}FO}%
\colorbox{green}{\color[gray]{0.75}FO}%
\colorbox{green}{\color[gray]{0.75}FO}%
\colorbox{green}{\color[gray]{0.75}FO}%
\colorbox{green}{\color[gray]{0.75}FO}%
\colorbox{green}{\color[gray]{0.75}FO}%
\colorbox{green}{\color[gray]{0.75}FO}%
\colorbox{green}{\color[gray]{0.75}FO}%
\colorbox{green}{\color[gray]{0.75}FO}%
\colorbox{green}{\color[gray]{0.75}FO}%
\colorbox{green}{\color[gray]{0.75}FO}%
\colorbox{green}{\color[gray]{0.75}FO}%
\colorbox{green}{\color[gray]{0.75}FO}%
\colorbox{green}{\color[gray]{0.75}FO}%
\colorbox{green}{\color[gray]{0.75}FO}%
\colorbox{green}{\color[gray]{0.75}FO}%
\colorbox{green}{\color[gray]{0.75}FO}%
\colorbox{green}{\color[gray]{0.75}FO}%
\colorbox{green}{\color[gray]{0.75}FO}%
\colorbox{green}{\color[gray]{0.75}FO}%
\colorbox{green}{\color[gray]{0.75}FO}%
\colorbox{green}{\color[gray]{0.75}FO}%
\colorbox{green}{\color[gray]{0.75}FO}%
\colorbox{green}{\color[gray]{0.75}FO}%
\colorbox{green}{\color[gray]{0.75}FO}%
\colorbox{green}{\color[gray]{0.75}FO}%
\colorbox{green}{\color[gray]{0.75}FO}%
\colorbox{green}{\color[gray]{0.75}FO}%
\colorbox{green}{\color[gray]{0.75}FO}%
\colorbox{green}{\color[gray]{0.75}FO}%
\colorbox{green}{\color[gray]{0.75}FO}%
\colorbox{green}{\color[gray]{0.75}FO}%
\colorbox{green}{\color[gray]{0.75}FO}%
\colorbox{green}{\color[gray]{0.75}FO}%
\colorbox{green}{\color[gray]{0.75}FO}%
\colorbox{green}{\color[gray]{0.75}FO}%
\colorbox{green}{\color[gray]{0.75}FO}%
\colorbox{green}{\color[gray]{0.75}FO}%
\colorbox{green}{\color[gray]{0.75}FO}%
\colorbox{green}{\color[gray]{0.75}FO}%
\colorbox{green}{\color[gray]{0.75}FO}%
\colorbox{green}{\color[gray]{0.75}FO}%
\colorbox{green}{\color[gray]{0.75}FO}%
\colorbox{green}{\color[gray]{0.75}FO}%
\colorbox{green}{\color[gray]{0.75}FO}%
\colorbox{green}{\color[gray]{0.75}FO}%
\colorbox{green}{\color[gray]{0.75}FO}%
\colorbox{green}{\color[gray]{0.75}FO}%
\colorbox{green}{\color[gray]{0.75}FO}%
\colorbox{green}{\color[gray]{0.75}FO}%
\colorbox{green}{\color[gray]{0.75}FO}%
\colorbox{green}{\color[gray]{0.75}FO}%
\colorbox{green}{\color[gray]{0.75}FO}%
\colorbox{green}{\color[gray]{0.75}FO}%
\colorbox{green}{\color[gray]{0.75}FO}%
\colorbox{green}{\color[gray]{0.75}FO}%
\colorbox{green}{\color[gray]{0.75}FO}%
\colorbox{green}{\color[gray]{0.75}FO}%
\colorbox{green}{\color[gray]{0.75}FO}%
\colorbox{green}{\color[gray]{0.75}FO}%
\colorbox{green}{\color[gray]{0.75}FO}%
\colorbox{green}{\color[gray]{0.75}FO}%
\colorbox{green}{\color[gray]{0.75}FO}%
\colorbox{green}{\color[gray]{0.75}FO}%
\colorbox{green}{\color[gray]{0.75}FO}%
\colorbox{green}{\color[gray]{0.75}FO}%
\\
\colorbox{green}{\color[gray]{0.75}FO}%
\colorbox{green}{\color[gray]{0.75}FO}%
\colorbox{green}{\color[gray]{0.75}FO}%
\colorbox{green}{\color[gray]{0.75}FO}%
\colorbox{green}{\color[gray]{0.75}FO}%
\colorbox{green}{\color[gray]{0.75}FO}%
\colorbox{green}{\color[gray]{0.75}FO}%
\colorbox{green}{\color[gray]{0.75}FO}%
\colorbox{green}{\color[gray]{0.75}FO}%
\colorbox{green}{\color[gray]{0.75}FO}%
\colorbox{green}{\color[gray]{0.75}FO}%
\colorbox{green}{\color[gray]{0.75}FO}%
\colorbox{green}{\color[gray]{0.75}FO}%
\colorbox{green}{\color[gray]{0.75}FO}%
\colorbox{green}{\color[gray]{0.75}FO}%
\colorbox{green}{\color[gray]{0.75}FO}%
\colorbox{green}{\color[gray]{0.75}FO}%
\colorbox{green}{\color[gray]{0.75}FO}%
\colorbox{green}{\color[gray]{0.75}FO}%
\colorbox{green}{\color[gray]{0.75}FO}%
\colorbox{green}{\color[gray]{0.75}FO}%
\colorbox{green}{\color[gray]{0.75}FO}%
\colorbox{green}{\color[gray]{0.75}FO}%
\colorbox{green}{\color[gray]{0.75}FO}%
\colorbox{green}{\color[gray]{0.75}FO}%
\colorbox{green}{\color[gray]{0.75}FO}%
\colorbox{green}{\color[gray]{0.75}FO}%
\colorbox{green}{\color[gray]{0.75}FO}%
\colorbox{green}{\color[gray]{0.75}FO}%
\colorbox{green}{\color[gray]{0.75}FO}%
\colorbox{green}{\color[gray]{0.75}FO}%
\colorbox{green}{\color[gray]{0.75}FO}%
\colorbox{green}{\color[gray]{0.75}FO}%
\colorbox{green}{\color[gray]{0.75}FO}%
\colorbox{green}{\color[gray]{0.75}FO}%
\colorbox{green}{\color[gray]{0.75}FO}%
\colorbox{green}{\color[gray]{0.75}FO}%
\colorbox{green}{\color[gray]{0.75}FO}%
\colorbox{green}{\color[gray]{0.75}FO}%
\colorbox{green}{\color[gray]{0.75}FO}%
\colorbox{green}{\color[gray]{0.75}FO}%
\colorbox{green}{\color[gray]{0.75}FO}%
\colorbox{green}{\color[gray]{0.75}FO}%
\colorbox{green}{\color[gray]{0.75}FO}%
\colorbox{green}{\color[gray]{0.75}FO}%
\colorbox{green}{\color[gray]{0.75}FO}%
\colorbox{green}{\color[gray]{0.75}FO}%
\colorbox{green}{\color[gray]{0.75}FO}%
\colorbox{green}{\color[gray]{0.75}FO}%
\colorbox{green}{\color[gray]{0.75}FO}%
\colorbox{green}{\color[gray]{0.75}FO}%
\colorbox{green}{\color[gray]{0.75}FO}%
\colorbox{green}{\color[gray]{0.75}FO}%
\colorbox{green}{\color[gray]{0.75}FO}%
\colorbox{green}{\color[gray]{0.75}FO}%
\colorbox{green}{\color[gray]{0.75}FO}%
\colorbox{green}{\color[gray]{0.75}FO}%
\colorbox{green}{\color[gray]{0.75}FO}%
\colorbox{green}{\color[gray]{0.75}FO}%
\colorbox{green}{\color[gray]{0.75}FO}%
\colorbox{green}{\color[gray]{0.75}FO}%
\colorbox{green}{\color[gray]{0.75}FO}%
\colorbox{green}{\color[gray]{0.75}FO}%
\colorbox{green}{\color[gray]{0.75}FO}%
\colorbox{green}{\color[gray]{0.75}FO}%
\colorbox{green}{\color[gray]{0.75}FO}%
\colorbox{green}{\color[gray]{0.75}FO}%
\colorbox{green}{\color[gray]{0.75}FO}%
\colorbox{green}{\color[gray]{0.75}FO}%
\colorbox{green}{\color[gray]{0.75}FO}%
\colorbox{green}{\color[gray]{0.75}FO}%
\colorbox{green}{\color[gray]{0.75}FO}%
\colorbox{green}{\color[gray]{0.75}FO}%
\colorbox{green}{\color[gray]{0.75}FO}%
\colorbox{green}{\color[gray]{0.75}FO}%
\colorbox{green}{\color[gray]{0.75}FO}%
\colorbox{green}{\color[gray]{0.75}FO}%
\colorbox{green}{\color[gray]{0.75}FO}%
\colorbox{green}{\color[gray]{0.75}FO}%
\colorbox{green}{\color[gray]{0.75}FO}%
\colorbox{green}{\color[gray]{0.75}FO}%
\colorbox{green}{\color[gray]{0.75}FO}%
\colorbox{green}{\color[gray]{0.75}FO}%
\colorbox{green}{\color[gray]{0.75}FO}%
\colorbox{green}{\color[gray]{0.75}FO}%
\colorbox{green}{\color[gray]{0.75}FO}%
\colorbox{green}{\color[gray]{0.75}FO}%
\colorbox{green}{\color[gray]{0.75}FO}%
\colorbox{green}{\color[gray]{0.75}FO}%
\colorbox{green}{\color[gray]{0.75}FO}%
\colorbox{green}{\color[gray]{0.75}FO}%
\colorbox{green}{\color[gray]{0.75}FO}%
\colorbox{green}{\color[gray]{0.75}FO}%
\colorbox{green}{\color[gray]{0.75}FO}%
\colorbox{green}{\color[gray]{0.75}FO}%
\colorbox{green}{\color[gray]{0.75}FO}%
\colorbox{green}{\color[gray]{0.75}FO}%
\colorbox{green}{\color[gray]{0.75}FO}%
\colorbox{green}{\color[gray]{0.75}FO}%
\colorbox{green}{\color[gray]{0.75}FO}%
\\
\colorbox{green}{\color[gray]{0.75}FO}%
\colorbox{green}{\color[gray]{0.75}FO}%
\colorbox{green}{\color[gray]{0.75}FO}%
\colorbox{green}{\color[gray]{0.75}FO}%
\colorbox{green}{\color[gray]{0.75}FO}%
\colorbox{green}{\color[gray]{0.75}FO}%
\colorbox{green}{\color[gray]{0.75}FO}%
\colorbox{green}{\color[gray]{0.75}FO}%
\colorbox{green}{\color[gray]{0.75}FO}%
\colorbox{green}{\color[gray]{0.75}FO}%
\colorbox{green}{\color[gray]{0.75}FO}%
\colorbox{green}{\color[gray]{0.75}FO}%
\colorbox{green}{\color[gray]{0.75}FO}%
\colorbox{green}{\color[gray]{0.75}FO}%
\colorbox{green}{\color[gray]{0.75}FO}%
\colorbox{green}{\color[gray]{0.75}FO}%
\colorbox{green}{\color[gray]{0.75}FO}%
\colorbox{green}{\color[gray]{0.75}FO}%
\colorbox{green}{\color[gray]{0.75}FO}%
\colorbox{green}{\color[gray]{0.75}FO}%
\colorbox{green}{\color[gray]{0.75}FO}%
\colorbox{green}{\color[gray]{0.75}FO}%
\colorbox{green}{\color[gray]{0.75}FO}%
\colorbox{green}{\color[gray]{0.75}FO}%
\colorbox{green}{\color[gray]{0.75}FO}%
\colorbox{green}{\color[gray]{0.75}FO}%
\colorbox{green}{\color[gray]{0.75}FO}%
\colorbox{green}{\color[gray]{0.75}FO}%
\colorbox{green}{\color[gray]{0.75}FO}%
\colorbox{green}{\color[gray]{0.75}FO}%
\colorbox{green}{\color[gray]{0.75}FO}%
\colorbox{green}{\color[gray]{0.75}FO}%
\colorbox{green}{\color[gray]{0.75}FO}%
\colorbox{green}{\color[gray]{0.75}FO}%
\colorbox{green}{\color[gray]{0.75}FO}%
\colorbox{green}{\color[gray]{0.75}FO}%
\colorbox{green}{\color[gray]{0.75}FO}%
\colorbox{green}{\color[gray]{0.75}FO}%
\colorbox{green}{\color[gray]{0.75}FO}%
\colorbox{green}{\color[gray]{0.75}FO}%
\colorbox{green}{\color[gray]{0.75}FO}%
\colorbox{green}{\color[gray]{0.75}FO}%
\colorbox{green}{\color[gray]{0.75}FO}%
\colorbox{green}{\color[gray]{0.75}FO}%
\colorbox{green}{\color[gray]{0.75}FO}%
\colorbox{green}{\color[gray]{0.75}FO}%
\colorbox{green}{\color[gray]{0.75}FO}%
\colorbox{green}{\color[gray]{0.75}FO}%
\colorbox{green}{\color[gray]{0.75}FO}%
\colorbox{green}{\color[gray]{0.75}FO}%
\colorbox{green}{\color[gray]{0.75}FO}%
\colorbox{green}{\color[gray]{0.75}FO}%
\colorbox{green}{\color[gray]{0.75}FO}%
\colorbox{green}{\color[gray]{0.75}FO}%
\colorbox{green}{\color[gray]{0.75}FO}%
\colorbox{green}{\color[gray]{0.75}FO}%
\colorbox{green}{\color[gray]{0.75}FO}%
\colorbox{green}{\color[gray]{0.75}FO}%
\colorbox{green}{\color[gray]{0.75}FO}%
\colorbox{green}{\color[gray]{0.75}FO}%
\colorbox{green}{\color[gray]{0.75}FO}%
\colorbox{green}{\color[gray]{0.75}FO}%
\colorbox{green}{\color[gray]{0.75}FO}%
\colorbox{green}{\color[gray]{0.75}FO}%
\colorbox{green}{\color[gray]{0.75}FO}%
\colorbox{green}{\color[gray]{0.75}FO}%
\colorbox{green}{\color[gray]{0.75}FO}%
\colorbox{green}{\color[gray]{0.75}FO}%
\colorbox{green}{\color[gray]{0.75}FO}%
\colorbox{green}{\color[gray]{0.75}FO}%
\colorbox{green}{\color[gray]{0.75}FO}%
\colorbox{green}{\color[gray]{0.75}FO}%
\colorbox{green}{\color[gray]{0.75}FO}%
\colorbox{green}{\color[gray]{0.75}FO}%
\colorbox{green}{\color[gray]{0.75}FO}%
\colorbox{green}{\color[gray]{0.75}FO}%
\colorbox{green}{\color[gray]{0.75}FO}%
\colorbox{green}{\color[gray]{0.75}FO}%
\colorbox{green}{\color[gray]{0.75}FO}%
\colorbox{green}{\color[gray]{0.75}FO}%
\colorbox{green}{\color[gray]{0.75}FO}%
\colorbox{green}{\color[gray]{0.75}FO}%
\colorbox{green}{\color[gray]{0.75}FO}%
\colorbox{green}{\color[gray]{0.75}FO}%
\colorbox{green}{\color[gray]{0.75}FO}%
\colorbox{green}{\color[gray]{0.75}FO}%
\colorbox{green}{\color[gray]{0.75}FO}%
\colorbox{green}{\color[gray]{0.75}FO}%
\colorbox{green}{\color[gray]{0.75}FO}%
\colorbox{green}{\color[gray]{0.75}FO}%
\colorbox{green}{\color[gray]{0.75}FO}%
\colorbox{green}{\color[gray]{0.75}FO}%
\colorbox{green}{\color[gray]{0.75}FO}%
\colorbox{green}{\color[gray]{0.75}FO}%
\colorbox{green}{\color[gray]{0.75}FO}%
\colorbox{green}{\color[gray]{0.75}FO}%
\colorbox{green}{\color[gray]{0.75}FO}%
\colorbox{green}{\color[gray]{0.75}FO}%
\colorbox{green}{\color[gray]{0.75}FO}%
\colorbox{green}{\color[gray]{0.75}FO}%
\\
\colorbox{green}{\color[gray]{0.75}FO}%
\colorbox{green}{\color[gray]{0.75}FO}%
\colorbox{green}{\color[gray]{0.75}FO}%
\colorbox{green}{\color[gray]{0.75}FO}%
\colorbox{green}{\color[gray]{0.75}FO}%
\colorbox{green}{\color[gray]{0.75}FO}%
\colorbox{green}{\color[gray]{0.75}FO}%
\colorbox{green}{\color[gray]{0.75}FO}%
\colorbox{green}{\color[gray]{0.75}FO}%
\colorbox{green}{\color[gray]{0.75}FO}%
\colorbox{green}{\color[gray]{0.75}FO}%
\colorbox{green}{\color[gray]{0.75}FO}%
\colorbox{green}{\color[gray]{0.75}FO}%
\colorbox{green}{\color[gray]{0.75}FO}%
\colorbox{green}{\color[gray]{0.75}FO}%
\colorbox{green}{\color[gray]{0.75}FO}%
\colorbox{green}{\color[gray]{0.75}FO}%
\colorbox{green}{\color[gray]{0.75}FO}%
\colorbox{green}{\color[gray]{0.75}FO}%
\colorbox{green}{\color[gray]{0.75}FO}%
\colorbox{green}{\color[gray]{0.75}FO}%
\colorbox{green}{\color[gray]{0.75}FO}%
\colorbox{green}{\color[gray]{0.75}FO}%
\colorbox{green}{\color[gray]{0.75}FO}%
\colorbox{green}{\color[gray]{0.75}FO}%
\colorbox{green}{\color[gray]{0.75}FO}%
\colorbox{green}{\color[gray]{0.75}FO}%
\colorbox{green}{\color[gray]{0.75}FO}%
\colorbox{green}{\color[gray]{0.75}FO}%
\colorbox{green}{\color[gray]{0.75}FO}%
\colorbox{green}{\color[gray]{0.75}FO}%
\colorbox{green}{\color[gray]{0.75}FO}%
\colorbox{green}{\color[gray]{0.75}FO}%
\colorbox{green}{\color[gray]{0.75}FO}%
\colorbox{green}{\color[gray]{0.75}FO}%
\colorbox{green}{\color[gray]{0.75}FO}%
\colorbox{green}{\color[gray]{0.75}FO}%
\colorbox{green}{\color[gray]{0.75}FO}%
\colorbox{green}{\color[gray]{0.75}FO}%
\colorbox{green}{\color[gray]{0.75}FO}%
\colorbox{green}{\color[gray]{0.75}FO}%
\colorbox{green}{\color[gray]{0.75}FO}%
\colorbox{green}{\color[gray]{0.75}FO}%
\colorbox{green}{\color[gray]{0.75}FO}%
\colorbox{green}{\color[gray]{0.75}FO}%
\colorbox{green}{\color[gray]{0.75}FO}%
\colorbox{green}{\color[gray]{0.75}FO}%
\colorbox{green}{\color[gray]{0.75}FO}%
\colorbox{green}{\color[gray]{0.75}FO}%
\colorbox{green}{\color[gray]{0.75}FO}%
\colorbox{green}{\color[gray]{0.75}FO}%
\colorbox{green}{\color[gray]{0.75}FO}%
\colorbox{green}{\color[gray]{0.75}FO}%
\colorbox{green}{\color[gray]{0.75}FO}%
\colorbox{green}{\color[gray]{0.75}FO}%
\colorbox{green}{\color[gray]{0.75}FO}%
\colorbox{green}{\color[gray]{0.75}FO}%
\colorbox{green}{\color[gray]{0.75}FO}%
\colorbox{green}{\color[gray]{0.75}FO}%
\colorbox{green}{\color[gray]{0.75}FO}%
\colorbox{green}{\color[gray]{0.75}FO}%
\colorbox{green}{\color[gray]{0.75}FO}%
\colorbox{green}{\color[gray]{0.75}FO}%
\colorbox{green}{\color[gray]{0.75}FO}%
\colorbox{green}{\color[gray]{0.75}FO}%
\colorbox{green}{\color[gray]{0.75}FO}%
\colorbox{green}{\color[gray]{0.75}FO}%
\colorbox{green}{\color[gray]{0.75}FO}%
\colorbox{green}{\color[gray]{0.75}FO}%
\colorbox{green}{\color[gray]{0.75}FO}%
\colorbox{green}{\color[gray]{0.75}FO}%
\colorbox{green}{\color[gray]{0.75}FO}%
\colorbox{green}{\color[gray]{0.75}FO}%
\colorbox{green}{\color[gray]{0.75}FO}%
\colorbox{green}{\color[gray]{0.75}FO}%
\colorbox{green}{\color[gray]{0.75}FO}%
\colorbox{green}{\color[gray]{0.75}FO}%
\colorbox{green}{\color[gray]{0.75}FO}%
\colorbox{green}{\color[gray]{0.75}FO}%
\colorbox{green}{\color[gray]{0.75}FO}%
\colorbox{green}{\color[gray]{0.75}FO}%
\colorbox{green}{\color[gray]{0.75}FO}%
\colorbox{green}{\color[gray]{0.75}FO}%
\colorbox{green}{\color[gray]{0.75}FO}%
\colorbox{green}{\color[gray]{0.75}FO}%
\colorbox{green}{\color[gray]{0.75}FO}%
\colorbox{green}{\color[gray]{0.75}FO}%
\colorbox{green}{\color[gray]{0.75}FO}%
\colorbox{green}{\color[gray]{0.75}FO}%
\colorbox{green}{\color[gray]{0.75}FO}%
\colorbox{green}{\color[gray]{0.75}FO}%
\colorbox{green}{\color[gray]{0.75}FO}%
\colorbox{green}{\color[gray]{0.75}FO}%
\colorbox{green}{\color[gray]{0.75}FO}%
\colorbox{green}{\color[gray]{0.75}FO}%
\colorbox{green}{\color[gray]{0.75}FO}%
\colorbox{green}{\color[gray]{0.75}FO}%
\colorbox{green}{\color[gray]{0.75}FO}%
\colorbox{green}{\color[gray]{0.75}FO}%
\colorbox{green}{\color[gray]{0.75}FO}%
\\
\colorbox{green}{\color[gray]{0.75}FO}%
\colorbox{green}{\color[gray]{0.75}FO}%
\colorbox{green}{\color[gray]{0.75}FO}%
\colorbox{green}{\color[gray]{0.75}FO}%
\colorbox{green}{\color[gray]{0.75}FO}%
\colorbox{green}{\color[gray]{0.75}FO}%
\colorbox{green}{\color[gray]{0.75}FO}%
\colorbox{green}{\color[gray]{0.75}FO}%
\colorbox{green}{\color[gray]{0.75}FO}%
\colorbox{green}{\color[gray]{0.75}FO}%
\colorbox{green}{\color[gray]{0.75}FO}%
\colorbox{green}{\color[gray]{0.75}FO}%
\colorbox{green}{\color[gray]{0.75}FO}%
\colorbox{green}{\color[gray]{0.75}FO}%
\colorbox{green}{\color[gray]{0.75}FO}%
\colorbox{green}{\color[gray]{0.75}FO}%
\colorbox{green}{\color[gray]{0.75}FO}%
\colorbox{green}{\color[gray]{0.75}FO}%
\colorbox{green}{\color[gray]{0.75}FO}%
\colorbox{green}{\color[gray]{0.75}FO}%
\colorbox{green}{\color[gray]{0.75}FO}%
\colorbox{green}{\color[gray]{0.75}FO}%
\colorbox{green}{\color[gray]{0.75}FO}%
\colorbox{green}{\color[gray]{0.75}FO}%
\colorbox{green}{\color[gray]{0.75}FO}%
\colorbox{green}{\color[gray]{0.75}FO}%
\colorbox{green}{\color[gray]{0.75}FO}%
\colorbox{green}{\color[gray]{0.75}FO}%
\colorbox{green}{\color[gray]{0.75}FO}%
\colorbox{green}{\color[gray]{0.75}FO}%
\colorbox{green}{\color[gray]{0.75}FO}%
\colorbox{green}{\color[gray]{0.75}FO}%
\colorbox{green}{\color[gray]{0.75}FO}%
\colorbox{green}{\color[gray]{0.75}FO}%
\colorbox{green}{\color[gray]{0.75}FO}%
\colorbox{green}{\color[gray]{0.75}FO}%
\colorbox{green}{\color[gray]{0.75}FO}%
\colorbox{green}{\color[gray]{0.75}FO}%
\colorbox{green}{\color[gray]{0.75}FO}%
\colorbox{green}{\color[gray]{0.75}FO}%
\colorbox{green}{\color[gray]{0.75}FO}%
\colorbox{green}{\color[gray]{0.75}FO}%
\colorbox{green}{\color[gray]{0.75}FO}%
\colorbox{green}{\color[gray]{0.75}FO}%
\colorbox{green}{\color[gray]{0.75}FO}%
\colorbox{green}{\color[gray]{0.75}FO}%
\colorbox{green}{\color[gray]{0.75}FO}%
\colorbox{green}{\color[gray]{0.75}FO}%
\colorbox{green}{\color[gray]{0.75}FO}%
\colorbox{green}{\color[gray]{0.75}FO}%
\colorbox{green}{\color[gray]{0.75}FO}%
\colorbox{green}{\color[gray]{0.75}FO}%
\colorbox{green}{\color[gray]{0.75}FO}%
\colorbox{green}{\color[gray]{0.75}FO}%
\colorbox{green}{\color[gray]{0.75}FO}%
\colorbox{green}{\color[gray]{0.75}FO}%
\colorbox{green}{\color[gray]{0.75}FO}%
\colorbox{green}{\color[gray]{0.75}FO}%
\colorbox{green}{\color[gray]{0.75}FO}%
\colorbox{green}{\color[gray]{0.75}FO}%
\colorbox{green}{\color[gray]{0.75}FO}%
\colorbox{green}{\color[gray]{0.75}FO}%
\colorbox{green}{\color[gray]{0.75}FO}%
\colorbox{green}{\color[gray]{0.75}FO}%
\colorbox{green}{\color[gray]{0.75}FO}%
\colorbox{green}{\color[gray]{0.75}FO}%
\colorbox{green}{\color[gray]{0.75}FO}%
\colorbox{green}{\color[gray]{0.75}FO}%
\colorbox{green}{\color[gray]{0.75}FO}%
\colorbox{green}{\color[gray]{0.75}FO}%
\colorbox{green}{\color[gray]{0.75}FO}%
\colorbox{green}{\color[gray]{0.75}FO}%
\colorbox{green}{\color[gray]{0.75}FO}%
\colorbox{green}{\color[gray]{0.75}FO}%
\colorbox{green}{\color[gray]{0.75}FO}%
\colorbox{green}{\color[gray]{0.75}FO}%
\colorbox{green}{\color[gray]{0.75}FO}%
\colorbox{green}{\color[gray]{0.75}FO}%
\colorbox{green}{\color[gray]{0.75}FO}%
\colorbox{green}{\color[gray]{0.75}FO}%
\colorbox{green}{\color[gray]{0.75}FO}%
\colorbox{green}{\color[gray]{0.75}FO}%
\colorbox{green}{\color[gray]{0.75}FO}%
\colorbox{green}{\color[gray]{0.75}FO}%
\colorbox{green}{\color[gray]{0.75}FO}%
\colorbox{green}{\color[gray]{0.75}FO}%
\colorbox{green}{\color[gray]{0.75}FO}%
\colorbox{green}{\color[gray]{0.75}FO}%
\colorbox{green}{\color[gray]{0.75}FO}%
\colorbox{green}{\color[gray]{0.75}FO}%
\colorbox{green}{\color[gray]{0.75}FO}%
\colorbox{green}{\color[gray]{0.75}FO}%
\colorbox{green}{\color[gray]{0.75}FO}%
\colorbox{green}{\color[gray]{0.75}FO}%
\colorbox{green}{\color[gray]{0.75}FO}%
\colorbox{green}{\color[gray]{0.75}FO}%
\colorbox{green}{\color[gray]{0.75}FO}%
\colorbox{green}{\color[gray]{0.75}FO}%
\colorbox{green}{\color[gray]{0.75}FO}%
\colorbox{green}{\color[gray]{0.75}FO}%
\\
\colorbox{green}{\color[gray]{0.75}FO}%
\colorbox{green}{\color[gray]{0.75}FO}%
\colorbox{green}{\color[gray]{0.75}FO}%
\colorbox{green}{\color[gray]{0.75}FO}%
\colorbox{green}{\color[gray]{0.75}FO}%
\colorbox{green}{\color[gray]{0.75}FO}%
\colorbox{green}{\color[gray]{0.75}FO}%
\colorbox{green}{\color[gray]{0.75}FO}%
\colorbox{green}{\color[gray]{0.75}FO}%
\colorbox{green}{\color[gray]{0.75}FO}%
\colorbox{green}{\color[gray]{0.75}FO}%
\colorbox{green}{\color[gray]{0.75}FO}%
\colorbox{green}{\color[gray]{0.75}FO}%
\colorbox{green}{\color[gray]{0.75}FO}%
\colorbox{green}{\color[gray]{0.75}FO}%
\colorbox{green}{\color[gray]{0.75}FO}%
\colorbox{green}{\color[gray]{0.75}FO}%
\colorbox{green}{\color[gray]{0.75}FO}%
\colorbox{green}{\color[gray]{0.75}FO}%
\colorbox{green}{\color[gray]{0.75}FO}%
\colorbox{green}{\color[gray]{0.75}FO}%
\colorbox{green}{\color[gray]{0.75}FO}%
\colorbox{green}{\color[gray]{0.75}FO}%
\colorbox{green}{\color[gray]{0.75}FO}%
\colorbox{green}{\color[gray]{0.75}FO}%
\colorbox{green}{\color[gray]{0.75}FO}%
\colorbox{green}{\color[gray]{0.75}FO}%
\colorbox{green}{\color[gray]{0.75}FO}%
\colorbox{green}{\color[gray]{0.75}FO}%
\colorbox{green}{\color[gray]{0.75}FO}%
\colorbox{green}{\color[gray]{0.75}FO}%
\colorbox{green}{\color[gray]{0.75}FO}%
\colorbox{green}{\color[gray]{0.75}FO}%
\colorbox{green}{\color[gray]{0.75}FO}%
\colorbox{green}{\color[gray]{0.75}FO}%
\colorbox{green}{\color[gray]{0.75}FO}%
\colorbox{green}{\color[gray]{0.75}FO}%
\colorbox{green}{\color[gray]{0.75}FO}%
\colorbox{green}{\color[gray]{0.75}FO}%
\colorbox{green}{\color[gray]{0.75}FO}%
\colorbox{green}{\color[gray]{0.75}FO}%
\colorbox{green}{\color[gray]{0.75}FO}%
\colorbox{green}{\color[gray]{0.75}FO}%
\colorbox{green}{\color[gray]{0.75}FO}%
\colorbox{green}{\color[gray]{0.75}FO}%
\colorbox{green}{\color[gray]{0.75}FO}%
\colorbox{green}{\color[gray]{0.75}FO}%
\colorbox{green}{\color[gray]{0.75}FO}%
\colorbox{green}{\color[gray]{0.75}FO}%
\colorbox{green}{\color[gray]{0.75}FO}%
\colorbox{green}{\color[gray]{0.75}FO}%
\colorbox{green}{\color[gray]{0.75}FO}%
\colorbox{green}{\color[gray]{0.75}FO}%
\colorbox{green}{\color[gray]{0.75}FO}%
\colorbox{green}{\color[gray]{0.75}FO}%
\colorbox{green}{\color[gray]{0.75}FO}%
\colorbox{green}{\color[gray]{0.75}FO}%
\colorbox{green}{\color[gray]{0.75}FO}%
\colorbox{green}{\color[gray]{0.75}FO}%
\colorbox{green}{\color[gray]{0.75}FO}%
\colorbox{green}{\color[gray]{0.75}FO}%
\colorbox{green}{\color[gray]{0.75}FO}%
\colorbox{green}{\color[gray]{0.75}FO}%
\colorbox{green}{\color[gray]{0.75}FO}%
\colorbox{green}{\color[gray]{0.75}FO}%
\colorbox{green}{\color[gray]{0.75}FO}%
\colorbox{green}{\color[gray]{0.75}FO}%
\colorbox{green}{\color[gray]{0.75}FO}%
\colorbox{green}{\color[gray]{0.75}FO}%
\colorbox{green}{\color[gray]{0.75}FO}%
\colorbox{green}{\color[gray]{0.75}FO}%
\colorbox{green}{\color[gray]{0.75}FO}%
\colorbox{green}{\color[gray]{0.75}FO}%
\colorbox{green}{\color[gray]{0.75}FO}%
\colorbox{green}{\color[gray]{0.75}FO}%
\colorbox{green}{\color[gray]{0.75}FO}%
\colorbox{green}{\color[gray]{0.75}FO}%
\colorbox{green}{\color[gray]{0.75}FO}%
\colorbox{green}{\color[gray]{0.75}FO}%
\colorbox{green}{\color[gray]{0.75}FO}%
\colorbox{green}{\color[gray]{0.75}FO}%
\colorbox{green}{\color[gray]{0.75}FO}%
\colorbox{green}{\color[gray]{0.75}FO}%
\colorbox{green}{\color[gray]{0.75}FO}%
\colorbox{green}{\color[gray]{0.75}FO}%
\colorbox{green}{\color[gray]{0.75}FO}%
\colorbox{green}{\color[gray]{0.75}FO}%
\colorbox{green}{\color[gray]{0.75}FO}%
\colorbox{green}{\color[gray]{0.75}FO}%
\colorbox{green}{\color[gray]{0.75}FO}%
\colorbox{green}{\color[gray]{0.75}FO}%
\colorbox{green}{\color[gray]{0.75}FO}%
\colorbox{green}{\color[gray]{0.75}FO}%
\colorbox{green}{\color[gray]{0.75}FO}%
\colorbox{green}{\color[gray]{0.75}FO}%
\colorbox{green}{\color[gray]{0.75}FO}%
\colorbox{green}{\color[gray]{0.75}FO}%
\colorbox{green}{\color[gray]{0.75}FO}%
\colorbox{green}{\color[gray]{0.75}FO}%
\colorbox{green}{\color[gray]{0.75}FO}%
\\
\colorbox{green}{\color[gray]{0.75}FO}%
\colorbox{green}{\color[gray]{0.75}FO}%
\colorbox{green}{\color[gray]{0.75}FO}%
\colorbox{green}{\color[gray]{0.75}FO}%
\colorbox{green}{\color[gray]{0.75}FO}%
\colorbox{green}{\color[gray]{0.75}FO}%
\colorbox{green}{\color[gray]{0.75}FO}%
\colorbox{green}{\color[gray]{0.75}FO}%
\colorbox{green}{\color[gray]{0.75}FO}%
\colorbox{green}{\color[gray]{0.75}FO}%
\colorbox{green}{\color[gray]{0.75}FO}%
\colorbox{green}{\color[gray]{0.75}FO}%
\colorbox{green}{\color[gray]{0.75}FO}%
\colorbox{green}{\color[gray]{0.75}FO}%
\colorbox{green}{\color[gray]{0.75}FO}%
\colorbox{green}{\color[gray]{0.75}FO}%
\colorbox{green}{\color[gray]{0.75}FO}%
\colorbox{green}{\color[gray]{0.75}FO}%
\colorbox{green}{\color[gray]{0.75}FO}%
\colorbox{green}{\color[gray]{0.75}FO}%
\colorbox{green}{\color[gray]{0.75}FO}%
\colorbox{green}{\color[gray]{0.75}FO}%
\colorbox{green}{\color[gray]{0.75}FO}%
\colorbox{green}{\color[gray]{0.75}FO}%
\colorbox{green}{\color[gray]{0.75}FO}%
\colorbox{green}{\color[gray]{0.75}FO}%
\colorbox{green}{\color[gray]{0.75}FO}%
\colorbox{green}{\color[gray]{0.75}FO}%
\colorbox{green}{\color[gray]{0.75}FO}%
\colorbox{green}{\color[gray]{0.75}FO}%
\colorbox{green}{\color[gray]{0.75}FO}%
\colorbox{green}{\color[gray]{0.75}FO}%
\colorbox{green}{\color[gray]{0.75}FO}%
\colorbox{green}{\color[gray]{0.75}FO}%
\colorbox{green}{\color[gray]{0.75}FO}%
\colorbox{green}{\color[gray]{0.75}FO}%
\colorbox{green}{\color[gray]{0.75}FO}%
\colorbox{green}{\color[gray]{0.75}FO}%
\colorbox{green}{\color[gray]{0.75}FO}%
\colorbox{green}{\color[gray]{0.75}FO}%
\colorbox{green}{\color[gray]{0.75}FO}%
\colorbox{green}{\color[gray]{0.75}FO}%
\colorbox{green}{\color[gray]{0.75}FO}%
\colorbox{green}{\color[gray]{0.75}FO}%
\colorbox{green}{\color[gray]{0.75}FO}%
\colorbox{green}{\color[gray]{0.75}FO}%
\colorbox{green}{\color[gray]{0.75}FO}%
\colorbox{green}{\color[gray]{0.75}FO}%
\colorbox{green}{\color[gray]{0.75}FO}%
\colorbox{green}{\color[gray]{0.75}FO}%
\colorbox{green}{\color[gray]{0.75}FO}%
\colorbox{green}{\color[gray]{0.75}FO}%
\colorbox{green}{\color[gray]{0.75}FO}%
\colorbox{green}{\color[gray]{0.75}FO}%
\colorbox{green}{\color[gray]{0.75}FO}%
\colorbox{green}{\color[gray]{0.75}FO}%
\colorbox{green}{\color[gray]{0.75}FO}%
\colorbox{green}{\color[gray]{0.75}FO}%
\colorbox{green}{\color[gray]{0.75}FO}%
\colorbox{green}{\color[gray]{0.75}FO}%
\colorbox{green}{\color[gray]{0.75}FO}%
\colorbox{green}{\color[gray]{0.75}FO}%
\colorbox{green}{\color[gray]{0.75}FO}%
\colorbox{green}{\color[gray]{0.75}FO}%
\colorbox{green}{\color[gray]{0.75}FO}%
\colorbox{green}{\color[gray]{0.75}FO}%
\colorbox{green}{\color[gray]{0.75}FO}%
\colorbox{green}{\color[gray]{0.75}FO}%
\colorbox{green}{\color[gray]{0.75}FO}%
\colorbox{green}{\color[gray]{0.75}FO}%
\colorbox{green}{\color[gray]{0.75}FO}%
\colorbox{green}{\color[gray]{0.75}FO}%
\colorbox{green}{\color[gray]{0.75}FO}%
\colorbox{green}{\color[gray]{0.75}FO}%
\colorbox{green}{\color[gray]{0.75}FO}%
\colorbox{green}{\color[gray]{0.75}FO}%
\colorbox{green}{\color[gray]{0.75}FO}%
\colorbox{green}{\color[gray]{0.75}FO}%
\colorbox{green}{\color[gray]{0.75}FO}%
\colorbox{green}{\color[gray]{0.75}FO}%
\colorbox{green}{\color[gray]{0.75}FO}%
\colorbox{green}{\color[gray]{0.75}FO}%
\colorbox{green}{\color[gray]{0.75}FO}%
\colorbox{green}{\color[gray]{0.75}FO}%
\colorbox{green}{\color[gray]{0.75}FO}%
\colorbox{green}{\color[gray]{0.75}FO}%
\colorbox{green}{\color[gray]{0.75}FO}%
\colorbox{green}{\color[gray]{0.75}FO}%
\colorbox{green}{\color[gray]{0.75}FO}%
\colorbox{green}{\color[gray]{0.75}FO}%
\colorbox{green}{\color[gray]{0.75}FO}%
\colorbox{green}{\color[gray]{0.75}FO}%
\colorbox{green}{\color[gray]{0.75}FO}%
\colorbox{green}{\color[gray]{0.75}FO}%
\colorbox{green}{\color[gray]{0.75}FO}%
\colorbox{green}{\color[gray]{0.75}FO}%
\colorbox{green}{\color[gray]{0.75}FO}%
\colorbox{green}{\color[gray]{0.75}FO}%
\colorbox{green}{\color[gray]{0.75}FO}%
\colorbox{green}{\color[gray]{0.75}FO}%
\\
\colorbox{green}{\color[gray]{0.75}FO}%
\colorbox{green}{\color[gray]{0.75}FO}%
\colorbox{green}{\color[gray]{0.75}FO}%
\colorbox{green}{\color[gray]{0.75}FO}%
\colorbox{green}{\color[gray]{0.75}FO}%
\colorbox{green}{\color[gray]{0.75}FO}%
\colorbox{green}{\color[gray]{0.75}FO}%
\colorbox{green}{\color[gray]{0.75}FO}%
\colorbox{green}{\color[gray]{0.75}FO}%
\colorbox{green}{\color[gray]{0.75}FO}%
\colorbox{green}{\color[gray]{0.75}FO}%
\colorbox{green}{\color[gray]{0.75}FO}%
\colorbox{green}{\color[gray]{0.75}FO}%
\colorbox{green}{\color[gray]{0.75}FO}%
\colorbox{green}{\color[gray]{0.75}FO}%
\colorbox{green}{\color[gray]{0.75}FO}%
\colorbox{green}{\color[gray]{0.75}FO}%
\colorbox{green}{\color[gray]{0.75}FO}%
\colorbox{green}{\color[gray]{0.75}FO}%
\colorbox{green}{\color[gray]{0.75}FO}%
\colorbox{green}{\color[gray]{0.75}FO}%
\colorbox{green}{\color[gray]{0.75}FO}%
\colorbox{green}{\color[gray]{0.75}FO}%
\colorbox{green}{\color[gray]{0.75}FO}%
\colorbox{green}{\color[gray]{0.75}FO}%
\colorbox{green}{\color[gray]{0.75}FO}%
\colorbox{green}{\color[gray]{0.75}FO}%
\colorbox{green}{\color[gray]{0.75}FO}%
\colorbox{green}{\color[gray]{0.75}FO}%
\colorbox{green}{\color[gray]{0.75}FO}%
\colorbox{green}{\color[gray]{0.75}FO}%
\colorbox{green}{\color[gray]{0.75}FO}%
\colorbox{green}{\color[gray]{0.75}FO}%
\colorbox{green}{\color[gray]{0.75}FO}%
\colorbox{green}{\color[gray]{0.75}FO}%
\colorbox{green}{\color[gray]{0.75}FO}%
\colorbox{green}{\color[gray]{0.75}FO}%
\colorbox{green}{\color[gray]{0.75}FO}%
\colorbox{green}{\color[gray]{0.75}FO}%
\colorbox{green}{\color[gray]{0.75}FO}%
\colorbox{green}{\color[gray]{0.75}FO}%
\colorbox{green}{\color[gray]{0.75}FO}%
\colorbox{green}{\color[gray]{0.75}FO}%
\colorbox{green}{\color[gray]{0.75}FO}%
\colorbox{green}{\color[gray]{0.75}FO}%
\colorbox{green}{\color[gray]{0.75}FO}%
\colorbox{green}{\color[gray]{0.75}FO}%
\colorbox{green}{\color[gray]{0.75}FO}%
\colorbox{green}{\color[gray]{0.75}FO}%
\colorbox{green}{\color[gray]{0.75}FO}%
\colorbox{green}{\color[gray]{0.75}FO}%
\colorbox{green}{\color[gray]{0.75}FO}%
\colorbox{green}{\color[gray]{0.75}FO}%
\colorbox{green}{\color[gray]{0.75}FO}%
\colorbox{green}{\color[gray]{0.75}FO}%
\colorbox{green}{\color[gray]{0.75}FO}%
\colorbox{green}{\color[gray]{0.75}FO}%
\colorbox{green}{\color[gray]{0.75}FO}%
\colorbox{green}{\color[gray]{0.75}FO}%
\colorbox{green}{\color[gray]{0.75}FO}%
\colorbox{green}{\color[gray]{0.75}FO}%
\colorbox{green}{\color[gray]{0.75}FO}%
\colorbox{green}{\color[gray]{0.75}FO}%
\colorbox{green}{\color[gray]{0.75}FO}%
\colorbox{green}{\color[gray]{0.75}FO}%
\colorbox{green}{\color[gray]{0.75}FO}%
\colorbox{green}{\color[gray]{0.75}FO}%
\colorbox{green}{\color[gray]{0.75}FO}%
\colorbox{green}{\color[gray]{0.75}FO}%
\colorbox{green}{\color[gray]{0.75}FO}%
\colorbox{green}{\color[gray]{0.75}FO}%
\colorbox{green}{\color[gray]{0.75}FO}%
\colorbox{green}{\color[gray]{0.75}FO}%
\colorbox{green}{\color[gray]{0.75}FO}%
\colorbox{green}{\color[gray]{0.75}FO}%
\colorbox{green}{\color[gray]{0.75}FO}%
\colorbox{green}{\color[gray]{0.75}FO}%
\colorbox{green}{\color[gray]{0.75}FO}%
\colorbox{green}{\color[gray]{0.75}FO}%
\colorbox{green}{\color[gray]{0.75}FO}%
\colorbox{green}{\color[gray]{0.75}FO}%
\colorbox{green}{\color[gray]{0.75}FO}%
\colorbox{green}{\color[gray]{0.75}FO}%
\colorbox{green}{\color[gray]{0.75}FO}%
\colorbox{green}{\color[gray]{0.75}FO}%
\colorbox{green}{\color[gray]{0.75}FO}%
\colorbox{green}{\color[gray]{0.75}FO}%
\colorbox{green}{\color[gray]{0.75}FO}%
\colorbox{green}{\color[gray]{0.75}FO}%
\colorbox{green}{\color[gray]{0.75}FO}%
\colorbox{green}{\color[gray]{0.75}FO}%
\colorbox{green}{\color[gray]{0.75}FO}%
\colorbox{green}{\color[gray]{0.75}FO}%
\colorbox{green}{\color[gray]{0.75}FO}%
\colorbox{green}{\color[gray]{0.75}FO}%
\colorbox{green}{\color[gray]{0.75}FO}%
\colorbox{green}{\color[gray]{0.75}FO}%
\colorbox{green}{\color[gray]{0.75}FO}%
\colorbox{green}{\color[gray]{0.75}FO}%
\colorbox{green}{\color[gray]{0.75}FO}%
\\
\colorbox{green}{\color[gray]{0.75}FO}%
\colorbox{green}{\color[gray]{0.75}FO}%
\colorbox{green}{\color[gray]{0.75}FO}%
\colorbox{green}{\color[gray]{0.75}FO}%
\colorbox{green}{\color[gray]{0.75}FO}%
\colorbox{green}{\color[gray]{0.75}FO}%
\colorbox{green}{\color[gray]{0.75}FO}%
\colorbox{green}{\color[gray]{0.75}FO}%
\colorbox{green}{\color[gray]{0.75}FO}%
\colorbox{green}{\color[gray]{0.75}FO}%
\colorbox{green}{\color[gray]{0.75}FO}%
\colorbox{green}{\color[gray]{0.75}FO}%
\colorbox{green}{\color[gray]{0.75}FO}%
\colorbox{green}{\color[gray]{0.75}FO}%
\colorbox{green}{\color[gray]{0.75}FO}%
\colorbox{green}{\color[gray]{0.75}FO}%
\colorbox{green}{\color[gray]{0.75}FO}%
\colorbox{green}{\color[gray]{0.75}FO}%
\colorbox{green}{\color[gray]{0.75}FO}%
\colorbox{green}{\color[gray]{0.75}FO}%
\colorbox{green}{\color[gray]{0.75}FO}%
\colorbox{green}{\color[gray]{0.75}FO}%
\colorbox{green}{\color[gray]{0.75}FO}%
\colorbox{green}{\color[gray]{0.75}FO}%
\colorbox{green}{\color[gray]{0.75}FO}%
\colorbox{green}{\color[gray]{0.75}FO}%
\colorbox{green}{\color[gray]{0.75}FO}%
\colorbox{green}{\color[gray]{0.75}FO}%
\colorbox{green}{\color[gray]{0.75}FO}%
\colorbox{green}{\color[gray]{0.75}FO}%
\colorbox{green}{\color[gray]{0.75}FO}%
\colorbox{green}{\color[gray]{0.75}FO}%
\colorbox{green}{\color[gray]{0.75}FO}%
\colorbox{green}{\color[gray]{0.75}FO}%
\colorbox{green}{\color[gray]{0.75}FO}%
\colorbox{green}{\color[gray]{0.75}FO}%
\colorbox{green}{\color[gray]{0.75}FO}%
\colorbox{green}{\color[gray]{0.75}FO}%
\colorbox{green}{\color[gray]{0.75}FO}%
\colorbox{green}{\color[gray]{0.75}FO}%
\colorbox{green}{\color[gray]{0.75}FO}%
\colorbox{green}{\color[gray]{0.75}FO}%
\colorbox{green}{\color[gray]{0.75}FO}%
\colorbox{green}{\color[gray]{0.75}FO}%
\colorbox{green}{\color[gray]{0.75}FO}%
\colorbox{green}{\color[gray]{0.75}FO}%
\colorbox{green}{\color[gray]{0.75}FO}%
\colorbox{green}{\color[gray]{0.75}FO}%
\colorbox{green}{\color[gray]{0.75}FO}%
\colorbox{green}{\color[gray]{0.75}FO}%
\colorbox{green}{\color[gray]{0.75}FO}%
\colorbox{green}{\color[gray]{0.75}FO}%
\colorbox{green}{\color[gray]{0.75}FO}%
\colorbox{green}{\color[gray]{0.75}FO}%
\colorbox{green}{\color[gray]{0.75}FO}%
\colorbox{green}{\color[gray]{0.75}FO}%
\colorbox{green}{\color[gray]{0.75}FO}%
\colorbox{green}{\color[gray]{0.75}FO}%
\colorbox{green}{\color[gray]{0.75}FO}%
\colorbox{green}{\color[gray]{0.75}FO}%
\colorbox{green}{\color[gray]{0.75}FO}%
\colorbox{green}{\color[gray]{0.75}FO}%
\colorbox{green}{\color[gray]{0.75}FO}%
\colorbox{green}{\color[gray]{0.75}FO}%
\colorbox{green}{\color[gray]{0.75}FO}%
\colorbox{green}{\color[gray]{0.75}FO}%
\colorbox{green}{\color[gray]{0.75}FO}%
\colorbox{green}{\color[gray]{0.75}FO}%
\colorbox{green}{\color[gray]{0.75}FO}%
\colorbox{green}{\color[gray]{0.75}FO}%
\colorbox{green}{\color[gray]{0.75}FO}%
\colorbox{green}{\color[gray]{0.75}FO}%
\colorbox{green}{\color[gray]{0.75}FO}%
\colorbox{green}{\color[gray]{0.75}FO}%
\colorbox{green}{\color[gray]{0.75}FO}%
\colorbox{green}{\color[gray]{0.75}FO}%
\colorbox{green}{\color[gray]{0.75}FO}%
\colorbox{green}{\color[gray]{0.75}FO}%
\colorbox{green}{\color[gray]{0.75}FO}%
\colorbox{green}{\color[gray]{0.75}FO}%
\colorbox{green}{\color[gray]{0.75}FO}%
\colorbox{green}{\color[gray]{0.75}FO}%
\colorbox{green}{\color[gray]{0.75}FO}%
\colorbox{green}{\color[gray]{0.75}FO}%
\colorbox{green}{\color[gray]{0.75}FO}%
\colorbox{green}{\color[gray]{0.75}FO}%
\colorbox{green}{\color[gray]{0.75}FO}%
\colorbox{green}{\color[gray]{0.75}FO}%
\colorbox{green}{\color[gray]{0.75}FO}%
\colorbox{green}{\color[gray]{0.75}FO}%
\colorbox{green}{\color[gray]{0.75}FO}%
\colorbox{green}{\color[gray]{0.75}FO}%
\colorbox{green}{\color[gray]{0.75}FO}%
\colorbox{green}{\color[gray]{0.75}FO}%
\colorbox{green}{\color[gray]{0.75}FO}%
\colorbox{green}{\color[gray]{0.75}FO}%
\colorbox{green}{\color[gray]{0.75}FO}%
\colorbox{green}{\color[gray]{0.75}FO}%
\colorbox{green}{\color[gray]{0.75}FO}%
\colorbox{green}{\color[gray]{0.75}FO}%
\\
\colorbox{green}{\color[gray]{0.75}FO}%
\colorbox{green}{\color[gray]{0.75}FO}%
\colorbox{green}{\color[gray]{0.75}FO}%
\colorbox{green}{\color[gray]{0.75}FO}%
\colorbox{green}{\color[gray]{0.75}FO}%
\colorbox{green}{\color[gray]{0.75}FO}%
\colorbox{green}{\color[gray]{0.75}FO}%
\colorbox{green}{\color[gray]{0.75}FO}%
\colorbox{green}{\color[gray]{0.75}FO}%
\colorbox{green}{\color[gray]{0.75}FO}%
\colorbox{green}{\color[gray]{0.75}FO}%
\colorbox{green}{\color[gray]{0.75}FO}%
\colorbox{green}{\color[gray]{0.75}FO}%
\colorbox{green}{\color[gray]{0.75}FO}%
\colorbox{green}{\color[gray]{0.75}FO}%
\colorbox{green}{\color[gray]{0.75}FO}%
\colorbox{green}{\color[gray]{0.75}FO}%
\colorbox{green}{\color[gray]{0.75}FO}%
\colorbox{green}{\color[gray]{0.75}FO}%
\colorbox{green}{\color[gray]{0.75}FO}%
\colorbox{green}{\color[gray]{0.75}FO}%
\colorbox{green}{\color[gray]{0.75}FO}%
\colorbox{green}{\color[gray]{0.75}FO}%
\colorbox{green}{\color[gray]{0.75}FO}%
\colorbox{green}{\color[gray]{0.75}FO}%
\colorbox{green}{\color[gray]{0.75}FO}%
\colorbox{green}{\color[gray]{0.75}FO}%
\colorbox{green}{\color[gray]{0.75}FO}%
\colorbox{green}{\color[gray]{0.75}FO}%
\colorbox{green}{\color[gray]{0.75}FO}%
\colorbox{green}{\color[gray]{0.75}FO}%
\colorbox{green}{\color[gray]{0.75}FO}%
\colorbox{green}{\color[gray]{0.75}FO}%
\colorbox{green}{\color[gray]{0.75}FO}%
\colorbox{green}{\color[gray]{0.75}FO}%
\colorbox{green}{\color[gray]{0.75}FO}%
\colorbox{green}{\color[gray]{0.75}FO}%
\colorbox{green}{\color[gray]{0.75}FO}%
\colorbox{green}{\color[gray]{0.75}FO}%
\colorbox{green}{\color[gray]{0.75}FO}%
\colorbox{green}{\color[gray]{0.75}FO}%
\colorbox{green}{\color[gray]{0.75}FO}%
\colorbox{green}{\color[gray]{0.75}FO}%
\colorbox{green}{\color[gray]{0.75}FO}%
\colorbox{green}{\color[gray]{0.75}FO}%
\colorbox{green}{\color[gray]{0.75}FO}%
\colorbox{green}{\color[gray]{0.75}FO}%
\colorbox{green}{\color[gray]{0.75}FO}%
\colorbox{green}{\color[gray]{0.75}FO}%
\colorbox{green}{\color[gray]{0.75}FO}%
\colorbox{green}{\color[gray]{0.75}FO}%
\colorbox{green}{\color[gray]{0.75}FO}%
\colorbox{green}{\color[gray]{0.75}FO}%
\colorbox{green}{\color[gray]{0.75}FO}%
\colorbox{green}{\color[gray]{0.75}FO}%
\colorbox{green}{\color[gray]{0.75}FO}%
\colorbox{green}{\color[gray]{0.75}FO}%
\colorbox{green}{\color[gray]{0.75}FO}%
\colorbox{green}{\color[gray]{0.75}FO}%
\colorbox{green}{\color[gray]{0.75}FO}%
\colorbox{green}{\color[gray]{0.75}FO}%
\colorbox{green}{\color[gray]{0.75}FO}%
\colorbox{green}{\color[gray]{0.75}FO}%
\colorbox{green}{\color[gray]{0.75}FO}%
\colorbox{green}{\color[gray]{0.75}FO}%
\colorbox{green}{\color[gray]{0.75}FO}%
\colorbox{green}{\color[gray]{0.75}FO}%
\colorbox{green}{\color[gray]{0.75}FO}%
\colorbox{green}{\color[gray]{0.75}FO}%
\colorbox{green}{\color[gray]{0.75}FO}%
\colorbox{green}{\color[gray]{0.75}FO}%
\colorbox{green}{\color[gray]{0.75}FO}%
\colorbox{green}{\color[gray]{0.75}FO}%
\colorbox{green}{\color[gray]{0.75}FO}%
\colorbox{green}{\color[gray]{0.75}FO}%
\colorbox{green}{\color[gray]{0.75}FO}%
\colorbox{green}{\color[gray]{0.75}FO}%
\colorbox{green}{\color[gray]{0.75}FO}%
\colorbox{green}{\color[gray]{0.75}FO}%
\colorbox{green}{\color[gray]{0.75}FO}%
\colorbox{green}{\color[gray]{0.75}FO}%
\colorbox{green}{\color[gray]{0.75}FO}%
\colorbox{green}{\color[gray]{0.75}FO}%
\colorbox{green}{\color[gray]{0.75}FO}%
\colorbox{green}{\color[gray]{0.75}FO}%
\colorbox{green}{\color[gray]{0.75}FO}%
\colorbox{green}{\color[gray]{0.75}FO}%
\colorbox{green}{\color[gray]{0.75}FO}%
\colorbox{green}{\color[gray]{0.75}FO}%
\colorbox{green}{\color[gray]{0.75}FO}%
\colorbox{green}{\color[gray]{0.75}FO}%
\colorbox{green}{\color[gray]{0.75}FO}%
\colorbox{green}{\color[gray]{0.75}FO}%
\colorbox{green}{\color[gray]{0.75}FO}%
\colorbox{green}{\color[gray]{0.75}FO}%
\colorbox{green}{\color[gray]{0.75}FO}%
\colorbox{green}{\color[gray]{0.75}FO}%
\colorbox{green}{\color[gray]{0.75}FO}%
\colorbox{green}{\color[gray]{0.75}FO}%
\colorbox{green}{\color[gray]{0.75}FO}%
\\
\colorbox{green}{\color[gray]{0.75}FO}%
\colorbox{green}{\color[gray]{0.75}FO}%
\colorbox{green}{\color[gray]{0.75}FO}%
\colorbox{green}{\color[gray]{0.75}FO}%
\colorbox{green}{\color[gray]{0.75}FO}%
\colorbox{green}{\color[gray]{0.75}FO}%
\colorbox{green}{\color[gray]{0.75}FO}%
\colorbox{green}{\color[gray]{0.75}FO}%
\colorbox{green}{\color[gray]{0.75}FO}%
\colorbox{green}{\color[gray]{0.75}FO}%
\colorbox{green}{\color[gray]{0.75}FO}%
\colorbox{green}{\color[gray]{0.75}FO}%
\colorbox{green}{\color[gray]{0.75}FO}%
\colorbox{green}{\color[gray]{0.75}FO}%
\colorbox{green}{\color[gray]{0.75}FO}%
\colorbox{green}{\color[gray]{0.75}FO}%
\colorbox{green}{\color[gray]{0.75}FO}%
\colorbox{green}{\color[gray]{0.75}FO}%
\colorbox{green}{\color[gray]{0.75}FO}%
\colorbox{green}{\color[gray]{0.75}FO}%
\colorbox{green}{\color[gray]{0.75}FO}%
\colorbox{green}{\color[gray]{0.75}FO}%
\colorbox{green}{\color[gray]{0.75}FO}%
\colorbox{green}{\color[gray]{0.75}FO}%
\colorbox{green}{\color[gray]{0.75}FO}%
\colorbox{green}{\color[gray]{0.75}FO}%
\colorbox{green}{\color[gray]{0.75}FO}%
\colorbox{green}{\color[gray]{0.75}FO}%
\colorbox{green}{\color[gray]{0.75}FO}%
\colorbox{green}{\color[gray]{0.75}FO}%
\colorbox{green}{\color[gray]{0.75}FO}%
\colorbox{green}{\color[gray]{0.75}FO}%
\colorbox{green}{\color[gray]{0.75}FO}%
\colorbox{green}{\color[gray]{0.75}FO}%
\colorbox{green}{\color[gray]{0.75}FO}%
\colorbox{green}{\color[gray]{0.75}FO}%
\colorbox{green}{\color[gray]{0.75}FO}%
\colorbox{green}{\color[gray]{0.75}FO}%
\colorbox{green}{\color[gray]{0.75}FO}%
\colorbox{green}{\color[gray]{0.75}FO}%
\colorbox{green}{\color[gray]{0.75}FO}%
\colorbox{green}{\color[gray]{0.75}FO}%
\colorbox{green}{\color[gray]{0.75}FO}%
\colorbox{green}{\color[gray]{0.75}FO}%
\colorbox{green}{\color[gray]{0.75}FO}%
\colorbox{green}{\color[gray]{0.75}FO}%
\colorbox{green}{\color[gray]{0.75}FO}%
\colorbox{green}{\color[gray]{0.75}FO}%
\colorbox{green}{\color[gray]{0.75}FO}%
\colorbox{green}{\color[gray]{0.75}FO}%
\colorbox{green}{\color[gray]{0.75}FO}%
\colorbox{green}{\color[gray]{0.75}FO}%
\colorbox{green}{\color[gray]{0.75}FO}%
\colorbox{green}{\color[gray]{0.75}FO}%
\colorbox{green}{\color[gray]{0.75}FO}%
\colorbox{green}{\color[gray]{0.75}FO}%
\colorbox{green}{\color[gray]{0.75}FO}%
\colorbox{green}{\color[gray]{0.75}FO}%
\colorbox{green}{\color[gray]{0.75}FO}%
\colorbox{green}{\color[gray]{0.75}FO}%
\colorbox{green}{\color[gray]{0.75}FO}%
\colorbox{green}{\color[gray]{0.75}FO}%
\colorbox{green}{\color[gray]{0.75}FO}%
\colorbox{green}{\color[gray]{0.75}FO}%
\colorbox{green}{\color[gray]{0.75}FO}%
\colorbox{green}{\color[gray]{0.75}FO}%
\colorbox{green}{\color[gray]{0.75}FO}%
\colorbox{green}{\color[gray]{0.75}FO}%
\colorbox{green}{\color[gray]{0.75}FO}%
\colorbox{green}{\color[gray]{0.75}FO}%
\colorbox{green}{\color[gray]{0.75}FO}%
\colorbox{green}{\color[gray]{0.75}FO}%
\colorbox{green}{\color[gray]{0.75}FO}%
\colorbox{green}{\color[gray]{0.75}FO}%
\colorbox{green}{\color[gray]{0.75}FO}%
\colorbox{green}{\color[gray]{0.75}FO}%
\colorbox{green}{\color[gray]{0.75}FO}%
\colorbox{green}{\color[gray]{0.75}FO}%
\colorbox{green}{\color[gray]{0.75}FO}%
\colorbox{green}{\color[gray]{0.75}FO}%
\colorbox{green}{\color[gray]{0.75}FO}%
\colorbox{green}{\color[gray]{0.75}FO}%
\colorbox{green}{\color[gray]{0.75}FO}%
\colorbox{green}{\color[gray]{0.75}FO}%
\colorbox{green}{\color[gray]{0.75}FO}%
\colorbox{green}{\color[gray]{0.75}FO}%
\colorbox{green}{\color[gray]{0.75}FO}%
\colorbox{green}{\color[gray]{0.75}FO}%
\colorbox{green}{\color[gray]{0.75}FO}%
\colorbox{green}{\color[gray]{0.75}FO}%
\colorbox{green}{\color[gray]{0.75}FO}%
\colorbox{green}{\color[gray]{0.75}FO}%
\colorbox{green}{\color[gray]{0.75}FO}%
\colorbox{green}{\color[gray]{0.75}FO}%
\colorbox{green}{\color[gray]{0.75}FO}%
\colorbox{green}{\color[gray]{0.75}FO}%
\colorbox{green}{\color[gray]{0.75}FO}%
\colorbox{green}{\color[gray]{0.75}FO}%
\colorbox{green}{\color[gray]{0.75}FO}%
\colorbox{green}{\color[gray]{0.75}FO}%
\\
\colorbox{green}{\color[gray]{0.75}FO}%
\colorbox{green}{\color[gray]{0.75}FO}%
\colorbox{green}{\color[gray]{0.75}FO}%
\colorbox{green}{\color[gray]{0.75}FO}%
\colorbox{green}{\color[gray]{0.75}FO}%
\colorbox{green}{\color[gray]{0.75}FO}%
\colorbox{green}{\color[gray]{0.75}FO}%
\colorbox{green}{\color[gray]{0.75}FO}%
\colorbox{green}{\color[gray]{0.75}FO}%
\colorbox{green}{\color[gray]{0.75}FO}%
\colorbox{green}{\color[gray]{0.75}FO}%
\colorbox{green}{\color[gray]{0.75}FO}%
\colorbox{green}{\color[gray]{0.75}FO}%
\colorbox{green}{\color[gray]{0.75}FO}%
\colorbox{green}{\color[gray]{0.75}FO}%
\colorbox{green}{\color[gray]{0.75}FO}%
\colorbox{green}{\color[gray]{0.75}FO}%
\colorbox{green}{\color[gray]{0.75}FO}%
\colorbox{green}{\color[gray]{0.75}FO}%
\colorbox{green}{\color[gray]{0.75}FO}%
\colorbox{green}{\color[gray]{0.75}FO}%
\colorbox{green}{\color[gray]{0.75}FO}%
\colorbox{green}{\color[gray]{0.75}FO}%
\colorbox{green}{\color[gray]{0.75}FO}%
\colorbox{green}{\color[gray]{0.75}FO}%
\colorbox{green}{\color[gray]{0.75}FO}%
\colorbox{green}{\color[gray]{0.75}FO}%
\colorbox{green}{\color[gray]{0.75}FO}%
\colorbox{green}{\color[gray]{0.75}FO}%
\colorbox{green}{\color[gray]{0.75}FO}%
\colorbox{green}{\color[gray]{0.75}FO}%
\colorbox{green}{\color[gray]{0.75}FO}%
\colorbox{green}{\color[gray]{0.75}FO}%
\colorbox{green}{\color[gray]{0.75}FO}%
\colorbox{green}{\color[gray]{0.75}FO}%
\colorbox{green}{\color[gray]{0.75}FO}%
\colorbox{green}{\color[gray]{0.75}FO}%
\colorbox{green}{\color[gray]{0.75}FO}%
\colorbox{green}{\color[gray]{0.75}FO}%
\colorbox{green}{\color[gray]{0.75}FO}%
\colorbox{green}{\color[gray]{0.75}FO}%
\colorbox{green}{\color[gray]{0.75}FO}%
\colorbox{green}{\color[gray]{0.75}FO}%
\colorbox{green}{\color[gray]{0.75}FO}%
\colorbox{green}{\color[gray]{0.75}FO}%
\colorbox{green}{\color[gray]{0.75}FO}%
\colorbox{green}{\color[gray]{0.75}FO}%
\colorbox{green}{\color[gray]{0.75}FO}%
\colorbox{green}{\color[gray]{0.75}FO}%
\colorbox{green}{\color[gray]{0.75}FO}%
\colorbox{green}{\color[gray]{0.75}FO}%
\colorbox{green}{\color[gray]{0.75}FO}%
\colorbox{green}{\color[gray]{0.75}FO}%
\colorbox{green}{\color[gray]{0.75}FO}%
\colorbox{green}{\color[gray]{0.75}FO}%
\colorbox{green}{\color[gray]{0.75}FO}%
\colorbox{green}{\color[gray]{0.75}FO}%
\colorbox{green}{\color[gray]{0.75}FO}%
\colorbox{green}{\color[gray]{0.75}FO}%
\colorbox{green}{\color[gray]{0.75}FO}%
\colorbox{green}{\color[gray]{0.75}FO}%
\colorbox{green}{\color[gray]{0.75}FO}%
\colorbox{green}{\color[gray]{0.75}FO}%
\colorbox{green}{\color[gray]{0.75}FO}%
\colorbox{green}{\color[gray]{0.75}FO}%
\colorbox{green}{\color[gray]{0.75}FO}%
\colorbox{green}{\color[gray]{0.75}FO}%
\colorbox{green}{\color[gray]{0.75}FO}%
\colorbox{green}{\color[gray]{0.75}FO}%
\colorbox{green}{\color[gray]{0.75}FO}%
\colorbox{green}{\color[gray]{0.75}FO}%
\colorbox{green}{\color[gray]{0.75}FO}%
\colorbox{green}{\color[gray]{0.75}FO}%
\colorbox{green}{\color[gray]{0.75}FO}%
\colorbox{green}{\color[gray]{0.75}FO}%
\colorbox{green}{\color[gray]{0.75}FO}%
\colorbox{green}{\color[gray]{0.75}FO}%
\colorbox{green}{\color[gray]{0.75}FO}%
\colorbox{green}{\color[gray]{0.75}FO}%
\colorbox{green}{\color[gray]{0.75}FO}%
\colorbox{green}{\color[gray]{0.75}FO}%
\colorbox{green}{\color[gray]{0.75}FO}%
\colorbox{green}{\color[gray]{0.75}FO}%
\colorbox{green}{\color[gray]{0.75}FO}%
\colorbox{green}{\color[gray]{0.75}FO}%
\colorbox{green}{\color[gray]{0.75}FO}%
\colorbox{green}{\color[gray]{0.75}FO}%
\colorbox{green}{\color[gray]{0.75}FO}%
\colorbox{green}{\color[gray]{0.75}FO}%
\colorbox{green}{\color[gray]{0.75}FO}%
\colorbox{green}{\color[gray]{0.75}FO}%
\colorbox{green}{\color[gray]{0.75}FO}%
\colorbox{green}{\color[gray]{0.75}FO}%
\colorbox{green}{\color[gray]{0.75}FO}%
\colorbox{green}{\color[gray]{0.75}FO}%
\colorbox{green}{\color[gray]{0.75}FO}%
\colorbox{green}{\color[gray]{0.75}FO}%
\colorbox{green}{\color[gray]{0.75}FO}%
\colorbox{green}{\color[gray]{0.75}FO}%
\colorbox{green}{\color[gray]{0.75}FO}%
\\
\colorbox{green}{\color[gray]{0.75}FO}%
\colorbox{green}{\color[gray]{0.75}FO}%
\colorbox{green}{\color[gray]{0.75}FO}%
\colorbox{green}{\color[gray]{0.75}FO}%
\colorbox{green}{\color[gray]{0.75}FO}%
\colorbox{green}{\color[gray]{0.75}FO}%
\colorbox{green}{\color[gray]{0.75}FO}%
\colorbox{green}{\color[gray]{0.75}FO}%
\colorbox{green}{\color[gray]{0.75}FO}%
\colorbox{green}{\color[gray]{0.75}FO}%
\colorbox{green}{\color[gray]{0.75}FO}%
\colorbox{green}{\color[gray]{0.75}FO}%
\colorbox{green}{\color[gray]{0.75}FO}%
\colorbox{green}{\color[gray]{0.75}FO}%
\colorbox{green}{\color[gray]{0.75}FO}%
\colorbox{green}{\color[gray]{0.75}FO}%
\colorbox{green}{\color[gray]{0.75}FO}%
\colorbox{green}{\color[gray]{0.75}FO}%
\colorbox{green}{\color[gray]{0.75}FO}%
\colorbox{green}{\color[gray]{0.75}FO}%
\colorbox{green}{\color[gray]{0.75}FO}%
\colorbox{green}{\color[gray]{0.75}FO}%
\colorbox{green}{\color[gray]{0.75}FO}%
\colorbox{green}{\color[gray]{0.75}FO}%
\colorbox{green}{\color[gray]{0.75}FO}%
\colorbox{green}{\color[gray]{0.75}FO}%
\colorbox{green}{\color[gray]{0.75}FO}%
\colorbox{green}{\color[gray]{0.75}FO}%
\colorbox{green}{\color[gray]{0.75}FO}%
\colorbox{green}{\color[gray]{0.75}FO}%
\colorbox{green}{\color[gray]{0.75}FO}%
\colorbox{green}{\color[gray]{0.75}FO}%
\colorbox{green}{\color[gray]{0.75}FO}%
\colorbox{green}{\color[gray]{0.75}FO}%
\colorbox{green}{\color[gray]{0.75}FO}%
\colorbox{green}{\color[gray]{0.75}FO}%
\colorbox{green}{\color[gray]{0.75}FO}%
\colorbox{green}{\color[gray]{0.75}FO}%
\colorbox{green}{\color[gray]{0.75}FO}%
\colorbox{green}{\color[gray]{0.75}FO}%
\colorbox{green}{\color[gray]{0.75}FO}%
\colorbox{green}{\color[gray]{0.75}FO}%
\colorbox{green}{\color[gray]{0.75}FO}%
\colorbox{green}{\color[gray]{0.75}FO}%
\colorbox{green}{\color[gray]{0.75}FO}%
\colorbox{green}{\color[gray]{0.75}FO}%
\colorbox{green}{\color[gray]{0.75}FO}%
\colorbox{green}{\color[gray]{0.75}FO}%
\colorbox{green}{\color[gray]{0.75}FO}%
\colorbox{green}{\color[gray]{0.75}FO}%
\colorbox{green}{\color[gray]{0.75}FO}%
\colorbox{green}{\color[gray]{0.75}FO}%
\colorbox{green}{\color[gray]{0.75}FO}%
\colorbox{green}{\color[gray]{0.75}FO}%
\colorbox{green}{\color[gray]{0.75}FO}%
\colorbox{green}{\color[gray]{0.75}FO}%
\colorbox{green}{\color[gray]{0.75}FO}%
\colorbox{green}{\color[gray]{0.75}FO}%
\colorbox{green}{\color[gray]{0.75}FO}%
\colorbox{green}{\color[gray]{0.75}FO}%
\colorbox{green}{\color[gray]{0.75}FO}%
\colorbox{green}{\color[gray]{0.75}FO}%
\colorbox{green}{\color[gray]{0.75}FO}%
\colorbox{green}{\color[gray]{0.75}FO}%
\colorbox{green}{\color[gray]{0.75}FO}%
\colorbox{green}{\color[gray]{0.75}FO}%
\colorbox{green}{\color[gray]{0.75}FO}%
\colorbox{green}{\color[gray]{0.75}FO}%
\colorbox{green}{\color[gray]{0.75}FO}%
\colorbox{green}{\color[gray]{0.75}FO}%
\colorbox{green}{\color[gray]{0.75}FO}%
\colorbox{green}{\color[gray]{0.75}FO}%
\colorbox{green}{\color[gray]{0.75}FO}%
\colorbox{green}{\color[gray]{0.75}FO}%
\colorbox{green}{\color[gray]{0.75}FO}%
\colorbox{green}{\color[gray]{0.75}FO}%
\colorbox{green}{\color[gray]{0.75}FO}%
\colorbox{green}{\color[gray]{0.75}FO}%
\colorbox{green}{\color[gray]{0.75}FO}%
\colorbox{green}{\color[gray]{0.75}FO}%
\colorbox{green}{\color[gray]{0.75}FO}%
\colorbox{green}{\color[gray]{0.75}FO}%
\colorbox{green}{\color[gray]{0.75}FO}%
\colorbox{green}{\color[gray]{0.75}FO}%
\colorbox{green}{\color[gray]{0.75}FO}%
\colorbox{green}{\color[gray]{0.75}FO}%
\colorbox{green}{\color[gray]{0.75}FO}%
\colorbox{green}{\color[gray]{0.75}FO}%
\colorbox{green}{\color[gray]{0.75}FO}%
\colorbox{green}{\color[gray]{0.75}FO}%
\colorbox{green}{\color[gray]{0.75}FO}%
\colorbox{green}{\color[gray]{0.75}FO}%
\colorbox{green}{\color[gray]{0.75}FO}%
\colorbox{green}{\color[gray]{0.75}FO}%
\colorbox{green}{\color[gray]{0.75}FO}%
\colorbox{green}{\color[gray]{0.75}FO}%
\colorbox{green}{\color[gray]{0.75}FO}%
\colorbox{green}{\color[gray]{0.75}FO}%
\colorbox{green}{\color[gray]{0.75}FO}%
\colorbox{green}{\color[gray]{0.75}FO}%
\\
\colorbox{green}{\color[gray]{0.75}FO}%
\colorbox{green}{\color[gray]{0.75}FO}%
\colorbox{green}{\color[gray]{0.75}FO}%
\colorbox{green}{\color[gray]{0.75}FO}%
\colorbox{green}{\color[gray]{0.75}FO}%
\colorbox{green}{\color[gray]{0.75}FO}%
\colorbox{green}{\color[gray]{0.75}FO}%
\colorbox{green}{\color[gray]{0.75}FO}%
\colorbox{green}{\color[gray]{0.75}FO}%
\colorbox{green}{\color[gray]{0.75}FO}%
\colorbox{green}{\color[gray]{0.75}FO}%
\colorbox{green}{\color[gray]{0.75}FO}%
\colorbox{green}{\color[gray]{0.75}FO}%
\colorbox{green}{\color[gray]{0.75}FO}%
\colorbox{green}{\color[gray]{0.75}FO}%
\colorbox{green}{\color[gray]{0.75}FO}%
\colorbox{green}{\color[gray]{0.75}FO}%
\colorbox{green}{\color[gray]{0.75}FO}%
\colorbox{green}{\color[gray]{0.75}FO}%
\colorbox{green}{\color[gray]{0.75}FO}%
\colorbox{green}{\color[gray]{0.75}FO}%
\colorbox{green}{\color[gray]{0.75}FO}%
\colorbox{green}{\color[gray]{0.75}FO}%
\colorbox{green}{\color[gray]{0.75}FO}%
\colorbox{green}{\color[gray]{0.75}FO}%
\colorbox{green}{\color[gray]{0.75}FO}%
\colorbox{green}{\color[gray]{0.75}FO}%
\colorbox{green}{\color[gray]{0.75}FO}%
\colorbox{green}{\color[gray]{0.75}FO}%
\colorbox{green}{\color[gray]{0.75}FO}%
\colorbox{green}{\color[gray]{0.75}FO}%
\colorbox{green}{\color[gray]{0.75}FO}%
\colorbox{green}{\color[gray]{0.75}FO}%
\colorbox{green}{\color[gray]{0.75}FO}%
\colorbox{green}{\color[gray]{0.75}FO}%
\colorbox{green}{\color[gray]{0.75}FO}%
\colorbox{green}{\color[gray]{0.75}FO}%
\colorbox{green}{\color[gray]{0.75}FO}%
\colorbox{green}{\color[gray]{0.75}FO}%
\colorbox{green}{\color[gray]{0.75}FO}%
\colorbox{green}{\color[gray]{0.75}FO}%
\colorbox{green}{\color[gray]{0.75}FO}%
\colorbox{green}{\color[gray]{0.75}FO}%
\colorbox{green}{\color[gray]{0.75}FO}%
\colorbox{green}{\color[gray]{0.75}FO}%
\colorbox{green}{\color[gray]{0.75}FO}%
\colorbox{green}{\color[gray]{0.75}FO}%
\colorbox{green}{\color[gray]{0.75}FO}%
\colorbox{green}{\color[gray]{0.75}FO}%
\colorbox{green}{\color[gray]{0.75}FO}%
\colorbox{green}{\color[gray]{0.75}FO}%
\colorbox{green}{\color[gray]{0.75}FO}%
\colorbox{green}{\color[gray]{0.75}FO}%
\colorbox{green}{\color[gray]{0.75}FO}%
\colorbox{green}{\color[gray]{0.75}FO}%
\colorbox{green}{\color[gray]{0.75}FO}%
\colorbox{green}{\color[gray]{0.75}FO}%
\colorbox{green}{\color[gray]{0.75}FO}%
\colorbox{green}{\color[gray]{0.75}FO}%
\colorbox{green}{\color[gray]{0.75}FO}%
\colorbox{green}{\color[gray]{0.75}FO}%
\colorbox{green}{\color[gray]{0.75}FO}%
\colorbox{green}{\color[gray]{0.75}FO}%
\colorbox{green}{\color[gray]{0.75}FO}%
\colorbox{green}{\color[gray]{0.75}FO}%
\colorbox{green}{\color[gray]{0.75}FO}%
\colorbox{green}{\color[gray]{0.75}FO}%
\colorbox{green}{\color[gray]{0.75}FO}%
\colorbox{green}{\color[gray]{0.75}FO}%
\colorbox{green}{\color[gray]{0.75}FO}%
\colorbox{green}{\color[gray]{0.75}FO}%
\colorbox{green}{\color[gray]{0.75}FO}%
\colorbox{green}{\color[gray]{0.75}FO}%
\colorbox{green}{\color[gray]{0.75}FO}%
\colorbox{green}{\color[gray]{0.75}FO}%
\colorbox{green}{\color[gray]{0.75}FO}%
\colorbox{green}{\color[gray]{0.75}FO}%
\colorbox{green}{\color[gray]{0.75}FO}%
\colorbox{green}{\color[gray]{0.75}FO}%
\colorbox{green}{\color[gray]{0.75}FO}%
\colorbox{green}{\color[gray]{0.75}FO}%
\colorbox{green}{\color[gray]{0.75}FO}%
\colorbox{green}{\color[gray]{0.75}FO}%
\colorbox{green}{\color[gray]{0.75}FO}%
\colorbox{green}{\color[gray]{0.75}FO}%
\colorbox{green}{\color[gray]{0.75}FO}%
\colorbox{green}{\color[gray]{0.75}FO}%
\colorbox{green}{\color[gray]{0.75}FO}%
\colorbox{green}{\color[gray]{0.75}FO}%
\colorbox{green}{\color[gray]{0.75}FO}%
\colorbox{green}{\color[gray]{0.75}FO}%
\colorbox{green}{\color[gray]{0.75}FO}%
\colorbox{green}{\color[gray]{0.75}FO}%
\colorbox{green}{\color[gray]{0.75}FO}%
\colorbox{green}{\color[gray]{0.75}FO}%
\colorbox{green}{\color[gray]{0.75}FO}%
\colorbox{green}{\color[gray]{0.75}FO}%
\colorbox{green}{\color[gray]{0.75}FO}%
\colorbox{green}{\color[gray]{0.75}FO}%
\colorbox{green}{\color[gray]{0.75}FO}%
\\
\colorbox{green}{\color[gray]{0.75}FO}%
\colorbox{green}{\color[gray]{0.75}FO}%
\colorbox{green}{\color[gray]{0.75}FO}%
\colorbox{green}{\color[gray]{0.75}FO}%
\colorbox{green}{\color[gray]{0.75}FO}%
\colorbox{green}{\color[gray]{0.75}FO}%
\colorbox{green}{\color[gray]{0.75}FO}%
\colorbox{green}{\color[gray]{0.75}FO}%
\colorbox{green}{\color[gray]{0.75}FO}%
\colorbox{green}{\color[gray]{0.75}FO}%
\colorbox{green}{\color[gray]{0.75}FO}%
\colorbox{green}{\color[gray]{0.75}FO}%
\colorbox{green}{\color[gray]{0.75}FO}%
\colorbox{green}{\color[gray]{0.75}FO}%
\colorbox{green}{\color[gray]{0.75}FO}%
\colorbox{green}{\color[gray]{0.75}FO}%
\colorbox{green}{\color[gray]{0.75}FO}%
\colorbox{green}{\color[gray]{0.75}FO}%
\colorbox{green}{\color[gray]{0.75}FO}%
\colorbox{green}{\color[gray]{0.75}FO}%
\colorbox{green}{\color[gray]{0.75}FO}%
\colorbox{green}{\color[gray]{0.75}FO}%
\colorbox{green}{\color[gray]{0.75}FO}%
\colorbox{green}{\color[gray]{0.75}FO}%
\colorbox{green}{\color[gray]{0.75}FO}%
\colorbox{green}{\color[gray]{0.75}FO}%
\colorbox{green}{\color[gray]{0.75}FO}%
\colorbox{green}{\color[gray]{0.75}FO}%
\colorbox{green}{\color[gray]{0.75}FO}%
\colorbox{green}{\color[gray]{0.75}FO}%
\colorbox{green}{\color[gray]{0.75}FO}%
\colorbox{green}{\color[gray]{0.75}FO}%
\colorbox{green}{\color[gray]{0.75}FO}%
\colorbox{green}{\color[gray]{0.75}FO}%
\colorbox{green}{\color[gray]{0.75}FO}%
\colorbox{green}{\color[gray]{0.75}FO}%
\colorbox{green}{\color[gray]{0.75}FO}%
\colorbox{green}{\color[gray]{0.75}FO}%
\colorbox{green}{\color[gray]{0.75}FO}%
\colorbox{green}{\color[gray]{0.75}FO}%
\colorbox{green}{\color[gray]{0.75}FO}%
\colorbox{green}{\color[gray]{0.75}FO}%
\colorbox{green}{\color[gray]{0.75}FO}%
\colorbox{green}{\color[gray]{0.75}FO}%
\colorbox{green}{\color[gray]{0.75}FO}%
\colorbox{green}{\color[gray]{0.75}FO}%
\colorbox{green}{\color[gray]{0.75}FO}%
\colorbox{green}{\color[gray]{0.75}FO}%
\colorbox{green}{\color[gray]{0.75}FO}%
\colorbox{green}{\color[gray]{0.75}FO}%
\colorbox{green}{\color[gray]{0.75}FO}%
\colorbox{green}{\color[gray]{0.75}FO}%
\colorbox{green}{\color[gray]{0.75}FO}%
\colorbox{green}{\color[gray]{0.75}FO}%
\colorbox{green}{\color[gray]{0.75}FO}%
\colorbox{green}{\color[gray]{0.75}FO}%
\colorbox{green}{\color[gray]{0.75}FO}%
\colorbox{green}{\color[gray]{0.75}FO}%
\colorbox{green}{\color[gray]{0.75}FO}%
\colorbox{green}{\color[gray]{0.75}FO}%
\colorbox{green}{\color[gray]{0.75}FO}%
\colorbox{green}{\color[gray]{0.75}FO}%
\colorbox{green}{\color[gray]{0.75}FO}%
\colorbox{green}{\color[gray]{0.75}FO}%
\colorbox{green}{\color[gray]{0.75}FO}%
\colorbox{green}{\color[gray]{0.75}FO}%
\colorbox{green}{\color[gray]{0.75}FO}%
\colorbox{green}{\color[gray]{0.75}FO}%
\colorbox{green}{\color[gray]{0.75}FO}%
\colorbox{green}{\color[gray]{0.75}FO}%
\colorbox{green}{\color[gray]{0.75}FO}%
\colorbox{green}{\color[gray]{0.75}FO}%
\colorbox{green}{\color[gray]{0.75}FO}%
\colorbox{green}{\color[gray]{0.75}FO}%
\colorbox{green}{\color[gray]{0.75}FO}%
\colorbox{green}{\color[gray]{0.75}FO}%
\colorbox{green}{\color[gray]{0.75}FO}%
\colorbox{green}{\color[gray]{0.75}FO}%
\colorbox{green}{\color[gray]{0.75}FO}%
\colorbox{green}{\color[gray]{0.75}FO}%
\colorbox{green}{\color[gray]{0.75}FO}%
\colorbox{green}{\color[gray]{0.75}FO}%
\colorbox{green}{\color[gray]{0.75}FO}%
\colorbox{green}{\color[gray]{0.75}FO}%
\colorbox{green}{\color[gray]{0.75}FO}%
\colorbox{green}{\color[gray]{0.75}FO}%
\colorbox{green}{\color[gray]{0.75}FO}%
\colorbox{green}{\color[gray]{0.75}FO}%
\colorbox{green}{\color[gray]{0.75}FO}%
\colorbox{green}{\color[gray]{0.75}FO}%
\colorbox{green}{\color[gray]{0.75}FO}%
\colorbox{green}{\color[gray]{0.75}FO}%
\colorbox{green}{\color[gray]{0.75}FO}%
\colorbox{green}{\color[gray]{0.75}FO}%
\colorbox{green}{\color[gray]{0.75}FO}%
\colorbox{green}{\color[gray]{0.75}FO}%
\colorbox{green}{\color[gray]{0.75}FO}%
\colorbox{green}{\color[gray]{0.75}FO}%
\colorbox{green}{\color[gray]{0.75}FO}%
\colorbox{green}{\color[gray]{0.75}FO}%
\\
\colorbox{green}{\color[gray]{0.75}FO}%
\colorbox{green}{\color[gray]{0.75}FO}%
\colorbox{green}{\color[gray]{0.75}FO}%
\colorbox{green}{\color[gray]{0.75}FO}%
\colorbox{green}{\color[gray]{0.75}FO}%
\colorbox{green}{\color[gray]{0.75}FO}%
\colorbox{green}{\color[gray]{0.75}FO}%
\colorbox{green}{\color[gray]{0.75}FO}%
\colorbox{green}{\color[gray]{0.75}FO}%
\colorbox{green}{\color[gray]{0.75}FO}%
\colorbox{green}{\color[gray]{0.75}FO}%
\colorbox{green}{\color[gray]{0.75}FO}%
\colorbox{green}{\color[gray]{0.75}FO}%
\colorbox{green}{\color[gray]{0.75}FO}%
\colorbox{green}{\color[gray]{0.75}FO}%
\colorbox{green}{\color[gray]{0.75}FO}%
\colorbox{green}{\color[gray]{0.75}FO}%
\colorbox{green}{\color[gray]{0.75}FO}%
\colorbox{green}{\color[gray]{0.75}FO}%
\colorbox{green}{\color[gray]{0.75}FO}%
\colorbox{green}{\color[gray]{0.75}FO}%
\colorbox{green}{\color[gray]{0.75}FO}%
\colorbox{green}{\color[gray]{0.75}FO}%
\colorbox{green}{\color[gray]{0.75}FO}%
\colorbox{green}{\color[gray]{0.75}FO}%
\colorbox{green}{\color[gray]{0.75}FO}%
\colorbox{green}{\color[gray]{0.75}FO}%
\colorbox{green}{\color[gray]{0.75}FO}%
\colorbox{green}{\color[gray]{0.75}FO}%
\colorbox{green}{\color[gray]{0.75}FO}%
\colorbox{green}{\color[gray]{0.75}FO}%
\colorbox{green}{\color[gray]{0.75}FO}%
\colorbox{green}{\color[gray]{0.75}FO}%
\colorbox{green}{\color[gray]{0.75}FO}%
\colorbox{green}{\color[gray]{0.75}FO}%
\colorbox{green}{\color[gray]{0.75}FO}%
\colorbox{green}{\color[gray]{0.75}FO}%
\colorbox{green}{\color[gray]{0.75}FO}%
\colorbox{green}{\color[gray]{0.75}FO}%
\colorbox{green}{\color[gray]{0.75}FO}%
\colorbox{green}{\color[gray]{0.75}FO}%
\colorbox{green}{\color[gray]{0.75}FO}%
\colorbox{green}{\color[gray]{0.75}FO}%
\colorbox{green}{\color[gray]{0.75}FO}%
\colorbox{green}{\color[gray]{0.75}FO}%
\colorbox{green}{\color[gray]{0.75}FO}%
\colorbox{green}{\color[gray]{0.75}FO}%
\colorbox{green}{\color[gray]{0.75}FO}%
\colorbox{green}{\color[gray]{0.75}FO}%
\colorbox{green}{\color[gray]{0.75}FO}%
\colorbox{green}{\color[gray]{0.75}FO}%
\colorbox{green}{\color[gray]{0.75}FO}%
\colorbox{green}{\color[gray]{0.75}FO}%
\colorbox{green}{\color[gray]{0.75}FO}%
\colorbox{green}{\color[gray]{0.75}FO}%
\colorbox{green}{\color[gray]{0.75}FO}%
\colorbox{green}{\color[gray]{0.75}FO}%
\colorbox{green}{\color[gray]{0.75}FO}%
\colorbox{green}{\color[gray]{0.75}FO}%
\colorbox{green}{\color[gray]{0.75}FO}%
\colorbox{green}{\color[gray]{0.75}FO}%
\colorbox{green}{\color[gray]{0.75}FO}%
\colorbox{green}{\color[gray]{0.75}FO}%
\colorbox{green}{\color[gray]{0.75}FO}%
\colorbox{green}{\color[gray]{0.75}FO}%
\colorbox{green}{\color[gray]{0.75}FO}%
\colorbox{green}{\color[gray]{0.75}FO}%
\colorbox{green}{\color[gray]{0.75}FO}%
\colorbox{green}{\color[gray]{0.75}FO}%
\colorbox{green}{\color[gray]{0.75}FO}%
\colorbox{green}{\color[gray]{0.75}FO}%
\colorbox{green}{\color[gray]{0.75}FO}%
\colorbox{green}{\color[gray]{0.75}FO}%
\colorbox{green}{\color[gray]{0.75}FO}%
\colorbox{green}{\color[gray]{0.75}FO}%
\colorbox{green}{\color[gray]{0.75}FO}%
\colorbox{green}{\color[gray]{0.75}FO}%
\colorbox{green}{\color[gray]{0.75}FO}%
\colorbox{green}{\color[gray]{0.75}FO}%
\colorbox{green}{\color[gray]{0.75}FO}%
\colorbox{green}{\color[gray]{0.75}FO}%
\colorbox{green}{\color[gray]{0.75}FO}%
\colorbox{green}{\color[gray]{0.75}FO}%
\colorbox{green}{\color[gray]{0.75}FO}%
\colorbox{green}{\color[gray]{0.75}FO}%
\colorbox{green}{\color[gray]{0.75}FO}%
\colorbox{green}{\color[gray]{0.75}FO}%
\colorbox{green}{\color[gray]{0.75}FO}%
\colorbox{green}{\color[gray]{0.75}FO}%
\colorbox{green}{\color[gray]{0.75}FO}%
\colorbox{green}{\color[gray]{0.75}FO}%
\colorbox{green}{\color[gray]{0.75}FO}%
\colorbox{green}{\color[gray]{0.75}FO}%
\colorbox{green}{\color[gray]{0.75}FO}%
\colorbox{green}{\color[gray]{0.75}FO}%
\colorbox{green}{\color[gray]{0.75}FO}%
\colorbox{green}{\color[gray]{0.75}FO}%
\colorbox{green}{\color[gray]{0.75}FO}%
\colorbox{green}{\color[gray]{0.75}FO}%
\colorbox{green}{\color[gray]{0.75}FO}%
\\
\colorbox{green}{\color[gray]{0.75}FO}%
\colorbox{green}{\color[gray]{0.75}FO}%
\colorbox{green}{\color[gray]{0.75}FO}%
\colorbox{green}{\color[gray]{0.75}FO}%
\colorbox{green}{\color[gray]{0.75}FO}%
\colorbox{green}{\color[gray]{0.75}FO}%
\colorbox{green}{\color[gray]{0.75}FO}%
\colorbox{green}{\color[gray]{0.75}FO}%
\colorbox{green}{\color[gray]{0.75}FO}%
\colorbox{green}{\color[gray]{0.75}FO}%
\colorbox{green}{\color[gray]{0.75}FO}%
\colorbox{green}{\color[gray]{0.75}FO}%
\colorbox{green}{\color[gray]{0.75}FO}%
\colorbox{green}{\color[gray]{0.75}FO}%
\colorbox{green}{\color[gray]{0.75}FO}%
\colorbox{green}{\color[gray]{0.75}FO}%
\colorbox{green}{\color[gray]{0.75}FO}%
\colorbox{green}{\color[gray]{0.75}FO}%
\colorbox{green}{\color[gray]{0.75}FO}%
\colorbox{green}{\color[gray]{0.75}FO}%
\colorbox{green}{\color[gray]{0.75}FO}%
\colorbox{green}{\color[gray]{0.75}FO}%
\colorbox{green}{\color[gray]{0.75}FO}%
\colorbox{green}{\color[gray]{0.75}FO}%
\colorbox{green}{\color[gray]{0.75}FO}%
\colorbox{green}{\color[gray]{0.75}FO}%
\colorbox{green}{\color[gray]{0.75}FO}%
\colorbox{green}{\color[gray]{0.75}FO}%
\colorbox{green}{\color[gray]{0.75}FO}%
\colorbox{green}{\color[gray]{0.75}FO}%
\colorbox{green}{\color[gray]{0.75}FO}%
\colorbox{green}{\color[gray]{0.75}FO}%
\colorbox{green}{\color[gray]{0.75}FO}%
\colorbox{green}{\color[gray]{0.75}FO}%
\colorbox{green}{\color[gray]{0.75}FO}%
\colorbox{green}{\color[gray]{0.75}FO}%
\colorbox{green}{\color[gray]{0.75}FO}%
\colorbox{green}{\color[gray]{0.75}FO}%
\colorbox{green}{\color[gray]{0.75}FO}%
\colorbox{green}{\color[gray]{0.75}FO}%
\colorbox{green}{\color[gray]{0.75}FO}%
\colorbox{green}{\color[gray]{0.75}FO}%
\colorbox{green}{\color[gray]{0.75}FO}%
\colorbox{green}{\color[gray]{0.75}FO}%
\colorbox{green}{\color[gray]{0.75}FO}%
\colorbox{green}{\color[gray]{0.75}FO}%
\colorbox{green}{\color[gray]{0.75}FO}%
\colorbox{green}{\color[gray]{0.75}FO}%
\colorbox{green}{\color[gray]{0.75}FO}%
\colorbox{green}{\color[gray]{0.75}FO}%
\colorbox{green}{\color[gray]{0.75}FO}%
\colorbox{green}{\color[gray]{0.75}FO}%
\colorbox{green}{\color[gray]{0.75}FO}%
\colorbox{green}{\color[gray]{0.75}FO}%
\colorbox{green}{\color[gray]{0.75}FO}%
\colorbox{green}{\color[gray]{0.75}FO}%
\colorbox{green}{\color[gray]{0.75}FO}%
\colorbox{green}{\color[gray]{0.75}FO}%
\colorbox{green}{\color[gray]{0.75}FO}%
\colorbox{green}{\color[gray]{0.75}FO}%
\colorbox{green}{\color[gray]{0.75}FO}%
\colorbox{green}{\color[gray]{0.75}FO}%
\colorbox{green}{\color[gray]{0.75}FO}%
\colorbox{green}{\color[gray]{0.75}FO}%
\colorbox{green}{\color[gray]{0.75}FO}%
\colorbox{green}{\color[gray]{0.75}FO}%
\colorbox{green}{\color[gray]{0.75}FO}%
\colorbox{green}{\color[gray]{0.75}FO}%
\colorbox{green}{\color[gray]{0.75}FO}%
\colorbox{green}{\color[gray]{0.75}FO}%
\colorbox{green}{\color[gray]{0.75}FO}%
\colorbox{green}{\color[gray]{0.75}FO}%
\colorbox{green}{\color[gray]{0.75}FO}%
\colorbox{green}{\color[gray]{0.75}FO}%
\colorbox{green}{\color[gray]{0.75}FO}%
\colorbox{green}{\color[gray]{0.75}FO}%
\colorbox{green}{\color[gray]{0.75}FO}%
\colorbox{green}{\color[gray]{0.75}FO}%
\colorbox{green}{\color[gray]{0.75}FO}%
\colorbox{green}{\color[gray]{0.75}FO}%
\colorbox{green}{\color[gray]{0.75}FO}%
\colorbox{green}{\color[gray]{0.75}FO}%
\colorbox{green}{\color[gray]{0.75}FO}%
\colorbox{green}{\color[gray]{0.75}FO}%
\colorbox{green}{\color[gray]{0.75}FO}%
\colorbox{green}{\color[gray]{0.75}FO}%
\colorbox{green}{\color[gray]{0.75}FO}%
\colorbox{green}{\color[gray]{0.75}FO}%
\colorbox{green}{\color[gray]{0.75}FO}%
\colorbox{green}{\color[gray]{0.75}FO}%
\colorbox{green}{\color[gray]{0.75}FO}%
\colorbox{green}{\color[gray]{0.75}FO}%
\colorbox{green}{\color[gray]{0.75}FO}%
\colorbox{green}{\color[gray]{0.75}FO}%
\colorbox{green}{\color[gray]{0.75}FO}%
\colorbox{green}{\color[gray]{0.75}FO}%
\colorbox{green}{\color[gray]{0.75}FO}%
\colorbox{green}{\color[gray]{0.75}FO}%
\colorbox{green}{\color[gray]{0.75}FO}%
\colorbox{green}{\color[gray]{0.75}FO}%
\\
\colorbox{green}{\color[gray]{0.75}FO}%
\colorbox{green}{\color[gray]{0.75}FO}%
\colorbox{green}{\color[gray]{0.75}FO}%
\colorbox{green}{\color[gray]{0.75}FO}%
\colorbox{green}{\color[gray]{0.75}FO}%
\colorbox{green}{\color[gray]{0.75}FO}%
\colorbox{green}{\color[gray]{0.75}FO}%
\colorbox{green}{\color[gray]{0.75}FO}%
\colorbox{green}{\color[gray]{0.75}FO}%
\colorbox{green}{\color[gray]{0.75}FO}%
\colorbox{green}{\color[gray]{0.75}FO}%
\colorbox{green}{\color[gray]{0.75}FO}%
\colorbox{green}{\color[gray]{0.75}FO}%
\colorbox{green}{\color[gray]{0.75}FO}%
\colorbox{green}{\color[gray]{0.75}FO}%
\colorbox{green}{\color[gray]{0.75}FO}%
\colorbox{green}{\color[gray]{0.75}FO}%
\colorbox{green}{\color[gray]{0.75}FO}%
\colorbox{green}{\color[gray]{0.75}FO}%
\colorbox{green}{\color[gray]{0.75}FO}%
\colorbox{green}{\color[gray]{0.75}FO}%
\colorbox{green}{\color[gray]{0.75}FO}%
\colorbox{green}{\color[gray]{0.75}FO}%
\colorbox{green}{\color[gray]{0.75}FO}%
\colorbox{green}{\color[gray]{0.75}FO}%
\colorbox{green}{\color[gray]{0.75}FO}%
\colorbox{green}{\color[gray]{0.75}FO}%
\colorbox{green}{\color[gray]{0.75}FO}%
\colorbox{green}{\color[gray]{0.75}FO}%
\colorbox{green}{\color[gray]{0.75}FO}%
\colorbox{green}{\color[gray]{0.75}FO}%
\colorbox{green}{\color[gray]{0.75}FO}%
\colorbox{green}{\color[gray]{0.75}FO}%
\colorbox{green}{\color[gray]{0.75}FO}%
\colorbox{green}{\color[gray]{0.75}FO}%
\colorbox{green}{\color[gray]{0.75}FO}%
\colorbox{green}{\color[gray]{0.75}FO}%
\colorbox{green}{\color[gray]{0.75}FO}%
\colorbox{green}{\color[gray]{0.75}FO}%
\colorbox{green}{\color[gray]{0.75}FO}%
\colorbox{green}{\color[gray]{0.75}FO}%
\colorbox{green}{\color[gray]{0.75}FO}%
\colorbox{green}{\color[gray]{0.75}FO}%
\colorbox{green}{\color[gray]{0.75}FO}%
\colorbox{green}{\color[gray]{0.75}FO}%
\colorbox{green}{\color[gray]{0.75}FO}%
\colorbox{green}{\color[gray]{0.75}FO}%
\colorbox{green}{\color[gray]{0.75}FO}%
\colorbox{green}{\color[gray]{0.75}FO}%
\colorbox{green}{\color[gray]{0.75}FO}%
\colorbox{green}{\color[gray]{0.75}FO}%
\colorbox{green}{\color[gray]{0.75}FO}%
\colorbox{green}{\color[gray]{0.75}FO}%
\colorbox{green}{\color[gray]{0.75}FO}%
\colorbox{green}{\color[gray]{0.75}FO}%
\colorbox{green}{\color[gray]{0.75}FO}%
\colorbox{green}{\color[gray]{0.75}FO}%
\colorbox{green}{\color[gray]{0.75}FO}%
\colorbox{green}{\color[gray]{0.75}FO}%
\colorbox{green}{\color[gray]{0.75}FO}%
\colorbox{green}{\color[gray]{0.75}FO}%
\colorbox{green}{\color[gray]{0.75}FO}%
\colorbox{green}{\color[gray]{0.75}FO}%
\colorbox{green}{\color[gray]{0.75}FO}%
\colorbox{green}{\color[gray]{0.75}FO}%
\colorbox{green}{\color[gray]{0.75}FO}%
\colorbox{green}{\color[gray]{0.75}FO}%
\colorbox{green}{\color[gray]{0.75}FO}%
\colorbox{green}{\color[gray]{0.75}FO}%
\colorbox{green}{\color[gray]{0.75}FO}%
\colorbox{green}{\color[gray]{0.75}FO}%
\colorbox{green}{\color[gray]{0.75}FO}%
\colorbox{green}{\color[gray]{0.75}FO}%
\colorbox{green}{\color[gray]{0.75}FO}%
\colorbox{green}{\color[gray]{0.75}FO}%
\colorbox{green}{\color[gray]{0.75}FO}%
\colorbox{green}{\color[gray]{0.75}FO}%
\colorbox{green}{\color[gray]{0.75}FO}%
\colorbox{green}{\color[gray]{0.75}FO}%
\colorbox{green}{\color[gray]{0.75}FO}%
\colorbox{green}{\color[gray]{0.75}FO}%
\colorbox{green}{\color[gray]{0.75}FO}%
\colorbox{green}{\color[gray]{0.75}FO}%
\colorbox{green}{\color[gray]{0.75}FO}%
\colorbox{green}{\color[gray]{0.75}FO}%
\colorbox{green}{\color[gray]{0.75}FO}%
\colorbox{green}{\color[gray]{0.75}FO}%
\colorbox{green}{\color[gray]{0.75}FO}%
\colorbox{green}{\color[gray]{0.75}FO}%
\colorbox{green}{\color[gray]{0.75}FO}%
\colorbox{green}{\color[gray]{0.75}FO}%
\colorbox{green}{\color[gray]{0.75}FO}%
\colorbox{green}{\color[gray]{0.75}FO}%
\colorbox{green}{\color[gray]{0.75}FO}%
\colorbox{green}{\color[gray]{0.75}FO}%
\colorbox{green}{\color[gray]{0.75}FO}%
\colorbox{green}{\color[gray]{0.75}FO}%
\colorbox{green}{\color[gray]{0.75}FO}%
\colorbox{green}{\color[gray]{0.75}FO}%
\colorbox{green}{\color[gray]{0.75}FO}%
\\
\colorbox{green}{\color[gray]{0.75}FO}%
\colorbox{green}{\color[gray]{0.75}FO}%
\colorbox{green}{\color[gray]{0.75}FO}%
\colorbox{green}{\color[gray]{0.75}FO}%
\colorbox{green}{\color[gray]{0.75}FO}%
\colorbox{green}{\color[gray]{0.75}FO}%
\colorbox{green}{\color[gray]{0.75}FO}%
\colorbox{green}{\color[gray]{0.75}FO}%
\colorbox{green}{\color[gray]{0.75}FO}%
\colorbox{green}{\color[gray]{0.75}FO}%
\colorbox{green}{\color[gray]{0.75}FO}%
\colorbox{green}{\color[gray]{0.75}FO}%
\colorbox{green}{\color[gray]{0.75}FO}%
\colorbox{green}{\color[gray]{0.75}FO}%
\colorbox{green}{\color[gray]{0.75}FO}%
\colorbox{green}{\color[gray]{0.75}FO}%
\colorbox{green}{\color[gray]{0.75}FO}%
\colorbox{green}{\color[gray]{0.75}FO}%
\colorbox{green}{\color[gray]{0.75}FO}%
\colorbox{green}{\color[gray]{0.75}FO}%
\colorbox{green}{\color[gray]{0.75}FO}%
\colorbox{green}{\color[gray]{0.75}FO}%
\colorbox{green}{\color[gray]{0.75}FO}%
\colorbox{green}{\color[gray]{0.75}FO}%
\colorbox{green}{\color[gray]{0.75}FO}%
\colorbox{green}{\color[gray]{0.75}FO}%
\colorbox{green}{\color[gray]{0.75}FO}%
\colorbox{green}{\color[gray]{0.75}FO}%
\colorbox{green}{\color[gray]{0.75}FO}%
\colorbox{green}{\color[gray]{0.75}FO}%
\colorbox{green}{\color[gray]{0.75}FO}%
\colorbox{green}{\color[gray]{0.75}FO}%
\colorbox{green}{\color[gray]{0.75}FO}%
\colorbox{green}{\color[gray]{0.75}FO}%
\colorbox{green}{\color[gray]{0.75}FO}%
\colorbox{green}{\color[gray]{0.75}FO}%
\colorbox{green}{\color[gray]{0.75}FO}%
\colorbox{green}{\color[gray]{0.75}FO}%
\colorbox{green}{\color[gray]{0.75}FO}%
\colorbox{green}{\color[gray]{0.75}FO}%
\colorbox{green}{\color[gray]{0.75}FO}%
\colorbox{green}{\color[gray]{0.75}FO}%
\colorbox{green}{\color[gray]{0.75}FO}%
\colorbox{green}{\color[gray]{0.75}FO}%
\colorbox{green}{\color[gray]{0.75}FO}%
\colorbox{green}{\color[gray]{0.75}FO}%
\colorbox{green}{\color[gray]{0.75}FO}%
\colorbox{green}{\color[gray]{0.75}FO}%
\colorbox{green}{\color[gray]{0.75}FO}%
\colorbox{green}{\color[gray]{0.75}FO}%
\colorbox{green}{\color[gray]{0.75}FO}%
\colorbox{green}{\color[gray]{0.75}FO}%
\colorbox{green}{\color[gray]{0.75}FO}%
\colorbox{green}{\color[gray]{0.75}FO}%
\colorbox{green}{\color[gray]{0.75}FO}%
\colorbox{green}{\color[gray]{0.75}FO}%
\colorbox{green}{\color[gray]{0.75}FO}%
\colorbox{green}{\color[gray]{0.75}FO}%
\colorbox{green}{\color[gray]{0.75}FO}%
\colorbox{green}{\color[gray]{0.75}FO}%
\colorbox{green}{\color[gray]{0.75}FO}%
\colorbox{green}{\color[gray]{0.75}FO}%
\colorbox{green}{\color[gray]{0.75}FO}%
\colorbox{green}{\color[gray]{0.75}FO}%
\colorbox{green}{\color[gray]{0.75}FO}%
\colorbox{green}{\color[gray]{0.75}FO}%
\colorbox{green}{\color[gray]{0.75}FO}%
\colorbox{green}{\color[gray]{0.75}FO}%
\colorbox{green}{\color[gray]{0.75}FO}%
\colorbox{green}{\color[gray]{0.75}FO}%
\colorbox{green}{\color[gray]{0.75}FO}%
\colorbox{green}{\color[gray]{0.75}FO}%
\colorbox{green}{\color[gray]{0.75}FO}%
\colorbox{green}{\color[gray]{0.75}FO}%
\colorbox{green}{\color[gray]{0.75}FO}%
\colorbox{green}{\color[gray]{0.75}FO}%
\colorbox{green}{\color[gray]{0.75}FO}%
\colorbox{green}{\color[gray]{0.75}FO}%
\colorbox{green}{\color[gray]{0.75}FO}%
\colorbox{green}{\color[gray]{0.75}FO}%
\colorbox{green}{\color[gray]{0.75}FO}%
\colorbox{green}{\color[gray]{0.75}FO}%
\colorbox{green}{\color[gray]{0.75}FO}%
\colorbox{green}{\color[gray]{0.75}FO}%
\colorbox{green}{\color[gray]{0.75}FO}%
\colorbox{green}{\color[gray]{0.75}FO}%
\colorbox{green}{\color[gray]{0.75}FO}%
\colorbox{green}{\color[gray]{0.75}FO}%
\colorbox{green}{\color[gray]{0.75}FO}%
\colorbox{green}{\color[gray]{0.75}FO}%
\colorbox{green}{\color[gray]{0.75}FO}%
\colorbox{green}{\color[gray]{0.75}FO}%
\colorbox{green}{\color[gray]{0.75}FO}%
\colorbox{green}{\color[gray]{0.75}FO}%
\colorbox{green}{\color[gray]{0.75}FO}%
\colorbox{green}{\color[gray]{0.75}FO}%
\colorbox{green}{\color[gray]{0.75}FO}%
\colorbox{green}{\color[gray]{0.75}FO}%
\colorbox{green}{\color[gray]{0.75}FO}%
\colorbox{green}{\color[gray]{0.75}FO}%
\\
\colorbox{green}{\color[gray]{0.75}FO}%
\colorbox{green}{\color[gray]{0.75}FO}%
\colorbox{green}{\color[gray]{0.75}FO}%
\colorbox{green}{\color[gray]{0.75}FO}%
\colorbox{green}{\color[gray]{0.75}FO}%
\colorbox{green}{\color[gray]{0.75}FO}%
\colorbox{green}{\color[gray]{0.75}FO}%
\colorbox{green}{\color[gray]{0.75}FO}%
\colorbox{green}{\color[gray]{0.75}FO}%
\colorbox{green}{\color[gray]{0.75}FO}%
\colorbox{green}{\color[gray]{0.75}FO}%
\colorbox{green}{\color[gray]{0.75}FO}%
\colorbox{green}{\color[gray]{0.75}FO}%
\colorbox{green}{\color[gray]{0.75}FO}%
\colorbox{green}{\color[gray]{0.75}FO}%
\colorbox{green}{\color[gray]{0.75}FO}%
\colorbox{green}{\color[gray]{0.75}FO}%
\colorbox{green}{\color[gray]{0.75}FO}%
\colorbox{green}{\color[gray]{0.75}FO}%
\colorbox{green}{\color[gray]{0.75}FO}%
\colorbox{green}{\color[gray]{0.75}FO}%
\colorbox{green}{\color[gray]{0.75}FO}%
\colorbox{green}{\color[gray]{0.75}FO}%
\colorbox{green}{\color[gray]{0.75}FO}%
\colorbox{green}{\color[gray]{0.75}FO}%
\colorbox{green}{\color[gray]{0.75}FO}%
\colorbox{green}{\color[gray]{0.75}FO}%
\colorbox{green}{\color[gray]{0.75}FO}%
\colorbox{green}{\color[gray]{0.75}FO}%
\colorbox{green}{\color[gray]{0.75}FO}%
\colorbox{green}{\color[gray]{0.75}FO}%
\colorbox{green}{\color[gray]{0.75}FO}%
\colorbox{green}{\color[gray]{0.75}FO}%
\colorbox{green}{\color[gray]{0.75}FO}%
\colorbox{green}{\color[gray]{0.75}FO}%
\colorbox{green}{\color[gray]{0.75}FO}%
\colorbox{green}{\color[gray]{0.75}FO}%
\colorbox{green}{\color[gray]{0.75}FO}%
\colorbox{green}{\color[gray]{0.75}FO}%
\colorbox{green}{\color[gray]{0.75}FO}%
\colorbox{green}{\color[gray]{0.75}FO}%
\colorbox{green}{\color[gray]{0.75}FO}%
\colorbox{green}{\color[gray]{0.75}FO}%
\colorbox{green}{\color[gray]{0.75}FO}%
\colorbox{green}{\color[gray]{0.75}FO}%
\colorbox{green}{\color[gray]{0.75}FO}%
\colorbox{green}{\color[gray]{0.75}FO}%
\colorbox{green}{\color[gray]{0.75}FO}%
\colorbox{green}{\color[gray]{0.75}FO}%
\colorbox{green}{\color[gray]{0.75}FO}%
\colorbox{green}{\color[gray]{0.75}FO}%
\colorbox{green}{\color[gray]{0.75}FO}%
\colorbox{green}{\color[gray]{0.75}FO}%
\colorbox{green}{\color[gray]{0.75}FO}%
\colorbox{green}{\color[gray]{0.75}FO}%
\colorbox{green}{\color[gray]{0.75}FO}%
\colorbox{green}{\color[gray]{0.75}FO}%
\colorbox{green}{\color[gray]{0.75}FO}%
\colorbox{green}{\color[gray]{0.75}FO}%
\colorbox{green}{\color[gray]{0.75}FO}%
\colorbox{green}{\color[gray]{0.75}FO}%
\colorbox{green}{\color[gray]{0.75}FO}%
\colorbox{green}{\color[gray]{0.75}FO}%
\colorbox{green}{\color[gray]{0.75}FO}%
\colorbox{green}{\color[gray]{0.75}FO}%
\colorbox{green}{\color[gray]{0.75}FO}%
\colorbox{green}{\color[gray]{0.75}FO}%
\colorbox{green}{\color[gray]{0.75}FO}%
\colorbox{green}{\color[gray]{0.75}FO}%
\colorbox{green}{\color[gray]{0.75}FO}%
\colorbox{green}{\color[gray]{0.75}FO}%
\colorbox{green}{\color[gray]{0.75}FO}%
\colorbox{green}{\color[gray]{0.75}FO}%
\colorbox{green}{\color[gray]{0.75}FO}%
\colorbox{green}{\color[gray]{0.75}FO}%
\colorbox{green}{\color[gray]{0.75}FO}%
\colorbox{green}{\color[gray]{0.75}FO}%
\colorbox{green}{\color[gray]{0.75}FO}%
\colorbox{green}{\color[gray]{0.75}FO}%
\colorbox{green}{\color[gray]{0.75}FO}%
\colorbox{green}{\color[gray]{0.75}FO}%
\colorbox{green}{\color[gray]{0.75}FO}%
\colorbox{green}{\color[gray]{0.75}FO}%
\colorbox{green}{\color[gray]{0.75}FO}%
\colorbox{green}{\color[gray]{0.75}FO}%
\colorbox{green}{\color[gray]{0.75}FO}%
\colorbox{green}{\color[gray]{0.75}FO}%
\colorbox{green}{\color[gray]{0.75}FO}%
\colorbox{green}{\color[gray]{0.75}FO}%
\colorbox{green}{\color[gray]{0.75}FO}%
\colorbox{green}{\color[gray]{0.75}FO}%
\colorbox{green}{\color[gray]{0.75}FO}%
\colorbox{green}{\color[gray]{0.75}FO}%
\colorbox{green}{\color[gray]{0.75}FO}%
\colorbox{green}{\color[gray]{0.75}FO}%
\colorbox{green}{\color[gray]{0.75}FO}%
\colorbox{green}{\color[gray]{0.75}FO}%
\colorbox{green}{\color[gray]{0.75}FO}%
\colorbox{green}{\color[gray]{0.75}FO}%
\colorbox{green}{\color[gray]{0.75}FO}%
\\
\colorbox{green}{\color[gray]{0.75}FO}%
\colorbox{green}{\color[gray]{0.75}FO}%
\colorbox{green}{\color[gray]{0.75}FO}%
\colorbox{green}{\color[gray]{0.75}FO}%
\colorbox{green}{\color[gray]{0.75}FO}%
\colorbox{green}{\color[gray]{0.75}FO}%
\colorbox{green}{\color[gray]{0.75}FO}%
\colorbox{green}{\color[gray]{0.75}FO}%
\colorbox{green}{\color[gray]{0.75}FO}%
\colorbox{green}{\color[gray]{0.75}FO}%
\colorbox{green}{\color[gray]{0.75}FO}%
\colorbox{green}{\color[gray]{0.75}FO}%
\colorbox{green}{\color[gray]{0.75}FO}%
\colorbox{green}{\color[gray]{0.75}FO}%
\colorbox{green}{\color[gray]{0.75}FO}%
\colorbox{green}{\color[gray]{0.75}FO}%
\colorbox{green}{\color[gray]{0.75}FO}%
\colorbox{green}{\color[gray]{0.75}FO}%
\colorbox{green}{\color[gray]{0.75}FO}%
\colorbox{green}{\color[gray]{0.75}FO}%
\colorbox{green}{\color[gray]{0.75}FO}%
\colorbox{green}{\color[gray]{0.75}FO}%
\colorbox{green}{\color[gray]{0.75}FO}%
\colorbox{green}{\color[gray]{0.75}FO}%
\colorbox{green}{\color[gray]{0.75}FO}%
\colorbox{green}{\color[gray]{0.75}FO}%
\colorbox{green}{\color[gray]{0.75}FO}%
\colorbox{green}{\color[gray]{0.75}FO}%
\colorbox{green}{\color[gray]{0.75}FO}%
\colorbox{green}{\color[gray]{0.75}FO}%
\colorbox{green}{\color[gray]{0.75}FO}%
\colorbox{green}{\color[gray]{0.75}FO}%
\colorbox{green}{\color[gray]{0.75}FO}%
\colorbox{green}{\color[gray]{0.75}FO}%
\colorbox{green}{\color[gray]{0.75}FO}%
\colorbox{green}{\color[gray]{0.75}FO}%
\colorbox{green}{\color[gray]{0.75}FO}%
\colorbox{green}{\color[gray]{0.75}FO}%
\colorbox{green}{\color[gray]{0.75}FO}%
\colorbox{green}{\color[gray]{0.75}FO}%
\colorbox{green}{\color[gray]{0.75}FO}%
\colorbox{green}{\color[gray]{0.75}FO}%
\colorbox{green}{\color[gray]{0.75}FO}%
\colorbox{green}{\color[gray]{0.75}FO}%
\colorbox{green}{\color[rgb]{1,0,0}\textbf{BU}}%
\colorbox{green}{\color[gray]{0.75}FO}%
\colorbox{green}{\color[gray]{0.75}FO}%
\colorbox{green}{\color[gray]{0.75}FO}%
\colorbox{green}{\color[gray]{0.75}FO}%
\colorbox{green}{\color[gray]{0.75}FO}%
\colorbox{green}{\color[gray]{0.75}FO}%
\colorbox{green}{\color[gray]{0.75}FO}%
\colorbox{green}{\color[gray]{0.75}FO}%
\colorbox{green}{\color[gray]{0.75}FO}%
\colorbox{green}{\color[gray]{0.75}FO}%
\colorbox{green}{\color[gray]{0.75}FO}%
\colorbox{green}{\color[gray]{0.75}FO}%
\colorbox{green}{\color[gray]{0.75}FO}%
\colorbox{green}{\color[gray]{0.75}FO}%
\colorbox{green}{\color[gray]{0.75}FO}%
\colorbox{green}{\color[gray]{0.75}FO}%
\colorbox{green}{\color[gray]{0.75}FO}%
\colorbox{green}{\color[gray]{0.75}FO}%
\colorbox{green}{\color[gray]{0.75}FO}%
\colorbox{green}{\color[gray]{0.75}FO}%
\colorbox{green}{\color[gray]{0.75}FO}%
\colorbox{green}{\color[gray]{0.75}FO}%
\colorbox{green}{\color[gray]{0.75}FO}%
\colorbox{green}{\color[gray]{0.75}FO}%
\colorbox{green}{\color[gray]{0.75}FO}%
\colorbox{green}{\color[gray]{0.75}FO}%
\colorbox{green}{\color[gray]{0.75}FO}%
\colorbox{green}{\color[gray]{0.75}FO}%
\colorbox{green}{\color[gray]{0.75}FO}%
\colorbox{green}{\color[gray]{0.75}FO}%
\colorbox{green}{\color[gray]{0.75}FO}%
\colorbox{green}{\color[gray]{0.75}FO}%
\colorbox{green}{\color[gray]{0.75}FO}%
\colorbox{green}{\color[gray]{0.75}FO}%
\colorbox{green}{\color[gray]{0.75}FO}%
\colorbox{green}{\color[gray]{0.75}FO}%
\colorbox{green}{\color[gray]{0.75}FO}%
\colorbox{green}{\color[gray]{0.75}FO}%
\colorbox{green}{\color[gray]{0.75}FO}%
\colorbox{green}{\color[gray]{0.75}FO}%
\colorbox{green}{\color[gray]{0.75}FO}%
\colorbox{green}{\color[gray]{0.75}FO}%
\colorbox{green}{\color[gray]{0.75}FO}%
\colorbox{green}{\color[gray]{0.75}FO}%
\colorbox{green}{\color[gray]{0.75}FO}%
\colorbox{green}{\color[gray]{0.75}FO}%
\colorbox{green}{\color[gray]{0.75}FO}%
\colorbox{green}{\color[gray]{0.75}FO}%
\colorbox{green}{\color[gray]{0.75}FO}%
\colorbox{green}{\color[gray]{0.75}FO}%
\colorbox{green}{\color[gray]{0.75}FO}%
\colorbox{green}{\color[gray]{0.75}FO}%
\colorbox{green}{\color[gray]{0.75}FO}%
\colorbox{green}{\color[gray]{0.75}FO}%
\colorbox{green}{\color[gray]{0.75}FO}%
\\
\colorbox{green}{\color[gray]{0.75}FO}%
\colorbox{green}{\color[gray]{0.75}FO}%
\colorbox{green}{\color[gray]{0.75}FO}%
\colorbox{green}{\color[gray]{0.75}FO}%
\colorbox{green}{\color[gray]{0.75}FO}%
\colorbox{green}{\color[gray]{0.75}FO}%
\colorbox{green}{\color[gray]{0.75}FO}%
\colorbox{green}{\color[gray]{0.75}FO}%
\colorbox{green}{\color[gray]{0.75}FO}%
\colorbox{green}{\color[gray]{0.75}FO}%
\colorbox{green}{\color[gray]{0.75}FO}%
\colorbox{green}{\color[gray]{0.75}FO}%
\colorbox{green}{\color[gray]{0.75}FO}%
\colorbox{green}{\color[gray]{0.75}FO}%
\colorbox{green}{\color[gray]{0.75}FO}%
\colorbox{green}{\color[gray]{0.75}FO}%
\colorbox{green}{\color[gray]{0.75}FO}%
\colorbox{green}{\color[gray]{0.75}FO}%
\colorbox{green}{\color[gray]{0.75}FO}%
\colorbox{green}{\color[gray]{0.75}FO}%
\colorbox{green}{\color[gray]{0.75}FO}%
\colorbox{green}{\color[gray]{0.75}FO}%
\colorbox{green}{\color[gray]{0.75}FO}%
\colorbox{green}{\color[gray]{0.75}FO}%
\colorbox{green}{\color[gray]{0.75}FO}%
\colorbox{green}{\color[gray]{0.75}FO}%
\colorbox{green}{\color[gray]{0.75}FO}%
\colorbox{green}{\color[gray]{0.75}FO}%
\colorbox{green}{\color[gray]{0.75}FO}%
\colorbox{green}{\color[gray]{0.75}FO}%
\colorbox{green}{\color[gray]{0.75}FO}%
\colorbox{green}{\color[gray]{0.75}FO}%
\colorbox{green}{\color[gray]{0.75}FO}%
\colorbox{green}{\color[gray]{0.75}FO}%
\colorbox{green}{\color[gray]{0.75}FO}%
\colorbox{green}{\color[gray]{0.75}FO}%
\colorbox{green}{\color[gray]{0.75}FO}%
\colorbox{green}{\color[gray]{0.75}FO}%
\colorbox{green}{\color[gray]{0.75}FO}%
\colorbox{green}{\color[gray]{0.75}FO}%
\colorbox{green}{\color[gray]{0.75}FO}%
\colorbox{green}{\color[gray]{0.75}FO}%
\colorbox{green}{\color[gray]{0.75}FO}%
\colorbox{green}{\color[gray]{0.75}FO}%
\colorbox{green}{\color[gray]{0.75}FO}%
\colorbox{green}{\color[gray]{0.75}FO}%
\colorbox{green}{\color[gray]{0.75}FO}%
\colorbox{green}{\color[gray]{0.75}FO}%
\colorbox{green}{\color[gray]{0.75}FO}%
\colorbox{green}{\color[gray]{0.75}FO}%
\colorbox{green}{\color[gray]{0.75}FO}%
\colorbox{green}{\color[gray]{0.75}FO}%
\colorbox{green}{\color[gray]{0.75}FO}%
\colorbox{green}{\color[gray]{0.75}FO}%
\colorbox{green}{\color[gray]{0.75}FO}%
\colorbox{green}{\color[gray]{0.75}FO}%
\colorbox{green}{\color[gray]{0.75}FO}%
\colorbox{green}{\color[gray]{0.75}FO}%
\colorbox{green}{\color[gray]{0.75}FO}%
\colorbox{green}{\color[gray]{0.75}FO}%
\colorbox{green}{\color[gray]{0.75}FO}%
\colorbox{green}{\color[gray]{0.75}FO}%
\colorbox{green}{\color[gray]{0.75}FO}%
\colorbox{green}{\color[gray]{0.75}FO}%
\colorbox{green}{\color[gray]{0.75}FO}%
\colorbox{green}{\color[gray]{0.75}FO}%
\colorbox{green}{\color[gray]{0.75}FO}%
\colorbox{green}{\color[gray]{0.75}FO}%
\colorbox{green}{\color[gray]{0.75}FO}%
\colorbox{green}{\color[gray]{0.75}FO}%
\colorbox{green}{\color[gray]{0.75}FO}%
\colorbox{green}{\color[gray]{0.75}FO}%
\colorbox{green}{\color[gray]{0.75}FO}%
\colorbox{green}{\color[gray]{0.75}FO}%
\colorbox{green}{\color[gray]{0.75}FO}%
\colorbox{green}{\color[gray]{0.75}FO}%
\colorbox{green}{\color[gray]{0.75}FO}%
\colorbox{green}{\color[gray]{0.75}FO}%
\colorbox{green}{\color[gray]{0.75}FO}%
\colorbox{green}{\color[gray]{0.75}FO}%
\colorbox{green}{\color[gray]{0.75}FO}%
\colorbox{green}{\color[gray]{0.75}FO}%
\colorbox{green}{\color[gray]{0.75}FO}%
\colorbox{green}{\color[gray]{0.75}FO}%
\colorbox{green}{\color[gray]{0.75}FO}%
\colorbox{green}{\color[gray]{0.75}FO}%
\colorbox{green}{\color[gray]{0.75}FO}%
\colorbox{green}{\color[gray]{0.75}FO}%
\colorbox{green}{\color[gray]{0.75}FO}%
\colorbox{green}{\color[gray]{0.75}FO}%
\colorbox{green}{\color[gray]{0.75}FO}%
\colorbox{green}{\color[gray]{0.75}FO}%
\colorbox{green}{\color[gray]{0.75}FO}%
\colorbox{green}{\color[gray]{0.75}FO}%
\colorbox{green}{\color[gray]{0.75}FO}%
\colorbox{green}{\color[gray]{0.75}FO}%
\colorbox{green}{\color[gray]{0.75}FO}%
\colorbox{green}{\color[gray]{0.75}FO}%
\colorbox{green}{\color[gray]{0.75}FO}%
\colorbox{green}{\color[gray]{0.75}FO}%
\\
\colorbox{green}{\color[gray]{0.75}FO}%
\colorbox{green}{\color[gray]{0.75}FO}%
\colorbox{green}{\color[gray]{0.75}FO}%
\colorbox{green}{\color[gray]{0.75}FO}%
\colorbox{green}{\color[gray]{0.75}FO}%
\colorbox{green}{\color[gray]{0.75}FO}%
\colorbox{green}{\color[gray]{0.75}FO}%
\colorbox{green}{\color[gray]{0.75}FO}%
\colorbox{green}{\color[gray]{0.75}FO}%
\colorbox{green}{\color[gray]{0.75}FO}%
\colorbox{green}{\color[gray]{0.75}FO}%
\colorbox{green}{\color[gray]{0.75}FO}%
\colorbox{green}{\color[gray]{0.75}FO}%
\colorbox{green}{\color[gray]{0.75}FO}%
\colorbox{green}{\color[gray]{0.75}FO}%
\colorbox{green}{\color[gray]{0.75}FO}%
\colorbox{green}{\color[gray]{0.75}FO}%
\colorbox{green}{\color[gray]{0.75}FO}%
\colorbox{green}{\color[gray]{0.75}FO}%
\colorbox{green}{\color[gray]{0.75}FO}%
\colorbox{green}{\color[gray]{0.75}FO}%
\colorbox{green}{\color[gray]{0.75}FO}%
\colorbox{green}{\color[gray]{0.75}FO}%
\colorbox{green}{\color[gray]{0.75}FO}%
\colorbox{green}{\color[gray]{0.75}FO}%
\colorbox{green}{\color[gray]{0.75}FO}%
\colorbox{green}{\color[gray]{0.75}FO}%
\colorbox{green}{\color[gray]{0.75}FO}%
\colorbox{green}{\color[gray]{0.75}FO}%
\colorbox{green}{\color[gray]{0.75}FO}%
\colorbox{green}{\color[gray]{0.75}FO}%
\colorbox{green}{\color[gray]{0.75}FO}%
\colorbox{green}{\color[gray]{0.75}FO}%
\colorbox{green}{\color[gray]{0.75}FO}%
\colorbox{green}{\color[gray]{0.75}FO}%
\colorbox{green}{\color[gray]{0.75}FO}%
\colorbox{green}{\color[gray]{0.75}FO}%
\colorbox{green}{\color[gray]{0.75}FO}%
\colorbox{green}{\color[gray]{0.75}FO}%
\colorbox{green}{\color[gray]{0.75}FO}%
\colorbox{green}{\color[gray]{0.75}FO}%
\colorbox{green}{\color[gray]{0.75}FO}%
\colorbox{green}{\color[gray]{0.75}FO}%
\colorbox{green}{\color[gray]{0.75}FO}%
\colorbox{green}{\color[gray]{0.75}FO}%
\colorbox{green}{\color[gray]{0.75}FO}%
\colorbox{green}{\color[gray]{0.75}FO}%
\colorbox{green}{\color[gray]{0.75}FO}%
\colorbox{green}{\color[gray]{0.75}FO}%
\colorbox{green}{\color[gray]{0.75}FO}%
\colorbox{green}{\color[gray]{0.75}FO}%
\colorbox{green}{\color[gray]{0.75}FO}%
\colorbox{green}{\color[gray]{0.75}FO}%
\colorbox{green}{\color[gray]{0.75}FO}%
\colorbox{green}{\color[gray]{0.75}FO}%
\colorbox{green}{\color[gray]{0.75}FO}%
\colorbox{green}{\color[gray]{0.75}FO}%
\colorbox{green}{\color[gray]{0.75}FO}%
\colorbox{green}{\color[gray]{0.75}FO}%
\colorbox{green}{\color[gray]{0.75}FO}%
\colorbox{green}{\color[gray]{0.75}FO}%
\colorbox{green}{\color[gray]{0.75}FO}%
\colorbox{green}{\color[gray]{0.75}FO}%
\colorbox{green}{\color[gray]{0.75}FO}%
\colorbox{green}{\color[gray]{0.75}FO}%
\colorbox{green}{\color[gray]{0.75}FO}%
\colorbox{green}{\color[gray]{0.75}FO}%
\colorbox{green}{\color[gray]{0.75}FO}%
\colorbox{green}{\color[gray]{0.75}FO}%
\colorbox{green}{\color[gray]{0.75}FO}%
\colorbox{green}{\color[gray]{0.75}FO}%
\colorbox{green}{\color[gray]{0.75}FO}%
\colorbox{green}{\color[gray]{0.75}FO}%
\colorbox{green}{\color[gray]{0.75}FO}%
\colorbox{green}{\color[gray]{0.75}FO}%
\colorbox{green}{\color[gray]{0.75}FO}%
\colorbox{green}{\color[gray]{0.75}FO}%
\colorbox{green}{\color[gray]{0.75}FO}%
\colorbox{green}{\color[gray]{0.75}FO}%
\colorbox{green}{\color[gray]{0.75}FO}%
\colorbox{green}{\color[gray]{0.75}FO}%
\colorbox{green}{\color[gray]{0.75}FO}%
\colorbox{green}{\color[gray]{0.75}FO}%
\colorbox{green}{\color[gray]{0.75}FO}%
\colorbox{green}{\color[gray]{0.75}FO}%
\colorbox{green}{\color[gray]{0.75}FO}%
\colorbox{green}{\color[gray]{0.75}FO}%
\colorbox{green}{\color[gray]{0.75}FO}%
\colorbox{green}{\color[gray]{0.75}FO}%
\colorbox{green}{\color[gray]{0.75}FO}%
\colorbox{green}{\color[gray]{0.75}FO}%
\colorbox{green}{\color[gray]{0.75}FO}%
\colorbox{green}{\color[gray]{0.75}FO}%
\colorbox{green}{\color[gray]{0.75}FO}%
\colorbox{green}{\color[gray]{0.75}FO}%
\colorbox{green}{\color[gray]{0.75}FO}%
\colorbox{green}{\color[gray]{0.75}FO}%
\colorbox{green}{\color[gray]{0.75}FO}%
\colorbox{green}{\color[gray]{0.75}FO}%
\colorbox{green}{\color[gray]{0.75}FO}%
\\
\colorbox{green}{\color[gray]{0.75}FO}%
\colorbox{green}{\color[gray]{0.75}FO}%
\colorbox{green}{\color[gray]{0.75}FO}%
\colorbox{green}{\color[gray]{0.75}FO}%
\colorbox{green}{\color[gray]{0.75}FO}%
\colorbox{green}{\color[gray]{0.75}FO}%
\colorbox{green}{\color[gray]{0.75}FO}%
\colorbox{green}{\color[gray]{0.75}FO}%
\colorbox{green}{\color[gray]{0.75}FO}%
\colorbox{green}{\color[gray]{0.75}FO}%
\colorbox{green}{\color[gray]{0.75}FO}%
\colorbox{green}{\color[gray]{0.75}FO}%
\colorbox{green}{\color[gray]{0.75}FO}%
\colorbox{green}{\color[gray]{0.75}FO}%
\colorbox{green}{\color[gray]{0.75}FO}%
\colorbox{green}{\color[gray]{0.75}FO}%
\colorbox{green}{\color[gray]{0.75}FO}%
\colorbox{green}{\color[gray]{0.75}FO}%
\colorbox{green}{\color[gray]{0.75}FO}%
\colorbox{green}{\color[gray]{0.75}FO}%
\colorbox{green}{\color[gray]{0.75}FO}%
\colorbox{green}{\color[gray]{0.75}FO}%
\colorbox{green}{\color[gray]{0.75}FO}%
\colorbox{green}{\color[gray]{0.75}FO}%
\colorbox{green}{\color[gray]{0.75}FO}%
\colorbox{green}{\color[gray]{0.75}FO}%
\colorbox{green}{\color[gray]{0.75}FO}%
\colorbox{green}{\color[gray]{0.75}FO}%
\colorbox{green}{\color[gray]{0.75}FO}%
\colorbox{green}{\color[gray]{0.75}FO}%
\colorbox{green}{\color[gray]{0.75}FO}%
\colorbox{green}{\color[gray]{0.75}FO}%
\colorbox{green}{\color[gray]{0.75}FO}%
\colorbox{green}{\color[gray]{0.75}FO}%
\colorbox{green}{\color[gray]{0.75}FO}%
\colorbox{green}{\color[gray]{0.75}FO}%
\colorbox{green}{\color[gray]{0.75}FO}%
\colorbox{green}{\color[gray]{0.75}FO}%
\colorbox{green}{\color[gray]{0.75}FO}%
\colorbox{green}{\color[gray]{0.75}FO}%
\colorbox{green}{\color[gray]{0.75}FO}%
\colorbox{green}{\color[gray]{0.75}FO}%
\colorbox{green}{\color[gray]{0.75}FO}%
\colorbox{green}{\color[gray]{0.75}FO}%
\colorbox{green}{\color[gray]{0.75}FO}%
\colorbox{green}{\color[gray]{0.75}FO}%
\colorbox{green}{\color[gray]{0.75}FO}%
\colorbox{green}{\color[gray]{0.75}FO}%
\colorbox{green}{\color[gray]{0.75}FO}%
\colorbox{green}{\color[gray]{0.75}FO}%
\colorbox{green}{\color[gray]{0.75}FO}%
\colorbox{green}{\color[gray]{0.75}FO}%
\colorbox{green}{\color[gray]{0.75}FO}%
\colorbox{green}{\color[gray]{0.75}FO}%
\colorbox{green}{\color[gray]{0.75}FO}%
\colorbox{green}{\color[gray]{0.75}FO}%
\colorbox{green}{\color[gray]{0.75}FO}%
\colorbox{green}{\color[gray]{0.75}FO}%
\colorbox{green}{\color[gray]{0.75}FO}%
\colorbox{green}{\color[gray]{0.75}FO}%
\colorbox{green}{\color[gray]{0.75}FO}%
\colorbox{green}{\color[gray]{0.75}FO}%
\colorbox{green}{\color[gray]{0.75}FO}%
\colorbox{green}{\color[gray]{0.75}FO}%
\colorbox{green}{\color[gray]{0.75}FO}%
\colorbox{green}{\color[gray]{0.75}FO}%
\colorbox{green}{\color[gray]{0.75}FO}%
\colorbox{green}{\color[gray]{0.75}FO}%
\colorbox{green}{\color[gray]{0.75}FO}%
\colorbox{green}{\color[gray]{0.75}FO}%
\colorbox{green}{\color[gray]{0.75}FO}%
\colorbox{green}{\color[gray]{0.75}FO}%
\colorbox{green}{\color[gray]{0.75}FO}%
\colorbox{green}{\color[gray]{0.75}FO}%
\colorbox{green}{\color[gray]{0.75}FO}%
\colorbox{green}{\color[gray]{0.75}FO}%
\colorbox{green}{\color[gray]{0.75}FO}%
\colorbox{green}{\color[gray]{0.75}FO}%
\colorbox{green}{\color[gray]{0.75}FO}%
\colorbox{green}{\color[gray]{0.75}FO}%
\colorbox{green}{\color[gray]{0.75}FO}%
\colorbox{green}{\color[gray]{0.75}FO}%
\colorbox{green}{\color[gray]{0.75}FO}%
\colorbox{green}{\color[gray]{0.75}FO}%
\colorbox{green}{\color[gray]{0.75}FO}%
\colorbox{green}{\color[gray]{0.75}FO}%
\colorbox{green}{\color[gray]{0.75}FO}%
\colorbox{green}{\color[gray]{0.75}FO}%
\colorbox{green}{\color[gray]{0.75}FO}%
\colorbox{green}{\color[gray]{0.75}FO}%
\colorbox{green}{\color[gray]{0.75}FO}%
\colorbox{green}{\color[gray]{0.75}FO}%
\colorbox{green}{\color[gray]{0.75}FO}%
\colorbox{green}{\color[gray]{0.75}FO}%
\colorbox{green}{\color[gray]{0.75}FO}%
\colorbox{green}{\color[gray]{0.75}FO}%
\colorbox{green}{\color[gray]{0.75}FO}%
\colorbox{green}{\color[gray]{0.75}FO}%
\colorbox{green}{\color[gray]{0.75}FO}%
\colorbox{green}{\color[gray]{0.75}FO}%
\\
\colorbox{green}{\color[gray]{0.75}FO}%
\colorbox{green}{\color[gray]{0.75}FO}%
\colorbox{green}{\color[gray]{0.75}FO}%
\colorbox{green}{\color[gray]{0.75}FO}%
\colorbox{green}{\color[gray]{0.75}FO}%
\colorbox{green}{\color[gray]{0.75}FO}%
\colorbox{green}{\color[gray]{0.75}FO}%
\colorbox{green}{\color[gray]{0.75}FO}%
\colorbox{green}{\color[gray]{0.75}FO}%
\colorbox{green}{\color[gray]{0.75}FO}%
\colorbox{green}{\color[gray]{0.75}FO}%
\colorbox{green}{\color[gray]{0.75}FO}%
\colorbox{green}{\color[gray]{0.75}FO}%
\colorbox{green}{\color[gray]{0.75}FO}%
\colorbox{green}{\color[gray]{0.75}FO}%
\colorbox{green}{\color[gray]{0.75}FO}%
\colorbox{green}{\color[gray]{0.75}FO}%
\colorbox{green}{\color[gray]{0.75}FO}%
\colorbox{green}{\color[gray]{0.75}FO}%
\colorbox{green}{\color[gray]{0.75}FO}%
\colorbox{green}{\color[gray]{0.75}FO}%
\colorbox{green}{\color[gray]{0.75}FO}%
\colorbox{green}{\color[gray]{0.75}FO}%
\colorbox{green}{\color[gray]{0.75}FO}%
\colorbox{green}{\color[gray]{0.75}FO}%
\colorbox{green}{\color[gray]{0.75}FO}%
\colorbox{green}{\color[gray]{0.75}FO}%
\colorbox{green}{\color[gray]{0.75}FO}%
\colorbox{green}{\color[gray]{0.75}FO}%
\colorbox{green}{\color[gray]{0.75}FO}%
\colorbox{green}{\color[gray]{0.75}FO}%
\colorbox{green}{\color[gray]{0.75}FO}%
\colorbox{green}{\color[gray]{0.75}FO}%
\colorbox{green}{\color[gray]{0.75}FO}%
\colorbox{green}{\color[gray]{0.75}FO}%
\colorbox{green}{\color[gray]{0.75}FO}%
\colorbox{green}{\color[gray]{0.75}FO}%
\colorbox{green}{\color[gray]{0.75}FO}%
\colorbox{green}{\color[gray]{0.75}FO}%
\colorbox{green}{\color[gray]{0.75}FO}%
\colorbox{green}{\color[gray]{0.75}FO}%
\colorbox{green}{\color[gray]{0.75}FO}%
\colorbox{green}{\color[gray]{0.75}FO}%
\colorbox{green}{\color[gray]{0.75}FO}%
\colorbox{green}{\color[gray]{0.75}FO}%
\colorbox{green}{\color[gray]{0.75}FO}%
\colorbox{green}{\color[gray]{0.75}FO}%
\colorbox{green}{\color[gray]{0.75}FO}%
\colorbox{green}{\color[gray]{0.75}FO}%
\colorbox{green}{\color[gray]{0.75}FO}%
\colorbox{green}{\color[gray]{0.75}FO}%
\colorbox{green}{\color[gray]{0.75}FO}%
\colorbox{green}{\color[gray]{0.75}FO}%
\colorbox{green}{\color[gray]{0.75}FO}%
\colorbox{green}{\color[gray]{0.75}FO}%
\colorbox{green}{\color[gray]{0.75}FO}%
\colorbox{green}{\color[gray]{0.75}FO}%
\colorbox{green}{\color[gray]{0.75}FO}%
\colorbox{green}{\color[gray]{0.75}FO}%
\colorbox{green}{\color[gray]{0.75}FO}%
\colorbox{green}{\color[gray]{0.75}FO}%
\colorbox{green}{\color[gray]{0.75}FO}%
\colorbox{green}{\color[gray]{0.75}FO}%
\colorbox{green}{\color[gray]{0.75}FO}%
\colorbox{green}{\color[gray]{0.75}FO}%
\colorbox{green}{\color[gray]{0.75}FO}%
\colorbox{green}{\color[gray]{0.75}FO}%
\colorbox{green}{\color[gray]{0.75}FO}%
\colorbox{green}{\color[gray]{0.75}FO}%
\colorbox{green}{\color[gray]{0.75}FO}%
\colorbox{green}{\color[gray]{0.75}FO}%
\colorbox{green}{\color[gray]{0.75}FO}%
\colorbox{green}{\color[gray]{0.75}FO}%
\colorbox{green}{\color[gray]{0.75}FO}%
\colorbox{green}{\color[gray]{0.75}FO}%
\colorbox{green}{\color[gray]{0.75}FO}%
\colorbox{green}{\color[gray]{0.75}FO}%
\colorbox{green}{\color[gray]{0.75}FO}%
\colorbox{green}{\color[gray]{0.75}FO}%
\colorbox{green}{\color[gray]{0.75}FO}%
\colorbox{green}{\color[gray]{0.75}FO}%
\colorbox{green}{\color[gray]{0.75}FO}%
\colorbox{green}{\color[gray]{0.75}FO}%
\colorbox{green}{\color[gray]{0.75}FO}%
\colorbox{green}{\color[gray]{0.75}FO}%
\colorbox{green}{\color[gray]{0.75}FO}%
\colorbox{green}{\color[gray]{0.75}FO}%
\colorbox{green}{\color[gray]{0.75}FO}%
\colorbox{green}{\color[gray]{0.75}FO}%
\colorbox{green}{\color[gray]{0.75}FO}%
\colorbox{green}{\color[gray]{0.75}FO}%
\colorbox{green}{\color[gray]{0.75}FO}%
\colorbox{green}{\color[gray]{0.75}FO}%
\colorbox{green}{\color[gray]{0.75}FO}%
\colorbox{green}{\color[gray]{0.75}FO}%
\colorbox{green}{\color[gray]{0.75}FO}%
\colorbox{green}{\color[gray]{0.75}FO}%
\colorbox{green}{\color[gray]{0.75}FO}%
\colorbox{green}{\color[gray]{0.75}FO}%
\colorbox{green}{\color[gray]{0.75}FO}%
\\
\colorbox{green}{\color[gray]{0.75}FO}%
\colorbox{green}{\color[gray]{0.75}FO}%
\colorbox{green}{\color[gray]{0.75}FO}%
\colorbox{green}{\color[gray]{0.75}FO}%
\colorbox{green}{\color[gray]{0.75}FO}%
\colorbox{green}{\color[gray]{0.75}FO}%
\colorbox{green}{\color[gray]{0.75}FO}%
\colorbox{green}{\color[gray]{0.75}FO}%
\colorbox{green}{\color[gray]{0.75}FO}%
\colorbox{green}{\color[gray]{0.75}FO}%
\colorbox{green}{\color[gray]{0.75}FO}%
\colorbox{green}{\color[gray]{0.75}FO}%
\colorbox{green}{\color[gray]{0.75}FO}%
\colorbox{green}{\color[gray]{0.75}FO}%
\colorbox{green}{\color[gray]{0.75}FO}%
\colorbox{green}{\color[gray]{0.75}FO}%
\colorbox{green}{\color[gray]{0.75}FO}%
\colorbox{green}{\color[gray]{0.75}FO}%
\colorbox{green}{\color[gray]{0.75}FO}%
\colorbox{green}{\color[gray]{0.75}FO}%
\colorbox{green}{\color[gray]{0.75}FO}%
\colorbox{green}{\color[gray]{0.75}FO}%
\colorbox{green}{\color[gray]{0.75}FO}%
\colorbox{green}{\color[gray]{0.75}FO}%
\colorbox{green}{\color[gray]{0.75}FO}%
\colorbox{green}{\color[gray]{0.75}FO}%
\colorbox{green}{\color[gray]{0.75}FO}%
\colorbox{green}{\color[gray]{0.75}FO}%
\colorbox{green}{\color[gray]{0.75}FO}%
\colorbox{green}{\color[gray]{0.75}FO}%
\colorbox{green}{\color[gray]{0.75}FO}%
\colorbox{green}{\color[gray]{0.75}FO}%
\colorbox{green}{\color[gray]{0.75}FO}%
\colorbox{green}{\color[gray]{0.75}FO}%
\colorbox{green}{\color[gray]{0.75}FO}%
\colorbox{green}{\color[gray]{0.75}FO}%
\colorbox{green}{\color[gray]{0.75}FO}%
\colorbox{green}{\color[gray]{0.75}FO}%
\colorbox{green}{\color[gray]{0.75}FO}%
\colorbox{green}{\color[gray]{0.75}FO}%
\colorbox{green}{\color[gray]{0.75}FO}%
\colorbox{green}{\color[gray]{0.75}FO}%
\colorbox{green}{\color[gray]{0.75}FO}%
\colorbox{green}{\color[gray]{0.75}FO}%
\colorbox{green}{\color[gray]{0.75}FO}%
\colorbox{green}{\color[gray]{0.75}FO}%
\colorbox{green}{\color[gray]{0.75}FO}%
\colorbox{green}{\color[gray]{0.75}FO}%
\colorbox{green}{\color[gray]{0.75}FO}%
\colorbox{green}{\color[gray]{0.75}FO}%
\colorbox{green}{\color[gray]{0.75}FO}%
\colorbox{green}{\color[gray]{0.75}FO}%
\colorbox{green}{\color[gray]{0.75}FO}%
\colorbox{green}{\color[gray]{0.75}FO}%
\colorbox{green}{\color[gray]{0.75}FO}%
\colorbox{green}{\color[gray]{0.75}FO}%
\colorbox{green}{\color[gray]{0.75}FO}%
\colorbox{green}{\color[gray]{0.75}FO}%
\colorbox{green}{\color[gray]{0.75}FO}%
\colorbox{green}{\color[gray]{0.75}FO}%
\colorbox{green}{\color[gray]{0.75}FO}%
\colorbox{green}{\color[gray]{0.75}FO}%
\colorbox{green}{\color[gray]{0.75}FO}%
\colorbox{green}{\color[gray]{0.75}FO}%
\colorbox{green}{\color[gray]{0.75}FO}%
\colorbox{green}{\color[gray]{0.75}FO}%
\colorbox{green}{\color[gray]{0.75}FO}%
\colorbox{green}{\color[gray]{0.75}FO}%
\colorbox{green}{\color[gray]{0.75}FO}%
\colorbox{green}{\color[gray]{0.75}FO}%
\colorbox{green}{\color[gray]{0.75}FO}%
\colorbox{green}{\color[gray]{0.75}FO}%
\colorbox{green}{\color[gray]{0.75}FO}%
\colorbox{green}{\color[gray]{0.75}FO}%
\colorbox{green}{\color[gray]{0.75}FO}%
\colorbox{green}{\color[gray]{0.75}FO}%
\colorbox{green}{\color[gray]{0.75}FO}%
\colorbox{green}{\color[gray]{0.75}FO}%
\colorbox{green}{\color[gray]{0.75}FO}%
\colorbox{green}{\color[gray]{0.75}FO}%
\colorbox{green}{\color[gray]{0.75}FO}%
\colorbox{green}{\color[gray]{0.75}FO}%
\colorbox{green}{\color[gray]{0.75}FO}%
\colorbox{green}{\color[gray]{0.75}FO}%
\colorbox{green}{\color[gray]{0.75}FO}%
\colorbox{green}{\color[gray]{0.75}FO}%
\colorbox{green}{\color[gray]{0.75}FO}%
\colorbox{green}{\color[gray]{0.75}FO}%
\colorbox{green}{\color[gray]{0.75}FO}%
\colorbox{green}{\color[gray]{0.75}FO}%
\colorbox{green}{\color[gray]{0.75}FO}%
\colorbox{green}{\color[gray]{0.75}FO}%
\colorbox{green}{\color[gray]{0.75}FO}%
\colorbox{green}{\color[gray]{0.75}FO}%
\colorbox{green}{\color[gray]{0.75}FO}%
\colorbox{green}{\color[gray]{0.75}FO}%
\colorbox{green}{\color[gray]{0.75}FO}%
\colorbox{green}{\color[gray]{0.75}FO}%
\colorbox{green}{\color[gray]{0.75}FO}%
\colorbox{green}{\color[gray]{0.75}FO}%
\\
\colorbox{green}{\color[gray]{0.75}FO}%
\colorbox{green}{\color[gray]{0.75}FO}%
\colorbox{green}{\color[gray]{0.75}FO}%
\colorbox{green}{\color[gray]{0.75}FO}%
\colorbox{green}{\color[gray]{0.75}FO}%
\colorbox{green}{\color[gray]{0.75}FO}%
\colorbox{green}{\color[gray]{0.75}FO}%
\colorbox{green}{\color[gray]{0.75}FO}%
\colorbox{green}{\color[gray]{0.75}FO}%
\colorbox{green}{\color[gray]{0.75}FO}%
\colorbox{green}{\color[gray]{0.75}FO}%
\colorbox{green}{\color[gray]{0.75}FO}%
\colorbox{green}{\color[gray]{0.75}FO}%
\colorbox{green}{\color[gray]{0.75}FO}%
\colorbox{green}{\color[gray]{0.75}FO}%
\colorbox{green}{\color[gray]{0.75}FO}%
\colorbox{green}{\color[gray]{0.75}FO}%
\colorbox{green}{\color[gray]{0.75}FO}%
\colorbox{green}{\color[gray]{0.75}FO}%
\colorbox{green}{\color[gray]{0.75}FO}%
\colorbox{green}{\color[gray]{0.75}FO}%
\colorbox{green}{\color[gray]{0.75}FO}%
\colorbox{green}{\color[gray]{0.75}FO}%
\colorbox{green}{\color[gray]{0.75}FO}%
\colorbox{green}{\color[gray]{0.75}FO}%
\colorbox{green}{\color[gray]{0.75}FO}%
\colorbox{green}{\color[gray]{0.75}FO}%
\colorbox{green}{\color[gray]{0.75}FO}%
\colorbox{green}{\color[gray]{0.75}FO}%
\colorbox{green}{\color[gray]{0.75}FO}%
\colorbox{green}{\color[gray]{0.75}FO}%
\colorbox{green}{\color[gray]{0.75}FO}%
\colorbox{green}{\color[gray]{0.75}FO}%
\colorbox{green}{\color[gray]{0.75}FO}%
\colorbox{green}{\color[gray]{0.75}FO}%
\colorbox{green}{\color[gray]{0.75}FO}%
\colorbox{green}{\color[gray]{0.75}FO}%
\colorbox{green}{\color[gray]{0.75}FO}%
\colorbox{green}{\color[gray]{0.75}FO}%
\colorbox{green}{\color[gray]{0.75}FO}%
\colorbox{green}{\color[gray]{0.75}FO}%
\colorbox{green}{\color[gray]{0.75}FO}%
\colorbox{green}{\color[gray]{0.75}FO}%
\colorbox{green}{\color[gray]{0.75}FO}%
\colorbox{green}{\color[gray]{0.75}FO}%
\colorbox{green}{\color[gray]{0.75}FO}%
\colorbox{green}{\color[gray]{0.75}FO}%
\colorbox{green}{\color[gray]{0.75}FO}%
\colorbox{green}{\color[gray]{0.75}FO}%
\colorbox{green}{\color[gray]{0.75}FO}%
\colorbox{green}{\color[gray]{0.75}FO}%
\colorbox{green}{\color[gray]{0.75}FO}%
\colorbox{green}{\color[gray]{0.75}FO}%
\colorbox{green}{\color[gray]{0.75}FO}%
\colorbox{green}{\color[gray]{0.75}FO}%
\colorbox{green}{\color[gray]{0.75}FO}%
\colorbox{green}{\color[gray]{0.75}FO}%
\colorbox{green}{\color[gray]{0.75}FO}%
\colorbox{green}{\color[gray]{0.75}FO}%
\colorbox{green}{\color[gray]{0.75}FO}%
\colorbox{green}{\color[gray]{0.75}FO}%
\colorbox{green}{\color[gray]{0.75}FO}%
\colorbox{green}{\color[gray]{0.75}FO}%
\colorbox{green}{\color[gray]{0.75}FO}%
\colorbox{green}{\color[gray]{0.75}FO}%
\colorbox{green}{\color[gray]{0.75}FO}%
\colorbox{green}{\color[gray]{0.75}FO}%
\colorbox{green}{\color[gray]{0.75}FO}%
\colorbox{green}{\color[gray]{0.75}FO}%
\colorbox{green}{\color[gray]{0.75}FO}%
\colorbox{green}{\color[gray]{0.75}FO}%
\colorbox{green}{\color[gray]{0.75}FO}%
\colorbox{green}{\color[gray]{0.75}FO}%
\colorbox{green}{\color[gray]{0.75}FO}%
\colorbox{green}{\color[gray]{0.75}FO}%
\colorbox{green}{\color[gray]{0.75}FO}%
\colorbox{green}{\color[gray]{0.75}FO}%
\colorbox{green}{\color[gray]{0.75}FO}%
\colorbox{green}{\color[gray]{0.75}FO}%
\colorbox{green}{\color[gray]{0.75}FO}%
\colorbox{green}{\color[gray]{0.75}FO}%
\colorbox{green}{\color[gray]{0.75}FO}%
\colorbox{green}{\color[gray]{0.75}FO}%
\colorbox{green}{\color[gray]{0.75}FO}%
\colorbox{green}{\color[gray]{0.75}FO}%
\colorbox{green}{\color[gray]{0.75}FO}%
\colorbox{green}{\color[gray]{0.75}FO}%
\colorbox{green}{\color[gray]{0.75}FO}%
\colorbox{green}{\color[gray]{0.75}FO}%
\colorbox{green}{\color[gray]{0.75}FO}%
\colorbox{green}{\color[gray]{0.75}FO}%
\colorbox{green}{\color[gray]{0.75}FO}%
\colorbox{green}{\color[gray]{0.75}FO}%
\colorbox{green}{\color[gray]{0.75}FO}%
\colorbox{green}{\color[gray]{0.75}FO}%
\colorbox{green}{\color[gray]{0.75}FO}%
\colorbox{green}{\color[gray]{0.75}FO}%
\colorbox{green}{\color[gray]{0.75}FO}%
\colorbox{green}{\color[gray]{0.75}FO}%
\colorbox{green}{\color[gray]{0.75}FO}%
\\
\colorbox{green}{\color[gray]{0.75}FO}%
\colorbox{green}{\color[gray]{0.75}FO}%
\colorbox{green}{\color[gray]{0.75}FO}%
\colorbox{green}{\color[gray]{0.75}FO}%
\colorbox{green}{\color[gray]{0.75}FO}%
\colorbox{green}{\color[gray]{0.75}FO}%
\colorbox{green}{\color[gray]{0.75}FO}%
\colorbox{green}{\color[gray]{0.75}FO}%
\colorbox{green}{\color[gray]{0.75}FO}%
\colorbox{green}{\color[gray]{0.75}FO}%
\colorbox{green}{\color[gray]{0.75}FO}%
\colorbox{green}{\color[gray]{0.75}FO}%
\colorbox{green}{\color[gray]{0.75}FO}%
\colorbox{green}{\color[gray]{0.75}FO}%
\colorbox{green}{\color[gray]{0.75}FO}%
\colorbox{green}{\color[gray]{0.75}FO}%
\colorbox{green}{\color[gray]{0.75}FO}%
\colorbox{green}{\color[gray]{0.75}FO}%
\colorbox{green}{\color[gray]{0.75}FO}%
\colorbox{green}{\color[gray]{0.75}FO}%
\colorbox{green}{\color[gray]{0.75}FO}%
\colorbox{green}{\color[gray]{0.75}FO}%
\colorbox{green}{\color[gray]{0.75}FO}%
\colorbox{green}{\color[gray]{0.75}FO}%
\colorbox{green}{\color[gray]{0.75}FO}%
\colorbox{green}{\color[gray]{0.75}FO}%
\colorbox{green}{\color[gray]{0.75}FO}%
\colorbox{green}{\color[gray]{0.75}FO}%
\colorbox{green}{\color[gray]{0.75}FO}%
\colorbox{green}{\color[gray]{0.75}FO}%
\colorbox{green}{\color[gray]{0.75}FO}%
\colorbox{green}{\color[gray]{0.75}FO}%
\colorbox{green}{\color[gray]{0.75}FO}%
\colorbox{green}{\color[gray]{0.75}FO}%
\colorbox{green}{\color[gray]{0.75}FO}%
\colorbox{green}{\color[gray]{0.75}FO}%
\colorbox{green}{\color[gray]{0.75}FO}%
\colorbox{green}{\color[gray]{0.75}FO}%
\colorbox{green}{\color[gray]{0.75}FO}%
\colorbox{green}{\color[gray]{0.75}FO}%
\colorbox{green}{\color[gray]{0.75}FO}%
\colorbox{green}{\color[gray]{0.75}FO}%
\colorbox{green}{\color[gray]{0.75}FO}%
\colorbox{green}{\color[gray]{0.75}FO}%
\colorbox{green}{\color[gray]{0.75}FO}%
\colorbox{green}{\color[gray]{0.75}FO}%
\colorbox{green}{\color[gray]{0.75}FO}%
\colorbox{green}{\color[gray]{0.75}FO}%
\colorbox{green}{\color[gray]{0.75}FO}%
\colorbox{green}{\color[gray]{0.75}FO}%
\colorbox{green}{\color[gray]{0.75}FO}%
\colorbox{green}{\color[gray]{0.75}FO}%
\colorbox{green}{\color[gray]{0.75}FO}%
\colorbox{green}{\color[gray]{0.75}FO}%
\colorbox{green}{\color[gray]{0.75}FO}%
\colorbox{green}{\color[gray]{0.75}FO}%
\colorbox{green}{\color[gray]{0.75}FO}%
\colorbox{green}{\color[gray]{0.75}FO}%
\colorbox{green}{\color[gray]{0.75}FO}%
\colorbox{green}{\color[gray]{0.75}FO}%
\colorbox{green}{\color[gray]{0.75}FO}%
\colorbox{green}{\color[gray]{0.75}FO}%
\colorbox{green}{\color[gray]{0.75}FO}%
\colorbox{green}{\color[gray]{0.75}FO}%
\colorbox{green}{\color[gray]{0.75}FO}%
\colorbox{green}{\color[gray]{0.75}FO}%
\colorbox{green}{\color[gray]{0.75}FO}%
\colorbox{green}{\color[gray]{0.75}FO}%
\colorbox{green}{\color[gray]{0.75}FO}%
\colorbox{green}{\color[gray]{0.75}FO}%
\colorbox{green}{\color[gray]{0.75}FO}%
\colorbox{green}{\color[gray]{0.75}FO}%
\colorbox{green}{\color[gray]{0.75}FO}%
\colorbox{green}{\color[gray]{0.75}FO}%
\colorbox{green}{\color[gray]{0.75}FO}%
\colorbox{green}{\color[gray]{0.75}FO}%
\colorbox{green}{\color[gray]{0.75}FO}%
\colorbox{green}{\color[gray]{0.75}FO}%
\colorbox{green}{\color[gray]{0.75}FO}%
\colorbox{green}{\color[gray]{0.75}FO}%
\colorbox{green}{\color[gray]{0.75}FO}%
\colorbox{green}{\color[gray]{0.75}FO}%
\colorbox{green}{\color[gray]{0.75}FO}%
\colorbox{green}{\color[gray]{0.75}FO}%
\colorbox{green}{\color[gray]{0.75}FO}%
\colorbox{green}{\color[gray]{0.75}FO}%
\colorbox{green}{\color[gray]{0.75}FO}%
\colorbox{green}{\color[gray]{0.75}FO}%
\colorbox{green}{\color[gray]{0.75}FO}%
\colorbox{green}{\color[gray]{0.75}FO}%
\colorbox{green}{\color[gray]{0.75}FO}%
\colorbox{green}{\color[gray]{0.75}FO}%
\colorbox{green}{\color[gray]{0.75}FO}%
\colorbox{green}{\color[gray]{0.75}FO}%
\colorbox{green}{\color[gray]{0.75}FO}%
\colorbox{green}{\color[gray]{0.75}FO}%
\colorbox{green}{\color[gray]{0.75}FO}%
\colorbox{green}{\color[gray]{0.75}FO}%
\colorbox{green}{\color[gray]{0.75}FO}%
\colorbox{green}{\color[gray]{0.75}FO}%
\\
\colorbox{green}{\color[gray]{0.75}FO}%
\colorbox{green}{\color[gray]{0.75}FO}%
\colorbox{green}{\color[gray]{0.75}FO}%
\colorbox{green}{\color[gray]{0.75}FO}%
\colorbox{green}{\color[gray]{0.75}FO}%
\colorbox{green}{\color[gray]{0.75}FO}%
\colorbox{green}{\color[gray]{0.75}FO}%
\colorbox{green}{\color[gray]{0.75}FO}%
\colorbox{green}{\color[gray]{0.75}FO}%
\colorbox{green}{\color[gray]{0.75}FO}%
\colorbox{green}{\color[gray]{0.75}FO}%
\colorbox{green}{\color[gray]{0.75}FO}%
\colorbox{green}{\color[gray]{0.75}FO}%
\colorbox{green}{\color[gray]{0.75}FO}%
\colorbox{green}{\color[gray]{0.75}FO}%
\colorbox{green}{\color[gray]{0.75}FO}%
\colorbox{green}{\color[gray]{0.75}FO}%
\colorbox{green}{\color[gray]{0.75}FO}%
\colorbox{green}{\color[gray]{0.75}FO}%
\colorbox{green}{\color[gray]{0.75}FO}%
\colorbox{green}{\color[gray]{0.75}FO}%
\colorbox{green}{\color[gray]{0.75}FO}%
\colorbox{green}{\color[gray]{0.75}FO}%
\colorbox{green}{\color[gray]{0.75}FO}%
\colorbox{green}{\color[gray]{0.75}FO}%
\colorbox{green}{\color[gray]{0.75}FO}%
\colorbox{green}{\color[gray]{0.75}FO}%
\colorbox{green}{\color[gray]{0.75}FO}%
\colorbox{green}{\color[gray]{0.75}FO}%
\colorbox{green}{\color[gray]{0.75}FO}%
\colorbox{green}{\color[gray]{0.75}FO}%
\colorbox{green}{\color[gray]{0.75}FO}%
\colorbox{green}{\color[gray]{0.75}FO}%
\colorbox{green}{\color[gray]{0.75}FO}%
\colorbox{green}{\color[gray]{0.75}FO}%
\colorbox{green}{\color[gray]{0.75}FO}%
\colorbox{green}{\color[gray]{0.75}FO}%
\colorbox{green}{\color[gray]{0.75}FO}%
\colorbox{green}{\color[gray]{0.75}FO}%
\colorbox{green}{\color[gray]{0.75}FO}%
\colorbox{green}{\color[gray]{0.75}FO}%
\colorbox{green}{\color[gray]{0.75}FO}%
\colorbox{green}{\color[gray]{0.75}FO}%
\colorbox{green}{\color[gray]{0.75}FO}%
\colorbox{green}{\color[gray]{0.75}FO}%
\colorbox{green}{\color[gray]{0.75}FO}%
\colorbox{green}{\color[gray]{0.75}FO}%
\colorbox{green}{\color[gray]{0.75}FO}%
\colorbox{green}{\color[gray]{0.75}FO}%
\colorbox{green}{\color[gray]{0.75}FO}%
\colorbox{green}{\color[gray]{0.75}FO}%
\colorbox{green}{\color[gray]{0.75}FO}%
\colorbox{green}{\color[gray]{0.75}FO}%
\colorbox{green}{\color[gray]{0.75}FO}%
\colorbox{green}{\color[gray]{0.75}FO}%
\colorbox{green}{\color[gray]{0.75}FO}%
\colorbox{green}{\color[gray]{0.75}FO}%
\colorbox{green}{\color[gray]{0.75}FO}%
\colorbox{green}{\color[gray]{0.75}FO}%
\colorbox{green}{\color[gray]{0.75}FO}%
\colorbox{green}{\color[gray]{0.75}FO}%
\colorbox{green}{\color[gray]{0.75}FO}%
\colorbox{green}{\color[gray]{0.75}FO}%
\colorbox{green}{\color[gray]{0.75}FO}%
\colorbox{green}{\color[gray]{0.75}FO}%
\colorbox{green}{\color[gray]{0.75}FO}%
\colorbox{green}{\color[gray]{0.75}FO}%
\colorbox{green}{\color[gray]{0.75}FO}%
\colorbox{green}{\color[gray]{0.75}FO}%
\colorbox{green}{\color[gray]{0.75}FO}%
\colorbox{green}{\color[gray]{0.75}FO}%
\colorbox{green}{\color[gray]{0.75}FO}%
\colorbox{green}{\color[gray]{0.75}FO}%
\colorbox{green}{\color[gray]{0.75}FO}%
\colorbox{green}{\color[gray]{0.75}FO}%
\colorbox{green}{\color[gray]{0.75}FO}%
\colorbox{green}{\color[gray]{0.75}FO}%
\colorbox{green}{\color[gray]{0.75}FO}%
\colorbox{green}{\color[gray]{0.75}FO}%
\colorbox{green}{\color[gray]{0.75}FO}%
\colorbox{green}{\color[gray]{0.75}FO}%
\colorbox{green}{\color[gray]{0.75}FO}%
\colorbox{green}{\color[gray]{0.75}FO}%
\colorbox{green}{\color[gray]{0.75}FO}%
\colorbox{green}{\color[gray]{0.75}FO}%
\colorbox{green}{\color[gray]{0.75}FO}%
\colorbox{green}{\color[gray]{0.75}FO}%
\colorbox{green}{\color[gray]{0.75}FO}%
\colorbox{green}{\color[gray]{0.75}FO}%
\colorbox{green}{\color[gray]{0.75}FO}%
\colorbox{green}{\color[gray]{0.75}FO}%
\colorbox{green}{\color[gray]{0.75}FO}%
\colorbox{green}{\color[gray]{0.75}FO}%
\colorbox{green}{\color[gray]{0.75}FO}%
\colorbox{green}{\color[gray]{0.75}FO}%
\colorbox{green}{\color[gray]{0.75}FO}%
\colorbox{green}{\color[gray]{0.75}FO}%
\colorbox{green}{\color[gray]{0.75}FO}%
\colorbox{green}{\color[gray]{0.75}FO}%
\colorbox{green}{\color[gray]{0.75}FO}%
\\
\colorbox{green}{\color[gray]{0.75}FO}%
\colorbox{green}{\color[gray]{0.75}FO}%
\colorbox{green}{\color[gray]{0.75}FO}%
\colorbox{green}{\color[gray]{0.75}FO}%
\colorbox{green}{\color[gray]{0.75}FO}%
\colorbox{green}{\color[gray]{0.75}FO}%
\colorbox{green}{\color[gray]{0.75}FO}%
\colorbox{green}{\color[gray]{0.75}FO}%
\colorbox{green}{\color[gray]{0.75}FO}%
\colorbox{green}{\color[gray]{0.75}FO}%
\colorbox{green}{\color[gray]{0.75}FO}%
\colorbox{green}{\color[gray]{0.75}FO}%
\colorbox{green}{\color[gray]{0.75}FO}%
\colorbox{green}{\color[gray]{0.75}FO}%
\colorbox{green}{\color[gray]{0.75}FO}%
\colorbox{green}{\color[gray]{0.75}FO}%
\colorbox{green}{\color[gray]{0.75}FO}%
\colorbox{green}{\color[gray]{0.75}FO}%
\colorbox{green}{\color[gray]{0.75}FO}%
\colorbox{green}{\color[gray]{0.75}FO}%
\colorbox{green}{\color[gray]{0.75}FO}%
\colorbox{green}{\color[gray]{0.75}FO}%
\colorbox{green}{\color[gray]{0.75}FO}%
\colorbox{green}{\color[gray]{0.75}FO}%
\colorbox{green}{\color[gray]{0.75}FO}%
\colorbox{green}{\color[gray]{0.75}FO}%
\colorbox{green}{\color[gray]{0.75}FO}%
\colorbox{green}{\color[gray]{0.75}FO}%
\colorbox{green}{\color[gray]{0.75}FO}%
\colorbox{green}{\color[gray]{0.75}FO}%
\colorbox{green}{\color[gray]{0.75}FO}%
\colorbox{green}{\color[gray]{0.75}FO}%
\colorbox{green}{\color[gray]{0.75}FO}%
\colorbox{green}{\color[gray]{0.75}FO}%
\colorbox{green}{\color[gray]{0.75}FO}%
\colorbox{green}{\color[gray]{0.75}FO}%
\colorbox{green}{\color[gray]{0.75}FO}%
\colorbox{green}{\color[gray]{0.75}FO}%
\colorbox{green}{\color[gray]{0.75}FO}%
\colorbox{green}{\color[gray]{0.75}FO}%
\colorbox{green}{\color[gray]{0.75}FO}%
\colorbox{green}{\color[gray]{0.75}FO}%
\colorbox{green}{\color[gray]{0.75}FO}%
\colorbox{green}{\color[gray]{0.75}FO}%
\colorbox{green}{\color[gray]{0.75}FO}%
\colorbox{green}{\color[gray]{0.75}FO}%
\colorbox{green}{\color[gray]{0.75}FO}%
\colorbox{green}{\color[gray]{0.75}FO}%
\colorbox{green}{\color[gray]{0.75}FO}%
\colorbox{green}{\color[gray]{0.75}FO}%
\colorbox{green}{\color[gray]{0.75}FO}%
\colorbox{green}{\color[gray]{0.75}FO}%
\colorbox{green}{\color[gray]{0.75}FO}%
\colorbox{green}{\color[gray]{0.75}FO}%
\colorbox{green}{\color[gray]{0.75}FO}%
\colorbox{green}{\color[gray]{0.75}FO}%
\colorbox{green}{\color[gray]{0.75}FO}%
\colorbox{green}{\color[gray]{0.75}FO}%
\colorbox{green}{\color[gray]{0.75}FO}%
\colorbox{green}{\color[gray]{0.75}FO}%
\colorbox{green}{\color[gray]{0.75}FO}%
\colorbox{green}{\color[gray]{0.75}FO}%
\colorbox{green}{\color[gray]{0.75}FO}%
\colorbox{green}{\color[gray]{0.75}FO}%
\colorbox{green}{\color[gray]{0.75}FO}%
\colorbox{green}{\color[gray]{0.75}FO}%
\colorbox{green}{\color[gray]{0.75}FO}%
\colorbox{green}{\color[gray]{0.75}FO}%
\colorbox{green}{\color[gray]{0.75}FO}%
\colorbox{green}{\color[gray]{0.75}FO}%
\colorbox{green}{\color[gray]{0.75}FO}%
\colorbox{green}{\color[gray]{0.75}FO}%
\colorbox{green}{\color[gray]{0.75}FO}%
\colorbox{green}{\color[gray]{0.75}FO}%
\colorbox{green}{\color[gray]{0.75}FO}%
\colorbox{green}{\color[gray]{0.75}FO}%
\colorbox{green}{\color[gray]{0.75}FO}%
\colorbox{green}{\color[gray]{0.75}FO}%
\colorbox{green}{\color[gray]{0.75}FO}%
\colorbox{green}{\color[gray]{0.75}FO}%
\colorbox{green}{\color[gray]{0.75}FO}%
\colorbox{green}{\color[gray]{0.75}FO}%
\colorbox{green}{\color[gray]{0.75}FO}%
\colorbox{green}{\color[gray]{0.75}FO}%
\colorbox{green}{\color[gray]{0.75}FO}%
\colorbox{green}{\color[gray]{0.75}FO}%
\colorbox{green}{\color[gray]{0.75}FO}%
\colorbox{green}{\color[gray]{0.75}FO}%
\colorbox{green}{\color[gray]{0.75}FO}%
\colorbox{green}{\color[gray]{0.75}FO}%
\colorbox{green}{\color[gray]{0.75}FO}%
\colorbox{green}{\color[gray]{0.75}FO}%
\colorbox{green}{\color[gray]{0.75}FO}%
\colorbox{green}{\color[gray]{0.75}FO}%
\colorbox{green}{\color[gray]{0.75}FO}%
\colorbox{green}{\color[gray]{0.75}FO}%
\colorbox{green}{\color[gray]{0.75}FO}%
\colorbox{green}{\color[gray]{0.75}FO}%
\colorbox{green}{\color[gray]{0.75}FO}%
\colorbox{green}{\color[gray]{0.75}FO}%
\\
\colorbox{green}{\color[gray]{0.75}FO}%
\colorbox{green}{\color[gray]{0.75}FO}%
\colorbox{green}{\color[gray]{0.75}FO}%
\colorbox{green}{\color[gray]{0.75}FO}%
\colorbox{green}{\color[gray]{0.75}FO}%
\colorbox{green}{\color[gray]{0.75}FO}%
\colorbox{green}{\color[gray]{0.75}FO}%
\colorbox{green}{\color[gray]{0.75}FO}%
\colorbox{green}{\color[gray]{0.75}FO}%
\colorbox{green}{\color[gray]{0.75}FO}%
\colorbox{green}{\color[gray]{0.75}FO}%
\colorbox{green}{\color[gray]{0.75}FO}%
\colorbox{green}{\color[gray]{0.75}FO}%
\colorbox{green}{\color[gray]{0.75}FO}%
\colorbox{green}{\color[gray]{0.75}FO}%
\colorbox{green}{\color[gray]{0.75}FO}%
\colorbox{green}{\color[gray]{0.75}FO}%
\colorbox{green}{\color[gray]{0.75}FO}%
\colorbox{green}{\color[gray]{0.75}FO}%
\colorbox{green}{\color[gray]{0.75}FO}%
\colorbox{green}{\color[gray]{0.75}FO}%
\colorbox{green}{\color[gray]{0.75}FO}%
\colorbox{green}{\color[gray]{0.75}FO}%
\colorbox{green}{\color[gray]{0.75}FO}%
\colorbox{green}{\color[gray]{0.75}FO}%
\colorbox{green}{\color[gray]{0.75}FO}%
\colorbox{green}{\color[gray]{0.75}FO}%
\colorbox{green}{\color[gray]{0.75}FO}%
\colorbox{green}{\color[gray]{0.75}FO}%
\colorbox{green}{\color[gray]{0.75}FO}%
\colorbox{green}{\color[gray]{0.75}FO}%
\colorbox{green}{\color[gray]{0.75}FO}%
\colorbox{green}{\color[gray]{0.75}FO}%
\colorbox{green}{\color[gray]{0.75}FO}%
\colorbox{green}{\color[gray]{0.75}FO}%
\colorbox{green}{\color[gray]{0.75}FO}%
\colorbox{green}{\color[gray]{0.75}FO}%
\colorbox{green}{\color[gray]{0.75}FO}%
\colorbox{green}{\color[gray]{0.75}FO}%
\colorbox{green}{\color[gray]{0.75}FO}%
\colorbox{green}{\color[gray]{0.75}FO}%
\colorbox{green}{\color[gray]{0.75}FO}%
\colorbox{green}{\color[gray]{0.75}FO}%
\colorbox{green}{\color[gray]{0.75}FO}%
\colorbox{green}{\color[gray]{0.75}FO}%
\colorbox{green}{\color[gray]{0.75}FO}%
\colorbox{green}{\color[gray]{0.75}FO}%
\colorbox{green}{\color[gray]{0.75}FO}%
\colorbox{green}{\color[gray]{0.75}FO}%
\colorbox{green}{\color[gray]{0.75}FO}%
\colorbox{green}{\color[gray]{0.75}FO}%
\colorbox{green}{\color[gray]{0.75}FO}%
\colorbox{green}{\color[gray]{0.75}FO}%
\colorbox{green}{\color[gray]{0.75}FO}%
\colorbox{green}{\color[gray]{0.75}FO}%
\colorbox{green}{\color[gray]{0.75}FO}%
\colorbox{green}{\color[gray]{0.75}FO}%
\colorbox{green}{\color[gray]{0.75}FO}%
\colorbox{green}{\color[gray]{0.75}FO}%
\colorbox{green}{\color[gray]{0.75}FO}%
\colorbox{green}{\color[gray]{0.75}FO}%
\colorbox{green}{\color[gray]{0.75}FO}%
\colorbox{green}{\color[gray]{0.75}FO}%
\colorbox{green}{\color[gray]{0.75}FO}%
\colorbox{green}{\color[gray]{0.75}FO}%
\colorbox{green}{\color[gray]{0.75}FO}%
\colorbox{green}{\color[gray]{0.75}FO}%
\colorbox{green}{\color[gray]{0.75}FO}%
\colorbox{green}{\color[gray]{0.75}FO}%
\colorbox{green}{\color[gray]{0.75}FO}%
\colorbox{green}{\color[gray]{0.75}FO}%
\colorbox{green}{\color[gray]{0.75}FO}%
\colorbox{green}{\color[gray]{0.75}FO}%
\colorbox{green}{\color[gray]{0.75}FO}%
\colorbox{green}{\color[gray]{0.75}FO}%
\colorbox{green}{\color[gray]{0.75}FO}%
\colorbox{green}{\color[gray]{0.75}FO}%
\colorbox{green}{\color[gray]{0.75}FO}%
\colorbox{green}{\color[gray]{0.75}FO}%
\colorbox{green}{\color[gray]{0.75}FO}%
\colorbox{green}{\color[gray]{0.75}FO}%
\colorbox{green}{\color[gray]{0.75}FO}%
\colorbox{green}{\color[gray]{0.75}FO}%
\colorbox{green}{\color[gray]{0.75}FO}%
\colorbox{green}{\color[gray]{0.75}FO}%
\colorbox{green}{\color[gray]{0.75}FO}%
\colorbox{green}{\color[gray]{0.75}FO}%
\colorbox{green}{\color[gray]{0.75}FO}%
\colorbox{green}{\color[gray]{0.75}FO}%
\colorbox{green}{\color[gray]{0.75}FO}%
\colorbox{green}{\color[gray]{0.75}FO}%
\colorbox{green}{\color[gray]{0.75}FO}%
\colorbox{green}{\color[gray]{0.75}FO}%
\colorbox{green}{\color[gray]{0.75}FO}%
\colorbox{green}{\color[gray]{0.75}FO}%
\colorbox{green}{\color[gray]{0.75}FO}%
\colorbox{green}{\color[gray]{0.75}FO}%
\colorbox{green}{\color[gray]{0.75}FO}%
\colorbox{green}{\color[gray]{0.75}FO}%
\colorbox{green}{\color[gray]{0.75}FO}%
\\
\colorbox{green}{\color[gray]{0.75}FO}%
\colorbox{green}{\color[gray]{0.75}FO}%
\colorbox{green}{\color[gray]{0.75}FO}%
\colorbox{green}{\color[gray]{0.75}FO}%
\colorbox{green}{\color[gray]{0.75}FO}%
\colorbox{green}{\color[gray]{0.75}FO}%
\colorbox{green}{\color[gray]{0.75}FO}%
\colorbox{green}{\color[gray]{0.75}FO}%
\colorbox{green}{\color[gray]{0.75}FO}%
\colorbox{green}{\color[gray]{0.75}FO}%
\colorbox{green}{\color[gray]{0.75}FO}%
\colorbox{green}{\color[gray]{0.75}FO}%
\colorbox{green}{\color[gray]{0.75}FO}%
\colorbox{green}{\color[gray]{0.75}FO}%
\colorbox{green}{\color[gray]{0.75}FO}%
\colorbox{green}{\color[gray]{0.75}FO}%
\colorbox{green}{\color[gray]{0.75}FO}%
\colorbox{green}{\color[gray]{0.75}FO}%
\colorbox{green}{\color[gray]{0.75}FO}%
\colorbox{green}{\color[gray]{0.75}FO}%
\colorbox{green}{\color[gray]{0.75}FO}%
\colorbox{green}{\color[gray]{0.75}FO}%
\colorbox{green}{\color[gray]{0.75}FO}%
\colorbox{green}{\color[gray]{0.75}FO}%
\colorbox{green}{\color[gray]{0.75}FO}%
\colorbox{green}{\color[gray]{0.75}FO}%
\colorbox{green}{\color[gray]{0.75}FO}%
\colorbox{green}{\color[gray]{0.75}FO}%
\colorbox{green}{\color[gray]{0.75}FO}%
\colorbox{green}{\color[gray]{0.75}FO}%
\colorbox{green}{\color[gray]{0.75}FO}%
\colorbox{green}{\color[gray]{0.75}FO}%
\colorbox{green}{\color[gray]{0.75}FO}%
\colorbox{green}{\color[gray]{0.75}FO}%
\colorbox{green}{\color[gray]{0.75}FO}%
\colorbox{green}{\color[gray]{0.75}FO}%
\colorbox{green}{\color[gray]{0.75}FO}%
\colorbox{green}{\color[gray]{0.75}FO}%
\colorbox{green}{\color[gray]{0.75}FO}%
\colorbox{green}{\color[gray]{0.75}FO}%
\colorbox{green}{\color[gray]{0.75}FO}%
\colorbox{green}{\color[gray]{0.75}FO}%
\colorbox{green}{\color[gray]{0.75}FO}%
\colorbox{green}{\color[gray]{0.75}FO}%
\colorbox{green}{\color[gray]{0.75}FO}%
\colorbox{green}{\color[gray]{0.75}FO}%
\colorbox{green}{\color[gray]{0.75}FO}%
\colorbox{green}{\color[gray]{0.75}FO}%
\colorbox{green}{\color[gray]{0.75}FO}%
\colorbox{green}{\color[gray]{0.75}FO}%
\colorbox{green}{\color[gray]{0.75}FO}%
\colorbox{green}{\color[gray]{0.75}FO}%
\colorbox{green}{\color[gray]{0.75}FO}%
\colorbox{green}{\color[gray]{0.75}FO}%
\colorbox{green}{\color[gray]{0.75}FO}%
\colorbox{green}{\color[gray]{0.75}FO}%
\colorbox{green}{\color[gray]{0.75}FO}%
\colorbox{green}{\color[gray]{0.75}FO}%
\colorbox{green}{\color[gray]{0.75}FO}%
\colorbox{green}{\color[gray]{0.75}FO}%
\colorbox{green}{\color[gray]{0.75}FO}%
\colorbox{green}{\color[gray]{0.75}FO}%
\colorbox{green}{\color[gray]{0.75}FO}%
\colorbox{green}{\color[gray]{0.75}FO}%
\colorbox{green}{\color[gray]{0.75}FO}%
\colorbox{green}{\color[gray]{0.75}FO}%
\colorbox{green}{\color[gray]{0.75}FO}%
\colorbox{green}{\color[gray]{0.75}FO}%
\colorbox{green}{\color[gray]{0.75}FO}%
\colorbox{green}{\color[gray]{0.75}FO}%
\colorbox{green}{\color[gray]{0.75}FO}%
\colorbox{green}{\color[gray]{0.75}FO}%
\colorbox{green}{\color[gray]{0.75}FO}%
\colorbox{green}{\color[gray]{0.75}FO}%
\colorbox{green}{\color[gray]{0.75}FO}%
\colorbox{green}{\color[gray]{0.75}FO}%
\colorbox{green}{\color[gray]{0.75}FO}%
\colorbox{green}{\color[gray]{0.75}FO}%
\colorbox{green}{\color[gray]{0.75}FO}%
\colorbox{green}{\color[gray]{0.75}FO}%
\colorbox{green}{\color[gray]{0.75}FO}%
\colorbox{green}{\color[gray]{0.75}FO}%
\colorbox{green}{\color[gray]{0.75}FO}%
\colorbox{green}{\color[gray]{0.75}FO}%
\colorbox{green}{\color[gray]{0.75}FO}%
\colorbox{green}{\color[gray]{0.75}FO}%
\colorbox{green}{\color[gray]{0.75}FO}%
\colorbox{green}{\color[gray]{0.75}FO}%
\colorbox{green}{\color[gray]{0.75}FO}%
\colorbox{green}{\color[gray]{0.75}FO}%
\colorbox{green}{\color[gray]{0.75}FO}%
\colorbox{green}{\color[gray]{0.75}FO}%
\colorbox{green}{\color[gray]{0.75}FO}%
\colorbox{green}{\color[gray]{0.75}FO}%
\colorbox{green}{\color[gray]{0.75}FO}%
\colorbox{green}{\color[gray]{0.75}FO}%
\colorbox{green}{\color[gray]{0.75}FO}%
\colorbox{green}{\color[gray]{0.75}FO}%
\colorbox{green}{\color[gray]{0.75}FO}%
\colorbox{green}{\color[gray]{0.75}FO}%
\\
\colorbox{green}{\color[gray]{0.75}FO}%
\colorbox{green}{\color[gray]{0.75}FO}%
\colorbox{green}{\color[gray]{0.75}FO}%
\colorbox{green}{\color[gray]{0.75}FO}%
\colorbox{green}{\color[gray]{0.75}FO}%
\colorbox{green}{\color[gray]{0.75}FO}%
\colorbox{green}{\color[gray]{0.75}FO}%
\colorbox{green}{\color[gray]{0.75}FO}%
\colorbox{green}{\color[gray]{0.75}FO}%
\colorbox{green}{\color[gray]{0.75}FO}%
\colorbox{green}{\color[gray]{0.75}FO}%
\colorbox{green}{\color[gray]{0.75}FO}%
\colorbox{green}{\color[gray]{0.75}FO}%
\colorbox{green}{\color[gray]{0.75}FO}%
\colorbox{green}{\color[gray]{0.75}FO}%
\colorbox{green}{\color[gray]{0.75}FO}%
\colorbox{green}{\color[gray]{0.75}FO}%
\colorbox{green}{\color[gray]{0.75}FO}%
\colorbox{green}{\color[gray]{0.75}FO}%
\colorbox{green}{\color[gray]{0.75}FO}%
\colorbox{green}{\color[gray]{0.75}FO}%
\colorbox{green}{\color[gray]{0.75}FO}%
\colorbox{green}{\color[gray]{0.75}FO}%
\colorbox{green}{\color[gray]{0.75}FO}%
\colorbox{green}{\color[gray]{0.75}FO}%
\colorbox{green}{\color[gray]{0.75}FO}%
\colorbox{green}{\color[gray]{0.75}FO}%
\colorbox{green}{\color[gray]{0.75}FO}%
\colorbox{green}{\color[gray]{0.75}FO}%
\colorbox{green}{\color[gray]{0.75}FO}%
\colorbox{green}{\color[gray]{0.75}FO}%
\colorbox{green}{\color[gray]{0.75}FO}%
\colorbox{green}{\color[gray]{0.75}FO}%
\colorbox{green}{\color[gray]{0.75}FO}%
\colorbox{green}{\color[gray]{0.75}FO}%
\colorbox{green}{\color[gray]{0.75}FO}%
\colorbox{green}{\color[gray]{0.75}FO}%
\colorbox{green}{\color[gray]{0.75}FO}%
\colorbox{green}{\color[gray]{0.75}FO}%
\colorbox{green}{\color[gray]{0.75}FO}%
\colorbox{green}{\color[gray]{0.75}FO}%
\colorbox{green}{\color[gray]{0.75}FO}%
\colorbox{green}{\color[gray]{0.75}FO}%
\colorbox{green}{\color[gray]{0.75}FO}%
\colorbox{green}{\color[gray]{0.75}FO}%
\colorbox{green}{\color[gray]{0.75}FO}%
\colorbox{green}{\color[gray]{0.75}FO}%
\colorbox{green}{\color[gray]{0.75}FO}%
\colorbox{green}{\color[gray]{0.75}FO}%
\colorbox{green}{\color[gray]{0.75}FO}%
\colorbox{green}{\color[gray]{0.75}FO}%
\colorbox{green}{\color[gray]{0.75}FO}%
\colorbox{green}{\color[gray]{0.75}FO}%
\colorbox{green}{\color[gray]{0.75}FO}%
\colorbox{green}{\color[gray]{0.75}FO}%
\colorbox{green}{\color[gray]{0.75}FO}%
\colorbox{green}{\color[gray]{0.75}FO}%
\colorbox{green}{\color[gray]{0.75}FO}%
\colorbox{green}{\color[gray]{0.75}FO}%
\colorbox{green}{\color[gray]{0.75}FO}%
\colorbox{green}{\color[gray]{0.75}FO}%
\colorbox{green}{\color[gray]{0.75}FO}%
\colorbox{green}{\color[gray]{0.75}FO}%
\colorbox{green}{\color[gray]{0.75}FO}%
\colorbox{green}{\color[gray]{0.75}FO}%
\colorbox{green}{\color[gray]{0.75}FO}%
\colorbox{green}{\color[gray]{0.75}FO}%
\colorbox{green}{\color[gray]{0.75}FO}%
\colorbox{green}{\color[gray]{0.75}FO}%
\colorbox{green}{\color[gray]{0.75}FO}%
\colorbox{green}{\color[gray]{0.75}FO}%
\colorbox{green}{\color[gray]{0.75}FO}%
\colorbox{green}{\color[gray]{0.75}FO}%
\colorbox{green}{\color[gray]{0.75}FO}%
\colorbox{green}{\color[gray]{0.75}FO}%
\colorbox{green}{\color[gray]{0.75}FO}%
\colorbox{green}{\color[gray]{0.75}FO}%
\colorbox{green}{\color[gray]{0.75}FO}%
\colorbox{green}{\color[gray]{0.75}FO}%
\colorbox{green}{\color[gray]{0.75}FO}%
\colorbox{green}{\color[gray]{0.75}FO}%
\colorbox{green}{\color[gray]{0.75}FO}%
\colorbox{green}{\color[gray]{0.75}FO}%
\colorbox{green}{\color[gray]{0.75}FO}%
\colorbox{green}{\color[gray]{0.75}FO}%
\colorbox{green}{\color[gray]{0.75}FO}%
\colorbox{green}{\color[gray]{0.75}FO}%
\colorbox{green}{\color[gray]{0.75}FO}%
\colorbox{green}{\color[gray]{0.75}FO}%
\colorbox{green}{\color[gray]{0.75}FO}%
\colorbox{green}{\color[gray]{0.75}FO}%
\colorbox{green}{\color[gray]{0.75}FO}%
\colorbox{green}{\color[gray]{0.75}FO}%
\colorbox{green}{\color[gray]{0.75}FO}%
\colorbox{green}{\color[gray]{0.75}FO}%
\colorbox{green}{\color[gray]{0.75}FO}%
\colorbox{green}{\color[gray]{0.75}FO}%
\colorbox{green}{\color[gray]{0.75}FO}%
\colorbox{green}{\color[gray]{0.75}FO}%
\colorbox{green}{\color[gray]{0.75}FO}%
\\
\colorbox{green}{\color[gray]{0.75}FO}%
\colorbox{green}{\color[gray]{0.75}FO}%
\colorbox{green}{\color[gray]{0.75}FO}%
\colorbox{green}{\color[gray]{0.75}FO}%
\colorbox{green}{\color[gray]{0.75}FO}%
\colorbox{green}{\color[gray]{0.75}FO}%
\colorbox{green}{\color[gray]{0.75}FO}%
\colorbox{green}{\color[gray]{0.75}FO}%
\colorbox{green}{\color[gray]{0.75}FO}%
\colorbox{green}{\color[gray]{0.75}FO}%
\colorbox{green}{\color[gray]{0.75}FO}%
\colorbox{green}{\color[gray]{0.75}FO}%
\colorbox{green}{\color[gray]{0.75}FO}%
\colorbox{green}{\color[gray]{0.75}FO}%
\colorbox{green}{\color[gray]{0.75}FO}%
\colorbox{green}{\color[gray]{0.75}FO}%
\colorbox{green}{\color[gray]{0.75}FO}%
\colorbox{green}{\color[gray]{0.75}FO}%
\colorbox{green}{\color[gray]{0.75}FO}%
\colorbox{green}{\color[gray]{0.75}FO}%
\colorbox{green}{\color[gray]{0.75}FO}%
\colorbox{green}{\color[gray]{0.75}FO}%
\colorbox{green}{\color[gray]{0.75}FO}%
\colorbox{green}{\color[gray]{0.75}FO}%
\colorbox{green}{\color[gray]{0.75}FO}%
\colorbox{green}{\color[gray]{0.75}FO}%
\colorbox{green}{\color[gray]{0.75}FO}%
\colorbox{green}{\color[gray]{0.75}FO}%
\colorbox{green}{\color[gray]{0.75}FO}%
\colorbox{green}{\color[gray]{0.75}FO}%
\colorbox{green}{\color[gray]{0.75}FO}%
\colorbox{green}{\color[gray]{0.75}FO}%
\colorbox{green}{\color[gray]{0.75}FO}%
\colorbox{green}{\color[gray]{0.75}FO}%
\colorbox{green}{\color[gray]{0.75}FO}%
\colorbox{green}{\color[gray]{0.75}FO}%
\colorbox{green}{\color[gray]{0.75}FO}%
\colorbox{green}{\color[gray]{0.75}FO}%
\colorbox{green}{\color[gray]{0.75}FO}%
\colorbox{green}{\color[gray]{0.75}FO}%
\colorbox{green}{\color[gray]{0.75}FO}%
\colorbox{green}{\color[gray]{0.75}FO}%
\colorbox{green}{\color[gray]{0.75}FO}%
\colorbox{green}{\color[gray]{0.75}FO}%
\colorbox{green}{\color[gray]{0.75}FO}%
\colorbox{green}{\color[gray]{0.75}FO}%
\colorbox{green}{\color[gray]{0.75}FO}%
\colorbox{green}{\color[gray]{0.75}FO}%
\colorbox{green}{\color[gray]{0.75}FO}%
\colorbox{green}{\color[gray]{0.75}FO}%
\colorbox{green}{\color[gray]{0.75}FO}%
\colorbox{green}{\color[gray]{0.75}FO}%
\colorbox{green}{\color[gray]{0.75}FO}%
\colorbox{green}{\color[gray]{0.75}FO}%
\colorbox{green}{\color[gray]{0.75}FO}%
\colorbox{green}{\color[gray]{0.75}FO}%
\colorbox{green}{\color[gray]{0.75}FO}%
\colorbox{green}{\color[gray]{0.75}FO}%
\colorbox{green}{\color[gray]{0.75}FO}%
\colorbox{green}{\color[gray]{0.75}FO}%
\colorbox{green}{\color[gray]{0.75}FO}%
\colorbox{green}{\color[gray]{0.75}FO}%
\colorbox{green}{\color[gray]{0.75}FO}%
\colorbox{green}{\color[gray]{0.75}FO}%
\colorbox{green}{\color[gray]{0.75}FO}%
\colorbox{green}{\color[gray]{0.75}FO}%
\colorbox{green}{\color[gray]{0.75}FO}%
\colorbox{green}{\color[gray]{0.75}FO}%
\colorbox{green}{\color[gray]{0.75}FO}%
\colorbox{green}{\color[gray]{0.75}FO}%
\colorbox{green}{\color[gray]{0.75}FO}%
\colorbox{green}{\color[gray]{0.75}FO}%
\colorbox{green}{\color[gray]{0.75}FO}%
\colorbox{green}{\color[gray]{0.75}FO}%
\colorbox{green}{\color[gray]{0.75}FO}%
\colorbox{green}{\color[gray]{0.75}FO}%
\colorbox{green}{\color[gray]{0.75}FO}%
\colorbox{green}{\color[gray]{0.75}FO}%
\colorbox{green}{\color[gray]{0.75}FO}%
\colorbox{green}{\color[gray]{0.75}FO}%
\colorbox{green}{\color[gray]{0.75}FO}%
\colorbox{green}{\color[gray]{0.75}FO}%
\colorbox{green}{\color[gray]{0.75}FO}%
\colorbox{green}{\color[gray]{0.75}FO}%
\colorbox{green}{\color[gray]{0.75}FO}%
\colorbox{green}{\color[gray]{0.75}FO}%
\colorbox{green}{\color[gray]{0.75}FO}%
\colorbox{green}{\color[gray]{0.75}FO}%
\colorbox{green}{\color[gray]{0.75}FO}%
\colorbox{green}{\color[gray]{0.75}FO}%
\colorbox{green}{\color[gray]{0.75}FO}%
\colorbox{green}{\color[gray]{0.75}FO}%
\colorbox{green}{\color[gray]{0.75}FO}%
\colorbox{green}{\color[gray]{0.75}FO}%
\colorbox{green}{\color[gray]{0.75}FO}%
\colorbox{green}{\color[gray]{0.75}FO}%
\colorbox{green}{\color[gray]{0.75}FO}%
\colorbox{green}{\color[gray]{0.75}FO}%
\colorbox{green}{\color[gray]{0.75}FO}%
\colorbox{green}{\color[gray]{0.75}FO}%
\\
\colorbox{green}{\color[gray]{0.75}FO}%
\colorbox{green}{\color[gray]{0.75}FO}%
\colorbox{green}{\color[gray]{0.75}FO}%
\colorbox{green}{\color[gray]{0.75}FO}%
\colorbox{green}{\color[gray]{0.75}FO}%
\colorbox{green}{\color[gray]{0.75}FO}%
\colorbox{green}{\color[gray]{0.75}FO}%
\colorbox{green}{\color[gray]{0.75}FO}%
\colorbox{green}{\color[gray]{0.75}FO}%
\colorbox{green}{\color[gray]{0.75}FO}%
\colorbox{green}{\color[gray]{0.75}FO}%
\colorbox{green}{\color[gray]{0.75}FO}%
\colorbox{green}{\color[gray]{0.75}FO}%
\colorbox{green}{\color[gray]{0.75}FO}%
\colorbox{green}{\color[gray]{0.75}FO}%
\colorbox{green}{\color[gray]{0.75}FO}%
\colorbox{green}{\color[gray]{0.75}FO}%
\colorbox{green}{\color[gray]{0.75}FO}%
\colorbox{green}{\color[gray]{0.75}FO}%
\colorbox{green}{\color[gray]{0.75}FO}%
\colorbox{green}{\color[gray]{0.75}FO}%
\colorbox{green}{\color[gray]{0.75}FO}%
\colorbox{green}{\color[gray]{0.75}FO}%
\colorbox{green}{\color[gray]{0.75}FO}%
\colorbox{green}{\color[gray]{0.75}FO}%
\colorbox{green}{\color[gray]{0.75}FO}%
\colorbox{green}{\color[gray]{0.75}FO}%
\colorbox{green}{\color[gray]{0.75}FO}%
\colorbox{green}{\color[gray]{0.75}FO}%
\colorbox{green}{\color[gray]{0.75}FO}%
\colorbox{green}{\color[gray]{0.75}FO}%
\colorbox{green}{\color[gray]{0.75}FO}%
\colorbox{green}{\color[gray]{0.75}FO}%
\colorbox{green}{\color[gray]{0.75}FO}%
\colorbox{green}{\color[gray]{0.75}FO}%
\colorbox{green}{\color[gray]{0.75}FO}%
\colorbox{green}{\color[gray]{0.75}FO}%
\colorbox{green}{\color[gray]{0.75}FO}%
\colorbox{green}{\color[gray]{0.75}FO}%
\colorbox{green}{\color[gray]{0.75}FO}%
\colorbox{green}{\color[gray]{0.75}FO}%
\colorbox{green}{\color[gray]{0.75}FO}%
\colorbox{green}{\color[gray]{0.75}FO}%
\colorbox{green}{\color[gray]{0.75}FO}%
\colorbox{green}{\color[gray]{0.75}FO}%
\colorbox{green}{\color[gray]{0.75}FO}%
\colorbox{green}{\color[gray]{0.75}FO}%
\colorbox{green}{\color[gray]{0.75}FO}%
\colorbox{green}{\color[gray]{0.75}FO}%
\colorbox{green}{\color[gray]{0.75}FO}%
\colorbox{green}{\color[gray]{0.75}FO}%
\colorbox{green}{\color[gray]{0.75}FO}%
\colorbox{green}{\color[gray]{0.75}FO}%
\colorbox{green}{\color[gray]{0.75}FO}%
\colorbox{green}{\color[gray]{0.75}FO}%
\colorbox{green}{\color[gray]{0.75}FO}%
\colorbox{green}{\color[gray]{0.75}FO}%
\colorbox{green}{\color[gray]{0.75}FO}%
\colorbox{green}{\color[gray]{0.75}FO}%
\colorbox{green}{\color[gray]{0.75}FO}%
\colorbox{green}{\color[gray]{0.75}FO}%
\colorbox{green}{\color[gray]{0.75}FO}%
\colorbox{green}{\color[gray]{0.75}FO}%
\colorbox{green}{\color[gray]{0.75}FO}%
\colorbox{green}{\color[gray]{0.75}FO}%
\colorbox{green}{\color[gray]{0.75}FO}%
\colorbox{green}{\color[gray]{0.75}FO}%
\colorbox{green}{\color[gray]{0.75}FO}%
\colorbox{green}{\color[gray]{0.75}FO}%
\colorbox{green}{\color[gray]{0.75}FO}%
\colorbox{green}{\color[gray]{0.75}FO}%
\colorbox{green}{\color[gray]{0.75}FO}%
\colorbox{green}{\color[gray]{0.75}FO}%
\colorbox{green}{\color[gray]{0.75}FO}%
\colorbox{green}{\color[gray]{0.75}FO}%
\colorbox{green}{\color[gray]{0.75}FO}%
\colorbox{green}{\color[gray]{0.75}FO}%
\colorbox{green}{\color[gray]{0.75}FO}%
\colorbox{green}{\color[gray]{0.75}FO}%
\colorbox{green}{\color[gray]{0.75}FO}%
\colorbox{green}{\color[gray]{0.75}FO}%
\colorbox{green}{\color[gray]{0.75}FO}%
\colorbox{green}{\color[gray]{0.75}FO}%
\colorbox{green}{\color[gray]{0.75}FO}%
\colorbox{green}{\color[gray]{0.75}FO}%
\colorbox{green}{\color[gray]{0.75}FO}%
\colorbox{green}{\color[gray]{0.75}FO}%
\colorbox{green}{\color[gray]{0.75}FO}%
\colorbox{green}{\color[gray]{0.75}FO}%
\colorbox{green}{\color[gray]{0.75}FO}%
\colorbox{green}{\color[gray]{0.75}FO}%
\colorbox{green}{\color[gray]{0.75}FO}%
\colorbox{green}{\color[gray]{0.75}FO}%
\colorbox{green}{\color[gray]{0.75}FO}%
\colorbox{green}{\color[gray]{0.75}FO}%
\colorbox{green}{\color[gray]{0.75}FO}%
\colorbox{green}{\color[gray]{0.75}FO}%
\colorbox{green}{\color[gray]{0.75}FO}%
\colorbox{green}{\color[gray]{0.75}FO}%
\colorbox{green}{\color[gray]{0.75}FO}%
\\
\colorbox{green}{\color[gray]{0.75}FO}%
\colorbox{green}{\color[gray]{0.75}FO}%
\colorbox{green}{\color[gray]{0.75}FO}%
\colorbox{green}{\color[gray]{0.75}FO}%
\colorbox{green}{\color[gray]{0.75}FO}%
\colorbox{green}{\color[gray]{0.75}FO}%
\colorbox{green}{\color[gray]{0.75}FO}%
\colorbox{green}{\color[gray]{0.75}FO}%
\colorbox{green}{\color[gray]{0.75}FO}%
\colorbox{green}{\color[gray]{0.75}FO}%
\colorbox{green}{\color[gray]{0.75}FO}%
\colorbox{green}{\color[gray]{0.75}FO}%
\colorbox{green}{\color[gray]{0.75}FO}%
\colorbox{green}{\color[gray]{0.75}FO}%
\colorbox{green}{\color[gray]{0.75}FO}%
\colorbox{green}{\color[gray]{0.75}FO}%
\colorbox{green}{\color[gray]{0.75}FO}%
\colorbox{green}{\color[gray]{0.75}FO}%
\colorbox{green}{\color[gray]{0.75}FO}%
\colorbox{green}{\color[gray]{0.75}FO}%
\colorbox{green}{\color[gray]{0.75}FO}%
\colorbox{green}{\color[gray]{0.75}FO}%
\colorbox{green}{\color[gray]{0.75}FO}%
\colorbox{green}{\color[gray]{0.75}FO}%
\colorbox{green}{\color[gray]{0.75}FO}%
\colorbox{green}{\color[gray]{0.75}FO}%
\colorbox{green}{\color[gray]{0.75}FO}%
\colorbox{green}{\color[gray]{0.75}FO}%
\colorbox{green}{\color[gray]{0.75}FO}%
\colorbox{green}{\color[gray]{0.75}FO}%
\colorbox{green}{\color[gray]{0.75}FO}%
\colorbox{green}{\color[gray]{0.75}FO}%
\colorbox{green}{\color[gray]{0.75}FO}%
\colorbox{green}{\color[gray]{0.75}FO}%
\colorbox{green}{\color[gray]{0.75}FO}%
\colorbox{green}{\color[gray]{0.75}FO}%
\colorbox{green}{\color[gray]{0.75}FO}%
\colorbox{green}{\color[gray]{0.75}FO}%
\colorbox{green}{\color[gray]{0.75}FO}%
\colorbox{green}{\color[gray]{0.75}FO}%
\colorbox{green}{\color[gray]{0.75}FO}%
\colorbox{green}{\color[gray]{0.75}FO}%
\colorbox{green}{\color[gray]{0.75}FO}%
\colorbox{green}{\color[gray]{0.75}FO}%
\colorbox{green}{\color[gray]{0.75}FO}%
\colorbox{green}{\color[gray]{0.75}FO}%
\colorbox{green}{\color[gray]{0.75}FO}%
\colorbox{green}{\color[gray]{0.75}FO}%
\colorbox{green}{\color[gray]{0.75}FO}%
\colorbox{green}{\color[gray]{0.75}FO}%
\colorbox{green}{\color[gray]{0.75}FO}%
\colorbox{green}{\color[gray]{0.75}FO}%
\colorbox{green}{\color[gray]{0.75}FO}%
\colorbox{green}{\color[gray]{0.75}FO}%
\colorbox{green}{\color[gray]{0.75}FO}%
\colorbox{green}{\color[gray]{0.75}FO}%
\colorbox{green}{\color[gray]{0.75}FO}%
\colorbox{green}{\color[gray]{0.75}FO}%
\colorbox{green}{\color[gray]{0.75}FO}%
\colorbox{green}{\color[gray]{0.75}FO}%
\colorbox{green}{\color[gray]{0.75}FO}%
\colorbox{green}{\color[gray]{0.75}FO}%
\colorbox{green}{\color[gray]{0.75}FO}%
\colorbox{green}{\color[gray]{0.75}FO}%
\colorbox{green}{\color[gray]{0.75}FO}%
\colorbox{green}{\color[gray]{0.75}FO}%
\colorbox{green}{\color[gray]{0.75}FO}%
\colorbox{green}{\color[gray]{0.75}FO}%
\colorbox{green}{\color[gray]{0.75}FO}%
\colorbox{green}{\color[gray]{0.75}FO}%
\colorbox{green}{\color[gray]{0.75}FO}%
\colorbox{green}{\color[gray]{0.75}FO}%
\colorbox{green}{\color[gray]{0.75}FO}%
\colorbox{green}{\color[gray]{0.75}FO}%
\colorbox{green}{\color[gray]{0.75}FO}%
\colorbox{green}{\color[gray]{0.75}FO}%
\colorbox{green}{\color[gray]{0.75}FO}%
\colorbox{green}{\color[gray]{0.75}FO}%
\colorbox{green}{\color[gray]{0.75}FO}%
\colorbox{green}{\color[gray]{0.75}FO}%
\colorbox{green}{\color[gray]{0.75}FO}%
\colorbox{green}{\color[gray]{0.75}FO}%
\colorbox{green}{\color[gray]{0.75}FO}%
\colorbox{green}{\color[gray]{0.75}FO}%
\colorbox{green}{\color[gray]{0.75}FO}%
\colorbox{green}{\color[gray]{0.75}FO}%
\colorbox{green}{\color[gray]{0.75}FO}%
\colorbox{green}{\color[gray]{0.75}FO}%
\colorbox{green}{\color[gray]{0.75}FO}%
\colorbox{green}{\color[gray]{0.75}FO}%
\colorbox{green}{\color[gray]{0.75}FO}%
\colorbox{green}{\color[gray]{0.75}FO}%
\colorbox{green}{\color[gray]{0.75}FO}%
\colorbox{green}{\color[gray]{0.75}FO}%
\colorbox{green}{\color[gray]{0.75}FO}%
\colorbox{green}{\color[gray]{0.75}FO}%
\colorbox{green}{\color[gray]{0.75}FO}%
\colorbox{green}{\color[gray]{0.75}FO}%
\colorbox{green}{\color[gray]{0.75}FO}%
\colorbox{green}{\color[gray]{0.75}FO}%
\\
\colorbox{green}{\color[gray]{0.75}FO}%
\colorbox{green}{\color[gray]{0.75}FO}%
\colorbox{green}{\color[gray]{0.75}FO}%
\colorbox{green}{\color[gray]{0.75}FO}%
\colorbox{green}{\color[gray]{0.75}FO}%
\colorbox{green}{\color[gray]{0.75}FO}%
\colorbox{green}{\color[gray]{0.75}FO}%
\colorbox{green}{\color[gray]{0.75}FO}%
\colorbox{green}{\color[gray]{0.75}FO}%
\colorbox{green}{\color[gray]{0.75}FO}%
\colorbox{green}{\color[gray]{0.75}FO}%
\colorbox{green}{\color[gray]{0.75}FO}%
\colorbox{green}{\color[gray]{0.75}FO}%
\colorbox{green}{\color[gray]{0.75}FO}%
\colorbox{green}{\color[gray]{0.75}FO}%
\colorbox{green}{\color[gray]{0.75}FO}%
\colorbox{green}{\color[gray]{0.75}FO}%
\colorbox{green}{\color[gray]{0.75}FO}%
\colorbox{green}{\color[gray]{0.75}FO}%
\colorbox{green}{\color[gray]{0.75}FO}%
\colorbox{green}{\color[gray]{0.75}FO}%
\colorbox{green}{\color[gray]{0.75}FO}%
\colorbox{green}{\color[gray]{0.75}FO}%
\colorbox{green}{\color[gray]{0.75}FO}%
\colorbox{green}{\color[gray]{0.75}FO}%
\colorbox{green}{\color[gray]{0.75}FO}%
\colorbox{green}{\color[gray]{0.75}FO}%
\colorbox{green}{\color[gray]{0.75}FO}%
\colorbox{green}{\color[gray]{0.75}FO}%
\colorbox{green}{\color[gray]{0.75}FO}%
\colorbox{green}{\color[gray]{0.75}FO}%
\colorbox{green}{\color[gray]{0.75}FO}%
\colorbox{green}{\color[gray]{0.75}FO}%
\colorbox{green}{\color[gray]{0.75}FO}%
\colorbox{green}{\color[gray]{0.75}FO}%
\colorbox{green}{\color[gray]{0.75}FO}%
\colorbox{green}{\color[gray]{0.75}FO}%
\colorbox{green}{\color[gray]{0.75}FO}%
\colorbox{green}{\color[gray]{0.75}FO}%
\colorbox{green}{\color[gray]{0.75}FO}%
\colorbox{green}{\color[gray]{0.75}FO}%
\colorbox{green}{\color[gray]{0.75}FO}%
\colorbox{green}{\color[gray]{0.75}FO}%
\colorbox{green}{\color[gray]{0.75}FO}%
\colorbox{green}{\color[gray]{0.75}FO}%
\colorbox{green}{\color[gray]{0.75}FO}%
\colorbox{green}{\color[gray]{0.75}FO}%
\colorbox{green}{\color[gray]{0.75}FO}%
\colorbox{green}{\color[gray]{0.75}FO}%
\colorbox{green}{\color[gray]{0.75}FO}%
\colorbox{green}{\color[gray]{0.75}FO}%
\colorbox{green}{\color[gray]{0.75}FO}%
\colorbox{green}{\color[gray]{0.75}FO}%
\colorbox{green}{\color[gray]{0.75}FO}%
\colorbox{green}{\color[gray]{0.75}FO}%
\colorbox{green}{\color[gray]{0.75}FO}%
\colorbox{green}{\color[gray]{0.75}FO}%
\colorbox{green}{\color[gray]{0.75}FO}%
\colorbox{green}{\color[gray]{0.75}FO}%
\colorbox{green}{\color[gray]{0.75}FO}%
\colorbox{green}{\color[gray]{0.75}FO}%
\colorbox{green}{\color[gray]{0.75}FO}%
\colorbox{green}{\color[gray]{0.75}FO}%
\colorbox{green}{\color[gray]{0.75}FO}%
\colorbox{green}{\color[gray]{0.75}FO}%
\colorbox{green}{\color[gray]{0.75}FO}%
\colorbox{green}{\color[gray]{0.75}FO}%
\colorbox{green}{\color[gray]{0.75}FO}%
\colorbox{green}{\color[gray]{0.75}FO}%
\colorbox{green}{\color[gray]{0.75}FO}%
\colorbox{green}{\color[gray]{0.75}FO}%
\colorbox{green}{\color[gray]{0.75}FO}%
\colorbox{green}{\color[gray]{0.75}FO}%
\colorbox{green}{\color[gray]{0.75}FO}%
\colorbox{green}{\color[gray]{0.75}FO}%
\colorbox{green}{\color[gray]{0.75}FO}%
\colorbox{green}{\color[gray]{0.75}FO}%
\colorbox{green}{\color[gray]{0.75}FO}%
\colorbox{green}{\color[gray]{0.75}FO}%
\colorbox{green}{\color[gray]{0.75}FO}%
\colorbox{green}{\color[gray]{0.75}FO}%
\colorbox{green}{\color[gray]{0.75}FO}%
\colorbox{green}{\color[gray]{0.75}FO}%
\colorbox{green}{\color[gray]{0.75}FO}%
\colorbox{green}{\color[gray]{0.75}FO}%
\colorbox{green}{\color[gray]{0.75}FO}%
\colorbox{green}{\color[gray]{0.75}FO}%
\colorbox{green}{\color[gray]{0.75}FO}%
\colorbox{green}{\color[gray]{0.75}FO}%
\colorbox{green}{\color[gray]{0.75}FO}%
\colorbox{green}{\color[gray]{0.75}FO}%
\colorbox{green}{\color[gray]{0.75}FO}%
\colorbox{green}{\color[gray]{0.75}FO}%
\colorbox{green}{\color[gray]{0.75}FO}%
\colorbox{green}{\color[gray]{0.75}FO}%
\colorbox{green}{\color[gray]{0.75}FO}%
\colorbox{green}{\color[gray]{0.75}FO}%
\colorbox{green}{\color[gray]{0.75}FO}%
\colorbox{green}{\color[gray]{0.75}FO}%
\colorbox{green}{\color[gray]{0.75}FO}%
\\
\colorbox{green}{\color[gray]{0.75}FO}%
\colorbox{green}{\color[gray]{0.75}FO}%
\colorbox{green}{\color[gray]{0.75}FO}%
\colorbox{green}{\color[gray]{0.75}FO}%
\colorbox{green}{\color[gray]{0.75}FO}%
\colorbox{green}{\color[gray]{0.75}FO}%
\colorbox{green}{\color[gray]{0.75}FO}%
\colorbox{green}{\color[gray]{0.75}FO}%
\colorbox{green}{\color[gray]{0.75}FO}%
\colorbox{green}{\color[gray]{0.75}FO}%
\colorbox{green}{\color[gray]{0.75}FO}%
\colorbox{green}{\color[gray]{0.75}FO}%
\colorbox{green}{\color[gray]{0.75}FO}%
\colorbox{green}{\color[gray]{0.75}FO}%
\colorbox{green}{\color[gray]{0.75}FO}%
\colorbox{green}{\color[gray]{0.75}FO}%
\colorbox{green}{\color[gray]{0.75}FO}%
\colorbox{green}{\color[gray]{0.75}FO}%
\colorbox{green}{\color[gray]{0.75}FO}%
\colorbox{green}{\color[gray]{0.75}FO}%
\colorbox{green}{\color[gray]{0.75}FO}%
\colorbox{green}{\color[gray]{0.75}FO}%
\colorbox{green}{\color[gray]{0.75}FO}%
\colorbox{green}{\color[gray]{0.75}FO}%
\colorbox{green}{\color[gray]{0.75}FO}%
\colorbox{green}{\color[gray]{0.75}FO}%
\colorbox{green}{\color[gray]{0.75}FO}%
\colorbox{green}{\color[gray]{0.75}FO}%
\colorbox{green}{\color[gray]{0.75}FO}%
\colorbox{green}{\color[gray]{0.75}FO}%
\colorbox{green}{\color[gray]{0.75}FO}%
\colorbox{green}{\color[gray]{0.75}FO}%
\colorbox{green}{\color[gray]{0.75}FO}%
\colorbox{green}{\color[gray]{0.75}FO}%
\colorbox{green}{\color[gray]{0.75}FO}%
\colorbox{green}{\color[gray]{0.75}FO}%
\colorbox{green}{\color[gray]{0.75}FO}%
\colorbox{green}{\color[gray]{0.75}FO}%
\colorbox{green}{\color[gray]{0.75}FO}%
\colorbox{green}{\color[gray]{0.75}FO}%
\colorbox{green}{\color[gray]{0.75}FO}%
\colorbox{green}{\color[gray]{0.75}FO}%
\colorbox{green}{\color[gray]{0.75}FO}%
\colorbox{green}{\color[gray]{0.75}FO}%
\colorbox{green}{\color[gray]{0.75}FO}%
\colorbox{green}{\color[gray]{0.75}FO}%
\colorbox{green}{\color[gray]{0.75}FO}%
\colorbox{green}{\color[gray]{0.75}FO}%
\colorbox{green}{\color[gray]{0.75}FO}%
\colorbox{green}{\color[gray]{0.75}FO}%
\colorbox{green}{\color[gray]{0.75}FO}%
\colorbox{green}{\color[gray]{0.75}FO}%
\colorbox{green}{\color[gray]{0.75}FO}%
\colorbox{green}{\color[gray]{0.75}FO}%
\colorbox{green}{\color[gray]{0.75}FO}%
\colorbox{green}{\color[gray]{0.75}FO}%
\colorbox{green}{\color[gray]{0.75}FO}%
\colorbox{green}{\color[gray]{0.75}FO}%
\colorbox{green}{\color[gray]{0.75}FO}%
\colorbox{green}{\color[gray]{0.75}FO}%
\colorbox{green}{\color[gray]{0.75}FO}%
\colorbox{green}{\color[gray]{0.75}FO}%
\colorbox{green}{\color[gray]{0.75}FO}%
\colorbox{green}{\color[gray]{0.75}FO}%
\colorbox{green}{\color[gray]{0.75}FO}%
\colorbox{green}{\color[gray]{0.75}FO}%
\colorbox{green}{\color[gray]{0.75}FO}%
\colorbox{green}{\color[gray]{0.75}FO}%
\colorbox{green}{\color[gray]{0.75}FO}%
\colorbox{green}{\color[gray]{0.75}FO}%
\colorbox{green}{\color[gray]{0.75}FO}%
\colorbox{green}{\color[gray]{0.75}FO}%
\colorbox{green}{\color[gray]{0.75}FO}%
\colorbox{green}{\color[gray]{0.75}FO}%
\colorbox{green}{\color[gray]{0.75}FO}%
\colorbox{green}{\color[gray]{0.75}FO}%
\colorbox{green}{\color[gray]{0.75}FO}%
\colorbox{green}{\color[gray]{0.75}FO}%
\colorbox{green}{\color[gray]{0.75}FO}%
\colorbox{green}{\color[gray]{0.75}FO}%
\colorbox{green}{\color[gray]{0.75}FO}%
\colorbox{green}{\color[gray]{0.75}FO}%
\colorbox{green}{\color[gray]{0.75}FO}%
\colorbox{green}{\color[gray]{0.75}FO}%
\colorbox{green}{\color[gray]{0.75}FO}%
\colorbox{green}{\color[gray]{0.75}FO}%
\colorbox{green}{\color[gray]{0.75}FO}%
\colorbox{green}{\color[gray]{0.75}FO}%
\colorbox{green}{\color[gray]{0.75}FO}%
\colorbox{green}{\color[gray]{0.75}FO}%
\colorbox{green}{\color[gray]{0.75}FO}%
\colorbox{green}{\color[gray]{0.75}FO}%
\colorbox{green}{\color[gray]{0.75}FO}%
\colorbox{green}{\color[gray]{0.75}FO}%
\colorbox{green}{\color[gray]{0.75}FO}%
\colorbox{green}{\color[gray]{0.75}FO}%
\colorbox{green}{\color[gray]{0.75}FO}%
\colorbox{green}{\color[gray]{0.75}FO}%
\colorbox{green}{\color[gray]{0.75}FO}%
\colorbox{green}{\color[gray]{0.75}FO}%
\\
\colorbox{green}{\color[gray]{0.75}FO}%
\colorbox{green}{\color[gray]{0.75}FO}%
\colorbox{green}{\color[gray]{0.75}FO}%
\colorbox{green}{\color[gray]{0.75}FO}%
\colorbox{green}{\color[gray]{0.75}FO}%
\colorbox{green}{\color[gray]{0.75}FO}%
\colorbox{green}{\color[gray]{0.75}FO}%
\colorbox{green}{\color[gray]{0.75}FO}%
\colorbox{green}{\color[gray]{0.75}FO}%
\colorbox{green}{\color[gray]{0.75}FO}%
\colorbox{green}{\color[gray]{0.75}FO}%
\colorbox{green}{\color[gray]{0.75}FO}%
\colorbox{green}{\color[gray]{0.75}FO}%
\colorbox{green}{\color[gray]{0.75}FO}%
\colorbox{green}{\color[gray]{0.75}FO}%
\colorbox{green}{\color[gray]{0.75}FO}%
\colorbox{green}{\color[gray]{0.75}FO}%
\colorbox{green}{\color[gray]{0.75}FO}%
\colorbox{green}{\color[gray]{0.75}FO}%
\colorbox{green}{\color[gray]{0.75}FO}%
\colorbox{green}{\color[gray]{0.75}FO}%
\colorbox{green}{\color[gray]{0.75}FO}%
\colorbox{green}{\color[gray]{0.75}FO}%
\colorbox{green}{\color[gray]{0.75}FO}%
\colorbox{green}{\color[gray]{0.75}FO}%
\colorbox{green}{\color[gray]{0.75}FO}%
\colorbox{green}{\color[gray]{0.75}FO}%
\colorbox{green}{\color[gray]{0.75}FO}%
\colorbox{green}{\color[gray]{0.75}FO}%
\colorbox{green}{\color[gray]{0.75}FO}%
\colorbox{green}{\color[gray]{0.75}FO}%
\colorbox{green}{\color[gray]{0.75}FO}%
\colorbox{green}{\color[gray]{0.75}FO}%
\colorbox{green}{\color[gray]{0.75}FO}%
\colorbox{green}{\color[gray]{0.75}FO}%
\colorbox{green}{\color[gray]{0.75}FO}%
\colorbox{green}{\color[gray]{0.75}FO}%
\colorbox{green}{\color[gray]{0.75}FO}%
\colorbox{green}{\color[gray]{0.75}FO}%
\colorbox{green}{\color[gray]{0.75}FO}%
\colorbox{green}{\color[gray]{0.75}FO}%
\colorbox{green}{\color[gray]{0.75}FO}%
\colorbox{green}{\color[gray]{0.75}FO}%
\colorbox{green}{\color[gray]{0.75}FO}%
\colorbox{green}{\color[gray]{0.75}FO}%
\colorbox{green}{\color[gray]{0.75}FO}%
\colorbox{green}{\color[gray]{0.75}FO}%
\colorbox{green}{\color[gray]{0.75}FO}%
\colorbox{green}{\color[gray]{0.75}FO}%
\colorbox{green}{\color[gray]{0.75}FO}%
\colorbox{green}{\color[gray]{0.75}FO}%
\colorbox{green}{\color[gray]{0.75}FO}%
\colorbox{green}{\color[gray]{0.75}FO}%
\colorbox{green}{\color[gray]{0.75}FO}%
\colorbox{green}{\color[gray]{0.75}FO}%
\colorbox{green}{\color[gray]{0.75}FO}%
\colorbox{green}{\color[gray]{0.75}FO}%
\colorbox{green}{\color[gray]{0.75}FO}%
\colorbox{green}{\color[gray]{0.75}FO}%
\colorbox{green}{\color[gray]{0.75}FO}%
\colorbox{green}{\color[gray]{0.75}FO}%
\colorbox{green}{\color[gray]{0.75}FO}%
\colorbox{green}{\color[gray]{0.75}FO}%
\colorbox{green}{\color[gray]{0.75}FO}%
\colorbox{green}{\color[gray]{0.75}FO}%
\colorbox{green}{\color[gray]{0.75}FO}%
\colorbox{green}{\color[gray]{0.75}FO}%
\colorbox{green}{\color[gray]{0.75}FO}%
\colorbox{green}{\color[gray]{0.75}FO}%
\colorbox{green}{\color[gray]{0.75}FO}%
\colorbox{green}{\color[gray]{0.75}FO}%
\colorbox{green}{\color[gray]{0.75}FO}%
\colorbox{green}{\color[gray]{0.75}FO}%
\colorbox{green}{\color[gray]{0.75}FO}%
\colorbox{green}{\color[gray]{0.75}FO}%
\colorbox{green}{\color[gray]{0.75}FO}%
\colorbox{green}{\color[gray]{0.75}FO}%
\colorbox{green}{\color[gray]{0.75}FO}%
\colorbox{green}{\color[gray]{0.75}FO}%
\colorbox{green}{\color[gray]{0.75}FO}%
\colorbox{green}{\color[gray]{0.75}FO}%
\colorbox{green}{\color[gray]{0.75}FO}%
\colorbox{green}{\color[gray]{0.75}FO}%
\colorbox{green}{\color[gray]{0.75}FO}%
\colorbox{green}{\color[gray]{0.75}FO}%
\colorbox{green}{\color[gray]{0.75}FO}%
\colorbox{green}{\color[gray]{0.75}FO}%
\colorbox{green}{\color[gray]{0.75}FO}%
\colorbox{green}{\color[gray]{0.75}FO}%
\colorbox{green}{\color[gray]{0.75}FO}%
\colorbox{green}{\color[gray]{0.75}FO}%
\colorbox{green}{\color[gray]{0.75}FO}%
\colorbox{green}{\color[gray]{0.75}FO}%
\colorbox{green}{\color[gray]{0.75}FO}%
\colorbox{green}{\color[gray]{0.75}FO}%
\colorbox{green}{\color[gray]{0.75}FO}%
\colorbox{green}{\color[gray]{0.75}FO}%
\colorbox{green}{\color[gray]{0.75}FO}%
\colorbox{green}{\color[gray]{0.75}FO}%
\colorbox{green}{\color[gray]{0.75}FO}%
\\
\colorbox{green}{\color[gray]{0.75}FO}%
\colorbox{green}{\color[gray]{0.75}FO}%
\colorbox{green}{\color[gray]{0.75}FO}%
\colorbox{green}{\color[gray]{0.75}FO}%
\colorbox{green}{\color[gray]{0.75}FO}%
\colorbox{green}{\color[gray]{0.75}FO}%
\colorbox{green}{\color[gray]{0.75}FO}%
\colorbox{green}{\color[gray]{0.75}FO}%
\colorbox{green}{\color[gray]{0.75}FO}%
\colorbox{green}{\color[gray]{0.75}FO}%
\colorbox{green}{\color[gray]{0.75}FO}%
\colorbox{green}{\color[gray]{0.75}FO}%
\colorbox{green}{\color[gray]{0.75}FO}%
\colorbox{green}{\color[gray]{0.75}FO}%
\colorbox{green}{\color[gray]{0.75}FO}%
\colorbox{green}{\color[gray]{0.75}FO}%
\colorbox{green}{\color[gray]{0.75}FO}%
\colorbox{green}{\color[gray]{0.75}FO}%
\colorbox{green}{\color[gray]{0.75}FO}%
\colorbox{green}{\color[gray]{0.75}FO}%
\colorbox{green}{\color[gray]{0.75}FO}%
\colorbox{green}{\color[gray]{0.75}FO}%
\colorbox{green}{\color[gray]{0.75}FO}%
\colorbox{green}{\color[gray]{0.75}FO}%
\colorbox{green}{\color[gray]{0.75}FO}%
\colorbox{green}{\color[gray]{0.75}FO}%
\colorbox{green}{\color[gray]{0.75}FO}%
\colorbox{green}{\color[gray]{0.75}FO}%
\colorbox{green}{\color[gray]{0.75}FO}%
\colorbox{green}{\color[gray]{0.75}FO}%
\colorbox{green}{\color[gray]{0.75}FO}%
\colorbox{green}{\color[gray]{0.75}FO}%
\colorbox{green}{\color[gray]{0.75}FO}%
\colorbox{green}{\color[gray]{0.75}FO}%
\colorbox{green}{\color[gray]{0.75}FO}%
\colorbox{green}{\color[gray]{0.75}FO}%
\colorbox{green}{\color[gray]{0.75}FO}%
\colorbox{green}{\color[gray]{0.75}FO}%
\colorbox{green}{\color[gray]{0.75}FO}%
\colorbox{green}{\color[gray]{0.75}FO}%
\colorbox{green}{\color[gray]{0.75}FO}%
\colorbox{green}{\color[gray]{0.75}FO}%
\colorbox{green}{\color[gray]{0.75}FO}%
\colorbox{green}{\color[gray]{0.75}FO}%
\colorbox{green}{\color[gray]{0.75}FO}%
\colorbox{green}{\color[gray]{0.75}FO}%
\colorbox{green}{\color[gray]{0.75}FO}%
\colorbox{green}{\color[gray]{0.75}FO}%
\colorbox{green}{\color[gray]{0.75}FO}%
\colorbox{green}{\color[gray]{0.75}FO}%
\colorbox{green}{\color[gray]{0.75}FO}%
\colorbox{green}{\color[gray]{0.75}FO}%
\colorbox{green}{\color[gray]{0.75}FO}%
\colorbox{green}{\color[gray]{0.75}FO}%
\colorbox{green}{\color[gray]{0.75}FO}%
\colorbox{green}{\color[gray]{0.75}FO}%
\colorbox{green}{\color[gray]{0.75}FO}%
\colorbox{green}{\color[gray]{0.75}FO}%
\colorbox{green}{\color[gray]{0.75}FO}%
\colorbox{green}{\color[gray]{0.75}FO}%
\colorbox{green}{\color[gray]{0.75}FO}%
\colorbox{green}{\color[gray]{0.75}FO}%
\colorbox{green}{\color[gray]{0.75}FO}%
\colorbox{green}{\color[gray]{0.75}FO}%
\colorbox{green}{\color[gray]{0.75}FO}%
\colorbox{green}{\color[gray]{0.75}FO}%
\colorbox{green}{\color[gray]{0.75}FO}%
\colorbox{green}{\color[gray]{0.75}FO}%
\colorbox{green}{\color[gray]{0.75}FO}%
\colorbox{green}{\color[gray]{0.75}FO}%
\colorbox{green}{\color[gray]{0.75}FO}%
\colorbox{green}{\color[gray]{0.75}FO}%
\colorbox{green}{\color[gray]{0.75}FO}%
\colorbox{green}{\color[gray]{0.75}FO}%
\colorbox{green}{\color[gray]{0.75}FO}%
\colorbox{green}{\color[gray]{0.75}FO}%
\colorbox{green}{\color[gray]{0.75}FO}%
\colorbox{green}{\color[gray]{0.75}FO}%
\colorbox{green}{\color[gray]{0.75}FO}%
\colorbox{green}{\color[gray]{0.75}FO}%
\colorbox{green}{\color[gray]{0.75}FO}%
\colorbox{green}{\color[gray]{0.75}FO}%
\colorbox{green}{\color[gray]{0.75}FO}%
\colorbox{green}{\color[gray]{0.75}FO}%
\colorbox{green}{\color[gray]{0.75}FO}%
\colorbox{green}{\color[gray]{0.75}FO}%
\colorbox{green}{\color[gray]{0.75}FO}%
\colorbox{green}{\color[gray]{0.75}FO}%
\colorbox{green}{\color[gray]{0.75}FO}%
\colorbox{green}{\color[gray]{0.75}FO}%
\colorbox{green}{\color[gray]{0.75}FO}%
\colorbox{green}{\color[gray]{0.75}FO}%
\colorbox{green}{\color[gray]{0.75}FO}%
\colorbox{green}{\color[gray]{0.75}FO}%
\colorbox{green}{\color[gray]{0.75}FO}%
\colorbox{green}{\color[gray]{0.75}FO}%
\colorbox{green}{\color[gray]{0.75}FO}%
\colorbox{green}{\color[gray]{0.75}FO}%
\colorbox{green}{\color[gray]{0.75}FO}%
\colorbox{green}{\color[gray]{0.75}FO}%
\\
\colorbox{green}{\color[gray]{0.75}FO}%
\colorbox{green}{\color[gray]{0.75}FO}%
\colorbox{green}{\color[gray]{0.75}FO}%
\colorbox{green}{\color[gray]{0.75}FO}%
\colorbox{green}{\color[gray]{0.75}FO}%
\colorbox{green}{\color[gray]{0.75}FO}%
\colorbox{green}{\color[gray]{0.75}FO}%
\colorbox{green}{\color[gray]{0.75}FO}%
\colorbox{green}{\color[gray]{0.75}FO}%
\colorbox{green}{\color[gray]{0.75}FO}%
\colorbox{green}{\color[gray]{0.75}FO}%
\colorbox{green}{\color[gray]{0.75}FO}%
\colorbox{green}{\color[gray]{0.75}FO}%
\colorbox{green}{\color[gray]{0.75}FO}%
\colorbox{green}{\color[gray]{0.75}FO}%
\colorbox{green}{\color[gray]{0.75}FO}%
\colorbox{green}{\color[gray]{0.75}FO}%
\colorbox{green}{\color[gray]{0.75}FO}%
\colorbox{green}{\color[gray]{0.75}FO}%
\colorbox{green}{\color[gray]{0.75}FO}%
\colorbox{green}{\color[gray]{0.75}FO}%
\colorbox{green}{\color[gray]{0.75}FO}%
\colorbox{green}{\color[gray]{0.75}FO}%
\colorbox{green}{\color[gray]{0.75}FO}%
\colorbox{green}{\color[gray]{0.75}FO}%
\colorbox{green}{\color[gray]{0.75}FO}%
\colorbox{green}{\color[gray]{0.75}FO}%
\colorbox{green}{\color[gray]{0.75}FO}%
\colorbox{green}{\color[gray]{0.75}FO}%
\colorbox{green}{\color[gray]{0.75}FO}%
\colorbox{green}{\color[gray]{0.75}FO}%
\colorbox{green}{\color[gray]{0.75}FO}%
\colorbox{green}{\color[gray]{0.75}FO}%
\colorbox{green}{\color[gray]{0.75}FO}%
\colorbox{green}{\color[gray]{0.75}FO}%
\colorbox{green}{\color[gray]{0.75}FO}%
\colorbox{green}{\color[gray]{0.75}FO}%
\colorbox{green}{\color[gray]{0.75}FO}%
\colorbox{green}{\color[gray]{0.75}FO}%
\colorbox{green}{\color[gray]{0.75}FO}%
\colorbox{green}{\color[gray]{0.75}FO}%
\colorbox{green}{\color[gray]{0.75}FO}%
\colorbox{green}{\color[gray]{0.75}FO}%
\colorbox{green}{\color[gray]{0.75}FO}%
\colorbox{green}{\color[gray]{0.75}FO}%
\colorbox{green}{\color[gray]{0.75}FO}%
\colorbox{green}{\color[gray]{0.75}FO}%
\colorbox{green}{\color[gray]{0.75}FO}%
\colorbox{green}{\color[gray]{0.75}FO}%
\colorbox{green}{\color[gray]{0.75}FO}%
\colorbox{green}{\color[gray]{0.75}FO}%
\colorbox{green}{\color[gray]{0.75}FO}%
\colorbox{green}{\color[gray]{0.75}FO}%
\colorbox{green}{\color[gray]{0.75}FO}%
\colorbox{green}{\color[gray]{0.75}FO}%
\colorbox{green}{\color[gray]{0.75}FO}%
\colorbox{green}{\color[gray]{0.75}FO}%
\colorbox{green}{\color[gray]{0.75}FO}%
\colorbox{green}{\color[gray]{0.75}FO}%
\colorbox{green}{\color[gray]{0.75}FO}%
\colorbox{green}{\color[gray]{0.75}FO}%
\colorbox{green}{\color[gray]{0.75}FO}%
\colorbox{green}{\color[gray]{0.75}FO}%
\colorbox{green}{\color[gray]{0.75}FO}%
\colorbox{green}{\color[gray]{0.75}FO}%
\colorbox{green}{\color[gray]{0.75}FO}%
\colorbox{green}{\color[gray]{0.75}FO}%
\colorbox{green}{\color[gray]{0.75}FO}%
\colorbox{green}{\color[gray]{0.75}FO}%
\colorbox{green}{\color[gray]{0.75}FO}%
\colorbox{green}{\color[gray]{0.75}FO}%
\colorbox{green}{\color[gray]{0.75}FO}%
\colorbox{green}{\color[gray]{0.75}FO}%
\colorbox{green}{\color[gray]{0.75}FO}%
\colorbox{green}{\color[gray]{0.75}FO}%
\colorbox{green}{\color[gray]{0.75}FO}%
\colorbox{green}{\color[gray]{0.75}FO}%
\colorbox{green}{\color[gray]{0.75}FO}%
\colorbox{green}{\color[gray]{0.75}FO}%
\colorbox{green}{\color[gray]{0.75}FO}%
\colorbox{green}{\color[gray]{0.75}FO}%
\colorbox{green}{\color[gray]{0.75}FO}%
\colorbox{green}{\color[gray]{0.75}FO}%
\colorbox{green}{\color[gray]{0.75}FO}%
\colorbox{green}{\color[gray]{0.75}FO}%
\colorbox{green}{\color[gray]{0.75}FO}%
\colorbox{green}{\color[gray]{0.75}FO}%
\colorbox{green}{\color[gray]{0.75}FO}%
\colorbox{green}{\color[gray]{0.75}FO}%
\colorbox{green}{\color[gray]{0.75}FO}%
\colorbox{green}{\color[gray]{0.75}FO}%
\colorbox{green}{\color[gray]{0.75}FO}%
\colorbox{green}{\color[gray]{0.75}FO}%
\colorbox{green}{\color[gray]{0.75}FO}%
\colorbox{green}{\color[gray]{0.75}FO}%
\colorbox{green}{\color[gray]{0.75}FO}%
\colorbox{green}{\color[gray]{0.75}FO}%
\colorbox{green}{\color[gray]{0.75}FO}%
\colorbox{green}{\color[gray]{0.75}FO}%
\colorbox{green}{\color[gray]{0.75}FO}%
\\
\colorbox{green}{\color[gray]{0.75}FO}%
\colorbox{green}{\color[gray]{0.75}FO}%
\colorbox{green}{\color[gray]{0.75}FO}%
\colorbox{green}{\color[gray]{0.75}FO}%
\colorbox{green}{\color[gray]{0.75}FO}%
\colorbox{green}{\color[gray]{0.75}FO}%
\colorbox{green}{\color[gray]{0.75}FO}%
\colorbox{green}{\color[gray]{0.75}FO}%
\colorbox{green}{\color[gray]{0.75}FO}%
\colorbox{green}{\color[gray]{0.75}FO}%
\colorbox{green}{\color[gray]{0.75}FO}%
\colorbox{green}{\color[gray]{0.75}FO}%
\colorbox{green}{\color[gray]{0.75}FO}%
\colorbox{green}{\color[gray]{0.75}FO}%
\colorbox{green}{\color[gray]{0.75}FO}%
\colorbox{green}{\color[gray]{0.75}FO}%
\colorbox{green}{\color[gray]{0.75}FO}%
\colorbox{green}{\color[gray]{0.75}FO}%
\colorbox{green}{\color[gray]{0.75}FO}%
\colorbox{green}{\color[gray]{0.75}FO}%
\colorbox{green}{\color[gray]{0.75}FO}%
\colorbox{green}{\color[gray]{0.75}FO}%
\colorbox{green}{\color[gray]{0.75}FO}%
\colorbox{green}{\color[gray]{0.75}FO}%
\colorbox{green}{\color[gray]{0.75}FO}%
\colorbox{green}{\color[gray]{0.75}FO}%
\colorbox{green}{\color[gray]{0.75}FO}%
\colorbox{green}{\color[gray]{0.75}FO}%
\colorbox{green}{\color[gray]{0.75}FO}%
\colorbox{green}{\color[gray]{0.75}FO}%
\colorbox{green}{\color[gray]{0.75}FO}%
\colorbox{green}{\color[gray]{0.75}FO}%
\colorbox{green}{\color[gray]{0.75}FO}%
\colorbox{green}{\color[gray]{0.75}FO}%
\colorbox{green}{\color[gray]{0.75}FO}%
\colorbox{green}{\color[gray]{0.75}FO}%
\colorbox{green}{\color[gray]{0.75}FO}%
\colorbox{green}{\color[gray]{0.75}FO}%
\colorbox{green}{\color[gray]{0.75}FO}%
\colorbox{green}{\color[gray]{0.75}FO}%
\colorbox{green}{\color[gray]{0.75}FO}%
\colorbox{green}{\color[gray]{0.75}FO}%
\colorbox{green}{\color[gray]{0.75}FO}%
\colorbox{green}{\color[gray]{0.75}FO}%
\colorbox{green}{\color[gray]{0.75}FO}%
\colorbox{green}{\color[gray]{0.75}FO}%
\colorbox{green}{\color[gray]{0.75}FO}%
\colorbox{green}{\color[gray]{0.75}FO}%
\colorbox{green}{\color[gray]{0.75}FO}%
\colorbox{green}{\color[gray]{0.75}FO}%
\colorbox{green}{\color[gray]{0.75}FO}%
\colorbox{green}{\color[gray]{0.75}FO}%
\colorbox{green}{\color[gray]{0.75}FO}%
\colorbox{green}{\color[gray]{0.75}FO}%
\colorbox{green}{\color[gray]{0.75}FO}%
\colorbox{green}{\color[gray]{0.75}FO}%
\colorbox{green}{\color[gray]{0.75}FO}%
\colorbox{green}{\color[gray]{0.75}FO}%
\colorbox{green}{\color[gray]{0.75}FO}%
\colorbox{green}{\color[gray]{0.75}FO}%
\colorbox{green}{\color[gray]{0.75}FO}%
\colorbox{green}{\color[gray]{0.75}FO}%
\colorbox{green}{\color[gray]{0.75}FO}%
\colorbox{green}{\color[gray]{0.75}FO}%
\colorbox{green}{\color[gray]{0.75}FO}%
\colorbox{green}{\color[gray]{0.75}FO}%
\colorbox{green}{\color[gray]{0.75}FO}%
\colorbox{green}{\color[gray]{0.75}FO}%
\colorbox{green}{\color[gray]{0.75}FO}%
\colorbox{green}{\color[gray]{0.75}FO}%
\colorbox{green}{\color[gray]{0.75}FO}%
\colorbox{green}{\color[gray]{0.75}FO}%
\colorbox{green}{\color[gray]{0.75}FO}%
\colorbox{green}{\color[gray]{0.75}FO}%
\colorbox{green}{\color[gray]{0.75}FO}%
\colorbox{green}{\color[gray]{0.75}FO}%
\colorbox{green}{\color[gray]{0.75}FO}%
\colorbox{green}{\color[gray]{0.75}FO}%
\colorbox{green}{\color[gray]{0.75}FO}%
\colorbox{green}{\color[gray]{0.75}FO}%
\colorbox{green}{\color[gray]{0.75}FO}%
\colorbox{green}{\color[gray]{0.75}FO}%
\colorbox{green}{\color[gray]{0.75}FO}%
\colorbox{green}{\color[gray]{0.75}FO}%
\colorbox{green}{\color[gray]{0.75}FO}%
\colorbox{green}{\color[gray]{0.75}FO}%
\colorbox{green}{\color[gray]{0.75}FO}%
\colorbox{green}{\color[gray]{0.75}FO}%
\colorbox{green}{\color[gray]{0.75}FO}%
\colorbox{green}{\color[gray]{0.75}FO}%
\colorbox{green}{\color[gray]{0.75}FO}%
\colorbox{green}{\color[gray]{0.75}FO}%
\colorbox{green}{\color[gray]{0.75}FO}%
\colorbox{green}{\color[gray]{0.75}FO}%
\colorbox{green}{\color[gray]{0.75}FO}%
\colorbox{green}{\color[gray]{0.75}FO}%
\colorbox{green}{\color[gray]{0.75}FO}%
\colorbox{green}{\color[gray]{0.75}FO}%
\colorbox{green}{\color[gray]{0.75}FO}%
\colorbox{green}{\color[gray]{0.75}FO}%
\\
\colorbox{green}{\color[gray]{0.75}FO}%
\colorbox{green}{\color[gray]{0.75}FO}%
\colorbox{green}{\color[gray]{0.75}FO}%
\colorbox{green}{\color[gray]{0.75}FO}%
\colorbox{green}{\color[gray]{0.75}FO}%
\colorbox{green}{\color[gray]{0.75}FO}%
\colorbox{green}{\color[gray]{0.75}FO}%
\colorbox{green}{\color[gray]{0.75}FO}%
\colorbox{green}{\color[gray]{0.75}FO}%
\colorbox{green}{\color[gray]{0.75}FO}%
\colorbox{green}{\color[gray]{0.75}FO}%
\colorbox{green}{\color[gray]{0.75}FO}%
\colorbox{green}{\color[gray]{0.75}FO}%
\colorbox{green}{\color[gray]{0.75}FO}%
\colorbox{green}{\color[gray]{0.75}FO}%
\colorbox{green}{\color[gray]{0.75}FO}%
\colorbox{green}{\color[gray]{0.75}FO}%
\colorbox{green}{\color[gray]{0.75}FO}%
\colorbox{green}{\color[gray]{0.75}FO}%
\colorbox{green}{\color[gray]{0.75}FO}%
\colorbox{green}{\color[gray]{0.75}FO}%
\colorbox{green}{\color[gray]{0.75}FO}%
\colorbox{green}{\color[gray]{0.75}FO}%
\colorbox{green}{\color[gray]{0.75}FO}%
\colorbox{green}{\color[gray]{0.75}FO}%
\colorbox{green}{\color[gray]{0.75}FO}%
\colorbox{green}{\color[gray]{0.75}FO}%
\colorbox{green}{\color[gray]{0.75}FO}%
\colorbox{green}{\color[gray]{0.75}FO}%
\colorbox{green}{\color[gray]{0.75}FO}%
\colorbox{green}{\color[gray]{0.75}FO}%
\colorbox{green}{\color[gray]{0.75}FO}%
\colorbox{green}{\color[gray]{0.75}FO}%
\colorbox{green}{\color[gray]{0.75}FO}%
\colorbox{green}{\color[gray]{0.75}FO}%
\colorbox{green}{\color[gray]{0.75}FO}%
\colorbox{green}{\color[gray]{0.75}FO}%
\colorbox{green}{\color[gray]{0.75}FO}%
\colorbox{green}{\color[gray]{0.75}FO}%
\colorbox{green}{\color[gray]{0.75}FO}%
\colorbox{green}{\color[gray]{0.75}FO}%
\colorbox{green}{\color[gray]{0.75}FO}%
\colorbox{green}{\color[gray]{0.75}FO}%
\colorbox{green}{\color[gray]{0.75}FO}%
\colorbox{green}{\color[gray]{0.75}FO}%
\colorbox{green}{\color[gray]{0.75}FO}%
\colorbox{green}{\color[gray]{0.75}FO}%
\colorbox{green}{\color[gray]{0.75}FO}%
\colorbox{green}{\color[gray]{0.75}FO}%
\colorbox{green}{\color[gray]{0.75}FO}%
\colorbox{green}{\color[gray]{0.75}FO}%
\colorbox{green}{\color[gray]{0.75}FO}%
\colorbox{green}{\color[gray]{0.75}FO}%
\colorbox{green}{\color[gray]{0.75}FO}%
\colorbox{green}{\color[gray]{0.75}FO}%
\colorbox{green}{\color[gray]{0.75}FO}%
\colorbox{green}{\color[gray]{0.75}FO}%
\colorbox{green}{\color[gray]{0.75}FO}%
\colorbox{green}{\color[gray]{0.75}FO}%
\colorbox{green}{\color[gray]{0.75}FO}%
\colorbox{green}{\color[gray]{0.75}FO}%
\colorbox{green}{\color[gray]{0.75}FO}%
\colorbox{green}{\color[gray]{0.75}FO}%
\colorbox{green}{\color[gray]{0.75}FO}%
\colorbox{green}{\color[gray]{0.75}FO}%
\colorbox{green}{\color[gray]{0.75}FO}%
\colorbox{green}{\color[gray]{0.75}FO}%
\colorbox{green}{\color[gray]{0.75}FO}%
\colorbox{green}{\color[gray]{0.75}FO}%
\colorbox{green}{\color[gray]{0.75}FO}%
\colorbox{green}{\color[gray]{0.75}FO}%
\colorbox{green}{\color[gray]{0.75}FO}%
\colorbox{green}{\color[gray]{0.75}FO}%
\colorbox{green}{\color[gray]{0.75}FO}%
\colorbox{green}{\color[gray]{0.75}FO}%
\colorbox{green}{\color[gray]{0.75}FO}%
\colorbox{green}{\color[gray]{0.75}FO}%
\colorbox{green}{\color[gray]{0.75}FO}%
\colorbox{green}{\color[gray]{0.75}FO}%
\colorbox{green}{\color[gray]{0.75}FO}%
\colorbox{green}{\color[gray]{0.75}FO}%
\colorbox{green}{\color[gray]{0.75}FO}%
\colorbox{green}{\color[gray]{0.75}FO}%
\colorbox{green}{\color[gray]{0.75}FO}%
\colorbox{green}{\color[gray]{0.75}FO}%
\colorbox{green}{\color[gray]{0.75}FO}%
\colorbox{green}{\color[gray]{0.75}FO}%
\colorbox{green}{\color[gray]{0.75}FO}%
\colorbox{green}{\color[gray]{0.75}FO}%
\colorbox{green}{\color[gray]{0.75}FO}%
\colorbox{green}{\color[gray]{0.75}FO}%
\colorbox{green}{\color[gray]{0.75}FO}%
\colorbox{green}{\color[gray]{0.75}FO}%
\colorbox{green}{\color[gray]{0.75}FO}%
\colorbox{green}{\color[gray]{0.75}FO}%
\colorbox{green}{\color[gray]{0.75}FO}%
\colorbox{green}{\color[gray]{0.75}FO}%
\colorbox{green}{\color[gray]{0.75}FO}%
\colorbox{green}{\color[gray]{0.75}FO}%
\colorbox{green}{\color[gray]{0.75}FO}%
\\
\colorbox{green}{\color[gray]{0.75}FO}%
\colorbox{green}{\color[gray]{0.75}FO}%
\colorbox{green}{\color[gray]{0.75}FO}%
\colorbox{green}{\color[gray]{0.75}FO}%
\colorbox{green}{\color[gray]{0.75}FO}%
\colorbox{green}{\color[gray]{0.75}FO}%
\colorbox{green}{\color[gray]{0.75}FO}%
\colorbox{green}{\color[gray]{0.75}FO}%
\colorbox{green}{\color[gray]{0.75}FO}%
\colorbox{green}{\color[gray]{0.75}FO}%
\colorbox{green}{\color[gray]{0.75}FO}%
\colorbox{green}{\color[gray]{0.75}FO}%
\colorbox{green}{\color[gray]{0.75}FO}%
\colorbox{green}{\color[gray]{0.75}FO}%
\colorbox{green}{\color[gray]{0.75}FO}%
\colorbox{green}{\color[gray]{0.75}FO}%
\colorbox{green}{\color[gray]{0.75}FO}%
\colorbox{green}{\color[gray]{0.75}FO}%
\colorbox{green}{\color[gray]{0.75}FO}%
\colorbox{green}{\color[gray]{0.75}FO}%
\colorbox{green}{\color[gray]{0.75}FO}%
\colorbox{green}{\color[gray]{0.75}FO}%
\colorbox{green}{\color[gray]{0.75}FO}%
\colorbox{green}{\color[gray]{0.75}FO}%
\colorbox{green}{\color[gray]{0.75}FO}%
\colorbox{green}{\color[gray]{0.75}FO}%
\colorbox{green}{\color[gray]{0.75}FO}%
\colorbox{green}{\color[gray]{0.75}FO}%
\colorbox{green}{\color[gray]{0.75}FO}%
\colorbox{green}{\color[gray]{0.75}FO}%
\colorbox{green}{\color[gray]{0.75}FO}%
\colorbox{green}{\color[gray]{0.75}FO}%
\colorbox{green}{\color[gray]{0.75}FO}%
\colorbox{green}{\color[gray]{0.75}FO}%
\colorbox{green}{\color[gray]{0.75}FO}%
\colorbox{green}{\color[gray]{0.75}FO}%
\colorbox{green}{\color[gray]{0.75}FO}%
\colorbox{green}{\color[gray]{0.75}FO}%
\colorbox{green}{\color[gray]{0.75}FO}%
\colorbox{green}{\color[gray]{0.75}FO}%
\colorbox{green}{\color[gray]{0.75}FO}%
\colorbox{green}{\color[gray]{0.75}FO}%
\colorbox{green}{\color[gray]{0.75}FO}%
\colorbox{green}{\color[gray]{0.75}FO}%
\colorbox{green}{\color[gray]{0.75}FO}%
\colorbox{green}{\color[gray]{0.75}FO}%
\colorbox{green}{\color[gray]{0.75}FO}%
\colorbox{green}{\color[gray]{0.75}FO}%
\colorbox{green}{\color[gray]{0.75}FO}%
\colorbox{green}{\color[gray]{0.75}FO}%
\colorbox{green}{\color[gray]{0.75}FO}%
\colorbox{green}{\color[gray]{0.75}FO}%
\colorbox{green}{\color[gray]{0.75}FO}%
\colorbox{green}{\color[gray]{0.75}FO}%
\colorbox{green}{\color[gray]{0.75}FO}%
\colorbox{green}{\color[gray]{0.75}FO}%
\colorbox{green}{\color[gray]{0.75}FO}%
\colorbox{green}{\color[gray]{0.75}FO}%
\colorbox{green}{\color[gray]{0.75}FO}%
\colorbox{green}{\color[gray]{0.75}FO}%
\colorbox{green}{\color[gray]{0.75}FO}%
\colorbox{green}{\color[gray]{0.75}FO}%
\colorbox{green}{\color[gray]{0.75}FO}%
\colorbox{green}{\color[gray]{0.75}FO}%
\colorbox{green}{\color[gray]{0.75}FO}%
\colorbox{green}{\color[gray]{0.75}FO}%
\colorbox{green}{\color[gray]{0.75}FO}%
\colorbox{green}{\color[gray]{0.75}FO}%
\colorbox{green}{\color[gray]{0.75}FO}%
\colorbox{green}{\color[gray]{0.75}FO}%
\colorbox{green}{\color[gray]{0.75}FO}%
\colorbox{green}{\color[gray]{0.75}FO}%
\colorbox{green}{\color[gray]{0.75}FO}%
\colorbox{green}{\color[gray]{0.75}FO}%
\colorbox{green}{\color[gray]{0.75}FO}%
\colorbox{green}{\color[gray]{0.75}FO}%
\colorbox{green}{\color[gray]{0.75}FO}%
\colorbox{green}{\color[gray]{0.75}FO}%
\colorbox{green}{\color[gray]{0.75}FO}%
\colorbox{green}{\color[gray]{0.75}FO}%
\colorbox{green}{\color[gray]{0.75}FO}%
\colorbox{green}{\color[gray]{0.75}FO}%
\colorbox{green}{\color[gray]{0.75}FO}%
\colorbox{green}{\color[gray]{0.75}FO}%
\colorbox{green}{\color[gray]{0.75}FO}%
\colorbox{green}{\color[gray]{0.75}FO}%
\colorbox{green}{\color[gray]{0.75}FO}%
\colorbox{green}{\color[gray]{0.75}FO}%
\colorbox{green}{\color[gray]{0.75}FO}%
\colorbox{green}{\color[gray]{0.75}FO}%
\colorbox{green}{\color[gray]{0.75}FO}%
\colorbox{green}{\color[gray]{0.75}FO}%
\colorbox{green}{\color[gray]{0.75}FO}%
\colorbox{green}{\color[gray]{0.75}FO}%
\colorbox{green}{\color[gray]{0.75}FO}%
\colorbox{green}{\color[gray]{0.75}FO}%
\colorbox{green}{\color[gray]{0.75}FO}%
\colorbox{green}{\color[gray]{0.75}FO}%
\colorbox{green}{\color[gray]{0.75}FO}%
\colorbox{green}{\color[gray]{0.75}FO}%
\\
\colorbox{green}{\color[gray]{0.75}FO}%
\colorbox{green}{\color[gray]{0.75}FO}%
\colorbox{green}{\color[gray]{0.75}FO}%
\colorbox{green}{\color[gray]{0.75}FO}%
\colorbox{green}{\color[gray]{0.75}FO}%
\colorbox{green}{\color[gray]{0.75}FO}%
\colorbox{green}{\color[gray]{0.75}FO}%
\colorbox{green}{\color[gray]{0.75}FO}%
\colorbox{green}{\color[gray]{0.75}FO}%
\colorbox{green}{\color[gray]{0.75}FO}%
\colorbox{green}{\color[gray]{0.75}FO}%
\colorbox{green}{\color[gray]{0.75}FO}%
\colorbox{green}{\color[gray]{0.75}FO}%
\colorbox{green}{\color[gray]{0.75}FO}%
\colorbox{green}{\color[gray]{0.75}FO}%
\colorbox{green}{\color[gray]{0.75}FO}%
\colorbox{green}{\color[gray]{0.75}FO}%
\colorbox{green}{\color[gray]{0.75}FO}%
\colorbox{green}{\color[gray]{0.75}FO}%
\colorbox{green}{\color[gray]{0.75}FO}%
\colorbox{green}{\color[gray]{0.75}FO}%
\colorbox{green}{\color[gray]{0.75}FO}%
\colorbox{green}{\color[gray]{0.75}FO}%
\colorbox{green}{\color[gray]{0.75}FO}%
\colorbox{green}{\color[gray]{0.75}FO}%
\colorbox{green}{\color[gray]{0.75}FO}%
\colorbox{green}{\color[gray]{0.75}FO}%
\colorbox{green}{\color[gray]{0.75}FO}%
\colorbox{green}{\color[gray]{0.75}FO}%
\colorbox{green}{\color[gray]{0.75}FO}%
\colorbox{green}{\color[gray]{0.75}FO}%
\colorbox{green}{\color[gray]{0.75}FO}%
\colorbox{green}{\color[gray]{0.75}FO}%
\colorbox{green}{\color[gray]{0.75}FO}%
\colorbox{green}{\color[gray]{0.75}FO}%
\colorbox{green}{\color[gray]{0.75}FO}%
\colorbox{green}{\color[gray]{0.75}FO}%
\colorbox{green}{\color[gray]{0.75}FO}%
\colorbox{green}{\color[gray]{0.75}FO}%
\colorbox{green}{\color[gray]{0.75}FO}%
\colorbox{green}{\color[gray]{0.75}FO}%
\colorbox{green}{\color[gray]{0.75}FO}%
\colorbox{green}{\color[gray]{0.75}FO}%
\colorbox{green}{\color[gray]{0.75}FO}%
\colorbox{green}{\color[gray]{0.75}FO}%
\colorbox{green}{\color[gray]{0.75}FO}%
\colorbox{green}{\color[gray]{0.75}FO}%
\colorbox{green}{\color[gray]{0.75}FO}%
\colorbox{green}{\color[gray]{0.75}FO}%
\colorbox{green}{\color[gray]{0.75}FO}%
\colorbox{green}{\color[gray]{0.75}FO}%
\colorbox{green}{\color[gray]{0.75}FO}%
\colorbox{green}{\color[gray]{0.75}FO}%
\colorbox{green}{\color[gray]{0.75}FO}%
\colorbox{green}{\color[gray]{0.75}FO}%
\colorbox{green}{\color[gray]{0.75}FO}%
\colorbox{green}{\color[gray]{0.75}FO}%
\colorbox{green}{\color[gray]{0.75}FO}%
\colorbox{green}{\color[gray]{0.75}FO}%
\colorbox{green}{\color[gray]{0.75}FO}%
\colorbox{green}{\color[gray]{0.75}FO}%
\colorbox{green}{\color[gray]{0.75}FO}%
\colorbox{green}{\color[gray]{0.75}FO}%
\colorbox{green}{\color[gray]{0.75}FO}%
\colorbox{green}{\color[gray]{0.75}FO}%
\colorbox{green}{\color[gray]{0.75}FO}%
\colorbox{green}{\color[gray]{0.75}FO}%
\colorbox{green}{\color[gray]{0.75}FO}%
\colorbox{green}{\color[gray]{0.75}FO}%
\colorbox{green}{\color[gray]{0.75}FO}%
\colorbox{green}{\color[gray]{0.75}FO}%
\colorbox{green}{\color[gray]{0.75}FO}%
\colorbox{green}{\color[gray]{0.75}FO}%
\colorbox{green}{\color[gray]{0.75}FO}%
\colorbox{green}{\color[gray]{0.75}FO}%
\colorbox{green}{\color[gray]{0.75}FO}%
\colorbox{green}{\color[gray]{0.75}FO}%
\colorbox{green}{\color[gray]{0.75}FO}%
\colorbox{green}{\color[gray]{0.75}FO}%
\colorbox{green}{\color[gray]{0.75}FO}%
\colorbox{green}{\color[gray]{0.75}FO}%
\colorbox{green}{\color[gray]{0.75}FO}%
\colorbox{green}{\color[gray]{0.75}FO}%
\colorbox{green}{\color[gray]{0.75}FO}%
\colorbox{green}{\color[gray]{0.75}FO}%
\colorbox{green}{\color[gray]{0.75}FO}%
\colorbox{green}{\color[gray]{0.75}FO}%
\colorbox{green}{\color[gray]{0.75}FO}%
\colorbox{green}{\color[gray]{0.75}FO}%
\colorbox{green}{\color[gray]{0.75}FO}%
\colorbox{green}{\color[gray]{0.75}FO}%
\colorbox{green}{\color[gray]{0.75}FO}%
\colorbox{green}{\color[gray]{0.75}FO}%
\colorbox{green}{\color[gray]{0.75}FO}%
\colorbox{green}{\color[gray]{0.75}FO}%
\colorbox{green}{\color[gray]{0.75}FO}%
\colorbox{green}{\color[gray]{0.75}FO}%
\colorbox{green}{\color[gray]{0.75}FO}%
\colorbox{green}{\color[gray]{0.75}FO}%
\colorbox{green}{\color[gray]{0.75}FO}%
\\
\colorbox{green}{\color[gray]{0.75}FO}%
\colorbox{green}{\color[gray]{0.75}FO}%
\colorbox{green}{\color[gray]{0.75}FO}%
\colorbox{green}{\color[gray]{0.75}FO}%
\colorbox{green}{\color[gray]{0.75}FO}%
\colorbox{green}{\color[gray]{0.75}FO}%
\colorbox{green}{\color[gray]{0.75}FO}%
\colorbox{green}{\color[gray]{0.75}FO}%
\colorbox{green}{\color[gray]{0.75}FO}%
\colorbox{green}{\color[gray]{0.75}FO}%
\colorbox{green}{\color[gray]{0.75}FO}%
\colorbox{green}{\color[gray]{0.75}FO}%
\colorbox{green}{\color[gray]{0.75}FO}%
\colorbox{green}{\color[gray]{0.75}FO}%
\colorbox{green}{\color[gray]{0.75}FO}%
\colorbox{green}{\color[gray]{0.75}FO}%
\colorbox{green}{\color[gray]{0.75}FO}%
\colorbox{green}{\color[gray]{0.75}FO}%
\colorbox{green}{\color[gray]{0.75}FO}%
\colorbox{green}{\color[gray]{0.75}FO}%
\colorbox{green}{\color[gray]{0.75}FO}%
\colorbox{green}{\color[gray]{0.75}FO}%
\colorbox{green}{\color[gray]{0.75}FO}%
\colorbox{green}{\color[gray]{0.75}FO}%
\colorbox{green}{\color[gray]{0.75}FO}%
\colorbox{green}{\color[gray]{0.75}FO}%
\colorbox{green}{\color[gray]{0.75}FO}%
\colorbox{green}{\color[gray]{0.75}FO}%
\colorbox{green}{\color[gray]{0.75}FO}%
\colorbox{green}{\color[gray]{0.75}FO}%
\colorbox{green}{\color[gray]{0.75}FO}%
\colorbox{green}{\color[gray]{0.75}FO}%
\colorbox{green}{\color[gray]{0.75}FO}%
\colorbox{green}{\color[gray]{0.75}FO}%
\colorbox{green}{\color[gray]{0.75}FO}%
\colorbox{green}{\color[gray]{0.75}FO}%
\colorbox{green}{\color[gray]{0.75}FO}%
\colorbox{green}{\color[gray]{0.75}FO}%
\colorbox{green}{\color[gray]{0.75}FO}%
\colorbox{green}{\color[gray]{0.75}FO}%
\colorbox{green}{\color[gray]{0.75}FO}%
\colorbox{green}{\color[gray]{0.75}FO}%
\colorbox{green}{\color[gray]{0.75}FO}%
\colorbox{green}{\color[gray]{0.75}FO}%
\colorbox{green}{\color[gray]{0.75}FO}%
\colorbox{green}{\color[gray]{0.75}FO}%
\colorbox{green}{\color[gray]{0.75}FO}%
\colorbox{green}{\color[gray]{0.75}FO}%
\colorbox{green}{\color[gray]{0.75}FO}%
\colorbox{green}{\color[gray]{0.75}FO}%
\colorbox{green}{\color[gray]{0.75}FO}%
\colorbox{green}{\color[gray]{0.75}FO}%
\colorbox{green}{\color[gray]{0.75}FO}%
\colorbox{green}{\color[gray]{0.75}FO}%
\colorbox{green}{\color[gray]{0.75}FO}%
\colorbox{green}{\color[gray]{0.75}FO}%
\colorbox{green}{\color[gray]{0.75}FO}%
\colorbox{green}{\color[gray]{0.75}FO}%
\colorbox{green}{\color[gray]{0.75}FO}%
\colorbox{green}{\color[gray]{0.75}FO}%
\colorbox{green}{\color[gray]{0.75}FO}%
\colorbox{green}{\color[gray]{0.75}FO}%
\colorbox{green}{\color[gray]{0.75}FO}%
\colorbox{green}{\color[gray]{0.75}FO}%
\colorbox{green}{\color[gray]{0.75}FO}%
\colorbox{green}{\color[gray]{0.75}FO}%
\colorbox{green}{\color[gray]{0.75}FO}%
\colorbox{green}{\color[gray]{0.75}FO}%
\colorbox{green}{\color[gray]{0.75}FO}%
\colorbox{green}{\color[gray]{0.75}FO}%
\colorbox{green}{\color[gray]{0.75}FO}%
\colorbox{green}{\color[gray]{0.75}FO}%
\colorbox{green}{\color[gray]{0.75}FO}%
\colorbox{green}{\color[gray]{0.75}FO}%
\colorbox{green}{\color[gray]{0.75}FO}%
\colorbox{green}{\color[gray]{0.75}FO}%
\colorbox{green}{\color[gray]{0.75}FO}%
\colorbox{green}{\color[gray]{0.75}FO}%
\colorbox{green}{\color[gray]{0.75}FO}%
\colorbox{green}{\color[gray]{0.75}FO}%
\colorbox{green}{\color[gray]{0.75}FO}%
\colorbox{green}{\color[gray]{0.75}FO}%
\colorbox{green}{\color[gray]{0.75}FO}%
\colorbox{green}{\color[gray]{0.75}FO}%
\colorbox{green}{\color[gray]{0.75}FO}%
\colorbox{green}{\color[gray]{0.75}FO}%
\colorbox{green}{\color[gray]{0.75}FO}%
\colorbox{green}{\color[gray]{0.75}FO}%
\colorbox{green}{\color[gray]{0.75}FO}%
\colorbox{green}{\color[gray]{0.75}FO}%
\colorbox{green}{\color[gray]{0.75}FO}%
\colorbox{green}{\color[gray]{0.75}FO}%
\colorbox{green}{\color[gray]{0.75}FO}%
\colorbox{green}{\color[gray]{0.75}FO}%
\colorbox{green}{\color[gray]{0.75}FO}%
\colorbox{green}{\color[gray]{0.75}FO}%
\colorbox{green}{\color[gray]{0.75}FO}%
\colorbox{green}{\color[gray]{0.75}FO}%
\colorbox{green}{\color[gray]{0.75}FO}%
\colorbox{green}{\color[gray]{0.75}FO}%
\\
\colorbox{green}{\color[gray]{0.75}FO}%
\colorbox{green}{\color[gray]{0.75}FO}%
\colorbox{green}{\color[gray]{0.75}FO}%
\colorbox{green}{\color[gray]{0.75}FO}%
\colorbox{green}{\color[gray]{0.75}FO}%
\colorbox{green}{\color[gray]{0.75}FO}%
\colorbox{green}{\color[gray]{0.75}FO}%
\colorbox{green}{\color[gray]{0.75}FO}%
\colorbox{green}{\color[gray]{0.75}FO}%
\colorbox{green}{\color[gray]{0.75}FO}%
\colorbox{green}{\color[gray]{0.75}FO}%
\colorbox{green}{\color[gray]{0.75}FO}%
\colorbox{green}{\color[gray]{0.75}FO}%
\colorbox{green}{\color[gray]{0.75}FO}%
\colorbox{green}{\color[gray]{0.75}FO}%
\colorbox{green}{\color[gray]{0.75}FO}%
\colorbox{green}{\color[gray]{0.75}FO}%
\colorbox{green}{\color[gray]{0.75}FO}%
\colorbox{green}{\color[gray]{0.75}FO}%
\colorbox{green}{\color[gray]{0.75}FO}%
\colorbox{green}{\color[gray]{0.75}FO}%
\colorbox{green}{\color[gray]{0.75}FO}%
\colorbox{green}{\color[gray]{0.75}FO}%
\colorbox{green}{\color[gray]{0.75}FO}%
\colorbox{green}{\color[gray]{0.75}FO}%
\colorbox{green}{\color[gray]{0.75}FO}%
\colorbox{green}{\color[gray]{0.75}FO}%
\colorbox{green}{\color[gray]{0.75}FO}%
\colorbox{green}{\color[gray]{0.75}FO}%
\colorbox{green}{\color[gray]{0.75}FO}%
\colorbox{green}{\color[gray]{0.75}FO}%
\colorbox{green}{\color[gray]{0.75}FO}%
\colorbox{green}{\color[gray]{0.75}FO}%
\colorbox{green}{\color[gray]{0.75}FO}%
\colorbox{green}{\color[gray]{0.75}FO}%
\colorbox{green}{\color[gray]{0.75}FO}%
\colorbox{green}{\color[gray]{0.75}FO}%
\colorbox{green}{\color[gray]{0.75}FO}%
\colorbox{green}{\color[gray]{0.75}FO}%
\colorbox{green}{\color[gray]{0.75}FO}%
\colorbox{green}{\color[gray]{0.75}FO}%
\colorbox{green}{\color[gray]{0.75}FO}%
\colorbox{green}{\color[gray]{0.75}FO}%
\colorbox{green}{\color[gray]{0.75}FO}%
\colorbox{green}{\color[gray]{0.75}FO}%
\colorbox{green}{\color[gray]{0.75}FO}%
\colorbox{green}{\color[gray]{0.75}FO}%
\colorbox{green}{\color[gray]{0.75}FO}%
\colorbox{green}{\color[gray]{0.75}FO}%
\colorbox{green}{\color[gray]{0.75}FO}%
\colorbox{green}{\color[gray]{0.75}FO}%
\colorbox{green}{\color[gray]{0.75}FO}%
\colorbox{green}{\color[gray]{0.75}FO}%
\colorbox{green}{\color[gray]{0.75}FO}%
\colorbox{green}{\color[gray]{0.75}FO}%
\colorbox{green}{\color[gray]{0.75}FO}%
\colorbox{green}{\color[gray]{0.75}FO}%
\colorbox{green}{\color[gray]{0.75}FO}%
\colorbox{green}{\color[gray]{0.75}FO}%
\colorbox{green}{\color[gray]{0.75}FO}%
\colorbox{green}{\color[gray]{0.75}FO}%
\colorbox{green}{\color[gray]{0.75}FO}%
\colorbox{green}{\color[gray]{0.75}FO}%
\colorbox{green}{\color[gray]{0.75}FO}%
\colorbox{green}{\color[gray]{0.75}FO}%
\colorbox{green}{\color[gray]{0.75}FO}%
\colorbox{green}{\color[gray]{0.75}FO}%
\colorbox{green}{\color[gray]{0.75}FO}%
\colorbox{green}{\color[gray]{0.75}FO}%
\colorbox{green}{\color[gray]{0.75}FO}%
\colorbox{green}{\color[gray]{0.75}FO}%
\colorbox{green}{\color[gray]{0.75}FO}%
\colorbox{green}{\color[gray]{0.75}FO}%
\colorbox{green}{\color[gray]{0.75}FO}%
\colorbox{green}{\color[gray]{0.75}FO}%
\colorbox{green}{\color[gray]{0.75}FO}%
\colorbox{green}{\color[gray]{0.75}FO}%
\colorbox{green}{\color[gray]{0.75}FO}%
\colorbox{green}{\color[gray]{0.75}FO}%
\colorbox{green}{\color[gray]{0.75}FO}%
\colorbox{green}{\color[gray]{0.75}FO}%
\colorbox{green}{\color[gray]{0.75}FO}%
\colorbox{green}{\color[gray]{0.75}FO}%
\colorbox{green}{\color[gray]{0.75}FO}%
\colorbox{green}{\color[gray]{0.75}FO}%
\colorbox{green}{\color[gray]{0.75}FO}%
\colorbox{green}{\color[gray]{0.75}FO}%
\colorbox{green}{\color[gray]{0.75}FO}%
\colorbox{green}{\color[gray]{0.75}FO}%
\colorbox{green}{\color[gray]{0.75}FO}%
\colorbox{green}{\color[gray]{0.75}FO}%
\colorbox{green}{\color[gray]{0.75}FO}%
\colorbox{green}{\color[gray]{0.75}FO}%
\colorbox{green}{\color[gray]{0.75}FO}%
\colorbox{green}{\color[gray]{0.75}FO}%
\colorbox{green}{\color[gray]{0.75}FO}%
\colorbox{green}{\color[gray]{0.75}FO}%
\colorbox{green}{\color[gray]{0.75}FO}%
\colorbox{green}{\color[gray]{0.75}FO}%
\colorbox{green}{\color[gray]{0.75}FO}%
\\
\colorbox{green}{\color[gray]{0.75}FO}%
\colorbox{green}{\color[gray]{0.75}FO}%
\colorbox{green}{\color[gray]{0.75}FO}%
\colorbox{green}{\color[gray]{0.75}FO}%
\colorbox{green}{\color[gray]{0.75}FO}%
\colorbox{green}{\color[gray]{0.75}FO}%
\colorbox{green}{\color[gray]{0.75}FO}%
\colorbox{green}{\color[gray]{0.75}FO}%
\colorbox{green}{\color[gray]{0.75}FO}%
\colorbox{green}{\color[gray]{0.75}FO}%
\colorbox{green}{\color[gray]{0.75}FO}%
\colorbox{green}{\color[gray]{0.75}FO}%
\colorbox{green}{\color[gray]{0.75}FO}%
\colorbox{green}{\color[gray]{0.75}FO}%
\colorbox{green}{\color[gray]{0.75}FO}%
\colorbox{green}{\color[gray]{0.75}FO}%
\colorbox{green}{\color[gray]{0.75}FO}%
\colorbox{green}{\color[gray]{0.75}FO}%
\colorbox{green}{\color[gray]{0.75}FO}%
\colorbox{green}{\color[gray]{0.75}FO}%
\colorbox{green}{\color[gray]{0.75}FO}%
\colorbox{green}{\color[gray]{0.75}FO}%
\colorbox{green}{\color[gray]{0.75}FO}%
\colorbox{green}{\color[gray]{0.75}FO}%
\colorbox{green}{\color[gray]{0.75}FO}%
\colorbox{green}{\color[gray]{0.75}FO}%
\colorbox{green}{\color[gray]{0.75}FO}%
\colorbox{green}{\color[gray]{0.75}FO}%
\colorbox{green}{\color[gray]{0.75}FO}%
\colorbox{green}{\color[gray]{0.75}FO}%
\colorbox{green}{\color[gray]{0.75}FO}%
\colorbox{green}{\color[gray]{0.75}FO}%
\colorbox{green}{\color[gray]{0.75}FO}%
\colorbox{green}{\color[gray]{0.75}FO}%
\colorbox{green}{\color[gray]{0.75}FO}%
\colorbox{green}{\color[gray]{0.75}FO}%
\colorbox{green}{\color[gray]{0.75}FO}%
\colorbox{green}{\color[gray]{0.75}FO}%
\colorbox{green}{\color[gray]{0.75}FO}%
\colorbox{green}{\color[gray]{0.75}FO}%
\colorbox{green}{\color[gray]{0.75}FO}%
\colorbox{green}{\color[gray]{0.75}FO}%
\colorbox{green}{\color[gray]{0.75}FO}%
\colorbox{green}{\color[gray]{0.75}FO}%
\colorbox{green}{\color[gray]{0.75}FO}%
\colorbox{green}{\color[gray]{0.75}FO}%
\colorbox{green}{\color[gray]{0.75}FO}%
\colorbox{green}{\color[gray]{0.75}FO}%
\colorbox{green}{\color[gray]{0.75}FO}%
\colorbox{green}{\color[gray]{0.75}FO}%
\colorbox{green}{\color[gray]{0.75}FO}%
\colorbox{green}{\color[gray]{0.75}FO}%
\colorbox{green}{\color[gray]{0.75}FO}%
\colorbox{green}{\color[gray]{0.75}FO}%
\colorbox{green}{\color[gray]{0.75}FO}%
\colorbox{green}{\color[gray]{0.75}FO}%
\colorbox{green}{\color[gray]{0.75}FO}%
\colorbox{green}{\color[gray]{0.75}FO}%
\colorbox{green}{\color[gray]{0.75}FO}%
\colorbox{green}{\color[gray]{0.75}FO}%
\colorbox{green}{\color[gray]{0.75}FO}%
\colorbox{green}{\color[gray]{0.75}FO}%
\colorbox{green}{\color[gray]{0.75}FO}%
\colorbox{green}{\color[gray]{0.75}FO}%
\colorbox{green}{\color[gray]{0.75}FO}%
\colorbox{green}{\color[gray]{0.75}FO}%
\colorbox{green}{\color[gray]{0.75}FO}%
\colorbox{green}{\color[gray]{0.75}FO}%
\colorbox{green}{\color[gray]{0.75}FO}%
\colorbox{green}{\color[gray]{0.75}FO}%
\colorbox{green}{\color[gray]{0.75}FO}%
\colorbox{green}{\color[gray]{0.75}FO}%
\colorbox{green}{\color[gray]{0.75}FO}%
\colorbox{green}{\color[gray]{0.75}FO}%
\colorbox{green}{\color[gray]{0.75}FO}%
\colorbox{green}{\color[gray]{0.75}FO}%
\colorbox{green}{\color[gray]{0.75}FO}%
\colorbox{green}{\color[gray]{0.75}FO}%
\colorbox{green}{\color[gray]{0.75}FO}%
\colorbox{green}{\color[gray]{0.75}FO}%
\colorbox{green}{\color[gray]{0.75}FO}%
\colorbox{green}{\color[gray]{0.75}FO}%
\colorbox{green}{\color[gray]{0.75}FO}%
\colorbox{green}{\color[gray]{0.75}FO}%
\colorbox{green}{\color[gray]{0.75}FO}%
\colorbox{green}{\color[gray]{0.75}FO}%
\colorbox{green}{\color[gray]{0.75}FO}%
\colorbox{green}{\color[gray]{0.75}FO}%
\colorbox{green}{\color[gray]{0.75}FO}%
\colorbox{green}{\color[gray]{0.75}FO}%
\colorbox{green}{\color[gray]{0.75}FO}%
\colorbox{green}{\color[gray]{0.75}FO}%
\colorbox{green}{\color[gray]{0.75}FO}%
\colorbox{green}{\color[gray]{0.75}FO}%
\colorbox{green}{\color[gray]{0.75}FO}%
\colorbox{green}{\color[gray]{0.75}FO}%
\colorbox{green}{\color[gray]{0.75}FO}%
\colorbox{green}{\color[gray]{0.75}FO}%
\colorbox{green}{\color[gray]{0.75}FO}%
\colorbox{green}{\color[gray]{0.75}FO}%
\\
\colorbox{green}{\color[gray]{0.75}FO}%
\colorbox{green}{\color[gray]{0.75}FO}%
\colorbox{green}{\color[gray]{0.75}FO}%
\colorbox{green}{\color[gray]{0.75}FO}%
\colorbox{green}{\color[gray]{0.75}FO}%
\colorbox{green}{\color[gray]{0.75}FO}%
\colorbox{green}{\color[gray]{0.75}FO}%
\colorbox{green}{\color[gray]{0.75}FO}%
\colorbox{green}{\color[gray]{0.75}FO}%
\colorbox{green}{\color[gray]{0.75}FO}%
\colorbox{green}{\color[gray]{0.75}FO}%
\colorbox{green}{\color[gray]{0.75}FO}%
\colorbox{green}{\color[gray]{0.75}FO}%
\colorbox{green}{\color[gray]{0.75}FO}%
\colorbox{green}{\color[gray]{0.75}FO}%
\colorbox{green}{\color[gray]{0.75}FO}%
\colorbox{green}{\color[gray]{0.75}FO}%
\colorbox{green}{\color[gray]{0.75}FO}%
\colorbox{green}{\color[gray]{0.75}FO}%
\colorbox{green}{\color[gray]{0.75}FO}%
\colorbox{green}{\color[gray]{0.75}FO}%
\colorbox{green}{\color[gray]{0.75}FO}%
\colorbox{green}{\color[gray]{0.75}FO}%
\colorbox{green}{\color[gray]{0.75}FO}%
\colorbox{green}{\color[gray]{0.75}FO}%
\colorbox{green}{\color[gray]{0.75}FO}%
\colorbox{green}{\color[gray]{0.75}FO}%
\colorbox{green}{\color[gray]{0.75}FO}%
\colorbox{green}{\color[gray]{0.75}FO}%
\colorbox{green}{\color[gray]{0.75}FO}%
\colorbox{green}{\color[gray]{0.75}FO}%
\colorbox{green}{\color[gray]{0.75}FO}%
\colorbox{green}{\color[gray]{0.75}FO}%
\colorbox{green}{\color[gray]{0.75}FO}%
\colorbox{green}{\color[gray]{0.75}FO}%
\colorbox{green}{\color[gray]{0.75}FO}%
\colorbox{green}{\color[gray]{0.75}FO}%
\colorbox{green}{\color[gray]{0.75}FO}%
\colorbox{green}{\color[gray]{0.75}FO}%
\colorbox{green}{\color[gray]{0.75}FO}%
\colorbox{green}{\color[gray]{0.75}FO}%
\colorbox{green}{\color[gray]{0.75}FO}%
\colorbox{green}{\color[gray]{0.75}FO}%
\colorbox{green}{\color[gray]{0.75}FO}%
\colorbox{green}{\color[gray]{0.75}FO}%
\colorbox{green}{\color[gray]{0.75}FO}%
\colorbox{green}{\color[gray]{0.75}FO}%
\colorbox{green}{\color[gray]{0.75}FO}%
\colorbox{green}{\color[gray]{0.75}FO}%
\colorbox{green}{\color[gray]{0.75}FO}%
\colorbox{green}{\color[gray]{0.75}FO}%
\colorbox{green}{\color[gray]{0.75}FO}%
\colorbox{green}{\color[gray]{0.75}FO}%
\colorbox{green}{\color[gray]{0.75}FO}%
\colorbox{green}{\color[gray]{0.75}FO}%
\colorbox{green}{\color[gray]{0.75}FO}%
\colorbox{green}{\color[gray]{0.75}FO}%
\colorbox{green}{\color[gray]{0.75}FO}%
\colorbox{green}{\color[gray]{0.75}FO}%
\colorbox{green}{\color[gray]{0.75}FO}%
\colorbox{green}{\color[gray]{0.75}FO}%
\colorbox{green}{\color[gray]{0.75}FO}%
\colorbox{green}{\color[gray]{0.75}FO}%
\colorbox{green}{\color[gray]{0.75}FO}%
\colorbox{green}{\color[gray]{0.75}FO}%
\colorbox{green}{\color[gray]{0.75}FO}%
\colorbox{green}{\color[gray]{0.75}FO}%
\colorbox{green}{\color[gray]{0.75}FO}%
\colorbox{green}{\color[gray]{0.75}FO}%
\colorbox{green}{\color[gray]{0.75}FO}%
\colorbox{green}{\color[gray]{0.75}FO}%
\colorbox{green}{\color[gray]{0.75}FO}%
\colorbox{green}{\color[gray]{0.75}FO}%
\colorbox{green}{\color[gray]{0.75}FO}%
\colorbox{green}{\color[gray]{0.75}FO}%
\colorbox{green}{\color[gray]{0.75}FO}%
\colorbox{green}{\color[gray]{0.75}FO}%
\colorbox{green}{\color[gray]{0.75}FO}%
\colorbox{green}{\color[gray]{0.75}FO}%
\colorbox{green}{\color[gray]{0.75}FO}%
\colorbox{green}{\color[gray]{0.75}FO}%
\colorbox{green}{\color[gray]{0.75}FO}%
\colorbox{green}{\color[gray]{0.75}FO}%
\colorbox{green}{\color[gray]{0.75}FO}%
\colorbox{green}{\color[gray]{0.75}FO}%
\colorbox{green}{\color[gray]{0.75}FO}%
\colorbox{green}{\color[gray]{0.75}FO}%
\colorbox{green}{\color[gray]{0.75}FO}%
\colorbox{green}{\color[gray]{0.75}FO}%
\colorbox{green}{\color[gray]{0.75}FO}%
\colorbox{green}{\color[gray]{0.75}FO}%
\colorbox{green}{\color[gray]{0.75}FO}%
\colorbox{green}{\color[gray]{0.75}FO}%
\colorbox{green}{\color[gray]{0.75}FO}%
\colorbox{green}{\color[gray]{0.75}FO}%
\colorbox{green}{\color[gray]{0.75}FO}%
\colorbox{green}{\color[gray]{0.75}FO}%
\colorbox{green}{\color[gray]{0.75}FO}%
\colorbox{green}{\color[gray]{0.75}FO}%
\colorbox{green}{\color[gray]{0.75}FO}%
\\
\colorbox{green}{\color[gray]{0.75}FO}%
\colorbox{green}{\color[gray]{0.75}FO}%
\colorbox{green}{\color[gray]{0.75}FO}%
\colorbox{green}{\color[gray]{0.75}FO}%
\colorbox{green}{\color[gray]{0.75}FO}%
\colorbox{green}{\color[gray]{0.75}FO}%
\colorbox{green}{\color[gray]{0.75}FO}%
\colorbox{green}{\color[gray]{0.75}FO}%
\colorbox{green}{\color[gray]{0.75}FO}%
\colorbox{green}{\color[gray]{0.75}FO}%
\colorbox{green}{\color[gray]{0.75}FO}%
\colorbox{green}{\color[gray]{0.75}FO}%
\colorbox{green}{\color[gray]{0.75}FO}%
\colorbox{green}{\color[gray]{0.75}FO}%
\colorbox{green}{\color[gray]{0.75}FO}%
\colorbox{green}{\color[gray]{0.75}FO}%
\colorbox{green}{\color[gray]{0.75}FO}%
\colorbox{green}{\color[gray]{0.75}FO}%
\colorbox{green}{\color[gray]{0.75}FO}%
\colorbox{green}{\color[gray]{0.75}FO}%
\colorbox{green}{\color[gray]{0.75}FO}%
\colorbox{green}{\color[gray]{0.75}FO}%
\colorbox{green}{\color[gray]{0.75}FO}%
\colorbox{green}{\color[gray]{0.75}FO}%
\colorbox{green}{\color[gray]{0.75}FO}%
\colorbox{green}{\color[gray]{0.75}FO}%
\colorbox{green}{\color[gray]{0.75}FO}%
\colorbox{green}{\color[gray]{0.75}FO}%
\colorbox{green}{\color[gray]{0.75}FO}%
\colorbox{green}{\color[gray]{0.75}FO}%
\colorbox{green}{\color[gray]{0.75}FO}%
\colorbox{green}{\color[gray]{0.75}FO}%
\colorbox{green}{\color[gray]{0.75}FO}%
\colorbox{green}{\color[gray]{0.75}FO}%
\colorbox{green}{\color[gray]{0.75}FO}%
\colorbox{green}{\color[gray]{0.75}FO}%
\colorbox{green}{\color[gray]{0.75}FO}%
\colorbox{green}{\color[gray]{0.75}FO}%
\colorbox{green}{\color[gray]{0.75}FO}%
\colorbox{green}{\color[gray]{0.75}FO}%
\colorbox{green}{\color[gray]{0.75}FO}%
\colorbox{green}{\color[gray]{0.75}FO}%
\colorbox{green}{\color[gray]{0.75}FO}%
\colorbox{green}{\color[gray]{0.75}FO}%
\colorbox{green}{\color[gray]{0.75}FO}%
\colorbox{green}{\color[gray]{0.75}FO}%
\colorbox{green}{\color[gray]{0.75}FO}%
\colorbox{green}{\color[gray]{0.75}FO}%
\colorbox{green}{\color[gray]{0.75}FO}%
\colorbox{green}{\color[gray]{0.75}FO}%
\colorbox{green}{\color[gray]{0.75}FO}%
\colorbox{green}{\color[gray]{0.75}FO}%
\colorbox{green}{\color[gray]{0.75}FO}%
\colorbox{green}{\color[gray]{0.75}FO}%
\colorbox{green}{\color[gray]{0.75}FO}%
\colorbox{green}{\color[gray]{0.75}FO}%
\colorbox{green}{\color[gray]{0.75}FO}%
\colorbox{green}{\color[gray]{0.75}FO}%
\colorbox{green}{\color[gray]{0.75}FO}%
\colorbox{green}{\color[gray]{0.75}FO}%
\colorbox{green}{\color[gray]{0.75}FO}%
\colorbox{green}{\color[gray]{0.75}FO}%
\colorbox{green}{\color[gray]{0.75}FO}%
\colorbox{green}{\color[gray]{0.75}FO}%
\colorbox{green}{\color[gray]{0.75}FO}%
\colorbox{green}{\color[gray]{0.75}FO}%
\colorbox{green}{\color[gray]{0.75}FO}%
\colorbox{green}{\color[gray]{0.75}FO}%
\colorbox{green}{\color[gray]{0.75}FO}%
\colorbox{green}{\color[gray]{0.75}FO}%
\colorbox{green}{\color[gray]{0.75}FO}%
\colorbox{green}{\color[gray]{0.75}FO}%
\colorbox{green}{\color[gray]{0.75}FO}%
\colorbox{green}{\color[gray]{0.75}FO}%
\colorbox{green}{\color[gray]{0.75}FO}%
\colorbox{green}{\color[gray]{0.75}FO}%
\colorbox{green}{\color[gray]{0.75}FO}%
\colorbox{green}{\color[gray]{0.75}FO}%
\colorbox{green}{\color[gray]{0.75}FO}%
\colorbox{green}{\color[gray]{0.75}FO}%
\colorbox{green}{\color[gray]{0.75}FO}%
\colorbox{green}{\color[gray]{0.75}FO}%
\colorbox{green}{\color[gray]{0.75}FO}%
\colorbox{green}{\color[gray]{0.75}FO}%
\colorbox{green}{\color[gray]{0.75}FO}%
\colorbox{green}{\color[gray]{0.75}FO}%
\colorbox{green}{\color[gray]{0.75}FO}%
\colorbox{green}{\color[gray]{0.75}FO}%
\colorbox{green}{\color[gray]{0.75}FO}%
\colorbox{green}{\color[gray]{0.75}FO}%
\colorbox{green}{\color[gray]{0.75}FO}%
\colorbox{green}{\color[gray]{0.75}FO}%
\colorbox{green}{\color[gray]{0.75}FO}%
\colorbox{green}{\color[gray]{0.75}FO}%
\colorbox{green}{\color[gray]{0.75}FO}%
\colorbox{green}{\color[gray]{0.75}FO}%
\colorbox{green}{\color[gray]{0.75}FO}%
\colorbox{green}{\color[gray]{0.75}FO}%
\colorbox{green}{\color[gray]{0.75}FO}%
\colorbox{green}{\color[gray]{0.75}FO}%
\\
\colorbox{green}{\color[gray]{0.75}FO}%
\colorbox{green}{\color[gray]{0.75}FO}%
\colorbox{green}{\color[gray]{0.75}FO}%
\colorbox{green}{\color[gray]{0.75}FO}%
\colorbox{green}{\color[gray]{0.75}FO}%
\colorbox{green}{\color[gray]{0.75}FO}%
\colorbox{green}{\color[gray]{0.75}FO}%
\colorbox{green}{\color[gray]{0.75}FO}%
\colorbox{green}{\color[gray]{0.75}FO}%
\colorbox{green}{\color[gray]{0.75}FO}%
\colorbox{green}{\color[gray]{0.75}FO}%
\colorbox{green}{\color[gray]{0.75}FO}%
\colorbox{green}{\color[gray]{0.75}FO}%
\colorbox{green}{\color[gray]{0.75}FO}%
\colorbox{green}{\color[gray]{0.75}FO}%
\colorbox{green}{\color[gray]{0.75}FO}%
\colorbox{green}{\color[gray]{0.75}FO}%
\colorbox{green}{\color[gray]{0.75}FO}%
\colorbox{green}{\color[gray]{0.75}FO}%
\colorbox{green}{\color[gray]{0.75}FO}%
\colorbox{green}{\color[gray]{0.75}FO}%
\colorbox{green}{\color[gray]{0.75}FO}%
\colorbox{green}{\color[gray]{0.75}FO}%
\colorbox{green}{\color[gray]{0.75}FO}%
\colorbox{green}{\color[gray]{0.75}FO}%
\colorbox{green}{\color[gray]{0.75}FO}%
\colorbox{green}{\color[gray]{0.75}FO}%
\colorbox{green}{\color[gray]{0.75}FO}%
\colorbox{green}{\color[gray]{0.75}FO}%
\colorbox{green}{\color[gray]{0.75}FO}%
\colorbox{green}{\color[gray]{0.75}FO}%
\colorbox{green}{\color[gray]{0.75}FO}%
\colorbox{green}{\color[gray]{0.75}FO}%
\colorbox{green}{\color[gray]{0.75}FO}%
\colorbox{green}{\color[gray]{0.75}FO}%
\colorbox{green}{\color[gray]{0.75}FO}%
\colorbox{green}{\color[gray]{0.75}FO}%
\colorbox{green}{\color[gray]{0.75}FO}%
\colorbox{green}{\color[gray]{0.75}FO}%
\colorbox{green}{\color[gray]{0.75}FO}%
\colorbox{green}{\color[gray]{0.75}FO}%
\colorbox{green}{\color[gray]{0.75}FO}%
\colorbox{green}{\color[gray]{0.75}FO}%
\colorbox{green}{\color[gray]{0.75}FO}%
\colorbox{green}{\color[gray]{0.75}FO}%
\colorbox{green}{\color[gray]{0.75}FO}%
\colorbox{green}{\color[gray]{0.75}FO}%
\colorbox{green}{\color[gray]{0.75}FO}%
\colorbox{green}{\color[gray]{0.75}FO}%
\colorbox{green}{\color[gray]{0.75}FO}%
\colorbox{green}{\color[gray]{0.75}FO}%
\colorbox{green}{\color[gray]{0.75}FO}%
\colorbox{green}{\color[gray]{0.75}FO}%
\colorbox{green}{\color[gray]{0.75}FO}%
\colorbox{green}{\color[gray]{0.75}FO}%
\colorbox{green}{\color[gray]{0.75}FO}%
\colorbox{green}{\color[gray]{0.75}FO}%
\colorbox{green}{\color[gray]{0.75}FO}%
\colorbox{green}{\color[gray]{0.75}FO}%
\colorbox{green}{\color[gray]{0.75}FO}%
\colorbox{green}{\color[gray]{0.75}FO}%
\colorbox{green}{\color[gray]{0.75}FO}%
\colorbox{green}{\color[gray]{0.75}FO}%
\colorbox{green}{\color[gray]{0.75}FO}%
\colorbox{green}{\color[gray]{0.75}FO}%
\colorbox{green}{\color[gray]{0.75}FO}%
\colorbox{green}{\color[gray]{0.75}FO}%
\colorbox{green}{\color[gray]{0.75}FO}%
\colorbox{green}{\color[gray]{0.75}FO}%
\colorbox{green}{\color[gray]{0.75}FO}%
\colorbox{green}{\color[gray]{0.75}FO}%
\colorbox{green}{\color[gray]{0.75}FO}%
\colorbox{green}{\color[gray]{0.75}FO}%
\colorbox{green}{\color[gray]{0.75}FO}%
\colorbox{green}{\color[gray]{0.75}FO}%
\colorbox{green}{\color[gray]{0.75}FO}%
\colorbox{green}{\color[gray]{0.75}FO}%
\colorbox{green}{\color[gray]{0.75}FO}%
\colorbox{green}{\color[gray]{0.75}FO}%
\colorbox{green}{\color[gray]{0.75}FO}%
\colorbox{green}{\color[gray]{0.75}FO}%
\colorbox{green}{\color[gray]{0.75}FO}%
\colorbox{green}{\color[gray]{0.75}FO}%
\colorbox{green}{\color[gray]{0.75}FO}%
\colorbox{green}{\color[gray]{0.75}FO}%
\colorbox{green}{\color[gray]{0.75}FO}%
\colorbox{green}{\color[gray]{0.75}FO}%
\colorbox{green}{\color[gray]{0.75}FO}%
\colorbox{green}{\color[gray]{0.75}FO}%
\colorbox{green}{\color[gray]{0.75}FO}%
\colorbox{green}{\color[gray]{0.75}FO}%
\colorbox{green}{\color[gray]{0.75}FO}%
\colorbox{green}{\color[gray]{0.75}FO}%
\colorbox{green}{\color[gray]{0.75}FO}%
\colorbox{green}{\color[gray]{0.75}FO}%
\colorbox{green}{\color[gray]{0.75}FO}%
\colorbox{green}{\color[gray]{0.75}FO}%
\colorbox{green}{\color[gray]{0.75}FO}%
\colorbox{green}{\color[gray]{0.75}FO}%
\colorbox{green}{\color[gray]{0.75}FO}%
\\
\colorbox{green}{\color[gray]{0.75}FO}%
\colorbox{green}{\color[gray]{0.75}FO}%
\colorbox{green}{\color[gray]{0.75}FO}%
\colorbox{green}{\color[gray]{0.75}FO}%
\colorbox{green}{\color[gray]{0.75}FO}%
\colorbox{green}{\color[gray]{0.75}FO}%
\colorbox{green}{\color[gray]{0.75}FO}%
\colorbox{green}{\color[gray]{0.75}FO}%
\colorbox{green}{\color[gray]{0.75}FO}%
\colorbox{green}{\color[gray]{0.75}FO}%
\colorbox{green}{\color[gray]{0.75}FO}%
\colorbox{green}{\color[gray]{0.75}FO}%
\colorbox{green}{\color[gray]{0.75}FO}%
\colorbox{green}{\color[gray]{0.75}FO}%
\colorbox{green}{\color[gray]{0.75}FO}%
\colorbox{green}{\color[gray]{0.75}FO}%
\colorbox{green}{\color[gray]{0.75}FO}%
\colorbox{green}{\color[gray]{0.75}FO}%
\colorbox{green}{\color[gray]{0.75}FO}%
\colorbox{green}{\color[gray]{0.75}FO}%
\colorbox{green}{\color[gray]{0.75}FO}%
\colorbox{green}{\color[gray]{0.75}FO}%
\colorbox{green}{\color[gray]{0.75}FO}%
\colorbox{green}{\color[gray]{0.75}FO}%
\colorbox{green}{\color[gray]{0.75}FO}%
\colorbox{green}{\color[gray]{0.75}FO}%
\colorbox{green}{\color[gray]{0.75}FO}%
\colorbox{green}{\color[gray]{0.75}FO}%
\colorbox{green}{\color[gray]{0.75}FO}%
\colorbox{green}{\color[gray]{0.75}FO}%
\colorbox{green}{\color[gray]{0.75}FO}%
\colorbox{green}{\color[gray]{0.75}FO}%
\colorbox{green}{\color[gray]{0.75}FO}%
\colorbox{green}{\color[gray]{0.75}FO}%
\colorbox{green}{\color[gray]{0.75}FO}%
\colorbox{green}{\color[gray]{0.75}FO}%
\colorbox{green}{\color[gray]{0.75}FO}%
\colorbox{green}{\color[gray]{0.75}FO}%
\colorbox{green}{\color[gray]{0.75}FO}%
\colorbox{green}{\color[gray]{0.75}FO}%
\colorbox{green}{\color[gray]{0.75}FO}%
\colorbox{green}{\color[gray]{0.75}FO}%
\colorbox{green}{\color[gray]{0.75}FO}%
\colorbox{green}{\color[gray]{0.75}FO}%
\colorbox{green}{\color[gray]{0.75}FO}%
\colorbox{green}{\color[gray]{0.75}FO}%
\colorbox{green}{\color[gray]{0.75}FO}%
\colorbox{green}{\color[gray]{0.75}FO}%
\colorbox{green}{\color[gray]{0.75}FO}%
\colorbox{green}{\color[gray]{0.75}FO}%
\colorbox{green}{\color[gray]{0.75}FO}%
\colorbox{green}{\color[gray]{0.75}FO}%
\colorbox{green}{\color[gray]{0.75}FO}%
\colorbox{green}{\color[gray]{0.75}FO}%
\colorbox{green}{\color[gray]{0.75}FO}%
\colorbox{green}{\color[gray]{0.75}FO}%
\colorbox{green}{\color[gray]{0.75}FO}%
\colorbox{green}{\color[gray]{0.75}FO}%
\colorbox{green}{\color[gray]{0.75}FO}%
\colorbox{green}{\color[gray]{0.75}FO}%
\colorbox{green}{\color[gray]{0.75}FO}%
\colorbox{green}{\color[gray]{0.75}FO}%
\colorbox{green}{\color[gray]{0.75}FO}%
\colorbox{green}{\color[gray]{0.75}FO}%
\colorbox{green}{\color[gray]{0.75}FO}%
\colorbox{green}{\color[gray]{0.75}FO}%
\colorbox{green}{\color[gray]{0.75}FO}%
\colorbox{green}{\color[gray]{0.75}FO}%
\colorbox{green}{\color[gray]{0.75}FO}%
\colorbox{green}{\color[gray]{0.75}FO}%
\colorbox{green}{\color[gray]{0.75}FO}%
\colorbox{green}{\color[gray]{0.75}FO}%
\colorbox{green}{\color[gray]{0.75}FO}%
\colorbox{green}{\color[gray]{0.75}FO}%
\colorbox{green}{\color[gray]{0.75}FO}%
\colorbox{green}{\color[gray]{0.75}FO}%
\colorbox{green}{\color[gray]{0.75}FO}%
\colorbox{green}{\color[gray]{0.75}FO}%
\colorbox{green}{\color[gray]{0.75}FO}%
\colorbox{green}{\color[gray]{0.75}FO}%
\colorbox{green}{\color[gray]{0.75}FO}%
\colorbox{green}{\color[gray]{0.75}FO}%
\colorbox{green}{\color[gray]{0.75}FO}%
\colorbox{green}{\color[gray]{0.75}FO}%
\colorbox{green}{\color[gray]{0.75}FO}%
\colorbox{green}{\color[gray]{0.75}FO}%
\colorbox{green}{\color[gray]{0.75}FO}%
\colorbox{green}{\color[gray]{0.75}FO}%
\colorbox{green}{\color[gray]{0.75}FO}%
\colorbox{green}{\color[gray]{0.75}FO}%
\colorbox{green}{\color[gray]{0.75}FO}%
\colorbox{green}{\color[gray]{0.75}FO}%
\colorbox{green}{\color[gray]{0.75}FO}%
\colorbox{green}{\color[gray]{0.75}FO}%
\colorbox{green}{\color[gray]{0.75}FO}%
\colorbox{green}{\color[gray]{0.75}FO}%
\colorbox{green}{\color[gray]{0.75}FO}%
\colorbox{green}{\color[gray]{0.75}FO}%
\colorbox{green}{\color[gray]{0.75}FO}%
\colorbox{green}{\color[gray]{0.75}FO}%
\\
\colorbox{green}{\color[gray]{0.75}FO}%
\colorbox{green}{\color[gray]{0.75}FO}%
\colorbox{green}{\color[gray]{0.75}FO}%
\colorbox{green}{\color[gray]{0.75}FO}%
\colorbox{green}{\color[gray]{0.75}FO}%
\colorbox{green}{\color[gray]{0.75}FO}%
\colorbox{green}{\color[gray]{0.75}FO}%
\colorbox{green}{\color[gray]{0.75}FO}%
\colorbox{green}{\color[gray]{0.75}FO}%
\colorbox{green}{\color[gray]{0.75}FO}%
\colorbox{green}{\color[gray]{0.75}FO}%
\colorbox{green}{\color[gray]{0.75}FO}%
\colorbox{green}{\color[gray]{0.75}FO}%
\colorbox{green}{\color[gray]{0.75}FO}%
\colorbox{green}{\color[gray]{0.75}FO}%
\colorbox{green}{\color[gray]{0.75}FO}%
\colorbox{green}{\color[gray]{0.75}FO}%
\colorbox{green}{\color[gray]{0.75}FO}%
\colorbox{green}{\color[gray]{0.75}FO}%
\colorbox{green}{\color[gray]{0.75}FO}%
\colorbox{green}{\color[gray]{0.75}FO}%
\colorbox{green}{\color[gray]{0.75}FO}%
\colorbox{green}{\color[gray]{0.75}FO}%
\colorbox{green}{\color[gray]{0.75}FO}%
\colorbox{green}{\color[gray]{0.75}FO}%
\colorbox{green}{\color[gray]{0.75}FO}%
\colorbox{green}{\color[gray]{0.75}FO}%
\colorbox{green}{\color[gray]{0.75}FO}%
\colorbox{green}{\color[gray]{0.75}FO}%
\colorbox{green}{\color[gray]{0.75}FO}%
\colorbox{green}{\color[gray]{0.75}FO}%
\colorbox{green}{\color[gray]{0.75}FO}%
\colorbox{green}{\color[gray]{0.75}FO}%
\colorbox{green}{\color[gray]{0.75}FO}%
\colorbox{green}{\color[gray]{0.75}FO}%
\colorbox{green}{\color[gray]{0.75}FO}%
\colorbox{green}{\color[gray]{0.75}FO}%
\colorbox{green}{\color[gray]{0.75}FO}%
\colorbox{green}{\color[gray]{0.75}FO}%
\colorbox{green}{\color[gray]{0.75}FO}%
\colorbox{green}{\color[gray]{0.75}FO}%
\colorbox{green}{\color[gray]{0.75}FO}%
\colorbox{green}{\color[gray]{0.75}FO}%
\colorbox{green}{\color[gray]{0.75}FO}%
\colorbox{green}{\color[gray]{0.75}FO}%
\colorbox{green}{\color[gray]{0.75}FO}%
\colorbox{green}{\color[gray]{0.75}FO}%
\colorbox{green}{\color[gray]{0.75}FO}%
\colorbox{green}{\color[gray]{0.75}FO}%
\colorbox{green}{\color[gray]{0.75}FO}%
\colorbox{green}{\color[gray]{0.75}FO}%
\colorbox{green}{\color[gray]{0.75}FO}%
\colorbox{green}{\color[gray]{0.75}FO}%
\colorbox{green}{\color[gray]{0.75}FO}%
\colorbox{green}{\color[gray]{0.75}FO}%
\colorbox{green}{\color[gray]{0.75}FO}%
\colorbox{green}{\color[gray]{0.75}FO}%
\colorbox{green}{\color[gray]{0.75}FO}%
\colorbox{green}{\color[gray]{0.75}FO}%
\colorbox{green}{\color[gray]{0.75}FO}%
\colorbox{green}{\color[gray]{0.75}FO}%
\colorbox{green}{\color[gray]{0.75}FO}%
\colorbox{green}{\color[gray]{0.75}FO}%
\colorbox{green}{\color[gray]{0.75}FO}%
\colorbox{green}{\color[gray]{0.75}FO}%
\colorbox{green}{\color[gray]{0.75}FO}%
\colorbox{green}{\color[gray]{0.75}FO}%
\colorbox{green}{\color[gray]{0.75}FO}%
\colorbox{green}{\color[gray]{0.75}FO}%
\colorbox{green}{\color[gray]{0.75}FO}%
\colorbox{green}{\color[gray]{0.75}FO}%
\colorbox{green}{\color[gray]{0.75}FO}%
\colorbox{green}{\color[gray]{0.75}FO}%
\colorbox{green}{\color[gray]{0.75}FO}%
\colorbox{green}{\color[gray]{0.75}FO}%
\colorbox{green}{\color[gray]{0.75}FO}%
\colorbox{green}{\color[gray]{0.75}FO}%
\colorbox{green}{\color[gray]{0.75}FO}%
\colorbox{green}{\color[gray]{0.75}FO}%
\colorbox{green}{\color[gray]{0.75}FO}%
\colorbox{green}{\color[gray]{0.75}FO}%
\colorbox{green}{\color[gray]{0.75}FO}%
\colorbox{green}{\color[gray]{0.75}FO}%
\colorbox{green}{\color[gray]{0.75}FO}%
\colorbox{green}{\color[gray]{0.75}FO}%
\colorbox{green}{\color[gray]{0.75}FO}%
\colorbox{green}{\color[gray]{0.75}FO}%
\colorbox{green}{\color[gray]{0.75}FO}%
\colorbox{green}{\color[gray]{0.75}FO}%
\colorbox{green}{\color[gray]{0.75}FO}%
\colorbox{green}{\color[gray]{0.75}FO}%
\colorbox{green}{\color[gray]{0.75}FO}%
\colorbox{green}{\color[gray]{0.75}FO}%
\colorbox{green}{\color[gray]{0.75}FO}%
\colorbox{green}{\color[gray]{0.75}FO}%
\colorbox{green}{\color[gray]{0.75}FO}%
\colorbox{green}{\color[gray]{0.75}FO}%
\colorbox{green}{\color[gray]{0.75}FO}%
\colorbox{green}{\color[gray]{0.75}FO}%
\colorbox{green}{\color[gray]{0.75}FO}%
\\
\colorbox{green}{\color[gray]{0.75}FO}%
\colorbox{green}{\color[gray]{0.75}FO}%
\colorbox{green}{\color[gray]{0.75}FO}%
\colorbox{green}{\color[gray]{0.75}FO}%
\colorbox{green}{\color[gray]{0.75}FO}%
\colorbox{green}{\color[gray]{0.75}FO}%
\colorbox{green}{\color[gray]{0.75}FO}%
\colorbox{green}{\color[gray]{0.75}FO}%
\colorbox{green}{\color[gray]{0.75}FO}%
\colorbox{green}{\color[gray]{0.75}FO}%
\colorbox{green}{\color[gray]{0.75}FO}%
\colorbox{green}{\color[gray]{0.75}FO}%
\colorbox{green}{\color[gray]{0.75}FO}%
\colorbox{green}{\color[gray]{0.75}FO}%
\colorbox{green}{\color[gray]{0.75}FO}%
\colorbox{green}{\color[gray]{0.75}FO}%
\colorbox{green}{\color[gray]{0.75}FO}%
\colorbox{green}{\color[gray]{0.75}FO}%
\colorbox{green}{\color[gray]{0.75}FO}%
\colorbox{green}{\color[gray]{0.75}FO}%
\colorbox{green}{\color[gray]{0.75}FO}%
\colorbox{green}{\color[gray]{0.75}FO}%
\colorbox{green}{\color[gray]{0.75}FO}%
\colorbox{green}{\color[gray]{0.75}FO}%
\colorbox{green}{\color[gray]{0.75}FO}%
\colorbox{green}{\color[gray]{0.75}FO}%
\colorbox{green}{\color[gray]{0.75}FO}%
\colorbox{green}{\color[gray]{0.75}FO}%
\colorbox{green}{\color[gray]{0.75}FO}%
\colorbox{green}{\color[gray]{0.75}FO}%
\colorbox{green}{\color[gray]{0.75}FO}%
\colorbox{green}{\color[gray]{0.75}FO}%
\colorbox{green}{\color[gray]{0.75}FO}%
\colorbox{green}{\color[gray]{0.75}FO}%
\colorbox{green}{\color[gray]{0.75}FO}%
\colorbox{green}{\color[gray]{0.75}FO}%
\colorbox{green}{\color[gray]{0.75}FO}%
\colorbox{green}{\color[gray]{0.75}FO}%
\colorbox{green}{\color[gray]{0.75}FO}%
\colorbox{green}{\color[gray]{0.75}FO}%
\colorbox{green}{\color[gray]{0.75}FO}%
\colorbox{green}{\color[gray]{0.75}FO}%
\colorbox{green}{\color[gray]{0.75}FO}%
\colorbox{green}{\color[gray]{0.75}FO}%
\colorbox{green}{\color[gray]{0.75}FO}%
\colorbox{green}{\color[gray]{0.75}FO}%
\colorbox{green}{\color[gray]{0.75}FO}%
\colorbox{green}{\color[gray]{0.75}FO}%
\colorbox{green}{\color[gray]{0.75}FO}%
\colorbox{green}{\color[gray]{0.75}FO}%
\colorbox{green}{\color[gray]{0.75}FO}%
\colorbox{green}{\color[gray]{0.75}FO}%
\colorbox{green}{\color[gray]{0.75}FO}%
\colorbox{green}{\color[gray]{0.75}FO}%
\colorbox{green}{\color[gray]{0.75}FO}%
\colorbox{green}{\color[gray]{0.75}FO}%
\colorbox{green}{\color[gray]{0.75}FO}%
\colorbox{green}{\color[gray]{0.75}FO}%
\colorbox{green}{\color[gray]{0.75}FO}%
\colorbox{green}{\color[gray]{0.75}FO}%
\colorbox{green}{\color[gray]{0.75}FO}%
\colorbox{green}{\color[gray]{0.75}FO}%
\colorbox{green}{\color[gray]{0.75}FO}%
\colorbox{green}{\color[gray]{0.75}FO}%
\colorbox{green}{\color[gray]{0.75}FO}%
\colorbox{green}{\color[gray]{0.75}FO}%
\colorbox{green}{\color[gray]{0.75}FO}%
\colorbox{green}{\color[gray]{0.75}FO}%
\colorbox{green}{\color[gray]{0.75}FO}%
\colorbox{green}{\color[gray]{0.75}FO}%
\colorbox{green}{\color[gray]{0.75}FO}%
\colorbox{green}{\color[gray]{0.75}FO}%
\colorbox{green}{\color[gray]{0.75}FO}%
\colorbox{green}{\color[gray]{0.75}FO}%
\colorbox{green}{\color[gray]{0.75}FO}%
\colorbox{green}{\color[gray]{0.75}FO}%
\colorbox{green}{\color[gray]{0.75}FO}%
\colorbox{green}{\color[gray]{0.75}FO}%
\colorbox{green}{\color[gray]{0.75}FO}%
\colorbox{green}{\color[gray]{0.75}FO}%
\colorbox{green}{\color[gray]{0.75}FO}%
\colorbox{green}{\color[gray]{0.75}FO}%
\colorbox{green}{\color[gray]{0.75}FO}%
\colorbox{green}{\color[gray]{0.75}FO}%
\colorbox{green}{\color[gray]{0.75}FO}%
\colorbox{green}{\color[gray]{0.75}FO}%
\colorbox{green}{\color[gray]{0.75}FO}%
\colorbox{green}{\color[gray]{0.75}FO}%
\colorbox{green}{\color[gray]{0.75}FO}%
\colorbox{green}{\color[gray]{0.75}FO}%
\colorbox{green}{\color[gray]{0.75}FO}%
\colorbox{green}{\color[gray]{0.75}FO}%
\colorbox{green}{\color[gray]{0.75}FO}%
\colorbox{green}{\color[gray]{0.75}FO}%
\colorbox{green}{\color[gray]{0.75}FO}%
\colorbox{green}{\color[gray]{0.75}FO}%
\colorbox{green}{\color[gray]{0.75}FO}%
\colorbox{green}{\color[gray]{0.75}FO}%
\colorbox{green}{\color[gray]{0.75}FO}%
\colorbox{green}{\color[gray]{0.75}FO}%
\\
\colorbox{green}{\color[gray]{0.75}FO}%
\colorbox{green}{\color[gray]{0.75}FO}%
\colorbox{green}{\color[gray]{0.75}FO}%
\colorbox{green}{\color[gray]{0.75}FO}%
\colorbox{green}{\color[gray]{0.75}FO}%
\colorbox{green}{\color[gray]{0.75}FO}%
\colorbox{green}{\color[gray]{0.75}FO}%
\colorbox{green}{\color[gray]{0.75}FO}%
\colorbox{green}{\color[gray]{0.75}FO}%
\colorbox{green}{\color[gray]{0.75}FO}%
\colorbox{green}{\color[gray]{0.75}FO}%
\colorbox{green}{\color[gray]{0.75}FO}%
\colorbox{green}{\color[gray]{0.75}FO}%
\colorbox{green}{\color[gray]{0.75}FO}%
\colorbox{green}{\color[gray]{0.75}FO}%
\colorbox{green}{\color[gray]{0.75}FO}%
\colorbox{green}{\color[gray]{0.75}FO}%
\colorbox{green}{\color[gray]{0.75}FO}%
\colorbox{green}{\color[gray]{0.75}FO}%
\colorbox{green}{\color[gray]{0.75}FO}%
\colorbox{green}{\color[gray]{0.75}FO}%
\colorbox{green}{\color[gray]{0.75}FO}%
\colorbox{green}{\color[gray]{0.75}FO}%
\colorbox{green}{\color[gray]{0.75}FO}%
\colorbox{green}{\color[gray]{0.75}FO}%
\colorbox{green}{\color[gray]{0.75}FO}%
\colorbox{green}{\color[gray]{0.75}FO}%
\colorbox{green}{\color[gray]{0.75}FO}%
\colorbox{green}{\color[gray]{0.75}FO}%
\colorbox{green}{\color[gray]{0.75}FO}%
\colorbox{green}{\color[gray]{0.75}FO}%
\colorbox{green}{\color[gray]{0.75}FO}%
\colorbox{green}{\color[gray]{0.75}FO}%
\colorbox{green}{\color[gray]{0.75}FO}%
\colorbox{green}{\color[gray]{0.75}FO}%
\colorbox{green}{\color[gray]{0.75}FO}%
\colorbox{green}{\color[gray]{0.75}FO}%
\colorbox{green}{\color[gray]{0.75}FO}%
\colorbox{green}{\color[gray]{0.75}FO}%
\colorbox{green}{\color[gray]{0.75}FO}%
\colorbox{green}{\color[gray]{0.75}FO}%
\colorbox{green}{\color[gray]{0.75}FO}%
\colorbox{green}{\color[gray]{0.75}FO}%
\colorbox{green}{\color[gray]{0.75}FO}%
\colorbox{green}{\color[gray]{0.75}FO}%
\colorbox{green}{\color[gray]{0.75}FO}%
\colorbox{green}{\color[gray]{0.75}FO}%
\colorbox{green}{\color[gray]{0.75}FO}%
\colorbox{green}{\color[gray]{0.75}FO}%
\colorbox{green}{\color[gray]{0.75}FO}%
\colorbox{green}{\color[gray]{0.75}FO}%
\colorbox{green}{\color[gray]{0.75}FO}%
\colorbox{green}{\color[gray]{0.75}FO}%
\colorbox{green}{\color[gray]{0.75}FO}%
\colorbox{green}{\color[gray]{0.75}FO}%
\colorbox{green}{\color[gray]{0.75}FO}%
\colorbox{green}{\color[gray]{0.75}FO}%
\colorbox{green}{\color[gray]{0.75}FO}%
\colorbox{green}{\color[gray]{0.75}FO}%
\colorbox{green}{\color[gray]{0.75}FO}%
\colorbox{green}{\color[gray]{0.75}FO}%
\colorbox{green}{\color[gray]{0.75}FO}%
\colorbox{green}{\color[gray]{0.75}FO}%
\colorbox{green}{\color[gray]{0.75}FO}%
\colorbox{green}{\color[gray]{0.75}FO}%
\colorbox{green}{\color[gray]{0.75}FO}%
\colorbox{green}{\color[gray]{0.75}FO}%
\colorbox{green}{\color[gray]{0.75}FO}%
\colorbox{green}{\color[gray]{0.75}FO}%
\colorbox{green}{\color[gray]{0.75}FO}%
\colorbox{green}{\color[gray]{0.75}FO}%
\colorbox{green}{\color[gray]{0.75}FO}%
\colorbox{green}{\color[gray]{0.75}FO}%
\colorbox{green}{\color[gray]{0.75}FO}%
\colorbox{green}{\color[gray]{0.75}FO}%
\colorbox{green}{\color[gray]{0.75}FO}%
\colorbox{green}{\color[gray]{0.75}FO}%
\colorbox{green}{\color[gray]{0.75}FO}%
\colorbox{green}{\color[gray]{0.75}FO}%
\colorbox{green}{\color[gray]{0.75}FO}%
\colorbox{green}{\color[gray]{0.75}FO}%
\colorbox{green}{\color[gray]{0.75}FO}%
\colorbox{green}{\color[gray]{0.75}FO}%
\colorbox{green}{\color[gray]{0.75}FO}%
\colorbox{green}{\color[gray]{0.75}FO}%
\colorbox{green}{\color[gray]{0.75}FO}%
\colorbox{green}{\color[gray]{0.75}FO}%
\colorbox{green}{\color[gray]{0.75}FO}%
\colorbox{green}{\color[gray]{0.75}FO}%
\colorbox{green}{\color[gray]{0.75}FO}%
\colorbox{green}{\color[gray]{0.75}FO}%
\colorbox{green}{\color[gray]{0.75}FO}%
\colorbox{green}{\color[gray]{0.75}FO}%
\colorbox{green}{\color[gray]{0.75}FO}%
\colorbox{green}{\color[gray]{0.75}FO}%
\colorbox{green}{\color[gray]{0.75}FO}%
\colorbox{green}{\color[gray]{0.75}FO}%
\colorbox{green}{\color[gray]{0.75}FO}%
\colorbox{green}{\color[gray]{0.75}FO}%
\colorbox{green}{\color[gray]{0.75}FO}%
\\
\colorbox{green}{\color[gray]{0.75}FO}%
\colorbox{green}{\color[gray]{0.75}FO}%
\colorbox{green}{\color[gray]{0.75}FO}%
\colorbox{green}{\color[gray]{0.75}FO}%
\colorbox{green}{\color[gray]{0.75}FO}%
\colorbox{green}{\color[gray]{0.75}FO}%
\colorbox{green}{\color[gray]{0.75}FO}%
\colorbox{green}{\color[gray]{0.75}FO}%
\colorbox{green}{\color[gray]{0.75}FO}%
\colorbox{green}{\color[gray]{0.75}FO}%
\colorbox{green}{\color[gray]{0.75}FO}%
\colorbox{green}{\color[gray]{0.75}FO}%
\colorbox{green}{\color[gray]{0.75}FO}%
\colorbox{green}{\color[gray]{0.75}FO}%
\colorbox{green}{\color[gray]{0.75}FO}%
\colorbox{green}{\color[gray]{0.75}FO}%
\colorbox{green}{\color[gray]{0.75}FO}%
\colorbox{green}{\color[gray]{0.75}FO}%
\colorbox{green}{\color[gray]{0.75}FO}%
\colorbox{green}{\color[gray]{0.75}FO}%
\colorbox{green}{\color[gray]{0.75}FO}%
\colorbox{green}{\color[gray]{0.75}FO}%
\colorbox{green}{\color[gray]{0.75}FO}%
\colorbox{green}{\color[gray]{0.75}FO}%
\colorbox{green}{\color[gray]{0.75}FO}%
\colorbox{green}{\color[gray]{0.75}FO}%
\colorbox{green}{\color[gray]{0.75}FO}%
\colorbox{green}{\color[gray]{0.75}FO}%
\colorbox{green}{\color[gray]{0.75}FO}%
\colorbox{green}{\color[gray]{0.75}FO}%
\colorbox{green}{\color[gray]{0.75}FO}%
\colorbox{green}{\color[gray]{0.75}FO}%
\colorbox{green}{\color[gray]{0.75}FO}%
\colorbox{green}{\color[gray]{0.75}FO}%
\colorbox{green}{\color[gray]{0.75}FO}%
\colorbox{green}{\color[gray]{0.75}FO}%
\colorbox{green}{\color[gray]{0.75}FO}%
\colorbox{green}{\color[gray]{0.75}FO}%
\colorbox{green}{\color[gray]{0.75}FO}%
\colorbox{green}{\color[gray]{0.75}FO}%
\colorbox{green}{\color[gray]{0.75}FO}%
\colorbox{green}{\color[gray]{0.75}FO}%
\colorbox{green}{\color[gray]{0.75}FO}%
\colorbox{green}{\color[gray]{0.75}FO}%
\colorbox{green}{\color[gray]{0.75}FO}%
\colorbox{green}{\color[gray]{0.75}FO}%
\colorbox{green}{\color[gray]{0.75}FO}%
\colorbox{green}{\color[gray]{0.75}FO}%
\colorbox{green}{\color[gray]{0.75}FO}%
\colorbox{green}{\color[gray]{0.75}FO}%
\colorbox{green}{\color[gray]{0.75}FO}%
\colorbox{green}{\color[gray]{0.75}FO}%
\colorbox{green}{\color[gray]{0.75}FO}%
\colorbox{green}{\color[gray]{0.75}FO}%
\colorbox{green}{\color[gray]{0.75}FO}%
\colorbox{green}{\color[gray]{0.75}FO}%
\colorbox{green}{\color[gray]{0.75}FO}%
\colorbox{green}{\color[gray]{0.75}FO}%
\colorbox{green}{\color[gray]{0.75}FO}%
\colorbox{green}{\color[gray]{0.75}FO}%
\colorbox{green}{\color[gray]{0.75}FO}%
\colorbox{green}{\color[gray]{0.75}FO}%
\colorbox{green}{\color[gray]{0.75}FO}%
\colorbox{green}{\color[gray]{0.75}FO}%
\colorbox{green}{\color[gray]{0.75}FO}%
\colorbox{green}{\color[gray]{0.75}FO}%
\colorbox{green}{\color[gray]{0.75}FO}%
\colorbox{green}{\color[gray]{0.75}FO}%
\colorbox{green}{\color[gray]{0.75}FO}%
\colorbox{green}{\color[gray]{0.75}FO}%
\colorbox{green}{\color[gray]{0.75}FO}%
\colorbox{green}{\color[gray]{0.75}FO}%
\colorbox{green}{\color[gray]{0.75}FO}%
\colorbox{green}{\color[gray]{0.75}FO}%
\colorbox{green}{\color[gray]{0.75}FO}%
\colorbox{green}{\color[gray]{0.75}FO}%
\colorbox{green}{\color[gray]{0.75}FO}%
\colorbox{green}{\color[gray]{0.75}FO}%
\colorbox{green}{\color[gray]{0.75}FO}%
\colorbox{green}{\color[gray]{0.75}FO}%
\colorbox{green}{\color[gray]{0.75}FO}%
\colorbox{green}{\color[gray]{0.75}FO}%
\colorbox{green}{\color[gray]{0.75}FO}%
\colorbox{green}{\color[gray]{0.75}FO}%
\colorbox{green}{\color[gray]{0.75}FO}%
\colorbox{green}{\color[gray]{0.75}FO}%
\colorbox{green}{\color[gray]{0.75}FO}%
\colorbox{green}{\color[gray]{0.75}FO}%
\colorbox{green}{\color[gray]{0.75}FO}%
\colorbox{green}{\color[gray]{0.75}FO}%
\colorbox{green}{\color[gray]{0.75}FO}%
\colorbox{green}{\color[gray]{0.75}FO}%
\colorbox{green}{\color[gray]{0.75}FO}%
\colorbox{green}{\color[gray]{0.75}FO}%
\colorbox{green}{\color[gray]{0.75}FO}%
\colorbox{green}{\color[gray]{0.75}FO}%
\colorbox{green}{\color[gray]{0.75}FO}%
\colorbox{green}{\color[gray]{0.75}FO}%
\colorbox{green}{\color[gray]{0.75}FO}%
\colorbox{green}{\color[gray]{0.75}FO}%
\\
\colorbox{green}{\color[gray]{0.75}FO}%
\colorbox{green}{\color[gray]{0.75}FO}%
\colorbox{green}{\color[gray]{0.75}FO}%
\colorbox{green}{\color[gray]{0.75}FO}%
\colorbox{green}{\color[gray]{0.75}FO}%
\colorbox{green}{\color[gray]{0.75}FO}%
\colorbox{green}{\color[gray]{0.75}FO}%
\colorbox{green}{\color[gray]{0.75}FO}%
\colorbox{green}{\color[gray]{0.75}FO}%
\colorbox{green}{\color[gray]{0.75}FO}%
\colorbox{green}{\color[gray]{0.75}FO}%
\colorbox{green}{\color[gray]{0.75}FO}%
\colorbox{green}{\color[gray]{0.75}FO}%
\colorbox{green}{\color[gray]{0.75}FO}%
\colorbox{green}{\color[gray]{0.75}FO}%
\colorbox{green}{\color[gray]{0.75}FO}%
\colorbox{green}{\color[gray]{0.75}FO}%
\colorbox{green}{\color[gray]{0.75}FO}%
\colorbox{green}{\color[gray]{0.75}FO}%
\colorbox{green}{\color[gray]{0.75}FO}%
\colorbox{green}{\color[gray]{0.75}FO}%
\colorbox{green}{\color[gray]{0.75}FO}%
\colorbox{green}{\color[gray]{0.75}FO}%
\colorbox{green}{\color[gray]{0.75}FO}%
\colorbox{green}{\color[gray]{0.75}FO}%
\colorbox{green}{\color[gray]{0.75}FO}%
\colorbox{green}{\color[gray]{0.75}FO}%
\colorbox{green}{\color[gray]{0.75}FO}%
\colorbox{green}{\color[gray]{0.75}FO}%
\colorbox{green}{\color[gray]{0.75}FO}%
\colorbox{green}{\color[gray]{0.75}FO}%
\colorbox{green}{\color[gray]{0.75}FO}%
\colorbox{green}{\color[gray]{0.75}FO}%
\colorbox{green}{\color[gray]{0.75}FO}%
\colorbox{green}{\color[gray]{0.75}FO}%
\colorbox{green}{\color[gray]{0.75}FO}%
\colorbox{green}{\color[gray]{0.75}FO}%
\colorbox{green}{\color[gray]{0.75}FO}%
\colorbox{green}{\color[gray]{0.75}FO}%
\colorbox{green}{\color[gray]{0.75}FO}%
\colorbox{green}{\color[gray]{0.75}FO}%
\colorbox{green}{\color[gray]{0.75}FO}%
\colorbox{green}{\color[gray]{0.75}FO}%
\colorbox{green}{\color[gray]{0.75}FO}%
\colorbox{green}{\color[gray]{0.75}FO}%
\colorbox{green}{\color[gray]{0.75}FO}%
\colorbox{green}{\color[gray]{0.75}FO}%
\colorbox{green}{\color[gray]{0.75}FO}%
\colorbox{green}{\color[gray]{0.75}FO}%
\colorbox{green}{\color[gray]{0.75}FO}%
\colorbox{green}{\color[gray]{0.75}FO}%
\colorbox{green}{\color[gray]{0.75}FO}%
\colorbox{green}{\color[gray]{0.75}FO}%
\colorbox{green}{\color[gray]{0.75}FO}%
\colorbox{green}{\color[gray]{0.75}FO}%
\colorbox{green}{\color[gray]{0.75}FO}%
\colorbox{green}{\color[gray]{0.75}FO}%
\colorbox{green}{\color[gray]{0.75}FO}%
\colorbox{green}{\color[gray]{0.75}FO}%
\colorbox{green}{\color[gray]{0.75}FO}%
\colorbox{green}{\color[gray]{0.75}FO}%
\colorbox{green}{\color[gray]{0.75}FO}%
\colorbox{green}{\color[gray]{0.75}FO}%
\colorbox{green}{\color[gray]{0.75}FO}%
\colorbox{green}{\color[gray]{0.75}FO}%
\colorbox{green}{\color[gray]{0.75}FO}%
\colorbox{green}{\color[gray]{0.75}FO}%
\colorbox{green}{\color[gray]{0.75}FO}%
\colorbox{green}{\color[gray]{0.75}FO}%
\colorbox{green}{\color[gray]{0.75}FO}%
\colorbox{green}{\color[gray]{0.75}FO}%
\colorbox{green}{\color[gray]{0.75}FO}%
\colorbox{green}{\color[gray]{0.75}FO}%
\colorbox{green}{\color[gray]{0.75}FO}%
\colorbox{green}{\color[gray]{0.75}FO}%
\colorbox{green}{\color[gray]{0.75}FO}%
\colorbox{green}{\color[gray]{0.75}FO}%
\colorbox{green}{\color[gray]{0.75}FO}%
\colorbox{green}{\color[gray]{0.75}FO}%
\colorbox{green}{\color[gray]{0.75}FO}%
\colorbox{green}{\color[gray]{0.75}FO}%
\colorbox{green}{\color[gray]{0.75}FO}%
\colorbox{green}{\color[gray]{0.75}FO}%
\colorbox{green}{\color[gray]{0.75}FO}%
\colorbox{green}{\color[gray]{0.75}FO}%
\colorbox{green}{\color[gray]{0.75}FO}%
\colorbox{green}{\color[gray]{0.75}FO}%
\colorbox{green}{\color[gray]{0.75}FO}%
\colorbox{green}{\color[gray]{0.75}FO}%
\colorbox{green}{\color[gray]{0.75}FO}%
\colorbox{green}{\color[gray]{0.75}FO}%
\colorbox{green}{\color[gray]{0.75}FO}%
\colorbox{green}{\color[gray]{0.75}FO}%
\colorbox{green}{\color[gray]{0.75}FO}%
\colorbox{green}{\color[gray]{0.75}FO}%
\colorbox{green}{\color[gray]{0.75}FO}%
\colorbox{green}{\color[gray]{0.75}FO}%
\colorbox{green}{\color[gray]{0.75}FO}%
\colorbox{green}{\color[gray]{0.75}FO}%
\colorbox{green}{\color[gray]{0.75}FO}%
\\
\colorbox{green}{\color[gray]{0.75}FO}%
\colorbox{green}{\color[gray]{0.75}FO}%
\colorbox{green}{\color[gray]{0.75}FO}%
\colorbox{green}{\color[gray]{0.75}FO}%
\colorbox{green}{\color[gray]{0.75}FO}%
\colorbox{green}{\color[gray]{0.75}FO}%
\colorbox{green}{\color[gray]{0.75}FO}%
\colorbox{green}{\color[gray]{0.75}FO}%
\colorbox{green}{\color[gray]{0.75}FO}%
\colorbox{green}{\color[gray]{0.75}FO}%
\colorbox{green}{\color[gray]{0.75}FO}%
\colorbox{green}{\color[gray]{0.75}FO}%
\colorbox{green}{\color[gray]{0.75}FO}%
\colorbox{green}{\color[gray]{0.75}FO}%
\colorbox{green}{\color[gray]{0.75}FO}%
\colorbox{green}{\color[gray]{0.75}FO}%
\colorbox{green}{\color[gray]{0.75}FO}%
\colorbox{green}{\color[gray]{0.75}FO}%
\colorbox{green}{\color[gray]{0.75}FO}%
\colorbox{green}{\color[gray]{0.75}FO}%
\colorbox{green}{\color[gray]{0.75}FO}%
\colorbox{green}{\color[gray]{0.75}FO}%
\colorbox{green}{\color[gray]{0.75}FO}%
\colorbox{green}{\color[gray]{0.75}FO}%
\colorbox{green}{\color[gray]{0.75}FO}%
\colorbox{green}{\color[gray]{0.75}FO}%
\colorbox{green}{\color[gray]{0.75}FO}%
\colorbox{green}{\color[gray]{0.75}FO}%
\colorbox{green}{\color[gray]{0.75}FO}%
\colorbox{green}{\color[gray]{0.75}FO}%
\colorbox{green}{\color[gray]{0.75}FO}%
\colorbox{green}{\color[gray]{0.75}FO}%
\colorbox{green}{\color[gray]{0.75}FO}%
\colorbox{green}{\color[gray]{0.75}FO}%
\colorbox{green}{\color[gray]{0.75}FO}%
\colorbox{green}{\color[gray]{0.75}FO}%
\colorbox{green}{\color[gray]{0.75}FO}%
\colorbox{green}{\color[gray]{0.75}FO}%
\colorbox{green}{\color[gray]{0.75}FO}%
\colorbox{green}{\color[gray]{0.75}FO}%
\colorbox{green}{\color[gray]{0.75}FO}%
\colorbox{green}{\color[gray]{0.75}FO}%
\colorbox{green}{\color[gray]{0.75}FO}%
\colorbox{green}{\color[gray]{0.75}FO}%
\colorbox{green}{\color[gray]{0.75}FO}%
\colorbox{green}{\color[gray]{0.75}FO}%
\colorbox{green}{\color[gray]{0.75}FO}%
\colorbox{green}{\color[gray]{0.75}FO}%
\colorbox{green}{\color[gray]{0.75}FO}%
\colorbox{green}{\color[gray]{0.75}FO}%
\colorbox{green}{\color[gray]{0.75}FO}%
\colorbox{green}{\color[gray]{0.75}FO}%
\colorbox{green}{\color[gray]{0.75}FO}%
\colorbox{green}{\color[gray]{0.75}FO}%
\colorbox{green}{\color[gray]{0.75}FO}%
\colorbox{green}{\color[gray]{0.75}FO}%
\colorbox{green}{\color[gray]{0.75}FO}%
\colorbox{green}{\color[gray]{0.75}FO}%
\colorbox{green}{\color[gray]{0.75}FO}%
\colorbox{green}{\color[gray]{0.75}FO}%
\colorbox{green}{\color[gray]{0.75}FO}%
\colorbox{green}{\color[gray]{0.75}FO}%
\colorbox{green}{\color[gray]{0.75}FO}%
\colorbox{green}{\color[gray]{0.75}FO}%
\colorbox{green}{\color[gray]{0.75}FO}%
\colorbox{green}{\color[gray]{0.75}FO}%
\colorbox{green}{\color[gray]{0.75}FO}%
\colorbox{green}{\color[gray]{0.75}FO}%
\colorbox{green}{\color[gray]{0.75}FO}%
\colorbox{green}{\color[gray]{0.75}FO}%
\colorbox{green}{\color[gray]{0.75}FO}%
\colorbox{green}{\color[gray]{0.75}FO}%
\colorbox{green}{\color[gray]{0.75}FO}%
\colorbox{green}{\color[gray]{0.75}FO}%
\colorbox{green}{\color[gray]{0.75}FO}%
\colorbox{green}{\color[gray]{0.75}FO}%
\colorbox{green}{\color[gray]{0.75}FO}%
\colorbox{green}{\color[gray]{0.75}FO}%
\colorbox{green}{\color[gray]{0.75}FO}%
\colorbox{green}{\color[gray]{0.75}FO}%
\colorbox{green}{\color[gray]{0.75}FO}%
\colorbox{green}{\color[gray]{0.75}FO}%
\colorbox{green}{\color[gray]{0.75}FO}%
\colorbox{green}{\color[gray]{0.75}FO}%
\colorbox{green}{\color[gray]{0.75}FO}%
\colorbox{green}{\color[gray]{0.75}FO}%
\colorbox{green}{\color[gray]{0.75}FO}%
\colorbox{green}{\color[gray]{0.75}FO}%
\colorbox{green}{\color[gray]{0.75}FO}%
\colorbox{green}{\color[gray]{0.75}FO}%
\colorbox{green}{\color[gray]{0.75}FO}%
\colorbox{green}{\color[gray]{0.75}FO}%
\colorbox{green}{\color[gray]{0.75}FO}%
\colorbox{green}{\color[gray]{0.75}FO}%
\colorbox{green}{\color[gray]{0.75}FO}%
\colorbox{green}{\color[gray]{0.75}FO}%
\colorbox{green}{\color[gray]{0.75}FO}%
\colorbox{green}{\color[gray]{0.75}FO}%
\colorbox{green}{\color[gray]{0.75}FO}%
\colorbox{green}{\color[gray]{0.75}FO}%
\\
\colorbox{green}{\color[gray]{0.75}FO}%
\colorbox{green}{\color[gray]{0.75}FO}%
\colorbox{green}{\color[gray]{0.75}FO}%
\colorbox{green}{\color[gray]{0.75}FO}%
\colorbox{green}{\color[gray]{0.75}FO}%
\colorbox{green}{\color[gray]{0.75}FO}%
\colorbox{green}{\color[gray]{0.75}FO}%
\colorbox{green}{\color[gray]{0.75}FO}%
\colorbox{green}{\color[gray]{0.75}FO}%
\colorbox{green}{\color[gray]{0.75}FO}%
\colorbox{green}{\color[gray]{0.75}FO}%
\colorbox{green}{\color[gray]{0.75}FO}%
\colorbox{green}{\color[gray]{0.75}FO}%
\colorbox{green}{\color[gray]{0.75}FO}%
\colorbox{green}{\color[gray]{0.75}FO}%
\colorbox{green}{\color[gray]{0.75}FO}%
\colorbox{green}{\color[gray]{0.75}FO}%
\colorbox{green}{\color[gray]{0.75}FO}%
\colorbox{green}{\color[gray]{0.75}FO}%
\colorbox{green}{\color[gray]{0.75}FO}%
\colorbox{green}{\color[gray]{0.75}FO}%
\colorbox{green}{\color[gray]{0.75}FO}%
\colorbox{green}{\color[gray]{0.75}FO}%
\colorbox{green}{\color[gray]{0.75}FO}%
\colorbox{green}{\color[gray]{0.75}FO}%
\colorbox{green}{\color[gray]{0.75}FO}%
\colorbox{green}{\color[gray]{0.75}FO}%
\colorbox{green}{\color[gray]{0.75}FO}%
\colorbox{green}{\color[gray]{0.75}FO}%
\colorbox{green}{\color[gray]{0.75}FO}%
\colorbox{green}{\color[gray]{0.75}FO}%
\colorbox{green}{\color[gray]{0.75}FO}%
\colorbox{green}{\color[gray]{0.75}FO}%
\colorbox{green}{\color[gray]{0.75}FO}%
\colorbox{green}{\color[gray]{0.75}FO}%
\colorbox{green}{\color[gray]{0.75}FO}%
\colorbox{green}{\color[gray]{0.75}FO}%
\colorbox{green}{\color[gray]{0.75}FO}%
\colorbox{green}{\color[gray]{0.75}FO}%
\colorbox{green}{\color[gray]{0.75}FO}%
\colorbox{green}{\color[gray]{0.75}FO}%
\colorbox{green}{\color[gray]{0.75}FO}%
\colorbox{green}{\color[gray]{0.75}FO}%
\colorbox{green}{\color[gray]{0.75}FO}%
\colorbox{green}{\color[gray]{0.75}FO}%
\colorbox{green}{\color[gray]{0.75}FO}%
\colorbox{green}{\color[gray]{0.75}FO}%
\colorbox{green}{\color[gray]{0.75}FO}%
\colorbox{green}{\color[gray]{0.75}FO}%
\colorbox{green}{\color[gray]{0.75}FO}%
\colorbox{green}{\color[gray]{0.75}FO}%
\colorbox{green}{\color[gray]{0.75}FO}%
\colorbox{green}{\color[gray]{0.75}FO}%
\colorbox{green}{\color[gray]{0.75}FO}%
\colorbox{green}{\color[gray]{0.75}FO}%
\colorbox{green}{\color[gray]{0.75}FO}%
\colorbox{green}{\color[gray]{0.75}FO}%
\colorbox{green}{\color[gray]{0.75}FO}%
\colorbox{green}{\color[gray]{0.75}FO}%
\colorbox{green}{\color[gray]{0.75}FO}%
\colorbox{green}{\color[gray]{0.75}FO}%
\colorbox{green}{\color[gray]{0.75}FO}%
\colorbox{green}{\color[gray]{0.75}FO}%
\colorbox{green}{\color[gray]{0.75}FO}%
\colorbox{green}{\color[gray]{0.75}FO}%
\colorbox{green}{\color[gray]{0.75}FO}%
\colorbox{green}{\color[gray]{0.75}FO}%
\colorbox{green}{\color[gray]{0.75}FO}%
\colorbox{green}{\color[gray]{0.75}FO}%
\colorbox{green}{\color[gray]{0.75}FO}%
\colorbox{green}{\color[gray]{0.75}FO}%
\colorbox{green}{\color[gray]{0.75}FO}%
\colorbox{green}{\color[gray]{0.75}FO}%
\colorbox{green}{\color[gray]{0.75}FO}%
\colorbox{green}{\color[gray]{0.75}FO}%
\colorbox{green}{\color[gray]{0.75}FO}%
\colorbox{green}{\color[gray]{0.75}FO}%
\colorbox{green}{\color[gray]{0.75}FO}%
\colorbox{green}{\color[gray]{0.75}FO}%
\colorbox{green}{\color[gray]{0.75}FO}%
\colorbox{green}{\color[gray]{0.75}FO}%
\colorbox{green}{\color[gray]{0.75}FO}%
\colorbox{green}{\color[gray]{0.75}FO}%
\colorbox{green}{\color[gray]{0.75}FO}%
\colorbox{green}{\color[gray]{0.75}FO}%
\colorbox{green}{\color[gray]{0.75}FO}%
\colorbox{green}{\color[gray]{0.75}FO}%
\colorbox{green}{\color[gray]{0.75}FO}%
\colorbox{green}{\color[gray]{0.75}FO}%
\colorbox{green}{\color[gray]{0.75}FO}%
\colorbox{green}{\color[gray]{0.75}FO}%
\colorbox{green}{\color[gray]{0.75}FO}%
\colorbox{green}{\color[gray]{0.75}FO}%
\colorbox{green}{\color[gray]{0.75}FO}%
\colorbox{green}{\color[gray]{0.75}FO}%
\colorbox{green}{\color[gray]{0.75}FO}%
\colorbox{green}{\color[gray]{0.75}FO}%
\colorbox{green}{\color[gray]{0.75}FO}%
\colorbox{green}{\color[gray]{0.75}FO}%
\colorbox{green}{\color[gray]{0.75}FO}%
\\
\colorbox{green}{\color[gray]{0.75}FO}%
\colorbox{green}{\color[gray]{0.75}FO}%
\colorbox{green}{\color[gray]{0.75}FO}%
\colorbox{green}{\color[gray]{0.75}FO}%
\colorbox{green}{\color[gray]{0.75}FO}%
\colorbox{green}{\color[gray]{0.75}FO}%
\colorbox{green}{\color[gray]{0.75}FO}%
\colorbox{green}{\color[gray]{0.75}FO}%
\colorbox{green}{\color[gray]{0.75}FO}%
\colorbox{green}{\color[gray]{0.75}FO}%
\colorbox{green}{\color[gray]{0.75}FO}%
\colorbox{green}{\color[gray]{0.75}FO}%
\colorbox{green}{\color[gray]{0.75}FO}%
\colorbox{green}{\color[gray]{0.75}FO}%
\colorbox{green}{\color[gray]{0.75}FO}%
\colorbox{green}{\color[gray]{0.75}FO}%
\colorbox{green}{\color[gray]{0.75}FO}%
\colorbox{green}{\color[gray]{0.75}FO}%
\colorbox{green}{\color[gray]{0.75}FO}%
\colorbox{green}{\color[gray]{0.75}FO}%
\colorbox{green}{\color[gray]{0.75}FO}%
\colorbox{green}{\color[gray]{0.75}FO}%
\colorbox{green}{\color[gray]{0.75}FO}%
\colorbox{green}{\color[gray]{0.75}FO}%
\colorbox{green}{\color[gray]{0.75}FO}%
\colorbox{green}{\color[gray]{0.75}FO}%
\colorbox{green}{\color[gray]{0.75}FO}%
\colorbox{green}{\color[gray]{0.75}FO}%
\colorbox{green}{\color[gray]{0.75}FO}%
\colorbox{green}{\color[gray]{0.75}FO}%
\colorbox{green}{\color[gray]{0.75}FO}%
\colorbox{green}{\color[gray]{0.75}FO}%
\colorbox{green}{\color[gray]{0.75}FO}%
\colorbox{green}{\color[gray]{0.75}FO}%
\colorbox{green}{\color[gray]{0.75}FO}%
\colorbox{green}{\color[gray]{0.75}FO}%
\colorbox{green}{\color[gray]{0.75}FO}%
\colorbox{green}{\color[gray]{0.75}FO}%
\colorbox{green}{\color[gray]{0.75}FO}%
\colorbox{green}{\color[gray]{0.75}FO}%
\colorbox{green}{\color[gray]{0.75}FO}%
\colorbox{green}{\color[gray]{0.75}FO}%
\colorbox{green}{\color[gray]{0.75}FO}%
\colorbox{green}{\color[gray]{0.75}FO}%
\colorbox{green}{\color[gray]{0.75}FO}%
\colorbox{green}{\color[gray]{0.75}FO}%
\colorbox{green}{\color[gray]{0.75}FO}%
\colorbox{green}{\color[gray]{0.75}FO}%
\colorbox{green}{\color[gray]{0.75}FO}%
\colorbox{green}{\color[gray]{0.75}FO}%
\colorbox{green}{\color[gray]{0.75}FO}%
\colorbox{green}{\color[gray]{0.75}FO}%
\colorbox{green}{\color[gray]{0.75}FO}%
\colorbox{green}{\color[gray]{0.75}FO}%
\colorbox{green}{\color[gray]{0.75}FO}%
\colorbox{green}{\color[gray]{0.75}FO}%
\colorbox{green}{\color[gray]{0.75}FO}%
\colorbox{green}{\color[gray]{0.75}FO}%
\colorbox{green}{\color[gray]{0.75}FO}%
\colorbox{green}{\color[gray]{0.75}FO}%
\colorbox{green}{\color[gray]{0.75}FO}%
\colorbox{green}{\color[gray]{0.75}FO}%
\colorbox{green}{\color[gray]{0.75}FO}%
\colorbox{green}{\color[gray]{0.75}FO}%
\colorbox{green}{\color[gray]{0.75}FO}%
\colorbox{green}{\color[gray]{0.75}FO}%
\colorbox{green}{\color[gray]{0.75}FO}%
\colorbox{green}{\color[gray]{0.75}FO}%
\colorbox{green}{\color[gray]{0.75}FO}%
\colorbox{green}{\color[gray]{0.75}FO}%
\colorbox{green}{\color[gray]{0.75}FO}%
\colorbox{green}{\color[gray]{0.75}FO}%
\colorbox{green}{\color[gray]{0.75}FO}%
\colorbox{green}{\color[gray]{0.75}FO}%
\colorbox{green}{\color[gray]{0.75}FO}%
\colorbox{green}{\color[gray]{0.75}FO}%
\colorbox{green}{\color[gray]{0.75}FO}%
\colorbox{green}{\color[gray]{0.75}FO}%
\colorbox{green}{\color[gray]{0.75}FO}%
\colorbox{green}{\color[gray]{0.75}FO}%
\colorbox{green}{\color[gray]{0.75}FO}%
\colorbox{green}{\color[gray]{0.75}FO}%
\colorbox{green}{\color[gray]{0.75}FO}%
\colorbox{green}{\color[gray]{0.75}FO}%
\colorbox{green}{\color[gray]{0.75}FO}%
\colorbox{green}{\color[gray]{0.75}FO}%
\colorbox{green}{\color[gray]{0.75}FO}%
\colorbox{green}{\color[gray]{0.75}FO}%
\colorbox{green}{\color[gray]{0.75}FO}%
\colorbox{green}{\color[gray]{0.75}FO}%
\colorbox{green}{\color[gray]{0.75}FO}%
\colorbox{green}{\color[gray]{0.75}FO}%
\colorbox{green}{\color[gray]{0.75}FO}%
\colorbox{green}{\color[gray]{0.75}FO}%
\colorbox{green}{\color[gray]{0.75}FO}%
\colorbox{green}{\color[gray]{0.75}FO}%
\colorbox{green}{\color[gray]{0.75}FO}%
\colorbox{green}{\color[gray]{0.75}FO}%
\colorbox{green}{\color[gray]{0.75}FO}%
\colorbox{green}{\color[gray]{0.75}FO}%
\\
\colorbox{green}{\color[gray]{0.75}FO}%
\colorbox{green}{\color[gray]{0.75}FO}%
\colorbox{green}{\color[gray]{0.75}FO}%
\colorbox{green}{\color[gray]{0.75}FO}%
\colorbox{green}{\color[gray]{0.75}FO}%
\colorbox{green}{\color[gray]{0.75}FO}%
\colorbox{green}{\color[gray]{0.75}FO}%
\colorbox{green}{\color[gray]{0.75}FO}%
\colorbox{green}{\color[gray]{0.75}FO}%
\colorbox{green}{\color[gray]{0.75}FO}%
\colorbox{green}{\color[gray]{0.75}FO}%
\colorbox{green}{\color[gray]{0.75}FO}%
\colorbox{green}{\color[gray]{0.75}FO}%
\colorbox{green}{\color[gray]{0.75}FO}%
\colorbox{green}{\color[gray]{0.75}FO}%
\colorbox{green}{\color[gray]{0.75}FO}%
\colorbox{green}{\color[gray]{0.75}FO}%
\colorbox{green}{\color[gray]{0.75}FO}%
\colorbox{green}{\color[gray]{0.75}FO}%
\colorbox{green}{\color[gray]{0.75}FO}%
\colorbox{green}{\color[gray]{0.75}FO}%
\colorbox{green}{\color[gray]{0.75}FO}%
\colorbox{green}{\color[gray]{0.75}FO}%
\colorbox{green}{\color[gray]{0.75}FO}%
\colorbox{green}{\color[gray]{0.75}FO}%
\colorbox{green}{\color[gray]{0.75}FO}%
\colorbox{green}{\color[gray]{0.75}FO}%
\colorbox{green}{\color[gray]{0.75}FO}%
\colorbox{green}{\color[gray]{0.75}FO}%
\colorbox{green}{\color[gray]{0.75}FO}%
\colorbox{green}{\color[gray]{0.75}FO}%
\colorbox{green}{\color[gray]{0.75}FO}%
\colorbox{green}{\color[gray]{0.75}FO}%
\colorbox{green}{\color[gray]{0.75}FO}%
\colorbox{green}{\color[gray]{0.75}FO}%
\colorbox{green}{\color[gray]{0.75}FO}%
\colorbox{green}{\color[gray]{0.75}FO}%
\colorbox{green}{\color[gray]{0.75}FO}%
\colorbox{green}{\color[gray]{0.75}FO}%
\colorbox{green}{\color[gray]{0.75}FO}%
\colorbox{green}{\color[gray]{0.75}FO}%
\colorbox{green}{\color[gray]{0.75}FO}%
\colorbox{green}{\color[gray]{0.75}FO}%
\colorbox{green}{\color[gray]{0.75}FO}%
\colorbox{green}{\color[gray]{0.75}FO}%
\colorbox{green}{\color[gray]{0.75}FO}%
\colorbox{green}{\color[gray]{0.75}FO}%
\colorbox{green}{\color[gray]{0.75}FO}%
\colorbox{green}{\color[gray]{0.75}FO}%
\colorbox{green}{\color[gray]{0.75}FO}%
\colorbox{green}{\color[gray]{0.75}FO}%
\colorbox{green}{\color[gray]{0.75}FO}%
\colorbox{green}{\color[gray]{0.75}FO}%
\colorbox{green}{\color[gray]{0.75}FO}%
\colorbox{green}{\color[gray]{0.75}FO}%
\colorbox{green}{\color[gray]{0.75}FO}%
\colorbox{green}{\color[gray]{0.75}FO}%
\colorbox{green}{\color[gray]{0.75}FO}%
\colorbox{green}{\color[gray]{0.75}FO}%
\colorbox{green}{\color[gray]{0.75}FO}%
\colorbox{green}{\color[gray]{0.75}FO}%
\colorbox{green}{\color[gray]{0.75}FO}%
\colorbox{green}{\color[gray]{0.75}FO}%
\colorbox{green}{\color[gray]{0.75}FO}%
\colorbox{green}{\color[gray]{0.75}FO}%
\colorbox{green}{\color[gray]{0.75}FO}%
\colorbox{green}{\color[gray]{0.75}FO}%
\colorbox{green}{\color[gray]{0.75}FO}%
\colorbox{green}{\color[gray]{0.75}FO}%
\colorbox{green}{\color[gray]{0.75}FO}%
\colorbox{green}{\color[gray]{0.75}FO}%
\colorbox{green}{\color[gray]{0.75}FO}%
\colorbox{green}{\color[gray]{0.75}FO}%
\colorbox{green}{\color[gray]{0.75}FO}%
\colorbox{green}{\color[gray]{0.75}FO}%
\colorbox{green}{\color[gray]{0.75}FO}%
\colorbox{green}{\color[gray]{0.75}FO}%
\colorbox{green}{\color[gray]{0.75}FO}%
\colorbox{green}{\color[gray]{0.75}FO}%
\colorbox{green}{\color[gray]{0.75}FO}%
\colorbox{green}{\color[gray]{0.75}FO}%
\colorbox{green}{\color[gray]{0.75}FO}%
\colorbox{green}{\color[gray]{0.75}FO}%
\colorbox{green}{\color[gray]{0.75}FO}%
\colorbox{green}{\color[gray]{0.75}FO}%
\colorbox{green}{\color[gray]{0.75}FO}%
\colorbox{green}{\color[gray]{0.75}FO}%
\colorbox{green}{\color[gray]{0.75}FO}%
\colorbox{green}{\color[gray]{0.75}FO}%
\colorbox{green}{\color[gray]{0.75}FO}%
\colorbox{green}{\color[gray]{0.75}FO}%
\colorbox{green}{\color[gray]{0.75}FO}%
\colorbox{green}{\color[gray]{0.75}FO}%
\colorbox{green}{\color[gray]{0.75}FO}%
\colorbox{green}{\color[gray]{0.75}FO}%
\colorbox{green}{\color[gray]{0.75}FO}%
\colorbox{green}{\color[gray]{0.75}FO}%
\colorbox{green}{\color[gray]{0.75}FO}%
\colorbox{green}{\color[gray]{0.75}FO}%
\colorbox{green}{\color[gray]{0.75}FO}%
\\
\colorbox{green}{\color[gray]{0.75}FO}%
\colorbox{green}{\color[gray]{0.75}FO}%
\colorbox{green}{\color[gray]{0.75}FO}%
\colorbox{green}{\color[gray]{0.75}FO}%
\colorbox{green}{\color[gray]{0.75}FO}%
\colorbox{green}{\color[gray]{0.75}FO}%
\colorbox{green}{\color[gray]{0.75}FO}%
\colorbox{green}{\color[gray]{0.75}FO}%
\colorbox{green}{\color[gray]{0.75}FO}%
\colorbox{green}{\color[gray]{0.75}FO}%
\colorbox{green}{\color[gray]{0.75}FO}%
\colorbox{green}{\color[gray]{0.75}FO}%
\colorbox{green}{\color[gray]{0.75}FO}%
\colorbox{green}{\color[gray]{0.75}FO}%
\colorbox{green}{\color[gray]{0.75}FO}%
\colorbox{green}{\color[gray]{0.75}FO}%
\colorbox{green}{\color[gray]{0.75}FO}%
\colorbox{green}{\color[gray]{0.75}FO}%
\colorbox{green}{\color[gray]{0.75}FO}%
\colorbox{green}{\color[gray]{0.75}FO}%
\colorbox{green}{\color[gray]{0.75}FO}%
\colorbox{green}{\color[gray]{0.75}FO}%
\colorbox{green}{\color[gray]{0.75}FO}%
\colorbox{green}{\color[gray]{0.75}FO}%
\colorbox{green}{\color[gray]{0.75}FO}%
\colorbox{green}{\color[gray]{0.75}FO}%
\colorbox{green}{\color[gray]{0.75}FO}%
\colorbox{green}{\color[gray]{0.75}FO}%
\colorbox{green}{\color[gray]{0.75}FO}%
\colorbox{green}{\color[gray]{0.75}FO}%
\colorbox{green}{\color[gray]{0.75}FO}%
\colorbox{green}{\color[gray]{0.75}FO}%
\colorbox{green}{\color[gray]{0.75}FO}%
\colorbox{green}{\color[gray]{0.75}FO}%
\colorbox{green}{\color[gray]{0.75}FO}%
\colorbox{green}{\color[gray]{0.75}FO}%
\colorbox{green}{\color[gray]{0.75}FO}%
\colorbox{green}{\color[gray]{0.75}FO}%
\colorbox{green}{\color[gray]{0.75}FO}%
\colorbox{green}{\color[gray]{0.75}FO}%
\colorbox{green}{\color[gray]{0.75}FO}%
\colorbox{green}{\color[gray]{0.75}FO}%
\colorbox{green}{\color[gray]{0.75}FO}%
\colorbox{green}{\color[gray]{0.75}FO}%
\colorbox{green}{\color[gray]{0.75}FO}%
\colorbox{green}{\color[gray]{0.75}FO}%
\colorbox{green}{\color[gray]{0.75}FO}%
\colorbox{green}{\color[gray]{0.75}FO}%
\colorbox{green}{\color[gray]{0.75}FO}%
\colorbox{green}{\color[gray]{0.75}FO}%
\colorbox{green}{\color[gray]{0.75}FO}%
\colorbox{green}{\color[gray]{0.75}FO}%
\colorbox{green}{\color[gray]{0.75}FO}%
\colorbox{green}{\color[gray]{0.75}FO}%
\colorbox{green}{\color[gray]{0.75}FO}%
\colorbox{green}{\color[gray]{0.75}FO}%
\colorbox{green}{\color[gray]{0.75}FO}%
\colorbox{green}{\color[gray]{0.75}FO}%
\colorbox{green}{\color[gray]{0.75}FO}%
\colorbox{green}{\color[gray]{0.75}FO}%
\colorbox{green}{\color[gray]{0.75}FO}%
\colorbox{green}{\color[gray]{0.75}FO}%
\colorbox{green}{\color[gray]{0.75}FO}%
\colorbox{green}{\color[gray]{0.75}FO}%
\colorbox{green}{\color[gray]{0.75}FO}%
\colorbox{green}{\color[gray]{0.75}FO}%
\colorbox{green}{\color[gray]{0.75}FO}%
\colorbox{green}{\color[gray]{0.75}FO}%
\colorbox{green}{\color[gray]{0.75}FO}%
\colorbox{green}{\color[gray]{0.75}FO}%
\colorbox{green}{\color[gray]{0.75}FO}%
\colorbox{green}{\color[gray]{0.75}FO}%
\colorbox{green}{\color[gray]{0.75}FO}%
\colorbox{green}{\color[gray]{0.75}FO}%
\colorbox{green}{\color[gray]{0.75}FO}%
\colorbox{green}{\color[gray]{0.75}FO}%
\colorbox{green}{\color[gray]{0.75}FO}%
\colorbox{green}{\color[gray]{0.75}FO}%
\colorbox{green}{\color[gray]{0.75}FO}%
\colorbox{green}{\color[gray]{0.75}FO}%
\colorbox{green}{\color[gray]{0.75}FO}%
\colorbox{green}{\color[gray]{0.75}FO}%
\colorbox{green}{\color[gray]{0.75}FO}%
\colorbox{green}{\color[gray]{0.75}FO}%
\colorbox{green}{\color[gray]{0.75}FO}%
\colorbox{green}{\color[gray]{0.75}FO}%
\colorbox{green}{\color[gray]{0.75}FO}%
\colorbox{green}{\color[gray]{0.75}FO}%
\colorbox{green}{\color[gray]{0.75}FO}%
\colorbox{green}{\color[gray]{0.75}FO}%
\colorbox{green}{\color[gray]{0.75}FO}%
\colorbox{green}{\color[gray]{0.75}FO}%
\colorbox{green}{\color[gray]{0.75}FO}%
\colorbox{green}{\color[gray]{0.75}FO}%
\colorbox{green}{\color[gray]{0.75}FO}%
\colorbox{green}{\color[gray]{0.75}FO}%
\colorbox{green}{\color[gray]{0.75}FO}%
\colorbox{green}{\color[gray]{0.75}FO}%
\colorbox{green}{\color[gray]{0.75}FO}%
\colorbox{green}{\color[gray]{0.75}FO}%
\\
\colorbox{green}{\color[gray]{0.75}FO}%
\colorbox{green}{\color[gray]{0.75}FO}%
\colorbox{green}{\color[gray]{0.75}FO}%
\colorbox{green}{\color[gray]{0.75}FO}%
\colorbox{green}{\color[gray]{0.75}FO}%
\colorbox{green}{\color[gray]{0.75}FO}%
\colorbox{green}{\color[gray]{0.75}FO}%
\colorbox{green}{\color[gray]{0.75}FO}%
\colorbox{green}{\color[gray]{0.75}FO}%
\colorbox{green}{\color[gray]{0.75}FO}%
\colorbox{green}{\color[gray]{0.75}FO}%
\colorbox{green}{\color[gray]{0.75}FO}%
\colorbox{green}{\color[gray]{0.75}FO}%
\colorbox{green}{\color[gray]{0.75}FO}%
\colorbox{green}{\color[gray]{0.75}FO}%
\colorbox{green}{\color[gray]{0.75}FO}%
\colorbox{green}{\color[gray]{0.75}FO}%
\colorbox{green}{\color[gray]{0.75}FO}%
\colorbox{green}{\color[gray]{0.75}FO}%
\colorbox{green}{\color[gray]{0.75}FO}%
\colorbox{green}{\color[gray]{0.75}FO}%
\colorbox{green}{\color[gray]{0.75}FO}%
\colorbox{green}{\color[gray]{0.75}FO}%
\colorbox{green}{\color[gray]{0.75}FO}%
\colorbox{green}{\color[gray]{0.75}FO}%
\colorbox{green}{\color[gray]{0.75}FO}%
\colorbox{green}{\color[gray]{0.75}FO}%
\colorbox{green}{\color[gray]{0.75}FO}%
\colorbox{green}{\color[gray]{0.75}FO}%
\colorbox{green}{\color[gray]{0.75}FO}%
\colorbox{green}{\color[gray]{0.75}FO}%
\colorbox{green}{\color[gray]{0.75}FO}%
\colorbox{green}{\color[gray]{0.75}FO}%
\colorbox{green}{\color[gray]{0.75}FO}%
\colorbox{green}{\color[gray]{0.75}FO}%
\colorbox{green}{\color[gray]{0.75}FO}%
\colorbox{green}{\color[gray]{0.75}FO}%
\colorbox{green}{\color[gray]{0.75}FO}%
\colorbox{green}{\color[gray]{0.75}FO}%
\colorbox{green}{\color[gray]{0.75}FO}%
\colorbox{green}{\color[gray]{0.75}FO}%
\colorbox{green}{\color[gray]{0.75}FO}%
\colorbox{green}{\color[gray]{0.75}FO}%
\colorbox{green}{\color[gray]{0.75}FO}%
\colorbox{green}{\color[gray]{0.75}FO}%
\colorbox{green}{\color[gray]{0.75}FO}%
\colorbox{green}{\color[gray]{0.75}FO}%
\colorbox{green}{\color[gray]{0.75}FO}%
\colorbox{green}{\color[gray]{0.75}FO}%
\colorbox{green}{\color[gray]{0.75}FO}%
\colorbox{green}{\color[gray]{0.75}FO}%
\colorbox{green}{\color[gray]{0.75}FO}%
\colorbox{green}{\color[gray]{0.75}FO}%
\colorbox{green}{\color[gray]{0.75}FO}%
\colorbox{green}{\color[gray]{0.75}FO}%
\colorbox{green}{\color[gray]{0.75}FO}%
\colorbox{green}{\color[gray]{0.75}FO}%
\colorbox{green}{\color[gray]{0.75}FO}%
\colorbox{green}{\color[gray]{0.75}FO}%
\colorbox{green}{\color[gray]{0.75}FO}%
\colorbox{green}{\color[gray]{0.75}FO}%
\colorbox{green}{\color[gray]{0.75}FO}%
\colorbox{green}{\color[gray]{0.75}FO}%
\colorbox{green}{\color[gray]{0.75}FO}%
\colorbox{green}{\color[gray]{0.75}FO}%
\colorbox{green}{\color[gray]{0.75}FO}%
\colorbox{green}{\color[gray]{0.75}FO}%
\colorbox{green}{\color[gray]{0.75}FO}%
\colorbox{green}{\color[gray]{0.75}FO}%
\colorbox{green}{\color[gray]{0.75}FO}%
\colorbox{green}{\color[gray]{0.75}FO}%
\colorbox{green}{\color[gray]{0.75}FO}%
\colorbox{green}{\color[gray]{0.75}FO}%
\colorbox{green}{\color[gray]{0.75}FO}%
\colorbox{green}{\color[gray]{0.75}FO}%
\colorbox{green}{\color[gray]{0.75}FO}%
\colorbox{green}{\color[gray]{0.75}FO}%
\colorbox{green}{\color[gray]{0.75}FO}%
\colorbox{green}{\color[gray]{0.75}FO}%
\colorbox{green}{\color[gray]{0.75}FO}%
\colorbox{green}{\color[gray]{0.75}FO}%
\colorbox{green}{\color[gray]{0.75}FO}%
\colorbox{green}{\color[gray]{0.75}FO}%
\colorbox{green}{\color[gray]{0.75}FO}%
\colorbox{green}{\color[gray]{0.75}FO}%
\colorbox{green}{\color[gray]{0.75}FO}%
\colorbox{green}{\color[gray]{0.75}FO}%
\colorbox{green}{\color[gray]{0.75}FO}%
\colorbox{green}{\color[gray]{0.75}FO}%
\colorbox{green}{\color[gray]{0.75}FO}%
\colorbox{green}{\color[gray]{0.75}FO}%
\colorbox{green}{\color[gray]{0.75}FO}%
\colorbox{green}{\color[gray]{0.75}FO}%
\colorbox{green}{\color[gray]{0.75}FO}%
\colorbox{green}{\color[gray]{0.75}FO}%
\colorbox{green}{\color[gray]{0.75}FO}%
\colorbox{green}{\color[gray]{0.75}FO}%
\colorbox{green}{\color[gray]{0.75}FO}%
\colorbox{green}{\color[gray]{0.75}FO}%
\colorbox{green}{\color[gray]{0.75}FO}%
\\
\colorbox{green}{\color[gray]{0.75}FO}%
\colorbox{green}{\color[gray]{0.75}FO}%
\colorbox{green}{\color[gray]{0.75}FO}%
\colorbox{green}{\color[gray]{0.75}FO}%
\colorbox{green}{\color[gray]{0.75}FO}%
\colorbox{green}{\color[gray]{0.75}FO}%
\colorbox{green}{\color[gray]{0.75}FO}%
\colorbox{green}{\color[gray]{0.75}FO}%
\colorbox{green}{\color[gray]{0.75}FO}%
\colorbox{green}{\color[gray]{0.75}FO}%
\colorbox{green}{\color[gray]{0.75}FO}%
\colorbox{green}{\color[gray]{0.75}FO}%
\colorbox{green}{\color[gray]{0.75}FO}%
\colorbox{green}{\color[gray]{0.75}FO}%
\colorbox{green}{\color[gray]{0.75}FO}%
\colorbox{green}{\color[gray]{0.75}FO}%
\colorbox{green}{\color[gray]{0.75}FO}%
\colorbox{green}{\color[gray]{0.75}FO}%
\colorbox{green}{\color[gray]{0.75}FO}%
\colorbox{green}{\color[gray]{0.75}FO}%
\colorbox{green}{\color[gray]{0.75}FO}%
\colorbox{green}{\color[gray]{0.75}FO}%
\colorbox{green}{\color[gray]{0.75}FO}%
\colorbox{green}{\color[gray]{0.75}FO}%
\colorbox{green}{\color[gray]{0.75}FO}%
\colorbox{green}{\color[gray]{0.75}FO}%
\colorbox{green}{\color[gray]{0.75}FO}%
\colorbox{green}{\color[gray]{0.75}FO}%
\colorbox{green}{\color[gray]{0.75}FO}%
\colorbox{green}{\color[gray]{0.75}FO}%
\colorbox{green}{\color[gray]{0.75}FO}%
\colorbox{green}{\color[gray]{0.75}FO}%
\colorbox{green}{\color[gray]{0.75}FO}%
\colorbox{green}{\color[gray]{0.75}FO}%
\colorbox{green}{\color[gray]{0.75}FO}%
\colorbox{green}{\color[gray]{0.75}FO}%
\colorbox{green}{\color[gray]{0.75}FO}%
\colorbox{green}{\color[gray]{0.75}FO}%
\colorbox{green}{\color[gray]{0.75}FO}%
\colorbox{green}{\color[gray]{0.75}FO}%
\colorbox{green}{\color[gray]{0.75}FO}%
\colorbox{green}{\color[gray]{0.75}FO}%
\colorbox{green}{\color[gray]{0.75}FO}%
\colorbox{green}{\color[gray]{0.75}FO}%
\colorbox{green}{\color[gray]{0.75}FO}%
\colorbox{green}{\color[gray]{0.75}FO}%
\colorbox{green}{\color[gray]{0.75}FO}%
\colorbox{green}{\color[gray]{0.75}FO}%
\colorbox{green}{\color[gray]{0.75}FO}%
\colorbox{green}{\color[gray]{0.75}FO}%
\colorbox{green}{\color[gray]{0.75}FO}%
\colorbox{green}{\color[gray]{0.75}FO}%
\colorbox{green}{\color[gray]{0.75}FO}%
\colorbox{green}{\color[gray]{0.75}FO}%
\colorbox{green}{\color[gray]{0.75}FO}%
\colorbox{green}{\color[gray]{0.75}FO}%
\colorbox{green}{\color[gray]{0.75}FO}%
\colorbox{green}{\color[gray]{0.75}FO}%
\colorbox{green}{\color[gray]{0.75}FO}%
\colorbox{green}{\color[gray]{0.75}FO}%
\colorbox{green}{\color[gray]{0.75}FO}%
\colorbox{green}{\color[gray]{0.75}FO}%
\colorbox{green}{\color[gray]{0.75}FO}%
\colorbox{green}{\color[gray]{0.75}FO}%
\colorbox{green}{\color[gray]{0.75}FO}%
\colorbox{green}{\color[gray]{0.75}FO}%
\colorbox{green}{\color[gray]{0.75}FO}%
\colorbox{green}{\color[gray]{0.75}FO}%
\colorbox{green}{\color[gray]{0.75}FO}%
\colorbox{green}{\color[gray]{0.75}FO}%
\colorbox{green}{\color[gray]{0.75}FO}%
\colorbox{green}{\color[gray]{0.75}FO}%
\colorbox{green}{\color[gray]{0.75}FO}%
\colorbox{green}{\color[gray]{0.75}FO}%
\colorbox{green}{\color[gray]{0.75}FO}%
\colorbox{green}{\color[gray]{0.75}FO}%
\colorbox{green}{\color[gray]{0.75}FO}%
\colorbox{green}{\color[gray]{0.75}FO}%
\colorbox{green}{\color[gray]{0.75}FO}%
\colorbox{green}{\color[gray]{0.75}FO}%
\colorbox{green}{\color[gray]{0.75}FO}%
\colorbox{green}{\color[gray]{0.75}FO}%
\colorbox{green}{\color[gray]{0.75}FO}%
\colorbox{green}{\color[gray]{0.75}FO}%
\colorbox{green}{\color[gray]{0.75}FO}%
\colorbox{green}{\color[gray]{0.75}FO}%
\colorbox{green}{\color[gray]{0.75}FO}%
\colorbox{green}{\color[gray]{0.75}FO}%
\colorbox{green}{\color[gray]{0.75}FO}%
\colorbox{green}{\color[gray]{0.75}FO}%
\colorbox{green}{\color[gray]{0.75}FO}%
\colorbox{green}{\color[gray]{0.75}FO}%
\colorbox{green}{\color[gray]{0.75}FO}%
\colorbox{green}{\color[gray]{0.75}FO}%
\colorbox{green}{\color[gray]{0.75}FO}%
\colorbox{green}{\color[gray]{0.75}FO}%
\colorbox{green}{\color[gray]{0.75}FO}%
\colorbox{green}{\color[gray]{0.75}FO}%
\colorbox{green}{\color[gray]{0.75}FO}%
\colorbox{green}{\color[gray]{0.75}FO}%
\\
\colorbox{green}{\color[gray]{0.75}FO}%
\colorbox{green}{\color[gray]{0.75}FO}%
\colorbox{green}{\color[gray]{0.75}FO}%
\colorbox{green}{\color[gray]{0.75}FO}%
\colorbox{green}{\color[gray]{0.75}FO}%
\colorbox{green}{\color[gray]{0.75}FO}%
\colorbox{green}{\color[gray]{0.75}FO}%
\colorbox{green}{\color[gray]{0.75}FO}%
\colorbox{green}{\color[gray]{0.75}FO}%
\colorbox{green}{\color[gray]{0.75}FO}%
\colorbox{green}{\color[gray]{0.75}FO}%
\colorbox{green}{\color[gray]{0.75}FO}%
\colorbox{green}{\color[gray]{0.75}FO}%
\colorbox{green}{\color[gray]{0.75}FO}%
\colorbox{green}{\color[gray]{0.75}FO}%
\colorbox{green}{\color[gray]{0.75}FO}%
\colorbox{green}{\color[gray]{0.75}FO}%
\colorbox{green}{\color[gray]{0.75}FO}%
\colorbox{green}{\color[gray]{0.75}FO}%
\colorbox{green}{\color[gray]{0.75}FO}%
\colorbox{green}{\color[gray]{0.75}FO}%
\colorbox{green}{\color[gray]{0.75}FO}%
\colorbox{green}{\color[gray]{0.75}FO}%
\colorbox{green}{\color[gray]{0.75}FO}%
\colorbox{green}{\color[gray]{0.75}FO}%
\colorbox{green}{\color[gray]{0.75}FO}%
\colorbox{green}{\color[gray]{0.75}FO}%
\colorbox{green}{\color[gray]{0.75}FO}%
\colorbox{green}{\color[gray]{0.75}FO}%
\colorbox{green}{\color[gray]{0.75}FO}%
\colorbox{green}{\color[gray]{0.75}FO}%
\colorbox{green}{\color[gray]{0.75}FO}%
\colorbox{green}{\color[gray]{0.75}FO}%
\colorbox{green}{\color[gray]{0.75}FO}%
\colorbox{green}{\color[gray]{0.75}FO}%
\colorbox{green}{\color[gray]{0.75}FO}%
\colorbox{green}{\color[gray]{0.75}FO}%
\colorbox{green}{\color[gray]{0.75}FO}%
\colorbox{green}{\color[gray]{0.75}FO}%
\colorbox{green}{\color[gray]{0.75}FO}%
\colorbox{green}{\color[gray]{0.75}FO}%
\colorbox{green}{\color[gray]{0.75}FO}%
\colorbox{green}{\color[gray]{0.75}FO}%
\colorbox{green}{\color[gray]{0.75}FO}%
\colorbox{green}{\color[gray]{0.75}FO}%
\colorbox{green}{\color[gray]{0.75}FO}%
\colorbox{green}{\color[gray]{0.75}FO}%
\colorbox{green}{\color[gray]{0.75}FO}%
\colorbox{green}{\color[gray]{0.75}FO}%
\colorbox{green}{\color[gray]{0.75}FO}%
\colorbox{green}{\color[gray]{0.75}FO}%
\colorbox{green}{\color[gray]{0.75}FO}%
\colorbox{green}{\color[gray]{0.75}FO}%
\colorbox{green}{\color[gray]{0.75}FO}%
\colorbox{green}{\color[gray]{0.75}FO}%
\colorbox{green}{\color[gray]{0.75}FO}%
\colorbox{green}{\color[gray]{0.75}FO}%
\colorbox{green}{\color[gray]{0.75}FO}%
\colorbox{green}{\color[gray]{0.75}FO}%
\colorbox{green}{\color[gray]{0.75}FO}%
\colorbox{green}{\color[gray]{0.75}FO}%
\colorbox{green}{\color[gray]{0.75}FO}%
\colorbox{green}{\color[gray]{0.75}FO}%
\colorbox{green}{\color[gray]{0.75}FO}%
\colorbox{green}{\color[gray]{0.75}FO}%
\colorbox{green}{\color[gray]{0.75}FO}%
\colorbox{green}{\color[gray]{0.75}FO}%
\colorbox{green}{\color[gray]{0.75}FO}%
\colorbox{green}{\color[gray]{0.75}FO}%
\colorbox{green}{\color[gray]{0.75}FO}%
\colorbox{green}{\color[gray]{0.75}FO}%
\colorbox{green}{\color[gray]{0.75}FO}%
\colorbox{green}{\color[gray]{0.75}FO}%
\colorbox{green}{\color[gray]{0.75}FO}%
\colorbox{green}{\color[gray]{0.75}FO}%
\colorbox{green}{\color[gray]{0.75}FO}%
\colorbox{green}{\color[gray]{0.75}FO}%
\colorbox{green}{\color[gray]{0.75}FO}%
\colorbox{green}{\color[gray]{0.75}FO}%
\colorbox{green}{\color[gray]{0.75}FO}%
\colorbox{green}{\color[gray]{0.75}FO}%
\colorbox{green}{\color[gray]{0.75}FO}%
\colorbox{green}{\color[gray]{0.75}FO}%
\colorbox{green}{\color[gray]{0.75}FO}%
\colorbox{green}{\color[gray]{0.75}FO}%
\colorbox{green}{\color[gray]{0.75}FO}%
\colorbox{green}{\color[gray]{0.75}FO}%
\colorbox{green}{\color[gray]{0.75}FO}%
\colorbox{green}{\color[gray]{0.75}FO}%
\colorbox{green}{\color[gray]{0.75}FO}%
\colorbox{green}{\color[gray]{0.75}FO}%
\colorbox{green}{\color[gray]{0.75}FO}%
\colorbox{green}{\color[gray]{0.75}FO}%
\colorbox{green}{\color[gray]{0.75}FO}%
\colorbox{green}{\color[gray]{0.75}FO}%
\colorbox{green}{\color[gray]{0.75}FO}%
\colorbox{green}{\color[gray]{0.75}FO}%
\colorbox{green}{\color[gray]{0.75}FO}%
\colorbox{green}{\color[gray]{0.75}FO}%
\colorbox{green}{\color[gray]{0.75}FO}%
\\
\colorbox{green}{\color[gray]{0.75}FO}%
\colorbox{green}{\color[gray]{0.75}FO}%
\colorbox{green}{\color[gray]{0.75}FO}%
\colorbox{green}{\color[gray]{0.75}FO}%
\colorbox{green}{\color[gray]{0.75}FO}%
\colorbox{green}{\color[gray]{0.75}FO}%
\colorbox{green}{\color[gray]{0.75}FO}%
\colorbox{green}{\color[gray]{0.75}FO}%
\colorbox{green}{\color[gray]{0.75}FO}%
\colorbox{green}{\color[gray]{0.75}FO}%
\colorbox{green}{\color[gray]{0.75}FO}%
\colorbox{green}{\color[gray]{0.75}FO}%
\colorbox{green}{\color[gray]{0.75}FO}%
\colorbox{green}{\color[gray]{0.75}FO}%
\colorbox{green}{\color[gray]{0.75}FO}%
\colorbox{green}{\color[gray]{0.75}FO}%
\colorbox{green}{\color[gray]{0.75}FO}%
\colorbox{green}{\color[gray]{0.75}FO}%
\colorbox{green}{\color[gray]{0.75}FO}%
\colorbox{green}{\color[gray]{0.75}FO}%
\colorbox{green}{\color[gray]{0.75}FO}%
\colorbox{green}{\color[gray]{0.75}FO}%
\colorbox{green}{\color[gray]{0.75}FO}%
\colorbox{green}{\color[gray]{0.75}FO}%
\colorbox{green}{\color[gray]{0.75}FO}%
\colorbox{green}{\color[gray]{0.75}FO}%
\colorbox{green}{\color[gray]{0.75}FO}%
\colorbox{green}{\color[gray]{0.75}FO}%
\colorbox{green}{\color[gray]{0.75}FO}%
\colorbox{green}{\color[gray]{0.75}FO}%
\colorbox{green}{\color[gray]{0.75}FO}%
\colorbox{green}{\color[gray]{0.75}FO}%
\colorbox{green}{\color[gray]{0.75}FO}%
\colorbox{green}{\color[gray]{0.75}FO}%
\colorbox{green}{\color[gray]{0.75}FO}%
\colorbox{green}{\color[gray]{0.75}FO}%
\colorbox{green}{\color[gray]{0.75}FO}%
\colorbox{green}{\color[gray]{0.75}FO}%
\colorbox{green}{\color[gray]{0.75}FO}%
\colorbox{green}{\color[gray]{0.75}FO}%
\colorbox{green}{\color[gray]{0.75}FO}%
\colorbox{green}{\color[gray]{0.75}FO}%
\colorbox{green}{\color[gray]{0.75}FO}%
\colorbox{green}{\color[gray]{0.75}FO}%
\colorbox{green}{\color[gray]{0.75}FO}%
\colorbox{green}{\color[gray]{0.75}FO}%
\colorbox{green}{\color[gray]{0.75}FO}%
\colorbox{green}{\color[gray]{0.75}FO}%
\colorbox{green}{\color[gray]{0.75}FO}%
\colorbox{green}{\color[gray]{0.75}FO}%
\colorbox{green}{\color[gray]{0.75}FO}%
\colorbox{green}{\color[gray]{0.75}FO}%
\colorbox{green}{\color[gray]{0.75}FO}%
\colorbox{green}{\color[gray]{0.75}FO}%
\colorbox{green}{\color[gray]{0.75}FO}%
\colorbox{green}{\color[gray]{0.75}FO}%
\colorbox{green}{\color[gray]{0.75}FO}%
\colorbox{green}{\color[gray]{0.75}FO}%
\colorbox{green}{\color[gray]{0.75}FO}%
\colorbox{green}{\color[gray]{0.75}FO}%
\colorbox{green}{\color[gray]{0.75}FO}%
\colorbox{green}{\color[gray]{0.75}FO}%
\colorbox{green}{\color[gray]{0.75}FO}%
\colorbox{green}{\color[gray]{0.75}FO}%
\colorbox{green}{\color[gray]{0.75}FO}%
\colorbox{green}{\color[gray]{0.75}FO}%
\colorbox{green}{\color[gray]{0.75}FO}%
\colorbox{green}{\color[gray]{0.75}FO}%
\colorbox{green}{\color[gray]{0.75}FO}%
\colorbox{green}{\color[gray]{0.75}FO}%
\colorbox{green}{\color[gray]{0.75}FO}%
\colorbox{green}{\color[gray]{0.75}FO}%
\colorbox{green}{\color[gray]{0.75}FO}%
\colorbox{green}{\color[gray]{0.75}FO}%
\colorbox{green}{\color[gray]{0.75}FO}%
\colorbox{green}{\color[gray]{0.75}FO}%
\colorbox{green}{\color[gray]{0.75}FO}%
\colorbox{green}{\color[gray]{0.75}FO}%
\colorbox{green}{\color[gray]{0.75}FO}%
\colorbox{green}{\color[gray]{0.75}FO}%
\colorbox{green}{\color[gray]{0.75}FO}%
\colorbox{green}{\color[gray]{0.75}FO}%
\colorbox{green}{\color[gray]{0.75}FO}%
\colorbox{green}{\color[gray]{0.75}FO}%
\colorbox{green}{\color[gray]{0.75}FO}%
\colorbox{green}{\color[gray]{0.75}FO}%
\colorbox{green}{\color[gray]{0.75}FO}%
\colorbox{green}{\color[gray]{0.75}FO}%
\colorbox{green}{\color[gray]{0.75}FO}%
\colorbox{green}{\color[gray]{0.75}FO}%
\colorbox{green}{\color[gray]{0.75}FO}%
\colorbox{green}{\color[gray]{0.75}FO}%
\colorbox{green}{\color[gray]{0.75}FO}%
\colorbox{green}{\color[gray]{0.75}FO}%
\colorbox{green}{\color[gray]{0.75}FO}%
\colorbox{green}{\color[gray]{0.75}FO}%
\colorbox{green}{\color[gray]{0.75}FO}%
\colorbox{green}{\color[gray]{0.75}FO}%
\colorbox{green}{\color[gray]{0.75}FO}%
\colorbox{green}{\color[gray]{0.75}FO}%
\\
\colorbox{green}{\color[gray]{0.75}FO}%
\colorbox{green}{\color[gray]{0.75}FO}%
\colorbox{green}{\color[gray]{0.75}FO}%
\colorbox{green}{\color[gray]{0.75}FO}%
\colorbox{green}{\color[gray]{0.75}FO}%
\colorbox{green}{\color[gray]{0.75}FO}%
\colorbox{green}{\color[gray]{0.75}FO}%
\colorbox{green}{\color[gray]{0.75}FO}%
\colorbox{green}{\color[gray]{0.75}FO}%
\colorbox{green}{\color[gray]{0.75}FO}%
\colorbox{green}{\color[gray]{0.75}FO}%
\colorbox{green}{\color[gray]{0.75}FO}%
\colorbox{green}{\color[gray]{0.75}FO}%
\colorbox{green}{\color[gray]{0.75}FO}%
\colorbox{green}{\color[gray]{0.75}FO}%
\colorbox{green}{\color[gray]{0.75}FO}%
\colorbox{green}{\color[gray]{0.75}FO}%
\colorbox{green}{\color[gray]{0.75}FO}%
\colorbox{green}{\color[gray]{0.75}FO}%
\colorbox{green}{\color[gray]{0.75}FO}%
\colorbox{green}{\color[gray]{0.75}FO}%
\colorbox{green}{\color[gray]{0.75}FO}%
\colorbox{green}{\color[gray]{0.75}FO}%
\colorbox{green}{\color[gray]{0.75}FO}%
\colorbox{green}{\color[gray]{0.75}FO}%
\colorbox{green}{\color[gray]{0.75}FO}%
\colorbox{green}{\color[gray]{0.75}FO}%
\colorbox{green}{\color[gray]{0.75}FO}%
\colorbox{green}{\color[gray]{0.75}FO}%
\colorbox{green}{\color[gray]{0.75}FO}%
\colorbox{green}{\color[gray]{0.75}FO}%
\colorbox{green}{\color[gray]{0.75}FO}%
\colorbox{green}{\color[gray]{0.75}FO}%
\colorbox{green}{\color[gray]{0.75}FO}%
\colorbox{green}{\color[gray]{0.75}FO}%
\colorbox{green}{\color[gray]{0.75}FO}%
\colorbox{green}{\color[gray]{0.75}FO}%
\colorbox{green}{\color[gray]{0.75}FO}%
\colorbox{green}{\color[gray]{0.75}FO}%
\colorbox{green}{\color[gray]{0.75}FO}%
\colorbox{green}{\color[gray]{0.75}FO}%
\colorbox{green}{\color[gray]{0.75}FO}%
\colorbox{green}{\color[gray]{0.75}FO}%
\colorbox{green}{\color[gray]{0.75}FO}%
\colorbox{green}{\color[gray]{0.75}FO}%
\colorbox{green}{\color[gray]{0.75}FO}%
\colorbox{green}{\color[gray]{0.75}FO}%
\colorbox{green}{\color[gray]{0.75}FO}%
\colorbox{green}{\color[gray]{0.75}FO}%
\colorbox{green}{\color[gray]{0.75}FO}%
\colorbox{green}{\color[gray]{0.75}FO}%
\colorbox{green}{\color[gray]{0.75}FO}%
\colorbox{green}{\color[gray]{0.75}FO}%
\colorbox{green}{\color[gray]{0.75}FO}%
\colorbox{green}{\color[gray]{0.75}FO}%
\colorbox{green}{\color[gray]{0.75}FO}%
\colorbox{green}{\color[gray]{0.75}FO}%
\colorbox{green}{\color[gray]{0.75}FO}%
\colorbox{green}{\color[gray]{0.75}FO}%
\colorbox{green}{\color[gray]{0.75}FO}%
\colorbox{green}{\color[gray]{0.75}FO}%
\colorbox{green}{\color[gray]{0.75}FO}%
\colorbox{green}{\color[gray]{0.75}FO}%
\colorbox{green}{\color[gray]{0.75}FO}%
\colorbox{green}{\color[gray]{0.75}FO}%
\colorbox{green}{\color[gray]{0.75}FO}%
\colorbox{green}{\color[gray]{0.75}FO}%
\colorbox{green}{\color[gray]{0.75}FO}%
\colorbox{green}{\color[gray]{0.75}FO}%
\colorbox{green}{\color[gray]{0.75}FO}%
\colorbox{green}{\color[gray]{0.75}FO}%
\colorbox{green}{\color[gray]{0.75}FO}%
\colorbox{green}{\color[gray]{0.75}FO}%
\colorbox{green}{\color[gray]{0.75}FO}%
\colorbox{green}{\color[gray]{0.75}FO}%
\colorbox{green}{\color[gray]{0.75}FO}%
\colorbox{green}{\color[gray]{0.75}FO}%
\colorbox{green}{\color[gray]{0.75}FO}%
\colorbox{green}{\color[gray]{0.75}FO}%
\colorbox{green}{\color[gray]{0.75}FO}%
\colorbox{green}{\color[gray]{0.75}FO}%
\colorbox{green}{\color[gray]{0.75}FO}%
\colorbox{green}{\color[gray]{0.75}FO}%
\colorbox{green}{\color[gray]{0.75}FO}%
\colorbox{green}{\color[gray]{0.75}FO}%
\colorbox{green}{\color[gray]{0.75}FO}%
\colorbox{green}{\color[gray]{0.75}FO}%
\colorbox{green}{\color[gray]{0.75}FO}%
\colorbox{green}{\color[gray]{0.75}FO}%
\colorbox{green}{\color[gray]{0.75}FO}%
\colorbox{green}{\color[gray]{0.75}FO}%
\colorbox{green}{\color[gray]{0.75}FO}%
\colorbox{green}{\color[gray]{0.75}FO}%
\colorbox{green}{\color[gray]{0.75}FO}%
\colorbox{green}{\color[gray]{0.75}FO}%
\colorbox{green}{\color[gray]{0.75}FO}%
\colorbox{green}{\color[gray]{0.75}FO}%
\colorbox{green}{\color[gray]{0.75}FO}%
\colorbox{green}{\color[gray]{0.75}FO}%
\colorbox{green}{\color[gray]{0.75}FO}%
\\
\colorbox{green}{\color[gray]{0.75}FO}%
\colorbox{green}{\color[gray]{0.75}FO}%
\colorbox{green}{\color[gray]{0.75}FO}%
\colorbox{green}{\color[gray]{0.75}FO}%
\colorbox{green}{\color[gray]{0.75}FO}%
\colorbox{green}{\color[gray]{0.75}FO}%
\colorbox{green}{\color[gray]{0.75}FO}%
\colorbox{green}{\color[gray]{0.75}FO}%
\colorbox{green}{\color[gray]{0.75}FO}%
\colorbox{green}{\color[gray]{0.75}FO}%
\colorbox{green}{\color[gray]{0.75}FO}%
\colorbox{green}{\color[gray]{0.75}FO}%
\colorbox{green}{\color[gray]{0.75}FO}%
\colorbox{green}{\color[gray]{0.75}FO}%
\colorbox{green}{\color[gray]{0.75}FO}%
\colorbox{green}{\color[gray]{0.75}FO}%
\colorbox{green}{\color[gray]{0.75}FO}%
\colorbox{green}{\color[gray]{0.75}FO}%
\colorbox{green}{\color[gray]{0.75}FO}%
\colorbox{green}{\color[gray]{0.75}FO}%
\colorbox{green}{\color[gray]{0.75}FO}%
\colorbox{green}{\color[gray]{0.75}FO}%
\colorbox{green}{\color[gray]{0.75}FO}%
\colorbox{green}{\color[gray]{0.75}FO}%
\colorbox{green}{\color[gray]{0.75}FO}%
\colorbox{green}{\color[gray]{0.75}FO}%
\colorbox{green}{\color[gray]{0.75}FO}%
\colorbox{green}{\color[gray]{0.75}FO}%
\colorbox{green}{\color[gray]{0.75}FO}%
\colorbox{green}{\color[gray]{0.75}FO}%
\colorbox{green}{\color[gray]{0.75}FO}%
\colorbox{green}{\color[gray]{0.75}FO}%
\colorbox{green}{\color[gray]{0.75}FO}%
\colorbox{green}{\color[gray]{0.75}FO}%
\colorbox{green}{\color[gray]{0.75}FO}%
\colorbox{green}{\color[gray]{0.75}FO}%
\colorbox{green}{\color[gray]{0.75}FO}%
\colorbox{green}{\color[gray]{0.75}FO}%
\colorbox{green}{\color[gray]{0.75}FO}%
\colorbox{green}{\color[gray]{0.75}FO}%
\colorbox{green}{\color[gray]{0.75}FO}%
\colorbox{green}{\color[gray]{0.75}FO}%
\colorbox{green}{\color[gray]{0.75}FO}%
\colorbox{green}{\color[gray]{0.75}FO}%
\colorbox{green}{\color[gray]{0.75}FO}%
\colorbox{green}{\color[gray]{0.75}FO}%
\colorbox{green}{\color[gray]{0.75}FO}%
\colorbox{green}{\color[gray]{0.75}FO}%
\colorbox{green}{\color[gray]{0.75}FO}%
\colorbox{green}{\color[gray]{0.75}FO}%
\colorbox{green}{\color[gray]{0.75}FO}%
\colorbox{green}{\color[gray]{0.75}FO}%
\colorbox{green}{\color[gray]{0.75}FO}%
\colorbox{green}{\color[gray]{0.75}FO}%
\colorbox{green}{\color[gray]{0.75}FO}%
\colorbox{green}{\color[gray]{0.75}FO}%
\colorbox{green}{\color[gray]{0.75}FO}%
\colorbox{green}{\color[gray]{0.75}FO}%
\colorbox{green}{\color[gray]{0.75}FO}%
\colorbox{green}{\color[gray]{0.75}FO}%
\colorbox{green}{\color[gray]{0.75}FO}%
\colorbox{green}{\color[gray]{0.75}FO}%
\colorbox{green}{\color[gray]{0.75}FO}%
\colorbox{green}{\color[gray]{0.75}FO}%
\colorbox{green}{\color[gray]{0.75}FO}%
\colorbox{green}{\color[gray]{0.75}FO}%
\colorbox{green}{\color[gray]{0.75}FO}%
\colorbox{green}{\color[gray]{0.75}FO}%
\colorbox{green}{\color[gray]{0.75}FO}%
\colorbox{green}{\color[gray]{0.75}FO}%
\colorbox{green}{\color[gray]{0.75}FO}%
\colorbox{green}{\color[gray]{0.75}FO}%
\colorbox{green}{\color[gray]{0.75}FO}%
\colorbox{green}{\color[gray]{0.75}FO}%
\colorbox{green}{\color[gray]{0.75}FO}%
\colorbox{green}{\color[gray]{0.75}FO}%
\colorbox{green}{\color[gray]{0.75}FO}%
\colorbox{green}{\color[gray]{0.75}FO}%
\colorbox{green}{\color[gray]{0.75}FO}%
\colorbox{green}{\color[gray]{0.75}FO}%
\colorbox{green}{\color[gray]{0.75}FO}%
\colorbox{green}{\color[gray]{0.75}FO}%
\colorbox{green}{\color[gray]{0.75}FO}%
\colorbox{green}{\color[gray]{0.75}FO}%
\colorbox{green}{\color[gray]{0.75}FO}%
\colorbox{green}{\color[gray]{0.75}FO}%
\colorbox{green}{\color[gray]{0.75}FO}%
\colorbox{green}{\color[gray]{0.75}FO}%
\colorbox{green}{\color[gray]{0.75}FO}%
\colorbox{green}{\color[gray]{0.75}FO}%
\colorbox{green}{\color[gray]{0.75}FO}%
\colorbox{green}{\color[gray]{0.75}FO}%
\colorbox{green}{\color[gray]{0.75}FO}%
\colorbox{green}{\color[gray]{0.75}FO}%
\colorbox{green}{\color[gray]{0.75}FO}%
\colorbox{green}{\color[gray]{0.75}FO}%
\colorbox{green}{\color[gray]{0.75}FO}%
\colorbox{green}{\color[gray]{0.75}FO}%
\colorbox{green}{\color[gray]{0.75}FO}%
\colorbox{green}{\color[gray]{0.75}FO}%
\\
\colorbox{green}{\color[gray]{0.75}FO}%
\colorbox{green}{\color[gray]{0.75}FO}%
\colorbox{green}{\color[gray]{0.75}FO}%
\colorbox{green}{\color[gray]{0.75}FO}%
\colorbox{green}{\color[gray]{0.75}FO}%
\colorbox{green}{\color[gray]{0.75}FO}%
\colorbox{green}{\color[gray]{0.75}FO}%
\colorbox{green}{\color[gray]{0.75}FO}%
\colorbox{green}{\color[gray]{0.75}FO}%
\colorbox{green}{\color[gray]{0.75}FO}%
\colorbox{green}{\color[gray]{0.75}FO}%
\colorbox{green}{\color[gray]{0.75}FO}%
\colorbox{green}{\color[gray]{0.75}FO}%
\colorbox{green}{\color[gray]{0.75}FO}%
\colorbox{green}{\color[gray]{0.75}FO}%
\colorbox{green}{\color[gray]{0.75}FO}%
\colorbox{green}{\color[gray]{0.75}FO}%
\colorbox{green}{\color[gray]{0.75}FO}%
\colorbox{green}{\color[gray]{0.75}FO}%
\colorbox{green}{\color[gray]{0.75}FO}%
\colorbox{green}{\color[gray]{0.75}FO}%
\colorbox{green}{\color[gray]{0.75}FO}%
\colorbox{green}{\color[gray]{0.75}FO}%
\colorbox{green}{\color[gray]{0.75}FO}%
\colorbox{green}{\color[gray]{0.75}FO}%
\co
}

\subsection{Quelltext}
{\small
\lstinputlisting{../Aufgabe_1/Buschfeuer.cpp}
}
\newpage
\section{Aufgabe 2 - Lebenslinien}
\subsection{Lösungsidee}
Die \emphpar{Lebenszeit} eines Menschen ist ein abgeschlossenes Intervall $L = [a,b]$ zwischen 2 Zeitpunkten $a, b$. Da es eine Bijektion $J$ gibt, welche jeder Zeit eine reelle Zahl zuordnet, lässt sich die Lebenszeit eines Menschen auch als Intervall $L' = [J(a),J(b)]$ von reellen Zahlen auffassen. Dies wird im Folgenden getan.\\
Ein \emphpar{Lebensgraph} ist ein ungerichteter Graph $G = (V,E)$, auf dem eine Funktion $f$ \small{$ : V \mapsto P(\mathbb{R})$}
\footnote{$P(\mathbb{R})$ beschreibt die Potenzmenge von $\mathbb{R}$, also die Menge aller Teilmengen von $\mathbb{R}$ (Es sei der Einfachheit der Schreibweise wegen angenommen, dass eine solche Potenzmenge existiert.)} definiert ist, welche jedem Knoten eine Lebenszeit eines Menschen, also ein Intervall reeller Zahlen zuordnet und zusätzlich $\forall u,v \in V: (u,v) \in E \Leftrightarrow f(u) \cap f(v) \not= \emptyset $ gilt. Es gibt also genau dann eine Kante zwischen 2 Knoten, wenn der Schnitt der beiden Lebenszeiten der Knoten nicht leer ist, es also einen Zeitpunkt gibt, zudem beide Menschen gelebt haben.

Aufgabe ist es nun, für einen gegebenen ungerichteten Graphen $G = (V,E)$ zu prüfen, ob es eine Funktion $f$ \small{$ : V \mapsto P(\mathbb{R})$} gibt, sodass $G$ Lebensgraph wird.\\
Dabei soll, sofern es ein solches $f$ gibt, $f(v)$ für alle Knoten $v \in V$ ausgegeben werden, andernfalls soll der minimale Teilgraph von $G$ ausgegeben werden, für welchen allein es kein solches $f$ geben kann.

\begin{center}
\begin{figure}[h]
\begin{tikzpicture}[scale=0.75,transform shape]
  \SetVertexNormal[Shape      = rectangle]
  \Vertex[x=10,y=5,L={F. E. Allen, *1932}]{A}
  \Vertex[x=14,y=2,L={M. Mayer, *1975}]{B}
  \Vertex[x=14,y=-2,L={M. Grewing, *19.12.1992}]{C}
  \Vertex[x=10,y=-5,L={A. Borg, 1949-2003}]{D}
  \Vertex[x=6,y=-2,L={A. Goldstine, 1920-1964}]{E}
  \Vertex[x=6,y=2,L={G. Hopper, 1906-1.1.1992}]{F}
  \Vertex[x=0,y=0,L={A. Lovelace, 1815-1852}]{G}
  \tikzstyle{LabelStyle}=[fill=white,sloped]
  \Edge(A)(B)
  \Edge(B)(C)
  \Edge(C)(D)
  \Edge(D)(E)
  \Edge(E)(F)
  \Edge(A)(F)
  \Edge(A)(D)
  \tikzstyle{EdgeStyle}=[bend left]
  \Edge(B)(F)
  \Edge(A)(E)
  \tikzstyle{EdgeStyle}=[bend right]
  \Edge(B)(D)
  \Edge(A)(C)
  \Edge(D)(F)
\end{tikzpicture}
\caption{Der Lebensgraph aus der Aufgabenstellung}
\end{figure}
\end{center}

Im Folgenden wird nur von zusammenhängenden Graphen ausgegangen. Für aus mehreren Zusammenhangskomponenten bestehende Graphen lässt sich die Berechnung für jede dieser einzeln durchführen, eventuell muss allen einer Komponente zugewiesenen Intervalle eine reelle Konstante addiert werden, dies ändert jedoch nichts an der eigentlichen Lösung.

\subsubsection{Eigenschaften von Lebensgraphen}

Es ist leicht ersichtlich, dass ein naiver Algorithmus zur Prüfung eines Graphen auf Lebensgrapheneigenschaft, also ein Algorithmus der alle möglichen zeitlichen Anordnungen der Knoten zueinander  durchprobiert, nicht zum Ziel führt, da dieser mit einer grob approximierten Laufzeit von $\mathcal{O(V!)}$ wohl zu langsam ist.

Zur Überprüfung eines Graphen, ob dieser ein Lebensgraphen ist, ist es daher zunächst hilfreich sich Lebensgraphen etwas genauer zu betrachten. Es fällt zunächst auf, dass ein Graph, in dem ein \emphpar{Loch}\footnote{Ein Loch ist dabei ein Zyklus mit einer Länge größer 3, zwischen dessen einzelnen Knoten nur eine Kante existiert, wenn diese auch im Zyklus existiert.} auftritt niemals Lebensgraph sein kann:
\begin{center}
\begin{figure}[h]
\begin{tikzpicture}[scale=0.75,transform shape]

  \Vertex[x=0,y=0]{A}
  \Vertex[x=4,y=0]{B}
  \Vertex[x=4,y=4]{C}
  \Vertex[x=0,y=4]{D}
  
  \Vertex[x=8,y=1]{E}
  \Vertex[x=11,y=-1]{F}
  \Vertex[x=14,y=1]{G}
  \Vertex[x=12.5,y=4]{H}
  \Vertex[x=9.5,y=4]{I}
  \Vertex[x=10.5,y=3]{J}
  \tikzstyle{LabelStyle}=[fill=white,sloped]
  \Edge(A)(B)
  \Edge(B)(C)
  \Edge(C)(D)
  \Edge(D)(A)
  
  
  \Edge(E)(F)
  \Edge(F)(G)
  \Edge(G)(H)
  \Edge(H)(I)
  \Edge(I)(E)
  \Edge(I)(J)
\end{tikzpicture}
\caption{Graphen mit Löchern können niemals Lebensgraph sein \textit{(im 2. Graphen ist $J$ \textbf{nicht} Teil des Loches)}}
\end{figure}
\end{center}

Der Grund hierfür ist offensichtlich: Sei $Z = (v_i,v_k,...,v_i)$ ein Zyklus der Länge größer 3 eines Graphen $G = (V,E)$, und gelte für $G$: zwischen 2 Knoten aus $Z$ existiert nur genau dann eine Kante in $G$, sofern diese beiden Knoten im Zyklus nacheinander durchlaufen werden; bei $Z$ handelt es sich also um ein Loch von $G$.

Weise man einem Knoten $v_i \in Z$ nun ein Intervall $I_0 = [a_0,b_0]$ zu. 
Nun muss dem Nachfolger $v_{i_1}$ von $v_i$ im Zyklus $Z$ ein Intervall $I_1 = [a_1,b_1]$ zugewiesen werden, wobei entweder $a_0 < a_1 \leq b_0 < b_1$ oder $a_1 < a_0 \leq b_1 < b_0]$ gelten muss, da in $G$ zwischen $v_i$ und $v_{i_1}$ eine Kante existiert. Hat man sich jedoch für einen dieser beiden Fälle entschieden, so muss man sich bei der Zuweisung von Intervallen zu den nächsten Knoten in $Z$ immer für diesen Fall entscheiden. Sonst würde man Intervalle erhalten, welche einen nichtleeren Schnitt besitzen, deren Knoten in $G$ jedoch nicht durch eine Kante verbunden sind. Dies wäre ein Widerspruch zur Definition eines Lebensgraphen.\\
Setzt man diese Zuweisungen jedoch bis zum Ende des Zyklus fort, so erhält man zwangsläufig ein Problem mit der Kante zwischen dem Knoten $v_i$ und seinem Vorgänger im Zyklus $Z$. In jedem Fall muss der Schnitt der diesen beiden Knoten zugewiesenen Intervalle nach Konstruktion leer sein, da man ansonsten bei einem vorangegangenen Schritt einen Widerspruch zur Definition eines Lebensgraphen erhalten hatte. Dies an sich stellt jedoch auch einen Widerspruch dar, da diese beiden Konten in $G$ mit einer Kante verbunden sind.

Somit hat ein Lebensgraph kein Loch. \\
Graphen ohne Löcher werden in der Literatur \emphpar{Chordalgraph} oder \emph{Triangulierter Graph}\footnote{Der englischsprachige Wikipediaartikel ist in diesem Fall (mal wieder) deutlich informativer: \url{https://en.wikipedia.org/wiki/Chordal_graph}} genannt, es gibt effiziente Algorithmen zur Erkennung solcher Graphen.

Es sei an dieser Stelle angemerkt, dass ein Lebensgraph sehr wohl \emphpar{Dreiecke}, also Zyklen der Länge 3 haben darf. Dies liegt insbesondere daran, dass ein Dreieck eine \emphpar{Clique} der Größe 3 bildet, jeder der 3 Knoten also mit jedem anderen der 3 Knoten verbunden ist. Speziell bei Dreiecken muss es also einen Zeitpunkt geben, an dem alle 3 entsprechenden Menschen gelebt haben.

Weiterhin ist es für einen Lebensgraphen nur \emph{notwendig} Chordalgraph zu sein. Betrachte man folgenden Graphen, der Chordalgraph ist, jedoch nicht Lebensgraph sein kann:

\begin{center}
\begin{figure}[h]
\begin{tikzpicture}[scale=0.75,transform shape]

  \Vertex[x=2,y=0]{A}
  \Vertex[x=4,y=0]{B}
  \Vertex[x=8,y=0]{C}
  \Vertex[x=10,y=0]{D}
  \Vertex[x=6,y=0]{G}
  \Vertex[x=6,y=3]{E}
  \Vertex[x=10,y=3]{F}
  
  \tikzstyle{LabelStyle}=[fill=white,sloped]
  \Edge(A)(B)
  \Edge(B)(G)
  \Edge(C)(G)
  \Edge(C)(D)
  \Edge(G)(E)
  \Edge(F)(E)
\end{tikzpicture}
\caption{Chordalgraph, der \textbf{kein} Lebensgraph ist}
\end{figure}
\end{center}

Es gilt nun also ein \emph{hinreichendes Kriterium} dafür zu finden, ob ein Graph $G$ ein Lebensgraph ist.

Seien dazu die \emphpar{maximalen Cliquen} in $G$ betrachtet. Eine Clique $C$ eines Graphen $G = (V,E)$ heißt dabei maximal, wenn es keinen Knoten $v \in V\setminus C$ gibt, sodass $C \cup \{ v \}$ eine Clique in $G$ bildet; wenn also das Hinzufügen eines beliebigen weiteren Knotens des Graphen zu der Clique $C$ bewirkt, dass $C$ keine Clique mehr ist.

Es kann gezeigt werden, dass Chordalgraphen $G_C = (V,E)$ genau diejenigen Graphen sind, bei denen sich eben diese maximalen Cliquen so in einem \emphpar{Cliquenbaum} $T$ anordnen lassen, so dass für jeden Knoten $vSub$ aus $V$ gilt, dass die Cliquen, in denen $v$ enthalten ist einen zusammenhängenden Teilbaum von $T$ bilden.\\
Speziell bei Lebensgraphen vereinfacht sich dieser Baum jedoch zu einem Pfad, man kann die maximalen Cliquen also so anordnen, dass alle Cliquen, die einen Knoten $v$ enthalten in dieser Anordnung aufeinanderfolgend sind. Diese Anordnung sein im Folgenden mit \emphpar{Cliquenkette} beschrieben. Aus einer Cliquenkette lassen sich nun leicht Intervalle für Knoten ablesen, und umgekehrt, da zwei Konten nur in der selben Clique sind, wenn sie durch eine Kante verbunden sind, und ihnen so zurecht ein gemeinsamer Intervallabschnitt zugeordnet worden ist.

\subsubsection{Algorithmische Erkennung von Lebensgraphen}

Der vorangegangene Abschnitt liefert nun einen direkten Algorithmus zur Überprüfung, ob ein Graph ein Lebensgraph ist. Zunächst wird überprüft, ob der gegebene Graph ein Chordalgraph ist, dann wird ein Cliquenbaum erzeugt, überprüft, ob dieser eine Cliquenkette ist und zuletzt wird überprüft, ob jeder Knoten nur in aufeinanderfolgenden Cliquen vorkommt. Die hierzu notwendigen Algorithmen\footnote{nach Habib, M., McConnell, R., Paul, C. und Viennot , L.: \textit{"Lex-BFS and partition reÿnement, with applications to transitive orientation, interval graph recognition and consecutive ones testing"}, erschienen in \texttt{Theoretical Computer Science 234 (2000)}, Seiten 59–84} werden nun im Folgenden vorgestellt.

Die Überprüfung, ob ein gegebener Graph ein Chordalgraph ist, kann mithilfe einer \emph{lexikografischen Breitensuche}\marginpar{Lex-BFS} (im Folgenden Lex-BFS) geschehen. Dabei ist eine Lex-BFS ähnlich einer normalen Breitensuche. Anstatt einer Warteschlange (Queue) verwendet die Lex-BFS jedoch eine geordnete Folge von Knotenmengen. Die Lex-BFS wird speziell dazu benutzt, eine spezielle \emph{Abfolge} der Knoten zu erhalten, mit welcher im Folgenden dann weiter operiert werden kann.

\lstset{language=[Sharp]C}
\begin{lstlisting}
//Lex-BFS
//Eingabe: Graph G = (V,E), Knoten seien durchnummeriert 0..|V|-1
//Ausgabe: Reihenfolge der Knoten

begin
  int[] ausgabe := int[|V|];
  
  Liste<int> L	:= V; //Initiale Anordnung der Knoten (L[i] = i)
  
  Liste<int>[] S := {L};//Klassen
    
  int cnt = |V| - 1;  //Zähler für Ausgabe
  while S != { } do begin
    int x := letztes Element der letzten Klasse in S;
    entferne x aus der letzten Klasse in S,
       wird diese Klasse dadurch leer, entferne diese aus S;
  	
    ausgabe[x] := cnt; cnt := cnt - 1;
    
    //Klassen werden in 2 Teilklassen aufgespalten:
    //diejenigen Knoten, die Nachbar von x sind, 
    //und die, die es nicht sind
 
    foreach Liste<int> i in S do begin
      nachbarn := { Knoten in i, die benachbart zu x };
      nicht_nachbarn := i \ nachbarn;
    	
      //Ordne Nachbarn vor Nicht-Nachbarn in S    
      ersetzte { i } durch { nachbarn , nicht_nachbarn } in S;
   	    //Ignoriere leere Mengen
    end;
  end;
  return ausgabe;
end.
\end{lstlisting}

Eine \emph{perfekte Eliminationsordnung eines Graphen $G = (V,E)$}\marginpar{perfekte Eliminationsordnung} heißt eine Anordnung $A$ der Knoten $V$, sodass für jeden Konten $v \in V$ gilt:\\
$v$ und die Nachbarn von $v$, die nach $v$ in $A$ auftreten, bilden eine \emph{Clique} in $G$.

Ein Satz über Chordalgraphen besagt, dass ein Graph $G$ genau dann ein Chordalgraph ist, wenn $G$ eine perfekte Eliminationsordnung besitzt.\\
Auch kann bewiesen werden, dass die Lex-BFS bei einem Chordalgraphen $G$ eine perfekte Eliminationsordnung von $G$ erzeugt.\\
Speziell für die Überprüfung von Graphen auf Chordalität muss also nur noch die von der Lex-BFS erzeugte Abfolge $PI$ der Knoten darauf hin überprüft werden, ob diese eine perfekte Eliminationsordnung ist.

Dies kann beispielsweise mit folgendem Algorithmus geschehen, dabei seien mit $RN(v)$ die in der Eliminationsornung rechts gelegenen Nachbarknoten von $v$ bezeichnet und mit $P(v)$ der in der Eliminationsornung am weitesten links liegende Knoten von $RN(v)$.

\begin{lstlisting}
//Überprüfung einer Ordnung der Knoten V darauf, ob diese
//eine perfekte Eliminationsordnung ist
//Eingabe: Graph G = (V,E), Reihenfolge PI der Knoten V
//Ausgabe: true, falls PI perfekte Eliminationsordnung, false sonst
begin
  Ermittle RN(v) und P(v) für jeden Knoten v;
 
  foreach v in V do begin
    if ( RN(v) \ P(v) ist keine
      Teilmenge von RN(P(v)))
        then return false;
  end;
  return true;
end.
\end{lstlisting}

Obiger Algorithmus nutzt aus, dass bei einer perfekten Eliminationsordnung für jeden Knoten $v$ gilt, dass ${v} \cup RN(v)$ eine Clique bildet, somit muss auch $RN(v)$ eine Clique bilden.\\
Deshalb muss $RN(v) \setminus P(v) \subseteq RN(P(v))$ für jeden Knoten $v$ gelten.\\
Sollte es sich bei $PI$ nicht um eine perfekte Eliminationsordnung, so gibt es ein $v$, für dass ${v} \cup RN(v)$ keine Clique ist. Dies heißt aber im speziellen, dass $RN(v)$ keine Clique bilden und somit  $P(v)$ nicht mit allen Knoten aus $RN(v) \setminus P(v)$ durch eine Kante verbunden ist oder $RN(v) \setminus P(v)$ selbst keine Clique bildet. Der zweite Fall kann dann rekursiv weiter behandelt werden, bis einmal Fall 1 auftritt, dieser muss auftreten, da der betrachtete Graph endlich ist.\\
Speziell bei Fall 1 enthält dann $RN(P(v))$ jedoch mindestens einen Knoten nicht, der in $RN(v) \setminus P(v)$ enthalten ist.\\
Somit ist obiger Algorithmus also korrekt.

Ist der eingegebene Graph nun ein Chordalgraph, so wird nun der Cliquenbaum erzeugt. Dazu kann man, das schon für den vorangegangenen Algorithmus für jeden Knoten $v$ definierte, $P(v)$ nutzen. Es ist leicht zu sehen, dass dieses $P(v)$ schon einen Baum impliziert. Es werden nun also einfach die maximalen Cliquen ermittelt und dann entsprechend dieses Baumes geordnet:

\begin{lstlisting}
//Ermittlung des Cliquenbaumes
//Eingabe: Graph G = (V,E), Reihenfolge PI der Knoten V
//Ausgabe: Ein Cliquenbaum B
begin
  Ermittle RN(v) und P(v) für jeden Knoten v;
 
  Sei T der durch P(v) implizierte Baum;
  Sei r die Wurzel von T;
  
  Sei Clique ein Array, das Knoten eine Clique zuordnet;
  
  foreach v in T, v != r in preorder do begin
    if( RN(v) \ {P(v)} != RN(P(v)) ) then begin
      Sei c := {v} eine neue Clique;
      Clique[v] := c;
      PAR(c) := Clique[P(v)];
    end else begin
      Clique[P(v)] += { v };
      Clique[v] := Clique[P(v)];
    end;    
  end;
  Sei B der durch PAR(C) implizierte Baum;
  return B;   
end.
\end{lstlisting}

Anschließend wird nun versucht, die maximalen Cliquen zu ordnen, sodass eine Cliquenkette entsteht. Der naive Ansatz, alle möglichen Anordnungen durchzuprobieren ist allerdings zu langsam.\\
Eine solche Anordnung kann auch mit folgendem, der Lex-BFS sehr ähnlichen, Algorithmus von M. Habib, R. McConnell, C. Paul und L. Viennot geschehen, der diesen Test bei richtiger Implementierung in Linearzeit durchführen kann:

\begin{lstlisting}
//Lebensgraphen-Test
//Ermittlung einer Cliquenkette
//Eingabe: Graph G = (V,E), Reihenfolge PI der Knoten V
//Ausgabe: Eine Cliquenkette L
begin
  Sei B=(X,F) ein Cliquenbaum der mit dem vorangegangenen Algorithmus gefunden wurde;
  Sei X die Menge der maximalen Cliquen, X = {C1, C2, ... Cn};
  Sei L eine Liste von Mengen, L := ( X );
  Sei PIVOTS ein leerer Stack;
  Sei USED ein Array;
  
  while L enthält eine Menge Xc mit |Xc| > 1 do begin
    Sei b eine Menge;
    if PIVOTS == { } then begin
      Sei Cl die Clique in Xc mit der größten Nummer;
      Ersetze Xc durch Xc\{Cl}, {Cl} in L;
      b := {Cl};
    end else begin
      while USED[PIVOTS.top()] == TRUE do
        PIVOTS.pop();
      x := PIVOTS.top();
      USED[x] := TRUE;
      b := { W aus X | x in W };
      Seien Xa und Xb die erste bzw letzte Menge in L,
         die eine Klasse enthält die auch in b vorhanden ist;
      Ersetze Xa durch Xa\b, (Xa geschnitten b)
        und Xb durch (Xb geschnitten b), Xb\b;
    end;
    foreach (Ci,Cj) in F mit Ci in b und Cj nicht in b do begin
      PIVOTS += (Ci geschnitten Cj);
      entferne (Ci,Cj) aus F;
    end;
  end;
  
  foreach v in V do begin
    if Cliquen, in denen v vorkommt 
       sind nicht aufeinanderfolgend in L then
       return "G ist kein Lebensgraph";
  end;
  //L ist nun die Cliquenkette
  return "G ist ein Lebensgraph";
end.
\end{lstlisting}

Ist diese Anordnung möglich, so handelt es sich um einen Lebensgraphen, andernfalls nicht.
Zuallerletzt sollten nun noch die eigentlichen Intervalle für die Knoten ausgegeben werden. Dabei ist es nicht von Bedeutung, ob und wie nun Daten als Begrenzung für die Intervalle angegeben werden, oder ob einfach Zahlen ausgegeben werden. Dabei ist das eigentliche Finden der Intervalle aus einer Cliquenkette trivial realisierbar,
 da eine Cliquenkette schon Intervalle impliziert.
 
\subsubsection{Erkennung des kleinsten Teilgraphen, der kein Lebensgraph ist} 

Zunächst ist es einfach zu erkennen, dass es nach obigem Algorithmus 2 verschiedene Arten von Graphen gibt, die keine Lebensgraphen sind. Es gibt diejenigen Graphen, die keine Chordalgraphen sind und diejenigen Graphen, die zwar Chordalgraphen sind, die jedoch trotzdem keine Lebensgraphen sind.

Weiterhin ist es leicht zu sehen, dass jeder Graph mit 3 oder weniger Knoten ein Lebensgraph ist.

Ein naheliegender Ansatz zur Findung des kleinsten Teilgraphen ist es daher alle möglichen Teilgraphen der Größe nach zu überprüfen, ob diese kein Lebensgraph sind. Dabei können schon von vorne herein gewisse Teilmengen ausgeschlossen werden. So ist es nicht sinnvoll, Cliquen zu überprüfen, auch sollte der Teilgraph zusammenhängend sein.

 
\paragraph{Laufzeitanalyse}

Bei geschickter Implementierung kann erreicht werden, dass jeder der obigen Algorithmen eine Laufzeit von $\mathcal{O}(|V| + |E|)$ besitzt. Diese Schranke erfüllt auch die geforderte Effizienz des Verfahrens.

Das Ausprobieren aller Teilmengen einer Menge $V$ benötigt $\mathcal{O}(2^{|V|})$, diese Schranke ergibt sich also speziell auch für die naheliegende Variante des Findens des minimalen Teilgraphen, der kein Lebensgraph ist.

\subsection{Weitere Lösungsideen}

Neben der Erkennung mithilfe einer Lex-BFS sind durchaus auch andere Methoden denkbar. So kann man auch mithilfe eines PQ-Baumes diese Erkennung vornehmen\footnote{Booth, K., Lueker, S.: Testing for the Consecutive Ones Property, Interval Graphs, and Graph Planarity Using PQ-Tree Algorithms in Journal of Computer and System Sciences 13, 335--379 (1976)}. Es sind auch noch weitere Verfahren leicht im Internet auffindbar. Alle diese Verfahren eint jedoch, dass sie eine polynomielle Laufzeit für die Erkennung von Lebensgraphen besitzen, das gefundene Verfahren sollte daher auf jeden Fall auch eine solche Schranke für die Laufzeit besitzen.

Für das Finden des kleinsten Teilgraphen, der kein Lebensgraph ist, wären auch Heuristiken denkbar. Auch können die speziellen Eigenschaften von Chordalgraphen und Lebensgraphen ausgenutzt werden, um eine Suche stärker zu prunen.

\subsection{Erweiterungen}

Diese Aufgabe scheint zunächst in sich abgeschlossen, einige kleinere  Erweiterungen sind dennoch denkbar. Zum Einen wäre es denkbar, Graphen die keine Lebensgraphen sind so zu erweitern, dass sie Lebensgraphen werden. Dabei wäre allerdings nur die minimale Erweiterung interessant, jeder Graph kann schließlich zu einem vollständigen Graphen erweitert werden.\\
Zum Anderen können Menschen (in der Regel) nicht unbegrenzt leben. Man könnte also auch eine maximale Lebenszeit festlegen, diese Festlegung benötigt jedoch noch weitere Einschränkungen, um sinnvoll zu sein. Schließlich könnten die Grenzen aller Intervalle einfach mit einer reellen Konstanten multipliziert werden um jedwede Schranke zu unterschreiten. Daher könnte man Lebenszeiten auch nur auf ganzen Zahlen definieren um dies zu unterbinden.

\subsection{Umsetzung}

Zur Implementierung der Lösungsidee habe ich die Sprache \texttt{C++} gewählt.

Dabei finden sich alle Methoden in der Datei \texttt{Lebenslinien.cpp}. Dabei ist der Graph global als Adjazenzmatrix \texttt{G} gespeichert.

Das Lesen der Eingabe übernimmt die Methode \texttt{readInput}. Ist die Eingabe gelesen, so wird eine Ordnung der Knoten mithilfe einer Lex-BFS ermittelt, diese ist in der Methode \texttt{LexBFSOrder} implementiert. Die Überprüfung dieser Ordnung darauf, eine perfekte Eliminationsordnung zu sein geschieht in der Methode \texttt{isChordal}. Sofern die Ordnung dieser Überprüfung standhält wird ein Cliquenbaum ermittelt, implementiert in der Methode \texttt{getCliqueTree}. Abschließend wird noch überprüft, ob sich aus dem Cliquen des Cliquenbaumes auch eine Cliquenkette machen lässt. Dies ist in der Methode \texttt{isIntervalGraph} wiederzufinden. Die Ermittlung und Ausgabe der eigentlichen Intervalle übernimmt die Methode \texttt{main}.\\
Dabei kann aufgrund von naiver Implementierung nur eine Schranke von $\mathcal{O}((|V| + |E|)²)$ eingehalten werden.

Handelt es sich nicht um einen Lebensgraphen, so übernimmt die Methode \texttt{smallestFailingSubgraph} das Finden des minimalen Teilgraphen, der kein Lebensgraph ist. Auch in diesem Fall übernimmt die \texttt{main}-Methode die Ausgabe.

\subsubsection{Eingabeformat}

Die Eingabe in mein Programm kann über die Standardeingabe erfolgen. Dabei wird zunächst die Anzahl $N$ an Knoten in dem Folgenden Graphen angegeben. Darauf folgen $N²$ Zahlen, entweder 0 oder 1 und durch Leerzeichen getrennt; der Graph als Adjazenzmatrix. (Dabei steht eine 1 dafür, dass eine Kante zwischen den beiden entsprechenden Knoten existiert.)
 
\subsection{Beispiele}
\subsubsection*{Beispiel 0}
Das Beispiel aus der Aufgabenstellung.\footnote{Dieses Beispiel lässt sich auch in der Datei 0.in wiederfinden.}

{\small
\lstinputlisting{../Aufgabe_2/0.in}
}

Eine mögliche Belegung mit Intervallen\footnote{Diese Ausgabe lässt sich auch in der Datei 0.out wiederfinden.}:

{\small
\lstinputlisting{../Aufgabe_2/0.out}
}

Mein Programm benötigt für die Berechnung dieses Beispiels weniger als eine Sekunde.

Eine Visualisierung:

\begin{center}
\begin{figure}[h]
\begin{tikzpicture}[scale=0.75,transform shape]
  \Vertex[x=10,y=5,L={1}]{A}
  \Vertex[x=14,y=2,L={2}]{B}
  \Vertex[x=14,y=-2,L={3}]{C}
  \Vertex[x=10,y=-5,L={4}]{D}
  \Vertex[x=6,y=-2,L={5}]{E}
  \Vertex[x=6,y=2,L={0}]{F}
  \Vertex[x=0,y=0,L={6}]{G}
  \tikzstyle{LabelStyle}=[fill=white,sloped]
  \Edge(A)(B)
  \Edge(B)(C)
  \Edge(C)(D)
  \Edge(D)(E)
  \Edge(E)(F)
  \Edge(A)(F)
  \Edge(A)(D)
  \tikzstyle{EdgeStyle}=[bend left]
  \Edge(B)(F)
  \Edge(A)(E)
  \tikzstyle{EdgeStyle}=[bend right]
  \Edge(B)(D)
  \Edge(A)(C)
  \Edge(D)(F)
\end{tikzpicture}
\caption{Der Graph aus Beispiel 0.}
\end{figure}
\end{center}


\subsubsection*{Beispiel 1}
Ein Graph der kein Chordalgraph ist.\footnote{Dieses Beispiel lässt sich auch in der Datei 1.in wiederfinden.}

{\small
\lstinputlisting{../Aufgabe_2/1.in}
}

Die Ausgabe meines Programms\footnote{Diese Ausgabe lässt sich auch in der Datei 1.out wiederfinden.}:

{\small
\lstinputlisting{../Aufgabe_2/1.out}
}

Mein Programm benötigt für die Berechnung dieses Beispiels weniger als eine Sekunde.

Eine Visualisierung:
\begin{center}
\begin{figure}[h]
\begin{tikzpicture}[scale=0.75,transform shape]
  \Vertex[x=0,y=0,L={0}]{A}
  \Vertex[x=4,y=0,L={1}]{B}
  \Vertex[x=4,y=4,L={2}]{C}
  \Vertex[x=0,y=4,L={3}]{D}
  
  \tikzstyle{LabelStyle}=[fill=white,sloped]
  \Edge(A)(B)
  \Edge(B)(C)
  \Edge(C)(D)
  \Edge(D)(A)
  
\end{tikzpicture}
\caption{Der Graph aus Beispiel 1.}
\end{figure}
\end{center}

\subsubsection*{Beispiel 2}
Ein weiterer Graph, der nicht chordal ist\footnote{Dieses Beispiel lässt sich auch in der Datei 2.in wiederfinden.}

{\small
\lstinputlisting{../Aufgabe_2/2.in}
}

Die Ausgabe meines Programms\footnote{Diese Ausgabe lässt sich auch in der Datei 2.out wiederfinden.}:

{\small
\lstinputlisting{../Aufgabe_2/2.out}
}

Mein Programm benötigt für die Berechnung dieses Beispiels weniger als eine Sekunde.

Eine Visualisierung:
\begin{center}
\begin{figure}[h]
\begin{tikzpicture}[scale=0.75,transform shape]
  \Vertex[x=8,y=1,L={0}]{E}
  \Vertex[x=11,y=-1,L={1}]{F}
  \Vertex[x=14,y=1,L={2}]{G}
  \Vertex[x=12.5,y=4,L={3}]{H}
  \Vertex[x=9.5,y=4,L={4}]{I}
  \Vertex[x=10.5,y=3,L={5}]{J}
  \tikzstyle{LabelStyle}=[fill=white,sloped]
  
  \Edge(E)(F)
  \Edge(F)(G)
  \Edge(G)(H)
  \Edge(H)(I)
  \Edge(I)(E)
  \Edge(I)(J)
\end{tikzpicture}
\caption{Der Graph aus Beispiel 2.}
\end{figure}
\end{center}

\subsubsection*{Beispiel 3}
Ein Dreieck, das offenkundig ein Lebensgraph ist\footnote{Dieses Beispiel lässt sich auch in der Datei a.in wiederfinden.}

{\small
\lstinputlisting{../Aufgabe_2/a.in}
}

Eine gefundene Belegung mit Intervallen\footnote{Diese Ausgabe lässt sich auch in der Datei a.out wiederfinden.}:

{\small
\lstinputlisting{../Aufgabe_2/a.out}
}

Mein Programm benötigt für die Berechnung dieses Beispiels weniger als eine Sekunde.

Eine Visualisierung:
\begin{center}
\begin{figure}[h]
\begin{tikzpicture}[scale=0.75,transform shape]
  \Vertex[x=-1,y=1,L={0}]{E}
  \Vertex[x=0,y=-1,L={1}]{F}
  \Vertex[x=1,y=1,L={2}]{G}
  \tikzstyle{LabelStyle}=[fill=white,sloped]
  
  \Edge(E)(F)
  \Edge(F)(G)
  \Edge(G)(E)
\end{tikzpicture}
\caption{Der Graph aus Beispiel 3.}
\end{figure}
\end{center}

\subsubsection*{Beispiel 4}
Ein weiteres Beisiel.\footnote{Dieses Beispiel lässt sich auch in der Datei 6.in wiederfinden.}

{\small
\lstinputlisting{../Aufgabe_2/6.in}
}

Auch dieser Graph enthält ein Loch.\footnote{Diese Ausgabe lässt sich auch in der Datei 6.out wiederfinden.}:

{\small
\lstinputlisting{../Aufgabe_2/6.out}
}

Mein Programm benötigt für die Berechnung dieses Beispiels weniger als eine Sekunde.

\subsubsection*{Beispiel 5}
Ein weiteres Beisiel für einen Lebensgraphen.\footnote{Dieses Beispiel lässt sich auch in der Datei 7.in wiederfinden.}

{\small
\lstinputlisting{../Aufgabe_2/7.in}
}

Die gefundene Belegung mit den Intervallen\footnote{Diese Ausgabe lässt sich auch in der Datei 7.out wiederfinden.}:

{\small
\lstinputlisting{../Aufgabe_2/7.out}
}

Mein Programm benötigt für die Berechnung dieses Beispiels weniger als eine Sekunde.

\subsubsection*{Beispiel 5}
Ein weiteres Beisiel für einen Lebensgraphen.\footnote{Dieses Beispiel lässt sich auch in der Datei 3.in wiederfinden.}

{\small
\lstinputlisting{../Aufgabe_2/3.in}
}

Die gefundene Belegung mit den Intervallen\footnote{Diese Ausgabe lässt sich auch in der Datei 3.out wiederfinden.}:

{\small
\lstinputlisting{../Aufgabe_2/3.out}
}

Mein Programm benötigt für die Berechnung dieses Beispiels weniger als eine Sekunde.


\subsection{Quelltext}

{\small
\lstinputlisting{../Aufgabe_2/Lebenslinien.cpp}
}

\subsection{Bewertungskriterien}

\begin{itemize}
\item Für die Erkennung von Lebensgraphen bzw. \emph{Intervallgraphen} lassen sich leicht Algorithmen im Internet finden. Das zur Erkennung dieser verwendete Verfahren sollte daher auf jeden Fall nicht wesentlich langsamer sein als solche leicht findbaren Algorithmen.
\item Weiterhin sollte die Laufzeit angegeben und (nicht notwendigerweise formal) begründet werden. Konkrete Laufzeiten für Beispiele sollten dazu auch nicht fehlen, speziell bei Beispielen, bei denen die Eingabe kein Lebensgraph ist.
\item Auch sollte bei der Verwendung von Werken aus dem Internet nicht auf eine angemessene Referenzierung verzichtet werden.
\item Das beschriebene Verfahren sollte implementiert worden sein, die Funktionalität sollte anhand von aussagekräftigen Beispielen dokumentiert worden sein.
\item Auch sollten die minimalen Teilgraphen korrekt ausgegeben werden, handelt es sich nicht um einen Lebensgraphen. Mögliche Einschränkungen in der Optimalität dieser Ausgabe sind mit einer entsprechenden Begründung durch die schlechte Laufzeit eines Brute-Force-Ansatzes ebenfalls akzeptabel.
\item Eine grafische Ausgabe einer Zuweisung von Lebenszeiten zu Knoten bzw. der Teilgraphen ist zwar schön, aber nicht gefordert; auf jeden Fall sollte die Ausgabe leicht verständlich sein.
\end{itemize}

\newpage
\end{document} 
