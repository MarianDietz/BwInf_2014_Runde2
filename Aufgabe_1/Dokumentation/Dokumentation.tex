\section{Aufgabe 1 - Buschfeuer}
\subsection{Lösungsidee}

Ein \emphpar{Feld} ist ein quadratisches Stück Land, welches genau einen folgender Zustände inne haben kann:
\begin{itemize}
\item[BRENNBAR] Das Stück Land ist ind der Lage, zu brennen.
\item[BRENNEND] Ein brennendes Stück Land.
\item[GELÖSCHT] Ein Stück Land, welches nie wieder brennen  wird.
\item[LEER] Ein leeres Stück Land.
\end{itemize}

Alle Felder haben die selbe Fläche.

Ein \emphpar{Wald} ist nun die rechteckig-gitterförmige Anordnung von $n\times m$ Feldern. Die \emphpar{Umgebung} $U(f)$ eines Feldes $f$ in einem Wald $W$ ist dabei die Menge an Feldern, welche in $W$ eine gemeinsame Kante mit $f$ haben.

Der Wald wird nun diskret beobachtet. Es ist dabei sichergestellt, dass nur sofern ein Feld bei einer Beobachtung brennend ist, dieses und jedes brennbare Feld seiner Umgebung bei der nächsten Beobachtung brennen werden, sofern diese nicht schon brennen. Diese Eigenschaft des Waldes sei mit \emph{Feuerausbreitung} bezeichnet.

Ab der 2. Beobachtung kann pro Beobachtung genau 1 (brennendes) Feld gelöscht werden. Wird ein brennendes Feld gelöscht, so fängt seine Umgebung bis zur nächsten Beobachtung nicht an zu brennen.\\
Die erste Beobachtung, ab der ein Feld $f$ brennt, heiße \emph{Entflammung} von $f$.


Ziel ist es nun, eine Folge von zu löschenden Feldern anzugeben, sodass bei deren Einhaltung die Anzahl der brennenden Felder minimiert wird.\\


Im Folgenden seinen diejenigen Felder, welche bei mindestens 2 Beobachtungen brennend waren, als \emph{verkohlt} bezeichnet.\\
Nach der Feuerausbreiteung muss jedes Feld der Umgebung eines verkohlten Feldes $c$ brennend sein oder gewesen sein oder seit der Entflammung von $c$ nicht brennbar gewesen sein.

Sei nun zunächst der Fall betrachtet, dass nur brennende Feler gelöscht werden können.

Es ist leicht zu erkennen, dass es die Lösung nicht verschlechtert, wenn ab der 2. Beobachtung bei jeder Beobachtung 1 brennendes Feld gelöscht wird. Daher wird im Folgenden davon ausgegangen, dass bei jeder Beobachtung (ab der 2.) 1 brennendes Feld gelöscht wird. Es gilt nun also für jede dieser Beobachtungen dasjenige brennende Feld zu finden, durch dessen Löschung die Anzahl der im Folgenden (nicht unbedingt umittelbar folgend) zu brennen anfangenden Felder minimiert.

Sei nun eine Beobachtung fixiert.\\
Nun soll für ein brennendes Feld $F$ ein Maß $\mu(F)$ dafür gefunden werden, mit dem bestimmt werden kann, welches Feld zum Löschen in obigem Sinne am Besten ist. Sei $\mu(F)$ daher die Anzahl der brennbaren Felder, zu denen $F$ das brennende Feld mit dem \emph{kleinster Abstand} ist. Dieser kürzeste Abstand ist dabei die minimale Anzahl an Beobachtungen, bis das Feld anfängt zu brennen. (Unter der Annahme, dass keine weiteren Felder gelöscht werden.)\\
Löscht man nun $F$, so wird der kleinste Abstand aller Felder höchstens größer; bei allen Feldern, bei deren kürzestem Abstand $F$ jedoch keine Rolle spielte (bei denen der Abstand zu einem anderen brennenden Feld also kleiner oder gleich dem Abstand zu $F$ ist), tritt keine Veränderung auf.\\
Für 2 Werte $\mu(F_1)$ und $\mu(F_2)$ gilt nun: ist $\mu(F_1) < \mu(F_2)$, so erzeugte $F_2$ bei mehr Feldern eine Vergrößerung des kleinster Abstands als $F_1$.\\
Die \emph{minimale Lebenszeit} eines Feldes sei nun eben der kleinste Abstand zu einem brennenden Feld. Es ist leicht zu erkennen, dass nach mindesten so vielen Beobachtungen, wie die minimale Lebenszeit eines Feldes ist, das Feld zu brennen beginnt.\\
$\mu(F)$ gibt also auch die Anzahl der Felder an, deren minimale Lebenszeit allein durch $F$ besitmmt ist. Löscht man $F$, so wird, wie schon gesehen, die minimale Lebenszeit all dieser Felder höchstens größer, es ist also am Besten, dasjenige Feld $F^\star$ zum Löschen auszuwählen, welches $\mu(\cdot)$ für alle aktuell brennenden Felder maximiert.

Es gilt nun noch $\mu$ effizient zu bestimmen. Da ein Wald eine rechteckige Gitterform besitzt, ist der kürzeste Abstand zwischen 2 Feldern 1, ganau dann, wenn diese Felder eine gemeinsame Kante haben.\\
Fasse man das Gitter nun als Graphen auf, wobei die Felder die Knoten sind und zwischen 2 Knoten eine Kante ist, genau dann, wenn zwischen diesen Feldern eine Kante ist. Es nun offensichtlich, dass dieser Graph ungewichtet und ungerichtet  ist. Somit ist das Finden von kleinsten Abständen mittels einer \emph{Breitensuche} möglich.\\
Dabei sind die Startfelder der Breitensuche die brennenden Felder. Dabei muss für jedes dieser brennenden Felder eine eigene Breitensuche gestartet werden; wobei für alle Breitensuchen gemeinsam die ermittelten kleinsten Abstände gespeichert werden müssen. Zusätzlich zu den kleinsten Abständen müssen auch die dazugehörigen brennenden Felder gespeichert werden, von denen pro Feld eventuell mehr als 1 existiert. Weiterhin muss die Breitensuche nur brennbare Felder besuchen.\\
Sind die kleinsten Abstände gefunden, so kann $\mu$ ermittelt werden, mithilfe simplem durchiterieren über alle Felder und gleichzeitigem Zählen der Felder, für die nur 1 brennendes Feld gespeichert wurde.

In Pseudocode:
{\small
\begin{lstlisting}
Wald	; //Der Wald; ein 2D-Container

AnfangsBrennendeFelder()	{ //Ermittelt die von Anfang brennenden Felder
  brennendeFelder := null; //1D-Container für Positionen brennender Felder
  for (i = 0..Wald.Höhe())
    for (j = 0..Wald.Breite())
       if (Wald[i,j] == BRENNEND)
         brennendeFelder.Add((i;j)); //Gefundene Position hinzufügen
         
  return brennendeFelder; //Alle gefundenen Positionen zurückgeben
}

NächsteBeobachtung(aktBrennendeFelder) { //Ermittelt die bei der nächsten Beonachtung brennenden Felder, aus den Feldern, die aktuell brennen
  neuBrennendeFelder := null;
  for all((x;y) from aktBrennendeFelder)
    if(Wald[x,y] == GELÖSCHT)
      continue; //Feld kann kein Feuer verteilen
      
    Wald[x,y] := VERKOHLT; //2 mal brennende Felder sind verkohlt
    for all((x';y') from Umgebung((x;y)))
      if(Wald[x',y'] == BRENNBAR)
        neuBrennendeFelder.Add((x';y')); //Gefundene Position hinzufügen
        Wald[x',y'] := BRENNEND; //Wald beginnt zu brennen
        
  return neuBrennendeFelder;
}

GetOptBewässerungspunkt(aktBrennendeFelder) { //Ermittelt den besten Bewässerungspunkt
  kleinsterAbstand := null; //Speichert für alle Felder des Waldes den kleinsten Abstand zu jedem Feld aus aktBrennendeFelder
  
  for(i = 0..kleinsterAbstand.Size())
  	Fülle kleinsterAbstand[i] mithilfe einer Breitensuche

  anzEindeutigKleinstAbstände := null;
  
  for (i = 0..Wald.Höhe())
    for (j = 0..Wald.Breite())
      if(Es ex. k mit kleinsterAbstand[k][i,j] eindeutiges Minimum für alle mögliche k)
        anzEindeutigKleinstAbstände[k]++;
  
  return aktBrennendeFelder[k, sodass anzEindeutigKleinstAbstände[k] maximal];
}

SimuliereFeuer() { //Die eigentliche Berechnung
  aktBrennendeFelder := AnfangsBrennendeFelder(); //Anfangs interessante Felder; Kann brennende, von Feuer umschlossene Felder beinhalten
  while(!aktBrennendeFelder.Empty()) //Solange es brennende Felder gibt
    aktBrennendeFelder := NächsteBeobachtung( aktBrennendeFelder) //Ermittle die bei nächster Beobachtung brennenden Felder
    	if(aktBrennendeFelder.Empty())
    	  break;	 //Keine Felder brennen mehr
    	
  Wald[GetOptBewässerungspunkt(aktBrennendeFelder)] := GELÖSCHT; //Lösche das aktuell beste Feld
}
\end{lstlisting}
}

\subsubsection{Korrektheit}
Wie schon beschrieben, wird bei jeder Beobachtung das für diese Beobachtung nach $\mu$ beste Feld zum Löschen ausgewählt.\\
Es gilt also zu zeigen, dass insgesamt nicht weniger Felder abbrennen, sollte bei einer Beobachtung nicht das für diese Beobachtung nach $\mu$ optimalste Feld gelöscht werden. Verallgemeinernd muss gezeigt werden, dass kein $\mu'$ existiert, welches bei midestens 1 Beobachtung 1 anderes Feld als $\mu$ vorschlägt und bei der insgesamt weniger Felder abbrennen als bei $\mu$; dass $\mu$ also \emph{optimal} ist.\\
Außerdem muss gezeigt werden, dass der Algorithmus terminiert. Da der Algorithmus jedoch nur brennende und nicht verkohlte Felder betrachtet und jedes brennendes Feld nach endlicher Zeit ist den Zustand verkohlt übergeht, gibt es einen Zeitpunkt, ab dem alle einst brennenden Felder vekohlt sind. Dann gibt es jedoch keine Felder, auf denen der Algorithmus operieren kann, der Algorithmus terminiert dann, und somit immer.

Nach der Definition von $\mu$ wird dasjenige, beliebige Feld $F_i$ aus allen möglichen Feldern $F_1..F_n$ zum Löschen ausgewählt, welches die minimale Lebenszeit von den meisten Feldern erhöht. Wählte man ein beliebiges Feld $F_i^<$ aus $\{F_1,...,F_n\}$, mit $\mu(F_i^<) < \mu(F_i)$ so erhöht sich nach Definition der minimalen Lebenszeit diese bei $\mu(F_i) - \mu(F_i^<) > 0$ Feldern weniger, als wenn man $F_i$ wählte. Erhalten diese Felder in den nächsten $z_o$ Beobachtungen keine Lebenszeitvelängerung, so brennen sie ab, wobei $z_o$ der kleinste Abstand des Feldes $o$ zu $F_i$ ist.\\
Es verbleibt also zu zeigen, dass es keine Situation geben kann, bei der die Wahl von $F_i^<$ zu einer insgesamt geringeren Anzahl an verbrannten Feldern führt.\\
Angenommen es gäbe solch eine Situation.\\
Dies heißt jedoch, dass es eine oder mehrere Löschungen von Feldern gibt, welche insgesamt dazu führen, dass die Lebenszeit von $\mu(F_i) - \mu(F_i^<) + 1$ Feldern verlängert wird. Außerdem dürfen diese Löschungen nicht möglich sein, wenn $F_i$ anstatt $F_i^<$ gelöscht wird. Dies im Speziellen heißt jedoch, dass Felder gelöscht werden, welche sonst durch die Löschung von $F_i$ eine Lebenszeitverlängerung erhielten. Somit wäre es aber besser gewesen, $F_i$ zu löschen, da bei den Beobachtungen danach auch andere Felder gelöscht werden könnten und die insgesamte Anzahl an verbrannten Feldern so insgesamt gesunken wäre.\footnote{Es ist theoretisch möglich, dass die Wahl zwischen $F_i^<$ und $F_i$ keinen Unterschied macht, beispielsweise, wenn alle Felder innerhalb der nächsten $o$ Beobachtungen verkohlen oder gelöscht werden. Dabei sei $o$ der maximale Abstand, der in $\mu(F_i^<)$ Berücksichtigung fand. Dies stellt jedoch keinen Widerspruch zur Behauptung dar.}
Es ist also optimal, ein Feld mit maximalem $\mu(F_i)$ auszuwählen. Bleibt zu zeigen, dass die Wahl eines speziellen $F_i$ mit maximalem $\mu(F_i)$ an der insgesamten Anzahl an verbrannten Feldern nichts ändert.\\
Sei $F_i^=$ ein beliebiges Feld aus $\{F_1,...,F_n\}$, mit $\mu(F_i^=) = \mu(F_i)$ und $F_i^= \not= F_i$.\footnote{Sei vorrausgestzt, dass ein solches $F_i^=$ existiert. Andernfalls existiert dieser Fall nicht, der Beweis ist dann hier beendet.}\\
Es genügt zu zeigen, dass das Wählen von $\mu(F_i^=)$ keine Verringerung der am Ende insgesamt brennenden Felder gegenüber  $\mu(F_i)$ darstellt, da $\mu(F_i)$ und $\mu(F_i^=)$ beliebig gewählt sind.\\
Angenommen dies sei der Fall.\\


Der Algorithmus ist also korrekt und optimal.

\subsubsection{Laufzeitanalyse}
Eine Breitensuche hat eine Laufzeit von $\mathcal{O}(V + E)$ in einem Graphen mit $E$ Kanten und $V$ Knoten. Speziell hat der Graph bei dieser Aufgabe $n\cdot m$ Knoten und $(n-1)\cdot (m-1)$ Kanten.\\
Eine Breitensuche wird nach obigem Algorithmus bei jeder der insgesamt $b$ Beobachtungen $f(b_i)$-mal benötigt, wobei $f(b_i)$ die Anzahl der zu betrachtenden brennenden Felder bei Beobachtung $b_i$ sei.\\
Eine Breitensuche besucht nach obigem Algorithmus höchstens $n\cdot m - f(b_i)$ Felder; die Breitensuchen haben also eine Laufzeit von $\mathcal{O}(f(b_i)\cdot (2\cdot n\cdot m - f(b_i)))$. Es ist leicht zu erkennen, dass die Funktion $F(x) = x(a-x)$ das Maximum an der Stelle $x_{max} = \frac{a}{2}$ hat. Somit gilt $\mathcal{O}(f(b_i)\cdot (2\cdot n\cdot m - f(b_i))) = \mathcal{O}(\frac{nm}{2}(2nm - \frac{nm}{2}) = \mathcal{O}(\frac{3n²m²}{4}) = \mathcal{O}(n²m²)$
 Es ergibt sich eine Gesamtlaufzeit von $\mathcal{O}(n²\cdot m² \cdot b)$. Mit $b = \mathcal{O}(n\cdot m)$ ergibt sich eine (wohl sehr grobe) obere Schranke der Laufzeit von $\mathcal{O}(n^3 \cdot m^3)$.\\
Mit diesem Algorithmus lassen sich also Lösungen für Wälder gut berechnen, deren Dimensionen 200 nicht überschreiten, bei denen also $\max{n,m} \leq 200$.

\subsection{Umsetzung}
Für die Umsetzung habe ich die Sprache \texttt{C++} verwendet.\\
Zunächst habe ich für ein Feld \texttt{FIELDSTATE} als \texttt{char} definiert.\footnote{Das Wort \enquote{definiert} ist durchaus ernst zu nehmen, da es hier beschreiben soll, dass etwas mittels \texttt{\# define} \enquote{gelöst} wurde.} Dabei kann ein \texttt{FIELDSTATE} einen oder mehrere, ebenfalls definierter, Zustände annehmen. Dabei handelt es sich um die in der Lösungsidee beschriebenen Zustände eines Feldes, \texttt{EMPTY}, \texttt{BURNABLE}, \texttt{BURNED}, \texttt{WATERED} und \texttt{COAL}.\\
Für Wälder habe ich eine Klasse \texttt{Woods} geschrieben. Jeder Wald hat dabei eine Breite (\texttt{Width}) und eine Höhe (\texttt{Height}) und hält sich ein 2-dimensional Variables Feld an \texttt{FIELDSTATE}s \texttt{Fields}.\\
Durch geschickte Operatorenüberladung und geeigntete Akzessormethoden können diese Attribute vollständig gekapselt werden.

Das Lesen der Eingabe übernimmt die Prozedur \texttt{parseInput}, welche die Daten in eine globale Instanz der Klasse \texttt{Woods} \texttt{Forest} einliest.\\
Ist die Eingabe gelesen, werden aus dieser die zu Beginn brennenden Felder mithilfe der Funktion \texttt{getInitialBurningFields} ermittelt und dann gleich an die Prozedur \texttt{simulateFire} weitergereicht. Diese Prozedur \texttt{simulateFire} simuliert nun das Feuer und ermittelt die zu löschenden Felder unter Zuhilfenahme der Funktion \texttt{getOptimalWaterSpot}.\\
Dabei wird nach jedem Löschvorgang eine Ausgabe getätigt, welche die zu löschende Position (oben links mit (0|0) beginnend) ausgibt. Auch wird unter Verwendung von ASCII-Escape-Sequenzen ein Bild in der Konsole angezeigt, welches den Wald darstellt.\\
Ist das Feuer gelöscht (kann es sich also nicht weiter ausbreiten), wird dem NUtzer eine Meldung ausgegeben, wie viele Felder verbrannten und wie viele Felder verbrannt und gelöscht wurden. (Diese beiden Zahlen beschreiben disjunkte Mengen.) Auch hier wird wieder ein Bild erzeugt und ausgegeben.
\newpage
\subsection{Beispiele}
\subsubsection{Beispiel 0}
Die ist das Beispiel aus der Aufgabenstellung. Umgewandelt für mein Programm sieht diese Eingabe folgendermaßen aus\footnote{Diese Eingabe finden Sie auch in der Datei \texttt{0.in}}:
{\small
\lstinputlisting{../Aufgabe_1/0.in}
}
Mein Programm produziert folgende Ausgabe\footnote{Diese Ausgabe finden Sie auch in der Datei \texttt{0.out.tex}; Eine Datei \texttt{0.out} mit den ASCII-Escape-Sequenzen exisitert ebenfalls.}\footnote{Um die ASCII-Escape-Sequenzen in \TeX\, korrekt darzustellen, habe ich spezielle Ausgabemethoden geschrieben. Diese produzieren anstatt der ASCII-Sequenzen \TeX -Befehle, welche optisch zu ähnlichen Ergebnissen führen.}:\\
{\ttfamily \small
\\
\begin{tikzpicture}
\tikzset{square matrix/.style={
matrix of nodes,
column sep=-\pgflinewidth, row sep=-\pgflinewidth,
nodes={draw,
minimum height=#1,
anchor=center,
text width=#1,
align=center,
inner sep=0pt
},
},
square matrix/.default=1.2cm
}
\matrix[square matrix=1.4em] {
|[fill=green]|\color[gray]{0.75} FO%
 &|[fill=green]|\color[gray]{0.75} FO%
 &|[fill=white]|\color[gray]{0.5}WA%
 &|[fill=green]|\color[gray]{0.75} FO%
 &|[fill=green]|\color[gray]{0.75} FO%
 &|[fill=green]|\color[gray]{0.75} FO%
 &|[fill=green]|\color[gray]{0.75} FO%
 &|[fill=green]|\color[gray]{0.75} FO%
 &|[fill=white]|\color[gray]{0.5}WA%
 &|[fill=green]|\color[gray]{0.75} FO%
\\
|[fill=green]|\color[gray]{0.75} FO%
 &|[fill=white]|\color[gray]{0.5}WA%
 &|[fill=white]|\color[gray]{0.5}WA%
 &|[fill=green]|\color[gray]{0.75} FO%
 &|[fill=green]|\color[gray]{0.75} FO%
 &|[fill=green]|\color[gray]{0.75} FO%
 &|[fill=green]|\color[gray]{0.75} FO%
 &|[fill=green]|\color[gray]{0.75} FO%
 &|[fill=green]|\color[gray]{0.75} FO%
 &|[fill=white]|\color[gray]{0.5}WA%
\\
|[fill=green]|\color[gray]{0.75} FO%
 &|[fill=green]|\color[gray]{0.75} FO%
 &|[fill=green]|\color[gray]{0.75} FO%
 &|[fill=green]|\color[gray]{0.75} FO%
 &|[fill=green]|\color[gray]{0.75} FO%
 &|[fill=green]|\color[gray]{0.75} FO%
 &|[fill=green]|\color[gray]{0.75} FO%
 &|[fill=green]|\color[gray]{0.75} FO%
 &|[fill=green]|\color[gray]{0.75} FO%
 &|[fill=green]|\color[gray]{0.75} FO%
\\
|[fill=green]|\color[gray]{0.75} FO%
 &|[fill=green]|\color[gray]{0.75} FO%
 &|[fill=white]|\color[gray]{0.5}WA%
 &|[fill=white]|\color[gray]{0.5}WA%
 &|[fill=white]|\color[gray]{0.5}WA%
 &|[fill=green]|\color[gray]{0.75} FO%
 &|[fill=white]|\color[gray]{0.5}WA%
 &|[fill=white]|\color[gray]{0.5}WA%
 &|[fill=white]|\color[gray]{0.5}WA%
 &|[fill=green]|\color[gray]{0.75} FO%
\\
|[fill=green]|\color[gray]{0.75} FO%
 &|[fill=green]|\color[gray]{0.75} FO%
 &|[fill=green]|\color[gray]{0.75} FO%
 &|[fill=green]|\color[gray]{0.75} FO%
 &|[fill=green]|\color[gray]{0.75} FO%
 &|[fill=green]|\color[rgb]{1,0,0}\textbf{BU}%
 &|[fill=green]|\color[gray]{0.75} FO%
 &|[fill=green]|\color[gray]{0.75} FO%
 &|[fill=green]|\color[gray]{0.75} FO%
 &|[fill=green]|\color[gray]{0.75} FO%
\\
|[fill=green]|\color[gray]{0.75} FO%
 &|[fill=green]|\color[gray]{0.75} FO%
 &|[fill=white]|\color[gray]{0.5}WA%
 &|[fill=white]|\color[gray]{0.5}WA%
 &|[fill=green]|\color[gray]{0.75} FO%
 &|[fill=green]|\color[gray]{0.75} FO%
 &|[fill=green]|\color[gray]{0.75} FO%
 &|[fill=green]|\color[gray]{0.75} FO%
 &|[fill=green]|\color[gray]{0.75} FO%
 &|[fill=green]|\color[gray]{0.75} FO%
\\
|[fill=green]|\color[gray]{0.75} FO%
 &|[fill=green]|\color[gray]{0.75} FO%
 &|[fill=green]|\color[gray]{0.75} FO%
 &|[fill=green]|\color[gray]{0.75} FO%
 &|[fill=white]|\color[gray]{0.5}WA%
 &|[fill=green]|\color[gray]{0.75} FO%
 &|[fill=green]|\color[gray]{0.75} FO%
 &|[fill=white]|\color[gray]{0.5}WA%
 &|[fill=green]|\color[gray]{0.75} FO%
 &|[fill=green]|\color[gray]{0.75} FO%
\\
|[fill=white]|\color[gray]{0.5}WA%
 &|[fill=green]|\color[gray]{0.75} FO%
 &|[fill=green]|\color[gray]{0.75} FO%
 &|[fill=green]|\color[gray]{0.75} FO%
 &|[fill=white]|\color[gray]{0.5}WA%
 &|[fill=green]|\color[gray]{0.75} FO%
 &|[fill=green]|\color[gray]{0.75} FO%
 &|[fill=white]|\color[gray]{0.5}WA%
 &|[fill=green]|\color[gray]{0.75} FO%
 &|[fill=white]|\color[gray]{0.5}WA%
\\
|[fill=green]|\color[gray]{0.75} FO%
 &|[fill=white]|\color[gray]{0.5}WA%
 &|[fill=green]|\color[gray]{0.75} FO%
 &|[fill=green]|\color[gray]{0.75} FO%
 &|[fill=white]|\color[gray]{0.5}WA%
 &|[fill=green]|\color[gray]{0.75} FO%
 &|[fill=green]|\color[gray]{0.75} FO%
 &|[fill=white]|\color[gray]{0.5}WA%
 &|[fill=green]|\color[gray]{0.75} FO%
 &|[fill=green]|\color[gray]{0.75} FO%
\\
|[fill=green]|\color[gray]{0.75} FO%
 &|[fill=green]|\color[gray]{0.75} FO%
 &|[fill=green]|\color[gray]{0.75} FO%
 &|[fill=green]|\color[gray]{0.75} FO%
 &|[fill=green]|\color[gray]{0.75} FO%
 &|[fill=green]|\color[gray]{0.75} FO%
 &|[fill=green]|\color[gray]{0.75} FO%
 &|[fill=green]|\color[gray]{0.75} FO%
 &|[fill=green]|\color[gray]{0.75} FO%
 &|[fill=green]|\color[gray]{0.75} FO%
\\
};
\end{tikzpicture}\\
---
At time 1: Water spot (5|3)
\\
\begin{tikzpicture}
\tikzset{square matrix/.style={
matrix of nodes,
column sep=-\pgflinewidth, row sep=-\pgflinewidth,
nodes={draw,
minimum height=#1,
anchor=center,
text width=#1,
align=center,
inner sep=0pt
},
},
square matrix/.default=1.2cm
}
\matrix[square matrix=1.4em] {
|[fill=green]|\color[gray]{0.75} FO%
 &|[fill=green]|\color[gray]{0.75} FO%
 &|[fill=white]|\color[gray]{0.5}WA%
 &|[fill=green]|\color[gray]{0.75} FO%
 &|[fill=green]|\color[gray]{0.75} FO%
 &|[fill=green]|\color[gray]{0.75} FO%
 &|[fill=green]|\color[gray]{0.75} FO%
 &|[fill=green]|\color[gray]{0.75} FO%
 &|[fill=white]|\color[gray]{0.5}WA%
 &|[fill=green]|\color[gray]{0.75} FO%
\\
|[fill=green]|\color[gray]{0.75} FO%
 &|[fill=white]|\color[gray]{0.5}WA%
 &|[fill=white]|\color[gray]{0.5}WA%
 &|[fill=green]|\color[gray]{0.75} FO%
 &|[fill=green]|\color[gray]{0.75} FO%
 &|[fill=green]|\color[gray]{0.75} FO%
 &|[fill=green]|\color[gray]{0.75} FO%
 &|[fill=green]|\color[gray]{0.75} FO%
 &|[fill=green]|\color[gray]{0.75} FO%
 &|[fill=white]|\color[gray]{0.5}WA%
\\
|[fill=green]|\color[gray]{0.75} FO%
 &|[fill=green]|\color[gray]{0.75} FO%
 &|[fill=green]|\color[gray]{0.75} FO%
 &|[fill=green]|\color[gray]{0.75} FO%
 &|[fill=green]|\color[gray]{0.75} FO%
 &|[fill=green]|\color[gray]{0.75} FO%
 &|[fill=green]|\color[gray]{0.75} FO%
 &|[fill=green]|\color[gray]{0.75} FO%
 &|[fill=green]|\color[gray]{0.75} FO%
 &|[fill=green]|\color[gray]{0.75} FO%
\\
|[fill=green]|\color[gray]{0.75} FO%
 &|[fill=green]|\color[gray]{0.75} FO%
 &|[fill=white]|\color[gray]{0.5}WA%
 &|[fill=white]|\color[gray]{0.5}WA%
 &|[fill=white]|\color[gray]{0.5}WA%
 &|[fill=cyan]|\color[rgb]{1,0,0}\textbf{01}%
 &|[fill=white]|\color[gray]{0.5}WA%
 &|[fill=white]|\color[gray]{0.5}WA%
 &|[fill=white]|\color[gray]{0.5}WA%
 &|[fill=green]|\color[gray]{0.75} FO%
\\
|[fill=green]|\color[gray]{0.75} FO%
 &|[fill=green]|\color[gray]{0.75} FO%
 &|[fill=green]|\color[gray]{0.75} FO%
 &|[fill=green]|\color[gray]{0.75} FO%
 &|[fill=green]|\color[rgb]{1,0,0}\textbf{BU}%
 &|[fill=green]|\color[rgb]{0,0,0}\textbf{CO}%
 &|[fill=green]|\color[rgb]{1,0,0}\textbf{BU}%
 &|[fill=green]|\color[gray]{0.75} FO%
 &|[fill=green]|\color[gray]{0.75} FO%
 &|[fill=green]|\color[gray]{0.75} FO%
\\
|[fill=green]|\color[gray]{0.75} FO%
 &|[fill=green]|\color[gray]{0.75} FO%
 &|[fill=white]|\color[gray]{0.5}WA%
 &|[fill=white]|\color[gray]{0.5}WA%
 &|[fill=green]|\color[gray]{0.75} FO%
 &|[fill=green]|\color[rgb]{1,0,0}\textbf{BU}%
 &|[fill=green]|\color[gray]{0.75} FO%
 &|[fill=green]|\color[gray]{0.75} FO%
 &|[fill=green]|\color[gray]{0.75} FO%
 &|[fill=green]|\color[gray]{0.75} FO%
\\
|[fill=green]|\color[gray]{0.75} FO%
 &|[fill=green]|\color[gray]{0.75} FO%
 &|[fill=green]|\color[gray]{0.75} FO%
 &|[fill=green]|\color[gray]{0.75} FO%
 &|[fill=white]|\color[gray]{0.5}WA%
 &|[fill=green]|\color[gray]{0.75} FO%
 &|[fill=green]|\color[gray]{0.75} FO%
 &|[fill=white]|\color[gray]{0.5}WA%
 &|[fill=green]|\color[gray]{0.75} FO%
 &|[fill=green]|\color[gray]{0.75} FO%
\\
|[fill=white]|\color[gray]{0.5}WA%
 &|[fill=green]|\color[gray]{0.75} FO%
 &|[fill=green]|\color[gray]{0.75} FO%
 &|[fill=green]|\color[gray]{0.75} FO%
 &|[fill=white]|\color[gray]{0.5}WA%
 &|[fill=green]|\color[gray]{0.75} FO%
 &|[fill=green]|\color[gray]{0.75} FO%
 &|[fill=white]|\color[gray]{0.5}WA%
 &|[fill=green]|\color[gray]{0.75} FO%
 &|[fill=white]|\color[gray]{0.5}WA%
\\
|[fill=green]|\color[gray]{0.75} FO%
 &|[fill=white]|\color[gray]{0.5}WA%
 &|[fill=green]|\color[gray]{0.75} FO%
 &|[fill=green]|\color[gray]{0.75} FO%
 &|[fill=white]|\color[gray]{0.5}WA%
 &|[fill=green]|\color[gray]{0.75} FO%
 &|[fill=green]|\color[gray]{0.75} FO%
 &|[fill=white]|\color[gray]{0.5}WA%
 &|[fill=green]|\color[gray]{0.75} FO%
 &|[fill=green]|\color[gray]{0.75} FO%
\\
|[fill=green]|\color[gray]{0.75} FO%
 &|[fill=green]|\color[gray]{0.75} FO%
 &|[fill=green]|\color[gray]{0.75} FO%
 &|[fill=green]|\color[gray]{0.75} FO%
 &|[fill=green]|\color[gray]{0.75} FO%
 &|[fill=green]|\color[gray]{0.75} FO%
 &|[fill=green]|\color[gray]{0.75} FO%
 &|[fill=green]|\color[gray]{0.75} FO%
 &|[fill=green]|\color[gray]{0.75} FO%
 &|[fill=green]|\color[gray]{0.75} FO%
\\
};
\end{tikzpicture}\\
---
At time 2: Water spot (3|4)
\\
\begin{tikzpicture}
\tikzset{square matrix/.style={
matrix of nodes,
column sep=-\pgflinewidth, row sep=-\pgflinewidth,
nodes={draw,
minimum height=#1,
anchor=center,
text width=#1,
align=center,
inner sep=0pt
},
},
square matrix/.default=1.2cm
}
\matrix[square matrix=1.4em] {
|[fill=green]|\color[gray]{0.75} FO%
 &|[fill=green]|\color[gray]{0.75} FO%
 &|[fill=white]|\color[gray]{0.5}WA%
 &|[fill=green]|\color[gray]{0.75} FO%
 &|[fill=green]|\color[gray]{0.75} FO%
 &|[fill=green]|\color[gray]{0.75} FO%
 &|[fill=green]|\color[gray]{0.75} FO%
 &|[fill=green]|\color[gray]{0.75} FO%
 &|[fill=white]|\color[gray]{0.5}WA%
 &|[fill=green]|\color[gray]{0.75} FO%
\\
|[fill=green]|\color[gray]{0.75} FO%
 &|[fill=white]|\color[gray]{0.5}WA%
 &|[fill=white]|\color[gray]{0.5}WA%
 &|[fill=green]|\color[gray]{0.75} FO%
 &|[fill=green]|\color[gray]{0.75} FO%
 &|[fill=green]|\color[gray]{0.75} FO%
 &|[fill=green]|\color[gray]{0.75} FO%
 &|[fill=green]|\color[gray]{0.75} FO%
 &|[fill=green]|\color[gray]{0.75} FO%
 &|[fill=white]|\color[gray]{0.5}WA%
\\
|[fill=green]|\color[gray]{0.75} FO%
 &|[fill=green]|\color[gray]{0.75} FO%
 &|[fill=green]|\color[gray]{0.75} FO%
 &|[fill=green]|\color[gray]{0.75} FO%
 &|[fill=green]|\color[gray]{0.75} FO%
 &|[fill=green]|\color[gray]{0.75} FO%
 &|[fill=green]|\color[gray]{0.75} FO%
 &|[fill=green]|\color[gray]{0.75} FO%
 &|[fill=green]|\color[gray]{0.75} FO%
 &|[fill=green]|\color[gray]{0.75} FO%
\\
|[fill=green]|\color[gray]{0.75} FO%
 &|[fill=green]|\color[gray]{0.75} FO%
 &|[fill=white]|\color[gray]{0.5}WA%
 &|[fill=white]|\color[gray]{0.5}WA%
 &|[fill=white]|\color[gray]{0.5}WA%
 &|[fill=cyan]|\color[rgb]{1,0,0}\textbf{01}%
 &|[fill=white]|\color[gray]{0.5}WA%
 &|[fill=white]|\color[gray]{0.5}WA%
 &|[fill=white]|\color[gray]{0.5}WA%
 &|[fill=green]|\color[gray]{0.75} FO%
\\
|[fill=green]|\color[gray]{0.75} FO%
 &|[fill=green]|\color[gray]{0.75} FO%
 &|[fill=green]|\color[gray]{0.75} FO%
 &|[fill=cyan]|\color[rgb]{1,0,0}\textbf{02}%
 &|[fill=green]|\color[rgb]{0,0,0}\textbf{CO}%
 &|[fill=green]|\color[rgb]{0,0,0}\textbf{CO}%
 &|[fill=green]|\color[rgb]{0,0,0}\textbf{CO}%
 &|[fill=green]|\color[rgb]{1,0,0}\textbf{BU}%
 &|[fill=green]|\color[gray]{0.75} FO%
 &|[fill=green]|\color[gray]{0.75} FO%
\\
|[fill=green]|\color[gray]{0.75} FO%
 &|[fill=green]|\color[gray]{0.75} FO%
 &|[fill=white]|\color[gray]{0.5}WA%
 &|[fill=white]|\color[gray]{0.5}WA%
 &|[fill=green]|\color[rgb]{1,0,0}\textbf{BU}%
 &|[fill=green]|\color[rgb]{0,0,0}\textbf{CO}%
 &|[fill=green]|\color[rgb]{1,0,0}\textbf{BU}%
 &|[fill=green]|\color[gray]{0.75} FO%
 &|[fill=green]|\color[gray]{0.75} FO%
 &|[fill=green]|\color[gray]{0.75} FO%
\\
|[fill=green]|\color[gray]{0.75} FO%
 &|[fill=green]|\color[gray]{0.75} FO%
 &|[fill=green]|\color[gray]{0.75} FO%
 &|[fill=green]|\color[gray]{0.75} FO%
 &|[fill=white]|\color[gray]{0.5}WA%
 &|[fill=green]|\color[rgb]{1,0,0}\textbf{BU}%
 &|[fill=green]|\color[gray]{0.75} FO%
 &|[fill=white]|\color[gray]{0.5}WA%
 &|[fill=green]|\color[gray]{0.75} FO%
 &|[fill=green]|\color[gray]{0.75} FO%
\\
|[fill=white]|\color[gray]{0.5}WA%
 &|[fill=green]|\color[gray]{0.75} FO%
 &|[fill=green]|\color[gray]{0.75} FO%
 &|[fill=green]|\color[gray]{0.75} FO%
 &|[fill=white]|\color[gray]{0.5}WA%
 &|[fill=green]|\color[gray]{0.75} FO%
 &|[fill=green]|\color[gray]{0.75} FO%
 &|[fill=white]|\color[gray]{0.5}WA%
 &|[fill=green]|\color[gray]{0.75} FO%
 &|[fill=white]|\color[gray]{0.5}WA%
\\
|[fill=green]|\color[gray]{0.75} FO%
 &|[fill=white]|\color[gray]{0.5}WA%
 &|[fill=green]|\color[gray]{0.75} FO%
 &|[fill=green]|\color[gray]{0.75} FO%
 &|[fill=white]|\color[gray]{0.5}WA%
 &|[fill=green]|\color[gray]{0.75} FO%
 &|[fill=green]|\color[gray]{0.75} FO%
 &|[fill=white]|\color[gray]{0.5}WA%
 &|[fill=green]|\color[gray]{0.75} FO%
 &|[fill=green]|\color[gray]{0.75} FO%
\\
|[fill=green]|\color[gray]{0.75} FO%
 &|[fill=green]|\color[gray]{0.75} FO%
 &|[fill=green]|\color[gray]{0.75} FO%
 &|[fill=green]|\color[gray]{0.75} FO%
 &|[fill=green]|\color[gray]{0.75} FO%
 &|[fill=green]|\color[gray]{0.75} FO%
 &|[fill=green]|\color[gray]{0.75} FO%
 &|[fill=green]|\color[gray]{0.75} FO%
 &|[fill=green]|\color[gray]{0.75} FO%
 &|[fill=green]|\color[gray]{0.75} FO%
\\
};
\end{tikzpicture}\\
---
At time 3: Water spot (8|4)
\\
\begin{tikzpicture}
\tikzset{square matrix/.style={
matrix of nodes,
column sep=-\pgflinewidth, row sep=-\pgflinewidth,
nodes={draw,
minimum height=#1,
anchor=center,
text width=#1,
align=center,
inner sep=0pt
},
},
square matrix/.default=1.2cm
}
\matrix[square matrix=1.4em] {
|[fill=green]|\color[gray]{0.75} FO%
 &|[fill=green]|\color[gray]{0.75} FO%
 &|[fill=white]|\color[gray]{0.5}WA%
 &|[fill=green]|\color[gray]{0.75} FO%
 &|[fill=green]|\color[gray]{0.75} FO%
 &|[fill=green]|\color[gray]{0.75} FO%
 &|[fill=green]|\color[gray]{0.75} FO%
 &|[fill=green]|\color[gray]{0.75} FO%
 &|[fill=white]|\color[gray]{0.5}WA%
 &|[fill=green]|\color[gray]{0.75} FO%
\\
|[fill=green]|\color[gray]{0.75} FO%
 &|[fill=white]|\color[gray]{0.5}WA%
 &|[fill=white]|\color[gray]{0.5}WA%
 &|[fill=green]|\color[gray]{0.75} FO%
 &|[fill=green]|\color[gray]{0.75} FO%
 &|[fill=green]|\color[gray]{0.75} FO%
 &|[fill=green]|\color[gray]{0.75} FO%
 &|[fill=green]|\color[gray]{0.75} FO%
 &|[fill=green]|\color[gray]{0.75} FO%
 &|[fill=white]|\color[gray]{0.5}WA%
\\
|[fill=green]|\color[gray]{0.75} FO%
 &|[fill=green]|\color[gray]{0.75} FO%
 &|[fill=green]|\color[gray]{0.75} FO%
 &|[fill=green]|\color[gray]{0.75} FO%
 &|[fill=green]|\color[gray]{0.75} FO%
 &|[fill=green]|\color[gray]{0.75} FO%
 &|[fill=green]|\color[gray]{0.75} FO%
 &|[fill=green]|\color[gray]{0.75} FO%
 &|[fill=green]|\color[gray]{0.75} FO%
 &|[fill=green]|\color[gray]{0.75} FO%
\\
|[fill=green]|\color[gray]{0.75} FO%
 &|[fill=green]|\color[gray]{0.75} FO%
 &|[fill=white]|\color[gray]{0.5}WA%
 &|[fill=white]|\color[gray]{0.5}WA%
 &|[fill=white]|\color[gray]{0.5}WA%
 &|[fill=cyan]|\color[rgb]{1,0,0}\textbf{01}%
 &|[fill=white]|\color[gray]{0.5}WA%
 &|[fill=white]|\color[gray]{0.5}WA%
 &|[fill=white]|\color[gray]{0.5}WA%
 &|[fill=green]|\color[gray]{0.75} FO%
\\
|[fill=green]|\color[gray]{0.75} FO%
 &|[fill=green]|\color[gray]{0.75} FO%
 &|[fill=green]|\color[gray]{0.75} FO%
 &|[fill=cyan]|\color[rgb]{1,0,0}\textbf{02}%
 &|[fill=green]|\color[rgb]{0,0,0}\textbf{CO}%
 &|[fill=green]|\color[rgb]{0,0,0}\textbf{CO}%
 &|[fill=green]|\color[rgb]{0,0,0}\textbf{CO}%
 &|[fill=green]|\color[rgb]{0,0,0}\textbf{CO}%
 &|[fill=cyan]|\color[rgb]{1,0,0}\textbf{03}%
 &|[fill=green]|\color[gray]{0.75} FO%
\\
|[fill=green]|\color[gray]{0.75} FO%
 &|[fill=green]|\color[gray]{0.75} FO%
 &|[fill=white]|\color[gray]{0.5}WA%
 &|[fill=white]|\color[gray]{0.5}WA%
 &|[fill=green]|\color[rgb]{0,0,0}\textbf{CO}%
 &|[fill=green]|\color[rgb]{0,0,0}\textbf{CO}%
 &|[fill=green]|\color[rgb]{0,0,0}\textbf{CO}%
 &|[fill=green]|\color[rgb]{1,0,0}\textbf{BU}%
 &|[fill=green]|\color[gray]{0.75} FO%
 &|[fill=green]|\color[gray]{0.75} FO%
\\
|[fill=green]|\color[gray]{0.75} FO%
 &|[fill=green]|\color[gray]{0.75} FO%
 &|[fill=green]|\color[gray]{0.75} FO%
 &|[fill=green]|\color[gray]{0.75} FO%
 &|[fill=white]|\color[gray]{0.5}WA%
 &|[fill=green]|\color[rgb]{0,0,0}\textbf{CO}%
 &|[fill=green]|\color[rgb]{1,0,0}\textbf{BU}%
 &|[fill=white]|\color[gray]{0.5}WA%
 &|[fill=green]|\color[gray]{0.75} FO%
 &|[fill=green]|\color[gray]{0.75} FO%
\\
|[fill=white]|\color[gray]{0.5}WA%
 &|[fill=green]|\color[gray]{0.75} FO%
 &|[fill=green]|\color[gray]{0.75} FO%
 &|[fill=green]|\color[gray]{0.75} FO%
 &|[fill=white]|\color[gray]{0.5}WA%
 &|[fill=green]|\color[rgb]{1,0,0}\textbf{BU}%
 &|[fill=green]|\color[gray]{0.75} FO%
 &|[fill=white]|\color[gray]{0.5}WA%
 &|[fill=green]|\color[gray]{0.75} FO%
 &|[fill=white]|\color[gray]{0.5}WA%
\\
|[fill=green]|\color[gray]{0.75} FO%
 &|[fill=white]|\color[gray]{0.5}WA%
 &|[fill=green]|\color[gray]{0.75} FO%
 &|[fill=green]|\color[gray]{0.75} FO%
 &|[fill=white]|\color[gray]{0.5}WA%
 &|[fill=green]|\color[gray]{0.75} FO%
 &|[fill=green]|\color[gray]{0.75} FO%
 &|[fill=white]|\color[gray]{0.5}WA%
 &|[fill=green]|\color[gray]{0.75} FO%
 &|[fill=green]|\color[gray]{0.75} FO%
\\
|[fill=green]|\color[gray]{0.75} FO%
 &|[fill=green]|\color[gray]{0.75} FO%
 &|[fill=green]|\color[gray]{0.75} FO%
 &|[fill=green]|\color[gray]{0.75} FO%
 &|[fill=green]|\color[gray]{0.75} FO%
 &|[fill=green]|\color[gray]{0.75} FO%
 &|[fill=green]|\color[gray]{0.75} FO%
 &|[fill=green]|\color[gray]{0.75} FO%
 &|[fill=green]|\color[gray]{0.75} FO%
 &|[fill=green]|\color[gray]{0.75} FO%
\\
};
\end{tikzpicture}\\
---
At time 4: Water spot (8|5)
\\
\begin{tikzpicture}
\tikzset{square matrix/.style={
matrix of nodes,
column sep=-\pgflinewidth, row sep=-\pgflinewidth,
nodes={draw,
minimum height=#1,
anchor=center,
text width=#1,
align=center,
inner sep=0pt
},
},
square matrix/.default=1.2cm
}
\matrix[square matrix=1.4em] {
|[fill=green]|\color[gray]{0.75} FO%
 &|[fill=green]|\color[gray]{0.75} FO%
 &|[fill=white]|\color[gray]{0.5}WA%
 &|[fill=green]|\color[gray]{0.75} FO%
 &|[fill=green]|\color[gray]{0.75} FO%
 &|[fill=green]|\color[gray]{0.75} FO%
 &|[fill=green]|\color[gray]{0.75} FO%
 &|[fill=green]|\color[gray]{0.75} FO%
 &|[fill=white]|\color[gray]{0.5}WA%
 &|[fill=green]|\color[gray]{0.75} FO%
\\
|[fill=green]|\color[gray]{0.75} FO%
 &|[fill=white]|\color[gray]{0.5}WA%
 &|[fill=white]|\color[gray]{0.5}WA%
 &|[fill=green]|\color[gray]{0.75} FO%
 &|[fill=green]|\color[gray]{0.75} FO%
 &|[fill=green]|\color[gray]{0.75} FO%
 &|[fill=green]|\color[gray]{0.75} FO%
 &|[fill=green]|\color[gray]{0.75} FO%
 &|[fill=green]|\color[gray]{0.75} FO%
 &|[fill=white]|\color[gray]{0.5}WA%
\\
|[fill=green]|\color[gray]{0.75} FO%
 &|[fill=green]|\color[gray]{0.75} FO%
 &|[fill=green]|\color[gray]{0.75} FO%
 &|[fill=green]|\color[gray]{0.75} FO%
 &|[fill=green]|\color[gray]{0.75} FO%
 &|[fill=green]|\color[gray]{0.75} FO%
 &|[fill=green]|\color[gray]{0.75} FO%
 &|[fill=green]|\color[gray]{0.75} FO%
 &|[fill=green]|\color[gray]{0.75} FO%
 &|[fill=green]|\color[gray]{0.75} FO%
\\
|[fill=green]|\color[gray]{0.75} FO%
 &|[fill=green]|\color[gray]{0.75} FO%
 &|[fill=white]|\color[gray]{0.5}WA%
 &|[fill=white]|\color[gray]{0.5}WA%
 &|[fill=white]|\color[gray]{0.5}WA%
 &|[fill=cyan]|\color[rgb]{1,0,0}\textbf{01}%
 &|[fill=white]|\color[gray]{0.5}WA%
 &|[fill=white]|\color[gray]{0.5}WA%
 &|[fill=white]|\color[gray]{0.5}WA%
 &|[fill=green]|\color[gray]{0.75} FO%
\\
|[fill=green]|\color[gray]{0.75} FO%
 &|[fill=green]|\color[gray]{0.75} FO%
 &|[fill=green]|\color[gray]{0.75} FO%
 &|[fill=cyan]|\color[rgb]{1,0,0}\textbf{02}%
 &|[fill=green]|\color[rgb]{0,0,0}\textbf{CO}%
 &|[fill=green]|\color[rgb]{0,0,0}\textbf{CO}%
 &|[fill=green]|\color[rgb]{0,0,0}\textbf{CO}%
 &|[fill=green]|\color[rgb]{0,0,0}\textbf{CO}%
 &|[fill=cyan]|\color[rgb]{1,0,0}\textbf{03}%
 &|[fill=green]|\color[gray]{0.75} FO%
\\
|[fill=green]|\color[gray]{0.75} FO%
 &|[fill=green]|\color[gray]{0.75} FO%
 &|[fill=white]|\color[gray]{0.5}WA%
 &|[fill=white]|\color[gray]{0.5}WA%
 &|[fill=green]|\color[rgb]{0,0,0}\textbf{CO}%
 &|[fill=green]|\color[rgb]{0,0,0}\textbf{CO}%
 &|[fill=green]|\color[rgb]{0,0,0}\textbf{CO}%
 &|[fill=green]|\color[rgb]{0,0,0}\textbf{CO}%
 &|[fill=cyan]|\color[rgb]{1,0,0}\textbf{04}%
 &|[fill=green]|\color[gray]{0.75} FO%
\\
|[fill=green]|\color[gray]{0.75} FO%
 &|[fill=green]|\color[gray]{0.75} FO%
 &|[fill=green]|\color[gray]{0.75} FO%
 &|[fill=green]|\color[gray]{0.75} FO%
 &|[fill=white]|\color[gray]{0.5}WA%
 &|[fill=green]|\color[rgb]{0,0,0}\textbf{CO}%
 &|[fill=green]|\color[rgb]{0,0,0}\textbf{CO}%
 &|[fill=white]|\color[gray]{0.5}WA%
 &|[fill=green]|\color[gray]{0.75} FO%
 &|[fill=green]|\color[gray]{0.75} FO%
\\
|[fill=white]|\color[gray]{0.5}WA%
 &|[fill=green]|\color[gray]{0.75} FO%
 &|[fill=green]|\color[gray]{0.75} FO%
 &|[fill=green]|\color[gray]{0.75} FO%
 &|[fill=white]|\color[gray]{0.5}WA%
 &|[fill=green]|\color[rgb]{0,0,0}\textbf{CO}%
 &|[fill=green]|\color[rgb]{1,0,0}\textbf{BU}%
 &|[fill=white]|\color[gray]{0.5}WA%
 &|[fill=green]|\color[gray]{0.75} FO%
 &|[fill=white]|\color[gray]{0.5}WA%
\\
|[fill=green]|\color[gray]{0.75} FO%
 &|[fill=white]|\color[gray]{0.5}WA%
 &|[fill=green]|\color[gray]{0.75} FO%
 &|[fill=green]|\color[gray]{0.75} FO%
 &|[fill=white]|\color[gray]{0.5}WA%
 &|[fill=green]|\color[rgb]{1,0,0}\textbf{BU}%
 &|[fill=green]|\color[gray]{0.75} FO%
 &|[fill=white]|\color[gray]{0.5}WA%
 &|[fill=green]|\color[gray]{0.75} FO%
 &|[fill=green]|\color[gray]{0.75} FO%
\\
|[fill=green]|\color[gray]{0.75} FO%
 &|[fill=green]|\color[gray]{0.75} FO%
 &|[fill=green]|\color[gray]{0.75} FO%
 &|[fill=green]|\color[gray]{0.75} FO%
 &|[fill=green]|\color[gray]{0.75} FO%
 &|[fill=green]|\color[gray]{0.75} FO%
 &|[fill=green]|\color[gray]{0.75} FO%
 &|[fill=green]|\color[gray]{0.75} FO%
 &|[fill=green]|\color[gray]{0.75} FO%
 &|[fill=green]|\color[gray]{0.75} FO%
\\
};
\end{tikzpicture}\\
---
At time 5: Water spot (5|9)
\\
\begin{tikzpicture}
\tikzset{square matrix/.style={
matrix of nodes,
column sep=-\pgflinewidth, row sep=-\pgflinewidth,
nodes={draw,
minimum height=#1,
anchor=center,
text width=#1,
align=center,
inner sep=0pt
},
},
square matrix/.default=1.2cm
}
\matrix[square matrix=1.4em] {
|[fill=green]|\color[gray]{0.75} FO%
 &|[fill=green]|\color[gray]{0.75} FO%
 &|[fill=white]|\color[gray]{0.5}WA%
 &|[fill=green]|\color[gray]{0.75} FO%
 &|[fill=green]|\color[gray]{0.75} FO%
 &|[fill=green]|\color[gray]{0.75} FO%
 &|[fill=green]|\color[gray]{0.75} FO%
 &|[fill=green]|\color[gray]{0.75} FO%
 &|[fill=white]|\color[gray]{0.5}WA%
 &|[fill=green]|\color[gray]{0.75} FO%
\\
|[fill=green]|\color[gray]{0.75} FO%
 &|[fill=white]|\color[gray]{0.5}WA%
 &|[fill=white]|\color[gray]{0.5}WA%
 &|[fill=green]|\color[gray]{0.75} FO%
 &|[fill=green]|\color[gray]{0.75} FO%
 &|[fill=green]|\color[gray]{0.75} FO%
 &|[fill=green]|\color[gray]{0.75} FO%
 &|[fill=green]|\color[gray]{0.75} FO%
 &|[fill=green]|\color[gray]{0.75} FO%
 &|[fill=white]|\color[gray]{0.5}WA%
\\
|[fill=green]|\color[gray]{0.75} FO%
 &|[fill=green]|\color[gray]{0.75} FO%
 &|[fill=green]|\color[gray]{0.75} FO%
 &|[fill=green]|\color[gray]{0.75} FO%
 &|[fill=green]|\color[gray]{0.75} FO%
 &|[fill=green]|\color[gray]{0.75} FO%
 &|[fill=green]|\color[gray]{0.75} FO%
 &|[fill=green]|\color[gray]{0.75} FO%
 &|[fill=green]|\color[gray]{0.75} FO%
 &|[fill=green]|\color[gray]{0.75} FO%
\\
|[fill=green]|\color[gray]{0.75} FO%
 &|[fill=green]|\color[gray]{0.75} FO%
 &|[fill=white]|\color[gray]{0.5}WA%
 &|[fill=white]|\color[gray]{0.5}WA%
 &|[fill=white]|\color[gray]{0.5}WA%
 &|[fill=cyan]|\color[rgb]{1,0,0}\textbf{01}%
 &|[fill=white]|\color[gray]{0.5}WA%
 &|[fill=white]|\color[gray]{0.5}WA%
 &|[fill=white]|\color[gray]{0.5}WA%
 &|[fill=green]|\color[gray]{0.75} FO%
\\
|[fill=green]|\color[gray]{0.75} FO%
 &|[fill=green]|\color[gray]{0.75} FO%
 &|[fill=green]|\color[gray]{0.75} FO%
 &|[fill=cyan]|\color[rgb]{1,0,0}\textbf{02}%
 &|[fill=green]|\color[rgb]{0,0,0}\textbf{CO}%
 &|[fill=green]|\color[rgb]{0,0,0}\textbf{CO}%
 &|[fill=green]|\color[rgb]{0,0,0}\textbf{CO}%
 &|[fill=green]|\color[rgb]{0,0,0}\textbf{CO}%
 &|[fill=cyan]|\color[rgb]{1,0,0}\textbf{03}%
 &|[fill=green]|\color[gray]{0.75} FO%
\\
|[fill=green]|\color[gray]{0.75} FO%
 &|[fill=green]|\color[gray]{0.75} FO%
 &|[fill=white]|\color[gray]{0.5}WA%
 &|[fill=white]|\color[gray]{0.5}WA%
 &|[fill=green]|\color[rgb]{0,0,0}\textbf{CO}%
 &|[fill=green]|\color[rgb]{0,0,0}\textbf{CO}%
 &|[fill=green]|\color[rgb]{0,0,0}\textbf{CO}%
 &|[fill=green]|\color[rgb]{0,0,0}\textbf{CO}%
 &|[fill=cyan]|\color[rgb]{1,0,0}\textbf{04}%
 &|[fill=green]|\color[gray]{0.75} FO%
\\
|[fill=green]|\color[gray]{0.75} FO%
 &|[fill=green]|\color[gray]{0.75} FO%
 &|[fill=green]|\color[gray]{0.75} FO%
 &|[fill=green]|\color[gray]{0.75} FO%
 &|[fill=white]|\color[gray]{0.5}WA%
 &|[fill=green]|\color[rgb]{0,0,0}\textbf{CO}%
 &|[fill=green]|\color[rgb]{0,0,0}\textbf{CO}%
 &|[fill=white]|\color[gray]{0.5}WA%
 &|[fill=green]|\color[gray]{0.75} FO%
 &|[fill=green]|\color[gray]{0.75} FO%
\\
|[fill=white]|\color[gray]{0.5}WA%
 &|[fill=green]|\color[gray]{0.75} FO%
 &|[fill=green]|\color[gray]{0.75} FO%
 &|[fill=green]|\color[gray]{0.75} FO%
 &|[fill=white]|\color[gray]{0.5}WA%
 &|[fill=green]|\color[rgb]{0,0,0}\textbf{CO}%
 &|[fill=green]|\color[rgb]{0,0,0}\textbf{CO}%
 &|[fill=white]|\color[gray]{0.5}WA%
 &|[fill=green]|\color[gray]{0.75} FO%
 &|[fill=white]|\color[gray]{0.5}WA%
\\
|[fill=green]|\color[gray]{0.75} FO%
 &|[fill=white]|\color[gray]{0.5}WA%
 &|[fill=green]|\color[gray]{0.75} FO%
 &|[fill=green]|\color[gray]{0.75} FO%
 &|[fill=white]|\color[gray]{0.5}WA%
 &|[fill=green]|\color[rgb]{0,0,0}\textbf{CO}%
 &|[fill=green]|\color[rgb]{1,0,0}\textbf{BU}%
 &|[fill=white]|\color[gray]{0.5}WA%
 &|[fill=green]|\color[gray]{0.75} FO%
 &|[fill=green]|\color[gray]{0.75} FO%
\\
|[fill=green]|\color[gray]{0.75} FO%
 &|[fill=green]|\color[gray]{0.75} FO%
 &|[fill=green]|\color[gray]{0.75} FO%
 &|[fill=green]|\color[gray]{0.75} FO%
 &|[fill=green]|\color[gray]{0.75} FO%
 &|[fill=cyan]|\color[rgb]{1,0,0}\textbf{05}%
 &|[fill=green]|\color[gray]{0.75} FO%
 &|[fill=green]|\color[gray]{0.75} FO%
 &|[fill=green]|\color[gray]{0.75} FO%
 &|[fill=green]|\color[gray]{0.75} FO%
\\
};
\end{tikzpicture}\\
---
At time 6: Water spot (6|9)
\\
\begin{tikzpicture}
\tikzset{square matrix/.style={
matrix of nodes,
column sep=-\pgflinewidth, row sep=-\pgflinewidth,
nodes={draw,
minimum height=#1,
anchor=center,
text width=#1,
align=center,
inner sep=0pt
},
},
square matrix/.default=1.2cm
}
\matrix[square matrix=1.4em] {
|[fill=green]|\color[gray]{0.75} FO%
 &|[fill=green]|\color[gray]{0.75} FO%
 &|[fill=white]|\color[gray]{0.5}WA%
 &|[fill=green]|\color[gray]{0.75} FO%
 &|[fill=green]|\color[gray]{0.75} FO%
 &|[fill=green]|\color[gray]{0.75} FO%
 &|[fill=green]|\color[gray]{0.75} FO%
 &|[fill=green]|\color[gray]{0.75} FO%
 &|[fill=white]|\color[gray]{0.5}WA%
 &|[fill=green]|\color[gray]{0.75} FO%
\\
|[fill=green]|\color[gray]{0.75} FO%
 &|[fill=white]|\color[gray]{0.5}WA%
 &|[fill=white]|\color[gray]{0.5}WA%
 &|[fill=green]|\color[gray]{0.75} FO%
 &|[fill=green]|\color[gray]{0.75} FO%
 &|[fill=green]|\color[gray]{0.75} FO%
 &|[fill=green]|\color[gray]{0.75} FO%
 &|[fill=green]|\color[gray]{0.75} FO%
 &|[fill=green]|\color[gray]{0.75} FO%
 &|[fill=white]|\color[gray]{0.5}WA%
\\
|[fill=green]|\color[gray]{0.75} FO%
 &|[fill=green]|\color[gray]{0.75} FO%
 &|[fill=green]|\color[gray]{0.75} FO%
 &|[fill=green]|\color[gray]{0.75} FO%
 &|[fill=green]|\color[gray]{0.75} FO%
 &|[fill=green]|\color[gray]{0.75} FO%
 &|[fill=green]|\color[gray]{0.75} FO%
 &|[fill=green]|\color[gray]{0.75} FO%
 &|[fill=green]|\color[gray]{0.75} FO%
 &|[fill=green]|\color[gray]{0.75} FO%
\\
|[fill=green]|\color[gray]{0.75} FO%
 &|[fill=green]|\color[gray]{0.75} FO%
 &|[fill=white]|\color[gray]{0.5}WA%
 &|[fill=white]|\color[gray]{0.5}WA%
 &|[fill=white]|\color[gray]{0.5}WA%
 &|[fill=cyan]|\color[rgb]{1,0,0}\textbf{01}%
 &|[fill=white]|\color[gray]{0.5}WA%
 &|[fill=white]|\color[gray]{0.5}WA%
 &|[fill=white]|\color[gray]{0.5}WA%
 &|[fill=green]|\color[gray]{0.75} FO%
\\
|[fill=green]|\color[gray]{0.75} FO%
 &|[fill=green]|\color[gray]{0.75} FO%
 &|[fill=green]|\color[gray]{0.75} FO%
 &|[fill=cyan]|\color[rgb]{1,0,0}\textbf{02}%
 &|[fill=green]|\color[rgb]{0,0,0}\textbf{CO}%
 &|[fill=green]|\color[rgb]{0,0,0}\textbf{CO}%
 &|[fill=green]|\color[rgb]{0,0,0}\textbf{CO}%
 &|[fill=green]|\color[rgb]{0,0,0}\textbf{CO}%
 &|[fill=cyan]|\color[rgb]{1,0,0}\textbf{03}%
 &|[fill=green]|\color[gray]{0.75} FO%
\\
|[fill=green]|\color[gray]{0.75} FO%
 &|[fill=green]|\color[gray]{0.75} FO%
 &|[fill=white]|\color[gray]{0.5}WA%
 &|[fill=white]|\color[gray]{0.5}WA%
 &|[fill=green]|\color[rgb]{0,0,0}\textbf{CO}%
 &|[fill=green]|\color[rgb]{0,0,0}\textbf{CO}%
 &|[fill=green]|\color[rgb]{0,0,0}\textbf{CO}%
 &|[fill=green]|\color[rgb]{0,0,0}\textbf{CO}%
 &|[fill=cyan]|\color[rgb]{1,0,0}\textbf{04}%
 &|[fill=green]|\color[gray]{0.75} FO%
\\
|[fill=green]|\color[gray]{0.75} FO%
 &|[fill=green]|\color[gray]{0.75} FO%
 &|[fill=green]|\color[gray]{0.75} FO%
 &|[fill=green]|\color[gray]{0.75} FO%
 &|[fill=white]|\color[gray]{0.5}WA%
 &|[fill=green]|\color[rgb]{0,0,0}\textbf{CO}%
 &|[fill=green]|\color[rgb]{0,0,0}\textbf{CO}%
 &|[fill=white]|\color[gray]{0.5}WA%
 &|[fill=green]|\color[gray]{0.75} FO%
 &|[fill=green]|\color[gray]{0.75} FO%
\\
|[fill=white]|\color[gray]{0.5}WA%
 &|[fill=green]|\color[gray]{0.75} FO%
 &|[fill=green]|\color[gray]{0.75} FO%
 &|[fill=green]|\color[gray]{0.75} FO%
 &|[fill=white]|\color[gray]{0.5}WA%
 &|[fill=green]|\color[rgb]{0,0,0}\textbf{CO}%
 &|[fill=green]|\color[rgb]{0,0,0}\textbf{CO}%
 &|[fill=white]|\color[gray]{0.5}WA%
 &|[fill=green]|\color[gray]{0.75} FO%
 &|[fill=white]|\color[gray]{0.5}WA%
\\
|[fill=green]|\color[gray]{0.75} FO%
 &|[fill=white]|\color[gray]{0.5}WA%
 &|[fill=green]|\color[gray]{0.75} FO%
 &|[fill=green]|\color[gray]{0.75} FO%
 &|[fill=white]|\color[gray]{0.5}WA%
 &|[fill=green]|\color[rgb]{0,0,0}\textbf{CO}%
 &|[fill=green]|\color[rgb]{0,0,0}\textbf{CO}%
 &|[fill=white]|\color[gray]{0.5}WA%
 &|[fill=green]|\color[gray]{0.75} FO%
 &|[fill=green]|\color[gray]{0.75} FO%
\\
|[fill=green]|\color[gray]{0.75} FO%
 &|[fill=green]|\color[gray]{0.75} FO%
 &|[fill=green]|\color[gray]{0.75} FO%
 &|[fill=green]|\color[gray]{0.75} FO%
 &|[fill=green]|\color[gray]{0.75} FO%
 &|[fill=cyan]|\color[rgb]{1,0,0}\textbf{05}%
 &|[fill=cyan]|\color[rgb]{1,0,0}\textbf{06}%
 &|[fill=green]|\color[gray]{0.75} FO%
 &|[fill=green]|\color[gray]{0.75} FO%
 &|[fill=green]|\color[gray]{0.75} FO%
\\
};
\end{tikzpicture}\\
---
And you'll find 14 pieces of coal and 6 pieces of watered coal
\\
\begin{tikzpicture}
\tikzset{square matrix/.style={
matrix of nodes,
column sep=-\pgflinewidth, row sep=-\pgflinewidth,
nodes={draw,
minimum height=#1,
anchor=center,
text width=#1,
align=center,
inner sep=0pt
},
},
square matrix/.default=1.2cm
}
\matrix[square matrix=1.4em] {
|[fill=green]|\color[gray]{0.75} FO%
 &|[fill=green]|\color[gray]{0.75} FO%
 &|[fill=white]|\color[gray]{0.5}WA%
 &|[fill=green]|\color[gray]{0.75} FO%
 &|[fill=green]|\color[gray]{0.75} FO%
 &|[fill=green]|\color[gray]{0.75} FO%
 &|[fill=green]|\color[gray]{0.75} FO%
 &|[fill=green]|\color[gray]{0.75} FO%
 &|[fill=white]|\color[gray]{0.5}WA%
 &|[fill=green]|\color[gray]{0.75} FO%
\\
|[fill=green]|\color[gray]{0.75} FO%
 &|[fill=white]|\color[gray]{0.5}WA%
 &|[fill=white]|\color[gray]{0.5}WA%
 &|[fill=green]|\color[gray]{0.75} FO%
 &|[fill=green]|\color[gray]{0.75} FO%
 &|[fill=green]|\color[gray]{0.75} FO%
 &|[fill=green]|\color[gray]{0.75} FO%
 &|[fill=green]|\color[gray]{0.75} FO%
 &|[fill=green]|\color[gray]{0.75} FO%
 &|[fill=white]|\color[gray]{0.5}WA%
\\
|[fill=green]|\color[gray]{0.75} FO%
 &|[fill=green]|\color[gray]{0.75} FO%
 &|[fill=green]|\color[gray]{0.75} FO%
 &|[fill=green]|\color[gray]{0.75} FO%
 &|[fill=green]|\color[gray]{0.75} FO%
 &|[fill=green]|\color[gray]{0.75} FO%
 &|[fill=green]|\color[gray]{0.75} FO%
 &|[fill=green]|\color[gray]{0.75} FO%
 &|[fill=green]|\color[gray]{0.75} FO%
 &|[fill=green]|\color[gray]{0.75} FO%
\\
|[fill=green]|\color[gray]{0.75} FO%
 &|[fill=green]|\color[gray]{0.75} FO%
 &|[fill=white]|\color[gray]{0.5}WA%
 &|[fill=white]|\color[gray]{0.5}WA%
 &|[fill=white]|\color[gray]{0.5}WA%
 &|[fill=cyan]|\color[rgb]{1,0,0}\textbf{01}%
 &|[fill=white]|\color[gray]{0.5}WA%
 &|[fill=white]|\color[gray]{0.5}WA%
 &|[fill=white]|\color[gray]{0.5}WA%
 &|[fill=green]|\color[gray]{0.75} FO%
\\
|[fill=green]|\color[gray]{0.75} FO%
 &|[fill=green]|\color[gray]{0.75} FO%
 &|[fill=green]|\color[gray]{0.75} FO%
 &|[fill=cyan]|\color[rgb]{1,0,0}\textbf{02}%
 &|[fill=green]|\color[rgb]{0,0,0}\textbf{CO}%
 &|[fill=green]|\color[rgb]{0,0,0}\textbf{CO}%
 &|[fill=green]|\color[rgb]{0,0,0}\textbf{CO}%
 &|[fill=green]|\color[rgb]{0,0,0}\textbf{CO}%
 &|[fill=cyan]|\color[rgb]{1,0,0}\textbf{03}%
 &|[fill=green]|\color[gray]{0.75} FO%
\\
|[fill=green]|\color[gray]{0.75} FO%
 &|[fill=green]|\color[gray]{0.75} FO%
 &|[fill=white]|\color[gray]{0.5}WA%
 &|[fill=white]|\color[gray]{0.5}WA%
 &|[fill=green]|\color[rgb]{0,0,0}\textbf{CO}%
 &|[fill=green]|\color[rgb]{0,0,0}\textbf{CO}%
 &|[fill=green]|\color[rgb]{0,0,0}\textbf{CO}%
 &|[fill=green]|\color[rgb]{0,0,0}\textbf{CO}%
 &|[fill=cyan]|\color[rgb]{1,0,0}\textbf{04}%
 &|[fill=green]|\color[gray]{0.75} FO%
\\
|[fill=green]|\color[gray]{0.75} FO%
 &|[fill=green]|\color[gray]{0.75} FO%
 &|[fill=green]|\color[gray]{0.75} FO%
 &|[fill=green]|\color[gray]{0.75} FO%
 &|[fill=white]|\color[gray]{0.5}WA%
 &|[fill=green]|\color[rgb]{0,0,0}\textbf{CO}%
 &|[fill=green]|\color[rgb]{0,0,0}\textbf{CO}%
 &|[fill=white]|\color[gray]{0.5}WA%
 &|[fill=green]|\color[gray]{0.75} FO%
 &|[fill=green]|\color[gray]{0.75} FO%
\\
|[fill=white]|\color[gray]{0.5}WA%
 &|[fill=green]|\color[gray]{0.75} FO%
 &|[fill=green]|\color[gray]{0.75} FO%
 &|[fill=green]|\color[gray]{0.75} FO%
 &|[fill=white]|\color[gray]{0.5}WA%
 &|[fill=green]|\color[rgb]{0,0,0}\textbf{CO}%
 &|[fill=green]|\color[rgb]{0,0,0}\textbf{CO}%
 &|[fill=white]|\color[gray]{0.5}WA%
 &|[fill=green]|\color[gray]{0.75} FO%
 &|[fill=white]|\color[gray]{0.5}WA%
\\
|[fill=green]|\color[gray]{0.75} FO%
 &|[fill=white]|\color[gray]{0.5}WA%
 &|[fill=green]|\color[gray]{0.75} FO%
 &|[fill=green]|\color[gray]{0.75} FO%
 &|[fill=white]|\color[gray]{0.5}WA%
 &|[fill=green]|\color[rgb]{0,0,0}\textbf{CO}%
 &|[fill=green]|\color[rgb]{0,0,0}\textbf{CO}%
 &|[fill=white]|\color[gray]{0.5}WA%
 &|[fill=green]|\color[gray]{0.75} FO%
 &|[fill=green]|\color[gray]{0.75} FO%
\\
|[fill=green]|\color[gray]{0.75} FO%
 &|[fill=green]|\color[gray]{0.75} FO%
 &|[fill=green]|\color[gray]{0.75} FO%
 &|[fill=green]|\color[gray]{0.75} FO%
 &|[fill=green]|\color[gray]{0.75} FO%
 &|[fill=cyan]|\color[rgb]{1,0,0}\textbf{05}%
 &|[fill=cyan]|\color[rgb]{1,0,0}\textbf{06}%
 &|[fill=green]|\color[gray]{0.75} FO%
 &|[fill=green]|\color[gray]{0.75} FO%
 &|[fill=green]|\color[gray]{0.75} FO%
\\
};
\end{tikzpicture}\\
\\
Explanation:\\
\colorbox{white}{\color[gray]{0.5}WA}  ---  EMPTY\\
\colorbox{green}{\color[gray]{0.5}FO}  ---  BURNABLE\\
\colorbox{white}{\color[rgb]{1,0,0}\textbf{BU}}  ---  BURNED\\
\colorbox{white}{\color[rgb]{0,0,0}\textbf{CO}}  ---  COAL (doubly burned)\\
\colorbox{cyan}{\#\#}  ---  WATERED at time \#\#\\
Fields can have more than 1 state.
}
\subsubsection{Beispiel 1}
Eine Situation mit mehr als einem Feuer bei der ersten Beobachtung\footnote{Diese Eingabe finden Sie auch in der Datei \texttt{1.in}}:
{\small
\lstinputlisting{../Aufgabe_1/1.in}
}
Mein Programm produziert folgende Ausgabe\footnote{Diese Ausgabe finden Sie auch in der Datei \texttt{1.out.tex}; Eine Datei \texttt{1.out} mit den ASCII-Escape-Sequenzen exisitert ebenfalls.}:\\
{\ttfamily \small
\\
\begin{tikzpicture}
\tikzset{square matrix/.style={
matrix of nodes,
column sep=-\pgflinewidth, row sep=-\pgflinewidth,
nodes={draw,
minimum height=#1,
anchor=center,
text width=#1,
align=center,
inner sep=0pt
},
},
square matrix/.default=1.2cm
}
\matrix[square matrix=1.4em] {
|[fill=green]|\color[rgb]{1,0,0}\textbf{BU}%
 &|[fill=green]|\color[gray]{0.75} FO%
 &|[fill=white]|\color[gray]{0.5}WA%
 &|[fill=green]|\color[gray]{0.75} FO%
 &|[fill=green]|\color[gray]{0.75} FO%
 &|[fill=green]|\color[gray]{0.75} FO%
 &|[fill=green]|\color[gray]{0.75} FO%
 &|[fill=green]|\color[gray]{0.75} FO%
 &|[fill=white]|\color[gray]{0.5}WA%
 &|[fill=green]|\color[gray]{0.75} FO%
\\
|[fill=green]|\color[gray]{0.75} FO%
 &|[fill=white]|\color[gray]{0.5}WA%
 &|[fill=white]|\color[gray]{0.5}WA%
 &|[fill=green]|\color[gray]{0.75} FO%
 &|[fill=green]|\color[gray]{0.75} FO%
 &|[fill=green]|\color[gray]{0.75} FO%
 &|[fill=green]|\color[gray]{0.75} FO%
 &|[fill=green]|\color[gray]{0.75} FO%
 &|[fill=green]|\color[gray]{0.75} FO%
 &|[fill=white]|\color[gray]{0.5}WA%
\\
|[fill=green]|\color[gray]{0.75} FO%
 &|[fill=green]|\color[gray]{0.75} FO%
 &|[fill=green]|\color[gray]{0.75} FO%
 &|[fill=green]|\color[gray]{0.75} FO%
 &|[fill=green]|\color[gray]{0.75} FO%
 &|[fill=green]|\color[gray]{0.75} FO%
 &|[fill=green]|\color[gray]{0.75} FO%
 &|[fill=green]|\color[gray]{0.75} FO%
 &|[fill=green]|\color[gray]{0.75} FO%
 &|[fill=green]|\color[gray]{0.75} FO%
\\
|[fill=green]|\color[gray]{0.75} FO%
 &|[fill=green]|\color[gray]{0.75} FO%
 &|[fill=white]|\color[gray]{0.5}WA%
 &|[fill=white]|\color[gray]{0.5}WA%
 &|[fill=white]|\color[gray]{0.5}WA%
 &|[fill=green]|\color[gray]{0.75} FO%
 &|[fill=white]|\color[gray]{0.5}WA%
 &|[fill=white]|\color[gray]{0.5}WA%
 &|[fill=white]|\color[gray]{0.5}WA%
 &|[fill=green]|\color[gray]{0.75} FO%
\\
|[fill=green]|\color[gray]{0.75} FO%
 &|[fill=green]|\color[gray]{0.75} FO%
 &|[fill=green]|\color[gray]{0.75} FO%
 &|[fill=green]|\color[gray]{0.75} FO%
 &|[fill=green]|\color[gray]{0.75} FO%
 &|[fill=green]|\color[rgb]{1,0,0}\textbf{BU}%
 &|[fill=green]|\color[gray]{0.75} FO%
 &|[fill=green]|\color[gray]{0.75} FO%
 &|[fill=green]|\color[gray]{0.75} FO%
 &|[fill=green]|\color[gray]{0.75} FO%
\\
|[fill=green]|\color[gray]{0.75} FO%
 &|[fill=green]|\color[gray]{0.75} FO%
 &|[fill=white]|\color[gray]{0.5}WA%
 &|[fill=white]|\color[gray]{0.5}WA%
 &|[fill=green]|\color[gray]{0.75} FO%
 &|[fill=green]|\color[gray]{0.75} FO%
 &|[fill=green]|\color[gray]{0.75} FO%
 &|[fill=green]|\color[gray]{0.75} FO%
 &|[fill=green]|\color[gray]{0.75} FO%
 &|[fill=green]|\color[gray]{0.75} FO%
\\
|[fill=green]|\color[gray]{0.75} FO%
 &|[fill=green]|\color[gray]{0.75} FO%
 &|[fill=green]|\color[gray]{0.75} FO%
 &|[fill=green]|\color[gray]{0.75} FO%
 &|[fill=white]|\color[gray]{0.5}WA%
 &|[fill=green]|\color[gray]{0.75} FO%
 &|[fill=green]|\color[gray]{0.75} FO%
 &|[fill=white]|\color[gray]{0.5}WA%
 &|[fill=green]|\color[gray]{0.75} FO%
 &|[fill=green]|\color[gray]{0.75} FO%
\\
|[fill=white]|\color[gray]{0.5}WA%
 &|[fill=green]|\color[gray]{0.75} FO%
 &|[fill=green]|\color[gray]{0.75} FO%
 &|[fill=green]|\color[gray]{0.75} FO%
 &|[fill=white]|\color[gray]{0.5}WA%
 &|[fill=green]|\color[gray]{0.75} FO%
 &|[fill=green]|\color[gray]{0.75} FO%
 &|[fill=white]|\color[gray]{0.5}WA%
 &|[fill=green]|\color[gray]{0.75} FO%
 &|[fill=white]|\color[gray]{0.5}WA%
\\
|[fill=green]|\color[gray]{0.75} FO%
 &|[fill=white]|\color[gray]{0.5}WA%
 &|[fill=green]|\color[gray]{0.75} FO%
 &|[fill=green]|\color[gray]{0.75} FO%
 &|[fill=white]|\color[gray]{0.5}WA%
 &|[fill=green]|\color[gray]{0.75} FO%
 &|[fill=green]|\color[gray]{0.75} FO%
 &|[fill=white]|\color[gray]{0.5}WA%
 &|[fill=green]|\color[gray]{0.75} FO%
 &|[fill=green]|\color[gray]{0.75} FO%
\\
|[fill=green]|\color[gray]{0.75} FO%
 &|[fill=green]|\color[gray]{0.75} FO%
 &|[fill=green]|\color[gray]{0.75} FO%
 &|[fill=green]|\color[gray]{0.75} FO%
 &|[fill=green]|\color[gray]{0.75} FO%
 &|[fill=green]|\color[gray]{0.75} FO%
 &|[fill=green]|\color[rgb]{1,0,0}\textbf{BU}%
 &|[fill=green]|\color[gray]{0.75} FO%
 &|[fill=green]|\color[gray]{0.75} FO%
 &|[fill=green]|\color[gray]{0.75} FO%
\\
|[fill=green]|\color[gray]{0.75} FO%
 &|[fill=green]|\color[gray]{0.75} FO%
 &|[fill=green]|\color[gray]{0.75} FO%
 &|[fill=green]|\color[gray]{0.75} FO%
 &|[fill=green]|\color[gray]{0.75} FO%
 &|[fill=green]|\color[gray]{0.75} FO%
 &|[fill=green]|\color[gray]{0.75} FO%
 &|[fill=green]|\color[gray]{0.75} FO%
 &|[fill=green]|\color[gray]{0.75} FO%
 &|[fill=green]|\color[gray]{0.75} FO%
\\
};
\end{tikzpicture}\\
At time 1: Water spot (5|3):
\\
\begin{tikzpicture}
\tikzset{square matrix/.style={
matrix of nodes,
column sep=-\pgflinewidth, row sep=-\pgflinewidth,
nodes={draw,
minimum height=#1,
anchor=center,
text width=#1,
align=center,
inner sep=0pt
},
},
square matrix/.default=1.2cm
}
\matrix[square matrix=1.4em] {
|[fill=green]|\color[rgb]{0,0,0}\textbf{CO}%
 &|[fill=green]|\color[rgb]{1,0,0}\textbf{BU}%
 &|[fill=white]|\color[gray]{0.5}WA%
 &|[fill=green]|\color[gray]{0.75} FO%
 &|[fill=green]|\color[gray]{0.75} FO%
 &|[fill=green]|\color[gray]{0.75} FO%
 &|[fill=green]|\color[gray]{0.75} FO%
 &|[fill=green]|\color[gray]{0.75} FO%
 &|[fill=white]|\color[gray]{0.5}WA%
 &|[fill=green]|\color[gray]{0.75} FO%
\\
|[fill=green]|\color[rgb]{1,0,0}\textbf{BU}%
 &|[fill=white]|\color[gray]{0.5}WA%
 &|[fill=white]|\color[gray]{0.5}WA%
 &|[fill=green]|\color[gray]{0.75} FO%
 &|[fill=green]|\color[gray]{0.75} FO%
 &|[fill=green]|\color[gray]{0.75} FO%
 &|[fill=green]|\color[gray]{0.75} FO%
 &|[fill=green]|\color[gray]{0.75} FO%
 &|[fill=green]|\color[gray]{0.75} FO%
 &|[fill=white]|\color[gray]{0.5}WA%
\\
|[fill=green]|\color[gray]{0.75} FO%
 &|[fill=green]|\color[gray]{0.75} FO%
 &|[fill=green]|\color[gray]{0.75} FO%
 &|[fill=green]|\color[gray]{0.75} FO%
 &|[fill=green]|\color[gray]{0.75} FO%
 &|[fill=green]|\color[gray]{0.75} FO%
 &|[fill=green]|\color[gray]{0.75} FO%
 &|[fill=green]|\color[gray]{0.75} FO%
 &|[fill=green]|\color[gray]{0.75} FO%
 &|[fill=green]|\color[gray]{0.75} FO%
\\
|[fill=green]|\color[gray]{0.75} FO%
 &|[fill=green]|\color[gray]{0.75} FO%
 &|[fill=white]|\color[gray]{0.5}WA%
 &|[fill=white]|\color[gray]{0.5}WA%
 &|[fill=white]|\color[gray]{0.5}WA%
 &|[fill=cyan]|\color[rgb]{1,0,0}\textbf{01}%
 &|[fill=white]|\color[gray]{0.5}WA%
 &|[fill=white]|\color[gray]{0.5}WA%
 &|[fill=white]|\color[gray]{0.5}WA%
 &|[fill=green]|\color[gray]{0.75} FO%
\\
|[fill=green]|\color[gray]{0.75} FO%
 &|[fill=green]|\color[gray]{0.75} FO%
 &|[fill=green]|\color[gray]{0.75} FO%
 &|[fill=green]|\color[gray]{0.75} FO%
 &|[fill=green]|\color[rgb]{1,0,0}\textbf{BU}%
 &|[fill=green]|\color[rgb]{0,0,0}\textbf{CO}%
 &|[fill=green]|\color[rgb]{1,0,0}\textbf{BU}%
 &|[fill=green]|\color[gray]{0.75} FO%
 &|[fill=green]|\color[gray]{0.75} FO%
 &|[fill=green]|\color[gray]{0.75} FO%
\\
|[fill=green]|\color[gray]{0.75} FO%
 &|[fill=green]|\color[gray]{0.75} FO%
 &|[fill=white]|\color[gray]{0.5}WA%
 &|[fill=white]|\color[gray]{0.5}WA%
 &|[fill=green]|\color[gray]{0.75} FO%
 &|[fill=green]|\color[rgb]{1,0,0}\textbf{BU}%
 &|[fill=green]|\color[gray]{0.75} FO%
 &|[fill=green]|\color[gray]{0.75} FO%
 &|[fill=green]|\color[gray]{0.75} FO%
 &|[fill=green]|\color[gray]{0.75} FO%
\\
|[fill=green]|\color[gray]{0.75} FO%
 &|[fill=green]|\color[gray]{0.75} FO%
 &|[fill=green]|\color[gray]{0.75} FO%
 &|[fill=green]|\color[gray]{0.75} FO%
 &|[fill=white]|\color[gray]{0.5}WA%
 &|[fill=green]|\color[gray]{0.75} FO%
 &|[fill=green]|\color[gray]{0.75} FO%
 &|[fill=white]|\color[gray]{0.5}WA%
 &|[fill=green]|\color[gray]{0.75} FO%
 &|[fill=green]|\color[gray]{0.75} FO%
\\
|[fill=white]|\color[gray]{0.5}WA%
 &|[fill=green]|\color[gray]{0.75} FO%
 &|[fill=green]|\color[gray]{0.75} FO%
 &|[fill=green]|\color[gray]{0.75} FO%
 &|[fill=white]|\color[gray]{0.5}WA%
 &|[fill=green]|\color[gray]{0.75} FO%
 &|[fill=green]|\color[gray]{0.75} FO%
 &|[fill=white]|\color[gray]{0.5}WA%
 &|[fill=green]|\color[gray]{0.75} FO%
 &|[fill=white]|\color[gray]{0.5}WA%
\\
|[fill=green]|\color[gray]{0.75} FO%
 &|[fill=white]|\color[gray]{0.5}WA%
 &|[fill=green]|\color[gray]{0.75} FO%
 &|[fill=green]|\color[gray]{0.75} FO%
 &|[fill=white]|\color[gray]{0.5}WA%
 &|[fill=green]|\color[gray]{0.75} FO%
 &|[fill=green]|\color[rgb]{1,0,0}\textbf{BU}%
 &|[fill=white]|\color[gray]{0.5}WA%
 &|[fill=green]|\color[gray]{0.75} FO%
 &|[fill=green]|\color[gray]{0.75} FO%
\\
|[fill=green]|\color[gray]{0.75} FO%
 &|[fill=green]|\color[gray]{0.75} FO%
 &|[fill=green]|\color[gray]{0.75} FO%
 &|[fill=green]|\color[gray]{0.75} FO%
 &|[fill=green]|\color[gray]{0.75} FO%
 &|[fill=green]|\color[rgb]{1,0,0}\textbf{BU}%
 &|[fill=green]|\color[rgb]{0,0,0}\textbf{CO}%
 &|[fill=green]|\color[rgb]{1,0,0}\textbf{BU}%
 &|[fill=green]|\color[gray]{0.75} FO%
 &|[fill=green]|\color[gray]{0.75} FO%
\\
|[fill=green]|\color[gray]{0.75} FO%
 &|[fill=green]|\color[gray]{0.75} FO%
 &|[fill=green]|\color[gray]{0.75} FO%
 &|[fill=green]|\color[gray]{0.75} FO%
 &|[fill=green]|\color[gray]{0.75} FO%
 &|[fill=green]|\color[gray]{0.75} FO%
 &|[fill=green]|\color[rgb]{1,0,0}\textbf{BU}%
 &|[fill=green]|\color[gray]{0.75} FO%
 &|[fill=green]|\color[gray]{0.75} FO%
 &|[fill=green]|\color[gray]{0.75} FO%
\\
};
\end{tikzpicture}\\
At time 2: Water spot (0|2):
\\
\begin{tikzpicture}
\tikzset{square matrix/.style={
matrix of nodes,
column sep=-\pgflinewidth, row sep=-\pgflinewidth,
nodes={draw,
minimum height=#1,
anchor=center,
text width=#1,
align=center,
inner sep=0pt
},
},
square matrix/.default=1.2cm
}
\matrix[square matrix=1.4em] {
|[fill=green]|\color[rgb]{0,0,0}\textbf{CO}%
 &|[fill=green]|\color[rgb]{0,0,0}\textbf{CO}%
 &|[fill=white]|\color[gray]{0.5}WA%
 &|[fill=green]|\color[gray]{0.75} FO%
 &|[fill=green]|\color[gray]{0.75} FO%
 &|[fill=green]|\color[gray]{0.75} FO%
 &|[fill=green]|\color[gray]{0.75} FO%
 &|[fill=green]|\color[gray]{0.75} FO%
 &|[fill=white]|\color[gray]{0.5}WA%
 &|[fill=green]|\color[gray]{0.75} FO%
\\
|[fill=green]|\color[rgb]{0,0,0}\textbf{CO}%
 &|[fill=white]|\color[gray]{0.5}WA%
 &|[fill=white]|\color[gray]{0.5}WA%
 &|[fill=green]|\color[gray]{0.75} FO%
 &|[fill=green]|\color[gray]{0.75} FO%
 &|[fill=green]|\color[gray]{0.75} FO%
 &|[fill=green]|\color[gray]{0.75} FO%
 &|[fill=green]|\color[gray]{0.75} FO%
 &|[fill=green]|\color[gray]{0.75} FO%
 &|[fill=white]|\color[gray]{0.5}WA%
\\
|[fill=cyan]|\color[rgb]{1,0,0}\textbf{02}%
 &|[fill=green]|\color[gray]{0.75} FO%
 &|[fill=green]|\color[gray]{0.75} FO%
 &|[fill=green]|\color[gray]{0.75} FO%
 &|[fill=green]|\color[gray]{0.75} FO%
 &|[fill=green]|\color[gray]{0.75} FO%
 &|[fill=green]|\color[gray]{0.75} FO%
 &|[fill=green]|\color[gray]{0.75} FO%
 &|[fill=green]|\color[gray]{0.75} FO%
 &|[fill=green]|\color[gray]{0.75} FO%
\\
|[fill=green]|\color[gray]{0.75} FO%
 &|[fill=green]|\color[gray]{0.75} FO%
 &|[fill=white]|\color[gray]{0.5}WA%
 &|[fill=white]|\color[gray]{0.5}WA%
 &|[fill=white]|\color[gray]{0.5}WA%
 &|[fill=cyan]|\color[rgb]{1,0,0}\textbf{01}%
 &|[fill=white]|\color[gray]{0.5}WA%
 &|[fill=white]|\color[gray]{0.5}WA%
 &|[fill=white]|\color[gray]{0.5}WA%
 &|[fill=green]|\color[gray]{0.75} FO%
\\
|[fill=green]|\color[gray]{0.75} FO%
 &|[fill=green]|\color[gray]{0.75} FO%
 &|[fill=green]|\color[gray]{0.75} FO%
 &|[fill=green]|\color[rgb]{1,0,0}\textbf{BU}%
 &|[fill=green]|\color[rgb]{0,0,0}\textbf{CO}%
 &|[fill=green]|\color[rgb]{0,0,0}\textbf{CO}%
 &|[fill=green]|\color[rgb]{0,0,0}\textbf{CO}%
 &|[fill=green]|\color[rgb]{1,0,0}\textbf{BU}%
 &|[fill=green]|\color[gray]{0.75} FO%
 &|[fill=green]|\color[gray]{0.75} FO%
\\
|[fill=green]|\color[gray]{0.75} FO%
 &|[fill=green]|\color[gray]{0.75} FO%
 &|[fill=white]|\color[gray]{0.5}WA%
 &|[fill=white]|\color[gray]{0.5}WA%
 &|[fill=green]|\color[rgb]{1,0,0}\textbf{BU}%
 &|[fill=green]|\color[rgb]{0,0,0}\textbf{CO}%
 &|[fill=green]|\color[rgb]{1,0,0}\textbf{BU}%
 &|[fill=green]|\color[gray]{0.75} FO%
 &|[fill=green]|\color[gray]{0.75} FO%
 &|[fill=green]|\color[gray]{0.75} FO%
\\
|[fill=green]|\color[gray]{0.75} FO%
 &|[fill=green]|\color[gray]{0.75} FO%
 &|[fill=green]|\color[gray]{0.75} FO%
 &|[fill=green]|\color[gray]{0.75} FO%
 &|[fill=white]|\color[gray]{0.5}WA%
 &|[fill=green]|\color[rgb]{1,0,0}\textbf{BU}%
 &|[fill=green]|\color[gray]{0.75} FO%
 &|[fill=white]|\color[gray]{0.5}WA%
 &|[fill=green]|\color[gray]{0.75} FO%
 &|[fill=green]|\color[gray]{0.75} FO%
\\
|[fill=white]|\color[gray]{0.5}WA%
 &|[fill=green]|\color[gray]{0.75} FO%
 &|[fill=green]|\color[gray]{0.75} FO%
 &|[fill=green]|\color[gray]{0.75} FO%
 &|[fill=white]|\color[gray]{0.5}WA%
 &|[fill=green]|\color[gray]{0.75} FO%
 &|[fill=green]|\color[rgb]{1,0,0}\textbf{BU}%
 &|[fill=white]|\color[gray]{0.5}WA%
 &|[fill=green]|\color[gray]{0.75} FO%
 &|[fill=white]|\color[gray]{0.5}WA%
\\
|[fill=green]|\color[gray]{0.75} FO%
 &|[fill=white]|\color[gray]{0.5}WA%
 &|[fill=green]|\color[gray]{0.75} FO%
 &|[fill=green]|\color[gray]{0.75} FO%
 &|[fill=white]|\color[gray]{0.5}WA%
 &|[fill=green]|\color[rgb]{1,0,0}\textbf{BU}%
 &|[fill=green]|\color[rgb]{0,0,0}\textbf{CO}%
 &|[fill=white]|\color[gray]{0.5}WA%
 &|[fill=green]|\color[gray]{0.75} FO%
 &|[fill=green]|\color[gray]{0.75} FO%
\\
|[fill=green]|\color[gray]{0.75} FO%
 &|[fill=green]|\color[gray]{0.75} FO%
 &|[fill=green]|\color[gray]{0.75} FO%
 &|[fill=green]|\color[gray]{0.75} FO%
 &|[fill=green]|\color[rgb]{1,0,0}\textbf{BU}%
 &|[fill=green]|\color[rgb]{0,0,0}\textbf{CO}%
 &|[fill=green]|\color[rgb]{0,0,0}\textbf{CO}%
 &|[fill=green]|\color[rgb]{0,0,0}\textbf{CO}%
 &|[fill=green]|\color[rgb]{1,0,0}\textbf{BU}%
 &|[fill=green]|\color[gray]{0.75} FO%
\\
|[fill=green]|\color[gray]{0.75} FO%
 &|[fill=green]|\color[gray]{0.75} FO%
 &|[fill=green]|\color[gray]{0.75} FO%
 &|[fill=green]|\color[gray]{0.75} FO%
 &|[fill=green]|\color[gray]{0.75} FO%
 &|[fill=green]|\color[rgb]{1,0,0}\textbf{BU}%
 &|[fill=green]|\color[rgb]{0,0,0}\textbf{CO}%
 &|[fill=green]|\color[rgb]{1,0,0}\textbf{BU}%
 &|[fill=green]|\color[gray]{0.75} FO%
 &|[fill=green]|\color[gray]{0.75} FO%
\\
};
\end{tikzpicture}\\
At time 3: Water spot (2|4):
\\
\begin{tikzpicture}
\tikzset{square matrix/.style={
matrix of nodes,
column sep=-\pgflinewidth, row sep=-\pgflinewidth,
nodes={draw,
minimum height=#1,
anchor=center,
text width=#1,
align=center,
inner sep=0pt
},
},
square matrix/.default=1.2cm
}
\matrix[square matrix=1.4em] {
|[fill=green]|\color[rgb]{0,0,0}\textbf{CO}%
 &|[fill=green]|\color[rgb]{0,0,0}\textbf{CO}%
 &|[fill=white]|\color[gray]{0.5}WA%
 &|[fill=green]|\color[gray]{0.75} FO%
 &|[fill=green]|\color[gray]{0.75} FO%
 &|[fill=green]|\color[gray]{0.75} FO%
 &|[fill=green]|\color[gray]{0.75} FO%
 &|[fill=green]|\color[gray]{0.75} FO%
 &|[fill=white]|\color[gray]{0.5}WA%
 &|[fill=green]|\color[gray]{0.75} FO%
\\
|[fill=green]|\color[rgb]{0,0,0}\textbf{CO}%
 &|[fill=white]|\color[gray]{0.5}WA%
 &|[fill=white]|\color[gray]{0.5}WA%
 &|[fill=green]|\color[gray]{0.75} FO%
 &|[fill=green]|\color[gray]{0.75} FO%
 &|[fill=green]|\color[gray]{0.75} FO%
 &|[fill=green]|\color[gray]{0.75} FO%
 &|[fill=green]|\color[gray]{0.75} FO%
 &|[fill=green]|\color[gray]{0.75} FO%
 &|[fill=white]|\color[gray]{0.5}WA%
\\
|[fill=cyan]|\color[rgb]{1,0,0}\textbf{02}%
 &|[fill=green]|\color[gray]{0.75} FO%
 &|[fill=green]|\color[gray]{0.75} FO%
 &|[fill=green]|\color[gray]{0.75} FO%
 &|[fill=green]|\color[gray]{0.75} FO%
 &|[fill=green]|\color[gray]{0.75} FO%
 &|[fill=green]|\color[gray]{0.75} FO%
 &|[fill=green]|\color[gray]{0.75} FO%
 &|[fill=green]|\color[gray]{0.75} FO%
 &|[fill=green]|\color[gray]{0.75} FO%
\\
|[fill=green]|\color[gray]{0.75} FO%
 &|[fill=green]|\color[gray]{0.75} FO%
 &|[fill=white]|\color[gray]{0.5}WA%
 &|[fill=white]|\color[gray]{0.5}WA%
 &|[fill=white]|\color[gray]{0.5}WA%
 &|[fill=cyan]|\color[rgb]{1,0,0}\textbf{01}%
 &|[fill=white]|\color[gray]{0.5}WA%
 &|[fill=white]|\color[gray]{0.5}WA%
 &|[fill=white]|\color[gray]{0.5}WA%
 &|[fill=green]|\color[gray]{0.75} FO%
\\
|[fill=green]|\color[gray]{0.75} FO%
 &|[fill=green]|\color[gray]{0.75} FO%
 &|[fill=cyan]|\color[rgb]{1,0,0}\textbf{03}%
 &|[fill=green]|\color[rgb]{0,0,0}\textbf{CO}%
 &|[fill=green]|\color[rgb]{0,0,0}\textbf{CO}%
 &|[fill=green]|\color[rgb]{0,0,0}\textbf{CO}%
 &|[fill=green]|\color[rgb]{0,0,0}\textbf{CO}%
 &|[fill=green]|\color[rgb]{0,0,0}\textbf{CO}%
 &|[fill=green]|\color[rgb]{1,0,0}\textbf{BU}%
 &|[fill=green]|\color[gray]{0.75} FO%
\\
|[fill=green]|\color[gray]{0.75} FO%
 &|[fill=green]|\color[gray]{0.75} FO%
 &|[fill=white]|\color[gray]{0.5}WA%
 &|[fill=white]|\color[gray]{0.5}WA%
 &|[fill=green]|\color[rgb]{0,0,0}\textbf{CO}%
 &|[fill=green]|\color[rgb]{0,0,0}\textbf{CO}%
 &|[fill=green]|\color[rgb]{0,0,0}\textbf{CO}%
 &|[fill=green]|\color[rgb]{1,0,0}\textbf{BU}%
 &|[fill=green]|\color[gray]{0.75} FO%
 &|[fill=green]|\color[gray]{0.75} FO%
\\
|[fill=green]|\color[gray]{0.75} FO%
 &|[fill=green]|\color[gray]{0.75} FO%
 &|[fill=green]|\color[gray]{0.75} FO%
 &|[fill=green]|\color[gray]{0.75} FO%
 &|[fill=white]|\color[gray]{0.5}WA%
 &|[fill=green]|\color[rgb]{0,0,0}\textbf{CO}%
 &|[fill=green]|\color[rgb]{1,0,0}\textbf{BU}%
 &|[fill=white]|\color[gray]{0.5}WA%
 &|[fill=green]|\color[gray]{0.75} FO%
 &|[fill=green]|\color[gray]{0.75} FO%
\\
|[fill=white]|\color[gray]{0.5}WA%
 &|[fill=green]|\color[gray]{0.75} FO%
 &|[fill=green]|\color[gray]{0.75} FO%
 &|[fill=green]|\color[gray]{0.75} FO%
 &|[fill=white]|\color[gray]{0.5}WA%
 &|[fill=green]|\color[rgb]{1,0,0}\textbf{BU}%
 &|[fill=green]|\color[rgb]{0,0,0}\textbf{CO}%
 &|[fill=white]|\color[gray]{0.5}WA%
 &|[fill=green]|\color[gray]{0.75} FO%
 &|[fill=white]|\color[gray]{0.5}WA%
\\
|[fill=green]|\color[gray]{0.75} FO%
 &|[fill=white]|\color[gray]{0.5}WA%
 &|[fill=green]|\color[gray]{0.75} FO%
 &|[fill=green]|\color[gray]{0.75} FO%
 &|[fill=white]|\color[gray]{0.5}WA%
 &|[fill=green]|\color[rgb]{0,0,0}\textbf{CO}%
 &|[fill=green]|\color[rgb]{0,0,0}\textbf{CO}%
 &|[fill=white]|\color[gray]{0.5}WA%
 &|[fill=green]|\color[rgb]{1,0,0}\textbf{BU}%
 &|[fill=green]|\color[gray]{0.75} FO%
\\
|[fill=green]|\color[gray]{0.75} FO%
 &|[fill=green]|\color[gray]{0.75} FO%
 &|[fill=green]|\color[gray]{0.75} FO%
 &|[fill=green]|\color[rgb]{1,0,0}\textbf{BU}%
 &|[fill=green]|\color[rgb]{0,0,0}\textbf{CO}%
 &|[fill=green]|\color[rgb]{0,0,0}\textbf{CO}%
 &|[fill=green]|\color[rgb]{0,0,0}\textbf{CO}%
 &|[fill=green]|\color[rgb]{0,0,0}\textbf{CO}%
 &|[fill=green]|\color[rgb]{0,0,0}\textbf{CO}%
 &|[fill=green]|\color[rgb]{1,0,0}\textbf{BU}%
\\
|[fill=green]|\color[gray]{0.75} FO%
 &|[fill=green]|\color[gray]{0.75} FO%
 &|[fill=green]|\color[gray]{0.75} FO%
 &|[fill=green]|\color[gray]{0.75} FO%
 &|[fill=green]|\color[rgb]{1,0,0}\textbf{BU}%
 &|[fill=green]|\color[rgb]{0,0,0}\textbf{CO}%
 &|[fill=green]|\color[rgb]{0,0,0}\textbf{CO}%
 &|[fill=green]|\color[rgb]{0,0,0}\textbf{CO}%
 &|[fill=green]|\color[rgb]{1,0,0}\textbf{BU}%
 &|[fill=green]|\color[gray]{0.75} FO%
\\
};
\end{tikzpicture}\\
At time 4: Water spot (9|4):
\\
\begin{tikzpicture}
\tikzset{square matrix/.style={
matrix of nodes,
column sep=-\pgflinewidth, row sep=-\pgflinewidth,
nodes={draw,
minimum height=#1,
anchor=center,
text width=#1,
align=center,
inner sep=0pt
},
},
square matrix/.default=1.2cm
}
\matrix[square matrix=1.4em] {
|[fill=green]|\color[rgb]{0,0,0}\textbf{CO}%
 &|[fill=green]|\color[rgb]{0,0,0}\textbf{CO}%
 &|[fill=white]|\color[gray]{0.5}WA%
 &|[fill=green]|\color[gray]{0.75} FO%
 &|[fill=green]|\color[gray]{0.75} FO%
 &|[fill=green]|\color[gray]{0.75} FO%
 &|[fill=green]|\color[gray]{0.75} FO%
 &|[fill=green]|\color[gray]{0.75} FO%
 &|[fill=white]|\color[gray]{0.5}WA%
 &|[fill=green]|\color[gray]{0.75} FO%
\\
|[fill=green]|\color[rgb]{0,0,0}\textbf{CO}%
 &|[fill=white]|\color[gray]{0.5}WA%
 &|[fill=white]|\color[gray]{0.5}WA%
 &|[fill=green]|\color[gray]{0.75} FO%
 &|[fill=green]|\color[gray]{0.75} FO%
 &|[fill=green]|\color[gray]{0.75} FO%
 &|[fill=green]|\color[gray]{0.75} FO%
 &|[fill=green]|\color[gray]{0.75} FO%
 &|[fill=green]|\color[gray]{0.75} FO%
 &|[fill=white]|\color[gray]{0.5}WA%
\\
|[fill=cyan]|\color[rgb]{1,0,0}\textbf{02}%
 &|[fill=green]|\color[gray]{0.75} FO%
 &|[fill=green]|\color[gray]{0.75} FO%
 &|[fill=green]|\color[gray]{0.75} FO%
 &|[fill=green]|\color[gray]{0.75} FO%
 &|[fill=green]|\color[gray]{0.75} FO%
 &|[fill=green]|\color[gray]{0.75} FO%
 &|[fill=green]|\color[gray]{0.75} FO%
 &|[fill=green]|\color[gray]{0.75} FO%
 &|[fill=green]|\color[gray]{0.75} FO%
\\
|[fill=green]|\color[gray]{0.75} FO%
 &|[fill=green]|\color[gray]{0.75} FO%
 &|[fill=white]|\color[gray]{0.5}WA%
 &|[fill=white]|\color[gray]{0.5}WA%
 &|[fill=white]|\color[gray]{0.5}WA%
 &|[fill=cyan]|\color[rgb]{1,0,0}\textbf{01}%
 &|[fill=white]|\color[gray]{0.5}WA%
 &|[fill=white]|\color[gray]{0.5}WA%
 &|[fill=white]|\color[gray]{0.5}WA%
 &|[fill=green]|\color[gray]{0.75} FO%
\\
|[fill=green]|\color[gray]{0.75} FO%
 &|[fill=green]|\color[gray]{0.75} FO%
 &|[fill=cyan]|\color[rgb]{1,0,0}\textbf{03}%
 &|[fill=green]|\color[rgb]{0,0,0}\textbf{CO}%
 &|[fill=green]|\color[rgb]{0,0,0}\textbf{CO}%
 &|[fill=green]|\color[rgb]{0,0,0}\textbf{CO}%
 &|[fill=green]|\color[rgb]{0,0,0}\textbf{CO}%
 &|[fill=green]|\color[rgb]{0,0,0}\textbf{CO}%
 &|[fill=green]|\color[rgb]{0,0,0}\textbf{CO}%
 &|[fill=cyan]|\color[rgb]{1,0,0}\textbf{04}%
\\
|[fill=green]|\color[gray]{0.75} FO%
 &|[fill=green]|\color[gray]{0.75} FO%
 &|[fill=white]|\color[gray]{0.5}WA%
 &|[fill=white]|\color[gray]{0.5}WA%
 &|[fill=green]|\color[rgb]{0,0,0}\textbf{CO}%
 &|[fill=green]|\color[rgb]{0,0,0}\textbf{CO}%
 &|[fill=green]|\color[rgb]{0,0,0}\textbf{CO}%
 &|[fill=green]|\color[rgb]{0,0,0}\textbf{CO}%
 &|[fill=green]|\color[rgb]{1,0,0}\textbf{BU}%
 &|[fill=green]|\color[gray]{0.75} FO%
\\
|[fill=green]|\color[gray]{0.75} FO%
 &|[fill=green]|\color[gray]{0.75} FO%
 &|[fill=green]|\color[gray]{0.75} FO%
 &|[fill=green]|\color[gray]{0.75} FO%
 &|[fill=white]|\color[gray]{0.5}WA%
 &|[fill=green]|\color[rgb]{0,0,0}\textbf{CO}%
 &|[fill=green]|\color[rgb]{0,0,0}\textbf{CO}%
 &|[fill=white]|\color[gray]{0.5}WA%
 &|[fill=green]|\color[gray]{0.75} FO%
 &|[fill=green]|\color[gray]{0.75} FO%
\\
|[fill=white]|\color[gray]{0.5}WA%
 &|[fill=green]|\color[gray]{0.75} FO%
 &|[fill=green]|\color[gray]{0.75} FO%
 &|[fill=green]|\color[gray]{0.75} FO%
 &|[fill=white]|\color[gray]{0.5}WA%
 &|[fill=green]|\color[rgb]{0,0,0}\textbf{CO}%
 &|[fill=green]|\color[rgb]{0,0,0}\textbf{CO}%
 &|[fill=white]|\color[gray]{0.5}WA%
 &|[fill=green]|\color[rgb]{1,0,0}\textbf{BU}%
 &|[fill=white]|\color[gray]{0.5}WA%
\\
|[fill=green]|\color[gray]{0.75} FO%
 &|[fill=white]|\color[gray]{0.5}WA%
 &|[fill=green]|\color[gray]{0.75} FO%
 &|[fill=green]|\color[rgb]{1,0,0}\textbf{BU}%
 &|[fill=white]|\color[gray]{0.5}WA%
 &|[fill=green]|\color[rgb]{0,0,0}\textbf{CO}%
 &|[fill=green]|\color[rgb]{0,0,0}\textbf{CO}%
 &|[fill=white]|\color[gray]{0.5}WA%
 &|[fill=green]|\color[rgb]{0,0,0}\textbf{CO}%
 &|[fill=green]|\color[rgb]{1,0,0}\textbf{BU}%
\\
|[fill=green]|\color[gray]{0.75} FO%
 &|[fill=green]|\color[gray]{0.75} FO%
 &|[fill=green]|\color[rgb]{1,0,0}\textbf{BU}%
 &|[fill=green]|\color[rgb]{0,0,0}\textbf{CO}%
 &|[fill=green]|\color[rgb]{0,0,0}\textbf{CO}%
 &|[fill=green]|\color[rgb]{0,0,0}\textbf{CO}%
 &|[fill=green]|\color[rgb]{0,0,0}\textbf{CO}%
 &|[fill=green]|\color[rgb]{0,0,0}\textbf{CO}%
 &|[fill=green]|\color[rgb]{0,0,0}\textbf{CO}%
 &|[fill=green]|\color[rgb]{0,0,0}\textbf{CO}%
\\
|[fill=green]|\color[gray]{0.75} FO%
 &|[fill=green]|\color[gray]{0.75} FO%
 &|[fill=green]|\color[gray]{0.75} FO%
 &|[fill=green]|\color[rgb]{1,0,0}\textbf{BU}%
 &|[fill=green]|\color[rgb]{0,0,0}\textbf{CO}%
 &|[fill=green]|\color[rgb]{0,0,0}\textbf{CO}%
 &|[fill=green]|\color[rgb]{0,0,0}\textbf{CO}%
 &|[fill=green]|\color[rgb]{0,0,0}\textbf{CO}%
 &|[fill=green]|\color[rgb]{0,0,0}\textbf{CO}%
 &|[fill=green]|\color[rgb]{1,0,0}\textbf{BU}%
\\
};
\end{tikzpicture}\\
At time 5: Water spot (1|9):
\\
\begin{tikzpicture}
\tikzset{square matrix/.style={
matrix of nodes,
column sep=-\pgflinewidth, row sep=-\pgflinewidth,
nodes={draw,
minimum height=#1,
anchor=center,
text width=#1,
align=center,
inner sep=0pt
},
},
square matrix/.default=1.2cm
}
\matrix[square matrix=1.4em] {
|[fill=green]|\color[rgb]{0,0,0}\textbf{CO}%
 &|[fill=green]|\color[rgb]{0,0,0}\textbf{CO}%
 &|[fill=white]|\color[gray]{0.5}WA%
 &|[fill=green]|\color[gray]{0.75} FO%
 &|[fill=green]|\color[gray]{0.75} FO%
 &|[fill=green]|\color[gray]{0.75} FO%
 &|[fill=green]|\color[gray]{0.75} FO%
 &|[fill=green]|\color[gray]{0.75} FO%
 &|[fill=white]|\color[gray]{0.5}WA%
 &|[fill=green]|\color[gray]{0.75} FO%
\\
|[fill=green]|\color[rgb]{0,0,0}\textbf{CO}%
 &|[fill=white]|\color[gray]{0.5}WA%
 &|[fill=white]|\color[gray]{0.5}WA%
 &|[fill=green]|\color[gray]{0.75} FO%
 &|[fill=green]|\color[gray]{0.75} FO%
 &|[fill=green]|\color[gray]{0.75} FO%
 &|[fill=green]|\color[gray]{0.75} FO%
 &|[fill=green]|\color[gray]{0.75} FO%
 &|[fill=green]|\color[gray]{0.75} FO%
 &|[fill=white]|\color[gray]{0.5}WA%
\\
|[fill=cyan]|\color[rgb]{1,0,0}\textbf{02}%
 &|[fill=green]|\color[gray]{0.75} FO%
 &|[fill=green]|\color[gray]{0.75} FO%
 &|[fill=green]|\color[gray]{0.75} FO%
 &|[fill=green]|\color[gray]{0.75} FO%
 &|[fill=green]|\color[gray]{0.75} FO%
 &|[fill=green]|\color[gray]{0.75} FO%
 &|[fill=green]|\color[gray]{0.75} FO%
 &|[fill=green]|\color[gray]{0.75} FO%
 &|[fill=green]|\color[gray]{0.75} FO%
\\
|[fill=green]|\color[gray]{0.75} FO%
 &|[fill=green]|\color[gray]{0.75} FO%
 &|[fill=white]|\color[gray]{0.5}WA%
 &|[fill=white]|\color[gray]{0.5}WA%
 &|[fill=white]|\color[gray]{0.5}WA%
 &|[fill=cyan]|\color[rgb]{1,0,0}\textbf{01}%
 &|[fill=white]|\color[gray]{0.5}WA%
 &|[fill=white]|\color[gray]{0.5}WA%
 &|[fill=white]|\color[gray]{0.5}WA%
 &|[fill=green]|\color[gray]{0.75} FO%
\\
|[fill=green]|\color[gray]{0.75} FO%
 &|[fill=green]|\color[gray]{0.75} FO%
 &|[fill=cyan]|\color[rgb]{1,0,0}\textbf{03}%
 &|[fill=green]|\color[rgb]{0,0,0}\textbf{CO}%
 &|[fill=green]|\color[rgb]{0,0,0}\textbf{CO}%
 &|[fill=green]|\color[rgb]{0,0,0}\textbf{CO}%
 &|[fill=green]|\color[rgb]{0,0,0}\textbf{CO}%
 &|[fill=green]|\color[rgb]{0,0,0}\textbf{CO}%
 &|[fill=green]|\color[rgb]{0,0,0}\textbf{CO}%
 &|[fill=cyan]|\color[rgb]{1,0,0}\textbf{04}%
\\
|[fill=green]|\color[gray]{0.75} FO%
 &|[fill=green]|\color[gray]{0.75} FO%
 &|[fill=white]|\color[gray]{0.5}WA%
 &|[fill=white]|\color[gray]{0.5}WA%
 &|[fill=green]|\color[rgb]{0,0,0}\textbf{CO}%
 &|[fill=green]|\color[rgb]{0,0,0}\textbf{CO}%
 &|[fill=green]|\color[rgb]{0,0,0}\textbf{CO}%
 &|[fill=green]|\color[rgb]{0,0,0}\textbf{CO}%
 &|[fill=green]|\color[rgb]{0,0,0}\textbf{CO}%
 &|[fill=green]|\color[rgb]{1,0,0}\textbf{BU}%
\\
|[fill=green]|\color[gray]{0.75} FO%
 &|[fill=green]|\color[gray]{0.75} FO%
 &|[fill=green]|\color[gray]{0.75} FO%
 &|[fill=green]|\color[gray]{0.75} FO%
 &|[fill=white]|\color[gray]{0.5}WA%
 &|[fill=green]|\color[rgb]{0,0,0}\textbf{CO}%
 &|[fill=green]|\color[rgb]{0,0,0}\textbf{CO}%
 &|[fill=white]|\color[gray]{0.5}WA%
 &|[fill=green]|\color[rgb]{1,0,0}\textbf{BU}%
 &|[fill=green]|\color[gray]{0.75} FO%
\\
|[fill=white]|\color[gray]{0.5}WA%
 &|[fill=green]|\color[gray]{0.75} FO%
 &|[fill=green]|\color[gray]{0.75} FO%
 &|[fill=green]|\color[rgb]{1,0,0}\textbf{BU}%
 &|[fill=white]|\color[gray]{0.5}WA%
 &|[fill=green]|\color[rgb]{0,0,0}\textbf{CO}%
 &|[fill=green]|\color[rgb]{0,0,0}\textbf{CO}%
 &|[fill=white]|\color[gray]{0.5}WA%
 &|[fill=green]|\color[rgb]{0,0,0}\textbf{CO}%
 &|[fill=white]|\color[gray]{0.5}WA%
\\
|[fill=green]|\color[gray]{0.75} FO%
 &|[fill=white]|\color[gray]{0.5}WA%
 &|[fill=green]|\color[rgb]{1,0,0}\textbf{BU}%
 &|[fill=green]|\color[rgb]{0,0,0}\textbf{CO}%
 &|[fill=white]|\color[gray]{0.5}WA%
 &|[fill=green]|\color[rgb]{0,0,0}\textbf{CO}%
 &|[fill=green]|\color[rgb]{0,0,0}\textbf{CO}%
 &|[fill=white]|\color[gray]{0.5}WA%
 &|[fill=green]|\color[rgb]{0,0,0}\textbf{CO}%
 &|[fill=green]|\color[rgb]{0,0,0}\textbf{CO}%
\\
|[fill=green]|\color[gray]{0.75} FO%
 &|[fill=cyan]|\color[rgb]{1,0,0}\textbf{05}%
 &|[fill=green]|\color[rgb]{0,0,0}\textbf{CO}%
 &|[fill=green]|\color[rgb]{0,0,0}\textbf{CO}%
 &|[fill=green]|\color[rgb]{0,0,0}\textbf{CO}%
 &|[fill=green]|\color[rgb]{0,0,0}\textbf{CO}%
 &|[fill=green]|\color[rgb]{0,0,0}\textbf{CO}%
 &|[fill=green]|\color[rgb]{0,0,0}\textbf{CO}%
 &|[fill=green]|\color[rgb]{0,0,0}\textbf{CO}%
 &|[fill=green]|\color[rgb]{0,0,0}\textbf{CO}%
\\
|[fill=green]|\color[gray]{0.75} FO%
 &|[fill=green]|\color[gray]{0.75} FO%
 &|[fill=green]|\color[rgb]{1,0,0}\textbf{BU}%
 &|[fill=green]|\color[rgb]{0,0,0}\textbf{CO}%
 &|[fill=green]|\color[rgb]{0,0,0}\textbf{CO}%
 &|[fill=green]|\color[rgb]{0,0,0}\textbf{CO}%
 &|[fill=green]|\color[rgb]{0,0,0}\textbf{CO}%
 &|[fill=green]|\color[rgb]{0,0,0}\textbf{CO}%
 &|[fill=green]|\color[rgb]{0,0,0}\textbf{CO}%
 &|[fill=green]|\color[rgb]{0,0,0}\textbf{CO}%
\\
};
\end{tikzpicture}\\
At time 6: Water spot (1|10):
\\
\begin{tikzpicture}
\tikzset{square matrix/.style={
matrix of nodes,
column sep=-\pgflinewidth, row sep=-\pgflinewidth,
nodes={draw,
minimum height=#1,
anchor=center,
text width=#1,
align=center,
inner sep=0pt
},
},
square matrix/.default=1.2cm
}
\matrix[square matrix=1.4em] {
|[fill=green]|\color[rgb]{0,0,0}\textbf{CO}%
 &|[fill=green]|\color[rgb]{0,0,0}\textbf{CO}%
 &|[fill=white]|\color[gray]{0.5}WA%
 &|[fill=green]|\color[gray]{0.75} FO%
 &|[fill=green]|\color[gray]{0.75} FO%
 &|[fill=green]|\color[gray]{0.75} FO%
 &|[fill=green]|\color[gray]{0.75} FO%
 &|[fill=green]|\color[gray]{0.75} FO%
 &|[fill=white]|\color[gray]{0.5}WA%
 &|[fill=green]|\color[gray]{0.75} FO%
\\
|[fill=green]|\color[rgb]{0,0,0}\textbf{CO}%
 &|[fill=white]|\color[gray]{0.5}WA%
 &|[fill=white]|\color[gray]{0.5}WA%
 &|[fill=green]|\color[gray]{0.75} FO%
 &|[fill=green]|\color[gray]{0.75} FO%
 &|[fill=green]|\color[gray]{0.75} FO%
 &|[fill=green]|\color[gray]{0.75} FO%
 &|[fill=green]|\color[gray]{0.75} FO%
 &|[fill=green]|\color[gray]{0.75} FO%
 &|[fill=white]|\color[gray]{0.5}WA%
\\
|[fill=cyan]|\color[rgb]{1,0,0}\textbf{02}%
 &|[fill=green]|\color[gray]{0.75} FO%
 &|[fill=green]|\color[gray]{0.75} FO%
 &|[fill=green]|\color[gray]{0.75} FO%
 &|[fill=green]|\color[gray]{0.75} FO%
 &|[fill=green]|\color[gray]{0.75} FO%
 &|[fill=green]|\color[gray]{0.75} FO%
 &|[fill=green]|\color[gray]{0.75} FO%
 &|[fill=green]|\color[gray]{0.75} FO%
 &|[fill=green]|\color[gray]{0.75} FO%
\\
|[fill=green]|\color[gray]{0.75} FO%
 &|[fill=green]|\color[gray]{0.75} FO%
 &|[fill=white]|\color[gray]{0.5}WA%
 &|[fill=white]|\color[gray]{0.5}WA%
 &|[fill=white]|\color[gray]{0.5}WA%
 &|[fill=cyan]|\color[rgb]{1,0,0}\textbf{01}%
 &|[fill=white]|\color[gray]{0.5}WA%
 &|[fill=white]|\color[gray]{0.5}WA%
 &|[fill=white]|\color[gray]{0.5}WA%
 &|[fill=green]|\color[gray]{0.75} FO%
\\
|[fill=green]|\color[gray]{0.75} FO%
 &|[fill=green]|\color[gray]{0.75} FO%
 &|[fill=cyan]|\color[rgb]{1,0,0}\textbf{03}%
 &|[fill=green]|\color[rgb]{0,0,0}\textbf{CO}%
 &|[fill=green]|\color[rgb]{0,0,0}\textbf{CO}%
 &|[fill=green]|\color[rgb]{0,0,0}\textbf{CO}%
 &|[fill=green]|\color[rgb]{0,0,0}\textbf{CO}%
 &|[fill=green]|\color[rgb]{0,0,0}\textbf{CO}%
 &|[fill=green]|\color[rgb]{0,0,0}\textbf{CO}%
 &|[fill=cyan]|\color[rgb]{1,0,0}\textbf{04}%
\\
|[fill=green]|\color[gray]{0.75} FO%
 &|[fill=green]|\color[gray]{0.75} FO%
 &|[fill=white]|\color[gray]{0.5}WA%
 &|[fill=white]|\color[gray]{0.5}WA%
 &|[fill=green]|\color[rgb]{0,0,0}\textbf{CO}%
 &|[fill=green]|\color[rgb]{0,0,0}\textbf{CO}%
 &|[fill=green]|\color[rgb]{0,0,0}\textbf{CO}%
 &|[fill=green]|\color[rgb]{0,0,0}\textbf{CO}%
 &|[fill=green]|\color[rgb]{0,0,0}\textbf{CO}%
 &|[fill=green]|\color[rgb]{0,0,0}\textbf{CO}%
\\
|[fill=green]|\color[gray]{0.75} FO%
 &|[fill=green]|\color[gray]{0.75} FO%
 &|[fill=green]|\color[gray]{0.75} FO%
 &|[fill=green]|\color[rgb]{1,0,0}\textbf{BU}%
 &|[fill=white]|\color[gray]{0.5}WA%
 &|[fill=green]|\color[rgb]{0,0,0}\textbf{CO}%
 &|[fill=green]|\color[rgb]{0,0,0}\textbf{CO}%
 &|[fill=white]|\color[gray]{0.5}WA%
 &|[fill=green]|\color[rgb]{0,0,0}\textbf{CO}%
 &|[fill=green]|\color[rgb]{1,0,0}\textbf{BU}%
\\
|[fill=white]|\color[gray]{0.5}WA%
 &|[fill=green]|\color[gray]{0.75} FO%
 &|[fill=green]|\color[rgb]{1,0,0}\textbf{BU}%
 &|[fill=green]|\color[rgb]{0,0,0}\textbf{CO}%
 &|[fill=white]|\color[gray]{0.5}WA%
 &|[fill=green]|\color[rgb]{0,0,0}\textbf{CO}%
 &|[fill=green]|\color[rgb]{0,0,0}\textbf{CO}%
 &|[fill=white]|\color[gray]{0.5}WA%
 &|[fill=green]|\color[rgb]{0,0,0}\textbf{CO}%
 &|[fill=white]|\color[gray]{0.5}WA%
\\
|[fill=green]|\color[gray]{0.75} FO%
 &|[fill=white]|\color[gray]{0.5}WA%
 &|[fill=green]|\color[rgb]{0,0,0}\textbf{CO}%
 &|[fill=green]|\color[rgb]{0,0,0}\textbf{CO}%
 &|[fill=white]|\color[gray]{0.5}WA%
 &|[fill=green]|\color[rgb]{0,0,0}\textbf{CO}%
 &|[fill=green]|\color[rgb]{0,0,0}\textbf{CO}%
 &|[fill=white]|\color[gray]{0.5}WA%
 &|[fill=green]|\color[rgb]{0,0,0}\textbf{CO}%
 &|[fill=green]|\color[rgb]{0,0,0}\textbf{CO}%
\\
|[fill=green]|\color[gray]{0.75} FO%
 &|[fill=cyan]|\color[rgb]{1,0,0}\textbf{05}%
 &|[fill=green]|\color[rgb]{0,0,0}\textbf{CO}%
 &|[fill=green]|\color[rgb]{0,0,0}\textbf{CO}%
 &|[fill=green]|\color[rgb]{0,0,0}\textbf{CO}%
 &|[fill=green]|\color[rgb]{0,0,0}\textbf{CO}%
 &|[fill=green]|\color[rgb]{0,0,0}\textbf{CO}%
 &|[fill=green]|\color[rgb]{0,0,0}\textbf{CO}%
 &|[fill=green]|\color[rgb]{0,0,0}\textbf{CO}%
 &|[fill=green]|\color[rgb]{0,0,0}\textbf{CO}%
\\
|[fill=green]|\color[gray]{0.75} FO%
 &|[fill=cyan]|\color[rgb]{1,0,0}\textbf{06}%
 &|[fill=green]|\color[rgb]{0,0,0}\textbf{CO}%
 &|[fill=green]|\color[rgb]{0,0,0}\textbf{CO}%
 &|[fill=green]|\color[rgb]{0,0,0}\textbf{CO}%
 &|[fill=green]|\color[rgb]{0,0,0}\textbf{CO}%
 &|[fill=green]|\color[rgb]{0,0,0}\textbf{CO}%
 &|[fill=green]|\color[rgb]{0,0,0}\textbf{CO}%
 &|[fill=green]|\color[rgb]{0,0,0}\textbf{CO}%
 &|[fill=green]|\color[rgb]{0,0,0}\textbf{CO}%
\\
};
\end{tikzpicture}\\
At time 7: Water spot (1|7):
\\
\begin{tikzpicture}
\tikzset{square matrix/.style={
matrix of nodes,
column sep=-\pgflinewidth, row sep=-\pgflinewidth,
nodes={draw,
minimum height=#1,
anchor=center,
text width=#1,
align=center,
inner sep=0pt
},
},
square matrix/.default=1.2cm
}
\matrix[square matrix=1.4em] {
|[fill=green]|\color[rgb]{0,0,0}\textbf{CO}%
 &|[fill=green]|\color[rgb]{0,0,0}\textbf{CO}%
 &|[fill=white]|\color[gray]{0.5}WA%
 &|[fill=green]|\color[gray]{0.75} FO%
 &|[fill=green]|\color[gray]{0.75} FO%
 &|[fill=green]|\color[gray]{0.75} FO%
 &|[fill=green]|\color[gray]{0.75} FO%
 &|[fill=green]|\color[gray]{0.75} FO%
 &|[fill=white]|\color[gray]{0.5}WA%
 &|[fill=green]|\color[gray]{0.75} FO%
\\
|[fill=green]|\color[rgb]{0,0,0}\textbf{CO}%
 &|[fill=white]|\color[gray]{0.5}WA%
 &|[fill=white]|\color[gray]{0.5}WA%
 &|[fill=green]|\color[gray]{0.75} FO%
 &|[fill=green]|\color[gray]{0.75} FO%
 &|[fill=green]|\color[gray]{0.75} FO%
 &|[fill=green]|\color[gray]{0.75} FO%
 &|[fill=green]|\color[gray]{0.75} FO%
 &|[fill=green]|\color[gray]{0.75} FO%
 &|[fill=white]|\color[gray]{0.5}WA%
\\
|[fill=cyan]|\color[rgb]{1,0,0}\textbf{02}%
 &|[fill=green]|\color[gray]{0.75} FO%
 &|[fill=green]|\color[gray]{0.75} FO%
 &|[fill=green]|\color[gray]{0.75} FO%
 &|[fill=green]|\color[gray]{0.75} FO%
 &|[fill=green]|\color[gray]{0.75} FO%
 &|[fill=green]|\color[gray]{0.75} FO%
 &|[fill=green]|\color[gray]{0.75} FO%
 &|[fill=green]|\color[gray]{0.75} FO%
 &|[fill=green]|\color[gray]{0.75} FO%
\\
|[fill=green]|\color[gray]{0.75} FO%
 &|[fill=green]|\color[gray]{0.75} FO%
 &|[fill=white]|\color[gray]{0.5}WA%
 &|[fill=white]|\color[gray]{0.5}WA%
 &|[fill=white]|\color[gray]{0.5}WA%
 &|[fill=cyan]|\color[rgb]{1,0,0}\textbf{01}%
 &|[fill=white]|\color[gray]{0.5}WA%
 &|[fill=white]|\color[gray]{0.5}WA%
 &|[fill=white]|\color[gray]{0.5}WA%
 &|[fill=green]|\color[gray]{0.75} FO%
\\
|[fill=green]|\color[gray]{0.75} FO%
 &|[fill=green]|\color[gray]{0.75} FO%
 &|[fill=cyan]|\color[rgb]{1,0,0}\textbf{03}%
 &|[fill=green]|\color[rgb]{0,0,0}\textbf{CO}%
 &|[fill=green]|\color[rgb]{0,0,0}\textbf{CO}%
 &|[fill=green]|\color[rgb]{0,0,0}\textbf{CO}%
 &|[fill=green]|\color[rgb]{0,0,0}\textbf{CO}%
 &|[fill=green]|\color[rgb]{0,0,0}\textbf{CO}%
 &|[fill=green]|\color[rgb]{0,0,0}\textbf{CO}%
 &|[fill=cyan]|\color[rgb]{1,0,0}\textbf{04}%
\\
|[fill=green]|\color[gray]{0.75} FO%
 &|[fill=green]|\color[gray]{0.75} FO%
 &|[fill=white]|\color[gray]{0.5}WA%
 &|[fill=white]|\color[gray]{0.5}WA%
 &|[fill=green]|\color[rgb]{0,0,0}\textbf{CO}%
 &|[fill=green]|\color[rgb]{0,0,0}\textbf{CO}%
 &|[fill=green]|\color[rgb]{0,0,0}\textbf{CO}%
 &|[fill=green]|\color[rgb]{0,0,0}\textbf{CO}%
 &|[fill=green]|\color[rgb]{0,0,0}\textbf{CO}%
 &|[fill=green]|\color[rgb]{0,0,0}\textbf{CO}%
\\
|[fill=green]|\color[gray]{0.75} FO%
 &|[fill=green]|\color[gray]{0.75} FO%
 &|[fill=green]|\color[rgb]{1,0,0}\textbf{BU}%
 &|[fill=green]|\color[rgb]{0,0,0}\textbf{CO}%
 &|[fill=white]|\color[gray]{0.5}WA%
 &|[fill=green]|\color[rgb]{0,0,0}\textbf{CO}%
 &|[fill=green]|\color[rgb]{0,0,0}\textbf{CO}%
 &|[fill=white]|\color[gray]{0.5}WA%
 &|[fill=green]|\color[rgb]{0,0,0}\textbf{CO}%
 &|[fill=green]|\color[rgb]{0,0,0}\textbf{CO}%
\\
|[fill=white]|\color[gray]{0.5}WA%
 &|[fill=cyan]|\color[rgb]{1,0,0}\textbf{07}%
 &|[fill=green]|\color[rgb]{0,0,0}\textbf{CO}%
 &|[fill=green]|\color[rgb]{0,0,0}\textbf{CO}%
 &|[fill=white]|\color[gray]{0.5}WA%
 &|[fill=green]|\color[rgb]{0,0,0}\textbf{CO}%
 &|[fill=green]|\color[rgb]{0,0,0}\textbf{CO}%
 &|[fill=white]|\color[gray]{0.5}WA%
 &|[fill=green]|\color[rgb]{0,0,0}\textbf{CO}%
 &|[fill=white]|\color[gray]{0.5}WA%
\\
|[fill=green]|\color[gray]{0.75} FO%
 &|[fill=white]|\color[gray]{0.5}WA%
 &|[fill=green]|\color[rgb]{0,0,0}\textbf{CO}%
 &|[fill=green]|\color[rgb]{0,0,0}\textbf{CO}%
 &|[fill=white]|\color[gray]{0.5}WA%
 &|[fill=green]|\color[rgb]{0,0,0}\textbf{CO}%
 &|[fill=green]|\color[rgb]{0,0,0}\textbf{CO}%
 &|[fill=white]|\color[gray]{0.5}WA%
 &|[fill=green]|\color[rgb]{0,0,0}\textbf{CO}%
 &|[fill=green]|\color[rgb]{0,0,0}\textbf{CO}%
\\
|[fill=green]|\color[gray]{0.75} FO%
 &|[fill=cyan]|\color[rgb]{1,0,0}\textbf{05}%
 &|[fill=green]|\color[rgb]{0,0,0}\textbf{CO}%
 &|[fill=green]|\color[rgb]{0,0,0}\textbf{CO}%
 &|[fill=green]|\color[rgb]{0,0,0}\textbf{CO}%
 &|[fill=green]|\color[rgb]{0,0,0}\textbf{CO}%
 &|[fill=green]|\color[rgb]{0,0,0}\textbf{CO}%
 &|[fill=green]|\color[rgb]{0,0,0}\textbf{CO}%
 &|[fill=green]|\color[rgb]{0,0,0}\textbf{CO}%
 &|[fill=green]|\color[rgb]{0,0,0}\textbf{CO}%
\\
|[fill=green]|\color[gray]{0.75} FO%
 &|[fill=cyan]|\color[rgb]{1,0,0}\textbf{06}%
 &|[fill=green]|\color[rgb]{0,0,0}\textbf{CO}%
 &|[fill=green]|\color[rgb]{0,0,0}\textbf{CO}%
 &|[fill=green]|\color[rgb]{0,0,0}\textbf{CO}%
 &|[fill=green]|\color[rgb]{0,0,0}\textbf{CO}%
 &|[fill=green]|\color[rgb]{0,0,0}\textbf{CO}%
 &|[fill=green]|\color[rgb]{0,0,0}\textbf{CO}%
 &|[fill=green]|\color[rgb]{0,0,0}\textbf{CO}%
 &|[fill=green]|\color[rgb]{0,0,0}\textbf{CO}%
\\
};
\end{tikzpicture}\\
At time 8: Water spot (1|6):
\\
\begin{tikzpicture}
\tikzset{square matrix/.style={
matrix of nodes,
column sep=-\pgflinewidth, row sep=-\pgflinewidth,
nodes={draw,
minimum height=#1,
anchor=center,
text width=#1,
align=center,
inner sep=0pt
},
},
square matrix/.default=1.2cm
}
\matrix[square matrix=1.4em] {
|[fill=green]|\color[rgb]{0,0,0}\textbf{CO}%
 &|[fill=green]|\color[rgb]{0,0,0}\textbf{CO}%
 &|[fill=white]|\color[gray]{0.5}WA%
 &|[fill=green]|\color[gray]{0.75} FO%
 &|[fill=green]|\color[gray]{0.75} FO%
 &|[fill=green]|\color[gray]{0.75} FO%
 &|[fill=green]|\color[gray]{0.75} FO%
 &|[fill=green]|\color[gray]{0.75} FO%
 &|[fill=white]|\color[gray]{0.5}WA%
 &|[fill=green]|\color[gray]{0.75} FO%
\\
|[fill=green]|\color[rgb]{0,0,0}\textbf{CO}%
 &|[fill=white]|\color[gray]{0.5}WA%
 &|[fill=white]|\color[gray]{0.5}WA%
 &|[fill=green]|\color[gray]{0.75} FO%
 &|[fill=green]|\color[gray]{0.75} FO%
 &|[fill=green]|\color[gray]{0.75} FO%
 &|[fill=green]|\color[gray]{0.75} FO%
 &|[fill=green]|\color[gray]{0.75} FO%
 &|[fill=green]|\color[gray]{0.75} FO%
 &|[fill=white]|\color[gray]{0.5}WA%
\\
|[fill=cyan]|\color[rgb]{1,0,0}\textbf{02}%
 &|[fill=green]|\color[gray]{0.75} FO%
 &|[fill=green]|\color[gray]{0.75} FO%
 &|[fill=green]|\color[gray]{0.75} FO%
 &|[fill=green]|\color[gray]{0.75} FO%
 &|[fill=green]|\color[gray]{0.75} FO%
 &|[fill=green]|\color[gray]{0.75} FO%
 &|[fill=green]|\color[gray]{0.75} FO%
 &|[fill=green]|\color[gray]{0.75} FO%
 &|[fill=green]|\color[gray]{0.75} FO%
\\
|[fill=green]|\color[gray]{0.75} FO%
 &|[fill=green]|\color[gray]{0.75} FO%
 &|[fill=white]|\color[gray]{0.5}WA%
 &|[fill=white]|\color[gray]{0.5}WA%
 &|[fill=white]|\color[gray]{0.5}WA%
 &|[fill=cyan]|\color[rgb]{1,0,0}\textbf{01}%
 &|[fill=white]|\color[gray]{0.5}WA%
 &|[fill=white]|\color[gray]{0.5}WA%
 &|[fill=white]|\color[gray]{0.5}WA%
 &|[fill=green]|\color[gray]{0.75} FO%
\\
|[fill=green]|\color[gray]{0.75} FO%
 &|[fill=green]|\color[gray]{0.75} FO%
 &|[fill=cyan]|\color[rgb]{1,0,0}\textbf{03}%
 &|[fill=green]|\color[rgb]{0,0,0}\textbf{CO}%
 &|[fill=green]|\color[rgb]{0,0,0}\textbf{CO}%
 &|[fill=green]|\color[rgb]{0,0,0}\textbf{CO}%
 &|[fill=green]|\color[rgb]{0,0,0}\textbf{CO}%
 &|[fill=green]|\color[rgb]{0,0,0}\textbf{CO}%
 &|[fill=green]|\color[rgb]{0,0,0}\textbf{CO}%
 &|[fill=cyan]|\color[rgb]{1,0,0}\textbf{04}%
\\
|[fill=green]|\color[gray]{0.75} FO%
 &|[fill=green]|\color[gray]{0.75} FO%
 &|[fill=white]|\color[gray]{0.5}WA%
 &|[fill=white]|\color[gray]{0.5}WA%
 &|[fill=green]|\color[rgb]{0,0,0}\textbf{CO}%
 &|[fill=green]|\color[rgb]{0,0,0}\textbf{CO}%
 &|[fill=green]|\color[rgb]{0,0,0}\textbf{CO}%
 &|[fill=green]|\color[rgb]{0,0,0}\textbf{CO}%
 &|[fill=green]|\color[rgb]{0,0,0}\textbf{CO}%
 &|[fill=green]|\color[rgb]{0,0,0}\textbf{CO}%
\\
|[fill=green]|\color[gray]{0.75} FO%
 &|[fill=cyan]|\color[rgb]{1,0,0}\textbf{08}%
 &|[fill=green]|\color[rgb]{0,0,0}\textbf{CO}%
 &|[fill=green]|\color[rgb]{0,0,0}\textbf{CO}%
 &|[fill=white]|\color[gray]{0.5}WA%
 &|[fill=green]|\color[rgb]{0,0,0}\textbf{CO}%
 &|[fill=green]|\color[rgb]{0,0,0}\textbf{CO}%
 &|[fill=white]|\color[gray]{0.5}WA%
 &|[fill=green]|\color[rgb]{0,0,0}\textbf{CO}%
 &|[fill=green]|\color[rgb]{0,0,0}\textbf{CO}%
\\
|[fill=white]|\color[gray]{0.5}WA%
 &|[fill=cyan]|\color[rgb]{1,0,0}\textbf{07}%
 &|[fill=green]|\color[rgb]{0,0,0}\textbf{CO}%
 &|[fill=green]|\color[rgb]{0,0,0}\textbf{CO}%
 &|[fill=white]|\color[gray]{0.5}WA%
 &|[fill=green]|\color[rgb]{0,0,0}\textbf{CO}%
 &|[fill=green]|\color[rgb]{0,0,0}\textbf{CO}%
 &|[fill=white]|\color[gray]{0.5}WA%
 &|[fill=green]|\color[rgb]{0,0,0}\textbf{CO}%
 &|[fill=white]|\color[gray]{0.5}WA%
\\
|[fill=green]|\color[gray]{0.75} FO%
 &|[fill=white]|\color[gray]{0.5}WA%
 &|[fill=green]|\color[rgb]{0,0,0}\textbf{CO}%
 &|[fill=green]|\color[rgb]{0,0,0}\textbf{CO}%
 &|[fill=white]|\color[gray]{0.5}WA%
 &|[fill=green]|\color[rgb]{0,0,0}\textbf{CO}%
 &|[fill=green]|\color[rgb]{0,0,0}\textbf{CO}%
 &|[fill=white]|\color[gray]{0.5}WA%
 &|[fill=green]|\color[rgb]{0,0,0}\textbf{CO}%
 &|[fill=green]|\color[rgb]{0,0,0}\textbf{CO}%
\\
|[fill=green]|\color[gray]{0.75} FO%
 &|[fill=cyan]|\color[rgb]{1,0,0}\textbf{05}%
 &|[fill=green]|\color[rgb]{0,0,0}\textbf{CO}%
 &|[fill=green]|\color[rgb]{0,0,0}\textbf{CO}%
 &|[fill=green]|\color[rgb]{0,0,0}\textbf{CO}%
 &|[fill=green]|\color[rgb]{0,0,0}\textbf{CO}%
 &|[fill=green]|\color[rgb]{0,0,0}\textbf{CO}%
 &|[fill=green]|\color[rgb]{0,0,0}\textbf{CO}%
 &|[fill=green]|\color[rgb]{0,0,0}\textbf{CO}%
 &|[fill=green]|\color[rgb]{0,0,0}\textbf{CO}%
\\
|[fill=green]|\color[gray]{0.75} FO%
 &|[fill=cyan]|\color[rgb]{1,0,0}\textbf{06}%
 &|[fill=green]|\color[rgb]{0,0,0}\textbf{CO}%
 &|[fill=green]|\color[rgb]{0,0,0}\textbf{CO}%
 &|[fill=green]|\color[rgb]{0,0,0}\textbf{CO}%
 &|[fill=green]|\color[rgb]{0,0,0}\textbf{CO}%
 &|[fill=green]|\color[rgb]{0,0,0}\textbf{CO}%
 &|[fill=green]|\color[rgb]{0,0,0}\textbf{CO}%
 &|[fill=green]|\color[rgb]{0,0,0}\textbf{CO}%
 &|[fill=green]|\color[rgb]{0,0,0}\textbf{CO}%
\\
};
\end{tikzpicture}\\
And you'll find 48 pieces of coal and 8 pieces of watered coal:
\\
\begin{tikzpicture}
\tikzset{square matrix/.style={
matrix of nodes,
column sep=-\pgflinewidth, row sep=-\pgflinewidth,
nodes={draw,
minimum height=#1,
anchor=center,
text width=#1,
align=center,
inner sep=0pt
},
},
square matrix/.default=1.2cm
}
\matrix[square matrix=1.4em] {
|[fill=green]|\color[rgb]{0,0,0}\textbf{CO}%
 &|[fill=green]|\color[rgb]{0,0,0}\textbf{CO}%
 &|[fill=white]|\color[gray]{0.5}WA%
 &|[fill=green]|\color[gray]{0.75} FO%
 &|[fill=green]|\color[gray]{0.75} FO%
 &|[fill=green]|\color[gray]{0.75} FO%
 &|[fill=green]|\color[gray]{0.75} FO%
 &|[fill=green]|\color[gray]{0.75} FO%
 &|[fill=white]|\color[gray]{0.5}WA%
 &|[fill=green]|\color[gray]{0.75} FO%
\\
|[fill=green]|\color[rgb]{0,0,0}\textbf{CO}%
 &|[fill=white]|\color[gray]{0.5}WA%
 &|[fill=white]|\color[gray]{0.5}WA%
 &|[fill=green]|\color[gray]{0.75} FO%
 &|[fill=green]|\color[gray]{0.75} FO%
 &|[fill=green]|\color[gray]{0.75} FO%
 &|[fill=green]|\color[gray]{0.75} FO%
 &|[fill=green]|\color[gray]{0.75} FO%
 &|[fill=green]|\color[gray]{0.75} FO%
 &|[fill=white]|\color[gray]{0.5}WA%
\\
|[fill=cyan]|\color[rgb]{1,0,0}\textbf{02}%
 &|[fill=green]|\color[gray]{0.75} FO%
 &|[fill=green]|\color[gray]{0.75} FO%
 &|[fill=green]|\color[gray]{0.75} FO%
 &|[fill=green]|\color[gray]{0.75} FO%
 &|[fill=green]|\color[gray]{0.75} FO%
 &|[fill=green]|\color[gray]{0.75} FO%
 &|[fill=green]|\color[gray]{0.75} FO%
 &|[fill=green]|\color[gray]{0.75} FO%
 &|[fill=green]|\color[gray]{0.75} FO%
\\
|[fill=green]|\color[gray]{0.75} FO%
 &|[fill=green]|\color[gray]{0.75} FO%
 &|[fill=white]|\color[gray]{0.5}WA%
 &|[fill=white]|\color[gray]{0.5}WA%
 &|[fill=white]|\color[gray]{0.5}WA%
 &|[fill=cyan]|\color[rgb]{1,0,0}\textbf{01}%
 &|[fill=white]|\color[gray]{0.5}WA%
 &|[fill=white]|\color[gray]{0.5}WA%
 &|[fill=white]|\color[gray]{0.5}WA%
 &|[fill=green]|\color[gray]{0.75} FO%
\\
|[fill=green]|\color[gray]{0.75} FO%
 &|[fill=green]|\color[gray]{0.75} FO%
 &|[fill=cyan]|\color[rgb]{1,0,0}\textbf{03}%
 &|[fill=green]|\color[rgb]{0,0,0}\textbf{CO}%
 &|[fill=green]|\color[rgb]{0,0,0}\textbf{CO}%
 &|[fill=green]|\color[rgb]{0,0,0}\textbf{CO}%
 &|[fill=green]|\color[rgb]{0,0,0}\textbf{CO}%
 &|[fill=green]|\color[rgb]{0,0,0}\textbf{CO}%
 &|[fill=green]|\color[rgb]{0,0,0}\textbf{CO}%
 &|[fill=cyan]|\color[rgb]{1,0,0}\textbf{04}%
\\
|[fill=green]|\color[gray]{0.75} FO%
 &|[fill=green]|\color[gray]{0.75} FO%
 &|[fill=white]|\color[gray]{0.5}WA%
 &|[fill=white]|\color[gray]{0.5}WA%
 &|[fill=green]|\color[rgb]{0,0,0}\textbf{CO}%
 &|[fill=green]|\color[rgb]{0,0,0}\textbf{CO}%
 &|[fill=green]|\color[rgb]{0,0,0}\textbf{CO}%
 &|[fill=green]|\color[rgb]{0,0,0}\textbf{CO}%
 &|[fill=green]|\color[rgb]{0,0,0}\textbf{CO}%
 &|[fill=green]|\color[rgb]{0,0,0}\textbf{CO}%
\\
|[fill=green]|\color[gray]{0.75} FO%
 &|[fill=cyan]|\color[rgb]{1,0,0}\textbf{08}%
 &|[fill=green]|\color[rgb]{0,0,0}\textbf{CO}%
 &|[fill=green]|\color[rgb]{0,0,0}\textbf{CO}%
 &|[fill=white]|\color[gray]{0.5}WA%
 &|[fill=green]|\color[rgb]{0,0,0}\textbf{CO}%
 &|[fill=green]|\color[rgb]{0,0,0}\textbf{CO}%
 &|[fill=white]|\color[gray]{0.5}WA%
 &|[fill=green]|\color[rgb]{0,0,0}\textbf{CO}%
 &|[fill=green]|\color[rgb]{0,0,0}\textbf{CO}%
\\
|[fill=white]|\color[gray]{0.5}WA%
 &|[fill=cyan]|\color[rgb]{1,0,0}\textbf{07}%
 &|[fill=green]|\color[rgb]{0,0,0}\textbf{CO}%
 &|[fill=green]|\color[rgb]{0,0,0}\textbf{CO}%
 &|[fill=white]|\color[gray]{0.5}WA%
 &|[fill=green]|\color[rgb]{0,0,0}\textbf{CO}%
 &|[fill=green]|\color[rgb]{0,0,0}\textbf{CO}%
 &|[fill=white]|\color[gray]{0.5}WA%
 &|[fill=green]|\color[rgb]{0,0,0}\textbf{CO}%
 &|[fill=white]|\color[gray]{0.5}WA%
\\
|[fill=green]|\color[gray]{0.75} FO%
 &|[fill=white]|\color[gray]{0.5}WA%
 &|[fill=green]|\color[rgb]{0,0,0}\textbf{CO}%
 &|[fill=green]|\color[rgb]{0,0,0}\textbf{CO}%
 &|[fill=white]|\color[gray]{0.5}WA%
 &|[fill=green]|\color[rgb]{0,0,0}\textbf{CO}%
 &|[fill=green]|\color[rgb]{0,0,0}\textbf{CO}%
 &|[fill=white]|\color[gray]{0.5}WA%
 &|[fill=green]|\color[rgb]{0,0,0}\textbf{CO}%
 &|[fill=green]|\color[rgb]{0,0,0}\textbf{CO}%
\\
|[fill=green]|\color[gray]{0.75} FO%
 &|[fill=cyan]|\color[rgb]{1,0,0}\textbf{05}%
 &|[fill=green]|\color[rgb]{0,0,0}\textbf{CO}%
 &|[fill=green]|\color[rgb]{0,0,0}\textbf{CO}%
 &|[fill=green]|\color[rgb]{0,0,0}\textbf{CO}%
 &|[fill=green]|\color[rgb]{0,0,0}\textbf{CO}%
 &|[fill=green]|\color[rgb]{0,0,0}\textbf{CO}%
 &|[fill=green]|\color[rgb]{0,0,0}\textbf{CO}%
 &|[fill=green]|\color[rgb]{0,0,0}\textbf{CO}%
 &|[fill=green]|\color[rgb]{0,0,0}\textbf{CO}%
\\
|[fill=green]|\color[gray]{0.75} FO%
 &|[fill=cyan]|\color[rgb]{1,0,0}\textbf{06}%
 &|[fill=green]|\color[rgb]{0,0,0}\textbf{CO}%
 &|[fill=green]|\color[rgb]{0,0,0}\textbf{CO}%
 &|[fill=green]|\color[rgb]{0,0,0}\textbf{CO}%
 &|[fill=green]|\color[rgb]{0,0,0}\textbf{CO}%
 &|[fill=green]|\color[rgb]{0,0,0}\textbf{CO}%
 &|[fill=green]|\color[rgb]{0,0,0}\textbf{CO}%
 &|[fill=green]|\color[rgb]{0,0,0}\textbf{CO}%
 &|[fill=green]|\color[rgb]{0,0,0}\textbf{CO}%
\\
};
\end{tikzpicture}\\
\\
Explanation:\\
\colorbox{white}{\color[gray]{0.5}WA}  ---  WALL\\
\colorbox{green}{\color[gray]{0.5}FO}  ---  FOREST\\
{\color[rgb]{1,0,0}\textbf{BU}}  ---  BURNED\\
{\color[rgb]{0,0,0}\textbf{CO}}  ---  COAL (doubly burned)\\
\colorbox{cyan}{\#\#}  ---  WATERED at time \#\#\\
Fields can have more than 1 state.
}
\subsubsection{Beispiel 2}
\footnote{Diese Eingabe finden Sie auch in der Datei \texttt{2.in}}:
{\small
\lstinputlisting{../Aufgabe_1/2.in}
}
Mein Programm produziert folgende Ausgabe\footnote{Diese Ausgabe finden Sie auch in der Datei \texttt{2.out.tex}; Eine Datei \texttt{2.out} mit den ASCII-Escape-Sequenzen exisitert ebenfalls.}:\\
{\ttfamily \small
\\
\begin{tikzpicture}
\tikzset{square matrix/.style={
matrix of nodes,
column sep=-\pgflinewidth, row sep=-\pgflinewidth,
nodes={draw,
minimum height=#1,
anchor=center,
text width=#1,
align=center,
inner sep=0pt
},
},
square matrix/.default=1.2cm
}
\matrix[square matrix=1.4em] {
|[fill=green]|\color[gray]{0.75} FO%
 &|[fill=green]|\color[gray]{0.75} FO%
 &|[fill=green]|\color[gray]{0.75} FO%
 &|[fill=green]|\color[gray]{0.75} FO%
 &|[fill=green]|\color[gray]{0.75} FO%
 &|[fill=green]|\color[gray]{0.75} FO%
 &|[fill=green]|\color[gray]{0.75} FO%
 &|[fill=green]|\color[gray]{0.75} FO%
 &|[fill=green]|\color[gray]{0.75} FO%
 &|[fill=green]|\color[gray]{0.75} FO%
 &|[fill=green]|\color[gray]{0.75} FO%
 &|[fill=green]|\color[gray]{0.75} FO%
 &|[fill=green]|\color[gray]{0.75} FO%
\\
|[fill=green]|\color[gray]{0.75} FO%
 &|[fill=white]|\color[gray]{0.5}WA%
 &|[fill=white]|\color[gray]{0.5}WA%
 &|[fill=white]|\color[gray]{0.5}WA%
 &|[fill=white]|\color[gray]{0.5}WA%
 &|[fill=white]|\color[gray]{0.5}WA%
 &|[fill=green]|\color[gray]{0.75} FO%
 &|[fill=white]|\color[gray]{0.5}WA%
 &|[fill=white]|\color[gray]{0.5}WA%
 &|[fill=white]|\color[gray]{0.5}WA%
 &|[fill=white]|\color[gray]{0.5}WA%
 &|[fill=white]|\color[gray]{0.5}WA%
 &|[fill=green]|\color[gray]{0.75} FO%
\\
|[fill=green]|\color[gray]{0.75} FO%
 &|[fill=white]|\color[gray]{0.5}WA%
 &|[fill=green]|\color[gray]{0.75} FO%
 &|[fill=green]|\color[gray]{0.75} FO%
 &|[fill=green]|\color[gray]{0.75} FO%
 &|[fill=green]|\color[gray]{0.75} FO%
 &|[fill=green]|\color[gray]{0.75} FO%
 &|[fill=green]|\color[gray]{0.75} FO%
 &|[fill=green]|\color[gray]{0.75} FO%
 &|[fill=green]|\color[gray]{0.75} FO%
 &|[fill=green]|\color[gray]{0.75} FO%
 &|[fill=white]|\color[gray]{0.5}WA%
 &|[fill=green]|\color[gray]{0.75} FO%
\\
|[fill=green]|\color[gray]{0.75} FO%
 &|[fill=white]|\color[gray]{0.5}WA%
 &|[fill=green]|\color[gray]{0.75} FO%
 &|[fill=green]|\color[gray]{0.75} FO%
 &|[fill=green]|\color[gray]{0.75} FO%
 &|[fill=green]|\color[gray]{0.75} FO%
 &|[fill=green]|\color[gray]{0.75} FO%
 &|[fill=green]|\color[gray]{0.75} FO%
 &|[fill=green]|\color[gray]{0.75} FO%
 &|[fill=green]|\color[gray]{0.75} FO%
 &|[fill=green]|\color[gray]{0.75} FO%
 &|[fill=white]|\color[gray]{0.5}WA%
 &|[fill=green]|\color[gray]{0.75} FO%
\\
|[fill=green]|\color[gray]{0.75} FO%
 &|[fill=white]|\color[gray]{0.5}WA%
 &|[fill=green]|\color[gray]{0.75} FO%
 &|[fill=green]|\color[gray]{0.75} FO%
 &|[fill=green]|\color[gray]{0.75} FO%
 &|[fill=green]|\color[gray]{0.75} FO%
 &|[fill=green]|\color[gray]{0.75} FO%
 &|[fill=green]|\color[gray]{0.75} FO%
 &|[fill=green]|\color[gray]{0.75} FO%
 &|[fill=green]|\color[gray]{0.75} FO%
 &|[fill=green]|\color[gray]{0.75} FO%
 &|[fill=white]|\color[gray]{0.5}WA%
 &|[fill=green]|\color[gray]{0.75} FO%
\\
|[fill=green]|\color[gray]{0.75} FO%
 &|[fill=white]|\color[gray]{0.5}WA%
 &|[fill=green]|\color[gray]{0.75} FO%
 &|[fill=green]|\color[gray]{0.75} FO%
 &|[fill=green]|\color[gray]{0.75} FO%
 &|[fill=green]|\color[gray]{0.75} FO%
 &|[fill=green]|\color[gray]{0.75} FO%
 &|[fill=green]|\color[gray]{0.75} FO%
 &|[fill=green]|\color[gray]{0.75} FO%
 &|[fill=green]|\color[gray]{0.75} FO%
 &|[fill=green]|\color[gray]{0.75} FO%
 &|[fill=white]|\color[gray]{0.5}WA%
 &|[fill=green]|\color[gray]{0.75} FO%
\\
|[fill=green]|\color[gray]{0.75} FO%
 &|[fill=green]|\color[gray]{0.75} FO%
 &|[fill=green]|\color[gray]{0.75} FO%
 &|[fill=green]|\color[gray]{0.75} FO%
 &|[fill=green]|\color[gray]{0.75} FO%
 &|[fill=green]|\color[gray]{0.75} FO%
 &|[fill=green]|\color[rgb]{1,0,0}\textbf{BU}%
 &|[fill=green]|\color[gray]{0.75} FO%
 &|[fill=green]|\color[gray]{0.75} FO%
 &|[fill=green]|\color[gray]{0.75} FO%
 &|[fill=green]|\color[gray]{0.75} FO%
 &|[fill=green]|\color[gray]{0.75} FO%
 &|[fill=green]|\color[gray]{0.75} FO%
\\
|[fill=green]|\color[gray]{0.75} FO%
 &|[fill=white]|\color[gray]{0.5}WA%
 &|[fill=green]|\color[gray]{0.75} FO%
 &|[fill=green]|\color[gray]{0.75} FO%
 &|[fill=green]|\color[gray]{0.75} FO%
 &|[fill=green]|\color[gray]{0.75} FO%
 &|[fill=green]|\color[gray]{0.75} FO%
 &|[fill=green]|\color[gray]{0.75} FO%
 &|[fill=green]|\color[gray]{0.75} FO%
 &|[fill=green]|\color[gray]{0.75} FO%
 &|[fill=green]|\color[gray]{0.75} FO%
 &|[fill=white]|\color[gray]{0.5}WA%
 &|[fill=green]|\color[gray]{0.75} FO%
\\
|[fill=green]|\color[gray]{0.75} FO%
 &|[fill=white]|\color[gray]{0.5}WA%
 &|[fill=green]|\color[gray]{0.75} FO%
 &|[fill=green]|\color[gray]{0.75} FO%
 &|[fill=green]|\color[gray]{0.75} FO%
 &|[fill=green]|\color[gray]{0.75} FO%
 &|[fill=green]|\color[gray]{0.75} FO%
 &|[fill=green]|\color[gray]{0.75} FO%
 &|[fill=green]|\color[gray]{0.75} FO%
 &|[fill=green]|\color[gray]{0.75} FO%
 &|[fill=green]|\color[gray]{0.75} FO%
 &|[fill=white]|\color[gray]{0.5}WA%
 &|[fill=green]|\color[gray]{0.75} FO%
\\
|[fill=green]|\color[gray]{0.75} FO%
 &|[fill=white]|\color[gray]{0.5}WA%
 &|[fill=green]|\color[gray]{0.75} FO%
 &|[fill=green]|\color[gray]{0.75} FO%
 &|[fill=green]|\color[gray]{0.75} FO%
 &|[fill=green]|\color[gray]{0.75} FO%
 &|[fill=green]|\color[gray]{0.75} FO%
 &|[fill=green]|\color[gray]{0.75} FO%
 &|[fill=green]|\color[gray]{0.75} FO%
 &|[fill=green]|\color[gray]{0.75} FO%
 &|[fill=green]|\color[gray]{0.75} FO%
 &|[fill=white]|\color[gray]{0.5}WA%
 &|[fill=green]|\color[gray]{0.75} FO%
\\
|[fill=green]|\color[gray]{0.75} FO%
 &|[fill=white]|\color[gray]{0.5}WA%
 &|[fill=green]|\color[gray]{0.75} FO%
 &|[fill=green]|\color[gray]{0.75} FO%
 &|[fill=green]|\color[gray]{0.75} FO%
 &|[fill=green]|\color[gray]{0.75} FO%
 &|[fill=green]|\color[gray]{0.75} FO%
 &|[fill=green]|\color[gray]{0.75} FO%
 &|[fill=green]|\color[gray]{0.75} FO%
 &|[fill=green]|\color[gray]{0.75} FO%
 &|[fill=green]|\color[gray]{0.75} FO%
 &|[fill=white]|\color[gray]{0.5}WA%
 &|[fill=green]|\color[gray]{0.75} FO%
\\
|[fill=green]|\color[gray]{0.75} FO%
 &|[fill=white]|\color[gray]{0.5}WA%
 &|[fill=white]|\color[gray]{0.5}WA%
 &|[fill=white]|\color[gray]{0.5}WA%
 &|[fill=white]|\color[gray]{0.5}WA%
 &|[fill=white]|\color[gray]{0.5}WA%
 &|[fill=green]|\color[gray]{0.75} FO%
 &|[fill=white]|\color[gray]{0.5}WA%
 &|[fill=white]|\color[gray]{0.5}WA%
 &|[fill=white]|\color[gray]{0.5}WA%
 &|[fill=white]|\color[gray]{0.5}WA%
 &|[fill=white]|\color[gray]{0.5}WA%
 &|[fill=green]|\color[gray]{0.75} FO%
\\
|[fill=green]|\color[gray]{0.75} FO%
 &|[fill=green]|\color[gray]{0.75} FO%
 &|[fill=green]|\color[gray]{0.75} FO%
 &|[fill=green]|\color[gray]{0.75} FO%
 &|[fill=green]|\color[gray]{0.75} FO%
 &|[fill=green]|\color[gray]{0.75} FO%
 &|[fill=green]|\color[gray]{0.75} FO%
 &|[fill=green]|\color[gray]{0.75} FO%
 &|[fill=green]|\color[gray]{0.75} FO%
 &|[fill=green]|\color[gray]{0.75} FO%
 &|[fill=green]|\color[gray]{0.75} FO%
 &|[fill=green]|\color[gray]{0.75} FO%
 &|[fill=green]|\color[gray]{0.75} FO%
\\
};
\end{tikzpicture}\\
At time 1: Water spot (7|6):
\\
\begin{tikzpicture}
\tikzset{square matrix/.style={
matrix of nodes,
column sep=-\pgflinewidth, row sep=-\pgflinewidth,
nodes={draw,
minimum height=#1,
anchor=center,
text width=#1,
align=center,
inner sep=0pt
},
},
square matrix/.default=1.2cm
}
\matrix[square matrix=1.4em] {
|[fill=green]|\color[gray]{0.75} FO%
 &|[fill=green]|\color[gray]{0.75} FO%
 &|[fill=green]|\color[gray]{0.75} FO%
 &|[fill=green]|\color[gray]{0.75} FO%
 &|[fill=green]|\color[gray]{0.75} FO%
 &|[fill=green]|\color[gray]{0.75} FO%
 &|[fill=green]|\color[gray]{0.75} FO%
 &|[fill=green]|\color[gray]{0.75} FO%
 &|[fill=green]|\color[gray]{0.75} FO%
 &|[fill=green]|\color[gray]{0.75} FO%
 &|[fill=green]|\color[gray]{0.75} FO%
 &|[fill=green]|\color[gray]{0.75} FO%
 &|[fill=green]|\color[gray]{0.75} FO%
\\
|[fill=green]|\color[gray]{0.75} FO%
 &|[fill=white]|\color[gray]{0.5}WA%
 &|[fill=white]|\color[gray]{0.5}WA%
 &|[fill=white]|\color[gray]{0.5}WA%
 &|[fill=white]|\color[gray]{0.5}WA%
 &|[fill=white]|\color[gray]{0.5}WA%
 &|[fill=green]|\color[gray]{0.75} FO%
 &|[fill=white]|\color[gray]{0.5}WA%
 &|[fill=white]|\color[gray]{0.5}WA%
 &|[fill=white]|\color[gray]{0.5}WA%
 &|[fill=white]|\color[gray]{0.5}WA%
 &|[fill=white]|\color[gray]{0.5}WA%
 &|[fill=green]|\color[gray]{0.75} FO%
\\
|[fill=green]|\color[gray]{0.75} FO%
 &|[fill=white]|\color[gray]{0.5}WA%
 &|[fill=green]|\color[gray]{0.75} FO%
 &|[fill=green]|\color[gray]{0.75} FO%
 &|[fill=green]|\color[gray]{0.75} FO%
 &|[fill=green]|\color[gray]{0.75} FO%
 &|[fill=green]|\color[gray]{0.75} FO%
 &|[fill=green]|\color[gray]{0.75} FO%
 &|[fill=green]|\color[gray]{0.75} FO%
 &|[fill=green]|\color[gray]{0.75} FO%
 &|[fill=green]|\color[gray]{0.75} FO%
 &|[fill=white]|\color[gray]{0.5}WA%
 &|[fill=green]|\color[gray]{0.75} FO%
\\
|[fill=green]|\color[gray]{0.75} FO%
 &|[fill=white]|\color[gray]{0.5}WA%
 &|[fill=green]|\color[gray]{0.75} FO%
 &|[fill=green]|\color[gray]{0.75} FO%
 &|[fill=green]|\color[gray]{0.75} FO%
 &|[fill=green]|\color[gray]{0.75} FO%
 &|[fill=green]|\color[gray]{0.75} FO%
 &|[fill=green]|\color[gray]{0.75} FO%
 &|[fill=green]|\color[gray]{0.75} FO%
 &|[fill=green]|\color[gray]{0.75} FO%
 &|[fill=green]|\color[gray]{0.75} FO%
 &|[fill=white]|\color[gray]{0.5}WA%
 &|[fill=green]|\color[gray]{0.75} FO%
\\
|[fill=green]|\color[gray]{0.75} FO%
 &|[fill=white]|\color[gray]{0.5}WA%
 &|[fill=green]|\color[gray]{0.75} FO%
 &|[fill=green]|\color[gray]{0.75} FO%
 &|[fill=green]|\color[gray]{0.75} FO%
 &|[fill=green]|\color[gray]{0.75} FO%
 &|[fill=green]|\color[gray]{0.75} FO%
 &|[fill=green]|\color[gray]{0.75} FO%
 &|[fill=green]|\color[gray]{0.75} FO%
 &|[fill=green]|\color[gray]{0.75} FO%
 &|[fill=green]|\color[gray]{0.75} FO%
 &|[fill=white]|\color[gray]{0.5}WA%
 &|[fill=green]|\color[gray]{0.75} FO%
\\
|[fill=green]|\color[gray]{0.75} FO%
 &|[fill=white]|\color[gray]{0.5}WA%
 &|[fill=green]|\color[gray]{0.75} FO%
 &|[fill=green]|\color[gray]{0.75} FO%
 &|[fill=green]|\color[gray]{0.75} FO%
 &|[fill=green]|\color[gray]{0.75} FO%
 &|[fill=green]|\color[rgb]{1,0,0}\textbf{BU}%
 &|[fill=green]|\color[gray]{0.75} FO%
 &|[fill=green]|\color[gray]{0.75} FO%
 &|[fill=green]|\color[gray]{0.75} FO%
 &|[fill=green]|\color[gray]{0.75} FO%
 &|[fill=white]|\color[gray]{0.5}WA%
 &|[fill=green]|\color[gray]{0.75} FO%
\\
|[fill=green]|\color[gray]{0.75} FO%
 &|[fill=green]|\color[gray]{0.75} FO%
 &|[fill=green]|\color[gray]{0.75} FO%
 &|[fill=green]|\color[gray]{0.75} FO%
 &|[fill=green]|\color[gray]{0.75} FO%
 &|[fill=green]|\color[rgb]{1,0,0}\textbf{BU}%
 &|[fill=green]|\color[rgb]{0,0,0}\textbf{CO}%
 &|[fill=cyan]|\color[rgb]{1,0,0}\textbf{01}%
 &|[fill=green]|\color[gray]{0.75} FO%
 &|[fill=green]|\color[gray]{0.75} FO%
 &|[fill=green]|\color[gray]{0.75} FO%
 &|[fill=green]|\color[gray]{0.75} FO%
 &|[fill=green]|\color[gray]{0.75} FO%
\\
|[fill=green]|\color[gray]{0.75} FO%
 &|[fill=white]|\color[gray]{0.5}WA%
 &|[fill=green]|\color[gray]{0.75} FO%
 &|[fill=green]|\color[gray]{0.75} FO%
 &|[fill=green]|\color[gray]{0.75} FO%
 &|[fill=green]|\color[gray]{0.75} FO%
 &|[fill=green]|\color[rgb]{1,0,0}\textbf{BU}%
 &|[fill=green]|\color[gray]{0.75} FO%
 &|[fill=green]|\color[gray]{0.75} FO%
 &|[fill=green]|\color[gray]{0.75} FO%
 &|[fill=green]|\color[gray]{0.75} FO%
 &|[fill=white]|\color[gray]{0.5}WA%
 &|[fill=green]|\color[gray]{0.75} FO%
\\
|[fill=green]|\color[gray]{0.75} FO%
 &|[fill=white]|\color[gray]{0.5}WA%
 &|[fill=green]|\color[gray]{0.75} FO%
 &|[fill=green]|\color[gray]{0.75} FO%
 &|[fill=green]|\color[gray]{0.75} FO%
 &|[fill=green]|\color[gray]{0.75} FO%
 &|[fill=green]|\color[gray]{0.75} FO%
 &|[fill=green]|\color[gray]{0.75} FO%
 &|[fill=green]|\color[gray]{0.75} FO%
 &|[fill=green]|\color[gray]{0.75} FO%
 &|[fill=green]|\color[gray]{0.75} FO%
 &|[fill=white]|\color[gray]{0.5}WA%
 &|[fill=green]|\color[gray]{0.75} FO%
\\
|[fill=green]|\color[gray]{0.75} FO%
 &|[fill=white]|\color[gray]{0.5}WA%
 &|[fill=green]|\color[gray]{0.75} FO%
 &|[fill=green]|\color[gray]{0.75} FO%
 &|[fill=green]|\color[gray]{0.75} FO%
 &|[fill=green]|\color[gray]{0.75} FO%
 &|[fill=green]|\color[gray]{0.75} FO%
 &|[fill=green]|\color[gray]{0.75} FO%
 &|[fill=green]|\color[gray]{0.75} FO%
 &|[fill=green]|\color[gray]{0.75} FO%
 &|[fill=green]|\color[gray]{0.75} FO%
 &|[fill=white]|\color[gray]{0.5}WA%
 &|[fill=green]|\color[gray]{0.75} FO%
\\
|[fill=green]|\color[gray]{0.75} FO%
 &|[fill=white]|\color[gray]{0.5}WA%
 &|[fill=green]|\color[gray]{0.75} FO%
 &|[fill=green]|\color[gray]{0.75} FO%
 &|[fill=green]|\color[gray]{0.75} FO%
 &|[fill=green]|\color[gray]{0.75} FO%
 &|[fill=green]|\color[gray]{0.75} FO%
 &|[fill=green]|\color[gray]{0.75} FO%
 &|[fill=green]|\color[gray]{0.75} FO%
 &|[fill=green]|\color[gray]{0.75} FO%
 &|[fill=green]|\color[gray]{0.75} FO%
 &|[fill=white]|\color[gray]{0.5}WA%
 &|[fill=green]|\color[gray]{0.75} FO%
\\
|[fill=green]|\color[gray]{0.75} FO%
 &|[fill=white]|\color[gray]{0.5}WA%
 &|[fill=white]|\color[gray]{0.5}WA%
 &|[fill=white]|\color[gray]{0.5}WA%
 &|[fill=white]|\color[gray]{0.5}WA%
 &|[fill=white]|\color[gray]{0.5}WA%
 &|[fill=green]|\color[gray]{0.75} FO%
 &|[fill=white]|\color[gray]{0.5}WA%
 &|[fill=white]|\color[gray]{0.5}WA%
 &|[fill=white]|\color[gray]{0.5}WA%
 &|[fill=white]|\color[gray]{0.5}WA%
 &|[fill=white]|\color[gray]{0.5}WA%
 &|[fill=green]|\color[gray]{0.75} FO%
\\
|[fill=green]|\color[gray]{0.75} FO%
 &|[fill=green]|\color[gray]{0.75} FO%
 &|[fill=green]|\color[gray]{0.75} FO%
 &|[fill=green]|\color[gray]{0.75} FO%
 &|[fill=green]|\color[gray]{0.75} FO%
 &|[fill=green]|\color[gray]{0.75} FO%
 &|[fill=green]|\color[gray]{0.75} FO%
 &|[fill=green]|\color[gray]{0.75} FO%
 &|[fill=green]|\color[gray]{0.75} FO%
 &|[fill=green]|\color[gray]{0.75} FO%
 &|[fill=green]|\color[gray]{0.75} FO%
 &|[fill=green]|\color[gray]{0.75} FO%
 &|[fill=green]|\color[gray]{0.75} FO%
\\
};
\end{tikzpicture}\\
At time 2: Water spot (6|8):
\\
\begin{tikzpicture}
\tikzset{square matrix/.style={
matrix of nodes,
column sep=-\pgflinewidth, row sep=-\pgflinewidth,
nodes={draw,
minimum height=#1,
anchor=center,
text width=#1,
align=center,
inner sep=0pt
},
},
square matrix/.default=1.2cm
}
\matrix[square matrix=1.4em] {
|[fill=green]|\color[gray]{0.75} FO%
 &|[fill=green]|\color[gray]{0.75} FO%
 &|[fill=green]|\color[gray]{0.75} FO%
 &|[fill=green]|\color[gray]{0.75} FO%
 &|[fill=green]|\color[gray]{0.75} FO%
 &|[fill=green]|\color[gray]{0.75} FO%
 &|[fill=green]|\color[gray]{0.75} FO%
 &|[fill=green]|\color[gray]{0.75} FO%
 &|[fill=green]|\color[gray]{0.75} FO%
 &|[fill=green]|\color[gray]{0.75} FO%
 &|[fill=green]|\color[gray]{0.75} FO%
 &|[fill=green]|\color[gray]{0.75} FO%
 &|[fill=green]|\color[gray]{0.75} FO%
\\
|[fill=green]|\color[gray]{0.75} FO%
 &|[fill=white]|\color[gray]{0.5}WA%
 &|[fill=white]|\color[gray]{0.5}WA%
 &|[fill=white]|\color[gray]{0.5}WA%
 &|[fill=white]|\color[gray]{0.5}WA%
 &|[fill=white]|\color[gray]{0.5}WA%
 &|[fill=green]|\color[gray]{0.75} FO%
 &|[fill=white]|\color[gray]{0.5}WA%
 &|[fill=white]|\color[gray]{0.5}WA%
 &|[fill=white]|\color[gray]{0.5}WA%
 &|[fill=white]|\color[gray]{0.5}WA%
 &|[fill=white]|\color[gray]{0.5}WA%
 &|[fill=green]|\color[gray]{0.75} FO%
\\
|[fill=green]|\color[gray]{0.75} FO%
 &|[fill=white]|\color[gray]{0.5}WA%
 &|[fill=green]|\color[gray]{0.75} FO%
 &|[fill=green]|\color[gray]{0.75} FO%
 &|[fill=green]|\color[gray]{0.75} FO%
 &|[fill=green]|\color[gray]{0.75} FO%
 &|[fill=green]|\color[gray]{0.75} FO%
 &|[fill=green]|\color[gray]{0.75} FO%
 &|[fill=green]|\color[gray]{0.75} FO%
 &|[fill=green]|\color[gray]{0.75} FO%
 &|[fill=green]|\color[gray]{0.75} FO%
 &|[fill=white]|\color[gray]{0.5}WA%
 &|[fill=green]|\color[gray]{0.75} FO%
\\
|[fill=green]|\color[gray]{0.75} FO%
 &|[fill=white]|\color[gray]{0.5}WA%
 &|[fill=green]|\color[gray]{0.75} FO%
 &|[fill=green]|\color[gray]{0.75} FO%
 &|[fill=green]|\color[gray]{0.75} FO%
 &|[fill=green]|\color[gray]{0.75} FO%
 &|[fill=green]|\color[gray]{0.75} FO%
 &|[fill=green]|\color[gray]{0.75} FO%
 &|[fill=green]|\color[gray]{0.75} FO%
 &|[fill=green]|\color[gray]{0.75} FO%
 &|[fill=green]|\color[gray]{0.75} FO%
 &|[fill=white]|\color[gray]{0.5}WA%
 &|[fill=green]|\color[gray]{0.75} FO%
\\
|[fill=green]|\color[gray]{0.75} FO%
 &|[fill=white]|\color[gray]{0.5}WA%
 &|[fill=green]|\color[gray]{0.75} FO%
 &|[fill=green]|\color[gray]{0.75} FO%
 &|[fill=green]|\color[gray]{0.75} FO%
 &|[fill=green]|\color[gray]{0.75} FO%
 &|[fill=green]|\color[rgb]{1,0,0}\textbf{BU}%
 &|[fill=green]|\color[gray]{0.75} FO%
 &|[fill=green]|\color[gray]{0.75} FO%
 &|[fill=green]|\color[gray]{0.75} FO%
 &|[fill=green]|\color[gray]{0.75} FO%
 &|[fill=white]|\color[gray]{0.5}WA%
 &|[fill=green]|\color[gray]{0.75} FO%
\\
|[fill=green]|\color[gray]{0.75} FO%
 &|[fill=white]|\color[gray]{0.5}WA%
 &|[fill=green]|\color[gray]{0.75} FO%
 &|[fill=green]|\color[gray]{0.75} FO%
 &|[fill=green]|\color[gray]{0.75} FO%
 &|[fill=green]|\color[rgb]{1,0,0}\textbf{BU}%
 &|[fill=green]|\color[rgb]{0,0,0}\textbf{CO}%
 &|[fill=green]|\color[rgb]{1,0,0}\textbf{BU}%
 &|[fill=green]|\color[gray]{0.75} FO%
 &|[fill=green]|\color[gray]{0.75} FO%
 &|[fill=green]|\color[gray]{0.75} FO%
 &|[fill=white]|\color[gray]{0.5}WA%
 &|[fill=green]|\color[gray]{0.75} FO%
\\
|[fill=green]|\color[gray]{0.75} FO%
 &|[fill=green]|\color[gray]{0.75} FO%
 &|[fill=green]|\color[gray]{0.75} FO%
 &|[fill=green]|\color[gray]{0.75} FO%
 &|[fill=green]|\color[rgb]{1,0,0}\textbf{BU}%
 &|[fill=green]|\color[rgb]{0,0,0}\textbf{CO}%
 &|[fill=green]|\color[rgb]{0,0,0}\textbf{CO}%
 &|[fill=cyan]|\color[rgb]{1,0,0}\textbf{01}%
 &|[fill=green]|\color[gray]{0.75} FO%
 &|[fill=green]|\color[gray]{0.75} FO%
 &|[fill=green]|\color[gray]{0.75} FO%
 &|[fill=green]|\color[gray]{0.75} FO%
 &|[fill=green]|\color[gray]{0.75} FO%
\\
|[fill=green]|\color[gray]{0.75} FO%
 &|[fill=white]|\color[gray]{0.5}WA%
 &|[fill=green]|\color[gray]{0.75} FO%
 &|[fill=green]|\color[gray]{0.75} FO%
 &|[fill=green]|\color[gray]{0.75} FO%
 &|[fill=green]|\color[rgb]{1,0,0}\textbf{BU}%
 &|[fill=green]|\color[rgb]{0,0,0}\textbf{CO}%
 &|[fill=green]|\color[rgb]{1,0,0}\textbf{BU}%
 &|[fill=green]|\color[gray]{0.75} FO%
 &|[fill=green]|\color[gray]{0.75} FO%
 &|[fill=green]|\color[gray]{0.75} FO%
 &|[fill=white]|\color[gray]{0.5}WA%
 &|[fill=green]|\color[gray]{0.75} FO%
\\
|[fill=green]|\color[gray]{0.75} FO%
 &|[fill=white]|\color[gray]{0.5}WA%
 &|[fill=green]|\color[gray]{0.75} FO%
 &|[fill=green]|\color[gray]{0.75} FO%
 &|[fill=green]|\color[gray]{0.75} FO%
 &|[fill=green]|\color[gray]{0.75} FO%
 &|[fill=cyan]|\color[rgb]{1,0,0}\textbf{02}%
 &|[fill=green]|\color[gray]{0.75} FO%
 &|[fill=green]|\color[gray]{0.75} FO%
 &|[fill=green]|\color[gray]{0.75} FO%
 &|[fill=green]|\color[gray]{0.75} FO%
 &|[fill=white]|\color[gray]{0.5}WA%
 &|[fill=green]|\color[gray]{0.75} FO%
\\
|[fill=green]|\color[gray]{0.75} FO%
 &|[fill=white]|\color[gray]{0.5}WA%
 &|[fill=green]|\color[gray]{0.75} FO%
 &|[fill=green]|\color[gray]{0.75} FO%
 &|[fill=green]|\color[gray]{0.75} FO%
 &|[fill=green]|\color[gray]{0.75} FO%
 &|[fill=green]|\color[gray]{0.75} FO%
 &|[fill=green]|\color[gray]{0.75} FO%
 &|[fill=green]|\color[gray]{0.75} FO%
 &|[fill=green]|\color[gray]{0.75} FO%
 &|[fill=green]|\color[gray]{0.75} FO%
 &|[fill=white]|\color[gray]{0.5}WA%
 &|[fill=green]|\color[gray]{0.75} FO%
\\
|[fill=green]|\color[gray]{0.75} FO%
 &|[fill=white]|\color[gray]{0.5}WA%
 &|[fill=green]|\color[gray]{0.75} FO%
 &|[fill=green]|\color[gray]{0.75} FO%
 &|[fill=green]|\color[gray]{0.75} FO%
 &|[fill=green]|\color[gray]{0.75} FO%
 &|[fill=green]|\color[gray]{0.75} FO%
 &|[fill=green]|\color[gray]{0.75} FO%
 &|[fill=green]|\color[gray]{0.75} FO%
 &|[fill=green]|\color[gray]{0.75} FO%
 &|[fill=green]|\color[gray]{0.75} FO%
 &|[fill=white]|\color[gray]{0.5}WA%
 &|[fill=green]|\color[gray]{0.75} FO%
\\
|[fill=green]|\color[gray]{0.75} FO%
 &|[fill=white]|\color[gray]{0.5}WA%
 &|[fill=white]|\color[gray]{0.5}WA%
 &|[fill=white]|\color[gray]{0.5}WA%
 &|[fill=white]|\color[gray]{0.5}WA%
 &|[fill=white]|\color[gray]{0.5}WA%
 &|[fill=green]|\color[gray]{0.75} FO%
 &|[fill=white]|\color[gray]{0.5}WA%
 &|[fill=white]|\color[gray]{0.5}WA%
 &|[fill=white]|\color[gray]{0.5}WA%
 &|[fill=white]|\color[gray]{0.5}WA%
 &|[fill=white]|\color[gray]{0.5}WA%
 &|[fill=green]|\color[gray]{0.75} FO%
\\
|[fill=green]|\color[gray]{0.75} FO%
 &|[fill=green]|\color[gray]{0.75} FO%
 &|[fill=green]|\color[gray]{0.75} FO%
 &|[fill=green]|\color[gray]{0.75} FO%
 &|[fill=green]|\color[gray]{0.75} FO%
 &|[fill=green]|\color[gray]{0.75} FO%
 &|[fill=green]|\color[gray]{0.75} FO%
 &|[fill=green]|\color[gray]{0.75} FO%
 &|[fill=green]|\color[gray]{0.75} FO%
 &|[fill=green]|\color[gray]{0.75} FO%
 &|[fill=green]|\color[gray]{0.75} FO%
 &|[fill=green]|\color[gray]{0.75} FO%
 &|[fill=green]|\color[gray]{0.75} FO%
\\
};
\end{tikzpicture}\\
At time 3: Water spot (3|6):
\\
\begin{tikzpicture}
\tikzset{square matrix/.style={
matrix of nodes,
column sep=-\pgflinewidth, row sep=-\pgflinewidth,
nodes={draw,
minimum height=#1,
anchor=center,
text width=#1,
align=center,
inner sep=0pt
},
},
square matrix/.default=1.2cm
}
\matrix[square matrix=1.4em] {
|[fill=green]|\color[gray]{0.75} FO%
 &|[fill=green]|\color[gray]{0.75} FO%
 &|[fill=green]|\color[gray]{0.75} FO%
 &|[fill=green]|\color[gray]{0.75} FO%
 &|[fill=green]|\color[gray]{0.75} FO%
 &|[fill=green]|\color[gray]{0.75} FO%
 &|[fill=green]|\color[gray]{0.75} FO%
 &|[fill=green]|\color[gray]{0.75} FO%
 &|[fill=green]|\color[gray]{0.75} FO%
 &|[fill=green]|\color[gray]{0.75} FO%
 &|[fill=green]|\color[gray]{0.75} FO%
 &|[fill=green]|\color[gray]{0.75} FO%
 &|[fill=green]|\color[gray]{0.75} FO%
\\
|[fill=green]|\color[gray]{0.75} FO%
 &|[fill=white]|\color[gray]{0.5}WA%
 &|[fill=white]|\color[gray]{0.5}WA%
 &|[fill=white]|\color[gray]{0.5}WA%
 &|[fill=white]|\color[gray]{0.5}WA%
 &|[fill=white]|\color[gray]{0.5}WA%
 &|[fill=green]|\color[gray]{0.75} FO%
 &|[fill=white]|\color[gray]{0.5}WA%
 &|[fill=white]|\color[gray]{0.5}WA%
 &|[fill=white]|\color[gray]{0.5}WA%
 &|[fill=white]|\color[gray]{0.5}WA%
 &|[fill=white]|\color[gray]{0.5}WA%
 &|[fill=green]|\color[gray]{0.75} FO%
\\
|[fill=green]|\color[gray]{0.75} FO%
 &|[fill=white]|\color[gray]{0.5}WA%
 &|[fill=green]|\color[gray]{0.75} FO%
 &|[fill=green]|\color[gray]{0.75} FO%
 &|[fill=green]|\color[gray]{0.75} FO%
 &|[fill=green]|\color[gray]{0.75} FO%
 &|[fill=green]|\color[gray]{0.75} FO%
 &|[fill=green]|\color[gray]{0.75} FO%
 &|[fill=green]|\color[gray]{0.75} FO%
 &|[fill=green]|\color[gray]{0.75} FO%
 &|[fill=green]|\color[gray]{0.75} FO%
 &|[fill=white]|\color[gray]{0.5}WA%
 &|[fill=green]|\color[gray]{0.75} FO%
\\
|[fill=green]|\color[gray]{0.75} FO%
 &|[fill=white]|\color[gray]{0.5}WA%
 &|[fill=green]|\color[gray]{0.75} FO%
 &|[fill=green]|\color[gray]{0.75} FO%
 &|[fill=green]|\color[gray]{0.75} FO%
 &|[fill=green]|\color[gray]{0.75} FO%
 &|[fill=green]|\color[rgb]{1,0,0}\textbf{BU}%
 &|[fill=green]|\color[gray]{0.75} FO%
 &|[fill=green]|\color[gray]{0.75} FO%
 &|[fill=green]|\color[gray]{0.75} FO%
 &|[fill=green]|\color[gray]{0.75} FO%
 &|[fill=white]|\color[gray]{0.5}WA%
 &|[fill=green]|\color[gray]{0.75} FO%
\\
|[fill=green]|\color[gray]{0.75} FO%
 &|[fill=white]|\color[gray]{0.5}WA%
 &|[fill=green]|\color[gray]{0.75} FO%
 &|[fill=green]|\color[gray]{0.75} FO%
 &|[fill=green]|\color[gray]{0.75} FO%
 &|[fill=green]|\color[rgb]{1,0,0}\textbf{BU}%
 &|[fill=green]|\color[rgb]{0,0,0}\textbf{CO}%
 &|[fill=green]|\color[rgb]{1,0,0}\textbf{BU}%
 &|[fill=green]|\color[gray]{0.75} FO%
 &|[fill=green]|\color[gray]{0.75} FO%
 &|[fill=green]|\color[gray]{0.75} FO%
 &|[fill=white]|\color[gray]{0.5}WA%
 &|[fill=green]|\color[gray]{0.75} FO%
\\
|[fill=green]|\color[gray]{0.75} FO%
 &|[fill=white]|\color[gray]{0.5}WA%
 &|[fill=green]|\color[gray]{0.75} FO%
 &|[fill=green]|\color[gray]{0.75} FO%
 &|[fill=green]|\color[rgb]{1,0,0}\textbf{BU}%
 &|[fill=green]|\color[rgb]{0,0,0}\textbf{CO}%
 &|[fill=green]|\color[rgb]{0,0,0}\textbf{CO}%
 &|[fill=green]|\color[rgb]{0,0,0}\textbf{CO}%
 &|[fill=green]|\color[rgb]{1,0,0}\textbf{BU}%
 &|[fill=green]|\color[gray]{0.75} FO%
 &|[fill=green]|\color[gray]{0.75} FO%
 &|[fill=white]|\color[gray]{0.5}WA%
 &|[fill=green]|\color[gray]{0.75} FO%
\\
|[fill=green]|\color[gray]{0.75} FO%
 &|[fill=green]|\color[gray]{0.75} FO%
 &|[fill=green]|\color[gray]{0.75} FO%
 &|[fill=cyan]|\color[rgb]{1,0,0}\textbf{03}%
 &|[fill=green]|\color[rgb]{0,0,0}\textbf{CO}%
 &|[fill=green]|\color[rgb]{0,0,0}\textbf{CO}%
 &|[fill=green]|\color[rgb]{0,0,0}\textbf{CO}%
 &|[fill=cyan]|\color[rgb]{1,0,0}\textbf{01}%
 &|[fill=green]|\color[gray]{0.75} FO%
 &|[fill=green]|\color[gray]{0.75} FO%
 &|[fill=green]|\color[gray]{0.75} FO%
 &|[fill=green]|\color[gray]{0.75} FO%
 &|[fill=green]|\color[gray]{0.75} FO%
\\
|[fill=green]|\color[gray]{0.75} FO%
 &|[fill=white]|\color[gray]{0.5}WA%
 &|[fill=green]|\color[gray]{0.75} FO%
 &|[fill=green]|\color[gray]{0.75} FO%
 &|[fill=green]|\color[rgb]{1,0,0}\textbf{BU}%
 &|[fill=green]|\color[rgb]{0,0,0}\textbf{CO}%
 &|[fill=green]|\color[rgb]{0,0,0}\textbf{CO}%
 &|[fill=green]|\color[rgb]{0,0,0}\textbf{CO}%
 &|[fill=green]|\color[rgb]{1,0,0}\textbf{BU}%
 &|[fill=green]|\color[gray]{0.75} FO%
 &|[fill=green]|\color[gray]{0.75} FO%
 &|[fill=white]|\color[gray]{0.5}WA%
 &|[fill=green]|\color[gray]{0.75} FO%
\\
|[fill=green]|\color[gray]{0.75} FO%
 &|[fill=white]|\color[gray]{0.5}WA%
 &|[fill=green]|\color[gray]{0.75} FO%
 &|[fill=green]|\color[gray]{0.75} FO%
 &|[fill=green]|\color[gray]{0.75} FO%
 &|[fill=green]|\color[rgb]{1,0,0}\textbf{BU}%
 &|[fill=cyan]|\color[rgb]{1,0,0}\textbf{02}%
 &|[fill=green]|\color[rgb]{1,0,0}\textbf{BU}%
 &|[fill=green]|\color[gray]{0.75} FO%
 &|[fill=green]|\color[gray]{0.75} FO%
 &|[fill=green]|\color[gray]{0.75} FO%
 &|[fill=white]|\color[gray]{0.5}WA%
 &|[fill=green]|\color[gray]{0.75} FO%
\\
|[fill=green]|\color[gray]{0.75} FO%
 &|[fill=white]|\color[gray]{0.5}WA%
 &|[fill=green]|\color[gray]{0.75} FO%
 &|[fill=green]|\color[gray]{0.75} FO%
 &|[fill=green]|\color[gray]{0.75} FO%
 &|[fill=green]|\color[gray]{0.75} FO%
 &|[fill=green]|\color[gray]{0.75} FO%
 &|[fill=green]|\color[gray]{0.75} FO%
 &|[fill=green]|\color[gray]{0.75} FO%
 &|[fill=green]|\color[gray]{0.75} FO%
 &|[fill=green]|\color[gray]{0.75} FO%
 &|[fill=white]|\color[gray]{0.5}WA%
 &|[fill=green]|\color[gray]{0.75} FO%
\\
|[fill=green]|\color[gray]{0.75} FO%
 &|[fill=white]|\color[gray]{0.5}WA%
 &|[fill=green]|\color[gray]{0.75} FO%
 &|[fill=green]|\color[gray]{0.75} FO%
 &|[fill=green]|\color[gray]{0.75} FO%
 &|[fill=green]|\color[gray]{0.75} FO%
 &|[fill=green]|\color[gray]{0.75} FO%
 &|[fill=green]|\color[gray]{0.75} FO%
 &|[fill=green]|\color[gray]{0.75} FO%
 &|[fill=green]|\color[gray]{0.75} FO%
 &|[fill=green]|\color[gray]{0.75} FO%
 &|[fill=white]|\color[gray]{0.5}WA%
 &|[fill=green]|\color[gray]{0.75} FO%
\\
|[fill=green]|\color[gray]{0.75} FO%
 &|[fill=white]|\color[gray]{0.5}WA%
 &|[fill=white]|\color[gray]{0.5}WA%
 &|[fill=white]|\color[gray]{0.5}WA%
 &|[fill=white]|\color[gray]{0.5}WA%
 &|[fill=white]|\color[gray]{0.5}WA%
 &|[fill=green]|\color[gray]{0.75} FO%
 &|[fill=white]|\color[gray]{0.5}WA%
 &|[fill=white]|\color[gray]{0.5}WA%
 &|[fill=white]|\color[gray]{0.5}WA%
 &|[fill=white]|\color[gray]{0.5}WA%
 &|[fill=white]|\color[gray]{0.5}WA%
 &|[fill=green]|\color[gray]{0.75} FO%
\\
|[fill=green]|\color[gray]{0.75} FO%
 &|[fill=green]|\color[gray]{0.75} FO%
 &|[fill=green]|\color[gray]{0.75} FO%
 &|[fill=green]|\color[gray]{0.75} FO%
 &|[fill=green]|\color[gray]{0.75} FO%
 &|[fill=green]|\color[gray]{0.75} FO%
 &|[fill=green]|\color[gray]{0.75} FO%
 &|[fill=green]|\color[gray]{0.75} FO%
 &|[fill=green]|\color[gray]{0.75} FO%
 &|[fill=green]|\color[gray]{0.75} FO%
 &|[fill=green]|\color[gray]{0.75} FO%
 &|[fill=green]|\color[gray]{0.75} FO%
 &|[fill=green]|\color[gray]{0.75} FO%
\\
};
\end{tikzpicture}\\
At time 4: Water spot (6|2):
\\
\begin{tikzpicture}
\tikzset{square matrix/.style={
matrix of nodes,
column sep=-\pgflinewidth, row sep=-\pgflinewidth,
nodes={draw,
minimum height=#1,
anchor=center,
text width=#1,
align=center,
inner sep=0pt
},
},
square matrix/.default=1.2cm
}
\matrix[square matrix=1.4em] {
|[fill=green]|\color[gray]{0.75} FO%
 &|[fill=green]|\color[gray]{0.75} FO%
 &|[fill=green]|\color[gray]{0.75} FO%
 &|[fill=green]|\color[gray]{0.75} FO%
 &|[fill=green]|\color[gray]{0.75} FO%
 &|[fill=green]|\color[gray]{0.75} FO%
 &|[fill=green]|\color[gray]{0.75} FO%
 &|[fill=green]|\color[gray]{0.75} FO%
 &|[fill=green]|\color[gray]{0.75} FO%
 &|[fill=green]|\color[gray]{0.75} FO%
 &|[fill=green]|\color[gray]{0.75} FO%
 &|[fill=green]|\color[gray]{0.75} FO%
 &|[fill=green]|\color[gray]{0.75} FO%
\\
|[fill=green]|\color[gray]{0.75} FO%
 &|[fill=white]|\color[gray]{0.5}WA%
 &|[fill=white]|\color[gray]{0.5}WA%
 &|[fill=white]|\color[gray]{0.5}WA%
 &|[fill=white]|\color[gray]{0.5}WA%
 &|[fill=white]|\color[gray]{0.5}WA%
 &|[fill=green]|\color[gray]{0.75} FO%
 &|[fill=white]|\color[gray]{0.5}WA%
 &|[fill=white]|\color[gray]{0.5}WA%
 &|[fill=white]|\color[gray]{0.5}WA%
 &|[fill=white]|\color[gray]{0.5}WA%
 &|[fill=white]|\color[gray]{0.5}WA%
 &|[fill=green]|\color[gray]{0.75} FO%
\\
|[fill=green]|\color[gray]{0.75} FO%
 &|[fill=white]|\color[gray]{0.5}WA%
 &|[fill=green]|\color[gray]{0.75} FO%
 &|[fill=green]|\color[gray]{0.75} FO%
 &|[fill=green]|\color[gray]{0.75} FO%
 &|[fill=green]|\color[gray]{0.75} FO%
 &|[fill=cyan]|\color[rgb]{1,0,0}\textbf{04}%
 &|[fill=green]|\color[gray]{0.75} FO%
 &|[fill=green]|\color[gray]{0.75} FO%
 &|[fill=green]|\color[gray]{0.75} FO%
 &|[fill=green]|\color[gray]{0.75} FO%
 &|[fill=white]|\color[gray]{0.5}WA%
 &|[fill=green]|\color[gray]{0.75} FO%
\\
|[fill=green]|\color[gray]{0.75} FO%
 &|[fill=white]|\color[gray]{0.5}WA%
 &|[fill=green]|\color[gray]{0.75} FO%
 &|[fill=green]|\color[gray]{0.75} FO%
 &|[fill=green]|\color[gray]{0.75} FO%
 &|[fill=green]|\color[rgb]{1,0,0}\textbf{BU}%
 &|[fill=green]|\color[rgb]{0,0,0}\textbf{CO}%
 &|[fill=green]|\color[rgb]{1,0,0}\textbf{BU}%
 &|[fill=green]|\color[gray]{0.75} FO%
 &|[fill=green]|\color[gray]{0.75} FO%
 &|[fill=green]|\color[gray]{0.75} FO%
 &|[fill=white]|\color[gray]{0.5}WA%
 &|[fill=green]|\color[gray]{0.75} FO%
\\
|[fill=green]|\color[gray]{0.75} FO%
 &|[fill=white]|\color[gray]{0.5}WA%
 &|[fill=green]|\color[gray]{0.75} FO%
 &|[fill=green]|\color[gray]{0.75} FO%
 &|[fill=green]|\color[rgb]{1,0,0}\textbf{BU}%
 &|[fill=green]|\color[rgb]{0,0,0}\textbf{CO}%
 &|[fill=green]|\color[rgb]{0,0,0}\textbf{CO}%
 &|[fill=green]|\color[rgb]{0,0,0}\textbf{CO}%
 &|[fill=green]|\color[rgb]{1,0,0}\textbf{BU}%
 &|[fill=green]|\color[gray]{0.75} FO%
 &|[fill=green]|\color[gray]{0.75} FO%
 &|[fill=white]|\color[gray]{0.5}WA%
 &|[fill=green]|\color[gray]{0.75} FO%
\\
|[fill=green]|\color[gray]{0.75} FO%
 &|[fill=white]|\color[gray]{0.5}WA%
 &|[fill=green]|\color[gray]{0.75} FO%
 &|[fill=green]|\color[rgb]{1,0,0}\textbf{BU}%
 &|[fill=green]|\color[rgb]{0,0,0}\textbf{CO}%
 &|[fill=green]|\color[rgb]{0,0,0}\textbf{CO}%
 &|[fill=green]|\color[rgb]{0,0,0}\textbf{CO}%
 &|[fill=green]|\color[rgb]{0,0,0}\textbf{CO}%
 &|[fill=green]|\color[rgb]{0,0,0}\textbf{CO}%
 &|[fill=green]|\color[rgb]{1,0,0}\textbf{BU}%
 &|[fill=green]|\color[gray]{0.75} FO%
 &|[fill=white]|\color[gray]{0.5}WA%
 &|[fill=green]|\color[gray]{0.75} FO%
\\
|[fill=green]|\color[gray]{0.75} FO%
 &|[fill=green]|\color[gray]{0.75} FO%
 &|[fill=green]|\color[gray]{0.75} FO%
 &|[fill=cyan]|\color[rgb]{1,0,0}\textbf{03}%
 &|[fill=green]|\color[rgb]{0,0,0}\textbf{CO}%
 &|[fill=green]|\color[rgb]{0,0,0}\textbf{CO}%
 &|[fill=green]|\color[rgb]{0,0,0}\textbf{CO}%
 &|[fill=cyan]|\color[rgb]{1,0,0}\textbf{01}%
 &|[fill=green]|\color[rgb]{1,0,0}\textbf{BU}%
 &|[fill=green]|\color[gray]{0.75} FO%
 &|[fill=green]|\color[gray]{0.75} FO%
 &|[fill=green]|\color[gray]{0.75} FO%
 &|[fill=green]|\color[gray]{0.75} FO%
\\
|[fill=green]|\color[gray]{0.75} FO%
 &|[fill=white]|\color[gray]{0.5}WA%
 &|[fill=green]|\color[gray]{0.75} FO%
 &|[fill=green]|\color[rgb]{1,0,0}\textbf{BU}%
 &|[fill=green]|\color[rgb]{0,0,0}\textbf{CO}%
 &|[fill=green]|\color[rgb]{0,0,0}\textbf{CO}%
 &|[fill=green]|\color[rgb]{0,0,0}\textbf{CO}%
 &|[fill=green]|\color[rgb]{0,0,0}\textbf{CO}%
 &|[fill=green]|\color[rgb]{0,0,0}\textbf{CO}%
 &|[fill=green]|\color[rgb]{1,0,0}\textbf{BU}%
 &|[fill=green]|\color[gray]{0.75} FO%
 &|[fill=white]|\color[gray]{0.5}WA%
 &|[fill=green]|\color[gray]{0.75} FO%
\\
|[fill=green]|\color[gray]{0.75} FO%
 &|[fill=white]|\color[gray]{0.5}WA%
 &|[fill=green]|\color[gray]{0.75} FO%
 &|[fill=green]|\color[gray]{0.75} FO%
 &|[fill=green]|\color[rgb]{1,0,0}\textbf{BU}%
 &|[fill=green]|\color[rgb]{0,0,0}\textbf{CO}%
 &|[fill=cyan]|\color[rgb]{1,0,0}\textbf{02}%
 &|[fill=green]|\color[rgb]{0,0,0}\textbf{CO}%
 &|[fill=green]|\color[rgb]{1,0,0}\textbf{BU}%
 &|[fill=green]|\color[gray]{0.75} FO%
 &|[fill=green]|\color[gray]{0.75} FO%
 &|[fill=white]|\color[gray]{0.5}WA%
 &|[fill=green]|\color[gray]{0.75} FO%
\\
|[fill=green]|\color[gray]{0.75} FO%
 &|[fill=white]|\color[gray]{0.5}WA%
 &|[fill=green]|\color[gray]{0.75} FO%
 &|[fill=green]|\color[gray]{0.75} FO%
 &|[fill=green]|\color[gray]{0.75} FO%
 &|[fill=green]|\color[rgb]{1,0,0}\textbf{BU}%
 &|[fill=green]|\color[gray]{0.75} FO%
 &|[fill=green]|\color[rgb]{1,0,0}\textbf{BU}%
 &|[fill=green]|\color[gray]{0.75} FO%
 &|[fill=green]|\color[gray]{0.75} FO%
 &|[fill=green]|\color[gray]{0.75} FO%
 &|[fill=white]|\color[gray]{0.5}WA%
 &|[fill=green]|\color[gray]{0.75} FO%
\\
|[fill=green]|\color[gray]{0.75} FO%
 &|[fill=white]|\color[gray]{0.5}WA%
 &|[fill=green]|\color[gray]{0.75} FO%
 &|[fill=green]|\color[gray]{0.75} FO%
 &|[fill=green]|\color[gray]{0.75} FO%
 &|[fill=green]|\color[gray]{0.75} FO%
 &|[fill=green]|\color[gray]{0.75} FO%
 &|[fill=green]|\color[gray]{0.75} FO%
 &|[fill=green]|\color[gray]{0.75} FO%
 &|[fill=green]|\color[gray]{0.75} FO%
 &|[fill=green]|\color[gray]{0.75} FO%
 &|[fill=white]|\color[gray]{0.5}WA%
 &|[fill=green]|\color[gray]{0.75} FO%
\\
|[fill=green]|\color[gray]{0.75} FO%
 &|[fill=white]|\color[gray]{0.5}WA%
 &|[fill=white]|\color[gray]{0.5}WA%
 &|[fill=white]|\color[gray]{0.5}WA%
 &|[fill=white]|\color[gray]{0.5}WA%
 &|[fill=white]|\color[gray]{0.5}WA%
 &|[fill=green]|\color[gray]{0.75} FO%
 &|[fill=white]|\color[gray]{0.5}WA%
 &|[fill=white]|\color[gray]{0.5}WA%
 &|[fill=white]|\color[gray]{0.5}WA%
 &|[fill=white]|\color[gray]{0.5}WA%
 &|[fill=white]|\color[gray]{0.5}WA%
 &|[fill=green]|\color[gray]{0.75} FO%
\\
|[fill=green]|\color[gray]{0.75} FO%
 &|[fill=green]|\color[gray]{0.75} FO%
 &|[fill=green]|\color[gray]{0.75} FO%
 &|[fill=green]|\color[gray]{0.75} FO%
 &|[fill=green]|\color[gray]{0.75} FO%
 &|[fill=green]|\color[gray]{0.75} FO%
 &|[fill=green]|\color[gray]{0.75} FO%
 &|[fill=green]|\color[gray]{0.75} FO%
 &|[fill=green]|\color[gray]{0.75} FO%
 &|[fill=green]|\color[gray]{0.75} FO%
 &|[fill=green]|\color[gray]{0.75} FO%
 &|[fill=green]|\color[gray]{0.75} FO%
 &|[fill=green]|\color[gray]{0.75} FO%
\\
};
\end{tikzpicture}\\
At time 5: Water spot (10|7):
\\
\begin{tikzpicture}
\tikzset{square matrix/.style={
matrix of nodes,
column sep=-\pgflinewidth, row sep=-\pgflinewidth,
nodes={draw,
minimum height=#1,
anchor=center,
text width=#1,
align=center,
inner sep=0pt
},
},
square matrix/.default=1.2cm
}
\matrix[square matrix=1.4em] {
|[fill=green]|\color[gray]{0.75} FO%
 &|[fill=green]|\color[gray]{0.75} FO%
 &|[fill=green]|\color[gray]{0.75} FO%
 &|[fill=green]|\color[gray]{0.75} FO%
 &|[fill=green]|\color[gray]{0.75} FO%
 &|[fill=green]|\color[gray]{0.75} FO%
 &|[fill=green]|\color[gray]{0.75} FO%
 &|[fill=green]|\color[gray]{0.75} FO%
 &|[fill=green]|\color[gray]{0.75} FO%
 &|[fill=green]|\color[gray]{0.75} FO%
 &|[fill=green]|\color[gray]{0.75} FO%
 &|[fill=green]|\color[gray]{0.75} FO%
 &|[fill=green]|\color[gray]{0.75} FO%
\\
|[fill=green]|\color[gray]{0.75} FO%
 &|[fill=white]|\color[gray]{0.5}WA%
 &|[fill=white]|\color[gray]{0.5}WA%
 &|[fill=white]|\color[gray]{0.5}WA%
 &|[fill=white]|\color[gray]{0.5}WA%
 &|[fill=white]|\color[gray]{0.5}WA%
 &|[fill=green]|\color[gray]{0.75} FO%
 &|[fill=white]|\color[gray]{0.5}WA%
 &|[fill=white]|\color[gray]{0.5}WA%
 &|[fill=white]|\color[gray]{0.5}WA%
 &|[fill=white]|\color[gray]{0.5}WA%
 &|[fill=white]|\color[gray]{0.5}WA%
 &|[fill=green]|\color[gray]{0.75} FO%
\\
|[fill=green]|\color[gray]{0.75} FO%
 &|[fill=white]|\color[gray]{0.5}WA%
 &|[fill=green]|\color[gray]{0.75} FO%
 &|[fill=green]|\color[gray]{0.75} FO%
 &|[fill=green]|\color[gray]{0.75} FO%
 &|[fill=green]|\color[rgb]{1,0,0}\textbf{BU}%
 &|[fill=cyan]|\color[rgb]{1,0,0}\textbf{04}%
 &|[fill=green]|\color[rgb]{1,0,0}\textbf{BU}%
 &|[fill=green]|\color[gray]{0.75} FO%
 &|[fill=green]|\color[gray]{0.75} FO%
 &|[fill=green]|\color[gray]{0.75} FO%
 &|[fill=white]|\color[gray]{0.5}WA%
 &|[fill=green]|\color[gray]{0.75} FO%
\\
|[fill=green]|\color[gray]{0.75} FO%
 &|[fill=white]|\color[gray]{0.5}WA%
 &|[fill=green]|\color[gray]{0.75} FO%
 &|[fill=green]|\color[gray]{0.75} FO%
 &|[fill=green]|\color[rgb]{1,0,0}\textbf{BU}%
 &|[fill=green]|\color[rgb]{0,0,0}\textbf{CO}%
 &|[fill=green]|\color[rgb]{0,0,0}\textbf{CO}%
 &|[fill=green]|\color[rgb]{0,0,0}\textbf{CO}%
 &|[fill=green]|\color[rgb]{1,0,0}\textbf{BU}%
 &|[fill=green]|\color[gray]{0.75} FO%
 &|[fill=green]|\color[gray]{0.75} FO%
 &|[fill=white]|\color[gray]{0.5}WA%
 &|[fill=green]|\color[gray]{0.75} FO%
\\
|[fill=green]|\color[gray]{0.75} FO%
 &|[fill=white]|\color[gray]{0.5}WA%
 &|[fill=green]|\color[gray]{0.75} FO%
 &|[fill=green]|\color[rgb]{1,0,0}\textbf{BU}%
 &|[fill=green]|\color[rgb]{0,0,0}\textbf{CO}%
 &|[fill=green]|\color[rgb]{0,0,0}\textbf{CO}%
 &|[fill=green]|\color[rgb]{0,0,0}\textbf{CO}%
 &|[fill=green]|\color[rgb]{0,0,0}\textbf{CO}%
 &|[fill=green]|\color[rgb]{0,0,0}\textbf{CO}%
 &|[fill=green]|\color[rgb]{1,0,0}\textbf{BU}%
 &|[fill=green]|\color[gray]{0.75} FO%
 &|[fill=white]|\color[gray]{0.5}WA%
 &|[fill=green]|\color[gray]{0.75} FO%
\\
|[fill=green]|\color[gray]{0.75} FO%
 &|[fill=white]|\color[gray]{0.5}WA%
 &|[fill=green]|\color[rgb]{1,0,0}\textbf{BU}%
 &|[fill=green]|\color[rgb]{0,0,0}\textbf{CO}%
 &|[fill=green]|\color[rgb]{0,0,0}\textbf{CO}%
 &|[fill=green]|\color[rgb]{0,0,0}\textbf{CO}%
 &|[fill=green]|\color[rgb]{0,0,0}\textbf{CO}%
 &|[fill=green]|\color[rgb]{0,0,0}\textbf{CO}%
 &|[fill=green]|\color[rgb]{0,0,0}\textbf{CO}%
 &|[fill=green]|\color[rgb]{0,0,0}\textbf{CO}%
 &|[fill=green]|\color[rgb]{1,0,0}\textbf{BU}%
 &|[fill=white]|\color[gray]{0.5}WA%
 &|[fill=green]|\color[gray]{0.75} FO%
\\
|[fill=green]|\color[gray]{0.75} FO%
 &|[fill=green]|\color[gray]{0.75} FO%
 &|[fill=green]|\color[gray]{0.75} FO%
 &|[fill=cyan]|\color[rgb]{1,0,0}\textbf{03}%
 &|[fill=green]|\color[rgb]{0,0,0}\textbf{CO}%
 &|[fill=green]|\color[rgb]{0,0,0}\textbf{CO}%
 &|[fill=green]|\color[rgb]{0,0,0}\textbf{CO}%
 &|[fill=cyan]|\color[rgb]{1,0,0}\textbf{01}%
 &|[fill=green]|\color[rgb]{0,0,0}\textbf{CO}%
 &|[fill=green]|\color[rgb]{1,0,0}\textbf{BU}%
 &|[fill=green]|\color[gray]{0.75} FO%
 &|[fill=green]|\color[gray]{0.75} FO%
 &|[fill=green]|\color[gray]{0.75} FO%
\\
|[fill=green]|\color[gray]{0.75} FO%
 &|[fill=white]|\color[gray]{0.5}WA%
 &|[fill=green]|\color[rgb]{1,0,0}\textbf{BU}%
 &|[fill=green]|\color[rgb]{0,0,0}\textbf{CO}%
 &|[fill=green]|\color[rgb]{0,0,0}\textbf{CO}%
 &|[fill=green]|\color[rgb]{0,0,0}\textbf{CO}%
 &|[fill=green]|\color[rgb]{0,0,0}\textbf{CO}%
 &|[fill=green]|\color[rgb]{0,0,0}\textbf{CO}%
 &|[fill=green]|\color[rgb]{0,0,0}\textbf{CO}%
 &|[fill=green]|\color[rgb]{0,0,0}\textbf{CO}%
 &|[fill=cyan]|\color[rgb]{1,0,0}\textbf{05}%
 &|[fill=white]|\color[gray]{0.5}WA%
 &|[fill=green]|\color[gray]{0.75} FO%
\\
|[fill=green]|\color[gray]{0.75} FO%
 &|[fill=white]|\color[gray]{0.5}WA%
 &|[fill=green]|\color[gray]{0.75} FO%
 &|[fill=green]|\color[rgb]{1,0,0}\textbf{BU}%
 &|[fill=green]|\color[rgb]{0,0,0}\textbf{CO}%
 &|[fill=green]|\color[rgb]{0,0,0}\textbf{CO}%
 &|[fill=cyan]|\color[rgb]{1,0,0}\textbf{02}%
 &|[fill=green]|\color[rgb]{0,0,0}\textbf{CO}%
 &|[fill=green]|\color[rgb]{0,0,0}\textbf{CO}%
 &|[fill=green]|\color[rgb]{1,0,0}\textbf{BU}%
 &|[fill=green]|\color[gray]{0.75} FO%
 &|[fill=white]|\color[gray]{0.5}WA%
 &|[fill=green]|\color[gray]{0.75} FO%
\\
|[fill=green]|\color[gray]{0.75} FO%
 &|[fill=white]|\color[gray]{0.5}WA%
 &|[fill=green]|\color[gray]{0.75} FO%
 &|[fill=green]|\color[gray]{0.75} FO%
 &|[fill=green]|\color[rgb]{1,0,0}\textbf{BU}%
 &|[fill=green]|\color[rgb]{0,0,0}\textbf{CO}%
 &|[fill=green]|\color[rgb]{1,0,0}\textbf{BU}%
 &|[fill=green]|\color[rgb]{0,0,0}\textbf{CO}%
 &|[fill=green]|\color[rgb]{1,0,0}\textbf{BU}%
 &|[fill=green]|\color[gray]{0.75} FO%
 &|[fill=green]|\color[gray]{0.75} FO%
 &|[fill=white]|\color[gray]{0.5}WA%
 &|[fill=green]|\color[gray]{0.75} FO%
\\
|[fill=green]|\color[gray]{0.75} FO%
 &|[fill=white]|\color[gray]{0.5}WA%
 &|[fill=green]|\color[gray]{0.75} FO%
 &|[fill=green]|\color[gray]{0.75} FO%
 &|[fill=green]|\color[gray]{0.75} FO%
 &|[fill=green]|\color[rgb]{1,0,0}\textbf{BU}%
 &|[fill=green]|\color[gray]{0.75} FO%
 &|[fill=green]|\color[rgb]{1,0,0}\textbf{BU}%
 &|[fill=green]|\color[gray]{0.75} FO%
 &|[fill=green]|\color[gray]{0.75} FO%
 &|[fill=green]|\color[gray]{0.75} FO%
 &|[fill=white]|\color[gray]{0.5}WA%
 &|[fill=green]|\color[gray]{0.75} FO%
\\
|[fill=green]|\color[gray]{0.75} FO%
 &|[fill=white]|\color[gray]{0.5}WA%
 &|[fill=white]|\color[gray]{0.5}WA%
 &|[fill=white]|\color[gray]{0.5}WA%
 &|[fill=white]|\color[gray]{0.5}WA%
 &|[fill=white]|\color[gray]{0.5}WA%
 &|[fill=green]|\color[gray]{0.75} FO%
 &|[fill=white]|\color[gray]{0.5}WA%
 &|[fill=white]|\color[gray]{0.5}WA%
 &|[fill=white]|\color[gray]{0.5}WA%
 &|[fill=white]|\color[gray]{0.5}WA%
 &|[fill=white]|\color[gray]{0.5}WA%
 &|[fill=green]|\color[gray]{0.75} FO%
\\
|[fill=green]|\color[gray]{0.75} FO%
 &|[fill=green]|\color[gray]{0.75} FO%
 &|[fill=green]|\color[gray]{0.75} FO%
 &|[fill=green]|\color[gray]{0.75} FO%
 &|[fill=green]|\color[gray]{0.75} FO%
 &|[fill=green]|\color[gray]{0.75} FO%
 &|[fill=green]|\color[gray]{0.75} FO%
 &|[fill=green]|\color[gray]{0.75} FO%
 &|[fill=green]|\color[gray]{0.75} FO%
 &|[fill=green]|\color[gray]{0.75} FO%
 &|[fill=green]|\color[gray]{0.75} FO%
 &|[fill=green]|\color[gray]{0.75} FO%
 &|[fill=green]|\color[gray]{0.75} FO%
\\
};
\end{tikzpicture}\\
At time 6: Water spot (10|6):
\\
\begin{tikzpicture}
\tikzset{square matrix/.style={
matrix of nodes,
column sep=-\pgflinewidth, row sep=-\pgflinewidth,
nodes={draw,
minimum height=#1,
anchor=center,
text width=#1,
align=center,
inner sep=0pt
},
},
square matrix/.default=1.2cm
}
\matrix[square matrix=1.4em] {
|[fill=green]|\color[gray]{0.75} FO%
 &|[fill=green]|\color[gray]{0.75} FO%
 &|[fill=green]|\color[gray]{0.75} FO%
 &|[fill=green]|\color[gray]{0.75} FO%
 &|[fill=green]|\color[gray]{0.75} FO%
 &|[fill=green]|\color[gray]{0.75} FO%
 &|[fill=green]|\color[gray]{0.75} FO%
 &|[fill=green]|\color[gray]{0.75} FO%
 &|[fill=green]|\color[gray]{0.75} FO%
 &|[fill=green]|\color[gray]{0.75} FO%
 &|[fill=green]|\color[gray]{0.75} FO%
 &|[fill=green]|\color[gray]{0.75} FO%
 &|[fill=green]|\color[gray]{0.75} FO%
\\
|[fill=green]|\color[gray]{0.75} FO%
 &|[fill=white]|\color[gray]{0.5}WA%
 &|[fill=white]|\color[gray]{0.5}WA%
 &|[fill=white]|\color[gray]{0.5}WA%
 &|[fill=white]|\color[gray]{0.5}WA%
 &|[fill=white]|\color[gray]{0.5}WA%
 &|[fill=green]|\color[gray]{0.75} FO%
 &|[fill=white]|\color[gray]{0.5}WA%
 &|[fill=white]|\color[gray]{0.5}WA%
 &|[fill=white]|\color[gray]{0.5}WA%
 &|[fill=white]|\color[gray]{0.5}WA%
 &|[fill=white]|\color[gray]{0.5}WA%
 &|[fill=green]|\color[gray]{0.75} FO%
\\
|[fill=green]|\color[gray]{0.75} FO%
 &|[fill=white]|\color[gray]{0.5}WA%
 &|[fill=green]|\color[gray]{0.75} FO%
 &|[fill=green]|\color[gray]{0.75} FO%
 &|[fill=green]|\color[rgb]{1,0,0}\textbf{BU}%
 &|[fill=green]|\color[rgb]{0,0,0}\textbf{CO}%
 &|[fill=cyan]|\color[rgb]{1,0,0}\textbf{04}%
 &|[fill=green]|\color[rgb]{0,0,0}\textbf{CO}%
 &|[fill=green]|\color[rgb]{1,0,0}\textbf{BU}%
 &|[fill=green]|\color[gray]{0.75} FO%
 &|[fill=green]|\color[gray]{0.75} FO%
 &|[fill=white]|\color[gray]{0.5}WA%
 &|[fill=green]|\color[gray]{0.75} FO%
\\
|[fill=green]|\color[gray]{0.75} FO%
 &|[fill=white]|\color[gray]{0.5}WA%
 &|[fill=green]|\color[gray]{0.75} FO%
 &|[fill=green]|\color[rgb]{1,0,0}\textbf{BU}%
 &|[fill=green]|\color[rgb]{0,0,0}\textbf{CO}%
 &|[fill=green]|\color[rgb]{0,0,0}\textbf{CO}%
 &|[fill=green]|\color[rgb]{0,0,0}\textbf{CO}%
 &|[fill=green]|\color[rgb]{0,0,0}\textbf{CO}%
 &|[fill=green]|\color[rgb]{0,0,0}\textbf{CO}%
 &|[fill=green]|\color[rgb]{1,0,0}\textbf{BU}%
 &|[fill=green]|\color[gray]{0.75} FO%
 &|[fill=white]|\color[gray]{0.5}WA%
 &|[fill=green]|\color[gray]{0.75} FO%
\\
|[fill=green]|\color[gray]{0.75} FO%
 &|[fill=white]|\color[gray]{0.5}WA%
 &|[fill=green]|\color[rgb]{1,0,0}\textbf{BU}%
 &|[fill=green]|\color[rgb]{0,0,0}\textbf{CO}%
 &|[fill=green]|\color[rgb]{0,0,0}\textbf{CO}%
 &|[fill=green]|\color[rgb]{0,0,0}\textbf{CO}%
 &|[fill=green]|\color[rgb]{0,0,0}\textbf{CO}%
 &|[fill=green]|\color[rgb]{0,0,0}\textbf{CO}%
 &|[fill=green]|\color[rgb]{0,0,0}\textbf{CO}%
 &|[fill=green]|\color[rgb]{0,0,0}\textbf{CO}%
 &|[fill=green]|\color[rgb]{1,0,0}\textbf{BU}%
 &|[fill=white]|\color[gray]{0.5}WA%
 &|[fill=green]|\color[gray]{0.75} FO%
\\
|[fill=green]|\color[gray]{0.75} FO%
 &|[fill=white]|\color[gray]{0.5}WA%
 &|[fill=green]|\color[rgb]{0,0,0}\textbf{CO}%
 &|[fill=green]|\color[rgb]{0,0,0}\textbf{CO}%
 &|[fill=green]|\color[rgb]{0,0,0}\textbf{CO}%
 &|[fill=green]|\color[rgb]{0,0,0}\textbf{CO}%
 &|[fill=green]|\color[rgb]{0,0,0}\textbf{CO}%
 &|[fill=green]|\color[rgb]{0,0,0}\textbf{CO}%
 &|[fill=green]|\color[rgb]{0,0,0}\textbf{CO}%
 &|[fill=green]|\color[rgb]{0,0,0}\textbf{CO}%
 &|[fill=green]|\color[rgb]{0,0,0}\textbf{CO}%
 &|[fill=white]|\color[gray]{0.5}WA%
 &|[fill=green]|\color[gray]{0.75} FO%
\\
|[fill=green]|\color[gray]{0.75} FO%
 &|[fill=green]|\color[gray]{0.75} FO%
 &|[fill=green]|\color[rgb]{1,0,0}\textbf{BU}%
 &|[fill=cyan]|\color[rgb]{1,0,0}\textbf{03}%
 &|[fill=green]|\color[rgb]{0,0,0}\textbf{CO}%
 &|[fill=green]|\color[rgb]{0,0,0}\textbf{CO}%
 &|[fill=green]|\color[rgb]{0,0,0}\textbf{CO}%
 &|[fill=cyan]|\color[rgb]{1,0,0}\textbf{01}%
 &|[fill=green]|\color[rgb]{0,0,0}\textbf{CO}%
 &|[fill=green]|\color[rgb]{0,0,0}\textbf{CO}%
 &|[fill=cyan]|\color[rgb]{1,0,0}\textbf{06}%
 &|[fill=green]|\color[gray]{0.75} FO%
 &|[fill=green]|\color[gray]{0.75} FO%
\\
|[fill=green]|\color[gray]{0.75} FO%
 &|[fill=white]|\color[gray]{0.5}WA%
 &|[fill=green]|\color[rgb]{0,0,0}\textbf{CO}%
 &|[fill=green]|\color[rgb]{0,0,0}\textbf{CO}%
 &|[fill=green]|\color[rgb]{0,0,0}\textbf{CO}%
 &|[fill=green]|\color[rgb]{0,0,0}\textbf{CO}%
 &|[fill=green]|\color[rgb]{0,0,0}\textbf{CO}%
 &|[fill=green]|\color[rgb]{0,0,0}\textbf{CO}%
 &|[fill=green]|\color[rgb]{0,0,0}\textbf{CO}%
 &|[fill=green]|\color[rgb]{0,0,0}\textbf{CO}%
 &|[fill=cyan]|\color[rgb]{1,0,0}\textbf{05}%
 &|[fill=white]|\color[gray]{0.5}WA%
 &|[fill=green]|\color[gray]{0.75} FO%
\\
|[fill=green]|\color[gray]{0.75} FO%
 &|[fill=white]|\color[gray]{0.5}WA%
 &|[fill=green]|\color[rgb]{1,0,0}\textbf{BU}%
 &|[fill=green]|\color[rgb]{0,0,0}\textbf{CO}%
 &|[fill=green]|\color[rgb]{0,0,0}\textbf{CO}%
 &|[fill=green]|\color[rgb]{0,0,0}\textbf{CO}%
 &|[fill=cyan]|\color[rgb]{1,0,0}\textbf{02}%
 &|[fill=green]|\color[rgb]{0,0,0}\textbf{CO}%
 &|[fill=green]|\color[rgb]{0,0,0}\textbf{CO}%
 &|[fill=green]|\color[rgb]{0,0,0}\textbf{CO}%
 &|[fill=green]|\color[rgb]{1,0,0}\textbf{BU}%
 &|[fill=white]|\color[gray]{0.5}WA%
 &|[fill=green]|\color[gray]{0.75} FO%
\\
|[fill=green]|\color[gray]{0.75} FO%
 &|[fill=white]|\color[gray]{0.5}WA%
 &|[fill=green]|\color[gray]{0.75} FO%
 &|[fill=green]|\color[rgb]{1,0,0}\textbf{BU}%
 &|[fill=green]|\color[rgb]{0,0,0}\textbf{CO}%
 &|[fill=green]|\color[rgb]{0,0,0}\textbf{CO}%
 &|[fill=green]|\color[rgb]{0,0,0}\textbf{CO}%
 &|[fill=green]|\color[rgb]{0,0,0}\textbf{CO}%
 &|[fill=green]|\color[rgb]{0,0,0}\textbf{CO}%
 &|[fill=green]|\color[rgb]{1,0,0}\textbf{BU}%
 &|[fill=green]|\color[gray]{0.75} FO%
 &|[fill=white]|\color[gray]{0.5}WA%
 &|[fill=green]|\color[gray]{0.75} FO%
\\
|[fill=green]|\color[gray]{0.75} FO%
 &|[fill=white]|\color[gray]{0.5}WA%
 &|[fill=green]|\color[gray]{0.75} FO%
 &|[fill=green]|\color[gray]{0.75} FO%
 &|[fill=green]|\color[rgb]{1,0,0}\textbf{BU}%
 &|[fill=green]|\color[rgb]{0,0,0}\textbf{CO}%
 &|[fill=green]|\color[rgb]{1,0,0}\textbf{BU}%
 &|[fill=green]|\color[rgb]{0,0,0}\textbf{CO}%
 &|[fill=green]|\color[rgb]{1,0,0}\textbf{BU}%
 &|[fill=green]|\color[gray]{0.75} FO%
 &|[fill=green]|\color[gray]{0.75} FO%
 &|[fill=white]|\color[gray]{0.5}WA%
 &|[fill=green]|\color[gray]{0.75} FO%
\\
|[fill=green]|\color[gray]{0.75} FO%
 &|[fill=white]|\color[gray]{0.5}WA%
 &|[fill=white]|\color[gray]{0.5}WA%
 &|[fill=white]|\color[gray]{0.5}WA%
 &|[fill=white]|\color[gray]{0.5}WA%
 &|[fill=white]|\color[gray]{0.5}WA%
 &|[fill=green]|\color[gray]{0.75} FO%
 &|[fill=white]|\color[gray]{0.5}WA%
 &|[fill=white]|\color[gray]{0.5}WA%
 &|[fill=white]|\color[gray]{0.5}WA%
 &|[fill=white]|\color[gray]{0.5}WA%
 &|[fill=white]|\color[gray]{0.5}WA%
 &|[fill=green]|\color[gray]{0.75} FO%
\\
|[fill=green]|\color[gray]{0.75} FO%
 &|[fill=green]|\color[gray]{0.75} FO%
 &|[fill=green]|\color[gray]{0.75} FO%
 &|[fill=green]|\color[gray]{0.75} FO%
 &|[fill=green]|\color[gray]{0.75} FO%
 &|[fill=green]|\color[gray]{0.75} FO%
 &|[fill=green]|\color[gray]{0.75} FO%
 &|[fill=green]|\color[gray]{0.75} FO%
 &|[fill=green]|\color[gray]{0.75} FO%
 &|[fill=green]|\color[gray]{0.75} FO%
 &|[fill=green]|\color[gray]{0.75} FO%
 &|[fill=green]|\color[gray]{0.75} FO%
 &|[fill=green]|\color[gray]{0.75} FO%
\\
};
\end{tikzpicture}\\
At time 7: Water spot (6|11):
\\
\begin{tikzpicture}
\tikzset{square matrix/.style={
matrix of nodes,
column sep=-\pgflinewidth, row sep=-\pgflinewidth,
nodes={draw,
minimum height=#1,
anchor=center,
text width=#1,
align=center,
inner sep=0pt
},
},
square matrix/.default=1.2cm
}
\matrix[square matrix=1.4em] {
|[fill=green]|\color[gray]{0.75} FO%
 &|[fill=green]|\color[gray]{0.75} FO%
 &|[fill=green]|\color[gray]{0.75} FO%
 &|[fill=green]|\color[gray]{0.75} FO%
 &|[fill=green]|\color[gray]{0.75} FO%
 &|[fill=green]|\color[gray]{0.75} FO%
 &|[fill=green]|\color[gray]{0.75} FO%
 &|[fill=green]|\color[gray]{0.75} FO%
 &|[fill=green]|\color[gray]{0.75} FO%
 &|[fill=green]|\color[gray]{0.75} FO%
 &|[fill=green]|\color[gray]{0.75} FO%
 &|[fill=green]|\color[gray]{0.75} FO%
 &|[fill=green]|\color[gray]{0.75} FO%
\\
|[fill=green]|\color[gray]{0.75} FO%
 &|[fill=white]|\color[gray]{0.5}WA%
 &|[fill=white]|\color[gray]{0.5}WA%
 &|[fill=white]|\color[gray]{0.5}WA%
 &|[fill=white]|\color[gray]{0.5}WA%
 &|[fill=white]|\color[gray]{0.5}WA%
 &|[fill=green]|\color[gray]{0.75} FO%
 &|[fill=white]|\color[gray]{0.5}WA%
 &|[fill=white]|\color[gray]{0.5}WA%
 &|[fill=white]|\color[gray]{0.5}WA%
 &|[fill=white]|\color[gray]{0.5}WA%
 &|[fill=white]|\color[gray]{0.5}WA%
 &|[fill=green]|\color[gray]{0.75} FO%
\\
|[fill=green]|\color[gray]{0.75} FO%
 &|[fill=white]|\color[gray]{0.5}WA%
 &|[fill=green]|\color[gray]{0.75} FO%
 &|[fill=green]|\color[rgb]{1,0,0}\textbf{BU}%
 &|[fill=green]|\color[rgb]{0,0,0}\textbf{CO}%
 &|[fill=green]|\color[rgb]{0,0,0}\textbf{CO}%
 &|[fill=cyan]|\color[rgb]{1,0,0}\textbf{04}%
 &|[fill=green]|\color[rgb]{0,0,0}\textbf{CO}%
 &|[fill=green]|\color[rgb]{0,0,0}\textbf{CO}%
 &|[fill=green]|\color[rgb]{1,0,0}\textbf{BU}%
 &|[fill=green]|\color[gray]{0.75} FO%
 &|[fill=white]|\color[gray]{0.5}WA%
 &|[fill=green]|\color[gray]{0.75} FO%
\\
|[fill=green]|\color[gray]{0.75} FO%
 &|[fill=white]|\color[gray]{0.5}WA%
 &|[fill=green]|\color[rgb]{1,0,0}\textbf{BU}%
 &|[fill=green]|\color[rgb]{0,0,0}\textbf{CO}%
 &|[fill=green]|\color[rgb]{0,0,0}\textbf{CO}%
 &|[fill=green]|\color[rgb]{0,0,0}\textbf{CO}%
 &|[fill=green]|\color[rgb]{0,0,0}\textbf{CO}%
 &|[fill=green]|\color[rgb]{0,0,0}\textbf{CO}%
 &|[fill=green]|\color[rgb]{0,0,0}\textbf{CO}%
 &|[fill=green]|\color[rgb]{0,0,0}\textbf{CO}%
 &|[fill=green]|\color[rgb]{1,0,0}\textbf{BU}%
 &|[fill=white]|\color[gray]{0.5}WA%
 &|[fill=green]|\color[gray]{0.75} FO%
\\
|[fill=green]|\color[gray]{0.75} FO%
 &|[fill=white]|\color[gray]{0.5}WA%
 &|[fill=green]|\color[rgb]{0,0,0}\textbf{CO}%
 &|[fill=green]|\color[rgb]{0,0,0}\textbf{CO}%
 &|[fill=green]|\color[rgb]{0,0,0}\textbf{CO}%
 &|[fill=green]|\color[rgb]{0,0,0}\textbf{CO}%
 &|[fill=green]|\color[rgb]{0,0,0}\textbf{CO}%
 &|[fill=green]|\color[rgb]{0,0,0}\textbf{CO}%
 &|[fill=green]|\color[rgb]{0,0,0}\textbf{CO}%
 &|[fill=green]|\color[rgb]{0,0,0}\textbf{CO}%
 &|[fill=green]|\color[rgb]{0,0,0}\textbf{CO}%
 &|[fill=white]|\color[gray]{0.5}WA%
 &|[fill=green]|\color[gray]{0.75} FO%
\\
|[fill=green]|\color[gray]{0.75} FO%
 &|[fill=white]|\color[gray]{0.5}WA%
 &|[fill=green]|\color[rgb]{0,0,0}\textbf{CO}%
 &|[fill=green]|\color[rgb]{0,0,0}\textbf{CO}%
 &|[fill=green]|\color[rgb]{0,0,0}\textbf{CO}%
 &|[fill=green]|\color[rgb]{0,0,0}\textbf{CO}%
 &|[fill=green]|\color[rgb]{0,0,0}\textbf{CO}%
 &|[fill=green]|\color[rgb]{0,0,0}\textbf{CO}%
 &|[fill=green]|\color[rgb]{0,0,0}\textbf{CO}%
 &|[fill=green]|\color[rgb]{0,0,0}\textbf{CO}%
 &|[fill=green]|\color[rgb]{0,0,0}\textbf{CO}%
 &|[fill=white]|\color[gray]{0.5}WA%
 &|[fill=green]|\color[gray]{0.75} FO%
\\
|[fill=green]|\color[gray]{0.75} FO%
 &|[fill=green]|\color[rgb]{1,0,0}\textbf{BU}%
 &|[fill=green]|\color[rgb]{0,0,0}\textbf{CO}%
 &|[fill=cyan]|\color[rgb]{1,0,0}\textbf{03}%
 &|[fill=green]|\color[rgb]{0,0,0}\textbf{CO}%
 &|[fill=green]|\color[rgb]{0,0,0}\textbf{CO}%
 &|[fill=green]|\color[rgb]{0,0,0}\textbf{CO}%
 &|[fill=cyan]|\color[rgb]{1,0,0}\textbf{01}%
 &|[fill=green]|\color[rgb]{0,0,0}\textbf{CO}%
 &|[fill=green]|\color[rgb]{0,0,0}\textbf{CO}%
 &|[fill=cyan]|\color[rgb]{1,0,0}\textbf{06}%
 &|[fill=green]|\color[gray]{0.75} FO%
 &|[fill=green]|\color[gray]{0.75} FO%
\\
|[fill=green]|\color[gray]{0.75} FO%
 &|[fill=white]|\color[gray]{0.5}WA%
 &|[fill=green]|\color[rgb]{0,0,0}\textbf{CO}%
 &|[fill=green]|\color[rgb]{0,0,0}\textbf{CO}%
 &|[fill=green]|\color[rgb]{0,0,0}\textbf{CO}%
 &|[fill=green]|\color[rgb]{0,0,0}\textbf{CO}%
 &|[fill=green]|\color[rgb]{0,0,0}\textbf{CO}%
 &|[fill=green]|\color[rgb]{0,0,0}\textbf{CO}%
 &|[fill=green]|\color[rgb]{0,0,0}\textbf{CO}%
 &|[fill=green]|\color[rgb]{0,0,0}\textbf{CO}%
 &|[fill=cyan]|\color[rgb]{1,0,0}\textbf{05}%
 &|[fill=white]|\color[gray]{0.5}WA%
 &|[fill=green]|\color[gray]{0.75} FO%
\\
|[fill=green]|\color[gray]{0.75} FO%
 &|[fill=white]|\color[gray]{0.5}WA%
 &|[fill=green]|\color[rgb]{0,0,0}\textbf{CO}%
 &|[fill=green]|\color[rgb]{0,0,0}\textbf{CO}%
 &|[fill=green]|\color[rgb]{0,0,0}\textbf{CO}%
 &|[fill=green]|\color[rgb]{0,0,0}\textbf{CO}%
 &|[fill=cyan]|\color[rgb]{1,0,0}\textbf{02}%
 &|[fill=green]|\color[rgb]{0,0,0}\textbf{CO}%
 &|[fill=green]|\color[rgb]{0,0,0}\textbf{CO}%
 &|[fill=green]|\color[rgb]{0,0,0}\textbf{CO}%
 &|[fill=green]|\color[rgb]{0,0,0}\textbf{CO}%
 &|[fill=white]|\color[gray]{0.5}WA%
 &|[fill=green]|\color[gray]{0.75} FO%
\\
|[fill=green]|\color[gray]{0.75} FO%
 &|[fill=white]|\color[gray]{0.5}WA%
 &|[fill=green]|\color[rgb]{1,0,0}\textbf{BU}%
 &|[fill=green]|\color[rgb]{0,0,0}\textbf{CO}%
 &|[fill=green]|\color[rgb]{0,0,0}\textbf{CO}%
 &|[fill=green]|\color[rgb]{0,0,0}\textbf{CO}%
 &|[fill=green]|\color[rgb]{0,0,0}\textbf{CO}%
 &|[fill=green]|\color[rgb]{0,0,0}\textbf{CO}%
 &|[fill=green]|\color[rgb]{0,0,0}\textbf{CO}%
 &|[fill=green]|\color[rgb]{0,0,0}\textbf{CO}%
 &|[fill=green]|\color[rgb]{1,0,0}\textbf{BU}%
 &|[fill=white]|\color[gray]{0.5}WA%
 &|[fill=green]|\color[gray]{0.75} FO%
\\
|[fill=green]|\color[gray]{0.75} FO%
 &|[fill=white]|\color[gray]{0.5}WA%
 &|[fill=green]|\color[gray]{0.75} FO%
 &|[fill=green]|\color[rgb]{1,0,0}\textbf{BU}%
 &|[fill=green]|\color[rgb]{0,0,0}\textbf{CO}%
 &|[fill=green]|\color[rgb]{0,0,0}\textbf{CO}%
 &|[fill=green]|\color[rgb]{0,0,0}\textbf{CO}%
 &|[fill=green]|\color[rgb]{0,0,0}\textbf{CO}%
 &|[fill=green]|\color[rgb]{0,0,0}\textbf{CO}%
 &|[fill=green]|\color[rgb]{1,0,0}\textbf{BU}%
 &|[fill=green]|\color[gray]{0.75} FO%
 &|[fill=white]|\color[gray]{0.5}WA%
 &|[fill=green]|\color[gray]{0.75} FO%
\\
|[fill=green]|\color[gray]{0.75} FO%
 &|[fill=white]|\color[gray]{0.5}WA%
 &|[fill=white]|\color[gray]{0.5}WA%
 &|[fill=white]|\color[gray]{0.5}WA%
 &|[fill=white]|\color[gray]{0.5}WA%
 &|[fill=white]|\color[gray]{0.5}WA%
 &|[fill=cyan]|\color[rgb]{1,0,0}\textbf{07}%
 &|[fill=white]|\color[gray]{0.5}WA%
 &|[fill=white]|\color[gray]{0.5}WA%
 &|[fill=white]|\color[gray]{0.5}WA%
 &|[fill=white]|\color[gray]{0.5}WA%
 &|[fill=white]|\color[gray]{0.5}WA%
 &|[fill=green]|\color[gray]{0.75} FO%
\\
|[fill=green]|\color[gray]{0.75} FO%
 &|[fill=green]|\color[gray]{0.75} FO%
 &|[fill=green]|\color[gray]{0.75} FO%
 &|[fill=green]|\color[gray]{0.75} FO%
 &|[fill=green]|\color[gray]{0.75} FO%
 &|[fill=green]|\color[gray]{0.75} FO%
 &|[fill=green]|\color[gray]{0.75} FO%
 &|[fill=green]|\color[gray]{0.75} FO%
 &|[fill=green]|\color[gray]{0.75} FO%
 &|[fill=green]|\color[gray]{0.75} FO%
 &|[fill=green]|\color[gray]{0.75} FO%
 &|[fill=green]|\color[gray]{0.75} FO%
 &|[fill=green]|\color[gray]{0.75} FO%
\\
};
\end{tikzpicture}\\
At time 8: Water spot (0|6):
\\
\begin{tikzpicture}
\tikzset{square matrix/.style={
matrix of nodes,
column sep=-\pgflinewidth, row sep=-\pgflinewidth,
nodes={draw,
minimum height=#1,
anchor=center,
text width=#1,
align=center,
inner sep=0pt
},
},
square matrix/.default=1.2cm
}
\matrix[square matrix=1.4em] {
|[fill=green]|\color[gray]{0.75} FO%
 &|[fill=green]|\color[gray]{0.75} FO%
 &|[fill=green]|\color[gray]{0.75} FO%
 &|[fill=green]|\color[gray]{0.75} FO%
 &|[fill=green]|\color[gray]{0.75} FO%
 &|[fill=green]|\color[gray]{0.75} FO%
 &|[fill=green]|\color[gray]{0.75} FO%
 &|[fill=green]|\color[gray]{0.75} FO%
 &|[fill=green]|\color[gray]{0.75} FO%
 &|[fill=green]|\color[gray]{0.75} FO%
 &|[fill=green]|\color[gray]{0.75} FO%
 &|[fill=green]|\color[gray]{0.75} FO%
 &|[fill=green]|\color[gray]{0.75} FO%
\\
|[fill=green]|\color[gray]{0.75} FO%
 &|[fill=white]|\color[gray]{0.5}WA%
 &|[fill=white]|\color[gray]{0.5}WA%
 &|[fill=white]|\color[gray]{0.5}WA%
 &|[fill=white]|\color[gray]{0.5}WA%
 &|[fill=white]|\color[gray]{0.5}WA%
 &|[fill=green]|\color[gray]{0.75} FO%
 &|[fill=white]|\color[gray]{0.5}WA%
 &|[fill=white]|\color[gray]{0.5}WA%
 &|[fill=white]|\color[gray]{0.5}WA%
 &|[fill=white]|\color[gray]{0.5}WA%
 &|[fill=white]|\color[gray]{0.5}WA%
 &|[fill=green]|\color[gray]{0.75} FO%
\\
|[fill=green]|\color[gray]{0.75} FO%
 &|[fill=white]|\color[gray]{0.5}WA%
 &|[fill=green]|\color[rgb]{1,0,0}\textbf{BU}%
 &|[fill=green]|\color[rgb]{0,0,0}\textbf{CO}%
 &|[fill=green]|\color[rgb]{0,0,0}\textbf{CO}%
 &|[fill=green]|\color[rgb]{0,0,0}\textbf{CO}%
 &|[fill=cyan]|\color[rgb]{1,0,0}\textbf{04}%
 &|[fill=green]|\color[rgb]{0,0,0}\textbf{CO}%
 &|[fill=green]|\color[rgb]{0,0,0}\textbf{CO}%
 &|[fill=green]|\color[rgb]{0,0,0}\textbf{CO}%
 &|[fill=green]|\color[rgb]{1,0,0}\textbf{BU}%
 &|[fill=white]|\color[gray]{0.5}WA%
 &|[fill=green]|\color[gray]{0.75} FO%
\\
|[fill=green]|\color[gray]{0.75} FO%
 &|[fill=white]|\color[gray]{0.5}WA%
 &|[fill=green]|\color[rgb]{0,0,0}\textbf{CO}%
 &|[fill=green]|\color[rgb]{0,0,0}\textbf{CO}%
 &|[fill=green]|\color[rgb]{0,0,0}\textbf{CO}%
 &|[fill=green]|\color[rgb]{0,0,0}\textbf{CO}%
 &|[fill=green]|\color[rgb]{0,0,0}\textbf{CO}%
 &|[fill=green]|\color[rgb]{0,0,0}\textbf{CO}%
 &|[fill=green]|\color[rgb]{0,0,0}\textbf{CO}%
 &|[fill=green]|\color[rgb]{0,0,0}\textbf{CO}%
 &|[fill=green]|\color[rgb]{0,0,0}\textbf{CO}%
 &|[fill=white]|\color[gray]{0.5}WA%
 &|[fill=green]|\color[gray]{0.75} FO%
\\
|[fill=green]|\color[gray]{0.75} FO%
 &|[fill=white]|\color[gray]{0.5}WA%
 &|[fill=green]|\color[rgb]{0,0,0}\textbf{CO}%
 &|[fill=green]|\color[rgb]{0,0,0}\textbf{CO}%
 &|[fill=green]|\color[rgb]{0,0,0}\textbf{CO}%
 &|[fill=green]|\color[rgb]{0,0,0}\textbf{CO}%
 &|[fill=green]|\color[rgb]{0,0,0}\textbf{CO}%
 &|[fill=green]|\color[rgb]{0,0,0}\textbf{CO}%
 &|[fill=green]|\color[rgb]{0,0,0}\textbf{CO}%
 &|[fill=green]|\color[rgb]{0,0,0}\textbf{CO}%
 &|[fill=green]|\color[rgb]{0,0,0}\textbf{CO}%
 &|[fill=white]|\color[gray]{0.5}WA%
 &|[fill=green]|\color[gray]{0.75} FO%
\\
|[fill=green]|\color[gray]{0.75} FO%
 &|[fill=white]|\color[gray]{0.5}WA%
 &|[fill=green]|\color[rgb]{0,0,0}\textbf{CO}%
 &|[fill=green]|\color[rgb]{0,0,0}\textbf{CO}%
 &|[fill=green]|\color[rgb]{0,0,0}\textbf{CO}%
 &|[fill=green]|\color[rgb]{0,0,0}\textbf{CO}%
 &|[fill=green]|\color[rgb]{0,0,0}\textbf{CO}%
 &|[fill=green]|\color[rgb]{0,0,0}\textbf{CO}%
 &|[fill=green]|\color[rgb]{0,0,0}\textbf{CO}%
 &|[fill=green]|\color[rgb]{0,0,0}\textbf{CO}%
 &|[fill=green]|\color[rgb]{0,0,0}\textbf{CO}%
 &|[fill=white]|\color[gray]{0.5}WA%
 &|[fill=green]|\color[gray]{0.75} FO%
\\
|[fill=cyan]|\color[rgb]{1,0,0}\textbf{08}%
 &|[fill=green]|\color[rgb]{0,0,0}\textbf{CO}%
 &|[fill=green]|\color[rgb]{0,0,0}\textbf{CO}%
 &|[fill=cyan]|\color[rgb]{1,0,0}\textbf{03}%
 &|[fill=green]|\color[rgb]{0,0,0}\textbf{CO}%
 &|[fill=green]|\color[rgb]{0,0,0}\textbf{CO}%
 &|[fill=green]|\color[rgb]{0,0,0}\textbf{CO}%
 &|[fill=cyan]|\color[rgb]{1,0,0}\textbf{01}%
 &|[fill=green]|\color[rgb]{0,0,0}\textbf{CO}%
 &|[fill=green]|\color[rgb]{0,0,0}\textbf{CO}%
 &|[fill=cyan]|\color[rgb]{1,0,0}\textbf{06}%
 &|[fill=green]|\color[gray]{0.75} FO%
 &|[fill=green]|\color[gray]{0.75} FO%
\\
|[fill=green]|\color[gray]{0.75} FO%
 &|[fill=white]|\color[gray]{0.5}WA%
 &|[fill=green]|\color[rgb]{0,0,0}\textbf{CO}%
 &|[fill=green]|\color[rgb]{0,0,0}\textbf{CO}%
 &|[fill=green]|\color[rgb]{0,0,0}\textbf{CO}%
 &|[fill=green]|\color[rgb]{0,0,0}\textbf{CO}%
 &|[fill=green]|\color[rgb]{0,0,0}\textbf{CO}%
 &|[fill=green]|\color[rgb]{0,0,0}\textbf{CO}%
 &|[fill=green]|\color[rgb]{0,0,0}\textbf{CO}%
 &|[fill=green]|\color[rgb]{0,0,0}\textbf{CO}%
 &|[fill=cyan]|\color[rgb]{1,0,0}\textbf{05}%
 &|[fill=white]|\color[gray]{0.5}WA%
 &|[fill=green]|\color[gray]{0.75} FO%
\\
|[fill=green]|\color[gray]{0.75} FO%
 &|[fill=white]|\color[gray]{0.5}WA%
 &|[fill=green]|\color[rgb]{0,0,0}\textbf{CO}%
 &|[fill=green]|\color[rgb]{0,0,0}\textbf{CO}%
 &|[fill=green]|\color[rgb]{0,0,0}\textbf{CO}%
 &|[fill=green]|\color[rgb]{0,0,0}\textbf{CO}%
 &|[fill=cyan]|\color[rgb]{1,0,0}\textbf{02}%
 &|[fill=green]|\color[rgb]{0,0,0}\textbf{CO}%
 &|[fill=green]|\color[rgb]{0,0,0}\textbf{CO}%
 &|[fill=green]|\color[rgb]{0,0,0}\textbf{CO}%
 &|[fill=green]|\color[rgb]{0,0,0}\textbf{CO}%
 &|[fill=white]|\color[gray]{0.5}WA%
 &|[fill=green]|\color[gray]{0.75} FO%
\\
|[fill=green]|\color[gray]{0.75} FO%
 &|[fill=white]|\color[gray]{0.5}WA%
 &|[fill=green]|\color[rgb]{0,0,0}\textbf{CO}%
 &|[fill=green]|\color[rgb]{0,0,0}\textbf{CO}%
 &|[fill=green]|\color[rgb]{0,0,0}\textbf{CO}%
 &|[fill=green]|\color[rgb]{0,0,0}\textbf{CO}%
 &|[fill=green]|\color[rgb]{0,0,0}\textbf{CO}%
 &|[fill=green]|\color[rgb]{0,0,0}\textbf{CO}%
 &|[fill=green]|\color[rgb]{0,0,0}\textbf{CO}%
 &|[fill=green]|\color[rgb]{0,0,0}\textbf{CO}%
 &|[fill=green]|\color[rgb]{0,0,0}\textbf{CO}%
 &|[fill=white]|\color[gray]{0.5}WA%
 &|[fill=green]|\color[gray]{0.75} FO%
\\
|[fill=green]|\color[gray]{0.75} FO%
 &|[fill=white]|\color[gray]{0.5}WA%
 &|[fill=green]|\color[rgb]{1,0,0}\textbf{BU}%
 &|[fill=green]|\color[rgb]{0,0,0}\textbf{CO}%
 &|[fill=green]|\color[rgb]{0,0,0}\textbf{CO}%
 &|[fill=green]|\color[rgb]{0,0,0}\textbf{CO}%
 &|[fill=green]|\color[rgb]{0,0,0}\textbf{CO}%
 &|[fill=green]|\color[rgb]{0,0,0}\textbf{CO}%
 &|[fill=green]|\color[rgb]{0,0,0}\textbf{CO}%
 &|[fill=green]|\color[rgb]{0,0,0}\textbf{CO}%
 &|[fill=green]|\color[rgb]{1,0,0}\textbf{BU}%
 &|[fill=white]|\color[gray]{0.5}WA%
 &|[fill=green]|\color[gray]{0.75} FO%
\\
|[fill=green]|\color[gray]{0.75} FO%
 &|[fill=white]|\color[gray]{0.5}WA%
 &|[fill=white]|\color[gray]{0.5}WA%
 &|[fill=white]|\color[gray]{0.5}WA%
 &|[fill=white]|\color[gray]{0.5}WA%
 &|[fill=white]|\color[gray]{0.5}WA%
 &|[fill=cyan]|\color[rgb]{1,0,0}\textbf{07}%
 &|[fill=white]|\color[gray]{0.5}WA%
 &|[fill=white]|\color[gray]{0.5}WA%
 &|[fill=white]|\color[gray]{0.5}WA%
 &|[fill=white]|\color[gray]{0.5}WA%
 &|[fill=white]|\color[gray]{0.5}WA%
 &|[fill=green]|\color[gray]{0.75} FO%
\\
|[fill=green]|\color[gray]{0.75} FO%
 &|[fill=green]|\color[gray]{0.75} FO%
 &|[fill=green]|\color[gray]{0.75} FO%
 &|[fill=green]|\color[gray]{0.75} FO%
 &|[fill=green]|\color[gray]{0.75} FO%
 &|[fill=green]|\color[gray]{0.75} FO%
 &|[fill=green]|\color[gray]{0.75} FO%
 &|[fill=green]|\color[gray]{0.75} FO%
 &|[fill=green]|\color[gray]{0.75} FO%
 &|[fill=green]|\color[gray]{0.75} FO%
 &|[fill=green]|\color[gray]{0.75} FO%
 &|[fill=green]|\color[gray]{0.75} FO%
 &|[fill=green]|\color[gray]{0.75} FO%
\\
};
\end{tikzpicture}\\
And you'll find 76 pieces of coal and 8 pieces of watered coal:
\\
\begin{tikzpicture}
\tikzset{square matrix/.style={
matrix of nodes,
column sep=-\pgflinewidth, row sep=-\pgflinewidth,
nodes={draw,
minimum height=#1,
anchor=center,
text width=#1,
align=center,
inner sep=0pt
},
},
square matrix/.default=1.2cm
}
\matrix[square matrix=1.4em] {
|[fill=green]|\color[gray]{0.75} FO%
 &|[fill=green]|\color[gray]{0.75} FO%
 &|[fill=green]|\color[gray]{0.75} FO%
 &|[fill=green]|\color[gray]{0.75} FO%
 &|[fill=green]|\color[gray]{0.75} FO%
 &|[fill=green]|\color[gray]{0.75} FO%
 &|[fill=green]|\color[gray]{0.75} FO%
 &|[fill=green]|\color[gray]{0.75} FO%
 &|[fill=green]|\color[gray]{0.75} FO%
 &|[fill=green]|\color[gray]{0.75} FO%
 &|[fill=green]|\color[gray]{0.75} FO%
 &|[fill=green]|\color[gray]{0.75} FO%
 &|[fill=green]|\color[gray]{0.75} FO%
\\
|[fill=green]|\color[gray]{0.75} FO%
 &|[fill=white]|\color[gray]{0.5}WA%
 &|[fill=white]|\color[gray]{0.5}WA%
 &|[fill=white]|\color[gray]{0.5}WA%
 &|[fill=white]|\color[gray]{0.5}WA%
 &|[fill=white]|\color[gray]{0.5}WA%
 &|[fill=green]|\color[gray]{0.75} FO%
 &|[fill=white]|\color[gray]{0.5}WA%
 &|[fill=white]|\color[gray]{0.5}WA%
 &|[fill=white]|\color[gray]{0.5}WA%
 &|[fill=white]|\color[gray]{0.5}WA%
 &|[fill=white]|\color[gray]{0.5}WA%
 &|[fill=green]|\color[gray]{0.75} FO%
\\
|[fill=green]|\color[gray]{0.75} FO%
 &|[fill=white]|\color[gray]{0.5}WA%
 &|[fill=green]|\color[rgb]{0,0,0}\textbf{CO}%
 &|[fill=green]|\color[rgb]{0,0,0}\textbf{CO}%
 &|[fill=green]|\color[rgb]{0,0,0}\textbf{CO}%
 &|[fill=green]|\color[rgb]{0,0,0}\textbf{CO}%
 &|[fill=cyan]|\color[rgb]{1,0,0}\textbf{04}%
 &|[fill=green]|\color[rgb]{0,0,0}\textbf{CO}%
 &|[fill=green]|\color[rgb]{0,0,0}\textbf{CO}%
 &|[fill=green]|\color[rgb]{0,0,0}\textbf{CO}%
 &|[fill=green]|\color[rgb]{0,0,0}\textbf{CO}%
 &|[fill=white]|\color[gray]{0.5}WA%
 &|[fill=green]|\color[gray]{0.75} FO%
\\
|[fill=green]|\color[gray]{0.75} FO%
 &|[fill=white]|\color[gray]{0.5}WA%
 &|[fill=green]|\color[rgb]{0,0,0}\textbf{CO}%
 &|[fill=green]|\color[rgb]{0,0,0}\textbf{CO}%
 &|[fill=green]|\color[rgb]{0,0,0}\textbf{CO}%
 &|[fill=green]|\color[rgb]{0,0,0}\textbf{CO}%
 &|[fill=green]|\color[rgb]{0,0,0}\textbf{CO}%
 &|[fill=green]|\color[rgb]{0,0,0}\textbf{CO}%
 &|[fill=green]|\color[rgb]{0,0,0}\textbf{CO}%
 &|[fill=green]|\color[rgb]{0,0,0}\textbf{CO}%
 &|[fill=green]|\color[rgb]{0,0,0}\textbf{CO}%
 &|[fill=white]|\color[gray]{0.5}WA%
 &|[fill=green]|\color[gray]{0.75} FO%
\\
|[fill=green]|\color[gray]{0.75} FO%
 &|[fill=white]|\color[gray]{0.5}WA%
 &|[fill=green]|\color[rgb]{0,0,0}\textbf{CO}%
 &|[fill=green]|\color[rgb]{0,0,0}\textbf{CO}%
 &|[fill=green]|\color[rgb]{0,0,0}\textbf{CO}%
 &|[fill=green]|\color[rgb]{0,0,0}\textbf{CO}%
 &|[fill=green]|\color[rgb]{0,0,0}\textbf{CO}%
 &|[fill=green]|\color[rgb]{0,0,0}\textbf{CO}%
 &|[fill=green]|\color[rgb]{0,0,0}\textbf{CO}%
 &|[fill=green]|\color[rgb]{0,0,0}\textbf{CO}%
 &|[fill=green]|\color[rgb]{0,0,0}\textbf{CO}%
 &|[fill=white]|\color[gray]{0.5}WA%
 &|[fill=green]|\color[gray]{0.75} FO%
\\
|[fill=green]|\color[gray]{0.75} FO%
 &|[fill=white]|\color[gray]{0.5}WA%
 &|[fill=green]|\color[rgb]{0,0,0}\textbf{CO}%
 &|[fill=green]|\color[rgb]{0,0,0}\textbf{CO}%
 &|[fill=green]|\color[rgb]{0,0,0}\textbf{CO}%
 &|[fill=green]|\color[rgb]{0,0,0}\textbf{CO}%
 &|[fill=green]|\color[rgb]{0,0,0}\textbf{CO}%
 &|[fill=green]|\color[rgb]{0,0,0}\textbf{CO}%
 &|[fill=green]|\color[rgb]{0,0,0}\textbf{CO}%
 &|[fill=green]|\color[rgb]{0,0,0}\textbf{CO}%
 &|[fill=green]|\color[rgb]{0,0,0}\textbf{CO}%
 &|[fill=white]|\color[gray]{0.5}WA%
 &|[fill=green]|\color[gray]{0.75} FO%
\\
|[fill=cyan]|\color[rgb]{1,0,0}\textbf{08}%
 &|[fill=green]|\color[rgb]{0,0,0}\textbf{CO}%
 &|[fill=green]|\color[rgb]{0,0,0}\textbf{CO}%
 &|[fill=cyan]|\color[rgb]{1,0,0}\textbf{03}%
 &|[fill=green]|\color[rgb]{0,0,0}\textbf{CO}%
 &|[fill=green]|\color[rgb]{0,0,0}\textbf{CO}%
 &|[fill=green]|\color[rgb]{0,0,0}\textbf{CO}%
 &|[fill=cyan]|\color[rgb]{1,0,0}\textbf{01}%
 &|[fill=green]|\color[rgb]{0,0,0}\textbf{CO}%
 &|[fill=green]|\color[rgb]{0,0,0}\textbf{CO}%
 &|[fill=cyan]|\color[rgb]{1,0,0}\textbf{06}%
 &|[fill=green]|\color[gray]{0.75} FO%
 &|[fill=green]|\color[gray]{0.75} FO%
\\
|[fill=green]|\color[gray]{0.75} FO%
 &|[fill=white]|\color[gray]{0.5}WA%
 &|[fill=green]|\color[rgb]{0,0,0}\textbf{CO}%
 &|[fill=green]|\color[rgb]{0,0,0}\textbf{CO}%
 &|[fill=green]|\color[rgb]{0,0,0}\textbf{CO}%
 &|[fill=green]|\color[rgb]{0,0,0}\textbf{CO}%
 &|[fill=green]|\color[rgb]{0,0,0}\textbf{CO}%
 &|[fill=green]|\color[rgb]{0,0,0}\textbf{CO}%
 &|[fill=green]|\color[rgb]{0,0,0}\textbf{CO}%
 &|[fill=green]|\color[rgb]{0,0,0}\textbf{CO}%
 &|[fill=cyan]|\color[rgb]{1,0,0}\textbf{05}%
 &|[fill=white]|\color[gray]{0.5}WA%
 &|[fill=green]|\color[gray]{0.75} FO%
\\
|[fill=green]|\color[gray]{0.75} FO%
 &|[fill=white]|\color[gray]{0.5}WA%
 &|[fill=green]|\color[rgb]{0,0,0}\textbf{CO}%
 &|[fill=green]|\color[rgb]{0,0,0}\textbf{CO}%
 &|[fill=green]|\color[rgb]{0,0,0}\textbf{CO}%
 &|[fill=green]|\color[rgb]{0,0,0}\textbf{CO}%
 &|[fill=cyan]|\color[rgb]{1,0,0}\textbf{02}%
 &|[fill=green]|\color[rgb]{0,0,0}\textbf{CO}%
 &|[fill=green]|\color[rgb]{0,0,0}\textbf{CO}%
 &|[fill=green]|\color[rgb]{0,0,0}\textbf{CO}%
 &|[fill=green]|\color[rgb]{0,0,0}\textbf{CO}%
 &|[fill=white]|\color[gray]{0.5}WA%
 &|[fill=green]|\color[gray]{0.75} FO%
\\
|[fill=green]|\color[gray]{0.75} FO%
 &|[fill=white]|\color[gray]{0.5}WA%
 &|[fill=green]|\color[rgb]{0,0,0}\textbf{CO}%
 &|[fill=green]|\color[rgb]{0,0,0}\textbf{CO}%
 &|[fill=green]|\color[rgb]{0,0,0}\textbf{CO}%
 &|[fill=green]|\color[rgb]{0,0,0}\textbf{CO}%
 &|[fill=green]|\color[rgb]{0,0,0}\textbf{CO}%
 &|[fill=green]|\color[rgb]{0,0,0}\textbf{CO}%
 &|[fill=green]|\color[rgb]{0,0,0}\textbf{CO}%
 &|[fill=green]|\color[rgb]{0,0,0}\textbf{CO}%
 &|[fill=green]|\color[rgb]{0,0,0}\textbf{CO}%
 &|[fill=white]|\color[gray]{0.5}WA%
 &|[fill=green]|\color[gray]{0.75} FO%
\\
|[fill=green]|\color[gray]{0.75} FO%
 &|[fill=white]|\color[gray]{0.5}WA%
 &|[fill=green]|\color[rgb]{0,0,0}\textbf{CO}%
 &|[fill=green]|\color[rgb]{0,0,0}\textbf{CO}%
 &|[fill=green]|\color[rgb]{0,0,0}\textbf{CO}%
 &|[fill=green]|\color[rgb]{0,0,0}\textbf{CO}%
 &|[fill=green]|\color[rgb]{0,0,0}\textbf{CO}%
 &|[fill=green]|\color[rgb]{0,0,0}\textbf{CO}%
 &|[fill=green]|\color[rgb]{0,0,0}\textbf{CO}%
 &|[fill=green]|\color[rgb]{0,0,0}\textbf{CO}%
 &|[fill=green]|\color[rgb]{0,0,0}\textbf{CO}%
 &|[fill=white]|\color[gray]{0.5}WA%
 &|[fill=green]|\color[gray]{0.75} FO%
\\
|[fill=green]|\color[gray]{0.75} FO%
 &|[fill=white]|\color[gray]{0.5}WA%
 &|[fill=white]|\color[gray]{0.5}WA%
 &|[fill=white]|\color[gray]{0.5}WA%
 &|[fill=white]|\color[gray]{0.5}WA%
 &|[fill=white]|\color[gray]{0.5}WA%
 &|[fill=cyan]|\color[rgb]{1,0,0}\textbf{07}%
 &|[fill=white]|\color[gray]{0.5}WA%
 &|[fill=white]|\color[gray]{0.5}WA%
 &|[fill=white]|\color[gray]{0.5}WA%
 &|[fill=white]|\color[gray]{0.5}WA%
 &|[fill=white]|\color[gray]{0.5}WA%
 &|[fill=green]|\color[gray]{0.75} FO%
\\
|[fill=green]|\color[gray]{0.75} FO%
 &|[fill=green]|\color[gray]{0.75} FO%
 &|[fill=green]|\color[gray]{0.75} FO%
 &|[fill=green]|\color[gray]{0.75} FO%
 &|[fill=green]|\color[gray]{0.75} FO%
 &|[fill=green]|\color[gray]{0.75} FO%
 &|[fill=green]|\color[gray]{0.75} FO%
 &|[fill=green]|\color[gray]{0.75} FO%
 &|[fill=green]|\color[gray]{0.75} FO%
 &|[fill=green]|\color[gray]{0.75} FO%
 &|[fill=green]|\color[gray]{0.75} FO%
 &|[fill=green]|\color[gray]{0.75} FO%
 &|[fill=green]|\color[gray]{0.75} FO%
\\
};
\end{tikzpicture}\\
\\
Explanation:\\
\colorbox{white}{\color[gray]{0.5}WA}  ---  WALL\\
\colorbox{green}{\color[gray]{0.5}FO}  ---  FOREST\\
{\color[rgb]{1,0,0}\textbf{BU}}  ---  BURNED\\
{\color[rgb]{0,0,0}\textbf{CO}}  ---  COAL (doubly burned)\\
\colorbox{cyan}{\#\#}  ---  WATERED at time \#\#\\
Fields can have more than 1 state.
}
\subsubsection{Beispiel 3}
Ein (etwas) größeres Beispiel.\footnote{Diese Eingabe finden Sie auch in der Datei \texttt{3.in}}:
{\small
\lstinputlisting{../Aufgabe_1/3.in}
}
Mein Programm produziert folgende Ausgabe\footnote{Diese Ausgabe finden Sie auch in der Datei \texttt{3.out.tex2};} Dabei hat die Berechnung wenige Sekunden in Anspruch genommen, sofern nicht die Ausgabe der ASCII-Escape-Sequenzen gefordert wird. Dies erhöhte die Laufzeit auf ca. 30s.:\\
{\ttfamily \small
\\
\colorbox{green}{\color[gray]{0.75}FO}%
\colorbox{green}{\color[gray]{0.75}FO}%
\colorbox{green}{\color[gray]{0.75}FO}%
\colorbox{green}{\color[gray]{0.75}FO}%
\colorbox{green}{\color[gray]{0.75}FO}%
\colorbox{green}{\color[gray]{0.75}FO}%
\colorbox{green}{\color[gray]{0.75}FO}%
\colorbox{green}{\color[gray]{0.75}FO}%
\colorbox{green}{\color[gray]{0.75}FO}%
\colorbox{green}{\color[gray]{0.75}FO}%
\colorbox{green}{\color[gray]{0.75}FO}%
\colorbox{green}{\color[gray]{0.75}FO}%
\colorbox{green}{\color[gray]{0.75}FO}%
\colorbox{green}{\color[gray]{0.75}FO}%
\colorbox{green}{\color[gray]{0.75}FO}%
\colorbox{green}{\color[gray]{0.75}FO}%
\colorbox{green}{\color[gray]{0.75}FO}%
\colorbox{green}{\color[gray]{0.75}FO}%
\colorbox{green}{\color[gray]{0.75}FO}%
\colorbox{green}{\color[gray]{0.75}FO}%
\colorbox{green}{\color[gray]{0.75}FO}%
\colorbox{green}{\color[gray]{0.75}FO}%
\colorbox{green}{\color[gray]{0.75}FO}%
\colorbox{green}{\color[gray]{0.75}FO}%
\colorbox{green}{\color[gray]{0.75}FO}%
\colorbox{green}{\color[gray]{0.75}FO}%
\colorbox{green}{\color[gray]{0.75}FO}%
\colorbox{green}{\color[gray]{0.75}FO}%
\colorbox{green}{\color[gray]{0.75}FO}%
\colorbox{green}{\color[gray]{0.75}FO}%
\colorbox{green}{\color[gray]{0.75}FO}%
\colorbox{green}{\color[gray]{0.75}FO}%
\colorbox{green}{\color[gray]{0.75}FO}%
\colorbox{green}{\color[gray]{0.75}FO}%
\colorbox{green}{\color[gray]{0.75}FO}%
\colorbox{green}{\color[gray]{0.75}FO}%
\colorbox{green}{\color[gray]{0.75}FO}%
\colorbox{green}{\color[gray]{0.75}FO}%
\colorbox{green}{\color[gray]{0.75}FO}%
\colorbox{green}{\color[gray]{0.75}FO}%
\colorbox{green}{\color[gray]{0.75}FO}%
\colorbox{green}{\color[gray]{0.75}FO}%
\colorbox{green}{\color[gray]{0.75}FO}%
\colorbox{green}{\color[gray]{0.75}FO}%
\colorbox{green}{\color[gray]{0.75}FO}%
\colorbox{green}{\color[gray]{0.75}FO}%
\colorbox{green}{\color[gray]{0.75}FO}%
\colorbox{green}{\color[gray]{0.75}FO}%
\colorbox{green}{\color[gray]{0.75}FO}%
\colorbox{green}{\color[gray]{0.75}FO}%
\colorbox{green}{\color[gray]{0.75}FO}%
\colorbox{green}{\color[gray]{0.75}FO}%
\colorbox{green}{\color[gray]{0.75}FO}%
\colorbox{green}{\color[gray]{0.75}FO}%
\colorbox{green}{\color[gray]{0.75}FO}%
\colorbox{green}{\color[gray]{0.75}FO}%
\colorbox{green}{\color[gray]{0.75}FO}%
\colorbox{green}{\color[gray]{0.75}FO}%
\colorbox{green}{\color[gray]{0.75}FO}%
\colorbox{green}{\color[gray]{0.75}FO}%
\colorbox{green}{\color[gray]{0.75}FO}%
\colorbox{green}{\color[gray]{0.75}FO}%
\colorbox{green}{\color[gray]{0.75}FO}%
\colorbox{green}{\color[gray]{0.75}FO}%
\colorbox{green}{\color[gray]{0.75}FO}%
\colorbox{green}{\color[gray]{0.75}FO}%
\colorbox{green}{\color[gray]{0.75}FO}%
\colorbox{green}{\color[gray]{0.75}FO}%
\colorbox{green}{\color[gray]{0.75}FO}%
\colorbox{green}{\color[gray]{0.75}FO}%
\colorbox{green}{\color[gray]{0.75}FO}%
\colorbox{green}{\color[gray]{0.75}FO}%
\colorbox{green}{\color[gray]{0.75}FO}%
\colorbox{green}{\color[gray]{0.75}FO}%
\colorbox{green}{\color[gray]{0.75}FO}%
\colorbox{green}{\color[gray]{0.75}FO}%
\colorbox{green}{\color[gray]{0.75}FO}%
\colorbox{green}{\color[gray]{0.75}FO}%
\colorbox{green}{\color[gray]{0.75}FO}%
\colorbox{green}{\color[gray]{0.75}FO}%
\colorbox{green}{\color[gray]{0.75}FO}%
\colorbox{green}{\color[gray]{0.75}FO}%
\colorbox{green}{\color[gray]{0.75}FO}%
\colorbox{green}{\color[gray]{0.75}FO}%
\colorbox{green}{\color[gray]{0.75}FO}%
\colorbox{green}{\color[gray]{0.75}FO}%
\colorbox{green}{\color[gray]{0.75}FO}%
\colorbox{green}{\color[gray]{0.75}FO}%
\colorbox{green}{\color[gray]{0.75}FO}%
\colorbox{green}{\color[gray]{0.75}FO}%
\colorbox{green}{\color[gray]{0.75}FO}%
\colorbox{green}{\color[gray]{0.75}FO}%
\colorbox{green}{\color[gray]{0.75}FO}%
\colorbox{green}{\color[gray]{0.75}FO}%
\colorbox{green}{\color[gray]{0.75}FO}%
\colorbox{green}{\color[gray]{0.75}FO}%
\colorbox{green}{\color[gray]{0.75}FO}%
\colorbox{green}{\color[gray]{0.75}FO}%
\colorbox{green}{\color[gray]{0.75}FO}%
\colorbox{green}{\color[gray]{0.75}FO}%
\\
\colorbox{green}{\color[gray]{0.75}FO}%
\colorbox{green}{\color[gray]{0.75}FO}%
\colorbox{green}{\color[gray]{0.75}FO}%
\colorbox{green}{\color[gray]{0.75}FO}%
\colorbox{green}{\color[gray]{0.75}FO}%
\colorbox{green}{\color[gray]{0.75}FO}%
\colorbox{green}{\color[gray]{0.75}FO}%
\colorbox{green}{\color[gray]{0.75}FO}%
\colorbox{green}{\color[gray]{0.75}FO}%
\colorbox{green}{\color[gray]{0.75}FO}%
\colorbox{green}{\color[gray]{0.75}FO}%
\colorbox{green}{\color[gray]{0.75}FO}%
\colorbox{green}{\color[gray]{0.75}FO}%
\colorbox{green}{\color[gray]{0.75}FO}%
\colorbox{green}{\color[gray]{0.75}FO}%
\colorbox{green}{\color[gray]{0.75}FO}%
\colorbox{green}{\color[gray]{0.75}FO}%
\colorbox{green}{\color[gray]{0.75}FO}%
\colorbox{green}{\color[gray]{0.75}FO}%
\colorbox{green}{\color[gray]{0.75}FO}%
\colorbox{green}{\color[gray]{0.75}FO}%
\colorbox{green}{\color[gray]{0.75}FO}%
\colorbox{green}{\color[gray]{0.75}FO}%
\colorbox{green}{\color[gray]{0.75}FO}%
\colorbox{green}{\color[gray]{0.75}FO}%
\colorbox{green}{\color[gray]{0.75}FO}%
\colorbox{green}{\color[gray]{0.75}FO}%
\colorbox{green}{\color[gray]{0.75}FO}%
\colorbox{green}{\color[gray]{0.75}FO}%
\colorbox{green}{\color[gray]{0.75}FO}%
\colorbox{green}{\color[gray]{0.75}FO}%
\colorbox{green}{\color[gray]{0.75}FO}%
\colorbox{green}{\color[gray]{0.75}FO}%
\colorbox{green}{\color[gray]{0.75}FO}%
\colorbox{green}{\color[gray]{0.75}FO}%
\colorbox{green}{\color[gray]{0.75}FO}%
\colorbox{green}{\color[gray]{0.75}FO}%
\colorbox{green}{\color[gray]{0.75}FO}%
\colorbox{green}{\color[gray]{0.75}FO}%
\colorbox{green}{\color[gray]{0.75}FO}%
\colorbox{green}{\color[gray]{0.75}FO}%
\colorbox{green}{\color[gray]{0.75}FO}%
\colorbox{green}{\color[gray]{0.75}FO}%
\colorbox{green}{\color[gray]{0.75}FO}%
\colorbox{green}{\color[gray]{0.75}FO}%
\colorbox{green}{\color[gray]{0.75}FO}%
\colorbox{green}{\color[gray]{0.75}FO}%
\colorbox{green}{\color[gray]{0.75}FO}%
\colorbox{green}{\color[gray]{0.75}FO}%
\colorbox{green}{\color[gray]{0.75}FO}%
\colorbox{green}{\color[gray]{0.75}FO}%
\colorbox{green}{\color[gray]{0.75}FO}%
\colorbox{green}{\color[gray]{0.75}FO}%
\colorbox{green}{\color[gray]{0.75}FO}%
\colorbox{green}{\color[gray]{0.75}FO}%
\colorbox{green}{\color[gray]{0.75}FO}%
\colorbox{green}{\color[gray]{0.75}FO}%
\colorbox{green}{\color[gray]{0.75}FO}%
\colorbox{green}{\color[gray]{0.75}FO}%
\colorbox{green}{\color[gray]{0.75}FO}%
\colorbox{green}{\color[gray]{0.75}FO}%
\colorbox{green}{\color[gray]{0.75}FO}%
\colorbox{green}{\color[gray]{0.75}FO}%
\colorbox{green}{\color[gray]{0.75}FO}%
\colorbox{green}{\color[gray]{0.75}FO}%
\colorbox{green}{\color[gray]{0.75}FO}%
\colorbox{green}{\color[gray]{0.75}FO}%
\colorbox{green}{\color[gray]{0.75}FO}%
\colorbox{green}{\color[gray]{0.75}FO}%
\colorbox{green}{\color[gray]{0.75}FO}%
\colorbox{green}{\color[gray]{0.75}FO}%
\colorbox{green}{\color[gray]{0.75}FO}%
\colorbox{green}{\color[gray]{0.75}FO}%
\colorbox{green}{\color[gray]{0.75}FO}%
\colorbox{green}{\color[gray]{0.75}FO}%
\colorbox{green}{\color[gray]{0.75}FO}%
\colorbox{green}{\color[gray]{0.75}FO}%
\colorbox{green}{\color[gray]{0.75}FO}%
\colorbox{green}{\color[gray]{0.75}FO}%
\colorbox{green}{\color[gray]{0.75}FO}%
\colorbox{green}{\color[gray]{0.75}FO}%
\colorbox{green}{\color[gray]{0.75}FO}%
\colorbox{green}{\color[gray]{0.75}FO}%
\colorbox{green}{\color[gray]{0.75}FO}%
\colorbox{green}{\color[gray]{0.75}FO}%
\colorbox{green}{\color[gray]{0.75}FO}%
\colorbox{green}{\color[gray]{0.75}FO}%
\colorbox{green}{\color[gray]{0.75}FO}%
\colorbox{green}{\color[gray]{0.75}FO}%
\colorbox{green}{\color[gray]{0.75}FO}%
\colorbox{green}{\color[gray]{0.75}FO}%
\colorbox{green}{\color[gray]{0.75}FO}%
\colorbox{green}{\color[gray]{0.75}FO}%
\colorbox{green}{\color[gray]{0.75}FO}%
\colorbox{green}{\color[gray]{0.75}FO}%
\colorbox{green}{\color[gray]{0.75}FO}%
\colorbox{green}{\color[gray]{0.75}FO}%
\colorbox{green}{\color[gray]{0.75}FO}%
\colorbox{green}{\color[gray]{0.75}FO}%
\colorbox{green}{\color[gray]{0.75}FO}%
\\
\colorbox{green}{\color[gray]{0.75}FO}%
\colorbox{green}{\color[gray]{0.75}FO}%
\colorbox{green}{\color[gray]{0.75}FO}%
\colorbox{green}{\color[gray]{0.75}FO}%
\colorbox{green}{\color[gray]{0.75}FO}%
\colorbox{green}{\color[gray]{0.75}FO}%
\colorbox{green}{\color[gray]{0.75}FO}%
\colorbox{green}{\color[gray]{0.75}FO}%
\colorbox{green}{\color[gray]{0.75}FO}%
\colorbox{green}{\color[gray]{0.75}FO}%
\colorbox{green}{\color[gray]{0.75}FO}%
\colorbox{green}{\color[gray]{0.75}FO}%
\colorbox{green}{\color[gray]{0.75}FO}%
\colorbox{green}{\color[gray]{0.75}FO}%
\colorbox{green}{\color[gray]{0.75}FO}%
\colorbox{green}{\color[gray]{0.75}FO}%
\colorbox{green}{\color[gray]{0.75}FO}%
\colorbox{green}{\color[gray]{0.75}FO}%
\colorbox{green}{\color[gray]{0.75}FO}%
\colorbox{green}{\color[gray]{0.75}FO}%
\colorbox{green}{\color[gray]{0.75}FO}%
\colorbox{green}{\color[gray]{0.75}FO}%
\colorbox{green}{\color[gray]{0.75}FO}%
\colorbox{green}{\color[gray]{0.75}FO}%
\colorbox{green}{\color[gray]{0.75}FO}%
\colorbox{green}{\color[gray]{0.75}FO}%
\colorbox{green}{\color[gray]{0.75}FO}%
\colorbox{green}{\color[gray]{0.75}FO}%
\colorbox{green}{\color[gray]{0.75}FO}%
\colorbox{green}{\color[gray]{0.75}FO}%
\colorbox{green}{\color[gray]{0.75}FO}%
\colorbox{green}{\color[gray]{0.75}FO}%
\colorbox{green}{\color[gray]{0.75}FO}%
\colorbox{green}{\color[gray]{0.75}FO}%
\colorbox{green}{\color[gray]{0.75}FO}%
\colorbox{green}{\color[gray]{0.75}FO}%
\colorbox{green}{\color[gray]{0.75}FO}%
\colorbox{green}{\color[gray]{0.75}FO}%
\colorbox{green}{\color[gray]{0.75}FO}%
\colorbox{green}{\color[gray]{0.75}FO}%
\colorbox{green}{\color[gray]{0.75}FO}%
\colorbox{green}{\color[gray]{0.75}FO}%
\colorbox{green}{\color[gray]{0.75}FO}%
\colorbox{green}{\color[gray]{0.75}FO}%
\colorbox{green}{\color[gray]{0.75}FO}%
\colorbox{green}{\color[gray]{0.75}FO}%
\colorbox{green}{\color[gray]{0.75}FO}%
\colorbox{green}{\color[gray]{0.75}FO}%
\colorbox{green}{\color[gray]{0.75}FO}%
\colorbox{green}{\color[gray]{0.75}FO}%
\colorbox{green}{\color[gray]{0.75}FO}%
\colorbox{green}{\color[gray]{0.75}FO}%
\colorbox{green}{\color[gray]{0.75}FO}%
\colorbox{green}{\color[gray]{0.75}FO}%
\colorbox{green}{\color[gray]{0.75}FO}%
\colorbox{green}{\color[gray]{0.75}FO}%
\colorbox{green}{\color[gray]{0.75}FO}%
\colorbox{green}{\color[gray]{0.75}FO}%
\colorbox{green}{\color[gray]{0.75}FO}%
\colorbox{green}{\color[gray]{0.75}FO}%
\colorbox{green}{\color[gray]{0.75}FO}%
\colorbox{green}{\color[gray]{0.75}FO}%
\colorbox{green}{\color[gray]{0.75}FO}%
\colorbox{green}{\color[gray]{0.75}FO}%
\colorbox{green}{\color[gray]{0.75}FO}%
\colorbox{green}{\color[gray]{0.75}FO}%
\colorbox{green}{\color[gray]{0.75}FO}%
\colorbox{green}{\color[gray]{0.75}FO}%
\colorbox{green}{\color[gray]{0.75}FO}%
\colorbox{green}{\color[gray]{0.75}FO}%
\colorbox{green}{\color[gray]{0.75}FO}%
\colorbox{green}{\color[gray]{0.75}FO}%
\colorbox{green}{\color[gray]{0.75}FO}%
\colorbox{green}{\color[gray]{0.75}FO}%
\colorbox{green}{\color[gray]{0.75}FO}%
\colorbox{green}{\color[gray]{0.75}FO}%
\colorbox{green}{\color[gray]{0.75}FO}%
\colorbox{green}{\color[gray]{0.75}FO}%
\colorbox{green}{\color[gray]{0.75}FO}%
\colorbox{green}{\color[gray]{0.75}FO}%
\colorbox{green}{\color[gray]{0.75}FO}%
\colorbox{green}{\color[gray]{0.75}FO}%
\colorbox{green}{\color[gray]{0.75}FO}%
\colorbox{green}{\color[gray]{0.75}FO}%
\colorbox{green}{\color[gray]{0.75}FO}%
\colorbox{green}{\color[gray]{0.75}FO}%
\colorbox{green}{\color[gray]{0.75}FO}%
\colorbox{green}{\color[gray]{0.75}FO}%
\colorbox{green}{\color[gray]{0.75}FO}%
\colorbox{green}{\color[gray]{0.75}FO}%
\colorbox{green}{\color[gray]{0.75}FO}%
\colorbox{green}{\color[gray]{0.75}FO}%
\colorbox{green}{\color[gray]{0.75}FO}%
\colorbox{green}{\color[gray]{0.75}FO}%
\colorbox{green}{\color[gray]{0.75}FO}%
\colorbox{green}{\color[gray]{0.75}FO}%
\colorbox{green}{\color[gray]{0.75}FO}%
\colorbox{green}{\color[gray]{0.75}FO}%
\colorbox{green}{\color[gray]{0.75}FO}%
\colorbox{green}{\color[gray]{0.75}FO}%
\\
\colorbox{green}{\color[gray]{0.75}FO}%
\colorbox{green}{\color[gray]{0.75}FO}%
\colorbox{green}{\color[gray]{0.75}FO}%
\colorbox{green}{\color[gray]{0.75}FO}%
\colorbox{green}{\color[gray]{0.75}FO}%
\colorbox{green}{\color[gray]{0.75}FO}%
\colorbox{green}{\color[gray]{0.75}FO}%
\colorbox{green}{\color[gray]{0.75}FO}%
\colorbox{green}{\color[gray]{0.75}FO}%
\colorbox{green}{\color[gray]{0.75}FO}%
\colorbox{green}{\color[gray]{0.75}FO}%
\colorbox{green}{\color[gray]{0.75}FO}%
\colorbox{green}{\color[gray]{0.75}FO}%
\colorbox{green}{\color[gray]{0.75}FO}%
\colorbox{green}{\color[gray]{0.75}FO}%
\colorbox{green}{\color[gray]{0.75}FO}%
\colorbox{green}{\color[gray]{0.75}FO}%
\colorbox{green}{\color[gray]{0.75}FO}%
\colorbox{green}{\color[gray]{0.75}FO}%
\colorbox{green}{\color[gray]{0.75}FO}%
\colorbox{green}{\color[gray]{0.75}FO}%
\colorbox{green}{\color[gray]{0.75}FO}%
\colorbox{green}{\color[gray]{0.75}FO}%
\colorbox{green}{\color[gray]{0.75}FO}%
\colorbox{green}{\color[gray]{0.75}FO}%
\colorbox{green}{\color[gray]{0.75}FO}%
\colorbox{green}{\color[gray]{0.75}FO}%
\colorbox{green}{\color[gray]{0.75}FO}%
\colorbox{green}{\color[gray]{0.75}FO}%
\colorbox{green}{\color[gray]{0.75}FO}%
\colorbox{green}{\color[gray]{0.75}FO}%
\colorbox{green}{\color[gray]{0.75}FO}%
\colorbox{green}{\color[gray]{0.75}FO}%
\colorbox{green}{\color[gray]{0.75}FO}%
\colorbox{green}{\color[gray]{0.75}FO}%
\colorbox{green}{\color[gray]{0.75}FO}%
\colorbox{green}{\color[gray]{0.75}FO}%
\colorbox{green}{\color[gray]{0.75}FO}%
\colorbox{green}{\color[gray]{0.75}FO}%
\colorbox{green}{\color[gray]{0.75}FO}%
\colorbox{green}{\color[gray]{0.75}FO}%
\colorbox{green}{\color[gray]{0.75}FO}%
\colorbox{green}{\color[gray]{0.75}FO}%
\colorbox{green}{\color[gray]{0.75}FO}%
\colorbox{green}{\color[gray]{0.75}FO}%
\colorbox{green}{\color[gray]{0.75}FO}%
\colorbox{green}{\color[gray]{0.75}FO}%
\colorbox{green}{\color[gray]{0.75}FO}%
\colorbox{green}{\color[gray]{0.75}FO}%
\colorbox{green}{\color[gray]{0.75}FO}%
\colorbox{green}{\color[gray]{0.75}FO}%
\colorbox{green}{\color[gray]{0.75}FO}%
\colorbox{green}{\color[gray]{0.75}FO}%
\colorbox{green}{\color[gray]{0.75}FO}%
\colorbox{green}{\color[gray]{0.75}FO}%
\colorbox{green}{\color[gray]{0.75}FO}%
\colorbox{green}{\color[gray]{0.75}FO}%
\colorbox{green}{\color[gray]{0.75}FO}%
\colorbox{green}{\color[gray]{0.75}FO}%
\colorbox{green}{\color[gray]{0.75}FO}%
\colorbox{green}{\color[gray]{0.75}FO}%
\colorbox{green}{\color[gray]{0.75}FO}%
\colorbox{green}{\color[gray]{0.75}FO}%
\colorbox{green}{\color[gray]{0.75}FO}%
\colorbox{green}{\color[gray]{0.75}FO}%
\colorbox{green}{\color[gray]{0.75}FO}%
\colorbox{green}{\color[gray]{0.75}FO}%
\colorbox{green}{\color[gray]{0.75}FO}%
\colorbox{green}{\color[gray]{0.75}FO}%
\colorbox{green}{\color[gray]{0.75}FO}%
\colorbox{green}{\color[gray]{0.75}FO}%
\colorbox{green}{\color[gray]{0.75}FO}%
\colorbox{green}{\color[gray]{0.75}FO}%
\colorbox{green}{\color[gray]{0.75}FO}%
\colorbox{green}{\color[gray]{0.75}FO}%
\colorbox{green}{\color[gray]{0.75}FO}%
\colorbox{green}{\color[gray]{0.75}FO}%
\colorbox{green}{\color[gray]{0.75}FO}%
\colorbox{green}{\color[gray]{0.75}FO}%
\colorbox{green}{\color[gray]{0.75}FO}%
\colorbox{green}{\color[gray]{0.75}FO}%
\colorbox{green}{\color[gray]{0.75}FO}%
\colorbox{green}{\color[gray]{0.75}FO}%
\colorbox{green}{\color[gray]{0.75}FO}%
\colorbox{green}{\color[gray]{0.75}FO}%
\colorbox{green}{\color[gray]{0.75}FO}%
\colorbox{green}{\color[gray]{0.75}FO}%
\colorbox{green}{\color[gray]{0.75}FO}%
\colorbox{green}{\color[gray]{0.75}FO}%
\colorbox{green}{\color[gray]{0.75}FO}%
\colorbox{green}{\color[gray]{0.75}FO}%
\colorbox{green}{\color[gray]{0.75}FO}%
\colorbox{green}{\color[gray]{0.75}FO}%
\colorbox{green}{\color[gray]{0.75}FO}%
\colorbox{green}{\color[gray]{0.75}FO}%
\colorbox{green}{\color[gray]{0.75}FO}%
\colorbox{green}{\color[gray]{0.75}FO}%
\colorbox{green}{\color[gray]{0.75}FO}%
\colorbox{green}{\color[gray]{0.75}FO}%
\colorbox{green}{\color[gray]{0.75}FO}%
\\
\colorbox{green}{\color[gray]{0.75}FO}%
\colorbox{green}{\color[gray]{0.75}FO}%
\colorbox{green}{\color[gray]{0.75}FO}%
\colorbox{green}{\color[gray]{0.75}FO}%
\colorbox{green}{\color[gray]{0.75}FO}%
\colorbox{green}{\color[gray]{0.75}FO}%
\colorbox{green}{\color[gray]{0.75}FO}%
\colorbox{green}{\color[gray]{0.75}FO}%
\colorbox{green}{\color[gray]{0.75}FO}%
\colorbox{green}{\color[gray]{0.75}FO}%
\colorbox{green}{\color[gray]{0.75}FO}%
\colorbox{green}{\color[gray]{0.75}FO}%
\colorbox{green}{\color[gray]{0.75}FO}%
\colorbox{green}{\color[gray]{0.75}FO}%
\colorbox{green}{\color[gray]{0.75}FO}%
\colorbox{green}{\color[gray]{0.75}FO}%
\colorbox{green}{\color[gray]{0.75}FO}%
\colorbox{green}{\color[gray]{0.75}FO}%
\colorbox{green}{\color[gray]{0.75}FO}%
\colorbox{green}{\color[gray]{0.75}FO}%
\colorbox{green}{\color[gray]{0.75}FO}%
\colorbox{green}{\color[gray]{0.75}FO}%
\colorbox{green}{\color[gray]{0.75}FO}%
\colorbox{green}{\color[gray]{0.75}FO}%
\colorbox{green}{\color[gray]{0.75}FO}%
\colorbox{green}{\color[gray]{0.75}FO}%
\colorbox{green}{\color[gray]{0.75}FO}%
\colorbox{green}{\color[gray]{0.75}FO}%
\colorbox{green}{\color[gray]{0.75}FO}%
\colorbox{green}{\color[gray]{0.75}FO}%
\colorbox{green}{\color[gray]{0.75}FO}%
\colorbox{green}{\color[gray]{0.75}FO}%
\colorbox{green}{\color[gray]{0.75}FO}%
\colorbox{green}{\color[gray]{0.75}FO}%
\colorbox{green}{\color[gray]{0.75}FO}%
\colorbox{green}{\color[gray]{0.75}FO}%
\colorbox{green}{\color[gray]{0.75}FO}%
\colorbox{green}{\color[gray]{0.75}FO}%
\colorbox{green}{\color[gray]{0.75}FO}%
\colorbox{green}{\color[gray]{0.75}FO}%
\colorbox{green}{\color[gray]{0.75}FO}%
\colorbox{green}{\color[gray]{0.75}FO}%
\colorbox{green}{\color[gray]{0.75}FO}%
\colorbox{green}{\color[gray]{0.75}FO}%
\colorbox{green}{\color[gray]{0.75}FO}%
\colorbox{green}{\color[gray]{0.75}FO}%
\colorbox{green}{\color[gray]{0.75}FO}%
\colorbox{green}{\color[gray]{0.75}FO}%
\colorbox{green}{\color[gray]{0.75}FO}%
\colorbox{green}{\color[gray]{0.75}FO}%
\colorbox{green}{\color[gray]{0.75}FO}%
\colorbox{green}{\color[gray]{0.75}FO}%
\colorbox{green}{\color[gray]{0.75}FO}%
\colorbox{green}{\color[gray]{0.75}FO}%
\colorbox{green}{\color[gray]{0.75}FO}%
\colorbox{green}{\color[gray]{0.75}FO}%
\colorbox{green}{\color[gray]{0.75}FO}%
\colorbox{green}{\color[gray]{0.75}FO}%
\colorbox{green}{\color[gray]{0.75}FO}%
\colorbox{green}{\color[gray]{0.75}FO}%
\colorbox{green}{\color[gray]{0.75}FO}%
\colorbox{green}{\color[gray]{0.75}FO}%
\colorbox{green}{\color[gray]{0.75}FO}%
\colorbox{green}{\color[gray]{0.75}FO}%
\colorbox{green}{\color[gray]{0.75}FO}%
\colorbox{green}{\color[gray]{0.75}FO}%
\colorbox{green}{\color[gray]{0.75}FO}%
\colorbox{green}{\color[gray]{0.75}FO}%
\colorbox{green}{\color[gray]{0.75}FO}%
\colorbox{green}{\color[gray]{0.75}FO}%
\colorbox{green}{\color[gray]{0.75}FO}%
\colorbox{green}{\color[gray]{0.75}FO}%
\colorbox{green}{\color[gray]{0.75}FO}%
\colorbox{green}{\color[gray]{0.75}FO}%
\colorbox{green}{\color[gray]{0.75}FO}%
\colorbox{green}{\color[gray]{0.75}FO}%
\colorbox{green}{\color[gray]{0.75}FO}%
\colorbox{green}{\color[gray]{0.75}FO}%
\colorbox{green}{\color[gray]{0.75}FO}%
\colorbox{green}{\color[gray]{0.75}FO}%
\colorbox{green}{\color[gray]{0.75}FO}%
\colorbox{green}{\color[gray]{0.75}FO}%
\colorbox{green}{\color[gray]{0.75}FO}%
\colorbox{green}{\color[gray]{0.75}FO}%
\colorbox{green}{\color[gray]{0.75}FO}%
\colorbox{green}{\color[gray]{0.75}FO}%
\colorbox{green}{\color[gray]{0.75}FO}%
\colorbox{green}{\color[gray]{0.75}FO}%
\colorbox{green}{\color[gray]{0.75}FO}%
\colorbox{green}{\color[gray]{0.75}FO}%
\colorbox{green}{\color[gray]{0.75}FO}%
\colorbox{green}{\color[gray]{0.75}FO}%
\colorbox{green}{\color[gray]{0.75}FO}%
\colorbox{green}{\color[gray]{0.75}FO}%
\colorbox{green}{\color[gray]{0.75}FO}%
\colorbox{green}{\color[gray]{0.75}FO}%
\colorbox{green}{\color[gray]{0.75}FO}%
\colorbox{green}{\color[gray]{0.75}FO}%
\colorbox{green}{\color[gray]{0.75}FO}%
\colorbox{green}{\color[gray]{0.75}FO}%
\\
\colorbox{green}{\color[gray]{0.75}FO}%
\colorbox{green}{\color[gray]{0.75}FO}%
\colorbox{green}{\color[gray]{0.75}FO}%
\colorbox{green}{\color[gray]{0.75}FO}%
\colorbox{green}{\color[gray]{0.75}FO}%
\colorbox{green}{\color[gray]{0.75}FO}%
\colorbox{green}{\color[gray]{0.75}FO}%
\colorbox{green}{\color[gray]{0.75}FO}%
\colorbox{green}{\color[gray]{0.75}FO}%
\colorbox{green}{\color[gray]{0.75}FO}%
\colorbox{green}{\color[gray]{0.75}FO}%
\colorbox{green}{\color[gray]{0.75}FO}%
\colorbox{green}{\color[gray]{0.75}FO}%
\colorbox{green}{\color[gray]{0.75}FO}%
\colorbox{green}{\color[gray]{0.75}FO}%
\colorbox{green}{\color[gray]{0.75}FO}%
\colorbox{green}{\color[gray]{0.75}FO}%
\colorbox{green}{\color[gray]{0.75}FO}%
\colorbox{green}{\color[gray]{0.75}FO}%
\colorbox{green}{\color[gray]{0.75}FO}%
\colorbox{green}{\color[gray]{0.75}FO}%
\colorbox{green}{\color[gray]{0.75}FO}%
\colorbox{green}{\color[gray]{0.75}FO}%
\colorbox{green}{\color[gray]{0.75}FO}%
\colorbox{green}{\color[gray]{0.75}FO}%
\colorbox{green}{\color[gray]{0.75}FO}%
\colorbox{green}{\color[gray]{0.75}FO}%
\colorbox{green}{\color[gray]{0.75}FO}%
\colorbox{green}{\color[gray]{0.75}FO}%
\colorbox{green}{\color[gray]{0.75}FO}%
\colorbox{green}{\color[gray]{0.75}FO}%
\colorbox{green}{\color[gray]{0.75}FO}%
\colorbox{green}{\color[gray]{0.75}FO}%
\colorbox{green}{\color[gray]{0.75}FO}%
\colorbox{green}{\color[gray]{0.75}FO}%
\colorbox{green}{\color[gray]{0.75}FO}%
\colorbox{green}{\color[gray]{0.75}FO}%
\colorbox{green}{\color[gray]{0.75}FO}%
\colorbox{green}{\color[gray]{0.75}FO}%
\colorbox{green}{\color[gray]{0.75}FO}%
\colorbox{green}{\color[gray]{0.75}FO}%
\colorbox{green}{\color[gray]{0.75}FO}%
\colorbox{green}{\color[gray]{0.75}FO}%
\colorbox{green}{\color[gray]{0.75}FO}%
\colorbox{green}{\color[gray]{0.75}FO}%
\colorbox{green}{\color[gray]{0.75}FO}%
\colorbox{green}{\color[gray]{0.75}FO}%
\colorbox{green}{\color[gray]{0.75}FO}%
\colorbox{green}{\color[gray]{0.75}FO}%
\colorbox{green}{\color[gray]{0.75}FO}%
\colorbox{green}{\color[gray]{0.75}FO}%
\colorbox{green}{\color[gray]{0.75}FO}%
\colorbox{green}{\color[gray]{0.75}FO}%
\colorbox{green}{\color[gray]{0.75}FO}%
\colorbox{green}{\color[gray]{0.75}FO}%
\colorbox{green}{\color[gray]{0.75}FO}%
\colorbox{green}{\color[gray]{0.75}FO}%
\colorbox{green}{\color[gray]{0.75}FO}%
\colorbox{green}{\color[gray]{0.75}FO}%
\colorbox{green}{\color[gray]{0.75}FO}%
\colorbox{green}{\color[gray]{0.75}FO}%
\colorbox{green}{\color[gray]{0.75}FO}%
\colorbox{green}{\color[gray]{0.75}FO}%
\colorbox{green}{\color[gray]{0.75}FO}%
\colorbox{green}{\color[gray]{0.75}FO}%
\colorbox{green}{\color[gray]{0.75}FO}%
\colorbox{green}{\color[gray]{0.75}FO}%
\colorbox{green}{\color[gray]{0.75}FO}%
\colorbox{green}{\color[gray]{0.75}FO}%
\colorbox{green}{\color[gray]{0.75}FO}%
\colorbox{green}{\color[gray]{0.75}FO}%
\colorbox{green}{\color[gray]{0.75}FO}%
\colorbox{green}{\color[gray]{0.75}FO}%
\colorbox{green}{\color[gray]{0.75}FO}%
\colorbox{green}{\color[gray]{0.75}FO}%
\colorbox{green}{\color[gray]{0.75}FO}%
\colorbox{green}{\color[gray]{0.75}FO}%
\colorbox{green}{\color[gray]{0.75}FO}%
\colorbox{green}{\color[gray]{0.75}FO}%
\colorbox{green}{\color[gray]{0.75}FO}%
\colorbox{green}{\color[gray]{0.75}FO}%
\colorbox{green}{\color[gray]{0.75}FO}%
\colorbox{green}{\color[gray]{0.75}FO}%
\colorbox{green}{\color[gray]{0.75}FO}%
\colorbox{green}{\color[gray]{0.75}FO}%
\colorbox{green}{\color[gray]{0.75}FO}%
\colorbox{green}{\color[gray]{0.75}FO}%
\colorbox{green}{\color[gray]{0.75}FO}%
\colorbox{green}{\color[gray]{0.75}FO}%
\colorbox{green}{\color[gray]{0.75}FO}%
\colorbox{green}{\color[gray]{0.75}FO}%
\colorbox{green}{\color[gray]{0.75}FO}%
\colorbox{green}{\color[gray]{0.75}FO}%
\colorbox{green}{\color[gray]{0.75}FO}%
\colorbox{green}{\color[gray]{0.75}FO}%
\colorbox{green}{\color[gray]{0.75}FO}%
\colorbox{green}{\color[gray]{0.75}FO}%
\colorbox{green}{\color[gray]{0.75}FO}%
\colorbox{green}{\color[gray]{0.75}FO}%
\colorbox{green}{\color[gray]{0.75}FO}%
\\
\colorbox{green}{\color[gray]{0.75}FO}%
\colorbox{green}{\color[gray]{0.75}FO}%
\colorbox{green}{\color[gray]{0.75}FO}%
\colorbox{green}{\color[gray]{0.75}FO}%
\colorbox{green}{\color[gray]{0.75}FO}%
\colorbox{green}{\color[gray]{0.75}FO}%
\colorbox{green}{\color[gray]{0.75}FO}%
\colorbox{green}{\color[gray]{0.75}FO}%
\colorbox{green}{\color[gray]{0.75}FO}%
\colorbox{green}{\color[gray]{0.75}FO}%
\colorbox{green}{\color[gray]{0.75}FO}%
\colorbox{green}{\color[gray]{0.75}FO}%
\colorbox{green}{\color[gray]{0.75}FO}%
\colorbox{green}{\color[gray]{0.75}FO}%
\colorbox{green}{\color[gray]{0.75}FO}%
\colorbox{green}{\color[gray]{0.75}FO}%
\colorbox{green}{\color[gray]{0.75}FO}%
\colorbox{green}{\color[gray]{0.75}FO}%
\colorbox{green}{\color[gray]{0.75}FO}%
\colorbox{green}{\color[gray]{0.75}FO}%
\colorbox{green}{\color[gray]{0.75}FO}%
\colorbox{green}{\color[gray]{0.75}FO}%
\colorbox{green}{\color[gray]{0.75}FO}%
\colorbox{green}{\color[gray]{0.75}FO}%
\colorbox{green}{\color[gray]{0.75}FO}%
\colorbox{green}{\color[gray]{0.75}FO}%
\colorbox{green}{\color[gray]{0.75}FO}%
\colorbox{green}{\color[gray]{0.75}FO}%
\colorbox{green}{\color[gray]{0.75}FO}%
\colorbox{green}{\color[gray]{0.75}FO}%
\colorbox{green}{\color[gray]{0.75}FO}%
\colorbox{green}{\color[gray]{0.75}FO}%
\colorbox{green}{\color[gray]{0.75}FO}%
\colorbox{green}{\color[gray]{0.75}FO}%
\colorbox{green}{\color[gray]{0.75}FO}%
\colorbox{green}{\color[gray]{0.75}FO}%
\colorbox{green}{\color[gray]{0.75}FO}%
\colorbox{green}{\color[gray]{0.75}FO}%
\colorbox{green}{\color[gray]{0.75}FO}%
\colorbox{green}{\color[gray]{0.75}FO}%
\colorbox{green}{\color[gray]{0.75}FO}%
\colorbox{green}{\color[gray]{0.75}FO}%
\colorbox{green}{\color[gray]{0.75}FO}%
\colorbox{green}{\color[gray]{0.75}FO}%
\colorbox{green}{\color[gray]{0.75}FO}%
\colorbox{green}{\color[gray]{0.75}FO}%
\colorbox{green}{\color[gray]{0.75}FO}%
\colorbox{green}{\color[gray]{0.75}FO}%
\colorbox{green}{\color[gray]{0.75}FO}%
\colorbox{green}{\color[gray]{0.75}FO}%
\colorbox{green}{\color[gray]{0.75}FO}%
\colorbox{green}{\color[gray]{0.75}FO}%
\colorbox{green}{\color[gray]{0.75}FO}%
\colorbox{green}{\color[gray]{0.75}FO}%
\colorbox{green}{\color[gray]{0.75}FO}%
\colorbox{green}{\color[gray]{0.75}FO}%
\colorbox{green}{\color[gray]{0.75}FO}%
\colorbox{green}{\color[gray]{0.75}FO}%
\colorbox{green}{\color[gray]{0.75}FO}%
\colorbox{green}{\color[gray]{0.75}FO}%
\colorbox{green}{\color[gray]{0.75}FO}%
\colorbox{green}{\color[gray]{0.75}FO}%
\colorbox{green}{\color[gray]{0.75}FO}%
\colorbox{green}{\color[gray]{0.75}FO}%
\colorbox{green}{\color[gray]{0.75}FO}%
\colorbox{green}{\color[gray]{0.75}FO}%
\colorbox{green}{\color[gray]{0.75}FO}%
\colorbox{green}{\color[gray]{0.75}FO}%
\colorbox{green}{\color[gray]{0.75}FO}%
\colorbox{green}{\color[gray]{0.75}FO}%
\colorbox{green}{\color[gray]{0.75}FO}%
\colorbox{green}{\color[gray]{0.75}FO}%
\colorbox{green}{\color[gray]{0.75}FO}%
\colorbox{green}{\color[gray]{0.75}FO}%
\colorbox{green}{\color[gray]{0.75}FO}%
\colorbox{green}{\color[gray]{0.75}FO}%
\colorbox{green}{\color[gray]{0.75}FO}%
\colorbox{green}{\color[gray]{0.75}FO}%
\colorbox{green}{\color[gray]{0.75}FO}%
\colorbox{green}{\color[gray]{0.75}FO}%
\colorbox{green}{\color[gray]{0.75}FO}%
\colorbox{green}{\color[gray]{0.75}FO}%
\colorbox{green}{\color[gray]{0.75}FO}%
\colorbox{green}{\color[gray]{0.75}FO}%
\colorbox{green}{\color[gray]{0.75}FO}%
\colorbox{green}{\color[gray]{0.75}FO}%
\colorbox{green}{\color[gray]{0.75}FO}%
\colorbox{green}{\color[gray]{0.75}FO}%
\colorbox{green}{\color[gray]{0.75}FO}%
\colorbox{green}{\color[gray]{0.75}FO}%
\colorbox{green}{\color[gray]{0.75}FO}%
\colorbox{green}{\color[gray]{0.75}FO}%
\colorbox{green}{\color[gray]{0.75}FO}%
\colorbox{green}{\color[gray]{0.75}FO}%
\colorbox{green}{\color[gray]{0.75}FO}%
\colorbox{green}{\color[gray]{0.75}FO}%
\colorbox{green}{\color[gray]{0.75}FO}%
\colorbox{green}{\color[gray]{0.75}FO}%
\colorbox{green}{\color[gray]{0.75}FO}%
\colorbox{green}{\color[gray]{0.75}FO}%
\\
\colorbox{green}{\color[gray]{0.75}FO}%
\colorbox{green}{\color[gray]{0.75}FO}%
\colorbox{green}{\color[gray]{0.75}FO}%
\colorbox{green}{\color[gray]{0.75}FO}%
\colorbox{green}{\color[gray]{0.75}FO}%
\colorbox{green}{\color[gray]{0.75}FO}%
\colorbox{green}{\color[gray]{0.75}FO}%
\colorbox{green}{\color[gray]{0.75}FO}%
\colorbox{green}{\color[gray]{0.75}FO}%
\colorbox{green}{\color[gray]{0.75}FO}%
\colorbox{green}{\color[gray]{0.75}FO}%
\colorbox{green}{\color[gray]{0.75}FO}%
\colorbox{green}{\color[gray]{0.75}FO}%
\colorbox{green}{\color[gray]{0.75}FO}%
\colorbox{green}{\color[gray]{0.75}FO}%
\colorbox{green}{\color[gray]{0.75}FO}%
\colorbox{green}{\color[gray]{0.75}FO}%
\colorbox{green}{\color[gray]{0.75}FO}%
\colorbox{green}{\color[gray]{0.75}FO}%
\colorbox{green}{\color[gray]{0.75}FO}%
\colorbox{green}{\color[gray]{0.75}FO}%
\colorbox{green}{\color[gray]{0.75}FO}%
\colorbox{green}{\color[gray]{0.75}FO}%
\colorbox{green}{\color[gray]{0.75}FO}%
\colorbox{green}{\color[gray]{0.75}FO}%
\colorbox{green}{\color[gray]{0.75}FO}%
\colorbox{green}{\color[gray]{0.75}FO}%
\colorbox{green}{\color[gray]{0.75}FO}%
\colorbox{green}{\color[gray]{0.75}FO}%
\colorbox{green}{\color[gray]{0.75}FO}%
\colorbox{green}{\color[gray]{0.75}FO}%
\colorbox{green}{\color[gray]{0.75}FO}%
\colorbox{green}{\color[gray]{0.75}FO}%
\colorbox{green}{\color[gray]{0.75}FO}%
\colorbox{green}{\color[gray]{0.75}FO}%
\colorbox{green}{\color[gray]{0.75}FO}%
\colorbox{green}{\color[gray]{0.75}FO}%
\colorbox{green}{\color[gray]{0.75}FO}%
\colorbox{green}{\color[gray]{0.75}FO}%
\colorbox{green}{\color[gray]{0.75}FO}%
\colorbox{green}{\color[gray]{0.75}FO}%
\colorbox{green}{\color[gray]{0.75}FO}%
\colorbox{green}{\color[gray]{0.75}FO}%
\colorbox{green}{\color[gray]{0.75}FO}%
\colorbox{green}{\color[gray]{0.75}FO}%
\colorbox{green}{\color[gray]{0.75}FO}%
\colorbox{green}{\color[gray]{0.75}FO}%
\colorbox{green}{\color[gray]{0.75}FO}%
\colorbox{green}{\color[gray]{0.75}FO}%
\colorbox{green}{\color[gray]{0.75}FO}%
\colorbox{green}{\color[gray]{0.75}FO}%
\colorbox{green}{\color[gray]{0.75}FO}%
\colorbox{green}{\color[gray]{0.75}FO}%
\colorbox{green}{\color[gray]{0.75}FO}%
\colorbox{green}{\color[gray]{0.75}FO}%
\colorbox{green}{\color[gray]{0.75}FO}%
\colorbox{green}{\color[gray]{0.75}FO}%
\colorbox{green}{\color[gray]{0.75}FO}%
\colorbox{green}{\color[gray]{0.75}FO}%
\colorbox{green}{\color[gray]{0.75}FO}%
\colorbox{green}{\color[gray]{0.75}FO}%
\colorbox{green}{\color[gray]{0.75}FO}%
\colorbox{green}{\color[gray]{0.75}FO}%
\colorbox{green}{\color[gray]{0.75}FO}%
\colorbox{green}{\color[gray]{0.75}FO}%
\colorbox{green}{\color[gray]{0.75}FO}%
\colorbox{green}{\color[gray]{0.75}FO}%
\colorbox{green}{\color[gray]{0.75}FO}%
\colorbox{green}{\color[gray]{0.75}FO}%
\colorbox{green}{\color[gray]{0.75}FO}%
\colorbox{green}{\color[gray]{0.75}FO}%
\colorbox{green}{\color[gray]{0.75}FO}%
\colorbox{green}{\color[gray]{0.75}FO}%
\colorbox{green}{\color[gray]{0.75}FO}%
\colorbox{green}{\color[gray]{0.75}FO}%
\colorbox{green}{\color[gray]{0.75}FO}%
\colorbox{green}{\color[gray]{0.75}FO}%
\colorbox{green}{\color[gray]{0.75}FO}%
\colorbox{green}{\color[gray]{0.75}FO}%
\colorbox{green}{\color[gray]{0.75}FO}%
\colorbox{green}{\color[gray]{0.75}FO}%
\colorbox{green}{\color[gray]{0.75}FO}%
\colorbox{green}{\color[gray]{0.75}FO}%
\colorbox{green}{\color[gray]{0.75}FO}%
\colorbox{green}{\color[gray]{0.75}FO}%
\colorbox{green}{\color[gray]{0.75}FO}%
\colorbox{green}{\color[gray]{0.75}FO}%
\colorbox{green}{\color[gray]{0.75}FO}%
\colorbox{green}{\color[gray]{0.75}FO}%
\colorbox{green}{\color[gray]{0.75}FO}%
\colorbox{green}{\color[gray]{0.75}FO}%
\colorbox{green}{\color[gray]{0.75}FO}%
\colorbox{green}{\color[gray]{0.75}FO}%
\colorbox{green}{\color[gray]{0.75}FO}%
\colorbox{green}{\color[gray]{0.75}FO}%
\colorbox{green}{\color[gray]{0.75}FO}%
\colorbox{green}{\color[gray]{0.75}FO}%
\colorbox{green}{\color[gray]{0.75}FO}%
\colorbox{green}{\color[gray]{0.75}FO}%
\colorbox{green}{\color[gray]{0.75}FO}%
\\
\colorbox{green}{\color[gray]{0.75}FO}%
\colorbox{green}{\color[gray]{0.75}FO}%
\colorbox{green}{\color[gray]{0.75}FO}%
\colorbox{green}{\color[gray]{0.75}FO}%
\colorbox{green}{\color[gray]{0.75}FO}%
\colorbox{green}{\color[gray]{0.75}FO}%
\colorbox{green}{\color[gray]{0.75}FO}%
\colorbox{green}{\color[gray]{0.75}FO}%
\colorbox{green}{\color[gray]{0.75}FO}%
\colorbox{green}{\color[gray]{0.75}FO}%
\colorbox{green}{\color[gray]{0.75}FO}%
\colorbox{green}{\color[gray]{0.75}FO}%
\colorbox{green}{\color[gray]{0.75}FO}%
\colorbox{green}{\color[gray]{0.75}FO}%
\colorbox{green}{\color[gray]{0.75}FO}%
\colorbox{green}{\color[gray]{0.75}FO}%
\colorbox{green}{\color[gray]{0.75}FO}%
\colorbox{green}{\color[gray]{0.75}FO}%
\colorbox{green}{\color[gray]{0.75}FO}%
\colorbox{green}{\color[gray]{0.75}FO}%
\colorbox{green}{\color[gray]{0.75}FO}%
\colorbox{green}{\color[gray]{0.75}FO}%
\colorbox{green}{\color[gray]{0.75}FO}%
\colorbox{green}{\color[gray]{0.75}FO}%
\colorbox{green}{\color[gray]{0.75}FO}%
\colorbox{green}{\color[gray]{0.75}FO}%
\colorbox{green}{\color[gray]{0.75}FO}%
\colorbox{green}{\color[gray]{0.75}FO}%
\colorbox{green}{\color[gray]{0.75}FO}%
\colorbox{green}{\color[gray]{0.75}FO}%
\colorbox{green}{\color[gray]{0.75}FO}%
\colorbox{green}{\color[gray]{0.75}FO}%
\colorbox{green}{\color[gray]{0.75}FO}%
\colorbox{green}{\color[gray]{0.75}FO}%
\colorbox{green}{\color[gray]{0.75}FO}%
\colorbox{green}{\color[gray]{0.75}FO}%
\colorbox{green}{\color[gray]{0.75}FO}%
\colorbox{green}{\color[gray]{0.75}FO}%
\colorbox{green}{\color[gray]{0.75}FO}%
\colorbox{green}{\color[gray]{0.75}FO}%
\colorbox{green}{\color[gray]{0.75}FO}%
\colorbox{green}{\color[gray]{0.75}FO}%
\colorbox{green}{\color[gray]{0.75}FO}%
\colorbox{green}{\color[gray]{0.75}FO}%
\colorbox{green}{\color[gray]{0.75}FO}%
\colorbox{green}{\color[gray]{0.75}FO}%
\colorbox{green}{\color[gray]{0.75}FO}%
\colorbox{green}{\color[gray]{0.75}FO}%
\colorbox{green}{\color[gray]{0.75}FO}%
\colorbox{green}{\color[gray]{0.75}FO}%
\colorbox{green}{\color[gray]{0.75}FO}%
\colorbox{green}{\color[gray]{0.75}FO}%
\colorbox{green}{\color[gray]{0.75}FO}%
\colorbox{green}{\color[gray]{0.75}FO}%
\colorbox{green}{\color[gray]{0.75}FO}%
\colorbox{green}{\color[gray]{0.75}FO}%
\colorbox{green}{\color[gray]{0.75}FO}%
\colorbox{green}{\color[gray]{0.75}FO}%
\colorbox{green}{\color[gray]{0.75}FO}%
\colorbox{green}{\color[gray]{0.75}FO}%
\colorbox{green}{\color[gray]{0.75}FO}%
\colorbox{green}{\color[gray]{0.75}FO}%
\colorbox{green}{\color[gray]{0.75}FO}%
\colorbox{green}{\color[gray]{0.75}FO}%
\colorbox{green}{\color[gray]{0.75}FO}%
\colorbox{green}{\color[gray]{0.75}FO}%
\colorbox{green}{\color[gray]{0.75}FO}%
\colorbox{green}{\color[gray]{0.75}FO}%
\colorbox{green}{\color[gray]{0.75}FO}%
\colorbox{green}{\color[gray]{0.75}FO}%
\colorbox{green}{\color[gray]{0.75}FO}%
\colorbox{green}{\color[gray]{0.75}FO}%
\colorbox{green}{\color[gray]{0.75}FO}%
\colorbox{green}{\color[gray]{0.75}FO}%
\colorbox{green}{\color[gray]{0.75}FO}%
\colorbox{green}{\color[gray]{0.75}FO}%
\colorbox{green}{\color[gray]{0.75}FO}%
\colorbox{green}{\color[gray]{0.75}FO}%
\colorbox{green}{\color[gray]{0.75}FO}%
\colorbox{green}{\color[gray]{0.75}FO}%
\colorbox{green}{\color[gray]{0.75}FO}%
\colorbox{green}{\color[gray]{0.75}FO}%
\colorbox{green}{\color[gray]{0.75}FO}%
\colorbox{green}{\color[gray]{0.75}FO}%
\colorbox{green}{\color[gray]{0.75}FO}%
\colorbox{green}{\color[gray]{0.75}FO}%
\colorbox{green}{\color[gray]{0.75}FO}%
\colorbox{green}{\color[gray]{0.75}FO}%
\colorbox{green}{\color[gray]{0.75}FO}%
\colorbox{green}{\color[gray]{0.75}FO}%
\colorbox{green}{\color[gray]{0.75}FO}%
\colorbox{green}{\color[gray]{0.75}FO}%
\colorbox{green}{\color[gray]{0.75}FO}%
\colorbox{green}{\color[gray]{0.75}FO}%
\colorbox{green}{\color[gray]{0.75}FO}%
\colorbox{green}{\color[gray]{0.75}FO}%
\colorbox{green}{\color[gray]{0.75}FO}%
\colorbox{green}{\color[gray]{0.75}FO}%
\colorbox{green}{\color[gray]{0.75}FO}%
\colorbox{green}{\color[gray]{0.75}FO}%
\\
\colorbox{green}{\color[gray]{0.75}FO}%
\colorbox{green}{\color[gray]{0.75}FO}%
\colorbox{green}{\color[gray]{0.75}FO}%
\colorbox{green}{\color[gray]{0.75}FO}%
\colorbox{green}{\color[gray]{0.75}FO}%
\colorbox{green}{\color[gray]{0.75}FO}%
\colorbox{green}{\color[gray]{0.75}FO}%
\colorbox{green}{\color[gray]{0.75}FO}%
\colorbox{green}{\color[gray]{0.75}FO}%
\colorbox{green}{\color[gray]{0.75}FO}%
\colorbox{green}{\color[gray]{0.75}FO}%
\colorbox{green}{\color[gray]{0.75}FO}%
\colorbox{green}{\color[gray]{0.75}FO}%
\colorbox{green}{\color[gray]{0.75}FO}%
\colorbox{green}{\color[gray]{0.75}FO}%
\colorbox{green}{\color[gray]{0.75}FO}%
\colorbox{green}{\color[gray]{0.75}FO}%
\colorbox{green}{\color[gray]{0.75}FO}%
\colorbox{green}{\color[gray]{0.75}FO}%
\colorbox{green}{\color[gray]{0.75}FO}%
\colorbox{green}{\color[gray]{0.75}FO}%
\colorbox{green}{\color[gray]{0.75}FO}%
\colorbox{green}{\color[gray]{0.75}FO}%
\colorbox{green}{\color[gray]{0.75}FO}%
\colorbox{green}{\color[gray]{0.75}FO}%
\colorbox{green}{\color[gray]{0.75}FO}%
\colorbox{green}{\color[gray]{0.75}FO}%
\colorbox{green}{\color[gray]{0.75}FO}%
\colorbox{green}{\color[gray]{0.75}FO}%
\colorbox{green}{\color[gray]{0.75}FO}%
\colorbox{green}{\color[gray]{0.75}FO}%
\colorbox{green}{\color[gray]{0.75}FO}%
\colorbox{green}{\color[gray]{0.75}FO}%
\colorbox{green}{\color[gray]{0.75}FO}%
\colorbox{green}{\color[gray]{0.75}FO}%
\colorbox{green}{\color[gray]{0.75}FO}%
\colorbox{green}{\color[gray]{0.75}FO}%
\colorbox{green}{\color[gray]{0.75}FO}%
\colorbox{green}{\color[gray]{0.75}FO}%
\colorbox{green}{\color[gray]{0.75}FO}%
\colorbox{green}{\color[gray]{0.75}FO}%
\colorbox{green}{\color[gray]{0.75}FO}%
\colorbox{green}{\color[gray]{0.75}FO}%
\colorbox{green}{\color[gray]{0.75}FO}%
\colorbox{green}{\color[gray]{0.75}FO}%
\colorbox{green}{\color[gray]{0.75}FO}%
\colorbox{green}{\color[gray]{0.75}FO}%
\colorbox{green}{\color[gray]{0.75}FO}%
\colorbox{green}{\color[gray]{0.75}FO}%
\colorbox{green}{\color[gray]{0.75}FO}%
\colorbox{green}{\color[gray]{0.75}FO}%
\colorbox{green}{\color[gray]{0.75}FO}%
\colorbox{green}{\color[gray]{0.75}FO}%
\colorbox{green}{\color[gray]{0.75}FO}%
\colorbox{green}{\color[gray]{0.75}FO}%
\colorbox{green}{\color[gray]{0.75}FO}%
\colorbox{green}{\color[gray]{0.75}FO}%
\colorbox{green}{\color[gray]{0.75}FO}%
\colorbox{green}{\color[gray]{0.75}FO}%
\colorbox{green}{\color[gray]{0.75}FO}%
\colorbox{green}{\color[gray]{0.75}FO}%
\colorbox{green}{\color[gray]{0.75}FO}%
\colorbox{green}{\color[gray]{0.75}FO}%
\colorbox{green}{\color[gray]{0.75}FO}%
\colorbox{green}{\color[gray]{0.75}FO}%
\colorbox{green}{\color[gray]{0.75}FO}%
\colorbox{green}{\color[gray]{0.75}FO}%
\colorbox{green}{\color[gray]{0.75}FO}%
\colorbox{green}{\color[gray]{0.75}FO}%
\colorbox{green}{\color[gray]{0.75}FO}%
\colorbox{green}{\color[gray]{0.75}FO}%
\colorbox{green}{\color[gray]{0.75}FO}%
\colorbox{green}{\color[gray]{0.75}FO}%
\colorbox{green}{\color[gray]{0.75}FO}%
\colorbox{green}{\color[gray]{0.75}FO}%
\colorbox{green}{\color[gray]{0.75}FO}%
\colorbox{green}{\color[gray]{0.75}FO}%
\colorbox{green}{\color[gray]{0.75}FO}%
\colorbox{green}{\color[gray]{0.75}FO}%
\colorbox{green}{\color[gray]{0.75}FO}%
\colorbox{green}{\color[gray]{0.75}FO}%
\colorbox{green}{\color[gray]{0.75}FO}%
\colorbox{green}{\color[gray]{0.75}FO}%
\colorbox{green}{\color[gray]{0.75}FO}%
\colorbox{green}{\color[gray]{0.75}FO}%
\colorbox{green}{\color[gray]{0.75}FO}%
\colorbox{green}{\color[gray]{0.75}FO}%
\colorbox{green}{\color[gray]{0.75}FO}%
\colorbox{green}{\color[gray]{0.75}FO}%
\colorbox{green}{\color[gray]{0.75}FO}%
\colorbox{green}{\color[gray]{0.75}FO}%
\colorbox{green}{\color[gray]{0.75}FO}%
\colorbox{green}{\color[gray]{0.75}FO}%
\colorbox{green}{\color[gray]{0.75}FO}%
\colorbox{green}{\color[gray]{0.75}FO}%
\colorbox{green}{\color[gray]{0.75}FO}%
\colorbox{green}{\color[gray]{0.75}FO}%
\colorbox{green}{\color[gray]{0.75}FO}%
\colorbox{green}{\color[gray]{0.75}FO}%
\colorbox{green}{\color[gray]{0.75}FO}%
\\
\colorbox{green}{\color[gray]{0.75}FO}%
\colorbox{green}{\color[gray]{0.75}FO}%
\colorbox{green}{\color[gray]{0.75}FO}%
\colorbox{green}{\color[gray]{0.75}FO}%
\colorbox{green}{\color[gray]{0.75}FO}%
\colorbox{green}{\color[gray]{0.75}FO}%
\colorbox{green}{\color[gray]{0.75}FO}%
\colorbox{green}{\color[gray]{0.75}FO}%
\colorbox{green}{\color[gray]{0.75}FO}%
\colorbox{green}{\color[gray]{0.75}FO}%
\colorbox{green}{\color[gray]{0.75}FO}%
\colorbox{green}{\color[gray]{0.75}FO}%
\colorbox{green}{\color[gray]{0.75}FO}%
\colorbox{green}{\color[gray]{0.75}FO}%
\colorbox{green}{\color[gray]{0.75}FO}%
\colorbox{green}{\color[gray]{0.75}FO}%
\colorbox{green}{\color[gray]{0.75}FO}%
\colorbox{green}{\color[gray]{0.75}FO}%
\colorbox{green}{\color[gray]{0.75}FO}%
\colorbox{green}{\color[gray]{0.75}FO}%
\colorbox{green}{\color[gray]{0.75}FO}%
\colorbox{green}{\color[gray]{0.75}FO}%
\colorbox{green}{\color[gray]{0.75}FO}%
\colorbox{green}{\color[gray]{0.75}FO}%
\colorbox{green}{\color[gray]{0.75}FO}%
\colorbox{green}{\color[gray]{0.75}FO}%
\colorbox{green}{\color[gray]{0.75}FO}%
\colorbox{green}{\color[gray]{0.75}FO}%
\colorbox{green}{\color[gray]{0.75}FO}%
\colorbox{green}{\color[gray]{0.75}FO}%
\colorbox{green}{\color[gray]{0.75}FO}%
\colorbox{green}{\color[gray]{0.75}FO}%
\colorbox{green}{\color[gray]{0.75}FO}%
\colorbox{green}{\color[gray]{0.75}FO}%
\colorbox{green}{\color[gray]{0.75}FO}%
\colorbox{green}{\color[gray]{0.75}FO}%
\colorbox{green}{\color[gray]{0.75}FO}%
\colorbox{green}{\color[gray]{0.75}FO}%
\colorbox{green}{\color[gray]{0.75}FO}%
\colorbox{green}{\color[gray]{0.75}FO}%
\colorbox{green}{\color[gray]{0.75}FO}%
\colorbox{green}{\color[gray]{0.75}FO}%
\colorbox{green}{\color[gray]{0.75}FO}%
\colorbox{green}{\color[gray]{0.75}FO}%
\colorbox{green}{\color[gray]{0.75}FO}%
\colorbox{green}{\color[gray]{0.75}FO}%
\colorbox{green}{\color[gray]{0.75}FO}%
\colorbox{green}{\color[gray]{0.75}FO}%
\colorbox{green}{\color[gray]{0.75}FO}%
\colorbox{green}{\color[gray]{0.75}FO}%
\colorbox{green}{\color[gray]{0.75}FO}%
\colorbox{green}{\color[gray]{0.75}FO}%
\colorbox{green}{\color[gray]{0.75}FO}%
\colorbox{green}{\color[gray]{0.75}FO}%
\colorbox{green}{\color[gray]{0.75}FO}%
\colorbox{green}{\color[gray]{0.75}FO}%
\colorbox{green}{\color[gray]{0.75}FO}%
\colorbox{green}{\color[gray]{0.75}FO}%
\colorbox{green}{\color[gray]{0.75}FO}%
\colorbox{green}{\color[gray]{0.75}FO}%
\colorbox{green}{\color[gray]{0.75}FO}%
\colorbox{green}{\color[gray]{0.75}FO}%
\colorbox{green}{\color[gray]{0.75}FO}%
\colorbox{green}{\color[gray]{0.75}FO}%
\colorbox{green}{\color[gray]{0.75}FO}%
\colorbox{green}{\color[gray]{0.75}FO}%
\colorbox{green}{\color[gray]{0.75}FO}%
\colorbox{green}{\color[gray]{0.75}FO}%
\colorbox{green}{\color[gray]{0.75}FO}%
\colorbox{green}{\color[gray]{0.75}FO}%
\colorbox{green}{\color[gray]{0.75}FO}%
\colorbox{green}{\color[gray]{0.75}FO}%
\colorbox{green}{\color[gray]{0.75}FO}%
\colorbox{green}{\color[gray]{0.75}FO}%
\colorbox{green}{\color[gray]{0.75}FO}%
\colorbox{green}{\color[gray]{0.75}FO}%
\colorbox{green}{\color[gray]{0.75}FO}%
\colorbox{green}{\color[gray]{0.75}FO}%
\colorbox{green}{\color[gray]{0.75}FO}%
\colorbox{green}{\color[gray]{0.75}FO}%
\colorbox{green}{\color[gray]{0.75}FO}%
\colorbox{green}{\color[gray]{0.75}FO}%
\colorbox{green}{\color[gray]{0.75}FO}%
\colorbox{green}{\color[gray]{0.75}FO}%
\colorbox{green}{\color[gray]{0.75}FO}%
\colorbox{green}{\color[gray]{0.75}FO}%
\colorbox{green}{\color[gray]{0.75}FO}%
\colorbox{green}{\color[gray]{0.75}FO}%
\colorbox{green}{\color[gray]{0.75}FO}%
\colorbox{green}{\color[gray]{0.75}FO}%
\colorbox{green}{\color[gray]{0.75}FO}%
\colorbox{green}{\color[gray]{0.75}FO}%
\colorbox{green}{\color[gray]{0.75}FO}%
\colorbox{green}{\color[gray]{0.75}FO}%
\colorbox{green}{\color[gray]{0.75}FO}%
\colorbox{green}{\color[gray]{0.75}FO}%
\colorbox{green}{\color[gray]{0.75}FO}%
\colorbox{green}{\color[gray]{0.75}FO}%
\colorbox{green}{\color[gray]{0.75}FO}%
\colorbox{green}{\color[gray]{0.75}FO}%
\\
\colorbox{green}{\color[gray]{0.75}FO}%
\colorbox{green}{\color[gray]{0.75}FO}%
\colorbox{green}{\color[gray]{0.75}FO}%
\colorbox{green}{\color[gray]{0.75}FO}%
\colorbox{green}{\color[gray]{0.75}FO}%
\colorbox{green}{\color[gray]{0.75}FO}%
\colorbox{green}{\color[gray]{0.75}FO}%
\colorbox{green}{\color[gray]{0.75}FO}%
\colorbox{green}{\color[gray]{0.75}FO}%
\colorbox{green}{\color[gray]{0.75}FO}%
\colorbox{green}{\color[gray]{0.75}FO}%
\colorbox{green}{\color[gray]{0.75}FO}%
\colorbox{green}{\color[gray]{0.75}FO}%
\colorbox{green}{\color[gray]{0.75}FO}%
\colorbox{green}{\color[gray]{0.75}FO}%
\colorbox{green}{\color[gray]{0.75}FO}%
\colorbox{green}{\color[gray]{0.75}FO}%
\colorbox{green}{\color[gray]{0.75}FO}%
\colorbox{green}{\color[gray]{0.75}FO}%
\colorbox{green}{\color[gray]{0.75}FO}%
\colorbox{green}{\color[gray]{0.75}FO}%
\colorbox{green}{\color[gray]{0.75}FO}%
\colorbox{green}{\color[gray]{0.75}FO}%
\colorbox{green}{\color[gray]{0.75}FO}%
\colorbox{green}{\color[gray]{0.75}FO}%
\colorbox{green}{\color[gray]{0.75}FO}%
\colorbox{green}{\color[gray]{0.75}FO}%
\colorbox{green}{\color[gray]{0.75}FO}%
\colorbox{green}{\color[gray]{0.75}FO}%
\colorbox{green}{\color[gray]{0.75}FO}%
\colorbox{green}{\color[gray]{0.75}FO}%
\colorbox{green}{\color[gray]{0.75}FO}%
\colorbox{green}{\color[gray]{0.75}FO}%
\colorbox{green}{\color[gray]{0.75}FO}%
\colorbox{green}{\color[gray]{0.75}FO}%
\colorbox{green}{\color[gray]{0.75}FO}%
\colorbox{green}{\color[gray]{0.75}FO}%
\colorbox{green}{\color[gray]{0.75}FO}%
\colorbox{green}{\color[gray]{0.75}FO}%
\colorbox{green}{\color[gray]{0.75}FO}%
\colorbox{green}{\color[gray]{0.75}FO}%
\colorbox{green}{\color[gray]{0.75}FO}%
\colorbox{green}{\color[gray]{0.75}FO}%
\colorbox{green}{\color[gray]{0.75}FO}%
\colorbox{green}{\color[gray]{0.75}FO}%
\colorbox{green}{\color[gray]{0.75}FO}%
\colorbox{green}{\color[gray]{0.75}FO}%
\colorbox{green}{\color[gray]{0.75}FO}%
\colorbox{green}{\color[gray]{0.75}FO}%
\colorbox{green}{\color[gray]{0.75}FO}%
\colorbox{green}{\color[gray]{0.75}FO}%
\colorbox{green}{\color[gray]{0.75}FO}%
\colorbox{green}{\color[gray]{0.75}FO}%
\colorbox{green}{\color[gray]{0.75}FO}%
\colorbox{green}{\color[gray]{0.75}FO}%
\colorbox{green}{\color[gray]{0.75}FO}%
\colorbox{green}{\color[gray]{0.75}FO}%
\colorbox{green}{\color[gray]{0.75}FO}%
\colorbox{green}{\color[gray]{0.75}FO}%
\colorbox{green}{\color[gray]{0.75}FO}%
\colorbox{green}{\color[gray]{0.75}FO}%
\colorbox{green}{\color[gray]{0.75}FO}%
\colorbox{green}{\color[gray]{0.75}FO}%
\colorbox{green}{\color[gray]{0.75}FO}%
\colorbox{green}{\color[gray]{0.75}FO}%
\colorbox{green}{\color[gray]{0.75}FO}%
\colorbox{green}{\color[gray]{0.75}FO}%
\colorbox{green}{\color[gray]{0.75}FO}%
\colorbox{green}{\color[gray]{0.75}FO}%
\colorbox{green}{\color[gray]{0.75}FO}%
\colorbox{green}{\color[gray]{0.75}FO}%
\colorbox{green}{\color[gray]{0.75}FO}%
\colorbox{green}{\color[gray]{0.75}FO}%
\colorbox{green}{\color[gray]{0.75}FO}%
\colorbox{green}{\color[gray]{0.75}FO}%
\colorbox{green}{\color[gray]{0.75}FO}%
\colorbox{green}{\color[gray]{0.75}FO}%
\colorbox{green}{\color[gray]{0.75}FO}%
\colorbox{green}{\color[gray]{0.75}FO}%
\colorbox{green}{\color[gray]{0.75}FO}%
\colorbox{green}{\color[gray]{0.75}FO}%
\colorbox{green}{\color[gray]{0.75}FO}%
\colorbox{green}{\color[gray]{0.75}FO}%
\colorbox{green}{\color[gray]{0.75}FO}%
\colorbox{green}{\color[gray]{0.75}FO}%
\colorbox{green}{\color[gray]{0.75}FO}%
\colorbox{green}{\color[gray]{0.75}FO}%
\colorbox{green}{\color[gray]{0.75}FO}%
\colorbox{green}{\color[gray]{0.75}FO}%
\colorbox{green}{\color[gray]{0.75}FO}%
\colorbox{green}{\color[gray]{0.75}FO}%
\colorbox{green}{\color[gray]{0.75}FO}%
\colorbox{green}{\color[gray]{0.75}FO}%
\colorbox{green}{\color[gray]{0.75}FO}%
\colorbox{green}{\color[gray]{0.75}FO}%
\colorbox{green}{\color[gray]{0.75}FO}%
\colorbox{green}{\color[gray]{0.75}FO}%
\colorbox{green}{\color[gray]{0.75}FO}%
\colorbox{green}{\color[gray]{0.75}FO}%
\colorbox{green}{\color[gray]{0.75}FO}%
\\
\colorbox{green}{\color[gray]{0.75}FO}%
\colorbox{green}{\color[gray]{0.75}FO}%
\colorbox{green}{\color[gray]{0.75}FO}%
\colorbox{green}{\color[gray]{0.75}FO}%
\colorbox{green}{\color[gray]{0.75}FO}%
\colorbox{green}{\color[gray]{0.75}FO}%
\colorbox{green}{\color[gray]{0.75}FO}%
\colorbox{green}{\color[gray]{0.75}FO}%
\colorbox{green}{\color[gray]{0.75}FO}%
\colorbox{green}{\color[gray]{0.75}FO}%
\colorbox{green}{\color[gray]{0.75}FO}%
\colorbox{green}{\color[gray]{0.75}FO}%
\colorbox{green}{\color[gray]{0.75}FO}%
\colorbox{green}{\color[gray]{0.75}FO}%
\colorbox{green}{\color[gray]{0.75}FO}%
\colorbox{green}{\color[gray]{0.75}FO}%
\colorbox{green}{\color[gray]{0.75}FO}%
\colorbox{green}{\color[gray]{0.75}FO}%
\colorbox{green}{\color[gray]{0.75}FO}%
\colorbox{green}{\color[gray]{0.75}FO}%
\colorbox{green}{\color[gray]{0.75}FO}%
\colorbox{green}{\color[gray]{0.75}FO}%
\colorbox{green}{\color[gray]{0.75}FO}%
\colorbox{green}{\color[gray]{0.75}FO}%
\colorbox{green}{\color[gray]{0.75}FO}%
\colorbox{green}{\color[gray]{0.75}FO}%
\colorbox{green}{\color[gray]{0.75}FO}%
\colorbox{green}{\color[gray]{0.75}FO}%
\colorbox{green}{\color[gray]{0.75}FO}%
\colorbox{green}{\color[gray]{0.75}FO}%
\colorbox{green}{\color[gray]{0.75}FO}%
\colorbox{green}{\color[gray]{0.75}FO}%
\colorbox{green}{\color[gray]{0.75}FO}%
\colorbox{green}{\color[gray]{0.75}FO}%
\colorbox{green}{\color[gray]{0.75}FO}%
\colorbox{green}{\color[gray]{0.75}FO}%
\colorbox{green}{\color[gray]{0.75}FO}%
\colorbox{green}{\color[gray]{0.75}FO}%
\colorbox{green}{\color[gray]{0.75}FO}%
\colorbox{green}{\color[gray]{0.75}FO}%
\colorbox{green}{\color[gray]{0.75}FO}%
\colorbox{green}{\color[gray]{0.75}FO}%
\colorbox{green}{\color[gray]{0.75}FO}%
\colorbox{green}{\color[gray]{0.75}FO}%
\colorbox{green}{\color[gray]{0.75}FO}%
\colorbox{green}{\color[gray]{0.75}FO}%
\colorbox{green}{\color[gray]{0.75}FO}%
\colorbox{green}{\color[gray]{0.75}FO}%
\colorbox{green}{\color[gray]{0.75}FO}%
\colorbox{green}{\color[gray]{0.75}FO}%
\colorbox{green}{\color[gray]{0.75}FO}%
\colorbox{green}{\color[gray]{0.75}FO}%
\colorbox{green}{\color[gray]{0.75}FO}%
\colorbox{green}{\color[gray]{0.75}FO}%
\colorbox{green}{\color[gray]{0.75}FO}%
\colorbox{green}{\color[gray]{0.75}FO}%
\colorbox{green}{\color[gray]{0.75}FO}%
\colorbox{green}{\color[gray]{0.75}FO}%
\colorbox{green}{\color[gray]{0.75}FO}%
\colorbox{green}{\color[gray]{0.75}FO}%
\colorbox{green}{\color[gray]{0.75}FO}%
\colorbox{green}{\color[gray]{0.75}FO}%
\colorbox{green}{\color[gray]{0.75}FO}%
\colorbox{green}{\color[gray]{0.75}FO}%
\colorbox{green}{\color[gray]{0.75}FO}%
\colorbox{green}{\color[gray]{0.75}FO}%
\colorbox{green}{\color[gray]{0.75}FO}%
\colorbox{green}{\color[gray]{0.75}FO}%
\colorbox{green}{\color[gray]{0.75}FO}%
\colorbox{green}{\color[gray]{0.75}FO}%
\colorbox{green}{\color[gray]{0.75}FO}%
\colorbox{green}{\color[gray]{0.75}FO}%
\colorbox{green}{\color[gray]{0.75}FO}%
\colorbox{green}{\color[gray]{0.75}FO}%
\colorbox{green}{\color[gray]{0.75}FO}%
\colorbox{green}{\color[gray]{0.75}FO}%
\colorbox{green}{\color[gray]{0.75}FO}%
\colorbox{green}{\color[gray]{0.75}FO}%
\colorbox{green}{\color[gray]{0.75}FO}%
\colorbox{green}{\color[gray]{0.75}FO}%
\colorbox{green}{\color[gray]{0.75}FO}%
\colorbox{green}{\color[gray]{0.75}FO}%
\colorbox{green}{\color[gray]{0.75}FO}%
\colorbox{green}{\color[gray]{0.75}FO}%
\colorbox{green}{\color[gray]{0.75}FO}%
\colorbox{green}{\color[gray]{0.75}FO}%
\colorbox{green}{\color[gray]{0.75}FO}%
\colorbox{green}{\color[gray]{0.75}FO}%
\colorbox{green}{\color[gray]{0.75}FO}%
\colorbox{green}{\color[gray]{0.75}FO}%
\colorbox{green}{\color[gray]{0.75}FO}%
\colorbox{green}{\color[gray]{0.75}FO}%
\colorbox{green}{\color[gray]{0.75}FO}%
\colorbox{green}{\color[gray]{0.75}FO}%
\colorbox{green}{\color[gray]{0.75}FO}%
\colorbox{green}{\color[gray]{0.75}FO}%
\colorbox{green}{\color[gray]{0.75}FO}%
\colorbox{green}{\color[gray]{0.75}FO}%
\colorbox{green}{\color[gray]{0.75}FO}%
\colorbox{green}{\color[gray]{0.75}FO}%
\\
\colorbox{green}{\color[gray]{0.75}FO}%
\colorbox{green}{\color[gray]{0.75}FO}%
\colorbox{green}{\color[gray]{0.75}FO}%
\colorbox{green}{\color[gray]{0.75}FO}%
\colorbox{green}{\color[gray]{0.75}FO}%
\colorbox{green}{\color[gray]{0.75}FO}%
\colorbox{green}{\color[gray]{0.75}FO}%
\colorbox{green}{\color[gray]{0.75}FO}%
\colorbox{green}{\color[gray]{0.75}FO}%
\colorbox{green}{\color[gray]{0.75}FO}%
\colorbox{green}{\color[gray]{0.75}FO}%
\colorbox{green}{\color[gray]{0.75}FO}%
\colorbox{green}{\color[gray]{0.75}FO}%
\colorbox{green}{\color[gray]{0.75}FO}%
\colorbox{green}{\color[gray]{0.75}FO}%
\colorbox{green}{\color[gray]{0.75}FO}%
\colorbox{green}{\color[gray]{0.75}FO}%
\colorbox{green}{\color[gray]{0.75}FO}%
\colorbox{green}{\color[gray]{0.75}FO}%
\colorbox{green}{\color[gray]{0.75}FO}%
\colorbox{green}{\color[gray]{0.75}FO}%
\colorbox{green}{\color[gray]{0.75}FO}%
\colorbox{green}{\color[gray]{0.75}FO}%
\colorbox{green}{\color[gray]{0.75}FO}%
\colorbox{green}{\color[gray]{0.75}FO}%
\colorbox{green}{\color[gray]{0.75}FO}%
\colorbox{green}{\color[gray]{0.75}FO}%
\colorbox{green}{\color[gray]{0.75}FO}%
\colorbox{green}{\color[gray]{0.75}FO}%
\colorbox{green}{\color[gray]{0.75}FO}%
\colorbox{green}{\color[gray]{0.75}FO}%
\colorbox{green}{\color[gray]{0.75}FO}%
\colorbox{green}{\color[gray]{0.75}FO}%
\colorbox{green}{\color[gray]{0.75}FO}%
\colorbox{green}{\color[gray]{0.75}FO}%
\colorbox{green}{\color[gray]{0.75}FO}%
\colorbox{green}{\color[gray]{0.75}FO}%
\colorbox{green}{\color[gray]{0.75}FO}%
\colorbox{green}{\color[gray]{0.75}FO}%
\colorbox{green}{\color[gray]{0.75}FO}%
\colorbox{green}{\color[gray]{0.75}FO}%
\colorbox{green}{\color[gray]{0.75}FO}%
\colorbox{green}{\color[gray]{0.75}FO}%
\colorbox{green}{\color[gray]{0.75}FO}%
\colorbox{green}{\color[gray]{0.75}FO}%
\colorbox{green}{\color[gray]{0.75}FO}%
\colorbox{green}{\color[gray]{0.75}FO}%
\colorbox{green}{\color[gray]{0.75}FO}%
\colorbox{green}{\color[gray]{0.75}FO}%
\colorbox{green}{\color[gray]{0.75}FO}%
\colorbox{green}{\color[gray]{0.75}FO}%
\colorbox{green}{\color[gray]{0.75}FO}%
\colorbox{green}{\color[gray]{0.75}FO}%
\colorbox{green}{\color[gray]{0.75}FO}%
\colorbox{green}{\color[gray]{0.75}FO}%
\colorbox{green}{\color[gray]{0.75}FO}%
\colorbox{green}{\color[gray]{0.75}FO}%
\colorbox{green}{\color[gray]{0.75}FO}%
\colorbox{green}{\color[gray]{0.75}FO}%
\colorbox{green}{\color[gray]{0.75}FO}%
\colorbox{green}{\color[gray]{0.75}FO}%
\colorbox{green}{\color[gray]{0.75}FO}%
\colorbox{green}{\color[gray]{0.75}FO}%
\colorbox{green}{\color[gray]{0.75}FO}%
\colorbox{green}{\color[gray]{0.75}FO}%
\colorbox{green}{\color[gray]{0.75}FO}%
\colorbox{green}{\color[gray]{0.75}FO}%
\colorbox{green}{\color[gray]{0.75}FO}%
\colorbox{green}{\color[gray]{0.75}FO}%
\colorbox{green}{\color[gray]{0.75}FO}%
\colorbox{green}{\color[gray]{0.75}FO}%
\colorbox{green}{\color[gray]{0.75}FO}%
\colorbox{green}{\color[gray]{0.75}FO}%
\colorbox{green}{\color[gray]{0.75}FO}%
\colorbox{green}{\color[gray]{0.75}FO}%
\colorbox{green}{\color[gray]{0.75}FO}%
\colorbox{green}{\color[gray]{0.75}FO}%
\colorbox{green}{\color[gray]{0.75}FO}%
\colorbox{green}{\color[gray]{0.75}FO}%
\colorbox{green}{\color[gray]{0.75}FO}%
\colorbox{green}{\color[gray]{0.75}FO}%
\colorbox{green}{\color[gray]{0.75}FO}%
\colorbox{green}{\color[gray]{0.75}FO}%
\colorbox{green}{\color[gray]{0.75}FO}%
\colorbox{green}{\color[gray]{0.75}FO}%
\colorbox{green}{\color[gray]{0.75}FO}%
\colorbox{green}{\color[gray]{0.75}FO}%
\colorbox{green}{\color[gray]{0.75}FO}%
\colorbox{green}{\color[gray]{0.75}FO}%
\colorbox{green}{\color[gray]{0.75}FO}%
\colorbox{green}{\color[gray]{0.75}FO}%
\colorbox{green}{\color[gray]{0.75}FO}%
\colorbox{green}{\color[gray]{0.75}FO}%
\colorbox{green}{\color[gray]{0.75}FO}%
\colorbox{green}{\color[gray]{0.75}FO}%
\colorbox{green}{\color[gray]{0.75}FO}%
\colorbox{green}{\color[gray]{0.75}FO}%
\colorbox{green}{\color[gray]{0.75}FO}%
\colorbox{green}{\color[gray]{0.75}FO}%
\colorbox{green}{\color[gray]{0.75}FO}%
\\
\colorbox{green}{\color[gray]{0.75}FO}%
\colorbox{green}{\color[gray]{0.75}FO}%
\colorbox{green}{\color[gray]{0.75}FO}%
\colorbox{green}{\color[gray]{0.75}FO}%
\colorbox{green}{\color[gray]{0.75}FO}%
\colorbox{green}{\color[gray]{0.75}FO}%
\colorbox{green}{\color[gray]{0.75}FO}%
\colorbox{green}{\color[gray]{0.75}FO}%
\colorbox{green}{\color[gray]{0.75}FO}%
\colorbox{green}{\color[gray]{0.75}FO}%
\colorbox{green}{\color[gray]{0.75}FO}%
\colorbox{green}{\color[gray]{0.75}FO}%
\colorbox{green}{\color[gray]{0.75}FO}%
\colorbox{green}{\color[gray]{0.75}FO}%
\colorbox{green}{\color[gray]{0.75}FO}%
\colorbox{green}{\color[gray]{0.75}FO}%
\colorbox{green}{\color[gray]{0.75}FO}%
\colorbox{green}{\color[gray]{0.75}FO}%
\colorbox{green}{\color[gray]{0.75}FO}%
\colorbox{green}{\color[gray]{0.75}FO}%
\colorbox{green}{\color[gray]{0.75}FO}%
\colorbox{green}{\color[gray]{0.75}FO}%
\colorbox{green}{\color[gray]{0.75}FO}%
\colorbox{green}{\color[gray]{0.75}FO}%
\colorbox{green}{\color[gray]{0.75}FO}%
\colorbox{green}{\color[gray]{0.75}FO}%
\colorbox{green}{\color[gray]{0.75}FO}%
\colorbox{green}{\color[gray]{0.75}FO}%
\colorbox{green}{\color[gray]{0.75}FO}%
\colorbox{green}{\color[gray]{0.75}FO}%
\colorbox{green}{\color[gray]{0.75}FO}%
\colorbox{green}{\color[gray]{0.75}FO}%
\colorbox{green}{\color[gray]{0.75}FO}%
\colorbox{green}{\color[gray]{0.75}FO}%
\colorbox{green}{\color[gray]{0.75}FO}%
\colorbox{green}{\color[gray]{0.75}FO}%
\colorbox{green}{\color[gray]{0.75}FO}%
\colorbox{green}{\color[gray]{0.75}FO}%
\colorbox{green}{\color[gray]{0.75}FO}%
\colorbox{green}{\color[gray]{0.75}FO}%
\colorbox{green}{\color[gray]{0.75}FO}%
\colorbox{green}{\color[gray]{0.75}FO}%
\colorbox{green}{\color[gray]{0.75}FO}%
\colorbox{green}{\color[gray]{0.75}FO}%
\colorbox{green}{\color[gray]{0.75}FO}%
\colorbox{green}{\color[gray]{0.75}FO}%
\colorbox{green}{\color[gray]{0.75}FO}%
\colorbox{green}{\color[gray]{0.75}FO}%
\colorbox{green}{\color[gray]{0.75}FO}%
\colorbox{green}{\color[gray]{0.75}FO}%
\colorbox{green}{\color[gray]{0.75}FO}%
\colorbox{green}{\color[gray]{0.75}FO}%
\colorbox{green}{\color[gray]{0.75}FO}%
\colorbox{green}{\color[gray]{0.75}FO}%
\colorbox{green}{\color[gray]{0.75}FO}%
\colorbox{green}{\color[gray]{0.75}FO}%
\colorbox{green}{\color[gray]{0.75}FO}%
\colorbox{green}{\color[gray]{0.75}FO}%
\colorbox{green}{\color[gray]{0.75}FO}%
\colorbox{green}{\color[gray]{0.75}FO}%
\colorbox{green}{\color[gray]{0.75}FO}%
\colorbox{green}{\color[gray]{0.75}FO}%
\colorbox{green}{\color[gray]{0.75}FO}%
\colorbox{green}{\color[gray]{0.75}FO}%
\colorbox{green}{\color[gray]{0.75}FO}%
\colorbox{green}{\color[gray]{0.75}FO}%
\colorbox{green}{\color[gray]{0.75}FO}%
\colorbox{green}{\color[gray]{0.75}FO}%
\colorbox{green}{\color[gray]{0.75}FO}%
\colorbox{green}{\color[gray]{0.75}FO}%
\colorbox{green}{\color[gray]{0.75}FO}%
\colorbox{green}{\color[gray]{0.75}FO}%
\colorbox{green}{\color[gray]{0.75}FO}%
\colorbox{green}{\color[gray]{0.75}FO}%
\colorbox{green}{\color[gray]{0.75}FO}%
\colorbox{green}{\color[gray]{0.75}FO}%
\colorbox{green}{\color[gray]{0.75}FO}%
\colorbox{green}{\color[gray]{0.75}FO}%
\colorbox{green}{\color[gray]{0.75}FO}%
\colorbox{green}{\color[gray]{0.75}FO}%
\colorbox{green}{\color[gray]{0.75}FO}%
\colorbox{green}{\color[gray]{0.75}FO}%
\colorbox{green}{\color[gray]{0.75}FO}%
\colorbox{green}{\color[gray]{0.75}FO}%
\colorbox{green}{\color[gray]{0.75}FO}%
\colorbox{green}{\color[gray]{0.75}FO}%
\colorbox{green}{\color[gray]{0.75}FO}%
\colorbox{green}{\color[gray]{0.75}FO}%
\colorbox{green}{\color[gray]{0.75}FO}%
\colorbox{green}{\color[gray]{0.75}FO}%
\colorbox{green}{\color[gray]{0.75}FO}%
\colorbox{green}{\color[gray]{0.75}FO}%
\colorbox{green}{\color[gray]{0.75}FO}%
\colorbox{green}{\color[gray]{0.75}FO}%
\colorbox{green}{\color[gray]{0.75}FO}%
\colorbox{green}{\color[gray]{0.75}FO}%
\colorbox{green}{\color[gray]{0.75}FO}%
\colorbox{green}{\color[gray]{0.75}FO}%
\colorbox{green}{\color[gray]{0.75}FO}%
\colorbox{green}{\color[gray]{0.75}FO}%
\\
\colorbox{green}{\color[gray]{0.75}FO}%
\colorbox{green}{\color[gray]{0.75}FO}%
\colorbox{green}{\color[gray]{0.75}FO}%
\colorbox{green}{\color[gray]{0.75}FO}%
\colorbox{green}{\color[gray]{0.75}FO}%
\colorbox{green}{\color[gray]{0.75}FO}%
\colorbox{green}{\color[gray]{0.75}FO}%
\colorbox{green}{\color[gray]{0.75}FO}%
\colorbox{green}{\color[gray]{0.75}FO}%
\colorbox{green}{\color[gray]{0.75}FO}%
\colorbox{green}{\color[gray]{0.75}FO}%
\colorbox{green}{\color[gray]{0.75}FO}%
\colorbox{green}{\color[gray]{0.75}FO}%
\colorbox{green}{\color[gray]{0.75}FO}%
\colorbox{green}{\color[gray]{0.75}FO}%
\colorbox{green}{\color[gray]{0.75}FO}%
\colorbox{green}{\color[gray]{0.75}FO}%
\colorbox{green}{\color[gray]{0.75}FO}%
\colorbox{green}{\color[gray]{0.75}FO}%
\colorbox{green}{\color[gray]{0.75}FO}%
\colorbox{green}{\color[gray]{0.75}FO}%
\colorbox{green}{\color[gray]{0.75}FO}%
\colorbox{green}{\color[gray]{0.75}FO}%
\colorbox{green}{\color[gray]{0.75}FO}%
\colorbox{green}{\color[gray]{0.75}FO}%
\colorbox{green}{\color[gray]{0.75}FO}%
\colorbox{green}{\color[gray]{0.75}FO}%
\colorbox{green}{\color[gray]{0.75}FO}%
\colorbox{green}{\color[gray]{0.75}FO}%
\colorbox{green}{\color[gray]{0.75}FO}%
\colorbox{green}{\color[gray]{0.75}FO}%
\colorbox{green}{\color[gray]{0.75}FO}%
\colorbox{green}{\color[gray]{0.75}FO}%
\colorbox{green}{\color[gray]{0.75}FO}%
\colorbox{green}{\color[gray]{0.75}FO}%
\colorbox{green}{\color[gray]{0.75}FO}%
\colorbox{green}{\color[gray]{0.75}FO}%
\colorbox{green}{\color[gray]{0.75}FO}%
\colorbox{green}{\color[gray]{0.75}FO}%
\colorbox{green}{\color[gray]{0.75}FO}%
\colorbox{green}{\color[gray]{0.75}FO}%
\colorbox{green}{\color[gray]{0.75}FO}%
\colorbox{green}{\color[gray]{0.75}FO}%
\colorbox{green}{\color[gray]{0.75}FO}%
\colorbox{green}{\color[gray]{0.75}FO}%
\colorbox{green}{\color[gray]{0.75}FO}%
\colorbox{green}{\color[gray]{0.75}FO}%
\colorbox{green}{\color[gray]{0.75}FO}%
\colorbox{green}{\color[gray]{0.75}FO}%
\colorbox{green}{\color[gray]{0.75}FO}%
\colorbox{green}{\color[gray]{0.75}FO}%
\colorbox{green}{\color[gray]{0.75}FO}%
\colorbox{green}{\color[gray]{0.75}FO}%
\colorbox{green}{\color[gray]{0.75}FO}%
\colorbox{green}{\color[gray]{0.75}FO}%
\colorbox{green}{\color[gray]{0.75}FO}%
\colorbox{green}{\color[gray]{0.75}FO}%
\colorbox{green}{\color[gray]{0.75}FO}%
\colorbox{green}{\color[gray]{0.75}FO}%
\colorbox{green}{\color[gray]{0.75}FO}%
\colorbox{green}{\color[gray]{0.75}FO}%
\colorbox{green}{\color[gray]{0.75}FO}%
\colorbox{green}{\color[gray]{0.75}FO}%
\colorbox{green}{\color[gray]{0.75}FO}%
\colorbox{green}{\color[gray]{0.75}FO}%
\colorbox{green}{\color[gray]{0.75}FO}%
\colorbox{green}{\color[gray]{0.75}FO}%
\colorbox{green}{\color[gray]{0.75}FO}%
\colorbox{green}{\color[gray]{0.75}FO}%
\colorbox{green}{\color[gray]{0.75}FO}%
\colorbox{green}{\color[gray]{0.75}FO}%
\colorbox{green}{\color[gray]{0.75}FO}%
\colorbox{green}{\color[gray]{0.75}FO}%
\colorbox{green}{\color[gray]{0.75}FO}%
\colorbox{green}{\color[gray]{0.75}FO}%
\colorbox{green}{\color[gray]{0.75}FO}%
\colorbox{green}{\color[gray]{0.75}FO}%
\colorbox{green}{\color[gray]{0.75}FO}%
\colorbox{green}{\color[gray]{0.75}FO}%
\colorbox{green}{\color[gray]{0.75}FO}%
\colorbox{green}{\color[gray]{0.75}FO}%
\colorbox{green}{\color[gray]{0.75}FO}%
\colorbox{green}{\color[gray]{0.75}FO}%
\colorbox{green}{\color[gray]{0.75}FO}%
\colorbox{green}{\color[gray]{0.75}FO}%
\colorbox{green}{\color[gray]{0.75}FO}%
\colorbox{green}{\color[gray]{0.75}FO}%
\colorbox{green}{\color[gray]{0.75}FO}%
\colorbox{green}{\color[gray]{0.75}FO}%
\colorbox{green}{\color[gray]{0.75}FO}%
\colorbox{green}{\color[gray]{0.75}FO}%
\colorbox{green}{\color[gray]{0.75}FO}%
\colorbox{green}{\color[gray]{0.75}FO}%
\colorbox{green}{\color[gray]{0.75}FO}%
\colorbox{green}{\color[gray]{0.75}FO}%
\colorbox{green}{\color[gray]{0.75}FO}%
\colorbox{green}{\color[gray]{0.75}FO}%
\colorbox{green}{\color[gray]{0.75}FO}%
\colorbox{green}{\color[gray]{0.75}FO}%
\colorbox{green}{\color[gray]{0.75}FO}%
\\
\colorbox{green}{\color[gray]{0.75}FO}%
\colorbox{green}{\color[gray]{0.75}FO}%
\colorbox{green}{\color[gray]{0.75}FO}%
\colorbox{green}{\color[gray]{0.75}FO}%
\colorbox{green}{\color[gray]{0.75}FO}%
\colorbox{green}{\color[gray]{0.75}FO}%
\colorbox{green}{\color[gray]{0.75}FO}%
\colorbox{green}{\color[gray]{0.75}FO}%
\colorbox{green}{\color[gray]{0.75}FO}%
\colorbox{green}{\color[gray]{0.75}FO}%
\colorbox{green}{\color[gray]{0.75}FO}%
\colorbox{green}{\color[gray]{0.75}FO}%
\colorbox{green}{\color[gray]{0.75}FO}%
\colorbox{green}{\color[gray]{0.75}FO}%
\colorbox{green}{\color[gray]{0.75}FO}%
\colorbox{green}{\color[gray]{0.75}FO}%
\colorbox{green}{\color[gray]{0.75}FO}%
\colorbox{green}{\color[gray]{0.75}FO}%
\colorbox{green}{\color[gray]{0.75}FO}%
\colorbox{green}{\color[gray]{0.75}FO}%
\colorbox{green}{\color[gray]{0.75}FO}%
\colorbox{green}{\color[gray]{0.75}FO}%
\colorbox{green}{\color[gray]{0.75}FO}%
\colorbox{green}{\color[gray]{0.75}FO}%
\colorbox{green}{\color[gray]{0.75}FO}%
\colorbox{green}{\color[gray]{0.75}FO}%
\colorbox{green}{\color[gray]{0.75}FO}%
\colorbox{green}{\color[gray]{0.75}FO}%
\colorbox{green}{\color[gray]{0.75}FO}%
\colorbox{green}{\color[gray]{0.75}FO}%
\colorbox{green}{\color[gray]{0.75}FO}%
\colorbox{green}{\color[gray]{0.75}FO}%
\colorbox{green}{\color[gray]{0.75}FO}%
\colorbox{green}{\color[gray]{0.75}FO}%
\colorbox{green}{\color[gray]{0.75}FO}%
\colorbox{green}{\color[gray]{0.75}FO}%
\colorbox{green}{\color[gray]{0.75}FO}%
\colorbox{green}{\color[gray]{0.75}FO}%
\colorbox{green}{\color[gray]{0.75}FO}%
\colorbox{green}{\color[gray]{0.75}FO}%
\colorbox{green}{\color[gray]{0.75}FO}%
\colorbox{green}{\color[gray]{0.75}FO}%
\colorbox{green}{\color[gray]{0.75}FO}%
\colorbox{green}{\color[gray]{0.75}FO}%
\colorbox{green}{\color[gray]{0.75}FO}%
\colorbox{green}{\color[gray]{0.75}FO}%
\colorbox{green}{\color[gray]{0.75}FO}%
\colorbox{green}{\color[gray]{0.75}FO}%
\colorbox{green}{\color[gray]{0.75}FO}%
\colorbox{green}{\color[gray]{0.75}FO}%
\colorbox{green}{\color[gray]{0.75}FO}%
\colorbox{green}{\color[gray]{0.75}FO}%
\colorbox{green}{\color[gray]{0.75}FO}%
\colorbox{green}{\color[gray]{0.75}FO}%
\colorbox{green}{\color[gray]{0.75}FO}%
\colorbox{green}{\color[gray]{0.75}FO}%
\colorbox{green}{\color[gray]{0.75}FO}%
\colorbox{green}{\color[gray]{0.75}FO}%
\colorbox{green}{\color[gray]{0.75}FO}%
\colorbox{green}{\color[gray]{0.75}FO}%
\colorbox{green}{\color[gray]{0.75}FO}%
\colorbox{green}{\color[gray]{0.75}FO}%
\colorbox{green}{\color[gray]{0.75}FO}%
\colorbox{green}{\color[gray]{0.75}FO}%
\colorbox{green}{\color[gray]{0.75}FO}%
\colorbox{green}{\color[gray]{0.75}FO}%
\colorbox{green}{\color[gray]{0.75}FO}%
\colorbox{green}{\color[gray]{0.75}FO}%
\colorbox{green}{\color[gray]{0.75}FO}%
\colorbox{green}{\color[gray]{0.75}FO}%
\colorbox{green}{\color[gray]{0.75}FO}%
\colorbox{green}{\color[gray]{0.75}FO}%
\colorbox{green}{\color[gray]{0.75}FO}%
\colorbox{green}{\color[gray]{0.75}FO}%
\colorbox{green}{\color[gray]{0.75}FO}%
\colorbox{green}{\color[gray]{0.75}FO}%
\colorbox{green}{\color[gray]{0.75}FO}%
\colorbox{green}{\color[gray]{0.75}FO}%
\colorbox{green}{\color[gray]{0.75}FO}%
\colorbox{green}{\color[gray]{0.75}FO}%
\colorbox{green}{\color[gray]{0.75}FO}%
\colorbox{green}{\color[gray]{0.75}FO}%
\colorbox{green}{\color[gray]{0.75}FO}%
\colorbox{green}{\color[gray]{0.75}FO}%
\colorbox{green}{\color[gray]{0.75}FO}%
\colorbox{green}{\color[gray]{0.75}FO}%
\colorbox{green}{\color[gray]{0.75}FO}%
\colorbox{green}{\color[gray]{0.75}FO}%
\colorbox{green}{\color[gray]{0.75}FO}%
\colorbox{green}{\color[gray]{0.75}FO}%
\colorbox{green}{\color[gray]{0.75}FO}%
\colorbox{green}{\color[gray]{0.75}FO}%
\colorbox{green}{\color[gray]{0.75}FO}%
\colorbox{green}{\color[gray]{0.75}FO}%
\colorbox{green}{\color[gray]{0.75}FO}%
\colorbox{green}{\color[gray]{0.75}FO}%
\colorbox{green}{\color[gray]{0.75}FO}%
\colorbox{green}{\color[gray]{0.75}FO}%
\colorbox{green}{\color[gray]{0.75}FO}%
\colorbox{green}{\color[gray]{0.75}FO}%
\\
\colorbox{green}{\color[gray]{0.75}FO}%
\colorbox{green}{\color[gray]{0.75}FO}%
\colorbox{green}{\color[gray]{0.75}FO}%
\colorbox{green}{\color[gray]{0.75}FO}%
\colorbox{green}{\color[gray]{0.75}FO}%
\colorbox{green}{\color[gray]{0.75}FO}%
\colorbox{green}{\color[gray]{0.75}FO}%
\colorbox{green}{\color[gray]{0.75}FO}%
\colorbox{green}{\color[gray]{0.75}FO}%
\colorbox{green}{\color[gray]{0.75}FO}%
\colorbox{green}{\color[gray]{0.75}FO}%
\colorbox{green}{\color[gray]{0.75}FO}%
\colorbox{green}{\color[gray]{0.75}FO}%
\colorbox{green}{\color[gray]{0.75}FO}%
\colorbox{green}{\color[gray]{0.75}FO}%
\colorbox{green}{\color[gray]{0.75}FO}%
\colorbox{green}{\color[gray]{0.75}FO}%
\colorbox{green}{\color[gray]{0.75}FO}%
\colorbox{green}{\color[gray]{0.75}FO}%
\colorbox{green}{\color[gray]{0.75}FO}%
\colorbox{green}{\color[gray]{0.75}FO}%
\colorbox{green}{\color[gray]{0.75}FO}%
\colorbox{green}{\color[gray]{0.75}FO}%
\colorbox{green}{\color[gray]{0.75}FO}%
\colorbox{green}{\color[gray]{0.75}FO}%
\colorbox{green}{\color[gray]{0.75}FO}%
\colorbox{green}{\color[gray]{0.75}FO}%
\colorbox{green}{\color[gray]{0.75}FO}%
\colorbox{green}{\color[gray]{0.75}FO}%
\colorbox{green}{\color[gray]{0.75}FO}%
\colorbox{green}{\color[gray]{0.75}FO}%
\colorbox{green}{\color[gray]{0.75}FO}%
\colorbox{green}{\color[gray]{0.75}FO}%
\colorbox{green}{\color[gray]{0.75}FO}%
\colorbox{green}{\color[gray]{0.75}FO}%
\colorbox{green}{\color[gray]{0.75}FO}%
\colorbox{green}{\color[gray]{0.75}FO}%
\colorbox{green}{\color[gray]{0.75}FO}%
\colorbox{green}{\color[gray]{0.75}FO}%
\colorbox{green}{\color[gray]{0.75}FO}%
\colorbox{green}{\color[gray]{0.75}FO}%
\colorbox{green}{\color[gray]{0.75}FO}%
\colorbox{green}{\color[gray]{0.75}FO}%
\colorbox{green}{\color[gray]{0.75}FO}%
\colorbox{green}{\color[gray]{0.75}FO}%
\colorbox{green}{\color[gray]{0.75}FO}%
\colorbox{green}{\color[gray]{0.75}FO}%
\colorbox{green}{\color[gray]{0.75}FO}%
\colorbox{green}{\color[gray]{0.75}FO}%
\colorbox{green}{\color[gray]{0.75}FO}%
\colorbox{green}{\color[gray]{0.75}FO}%
\colorbox{green}{\color[gray]{0.75}FO}%
\colorbox{green}{\color[gray]{0.75}FO}%
\colorbox{green}{\color[gray]{0.75}FO}%
\colorbox{green}{\color[gray]{0.75}FO}%
\colorbox{green}{\color[gray]{0.75}FO}%
\colorbox{green}{\color[gray]{0.75}FO}%
\colorbox{green}{\color[gray]{0.75}FO}%
\colorbox{green}{\color[gray]{0.75}FO}%
\colorbox{green}{\color[gray]{0.75}FO}%
\colorbox{green}{\color[gray]{0.75}FO}%
\colorbox{green}{\color[gray]{0.75}FO}%
\colorbox{green}{\color[gray]{0.75}FO}%
\colorbox{green}{\color[gray]{0.75}FO}%
\colorbox{green}{\color[gray]{0.75}FO}%
\colorbox{green}{\color[gray]{0.75}FO}%
\colorbox{green}{\color[gray]{0.75}FO}%
\colorbox{green}{\color[gray]{0.75}FO}%
\colorbox{green}{\color[gray]{0.75}FO}%
\colorbox{green}{\color[gray]{0.75}FO}%
\colorbox{green}{\color[gray]{0.75}FO}%
\colorbox{green}{\color[gray]{0.75}FO}%
\colorbox{green}{\color[gray]{0.75}FO}%
\colorbox{green}{\color[gray]{0.75}FO}%
\colorbox{green}{\color[gray]{0.75}FO}%
\colorbox{green}{\color[gray]{0.75}FO}%
\colorbox{green}{\color[gray]{0.75}FO}%
\colorbox{green}{\color[gray]{0.75}FO}%
\colorbox{green}{\color[gray]{0.75}FO}%
\colorbox{green}{\color[gray]{0.75}FO}%
\colorbox{green}{\color[gray]{0.75}FO}%
\colorbox{green}{\color[gray]{0.75}FO}%
\colorbox{green}{\color[gray]{0.75}FO}%
\colorbox{green}{\color[gray]{0.75}FO}%
\colorbox{green}{\color[gray]{0.75}FO}%
\colorbox{green}{\color[gray]{0.75}FO}%
\colorbox{green}{\color[gray]{0.75}FO}%
\colorbox{green}{\color[gray]{0.75}FO}%
\colorbox{green}{\color[gray]{0.75}FO}%
\colorbox{green}{\color[gray]{0.75}FO}%
\colorbox{green}{\color[gray]{0.75}FO}%
\colorbox{green}{\color[gray]{0.75}FO}%
\colorbox{green}{\color[gray]{0.75}FO}%
\colorbox{green}{\color[gray]{0.75}FO}%
\colorbox{green}{\color[gray]{0.75}FO}%
\colorbox{green}{\color[gray]{0.75}FO}%
\colorbox{green}{\color[gray]{0.75}FO}%
\colorbox{green}{\color[gray]{0.75}FO}%
\colorbox{green}{\color[gray]{0.75}FO}%
\colorbox{green}{\color[gray]{0.75}FO}%
\\
\colorbox{green}{\color[gray]{0.75}FO}%
\colorbox{green}{\color[gray]{0.75}FO}%
\colorbox{green}{\color[gray]{0.75}FO}%
\colorbox{green}{\color[gray]{0.75}FO}%
\colorbox{green}{\color[gray]{0.75}FO}%
\colorbox{green}{\color[gray]{0.75}FO}%
\colorbox{green}{\color[gray]{0.75}FO}%
\colorbox{green}{\color[gray]{0.75}FO}%
\colorbox{green}{\color[gray]{0.75}FO}%
\colorbox{green}{\color[gray]{0.75}FO}%
\colorbox{green}{\color[gray]{0.75}FO}%
\colorbox{green}{\color[gray]{0.75}FO}%
\colorbox{green}{\color[gray]{0.75}FO}%
\colorbox{green}{\color[gray]{0.75}FO}%
\colorbox{green}{\color[gray]{0.75}FO}%
\colorbox{green}{\color[gray]{0.75}FO}%
\colorbox{green}{\color[gray]{0.75}FO}%
\colorbox{green}{\color[gray]{0.75}FO}%
\colorbox{green}{\color[gray]{0.75}FO}%
\colorbox{green}{\color[gray]{0.75}FO}%
\colorbox{green}{\color[gray]{0.75}FO}%
\colorbox{green}{\color[gray]{0.75}FO}%
\colorbox{green}{\color[gray]{0.75}FO}%
\colorbox{green}{\color[gray]{0.75}FO}%
\colorbox{green}{\color[gray]{0.75}FO}%
\colorbox{green}{\color[gray]{0.75}FO}%
\colorbox{green}{\color[gray]{0.75}FO}%
\colorbox{green}{\color[gray]{0.75}FO}%
\colorbox{green}{\color[gray]{0.75}FO}%
\colorbox{green}{\color[gray]{0.75}FO}%
\colorbox{green}{\color[gray]{0.75}FO}%
\colorbox{green}{\color[gray]{0.75}FO}%
\colorbox{green}{\color[gray]{0.75}FO}%
\colorbox{green}{\color[gray]{0.75}FO}%
\colorbox{green}{\color[gray]{0.75}FO}%
\colorbox{green}{\color[gray]{0.75}FO}%
\colorbox{green}{\color[gray]{0.75}FO}%
\colorbox{green}{\color[gray]{0.75}FO}%
\colorbox{green}{\color[gray]{0.75}FO}%
\colorbox{green}{\color[gray]{0.75}FO}%
\colorbox{green}{\color[gray]{0.75}FO}%
\colorbox{green}{\color[gray]{0.75}FO}%
\colorbox{green}{\color[gray]{0.75}FO}%
\colorbox{green}{\color[gray]{0.75}FO}%
\colorbox{green}{\color[gray]{0.75}FO}%
\colorbox{green}{\color[gray]{0.75}FO}%
\colorbox{green}{\color[gray]{0.75}FO}%
\colorbox{green}{\color[gray]{0.75}FO}%
\colorbox{green}{\color[gray]{0.75}FO}%
\colorbox{green}{\color[gray]{0.75}FO}%
\colorbox{green}{\color[gray]{0.75}FO}%
\colorbox{green}{\color[gray]{0.75}FO}%
\colorbox{green}{\color[gray]{0.75}FO}%
\colorbox{green}{\color[gray]{0.75}FO}%
\colorbox{green}{\color[gray]{0.75}FO}%
\colorbox{green}{\color[gray]{0.75}FO}%
\colorbox{green}{\color[gray]{0.75}FO}%
\colorbox{green}{\color[gray]{0.75}FO}%
\colorbox{green}{\color[gray]{0.75}FO}%
\colorbox{green}{\color[gray]{0.75}FO}%
\colorbox{green}{\color[gray]{0.75}FO}%
\colorbox{green}{\color[gray]{0.75}FO}%
\colorbox{green}{\color[gray]{0.75}FO}%
\colorbox{green}{\color[gray]{0.75}FO}%
\colorbox{green}{\color[gray]{0.75}FO}%
\colorbox{green}{\color[gray]{0.75}FO}%
\colorbox{green}{\color[gray]{0.75}FO}%
\colorbox{green}{\color[gray]{0.75}FO}%
\colorbox{green}{\color[gray]{0.75}FO}%
\colorbox{green}{\color[gray]{0.75}FO}%
\colorbox{green}{\color[gray]{0.75}FO}%
\colorbox{green}{\color[gray]{0.75}FO}%
\colorbox{green}{\color[gray]{0.75}FO}%
\colorbox{green}{\color[gray]{0.75}FO}%
\colorbox{green}{\color[gray]{0.75}FO}%
\colorbox{green}{\color[gray]{0.75}FO}%
\colorbox{green}{\color[gray]{0.75}FO}%
\colorbox{green}{\color[gray]{0.75}FO}%
\colorbox{green}{\color[gray]{0.75}FO}%
\colorbox{green}{\color[gray]{0.75}FO}%
\colorbox{green}{\color[gray]{0.75}FO}%
\colorbox{green}{\color[gray]{0.75}FO}%
\colorbox{green}{\color[gray]{0.75}FO}%
\colorbox{green}{\color[gray]{0.75}FO}%
\colorbox{green}{\color[gray]{0.75}FO}%
\colorbox{green}{\color[gray]{0.75}FO}%
\colorbox{green}{\color[gray]{0.75}FO}%
\colorbox{green}{\color[gray]{0.75}FO}%
\colorbox{green}{\color[gray]{0.75}FO}%
\colorbox{green}{\color[gray]{0.75}FO}%
\colorbox{green}{\color[gray]{0.75}FO}%
\colorbox{green}{\color[gray]{0.75}FO}%
\colorbox{green}{\color[gray]{0.75}FO}%
\colorbox{green}{\color[gray]{0.75}FO}%
\colorbox{green}{\color[gray]{0.75}FO}%
\colorbox{green}{\color[gray]{0.75}FO}%
\colorbox{green}{\color[gray]{0.75}FO}%
\colorbox{green}{\color[gray]{0.75}FO}%
\colorbox{green}{\color[gray]{0.75}FO}%
\colorbox{green}{\color[gray]{0.75}FO}%
\\
\colorbox{green}{\color[gray]{0.75}FO}%
\colorbox{green}{\color[gray]{0.75}FO}%
\colorbox{green}{\color[gray]{0.75}FO}%
\colorbox{green}{\color[gray]{0.75}FO}%
\colorbox{green}{\color[gray]{0.75}FO}%
\colorbox{green}{\color[gray]{0.75}FO}%
\colorbox{green}{\color[gray]{0.75}FO}%
\colorbox{green}{\color[gray]{0.75}FO}%
\colorbox{green}{\color[gray]{0.75}FO}%
\colorbox{green}{\color[gray]{0.75}FO}%
\colorbox{green}{\color[gray]{0.75}FO}%
\colorbox{green}{\color[gray]{0.75}FO}%
\colorbox{green}{\color[gray]{0.75}FO}%
\colorbox{green}{\color[gray]{0.75}FO}%
\colorbox{green}{\color[gray]{0.75}FO}%
\colorbox{green}{\color[gray]{0.75}FO}%
\colorbox{green}{\color[gray]{0.75}FO}%
\colorbox{green}{\color[gray]{0.75}FO}%
\colorbox{green}{\color[gray]{0.75}FO}%
\colorbox{green}{\color[gray]{0.75}FO}%
\colorbox{green}{\color[gray]{0.75}FO}%
\colorbox{green}{\color[gray]{0.75}FO}%
\colorbox{green}{\color[gray]{0.75}FO}%
\colorbox{green}{\color[gray]{0.75}FO}%
\colorbox{green}{\color[gray]{0.75}FO}%
\colorbox{green}{\color[gray]{0.75}FO}%
\colorbox{green}{\color[gray]{0.75}FO}%
\colorbox{green}{\color[gray]{0.75}FO}%
\colorbox{green}{\color[gray]{0.75}FO}%
\colorbox{green}{\color[gray]{0.75}FO}%
\colorbox{green}{\color[gray]{0.75}FO}%
\colorbox{green}{\color[gray]{0.75}FO}%
\colorbox{green}{\color[gray]{0.75}FO}%
\colorbox{green}{\color[gray]{0.75}FO}%
\colorbox{green}{\color[gray]{0.75}FO}%
\colorbox{green}{\color[gray]{0.75}FO}%
\colorbox{green}{\color[gray]{0.75}FO}%
\colorbox{green}{\color[gray]{0.75}FO}%
\colorbox{green}{\color[gray]{0.75}FO}%
\colorbox{green}{\color[gray]{0.75}FO}%
\colorbox{green}{\color[gray]{0.75}FO}%
\colorbox{green}{\color[gray]{0.75}FO}%
\colorbox{green}{\color[gray]{0.75}FO}%
\colorbox{green}{\color[gray]{0.75}FO}%
\colorbox{green}{\color[gray]{0.75}FO}%
\colorbox{green}{\color[gray]{0.75}FO}%
\colorbox{green}{\color[gray]{0.75}FO}%
\colorbox{green}{\color[gray]{0.75}FO}%
\colorbox{green}{\color[gray]{0.75}FO}%
\colorbox{green}{\color[gray]{0.75}FO}%
\colorbox{green}{\color[gray]{0.75}FO}%
\colorbox{green}{\color[gray]{0.75}FO}%
\colorbox{green}{\color[gray]{0.75}FO}%
\colorbox{green}{\color[gray]{0.75}FO}%
\colorbox{green}{\color[gray]{0.75}FO}%
\colorbox{green}{\color[gray]{0.75}FO}%
\colorbox{green}{\color[gray]{0.75}FO}%
\colorbox{green}{\color[gray]{0.75}FO}%
\colorbox{green}{\color[gray]{0.75}FO}%
\colorbox{green}{\color[gray]{0.75}FO}%
\colorbox{green}{\color[gray]{0.75}FO}%
\colorbox{green}{\color[gray]{0.75}FO}%
\colorbox{green}{\color[gray]{0.75}FO}%
\colorbox{green}{\color[gray]{0.75}FO}%
\colorbox{green}{\color[gray]{0.75}FO}%
\colorbox{green}{\color[gray]{0.75}FO}%
\colorbox{green}{\color[gray]{0.75}FO}%
\colorbox{green}{\color[gray]{0.75}FO}%
\colorbox{green}{\color[gray]{0.75}FO}%
\colorbox{green}{\color[gray]{0.75}FO}%
\colorbox{green}{\color[gray]{0.75}FO}%
\colorbox{green}{\color[gray]{0.75}FO}%
\colorbox{green}{\color[gray]{0.75}FO}%
\colorbox{green}{\color[gray]{0.75}FO}%
\colorbox{green}{\color[gray]{0.75}FO}%
\colorbox{green}{\color[gray]{0.75}FO}%
\colorbox{green}{\color[gray]{0.75}FO}%
\colorbox{green}{\color[gray]{0.75}FO}%
\colorbox{green}{\color[gray]{0.75}FO}%
\colorbox{green}{\color[gray]{0.75}FO}%
\colorbox{green}{\color[gray]{0.75}FO}%
\colorbox{green}{\color[gray]{0.75}FO}%
\colorbox{green}{\color[gray]{0.75}FO}%
\colorbox{green}{\color[gray]{0.75}FO}%
\colorbox{green}{\color[gray]{0.75}FO}%
\colorbox{green}{\color[gray]{0.75}FO}%
\colorbox{green}{\color[gray]{0.75}FO}%
\colorbox{green}{\color[gray]{0.75}FO}%
\colorbox{green}{\color[gray]{0.75}FO}%
\colorbox{green}{\color[gray]{0.75}FO}%
\colorbox{green}{\color[gray]{0.75}FO}%
\colorbox{green}{\color[gray]{0.75}FO}%
\colorbox{green}{\color[gray]{0.75}FO}%
\colorbox{green}{\color[gray]{0.75}FO}%
\colorbox{green}{\color[gray]{0.75}FO}%
\colorbox{green}{\color[gray]{0.75}FO}%
\colorbox{green}{\color[gray]{0.75}FO}%
\colorbox{green}{\color[gray]{0.75}FO}%
\colorbox{green}{\color[gray]{0.75}FO}%
\colorbox{green}{\color[gray]{0.75}FO}%
\\
\colorbox{green}{\color[gray]{0.75}FO}%
\colorbox{green}{\color[gray]{0.75}FO}%
\colorbox{green}{\color[gray]{0.75}FO}%
\colorbox{green}{\color[gray]{0.75}FO}%
\colorbox{green}{\color[gray]{0.75}FO}%
\colorbox{green}{\color[gray]{0.75}FO}%
\colorbox{green}{\color[gray]{0.75}FO}%
\colorbox{green}{\color[gray]{0.75}FO}%
\colorbox{green}{\color[gray]{0.75}FO}%
\colorbox{green}{\color[gray]{0.75}FO}%
\colorbox{green}{\color[gray]{0.75}FO}%
\colorbox{green}{\color[gray]{0.75}FO}%
\colorbox{green}{\color[gray]{0.75}FO}%
\colorbox{green}{\color[gray]{0.75}FO}%
\colorbox{green}{\color[gray]{0.75}FO}%
\colorbox{green}{\color[gray]{0.75}FO}%
\colorbox{green}{\color[gray]{0.75}FO}%
\colorbox{green}{\color[gray]{0.75}FO}%
\colorbox{green}{\color[gray]{0.75}FO}%
\colorbox{green}{\color[gray]{0.75}FO}%
\colorbox{green}{\color[gray]{0.75}FO}%
\colorbox{green}{\color[gray]{0.75}FO}%
\colorbox{green}{\color[gray]{0.75}FO}%
\colorbox{green}{\color[gray]{0.75}FO}%
\colorbox{green}{\color[gray]{0.75}FO}%
\colorbox{green}{\color[gray]{0.75}FO}%
\colorbox{green}{\color[gray]{0.75}FO}%
\colorbox{green}{\color[gray]{0.75}FO}%
\colorbox{green}{\color[gray]{0.75}FO}%
\colorbox{green}{\color[gray]{0.75}FO}%
\colorbox{green}{\color[gray]{0.75}FO}%
\colorbox{green}{\color[gray]{0.75}FO}%
\colorbox{green}{\color[gray]{0.75}FO}%
\colorbox{green}{\color[gray]{0.75}FO}%
\colorbox{green}{\color[gray]{0.75}FO}%
\colorbox{green}{\color[gray]{0.75}FO}%
\colorbox{green}{\color[gray]{0.75}FO}%
\colorbox{green}{\color[gray]{0.75}FO}%
\colorbox{green}{\color[gray]{0.75}FO}%
\colorbox{green}{\color[gray]{0.75}FO}%
\colorbox{green}{\color[gray]{0.75}FO}%
\colorbox{green}{\color[gray]{0.75}FO}%
\colorbox{green}{\color[gray]{0.75}FO}%
\colorbox{green}{\color[gray]{0.75}FO}%
\colorbox{green}{\color[gray]{0.75}FO}%
\colorbox{green}{\color[gray]{0.75}FO}%
\colorbox{green}{\color[gray]{0.75}FO}%
\colorbox{green}{\color[gray]{0.75}FO}%
\colorbox{green}{\color[gray]{0.75}FO}%
\colorbox{green}{\color[gray]{0.75}FO}%
\colorbox{green}{\color[gray]{0.75}FO}%
\colorbox{green}{\color[gray]{0.75}FO}%
\colorbox{green}{\color[gray]{0.75}FO}%
\colorbox{green}{\color[gray]{0.75}FO}%
\colorbox{green}{\color[gray]{0.75}FO}%
\colorbox{green}{\color[gray]{0.75}FO}%
\colorbox{green}{\color[gray]{0.75}FO}%
\colorbox{green}{\color[gray]{0.75}FO}%
\colorbox{green}{\color[gray]{0.75}FO}%
\colorbox{green}{\color[gray]{0.75}FO}%
\colorbox{green}{\color[gray]{0.75}FO}%
\colorbox{green}{\color[gray]{0.75}FO}%
\colorbox{green}{\color[gray]{0.75}FO}%
\colorbox{green}{\color[gray]{0.75}FO}%
\colorbox{green}{\color[gray]{0.75}FO}%
\colorbox{green}{\color[gray]{0.75}FO}%
\colorbox{green}{\color[gray]{0.75}FO}%
\colorbox{green}{\color[gray]{0.75}FO}%
\colorbox{green}{\color[gray]{0.75}FO}%
\colorbox{green}{\color[gray]{0.75}FO}%
\colorbox{green}{\color[gray]{0.75}FO}%
\colorbox{green}{\color[gray]{0.75}FO}%
\colorbox{green}{\color[gray]{0.75}FO}%
\colorbox{green}{\color[gray]{0.75}FO}%
\colorbox{green}{\color[gray]{0.75}FO}%
\colorbox{green}{\color[gray]{0.75}FO}%
\colorbox{green}{\color[gray]{0.75}FO}%
\colorbox{green}{\color[gray]{0.75}FO}%
\colorbox{green}{\color[gray]{0.75}FO}%
\colorbox{green}{\color[gray]{0.75}FO}%
\colorbox{green}{\color[gray]{0.75}FO}%
\colorbox{green}{\color[gray]{0.75}FO}%
\colorbox{green}{\color[gray]{0.75}FO}%
\colorbox{green}{\color[gray]{0.75}FO}%
\colorbox{green}{\color[gray]{0.75}FO}%
\colorbox{green}{\color[gray]{0.75}FO}%
\colorbox{green}{\color[gray]{0.75}FO}%
\colorbox{green}{\color[gray]{0.75}FO}%
\colorbox{green}{\color[gray]{0.75}FO}%
\colorbox{green}{\color[gray]{0.75}FO}%
\colorbox{green}{\color[gray]{0.75}FO}%
\colorbox{green}{\color[gray]{0.75}FO}%
\colorbox{green}{\color[gray]{0.75}FO}%
\colorbox{green}{\color[gray]{0.75}FO}%
\colorbox{green}{\color[gray]{0.75}FO}%
\colorbox{green}{\color[gray]{0.75}FO}%
\colorbox{green}{\color[gray]{0.75}FO}%
\colorbox{green}{\color[gray]{0.75}FO}%
\colorbox{green}{\color[gray]{0.75}FO}%
\colorbox{green}{\color[gray]{0.75}FO}%
\\
\colorbox{green}{\color[gray]{0.75}FO}%
\colorbox{green}{\color[gray]{0.75}FO}%
\colorbox{green}{\color[gray]{0.75}FO}%
\colorbox{green}{\color[gray]{0.75}FO}%
\colorbox{green}{\color[gray]{0.75}FO}%
\colorbox{green}{\color[gray]{0.75}FO}%
\colorbox{green}{\color[gray]{0.75}FO}%
\colorbox{green}{\color[gray]{0.75}FO}%
\colorbox{green}{\color[gray]{0.75}FO}%
\colorbox{green}{\color[gray]{0.75}FO}%
\colorbox{green}{\color[gray]{0.75}FO}%
\colorbox{green}{\color[gray]{0.75}FO}%
\colorbox{green}{\color[gray]{0.75}FO}%
\colorbox{green}{\color[gray]{0.75}FO}%
\colorbox{green}{\color[gray]{0.75}FO}%
\colorbox{green}{\color[gray]{0.75}FO}%
\colorbox{green}{\color[gray]{0.75}FO}%
\colorbox{green}{\color[gray]{0.75}FO}%
\colorbox{green}{\color[gray]{0.75}FO}%
\colorbox{green}{\color[gray]{0.75}FO}%
\colorbox{green}{\color[gray]{0.75}FO}%
\colorbox{green}{\color[gray]{0.75}FO}%
\colorbox{green}{\color[gray]{0.75}FO}%
\colorbox{green}{\color[gray]{0.75}FO}%
\colorbox{green}{\color[gray]{0.75}FO}%
\colorbox{green}{\color[gray]{0.75}FO}%
\colorbox{green}{\color[gray]{0.75}FO}%
\colorbox{green}{\color[gray]{0.75}FO}%
\colorbox{green}{\color[gray]{0.75}FO}%
\colorbox{green}{\color[gray]{0.75}FO}%
\colorbox{green}{\color[gray]{0.75}FO}%
\colorbox{green}{\color[gray]{0.75}FO}%
\colorbox{green}{\color[gray]{0.75}FO}%
\colorbox{green}{\color[gray]{0.75}FO}%
\colorbox{green}{\color[gray]{0.75}FO}%
\colorbox{green}{\color[gray]{0.75}FO}%
\colorbox{green}{\color[gray]{0.75}FO}%
\colorbox{green}{\color[gray]{0.75}FO}%
\colorbox{green}{\color[gray]{0.75}FO}%
\colorbox{green}{\color[gray]{0.75}FO}%
\colorbox{green}{\color[gray]{0.75}FO}%
\colorbox{green}{\color[gray]{0.75}FO}%
\colorbox{green}{\color[gray]{0.75}FO}%
\colorbox{green}{\color[gray]{0.75}FO}%
\colorbox{green}{\color[gray]{0.75}FO}%
\colorbox{green}{\color[gray]{0.75}FO}%
\colorbox{green}{\color[gray]{0.75}FO}%
\colorbox{green}{\color[gray]{0.75}FO}%
\colorbox{green}{\color[gray]{0.75}FO}%
\colorbox{green}{\color[gray]{0.75}FO}%
\colorbox{green}{\color[gray]{0.75}FO}%
\colorbox{green}{\color[gray]{0.75}FO}%
\colorbox{green}{\color[gray]{0.75}FO}%
\colorbox{green}{\color[gray]{0.75}FO}%
\colorbox{green}{\color[gray]{0.75}FO}%
\colorbox{green}{\color[gray]{0.75}FO}%
\colorbox{green}{\color[gray]{0.75}FO}%
\colorbox{green}{\color[gray]{0.75}FO}%
\colorbox{green}{\color[gray]{0.75}FO}%
\colorbox{green}{\color[gray]{0.75}FO}%
\colorbox{green}{\color[gray]{0.75}FO}%
\colorbox{green}{\color[gray]{0.75}FO}%
\colorbox{green}{\color[gray]{0.75}FO}%
\colorbox{green}{\color[gray]{0.75}FO}%
\colorbox{green}{\color[gray]{0.75}FO}%
\colorbox{green}{\color[gray]{0.75}FO}%
\colorbox{green}{\color[gray]{0.75}FO}%
\colorbox{green}{\color[gray]{0.75}FO}%
\colorbox{green}{\color[gray]{0.75}FO}%
\colorbox{green}{\color[gray]{0.75}FO}%
\colorbox{green}{\color[gray]{0.75}FO}%
\colorbox{green}{\color[gray]{0.75}FO}%
\colorbox{green}{\color[gray]{0.75}FO}%
\colorbox{green}{\color[gray]{0.75}FO}%
\colorbox{green}{\color[gray]{0.75}FO}%
\colorbox{green}{\color[gray]{0.75}FO}%
\colorbox{green}{\color[gray]{0.75}FO}%
\colorbox{green}{\color[gray]{0.75}FO}%
\colorbox{green}{\color[gray]{0.75}FO}%
\colorbox{green}{\color[gray]{0.75}FO}%
\colorbox{green}{\color[gray]{0.75}FO}%
\colorbox{green}{\color[gray]{0.75}FO}%
\colorbox{green}{\color[gray]{0.75}FO}%
\colorbox{green}{\color[gray]{0.75}FO}%
\colorbox{green}{\color[gray]{0.75}FO}%
\colorbox{green}{\color[gray]{0.75}FO}%
\colorbox{green}{\color[gray]{0.75}FO}%
\colorbox{green}{\color[gray]{0.75}FO}%
\colorbox{green}{\color[gray]{0.75}FO}%
\colorbox{green}{\color[gray]{0.75}FO}%
\colorbox{green}{\color[gray]{0.75}FO}%
\colorbox{green}{\color[gray]{0.75}FO}%
\colorbox{green}{\color[gray]{0.75}FO}%
\colorbox{green}{\color[gray]{0.75}FO}%
\colorbox{green}{\color[gray]{0.75}FO}%
\colorbox{green}{\color[gray]{0.75}FO}%
\colorbox{green}{\color[gray]{0.75}FO}%
\colorbox{green}{\color[gray]{0.75}FO}%
\colorbox{green}{\color[gray]{0.75}FO}%
\colorbox{green}{\color[gray]{0.75}FO}%
\\
\colorbox{green}{\color[gray]{0.75}FO}%
\colorbox{green}{\color[gray]{0.75}FO}%
\colorbox{green}{\color[gray]{0.75}FO}%
\colorbox{green}{\color[gray]{0.75}FO}%
\colorbox{green}{\color[gray]{0.75}FO}%
\colorbox{green}{\color[gray]{0.75}FO}%
\colorbox{green}{\color[gray]{0.75}FO}%
\colorbox{green}{\color[gray]{0.75}FO}%
\colorbox{green}{\color[gray]{0.75}FO}%
\colorbox{green}{\color[gray]{0.75}FO}%
\colorbox{green}{\color[gray]{0.75}FO}%
\colorbox{green}{\color[gray]{0.75}FO}%
\colorbox{green}{\color[gray]{0.75}FO}%
\colorbox{green}{\color[gray]{0.75}FO}%
\colorbox{green}{\color[gray]{0.75}FO}%
\colorbox{green}{\color[gray]{0.75}FO}%
\colorbox{green}{\color[gray]{0.75}FO}%
\colorbox{green}{\color[gray]{0.75}FO}%
\colorbox{green}{\color[gray]{0.75}FO}%
\colorbox{green}{\color[gray]{0.75}FO}%
\colorbox{green}{\color[gray]{0.75}FO}%
\colorbox{green}{\color[gray]{0.75}FO}%
\colorbox{green}{\color[gray]{0.75}FO}%
\colorbox{green}{\color[gray]{0.75}FO}%
\colorbox{green}{\color[gray]{0.75}FO}%
\colorbox{green}{\color[gray]{0.75}FO}%
\colorbox{green}{\color[gray]{0.75}FO}%
\colorbox{green}{\color[gray]{0.75}FO}%
\colorbox{green}{\color[gray]{0.75}FO}%
\colorbox{green}{\color[gray]{0.75}FO}%
\colorbox{green}{\color[gray]{0.75}FO}%
\colorbox{green}{\color[gray]{0.75}FO}%
\colorbox{green}{\color[gray]{0.75}FO}%
\colorbox{green}{\color[gray]{0.75}FO}%
\colorbox{green}{\color[gray]{0.75}FO}%
\colorbox{green}{\color[gray]{0.75}FO}%
\colorbox{green}{\color[gray]{0.75}FO}%
\colorbox{green}{\color[gray]{0.75}FO}%
\colorbox{green}{\color[gray]{0.75}FO}%
\colorbox{green}{\color[gray]{0.75}FO}%
\colorbox{green}{\color[gray]{0.75}FO}%
\colorbox{green}{\color[gray]{0.75}FO}%
\colorbox{green}{\color[gray]{0.75}FO}%
\colorbox{green}{\color[gray]{0.75}FO}%
\colorbox{green}{\color[gray]{0.75}FO}%
\colorbox{green}{\color[gray]{0.75}FO}%
\colorbox{green}{\color[gray]{0.75}FO}%
\colorbox{green}{\color[gray]{0.75}FO}%
\colorbox{green}{\color[gray]{0.75}FO}%
\colorbox{green}{\color[gray]{0.75}FO}%
\colorbox{green}{\color[gray]{0.75}FO}%
\colorbox{green}{\color[gray]{0.75}FO}%
\colorbox{green}{\color[gray]{0.75}FO}%
\colorbox{green}{\color[gray]{0.75}FO}%
\colorbox{green}{\color[gray]{0.75}FO}%
\colorbox{green}{\color[gray]{0.75}FO}%
\colorbox{green}{\color[gray]{0.75}FO}%
\colorbox{green}{\color[gray]{0.75}FO}%
\colorbox{green}{\color[gray]{0.75}FO}%
\colorbox{green}{\color[gray]{0.75}FO}%
\colorbox{green}{\color[gray]{0.75}FO}%
\colorbox{green}{\color[gray]{0.75}FO}%
\colorbox{green}{\color[gray]{0.75}FO}%
\colorbox{green}{\color[gray]{0.75}FO}%
\colorbox{green}{\color[gray]{0.75}FO}%
\colorbox{green}{\color[gray]{0.75}FO}%
\colorbox{green}{\color[gray]{0.75}FO}%
\colorbox{green}{\color[gray]{0.75}FO}%
\colorbox{green}{\color[gray]{0.75}FO}%
\colorbox{green}{\color[gray]{0.75}FO}%
\colorbox{green}{\color[gray]{0.75}FO}%
\colorbox{green}{\color[gray]{0.75}FO}%
\colorbox{green}{\color[gray]{0.75}FO}%
\colorbox{green}{\color[gray]{0.75}FO}%
\colorbox{green}{\color[gray]{0.75}FO}%
\colorbox{green}{\color[gray]{0.75}FO}%
\colorbox{green}{\color[gray]{0.75}FO}%
\colorbox{green}{\color[gray]{0.75}FO}%
\colorbox{green}{\color[gray]{0.75}FO}%
\colorbox{green}{\color[gray]{0.75}FO}%
\colorbox{green}{\color[gray]{0.75}FO}%
\colorbox{green}{\color[gray]{0.75}FO}%
\colorbox{green}{\color[gray]{0.75}FO}%
\colorbox{green}{\color[gray]{0.75}FO}%
\colorbox{green}{\color[gray]{0.75}FO}%
\colorbox{green}{\color[gray]{0.75}FO}%
\colorbox{green}{\color[gray]{0.75}FO}%
\colorbox{green}{\color[gray]{0.75}FO}%
\colorbox{green}{\color[gray]{0.75}FO}%
\colorbox{green}{\color[gray]{0.75}FO}%
\colorbox{green}{\color[gray]{0.75}FO}%
\colorbox{green}{\color[gray]{0.75}FO}%
\colorbox{green}{\color[gray]{0.75}FO}%
\colorbox{green}{\color[gray]{0.75}FO}%
\colorbox{green}{\color[gray]{0.75}FO}%
\colorbox{green}{\color[gray]{0.75}FO}%
\colorbox{green}{\color[gray]{0.75}FO}%
\colorbox{green}{\color[gray]{0.75}FO}%
\colorbox{green}{\color[gray]{0.75}FO}%
\colorbox{green}{\color[gray]{0.75}FO}%
\\
\colorbox{green}{\color[gray]{0.75}FO}%
\colorbox{green}{\color[gray]{0.75}FO}%
\colorbox{green}{\color[gray]{0.75}FO}%
\colorbox{green}{\color[gray]{0.75}FO}%
\colorbox{green}{\color[gray]{0.75}FO}%
\colorbox{green}{\color[gray]{0.75}FO}%
\colorbox{green}{\color[gray]{0.75}FO}%
\colorbox{green}{\color[gray]{0.75}FO}%
\colorbox{green}{\color[gray]{0.75}FO}%
\colorbox{green}{\color[gray]{0.75}FO}%
\colorbox{green}{\color[gray]{0.75}FO}%
\colorbox{green}{\color[gray]{0.75}FO}%
\colorbox{green}{\color[gray]{0.75}FO}%
\colorbox{green}{\color[gray]{0.75}FO}%
\colorbox{green}{\color[gray]{0.75}FO}%
\colorbox{green}{\color[gray]{0.75}FO}%
\colorbox{green}{\color[gray]{0.75}FO}%
\colorbox{green}{\color[gray]{0.75}FO}%
\colorbox{green}{\color[gray]{0.75}FO}%
\colorbox{green}{\color[gray]{0.75}FO}%
\colorbox{green}{\color[gray]{0.75}FO}%
\colorbox{green}{\color[gray]{0.75}FO}%
\colorbox{green}{\color[gray]{0.75}FO}%
\colorbox{green}{\color[gray]{0.75}FO}%
\colorbox{green}{\color[gray]{0.75}FO}%
\colorbox{green}{\color[gray]{0.75}FO}%
\colorbox{green}{\color[gray]{0.75}FO}%
\colorbox{green}{\color[gray]{0.75}FO}%
\colorbox{green}{\color[gray]{0.75}FO}%
\colorbox{green}{\color[gray]{0.75}FO}%
\colorbox{green}{\color[gray]{0.75}FO}%
\colorbox{green}{\color[gray]{0.75}FO}%
\colorbox{green}{\color[gray]{0.75}FO}%
\colorbox{green}{\color[gray]{0.75}FO}%
\colorbox{green}{\color[gray]{0.75}FO}%
\colorbox{green}{\color[gray]{0.75}FO}%
\colorbox{green}{\color[gray]{0.75}FO}%
\colorbox{green}{\color[gray]{0.75}FO}%
\colorbox{green}{\color[gray]{0.75}FO}%
\colorbox{green}{\color[gray]{0.75}FO}%
\colorbox{green}{\color[gray]{0.75}FO}%
\colorbox{green}{\color[gray]{0.75}FO}%
\colorbox{green}{\color[gray]{0.75}FO}%
\colorbox{green}{\color[gray]{0.75}FO}%
\colorbox{green}{\color[gray]{0.75}FO}%
\colorbox{green}{\color[gray]{0.75}FO}%
\colorbox{green}{\color[gray]{0.75}FO}%
\colorbox{green}{\color[gray]{0.75}FO}%
\colorbox{green}{\color[gray]{0.75}FO}%
\colorbox{green}{\color[gray]{0.75}FO}%
\colorbox{green}{\color[gray]{0.75}FO}%
\colorbox{green}{\color[gray]{0.75}FO}%
\colorbox{green}{\color[gray]{0.75}FO}%
\colorbox{green}{\color[gray]{0.75}FO}%
\colorbox{green}{\color[gray]{0.75}FO}%
\colorbox{green}{\color[gray]{0.75}FO}%
\colorbox{green}{\color[gray]{0.75}FO}%
\colorbox{green}{\color[gray]{0.75}FO}%
\colorbox{green}{\color[gray]{0.75}FO}%
\colorbox{green}{\color[gray]{0.75}FO}%
\colorbox{green}{\color[gray]{0.75}FO}%
\colorbox{green}{\color[gray]{0.75}FO}%
\colorbox{green}{\color[gray]{0.75}FO}%
\colorbox{green}{\color[gray]{0.75}FO}%
\colorbox{green}{\color[gray]{0.75}FO}%
\colorbox{green}{\color[gray]{0.75}FO}%
\colorbox{green}{\color[gray]{0.75}FO}%
\colorbox{green}{\color[gray]{0.75}FO}%
\colorbox{green}{\color[gray]{0.75}FO}%
\colorbox{green}{\color[gray]{0.75}FO}%
\colorbox{green}{\color[gray]{0.75}FO}%
\colorbox{green}{\color[gray]{0.75}FO}%
\colorbox{green}{\color[gray]{0.75}FO}%
\colorbox{green}{\color[gray]{0.75}FO}%
\colorbox{green}{\color[gray]{0.75}FO}%
\colorbox{green}{\color[gray]{0.75}FO}%
\colorbox{green}{\color[gray]{0.75}FO}%
\colorbox{green}{\color[gray]{0.75}FO}%
\colorbox{green}{\color[gray]{0.75}FO}%
\colorbox{green}{\color[gray]{0.75}FO}%
\colorbox{green}{\color[gray]{0.75}FO}%
\colorbox{green}{\color[gray]{0.75}FO}%
\colorbox{green}{\color[gray]{0.75}FO}%
\colorbox{green}{\color[gray]{0.75}FO}%
\colorbox{green}{\color[gray]{0.75}FO}%
\colorbox{green}{\color[gray]{0.75}FO}%
\colorbox{green}{\color[gray]{0.75}FO}%
\colorbox{green}{\color[gray]{0.75}FO}%
\colorbox{green}{\color[gray]{0.75}FO}%
\colorbox{green}{\color[gray]{0.75}FO}%
\colorbox{green}{\color[gray]{0.75}FO}%
\colorbox{green}{\color[gray]{0.75}FO}%
\colorbox{green}{\color[gray]{0.75}FO}%
\colorbox{green}{\color[gray]{0.75}FO}%
\colorbox{green}{\color[gray]{0.75}FO}%
\colorbox{green}{\color[gray]{0.75}FO}%
\colorbox{green}{\color[gray]{0.75}FO}%
\colorbox{green}{\color[gray]{0.75}FO}%
\colorbox{green}{\color[gray]{0.75}FO}%
\colorbox{green}{\color[gray]{0.75}FO}%
\\
\colorbox{green}{\color[gray]{0.75}FO}%
\colorbox{green}{\color[gray]{0.75}FO}%
\colorbox{green}{\color[gray]{0.75}FO}%
\colorbox{green}{\color[gray]{0.75}FO}%
\colorbox{green}{\color[gray]{0.75}FO}%
\colorbox{green}{\color[gray]{0.75}FO}%
\colorbox{green}{\color[gray]{0.75}FO}%
\colorbox{green}{\color[gray]{0.75}FO}%
\colorbox{green}{\color[gray]{0.75}FO}%
\colorbox{green}{\color[gray]{0.75}FO}%
\colorbox{green}{\color[gray]{0.75}FO}%
\colorbox{green}{\color[gray]{0.75}FO}%
\colorbox{green}{\color[gray]{0.75}FO}%
\colorbox{green}{\color[gray]{0.75}FO}%
\colorbox{green}{\color[gray]{0.75}FO}%
\colorbox{green}{\color[gray]{0.75}FO}%
\colorbox{green}{\color[gray]{0.75}FO}%
\colorbox{green}{\color[gray]{0.75}FO}%
\colorbox{green}{\color[gray]{0.75}FO}%
\colorbox{green}{\color[gray]{0.75}FO}%
\colorbox{green}{\color[gray]{0.75}FO}%
\colorbox{green}{\color[gray]{0.75}FO}%
\colorbox{green}{\color[gray]{0.75}FO}%
\colorbox{green}{\color[gray]{0.75}FO}%
\colorbox{green}{\color[gray]{0.75}FO}%
\colorbox{green}{\color[gray]{0.75}FO}%
\colorbox{green}{\color[gray]{0.75}FO}%
\colorbox{green}{\color[gray]{0.75}FO}%
\colorbox{green}{\color[gray]{0.75}FO}%
\colorbox{green}{\color[gray]{0.75}FO}%
\colorbox{green}{\color[gray]{0.75}FO}%
\colorbox{green}{\color[gray]{0.75}FO}%
\colorbox{green}{\color[gray]{0.75}FO}%
\colorbox{green}{\color[gray]{0.75}FO}%
\colorbox{green}{\color[gray]{0.75}FO}%
\colorbox{green}{\color[gray]{0.75}FO}%
\colorbox{green}{\color[gray]{0.75}FO}%
\colorbox{green}{\color[gray]{0.75}FO}%
\colorbox{green}{\color[gray]{0.75}FO}%
\colorbox{green}{\color[gray]{0.75}FO}%
\colorbox{green}{\color[gray]{0.75}FO}%
\colorbox{green}{\color[gray]{0.75}FO}%
\colorbox{green}{\color[gray]{0.75}FO}%
\colorbox{green}{\color[gray]{0.75}FO}%
\colorbox{green}{\color[gray]{0.75}FO}%
\colorbox{green}{\color[gray]{0.75}FO}%
\colorbox{green}{\color[gray]{0.75}FO}%
\colorbox{green}{\color[gray]{0.75}FO}%
\colorbox{green}{\color[gray]{0.75}FO}%
\colorbox{green}{\color[gray]{0.75}FO}%
\colorbox{green}{\color[gray]{0.75}FO}%
\colorbox{green}{\color[gray]{0.75}FO}%
\colorbox{green}{\color[gray]{0.75}FO}%
\colorbox{green}{\color[gray]{0.75}FO}%
\colorbox{green}{\color[gray]{0.75}FO}%
\colorbox{green}{\color[gray]{0.75}FO}%
\colorbox{green}{\color[gray]{0.75}FO}%
\colorbox{green}{\color[gray]{0.75}FO}%
\colorbox{green}{\color[gray]{0.75}FO}%
\colorbox{green}{\color[gray]{0.75}FO}%
\colorbox{green}{\color[gray]{0.75}FO}%
\colorbox{green}{\color[gray]{0.75}FO}%
\colorbox{green}{\color[gray]{0.75}FO}%
\colorbox{green}{\color[gray]{0.75}FO}%
\colorbox{green}{\color[gray]{0.75}FO}%
\colorbox{green}{\color[gray]{0.75}FO}%
\colorbox{green}{\color[gray]{0.75}FO}%
\colorbox{green}{\color[gray]{0.75}FO}%
\colorbox{green}{\color[gray]{0.75}FO}%
\colorbox{green}{\color[gray]{0.75}FO}%
\colorbox{green}{\color[gray]{0.75}FO}%
\colorbox{green}{\color[gray]{0.75}FO}%
\colorbox{green}{\color[gray]{0.75}FO}%
\colorbox{green}{\color[gray]{0.75}FO}%
\colorbox{green}{\color[gray]{0.75}FO}%
\colorbox{green}{\color[gray]{0.75}FO}%
\colorbox{green}{\color[gray]{0.75}FO}%
\colorbox{green}{\color[gray]{0.75}FO}%
\colorbox{green}{\color[gray]{0.75}FO}%
\colorbox{green}{\color[gray]{0.75}FO}%
\colorbox{green}{\color[gray]{0.75}FO}%
\colorbox{green}{\color[gray]{0.75}FO}%
\colorbox{green}{\color[gray]{0.75}FO}%
\colorbox{green}{\color[gray]{0.75}FO}%
\colorbox{green}{\color[gray]{0.75}FO}%
\colorbox{green}{\color[gray]{0.75}FO}%
\colorbox{green}{\color[gray]{0.75}FO}%
\colorbox{green}{\color[gray]{0.75}FO}%
\colorbox{green}{\color[gray]{0.75}FO}%
\colorbox{green}{\color[gray]{0.75}FO}%
\colorbox{green}{\color[gray]{0.75}FO}%
\colorbox{green}{\color[gray]{0.75}FO}%
\colorbox{green}{\color[gray]{0.75}FO}%
\colorbox{green}{\color[gray]{0.75}FO}%
\colorbox{green}{\color[gray]{0.75}FO}%
\colorbox{green}{\color[gray]{0.75}FO}%
\colorbox{green}{\color[gray]{0.75}FO}%
\colorbox{green}{\color[gray]{0.75}FO}%
\colorbox{green}{\color[gray]{0.75}FO}%
\colorbox{green}{\color[gray]{0.75}FO}%
\\
\colorbox{green}{\color[gray]{0.75}FO}%
\colorbox{green}{\color[gray]{0.75}FO}%
\colorbox{green}{\color[gray]{0.75}FO}%
\colorbox{green}{\color[gray]{0.75}FO}%
\colorbox{green}{\color[gray]{0.75}FO}%
\colorbox{green}{\color[gray]{0.75}FO}%
\colorbox{green}{\color[gray]{0.75}FO}%
\colorbox{green}{\color[gray]{0.75}FO}%
\colorbox{green}{\color[gray]{0.75}FO}%
\colorbox{green}{\color[gray]{0.75}FO}%
\colorbox{green}{\color[gray]{0.75}FO}%
\colorbox{green}{\color[gray]{0.75}FO}%
\colorbox{green}{\color[gray]{0.75}FO}%
\colorbox{green}{\color[gray]{0.75}FO}%
\colorbox{green}{\color[gray]{0.75}FO}%
\colorbox{green}{\color[gray]{0.75}FO}%
\colorbox{green}{\color[gray]{0.75}FO}%
\colorbox{green}{\color[gray]{0.75}FO}%
\colorbox{green}{\color[gray]{0.75}FO}%
\colorbox{green}{\color[gray]{0.75}FO}%
\colorbox{green}{\color[gray]{0.75}FO}%
\colorbox{green}{\color[gray]{0.75}FO}%
\colorbox{green}{\color[gray]{0.75}FO}%
\colorbox{green}{\color[gray]{0.75}FO}%
\colorbox{green}{\color[gray]{0.75}FO}%
\colorbox{green}{\color[gray]{0.75}FO}%
\colorbox{green}{\color[gray]{0.75}FO}%
\colorbox{green}{\color[gray]{0.75}FO}%
\colorbox{green}{\color[gray]{0.75}FO}%
\colorbox{green}{\color[gray]{0.75}FO}%
\colorbox{green}{\color[gray]{0.75}FO}%
\colorbox{green}{\color[gray]{0.75}FO}%
\colorbox{green}{\color[gray]{0.75}FO}%
\colorbox{green}{\color[gray]{0.75}FO}%
\colorbox{green}{\color[gray]{0.75}FO}%
\colorbox{green}{\color[gray]{0.75}FO}%
\colorbox{green}{\color[gray]{0.75}FO}%
\colorbox{green}{\color[gray]{0.75}FO}%
\colorbox{green}{\color[gray]{0.75}FO}%
\colorbox{green}{\color[gray]{0.75}FO}%
\colorbox{green}{\color[gray]{0.75}FO}%
\colorbox{green}{\color[gray]{0.75}FO}%
\colorbox{green}{\color[gray]{0.75}FO}%
\colorbox{green}{\color[gray]{0.75}FO}%
\colorbox{green}{\color[gray]{0.75}FO}%
\colorbox{green}{\color[gray]{0.75}FO}%
\colorbox{green}{\color[gray]{0.75}FO}%
\colorbox{green}{\color[gray]{0.75}FO}%
\colorbox{green}{\color[gray]{0.75}FO}%
\colorbox{green}{\color[gray]{0.75}FO}%
\colorbox{green}{\color[gray]{0.75}FO}%
\colorbox{green}{\color[gray]{0.75}FO}%
\colorbox{green}{\color[gray]{0.75}FO}%
\colorbox{green}{\color[gray]{0.75}FO}%
\colorbox{green}{\color[gray]{0.75}FO}%
\colorbox{green}{\color[gray]{0.75}FO}%
\colorbox{green}{\color[gray]{0.75}FO}%
\colorbox{green}{\color[gray]{0.75}FO}%
\colorbox{green}{\color[gray]{0.75}FO}%
\colorbox{green}{\color[gray]{0.75}FO}%
\colorbox{green}{\color[gray]{0.75}FO}%
\colorbox{green}{\color[gray]{0.75}FO}%
\colorbox{green}{\color[gray]{0.75}FO}%
\colorbox{green}{\color[gray]{0.75}FO}%
\colorbox{green}{\color[gray]{0.75}FO}%
\colorbox{green}{\color[gray]{0.75}FO}%
\colorbox{green}{\color[gray]{0.75}FO}%
\colorbox{green}{\color[gray]{0.75}FO}%
\colorbox{green}{\color[gray]{0.75}FO}%
\colorbox{green}{\color[gray]{0.75}FO}%
\colorbox{green}{\color[gray]{0.75}FO}%
\colorbox{green}{\color[gray]{0.75}FO}%
\colorbox{green}{\color[gray]{0.75}FO}%
\colorbox{green}{\color[gray]{0.75}FO}%
\colorbox{green}{\color[gray]{0.75}FO}%
\colorbox{green}{\color[gray]{0.75}FO}%
\colorbox{green}{\color[gray]{0.75}FO}%
\colorbox{green}{\color[gray]{0.75}FO}%
\colorbox{green}{\color[gray]{0.75}FO}%
\colorbox{green}{\color[gray]{0.75}FO}%
\colorbox{green}{\color[gray]{0.75}FO}%
\colorbox{green}{\color[gray]{0.75}FO}%
\colorbox{green}{\color[gray]{0.75}FO}%
\colorbox{green}{\color[gray]{0.75}FO}%
\colorbox{green}{\color[gray]{0.75}FO}%
\colorbox{green}{\color[gray]{0.75}FO}%
\colorbox{green}{\color[gray]{0.75}FO}%
\colorbox{green}{\color[gray]{0.75}FO}%
\colorbox{green}{\color[gray]{0.75}FO}%
\colorbox{green}{\color[gray]{0.75}FO}%
\colorbox{green}{\color[gray]{0.75}FO}%
\colorbox{green}{\color[gray]{0.75}FO}%
\colorbox{green}{\color[gray]{0.75}FO}%
\colorbox{green}{\color[gray]{0.75}FO}%
\colorbox{green}{\color[gray]{0.75}FO}%
\colorbox{green}{\color[gray]{0.75}FO}%
\colorbox{green}{\color[gray]{0.75}FO}%
\colorbox{green}{\color[gray]{0.75}FO}%
\colorbox{green}{\color[gray]{0.75}FO}%
\colorbox{green}{\color[gray]{0.75}FO}%
\\
\colorbox{green}{\color[gray]{0.75}FO}%
\colorbox{green}{\color[gray]{0.75}FO}%
\colorbox{green}{\color[gray]{0.75}FO}%
\colorbox{green}{\color[gray]{0.75}FO}%
\colorbox{green}{\color[gray]{0.75}FO}%
\colorbox{green}{\color[gray]{0.75}FO}%
\colorbox{green}{\color[gray]{0.75}FO}%
\colorbox{green}{\color[gray]{0.75}FO}%
\colorbox{green}{\color[gray]{0.75}FO}%
\colorbox{green}{\color[gray]{0.75}FO}%
\colorbox{green}{\color[gray]{0.75}FO}%
\colorbox{green}{\color[gray]{0.75}FO}%
\colorbox{green}{\color[gray]{0.75}FO}%
\colorbox{green}{\color[gray]{0.75}FO}%
\colorbox{green}{\color[gray]{0.75}FO}%
\colorbox{green}{\color[gray]{0.75}FO}%
\colorbox{green}{\color[gray]{0.75}FO}%
\colorbox{green}{\color[gray]{0.75}FO}%
\colorbox{green}{\color[gray]{0.75}FO}%
\colorbox{green}{\color[gray]{0.75}FO}%
\colorbox{green}{\color[gray]{0.75}FO}%
\colorbox{green}{\color[gray]{0.75}FO}%
\colorbox{green}{\color[gray]{0.75}FO}%
\colorbox{green}{\color[gray]{0.75}FO}%
\colorbox{green}{\color[gray]{0.75}FO}%
\colorbox{green}{\color[gray]{0.75}FO}%
\colorbox{green}{\color[gray]{0.75}FO}%
\colorbox{green}{\color[gray]{0.75}FO}%
\colorbox{green}{\color[gray]{0.75}FO}%
\colorbox{green}{\color[gray]{0.75}FO}%
\colorbox{green}{\color[gray]{0.75}FO}%
\colorbox{green}{\color[gray]{0.75}FO}%
\colorbox{green}{\color[gray]{0.75}FO}%
\colorbox{green}{\color[gray]{0.75}FO}%
\colorbox{green}{\color[gray]{0.75}FO}%
\colorbox{green}{\color[gray]{0.75}FO}%
\colorbox{green}{\color[gray]{0.75}FO}%
\colorbox{green}{\color[gray]{0.75}FO}%
\colorbox{green}{\color[gray]{0.75}FO}%
\colorbox{green}{\color[gray]{0.75}FO}%
\colorbox{green}{\color[gray]{0.75}FO}%
\colorbox{green}{\color[gray]{0.75}FO}%
\colorbox{green}{\color[gray]{0.75}FO}%
\colorbox{green}{\color[gray]{0.75}FO}%
\colorbox{green}{\color[gray]{0.75}FO}%
\colorbox{green}{\color[gray]{0.75}FO}%
\colorbox{green}{\color[gray]{0.75}FO}%
\colorbox{green}{\color[gray]{0.75}FO}%
\colorbox{green}{\color[gray]{0.75}FO}%
\colorbox{green}{\color[gray]{0.75}FO}%
\colorbox{green}{\color[gray]{0.75}FO}%
\colorbox{green}{\color[gray]{0.75}FO}%
\colorbox{green}{\color[gray]{0.75}FO}%
\colorbox{green}{\color[gray]{0.75}FO}%
\colorbox{green}{\color[gray]{0.75}FO}%
\colorbox{green}{\color[gray]{0.75}FO}%
\colorbox{green}{\color[gray]{0.75}FO}%
\colorbox{green}{\color[gray]{0.75}FO}%
\colorbox{green}{\color[gray]{0.75}FO}%
\colorbox{green}{\color[gray]{0.75}FO}%
\colorbox{green}{\color[gray]{0.75}FO}%
\colorbox{green}{\color[gray]{0.75}FO}%
\colorbox{green}{\color[gray]{0.75}FO}%
\colorbox{green}{\color[gray]{0.75}FO}%
\colorbox{green}{\color[gray]{0.75}FO}%
\colorbox{green}{\color[gray]{0.75}FO}%
\colorbox{green}{\color[gray]{0.75}FO}%
\colorbox{green}{\color[gray]{0.75}FO}%
\colorbox{green}{\color[gray]{0.75}FO}%
\colorbox{green}{\color[gray]{0.75}FO}%
\colorbox{green}{\color[gray]{0.75}FO}%
\colorbox{green}{\color[gray]{0.75}FO}%
\colorbox{green}{\color[gray]{0.75}FO}%
\colorbox{green}{\color[gray]{0.75}FO}%
\colorbox{green}{\color[gray]{0.75}FO}%
\colorbox{green}{\color[gray]{0.75}FO}%
\colorbox{green}{\color[gray]{0.75}FO}%
\colorbox{green}{\color[gray]{0.75}FO}%
\colorbox{green}{\color[gray]{0.75}FO}%
\colorbox{green}{\color[gray]{0.75}FO}%
\colorbox{green}{\color[gray]{0.75}FO}%
\colorbox{green}{\color[gray]{0.75}FO}%
\colorbox{green}{\color[gray]{0.75}FO}%
\colorbox{green}{\color[gray]{0.75}FO}%
\colorbox{green}{\color[gray]{0.75}FO}%
\colorbox{green}{\color[gray]{0.75}FO}%
\colorbox{green}{\color[gray]{0.75}FO}%
\colorbox{green}{\color[gray]{0.75}FO}%
\colorbox{green}{\color[gray]{0.75}FO}%
\colorbox{green}{\color[gray]{0.75}FO}%
\colorbox{green}{\color[gray]{0.75}FO}%
\colorbox{green}{\color[gray]{0.75}FO}%
\colorbox{green}{\color[gray]{0.75}FO}%
\colorbox{green}{\color[gray]{0.75}FO}%
\colorbox{green}{\color[gray]{0.75}FO}%
\colorbox{green}{\color[gray]{0.75}FO}%
\colorbox{green}{\color[gray]{0.75}FO}%
\colorbox{green}{\color[gray]{0.75}FO}%
\colorbox{green}{\color[gray]{0.75}FO}%
\colorbox{green}{\color[gray]{0.75}FO}%
\\
\colorbox{green}{\color[gray]{0.75}FO}%
\colorbox{green}{\color[gray]{0.75}FO}%
\colorbox{green}{\color[gray]{0.75}FO}%
\colorbox{green}{\color[gray]{0.75}FO}%
\colorbox{green}{\color[gray]{0.75}FO}%
\colorbox{green}{\color[gray]{0.75}FO}%
\colorbox{green}{\color[gray]{0.75}FO}%
\colorbox{green}{\color[gray]{0.75}FO}%
\colorbox{green}{\color[gray]{0.75}FO}%
\colorbox{green}{\color[gray]{0.75}FO}%
\colorbox{green}{\color[gray]{0.75}FO}%
\colorbox{green}{\color[gray]{0.75}FO}%
\colorbox{green}{\color[gray]{0.75}FO}%
\colorbox{green}{\color[gray]{0.75}FO}%
\colorbox{green}{\color[gray]{0.75}FO}%
\colorbox{green}{\color[gray]{0.75}FO}%
\colorbox{green}{\color[gray]{0.75}FO}%
\colorbox{green}{\color[gray]{0.75}FO}%
\colorbox{green}{\color[gray]{0.75}FO}%
\colorbox{green}{\color[gray]{0.75}FO}%
\colorbox{green}{\color[gray]{0.75}FO}%
\colorbox{green}{\color[gray]{0.75}FO}%
\colorbox{green}{\color[gray]{0.75}FO}%
\colorbox{green}{\color[gray]{0.75}FO}%
\colorbox{green}{\color[gray]{0.75}FO}%
\colorbox{green}{\color[gray]{0.75}FO}%
\colorbox{green}{\color[gray]{0.75}FO}%
\colorbox{green}{\color[gray]{0.75}FO}%
\colorbox{green}{\color[gray]{0.75}FO}%
\colorbox{green}{\color[gray]{0.75}FO}%
\colorbox{green}{\color[gray]{0.75}FO}%
\colorbox{green}{\color[gray]{0.75}FO}%
\colorbox{green}{\color[gray]{0.75}FO}%
\colorbox{green}{\color[gray]{0.75}FO}%
\colorbox{green}{\color[gray]{0.75}FO}%
\colorbox{green}{\color[gray]{0.75}FO}%
\colorbox{green}{\color[gray]{0.75}FO}%
\colorbox{green}{\color[gray]{0.75}FO}%
\colorbox{green}{\color[gray]{0.75}FO}%
\colorbox{green}{\color[gray]{0.75}FO}%
\colorbox{green}{\color[gray]{0.75}FO}%
\colorbox{green}{\color[gray]{0.75}FO}%
\colorbox{green}{\color[gray]{0.75}FO}%
\colorbox{green}{\color[gray]{0.75}FO}%
\colorbox{green}{\color[gray]{0.75}FO}%
\colorbox{green}{\color[gray]{0.75}FO}%
\colorbox{green}{\color[gray]{0.75}FO}%
\colorbox{green}{\color[gray]{0.75}FO}%
\colorbox{green}{\color[gray]{0.75}FO}%
\colorbox{green}{\color[gray]{0.75}FO}%
\colorbox{green}{\color[gray]{0.75}FO}%
\colorbox{green}{\color[gray]{0.75}FO}%
\colorbox{green}{\color[gray]{0.75}FO}%
\colorbox{green}{\color[gray]{0.75}FO}%
\colorbox{green}{\color[gray]{0.75}FO}%
\colorbox{green}{\color[gray]{0.75}FO}%
\colorbox{green}{\color[gray]{0.75}FO}%
\colorbox{green}{\color[gray]{0.75}FO}%
\colorbox{green}{\color[gray]{0.75}FO}%
\colorbox{green}{\color[gray]{0.75}FO}%
\colorbox{green}{\color[gray]{0.75}FO}%
\colorbox{green}{\color[gray]{0.75}FO}%
\colorbox{green}{\color[gray]{0.75}FO}%
\colorbox{green}{\color[gray]{0.75}FO}%
\colorbox{green}{\color[gray]{0.75}FO}%
\colorbox{green}{\color[gray]{0.75}FO}%
\colorbox{green}{\color[gray]{0.75}FO}%
\colorbox{green}{\color[gray]{0.75}FO}%
\colorbox{green}{\color[gray]{0.75}FO}%
\colorbox{green}{\color[gray]{0.75}FO}%
\colorbox{green}{\color[gray]{0.75}FO}%
\colorbox{green}{\color[gray]{0.75}FO}%
\colorbox{green}{\color[gray]{0.75}FO}%
\colorbox{green}{\color[gray]{0.75}FO}%
\colorbox{green}{\color[gray]{0.75}FO}%
\colorbox{green}{\color[gray]{0.75}FO}%
\colorbox{green}{\color[gray]{0.75}FO}%
\colorbox{green}{\color[gray]{0.75}FO}%
\colorbox{green}{\color[gray]{0.75}FO}%
\colorbox{green}{\color[gray]{0.75}FO}%
\colorbox{green}{\color[gray]{0.75}FO}%
\colorbox{green}{\color[gray]{0.75}FO}%
\colorbox{green}{\color[gray]{0.75}FO}%
\colorbox{green}{\color[gray]{0.75}FO}%
\colorbox{green}{\color[gray]{0.75}FO}%
\colorbox{green}{\color[gray]{0.75}FO}%
\colorbox{green}{\color[gray]{0.75}FO}%
\colorbox{green}{\color[gray]{0.75}FO}%
\colorbox{green}{\color[gray]{0.75}FO}%
\colorbox{green}{\color[gray]{0.75}FO}%
\colorbox{green}{\color[gray]{0.75}FO}%
\colorbox{green}{\color[gray]{0.75}FO}%
\colorbox{green}{\color[gray]{0.75}FO}%
\colorbox{green}{\color[gray]{0.75}FO}%
\colorbox{green}{\color[gray]{0.75}FO}%
\colorbox{green}{\color[gray]{0.75}FO}%
\colorbox{green}{\color[gray]{0.75}FO}%
\colorbox{green}{\color[gray]{0.75}FO}%
\colorbox{green}{\color[gray]{0.75}FO}%
\colorbox{green}{\color[gray]{0.75}FO}%
\\
\colorbox{green}{\color[gray]{0.75}FO}%
\colorbox{green}{\color[gray]{0.75}FO}%
\colorbox{green}{\color[gray]{0.75}FO}%
\colorbox{green}{\color[gray]{0.75}FO}%
\colorbox{green}{\color[gray]{0.75}FO}%
\colorbox{green}{\color[gray]{0.75}FO}%
\colorbox{green}{\color[gray]{0.75}FO}%
\colorbox{green}{\color[gray]{0.75}FO}%
\colorbox{green}{\color[gray]{0.75}FO}%
\colorbox{green}{\color[gray]{0.75}FO}%
\colorbox{green}{\color[gray]{0.75}FO}%
\colorbox{green}{\color[gray]{0.75}FO}%
\colorbox{green}{\color[gray]{0.75}FO}%
\colorbox{green}{\color[gray]{0.75}FO}%
\colorbox{green}{\color[gray]{0.75}FO}%
\colorbox{green}{\color[gray]{0.75}FO}%
\colorbox{green}{\color[gray]{0.75}FO}%
\colorbox{green}{\color[gray]{0.75}FO}%
\colorbox{green}{\color[gray]{0.75}FO}%
\colorbox{green}{\color[gray]{0.75}FO}%
\colorbox{green}{\color[gray]{0.75}FO}%
\colorbox{green}{\color[gray]{0.75}FO}%
\colorbox{green}{\color[gray]{0.75}FO}%
\colorbox{green}{\color[gray]{0.75}FO}%
\colorbox{green}{\color[gray]{0.75}FO}%
\colorbox{green}{\color[gray]{0.75}FO}%
\colorbox{green}{\color[gray]{0.75}FO}%
\colorbox{green}{\color[gray]{0.75}FO}%
\colorbox{green}{\color[gray]{0.75}FO}%
\colorbox{green}{\color[gray]{0.75}FO}%
\colorbox{green}{\color[gray]{0.75}FO}%
\colorbox{green}{\color[gray]{0.75}FO}%
\colorbox{green}{\color[gray]{0.75}FO}%
\colorbox{green}{\color[gray]{0.75}FO}%
\colorbox{green}{\color[gray]{0.75}FO}%
\colorbox{green}{\color[gray]{0.75}FO}%
\colorbox{green}{\color[gray]{0.75}FO}%
\colorbox{green}{\color[gray]{0.75}FO}%
\colorbox{green}{\color[gray]{0.75}FO}%
\colorbox{green}{\color[gray]{0.75}FO}%
\colorbox{green}{\color[gray]{0.75}FO}%
\colorbox{green}{\color[gray]{0.75}FO}%
\colorbox{green}{\color[gray]{0.75}FO}%
\colorbox{green}{\color[gray]{0.75}FO}%
\colorbox{green}{\color[gray]{0.75}FO}%
\colorbox{green}{\color[gray]{0.75}FO}%
\colorbox{green}{\color[gray]{0.75}FO}%
\colorbox{green}{\color[gray]{0.75}FO}%
\colorbox{green}{\color[gray]{0.75}FO}%
\colorbox{green}{\color[gray]{0.75}FO}%
\colorbox{green}{\color[gray]{0.75}FO}%
\colorbox{green}{\color[gray]{0.75}FO}%
\colorbox{green}{\color[gray]{0.75}FO}%
\colorbox{green}{\color[gray]{0.75}FO}%
\colorbox{green}{\color[gray]{0.75}FO}%
\colorbox{green}{\color[gray]{0.75}FO}%
\colorbox{green}{\color[gray]{0.75}FO}%
\colorbox{green}{\color[gray]{0.75}FO}%
\colorbox{green}{\color[gray]{0.75}FO}%
\colorbox{green}{\color[gray]{0.75}FO}%
\colorbox{green}{\color[gray]{0.75}FO}%
\colorbox{green}{\color[gray]{0.75}FO}%
\colorbox{green}{\color[gray]{0.75}FO}%
\colorbox{green}{\color[gray]{0.75}FO}%
\colorbox{green}{\color[gray]{0.75}FO}%
\colorbox{green}{\color[gray]{0.75}FO}%
\colorbox{green}{\color[gray]{0.75}FO}%
\colorbox{green}{\color[gray]{0.75}FO}%
\colorbox{green}{\color[gray]{0.75}FO}%
\colorbox{green}{\color[gray]{0.75}FO}%
\colorbox{green}{\color[gray]{0.75}FO}%
\colorbox{green}{\color[gray]{0.75}FO}%
\colorbox{green}{\color[gray]{0.75}FO}%
\colorbox{green}{\color[gray]{0.75}FO}%
\colorbox{green}{\color[gray]{0.75}FO}%
\colorbox{green}{\color[gray]{0.75}FO}%
\colorbox{green}{\color[gray]{0.75}FO}%
\colorbox{green}{\color[gray]{0.75}FO}%
\colorbox{green}{\color[gray]{0.75}FO}%
\colorbox{green}{\color[gray]{0.75}FO}%
\colorbox{green}{\color[gray]{0.75}FO}%
\colorbox{green}{\color[gray]{0.75}FO}%
\colorbox{green}{\color[gray]{0.75}FO}%
\colorbox{green}{\color[gray]{0.75}FO}%
\colorbox{green}{\color[gray]{0.75}FO}%
\colorbox{green}{\color[gray]{0.75}FO}%
\colorbox{green}{\color[gray]{0.75}FO}%
\colorbox{green}{\color[gray]{0.75}FO}%
\colorbox{green}{\color[gray]{0.75}FO}%
\colorbox{green}{\color[gray]{0.75}FO}%
\colorbox{green}{\color[gray]{0.75}FO}%
\colorbox{green}{\color[gray]{0.75}FO}%
\colorbox{green}{\color[gray]{0.75}FO}%
\colorbox{green}{\color[gray]{0.75}FO}%
\colorbox{green}{\color[gray]{0.75}FO}%
\colorbox{green}{\color[gray]{0.75}FO}%
\colorbox{green}{\color[gray]{0.75}FO}%
\colorbox{green}{\color[gray]{0.75}FO}%
\colorbox{green}{\color[gray]{0.75}FO}%
\colorbox{green}{\color[gray]{0.75}FO}%
\\
\colorbox{green}{\color[gray]{0.75}FO}%
\colorbox{green}{\color[gray]{0.75}FO}%
\colorbox{green}{\color[gray]{0.75}FO}%
\colorbox{green}{\color[gray]{0.75}FO}%
\colorbox{green}{\color[gray]{0.75}FO}%
\colorbox{green}{\color[gray]{0.75}FO}%
\colorbox{green}{\color[gray]{0.75}FO}%
\colorbox{green}{\color[gray]{0.75}FO}%
\colorbox{green}{\color[gray]{0.75}FO}%
\colorbox{green}{\color[gray]{0.75}FO}%
\colorbox{green}{\color[gray]{0.75}FO}%
\colorbox{green}{\color[gray]{0.75}FO}%
\colorbox{green}{\color[gray]{0.75}FO}%
\colorbox{green}{\color[gray]{0.75}FO}%
\colorbox{green}{\color[gray]{0.75}FO}%
\colorbox{green}{\color[gray]{0.75}FO}%
\colorbox{green}{\color[gray]{0.75}FO}%
\colorbox{green}{\color[gray]{0.75}FO}%
\colorbox{green}{\color[gray]{0.75}FO}%
\colorbox{green}{\color[gray]{0.75}FO}%
\colorbox{green}{\color[gray]{0.75}FO}%
\colorbox{green}{\color[gray]{0.75}FO}%
\colorbox{green}{\color[gray]{0.75}FO}%
\colorbox{green}{\color[gray]{0.75}FO}%
\colorbox{green}{\color[gray]{0.75}FO}%
\colorbox{green}{\color[gray]{0.75}FO}%
\colorbox{green}{\color[gray]{0.75}FO}%
\colorbox{green}{\color[gray]{0.75}FO}%
\colorbox{green}{\color[gray]{0.75}FO}%
\colorbox{green}{\color[gray]{0.75}FO}%
\colorbox{green}{\color[gray]{0.75}FO}%
\colorbox{green}{\color[gray]{0.75}FO}%
\colorbox{green}{\color[gray]{0.75}FO}%
\colorbox{green}{\color[gray]{0.75}FO}%
\colorbox{green}{\color[gray]{0.75}FO}%
\colorbox{green}{\color[gray]{0.75}FO}%
\colorbox{green}{\color[gray]{0.75}FO}%
\colorbox{green}{\color[gray]{0.75}FO}%
\colorbox{green}{\color[gray]{0.75}FO}%
\colorbox{green}{\color[gray]{0.75}FO}%
\colorbox{green}{\color[gray]{0.75}FO}%
\colorbox{green}{\color[gray]{0.75}FO}%
\colorbox{green}{\color[gray]{0.75}FO}%
\colorbox{green}{\color[gray]{0.75}FO}%
\colorbox{green}{\color[gray]{0.75}FO}%
\colorbox{green}{\color[gray]{0.75}FO}%
\colorbox{green}{\color[gray]{0.75}FO}%
\colorbox{green}{\color[gray]{0.75}FO}%
\colorbox{green}{\color[gray]{0.75}FO}%
\colorbox{green}{\color[gray]{0.75}FO}%
\colorbox{green}{\color[gray]{0.75}FO}%
\colorbox{green}{\color[gray]{0.75}FO}%
\colorbox{green}{\color[gray]{0.75}FO}%
\colorbox{green}{\color[gray]{0.75}FO}%
\colorbox{green}{\color[gray]{0.75}FO}%
\colorbox{green}{\color[gray]{0.75}FO}%
\colorbox{green}{\color[gray]{0.75}FO}%
\colorbox{green}{\color[gray]{0.75}FO}%
\colorbox{green}{\color[gray]{0.75}FO}%
\colorbox{green}{\color[gray]{0.75}FO}%
\colorbox{green}{\color[gray]{0.75}FO}%
\colorbox{green}{\color[gray]{0.75}FO}%
\colorbox{green}{\color[gray]{0.75}FO}%
\colorbox{green}{\color[gray]{0.75}FO}%
\colorbox{green}{\color[gray]{0.75}FO}%
\colorbox{green}{\color[gray]{0.75}FO}%
\colorbox{green}{\color[gray]{0.75}FO}%
\colorbox{green}{\color[gray]{0.75}FO}%
\colorbox{green}{\color[gray]{0.75}FO}%
\colorbox{green}{\color[gray]{0.75}FO}%
\colorbox{green}{\color[gray]{0.75}FO}%
\colorbox{green}{\color[gray]{0.75}FO}%
\colorbox{green}{\color[gray]{0.75}FO}%
\colorbox{green}{\color[gray]{0.75}FO}%
\colorbox{green}{\color[gray]{0.75}FO}%
\colorbox{green}{\color[gray]{0.75}FO}%
\colorbox{green}{\color[gray]{0.75}FO}%
\colorbox{green}{\color[gray]{0.75}FO}%
\colorbox{green}{\color[gray]{0.75}FO}%
\colorbox{green}{\color[gray]{0.75}FO}%
\colorbox{green}{\color[gray]{0.75}FO}%
\colorbox{green}{\color[gray]{0.75}FO}%
\colorbox{green}{\color[gray]{0.75}FO}%
\colorbox{green}{\color[gray]{0.75}FO}%
\colorbox{green}{\color[gray]{0.75}FO}%
\colorbox{green}{\color[gray]{0.75}FO}%
\colorbox{green}{\color[gray]{0.75}FO}%
\colorbox{green}{\color[gray]{0.75}FO}%
\colorbox{green}{\color[gray]{0.75}FO}%
\colorbox{green}{\color[gray]{0.75}FO}%
\colorbox{green}{\color[gray]{0.75}FO}%
\colorbox{green}{\color[gray]{0.75}FO}%
\colorbox{green}{\color[gray]{0.75}FO}%
\colorbox{green}{\color[gray]{0.75}FO}%
\colorbox{green}{\color[gray]{0.75}FO}%
\colorbox{green}{\color[gray]{0.75}FO}%
\colorbox{green}{\color[gray]{0.75}FO}%
\colorbox{green}{\color[gray]{0.75}FO}%
\colorbox{green}{\color[gray]{0.75}FO}%
\colorbox{green}{\color[gray]{0.75}FO}%
\\
\colorbox{green}{\color[gray]{0.75}FO}%
\colorbox{green}{\color[gray]{0.75}FO}%
\colorbox{green}{\color[gray]{0.75}FO}%
\colorbox{green}{\color[gray]{0.75}FO}%
\colorbox{green}{\color[gray]{0.75}FO}%
\colorbox{green}{\color[gray]{0.75}FO}%
\colorbox{green}{\color[gray]{0.75}FO}%
\colorbox{green}{\color[gray]{0.75}FO}%
\colorbox{green}{\color[gray]{0.75}FO}%
\colorbox{green}{\color[gray]{0.75}FO}%
\colorbox{green}{\color[gray]{0.75}FO}%
\colorbox{green}{\color[gray]{0.75}FO}%
\colorbox{green}{\color[gray]{0.75}FO}%
\colorbox{green}{\color[gray]{0.75}FO}%
\colorbox{green}{\color[gray]{0.75}FO}%
\colorbox{green}{\color[gray]{0.75}FO}%
\colorbox{green}{\color[gray]{0.75}FO}%
\colorbox{green}{\color[gray]{0.75}FO}%
\colorbox{green}{\color[gray]{0.75}FO}%
\colorbox{green}{\color[gray]{0.75}FO}%
\colorbox{green}{\color[gray]{0.75}FO}%
\colorbox{green}{\color[gray]{0.75}FO}%
\colorbox{green}{\color[gray]{0.75}FO}%
\colorbox{green}{\color[gray]{0.75}FO}%
\colorbox{green}{\color[gray]{0.75}FO}%
\colorbox{green}{\color[gray]{0.75}FO}%
\colorbox{green}{\color[gray]{0.75}FO}%
\colorbox{green}{\color[gray]{0.75}FO}%
\colorbox{green}{\color[gray]{0.75}FO}%
\colorbox{green}{\color[gray]{0.75}FO}%
\colorbox{green}{\color[gray]{0.75}FO}%
\colorbox{green}{\color[gray]{0.75}FO}%
\colorbox{green}{\color[gray]{0.75}FO}%
\colorbox{green}{\color[gray]{0.75}FO}%
\colorbox{green}{\color[gray]{0.75}FO}%
\colorbox{green}{\color[gray]{0.75}FO}%
\colorbox{green}{\color[gray]{0.75}FO}%
\colorbox{green}{\color[gray]{0.75}FO}%
\colorbox{green}{\color[gray]{0.75}FO}%
\colorbox{green}{\color[gray]{0.75}FO}%
\colorbox{green}{\color[gray]{0.75}FO}%
\colorbox{green}{\color[gray]{0.75}FO}%
\colorbox{green}{\color[gray]{0.75}FO}%
\colorbox{green}{\color[gray]{0.75}FO}%
\colorbox{green}{\color[gray]{0.75}FO}%
\colorbox{green}{\color[gray]{0.75}FO}%
\colorbox{green}{\color[gray]{0.75}FO}%
\colorbox{green}{\color[gray]{0.75}FO}%
\colorbox{green}{\color[gray]{0.75}FO}%
\colorbox{green}{\color[gray]{0.75}FO}%
\colorbox{green}{\color[gray]{0.75}FO}%
\colorbox{green}{\color[gray]{0.75}FO}%
\colorbox{green}{\color[gray]{0.75}FO}%
\colorbox{green}{\color[gray]{0.75}FO}%
\colorbox{green}{\color[gray]{0.75}FO}%
\colorbox{green}{\color[gray]{0.75}FO}%
\colorbox{green}{\color[gray]{0.75}FO}%
\colorbox{green}{\color[gray]{0.75}FO}%
\colorbox{green}{\color[gray]{0.75}FO}%
\colorbox{green}{\color[gray]{0.75}FO}%
\colorbox{green}{\color[gray]{0.75}FO}%
\colorbox{green}{\color[gray]{0.75}FO}%
\colorbox{green}{\color[gray]{0.75}FO}%
\colorbox{green}{\color[gray]{0.75}FO}%
\colorbox{green}{\color[gray]{0.75}FO}%
\colorbox{green}{\color[gray]{0.75}FO}%
\colorbox{green}{\color[gray]{0.75}FO}%
\colorbox{green}{\color[gray]{0.75}FO}%
\colorbox{green}{\color[gray]{0.75}FO}%
\colorbox{green}{\color[gray]{0.75}FO}%
\colorbox{green}{\color[gray]{0.75}FO}%
\colorbox{green}{\color[gray]{0.75}FO}%
\colorbox{green}{\color[gray]{0.75}FO}%
\colorbox{green}{\color[gray]{0.75}FO}%
\colorbox{green}{\color[gray]{0.75}FO}%
\colorbox{green}{\color[gray]{0.75}FO}%
\colorbox{green}{\color[gray]{0.75}FO}%
\colorbox{green}{\color[gray]{0.75}FO}%
\colorbox{green}{\color[gray]{0.75}FO}%
\colorbox{green}{\color[gray]{0.75}FO}%
\colorbox{green}{\color[gray]{0.75}FO}%
\colorbox{green}{\color[gray]{0.75}FO}%
\colorbox{green}{\color[gray]{0.75}FO}%
\colorbox{green}{\color[gray]{0.75}FO}%
\colorbox{green}{\color[gray]{0.75}FO}%
\colorbox{green}{\color[gray]{0.75}FO}%
\colorbox{green}{\color[gray]{0.75}FO}%
\colorbox{green}{\color[gray]{0.75}FO}%
\colorbox{green}{\color[gray]{0.75}FO}%
\colorbox{green}{\color[gray]{0.75}FO}%
\colorbox{green}{\color[gray]{0.75}FO}%
\colorbox{green}{\color[gray]{0.75}FO}%
\colorbox{green}{\color[gray]{0.75}FO}%
\colorbox{green}{\color[gray]{0.75}FO}%
\colorbox{green}{\color[gray]{0.75}FO}%
\colorbox{green}{\color[gray]{0.75}FO}%
\colorbox{green}{\color[gray]{0.75}FO}%
\colorbox{green}{\color[gray]{0.75}FO}%
\colorbox{green}{\color[gray]{0.75}FO}%
\colorbox{green}{\color[gray]{0.75}FO}%
\\
\colorbox{green}{\color[gray]{0.75}FO}%
\colorbox{green}{\color[gray]{0.75}FO}%
\colorbox{green}{\color[gray]{0.75}FO}%
\colorbox{green}{\color[gray]{0.75}FO}%
\colorbox{green}{\color[gray]{0.75}FO}%
\colorbox{green}{\color[gray]{0.75}FO}%
\colorbox{green}{\color[gray]{0.75}FO}%
\colorbox{green}{\color[gray]{0.75}FO}%
\colorbox{green}{\color[gray]{0.75}FO}%
\colorbox{green}{\color[gray]{0.75}FO}%
\colorbox{green}{\color[gray]{0.75}FO}%
\colorbox{green}{\color[gray]{0.75}FO}%
\colorbox{green}{\color[gray]{0.75}FO}%
\colorbox{green}{\color[gray]{0.75}FO}%
\colorbox{green}{\color[gray]{0.75}FO}%
\colorbox{green}{\color[gray]{0.75}FO}%
\colorbox{green}{\color[gray]{0.75}FO}%
\colorbox{green}{\color[gray]{0.75}FO}%
\colorbox{green}{\color[gray]{0.75}FO}%
\colorbox{green}{\color[gray]{0.75}FO}%
\colorbox{green}{\color[gray]{0.75}FO}%
\colorbox{green}{\color[gray]{0.75}FO}%
\colorbox{green}{\color[gray]{0.75}FO}%
\colorbox{green}{\color[gray]{0.75}FO}%
\colorbox{green}{\color[gray]{0.75}FO}%
\colorbox{green}{\color[gray]{0.75}FO}%
\colorbox{green}{\color[gray]{0.75}FO}%
\colorbox{green}{\color[gray]{0.75}FO}%
\colorbox{green}{\color[gray]{0.75}FO}%
\colorbox{green}{\color[gray]{0.75}FO}%
\colorbox{green}{\color[gray]{0.75}FO}%
\colorbox{green}{\color[gray]{0.75}FO}%
\colorbox{green}{\color[gray]{0.75}FO}%
\colorbox{green}{\color[gray]{0.75}FO}%
\colorbox{green}{\color[gray]{0.75}FO}%
\colorbox{green}{\color[gray]{0.75}FO}%
\colorbox{green}{\color[gray]{0.75}FO}%
\colorbox{green}{\color[gray]{0.75}FO}%
\colorbox{green}{\color[gray]{0.75}FO}%
\colorbox{green}{\color[gray]{0.75}FO}%
\colorbox{green}{\color[gray]{0.75}FO}%
\colorbox{green}{\color[gray]{0.75}FO}%
\colorbox{green}{\color[gray]{0.75}FO}%
\colorbox{green}{\color[gray]{0.75}FO}%
\colorbox{green}{\color[gray]{0.75}FO}%
\colorbox{green}{\color[gray]{0.75}FO}%
\colorbox{green}{\color[gray]{0.75}FO}%
\colorbox{green}{\color[gray]{0.75}FO}%
\colorbox{green}{\color[gray]{0.75}FO}%
\colorbox{green}{\color[gray]{0.75}FO}%
\colorbox{green}{\color[gray]{0.75}FO}%
\colorbox{green}{\color[gray]{0.75}FO}%
\colorbox{green}{\color[gray]{0.75}FO}%
\colorbox{green}{\color[gray]{0.75}FO}%
\colorbox{green}{\color[gray]{0.75}FO}%
\colorbox{green}{\color[gray]{0.75}FO}%
\colorbox{green}{\color[gray]{0.75}FO}%
\colorbox{green}{\color[gray]{0.75}FO}%
\colorbox{green}{\color[gray]{0.75}FO}%
\colorbox{green}{\color[gray]{0.75}FO}%
\colorbox{green}{\color[gray]{0.75}FO}%
\colorbox{green}{\color[gray]{0.75}FO}%
\colorbox{green}{\color[gray]{0.75}FO}%
\colorbox{green}{\color[gray]{0.75}FO}%
\colorbox{green}{\color[gray]{0.75}FO}%
\colorbox{green}{\color[gray]{0.75}FO}%
\colorbox{green}{\color[gray]{0.75}FO}%
\colorbox{green}{\color[gray]{0.75}FO}%
\colorbox{green}{\color[gray]{0.75}FO}%
\colorbox{green}{\color[gray]{0.75}FO}%
\colorbox{green}{\color[gray]{0.75}FO}%
\colorbox{green}{\color[gray]{0.75}FO}%
\colorbox{green}{\color[gray]{0.75}FO}%
\colorbox{green}{\color[gray]{0.75}FO}%
\colorbox{green}{\color[gray]{0.75}FO}%
\colorbox{green}{\color[gray]{0.75}FO}%
\colorbox{green}{\color[gray]{0.75}FO}%
\colorbox{green}{\color[gray]{0.75}FO}%
\colorbox{green}{\color[gray]{0.75}FO}%
\colorbox{green}{\color[gray]{0.75}FO}%
\colorbox{green}{\color[gray]{0.75}FO}%
\colorbox{green}{\color[gray]{0.75}FO}%
\colorbox{green}{\color[gray]{0.75}FO}%
\colorbox{green}{\color[gray]{0.75}FO}%
\colorbox{green}{\color[gray]{0.75}FO}%
\colorbox{green}{\color[gray]{0.75}FO}%
\colorbox{green}{\color[gray]{0.75}FO}%
\colorbox{green}{\color[gray]{0.75}FO}%
\colorbox{green}{\color[gray]{0.75}FO}%
\colorbox{green}{\color[gray]{0.75}FO}%
\colorbox{green}{\color[gray]{0.75}FO}%
\colorbox{green}{\color[gray]{0.75}FO}%
\colorbox{green}{\color[gray]{0.75}FO}%
\colorbox{green}{\color[gray]{0.75}FO}%
\colorbox{green}{\color[gray]{0.75}FO}%
\colorbox{green}{\color[gray]{0.75}FO}%
\colorbox{green}{\color[gray]{0.75}FO}%
\colorbox{green}{\color[gray]{0.75}FO}%
\colorbox{green}{\color[gray]{0.75}FO}%
\colorbox{green}{\color[gray]{0.75}FO}%
\\
\colorbox{green}{\color[gray]{0.75}FO}%
\colorbox{green}{\color[gray]{0.75}FO}%
\colorbox{green}{\color[gray]{0.75}FO}%
\colorbox{green}{\color[gray]{0.75}FO}%
\colorbox{green}{\color[gray]{0.75}FO}%
\colorbox{green}{\color[gray]{0.75}FO}%
\colorbox{green}{\color[gray]{0.75}FO}%
\colorbox{green}{\color[gray]{0.75}FO}%
\colorbox{green}{\color[gray]{0.75}FO}%
\colorbox{green}{\color[gray]{0.75}FO}%
\colorbox{green}{\color[gray]{0.75}FO}%
\colorbox{green}{\color[gray]{0.75}FO}%
\colorbox{green}{\color[gray]{0.75}FO}%
\colorbox{green}{\color[gray]{0.75}FO}%
\colorbox{green}{\color[gray]{0.75}FO}%
\colorbox{green}{\color[gray]{0.75}FO}%
\colorbox{green}{\color[gray]{0.75}FO}%
\colorbox{green}{\color[gray]{0.75}FO}%
\colorbox{green}{\color[gray]{0.75}FO}%
\colorbox{green}{\color[gray]{0.75}FO}%
\colorbox{green}{\color[gray]{0.75}FO}%
\colorbox{green}{\color[gray]{0.75}FO}%
\colorbox{green}{\color[gray]{0.75}FO}%
\colorbox{green}{\color[gray]{0.75}FO}%
\colorbox{green}{\color[gray]{0.75}FO}%
\colorbox{green}{\color[gray]{0.75}FO}%
\colorbox{green}{\color[gray]{0.75}FO}%
\colorbox{green}{\color[gray]{0.75}FO}%
\colorbox{green}{\color[gray]{0.75}FO}%
\colorbox{green}{\color[gray]{0.75}FO}%
\colorbox{green}{\color[gray]{0.75}FO}%
\colorbox{green}{\color[gray]{0.75}FO}%
\colorbox{green}{\color[gray]{0.75}FO}%
\colorbox{green}{\color[gray]{0.75}FO}%
\colorbox{green}{\color[gray]{0.75}FO}%
\colorbox{green}{\color[gray]{0.75}FO}%
\colorbox{green}{\color[gray]{0.75}FO}%
\colorbox{green}{\color[gray]{0.75}FO}%
\colorbox{green}{\color[gray]{0.75}FO}%
\colorbox{green}{\color[gray]{0.75}FO}%
\colorbox{green}{\color[gray]{0.75}FO}%
\colorbox{green}{\color[gray]{0.75}FO}%
\colorbox{green}{\color[gray]{0.75}FO}%
\colorbox{green}{\color[gray]{0.75}FO}%
\colorbox{green}{\color[gray]{0.75}FO}%
\colorbox{green}{\color[gray]{0.75}FO}%
\colorbox{green}{\color[gray]{0.75}FO}%
\colorbox{green}{\color[gray]{0.75}FO}%
\colorbox{green}{\color[gray]{0.75}FO}%
\colorbox{green}{\color[gray]{0.75}FO}%
\colorbox{green}{\color[gray]{0.75}FO}%
\colorbox{green}{\color[gray]{0.75}FO}%
\colorbox{green}{\color[gray]{0.75}FO}%
\colorbox{green}{\color[gray]{0.75}FO}%
\colorbox{green}{\color[gray]{0.75}FO}%
\colorbox{green}{\color[gray]{0.75}FO}%
\colorbox{green}{\color[gray]{0.75}FO}%
\colorbox{green}{\color[gray]{0.75}FO}%
\colorbox{green}{\color[gray]{0.75}FO}%
\colorbox{green}{\color[gray]{0.75}FO}%
\colorbox{green}{\color[gray]{0.75}FO}%
\colorbox{green}{\color[gray]{0.75}FO}%
\colorbox{green}{\color[gray]{0.75}FO}%
\colorbox{green}{\color[gray]{0.75}FO}%
\colorbox{green}{\color[gray]{0.75}FO}%
\colorbox{green}{\color[gray]{0.75}FO}%
\colorbox{green}{\color[gray]{0.75}FO}%
\colorbox{green}{\color[gray]{0.75}FO}%
\colorbox{green}{\color[gray]{0.75}FO}%
\colorbox{green}{\color[gray]{0.75}FO}%
\colorbox{green}{\color[gray]{0.75}FO}%
\colorbox{green}{\color[gray]{0.75}FO}%
\colorbox{green}{\color[gray]{0.75}FO}%
\colorbox{green}{\color[gray]{0.75}FO}%
\colorbox{green}{\color[gray]{0.75}FO}%
\colorbox{green}{\color[gray]{0.75}FO}%
\colorbox{green}{\color[gray]{0.75}FO}%
\colorbox{green}{\color[gray]{0.75}FO}%
\colorbox{green}{\color[gray]{0.75}FO}%
\colorbox{green}{\color[gray]{0.75}FO}%
\colorbox{green}{\color[gray]{0.75}FO}%
\colorbox{green}{\color[gray]{0.75}FO}%
\colorbox{green}{\color[gray]{0.75}FO}%
\colorbox{green}{\color[gray]{0.75}FO}%
\colorbox{green}{\color[gray]{0.75}FO}%
\colorbox{green}{\color[gray]{0.75}FO}%
\colorbox{green}{\color[gray]{0.75}FO}%
\colorbox{green}{\color[gray]{0.75}FO}%
\colorbox{green}{\color[gray]{0.75}FO}%
\colorbox{green}{\color[gray]{0.75}FO}%
\colorbox{green}{\color[gray]{0.75}FO}%
\colorbox{green}{\color[gray]{0.75}FO}%
\colorbox{green}{\color[gray]{0.75}FO}%
\colorbox{green}{\color[gray]{0.75}FO}%
\colorbox{green}{\color[gray]{0.75}FO}%
\colorbox{green}{\color[gray]{0.75}FO}%
\colorbox{green}{\color[gray]{0.75}FO}%
\colorbox{green}{\color[gray]{0.75}FO}%
\colorbox{green}{\color[gray]{0.75}FO}%
\colorbox{green}{\color[gray]{0.75}FO}%
\\
\colorbox{green}{\color[gray]{0.75}FO}%
\colorbox{green}{\color[gray]{0.75}FO}%
\colorbox{green}{\color[gray]{0.75}FO}%
\colorbox{green}{\color[gray]{0.75}FO}%
\colorbox{green}{\color[gray]{0.75}FO}%
\colorbox{green}{\color[gray]{0.75}FO}%
\colorbox{green}{\color[gray]{0.75}FO}%
\colorbox{green}{\color[gray]{0.75}FO}%
\colorbox{green}{\color[gray]{0.75}FO}%
\colorbox{green}{\color[gray]{0.75}FO}%
\colorbox{green}{\color[gray]{0.75}FO}%
\colorbox{green}{\color[gray]{0.75}FO}%
\colorbox{green}{\color[gray]{0.75}FO}%
\colorbox{green}{\color[gray]{0.75}FO}%
\colorbox{green}{\color[gray]{0.75}FO}%
\colorbox{green}{\color[gray]{0.75}FO}%
\colorbox{green}{\color[gray]{0.75}FO}%
\colorbox{green}{\color[gray]{0.75}FO}%
\colorbox{green}{\color[gray]{0.75}FO}%
\colorbox{green}{\color[gray]{0.75}FO}%
\colorbox{green}{\color[gray]{0.75}FO}%
\colorbox{green}{\color[gray]{0.75}FO}%
\colorbox{green}{\color[gray]{0.75}FO}%
\colorbox{green}{\color[gray]{0.75}FO}%
\colorbox{green}{\color[gray]{0.75}FO}%
\colorbox{green}{\color[gray]{0.75}FO}%
\colorbox{green}{\color[gray]{0.75}FO}%
\colorbox{green}{\color[gray]{0.75}FO}%
\colorbox{green}{\color[gray]{0.75}FO}%
\colorbox{green}{\color[gray]{0.75}FO}%
\colorbox{green}{\color[gray]{0.75}FO}%
\colorbox{green}{\color[gray]{0.75}FO}%
\colorbox{green}{\color[gray]{0.75}FO}%
\colorbox{green}{\color[gray]{0.75}FO}%
\colorbox{green}{\color[gray]{0.75}FO}%
\colorbox{green}{\color[gray]{0.75}FO}%
\colorbox{green}{\color[gray]{0.75}FO}%
\colorbox{green}{\color[gray]{0.75}FO}%
\colorbox{green}{\color[gray]{0.75}FO}%
\colorbox{green}{\color[gray]{0.75}FO}%
\colorbox{green}{\color[gray]{0.75}FO}%
\colorbox{green}{\color[gray]{0.75}FO}%
\colorbox{green}{\color[gray]{0.75}FO}%
\colorbox{green}{\color[gray]{0.75}FO}%
\colorbox{green}{\color[gray]{0.75}FO}%
\colorbox{green}{\color[gray]{0.75}FO}%
\colorbox{green}{\color[gray]{0.75}FO}%
\colorbox{green}{\color[gray]{0.75}FO}%
\colorbox{green}{\color[gray]{0.75}FO}%
\colorbox{green}{\color[gray]{0.75}FO}%
\colorbox{green}{\color[gray]{0.75}FO}%
\colorbox{green}{\color[gray]{0.75}FO}%
\colorbox{green}{\color[gray]{0.75}FO}%
\colorbox{green}{\color[gray]{0.75}FO}%
\colorbox{green}{\color[gray]{0.75}FO}%
\colorbox{green}{\color[gray]{0.75}FO}%
\colorbox{green}{\color[gray]{0.75}FO}%
\colorbox{green}{\color[gray]{0.75}FO}%
\colorbox{green}{\color[gray]{0.75}FO}%
\colorbox{green}{\color[gray]{0.75}FO}%
\colorbox{green}{\color[gray]{0.75}FO}%
\colorbox{green}{\color[gray]{0.75}FO}%
\colorbox{green}{\color[gray]{0.75}FO}%
\colorbox{green}{\color[gray]{0.75}FO}%
\colorbox{green}{\color[gray]{0.75}FO}%
\colorbox{green}{\color[gray]{0.75}FO}%
\colorbox{green}{\color[gray]{0.75}FO}%
\colorbox{green}{\color[gray]{0.75}FO}%
\colorbox{green}{\color[gray]{0.75}FO}%
\colorbox{green}{\color[gray]{0.75}FO}%
\colorbox{green}{\color[gray]{0.75}FO}%
\colorbox{green}{\color[gray]{0.75}FO}%
\colorbox{green}{\color[gray]{0.75}FO}%
\colorbox{green}{\color[gray]{0.75}FO}%
\colorbox{green}{\color[gray]{0.75}FO}%
\colorbox{green}{\color[gray]{0.75}FO}%
\colorbox{green}{\color[gray]{0.75}FO}%
\colorbox{green}{\color[gray]{0.75}FO}%
\colorbox{green}{\color[gray]{0.75}FO}%
\colorbox{green}{\color[gray]{0.75}FO}%
\colorbox{green}{\color[gray]{0.75}FO}%
\colorbox{green}{\color[gray]{0.75}FO}%
\colorbox{green}{\color[gray]{0.75}FO}%
\colorbox{green}{\color[gray]{0.75}FO}%
\colorbox{green}{\color[gray]{0.75}FO}%
\colorbox{green}{\color[gray]{0.75}FO}%
\colorbox{green}{\color[gray]{0.75}FO}%
\colorbox{green}{\color[gray]{0.75}FO}%
\colorbox{green}{\color[gray]{0.75}FO}%
\colorbox{green}{\color[gray]{0.75}FO}%
\colorbox{green}{\color[gray]{0.75}FO}%
\colorbox{green}{\color[gray]{0.75}FO}%
\colorbox{green}{\color[gray]{0.75}FO}%
\colorbox{green}{\color[gray]{0.75}FO}%
\colorbox{green}{\color[gray]{0.75}FO}%
\colorbox{green}{\color[gray]{0.75}FO}%
\colorbox{green}{\color[gray]{0.75}FO}%
\colorbox{green}{\color[gray]{0.75}FO}%
\colorbox{green}{\color[gray]{0.75}FO}%
\colorbox{green}{\color[gray]{0.75}FO}%
\\
\colorbox{green}{\color[gray]{0.75}FO}%
\colorbox{green}{\color[gray]{0.75}FO}%
\colorbox{green}{\color[gray]{0.75}FO}%
\colorbox{green}{\color[gray]{0.75}FO}%
\colorbox{green}{\color[gray]{0.75}FO}%
\colorbox{green}{\color[gray]{0.75}FO}%
\colorbox{green}{\color[gray]{0.75}FO}%
\colorbox{green}{\color[gray]{0.75}FO}%
\colorbox{green}{\color[gray]{0.75}FO}%
\colorbox{green}{\color[gray]{0.75}FO}%
\colorbox{green}{\color[gray]{0.75}FO}%
\colorbox{green}{\color[gray]{0.75}FO}%
\colorbox{green}{\color[gray]{0.75}FO}%
\colorbox{green}{\color[gray]{0.75}FO}%
\colorbox{green}{\color[gray]{0.75}FO}%
\colorbox{green}{\color[gray]{0.75}FO}%
\colorbox{green}{\color[gray]{0.75}FO}%
\colorbox{green}{\color[gray]{0.75}FO}%
\colorbox{green}{\color[gray]{0.75}FO}%
\colorbox{green}{\color[gray]{0.75}FO}%
\colorbox{green}{\color[gray]{0.75}FO}%
\colorbox{green}{\color[gray]{0.75}FO}%
\colorbox{green}{\color[gray]{0.75}FO}%
\colorbox{green}{\color[gray]{0.75}FO}%
\colorbox{green}{\color[gray]{0.75}FO}%
\colorbox{green}{\color[gray]{0.75}FO}%
\colorbox{green}{\color[gray]{0.75}FO}%
\colorbox{green}{\color[gray]{0.75}FO}%
\colorbox{green}{\color[gray]{0.75}FO}%
\colorbox{green}{\color[gray]{0.75}FO}%
\colorbox{green}{\color[gray]{0.75}FO}%
\colorbox{green}{\color[gray]{0.75}FO}%
\colorbox{green}{\color[gray]{0.75}FO}%
\colorbox{green}{\color[gray]{0.75}FO}%
\colorbox{green}{\color[gray]{0.75}FO}%
\colorbox{green}{\color[gray]{0.75}FO}%
\colorbox{green}{\color[gray]{0.75}FO}%
\colorbox{green}{\color[gray]{0.75}FO}%
\colorbox{green}{\color[gray]{0.75}FO}%
\colorbox{green}{\color[gray]{0.75}FO}%
\colorbox{green}{\color[gray]{0.75}FO}%
\colorbox{green}{\color[gray]{0.75}FO}%
\colorbox{green}{\color[gray]{0.75}FO}%
\colorbox{green}{\color[gray]{0.75}FO}%
\colorbox{green}{\color[gray]{0.75}FO}%
\colorbox{green}{\color[gray]{0.75}FO}%
\colorbox{green}{\color[gray]{0.75}FO}%
\colorbox{green}{\color[gray]{0.75}FO}%
\colorbox{green}{\color[gray]{0.75}FO}%
\colorbox{green}{\color[gray]{0.75}FO}%
\colorbox{green}{\color[gray]{0.75}FO}%
\colorbox{green}{\color[gray]{0.75}FO}%
\colorbox{green}{\color[gray]{0.75}FO}%
\colorbox{green}{\color[gray]{0.75}FO}%
\colorbox{green}{\color[gray]{0.75}FO}%
\colorbox{green}{\color[gray]{0.75}FO}%
\colorbox{green}{\color[gray]{0.75}FO}%
\colorbox{green}{\color[gray]{0.75}FO}%
\colorbox{green}{\color[gray]{0.75}FO}%
\colorbox{green}{\color[gray]{0.75}FO}%
\colorbox{green}{\color[gray]{0.75}FO}%
\colorbox{green}{\color[gray]{0.75}FO}%
\colorbox{green}{\color[gray]{0.75}FO}%
\colorbox{green}{\color[gray]{0.75}FO}%
\colorbox{green}{\color[gray]{0.75}FO}%
\colorbox{green}{\color[gray]{0.75}FO}%
\colorbox{green}{\color[gray]{0.75}FO}%
\colorbox{green}{\color[gray]{0.75}FO}%
\colorbox{green}{\color[gray]{0.75}FO}%
\colorbox{green}{\color[gray]{0.75}FO}%
\colorbox{green}{\color[gray]{0.75}FO}%
\colorbox{green}{\color[gray]{0.75}FO}%
\colorbox{green}{\color[gray]{0.75}FO}%
\colorbox{green}{\color[gray]{0.75}FO}%
\colorbox{green}{\color[gray]{0.75}FO}%
\colorbox{green}{\color[gray]{0.75}FO}%
\colorbox{green}{\color[gray]{0.75}FO}%
\colorbox{green}{\color[gray]{0.75}FO}%
\colorbox{green}{\color[gray]{0.75}FO}%
\colorbox{green}{\color[gray]{0.75}FO}%
\colorbox{green}{\color[gray]{0.75}FO}%
\colorbox{green}{\color[gray]{0.75}FO}%
\colorbox{green}{\color[gray]{0.75}FO}%
\colorbox{green}{\color[gray]{0.75}FO}%
\colorbox{green}{\color[gray]{0.75}FO}%
\colorbox{green}{\color[gray]{0.75}FO}%
\colorbox{green}{\color[gray]{0.75}FO}%
\colorbox{green}{\color[gray]{0.75}FO}%
\colorbox{green}{\color[gray]{0.75}FO}%
\colorbox{green}{\color[gray]{0.75}FO}%
\colorbox{green}{\color[gray]{0.75}FO}%
\colorbox{green}{\color[gray]{0.75}FO}%
\colorbox{green}{\color[gray]{0.75}FO}%
\colorbox{green}{\color[gray]{0.75}FO}%
\colorbox{green}{\color[gray]{0.75}FO}%
\colorbox{green}{\color[gray]{0.75}FO}%
\colorbox{green}{\color[gray]{0.75}FO}%
\colorbox{green}{\color[gray]{0.75}FO}%
\colorbox{green}{\color[gray]{0.75}FO}%
\colorbox{green}{\color[gray]{0.75}FO}%
\\
\colorbox{green}{\color[gray]{0.75}FO}%
\colorbox{green}{\color[gray]{0.75}FO}%
\colorbox{green}{\color[gray]{0.75}FO}%
\colorbox{green}{\color[gray]{0.75}FO}%
\colorbox{green}{\color[gray]{0.75}FO}%
\colorbox{green}{\color[gray]{0.75}FO}%
\colorbox{green}{\color[gray]{0.75}FO}%
\colorbox{green}{\color[gray]{0.75}FO}%
\colorbox{green}{\color[gray]{0.75}FO}%
\colorbox{green}{\color[gray]{0.75}FO}%
\colorbox{green}{\color[gray]{0.75}FO}%
\colorbox{green}{\color[gray]{0.75}FO}%
\colorbox{green}{\color[gray]{0.75}FO}%
\colorbox{green}{\color[gray]{0.75}FO}%
\colorbox{green}{\color[gray]{0.75}FO}%
\colorbox{green}{\color[gray]{0.75}FO}%
\colorbox{green}{\color[gray]{0.75}FO}%
\colorbox{green}{\color[gray]{0.75}FO}%
\colorbox{green}{\color[gray]{0.75}FO}%
\colorbox{green}{\color[gray]{0.75}FO}%
\colorbox{green}{\color[gray]{0.75}FO}%
\colorbox{green}{\color[gray]{0.75}FO}%
\colorbox{green}{\color[gray]{0.75}FO}%
\colorbox{green}{\color[gray]{0.75}FO}%
\colorbox{green}{\color[gray]{0.75}FO}%
\colorbox{green}{\color[gray]{0.75}FO}%
\colorbox{green}{\color[gray]{0.75}FO}%
\colorbox{green}{\color[gray]{0.75}FO}%
\colorbox{green}{\color[gray]{0.75}FO}%
\colorbox{green}{\color[gray]{0.75}FO}%
\colorbox{green}{\color[gray]{0.75}FO}%
\colorbox{green}{\color[gray]{0.75}FO}%
\colorbox{green}{\color[gray]{0.75}FO}%
\colorbox{green}{\color[gray]{0.75}FO}%
\colorbox{green}{\color[gray]{0.75}FO}%
\colorbox{green}{\color[gray]{0.75}FO}%
\colorbox{green}{\color[gray]{0.75}FO}%
\colorbox{green}{\color[gray]{0.75}FO}%
\colorbox{green}{\color[gray]{0.75}FO}%
\colorbox{green}{\color[gray]{0.75}FO}%
\colorbox{green}{\color[gray]{0.75}FO}%
\colorbox{green}{\color[gray]{0.75}FO}%
\colorbox{green}{\color[gray]{0.75}FO}%
\colorbox{green}{\color[gray]{0.75}FO}%
\colorbox{green}{\color[gray]{0.75}FO}%
\colorbox{green}{\color[gray]{0.75}FO}%
\colorbox{green}{\color[gray]{0.75}FO}%
\colorbox{green}{\color[gray]{0.75}FO}%
\colorbox{green}{\color[gray]{0.75}FO}%
\colorbox{green}{\color[gray]{0.75}FO}%
\colorbox{green}{\color[gray]{0.75}FO}%
\colorbox{green}{\color[gray]{0.75}FO}%
\colorbox{green}{\color[gray]{0.75}FO}%
\colorbox{green}{\color[gray]{0.75}FO}%
\colorbox{green}{\color[gray]{0.75}FO}%
\colorbox{green}{\color[gray]{0.75}FO}%
\colorbox{green}{\color[gray]{0.75}FO}%
\colorbox{green}{\color[gray]{0.75}FO}%
\colorbox{green}{\color[gray]{0.75}FO}%
\colorbox{green}{\color[gray]{0.75}FO}%
\colorbox{green}{\color[gray]{0.75}FO}%
\colorbox{green}{\color[gray]{0.75}FO}%
\colorbox{green}{\color[gray]{0.75}FO}%
\colorbox{green}{\color[gray]{0.75}FO}%
\colorbox{green}{\color[gray]{0.75}FO}%
\colorbox{green}{\color[gray]{0.75}FO}%
\colorbox{green}{\color[gray]{0.75}FO}%
\colorbox{green}{\color[gray]{0.75}FO}%
\colorbox{green}{\color[gray]{0.75}FO}%
\colorbox{green}{\color[gray]{0.75}FO}%
\colorbox{green}{\color[gray]{0.75}FO}%
\colorbox{green}{\color[gray]{0.75}FO}%
\colorbox{green}{\color[gray]{0.75}FO}%
\colorbox{green}{\color[gray]{0.75}FO}%
\colorbox{green}{\color[gray]{0.75}FO}%
\colorbox{green}{\color[gray]{0.75}FO}%
\colorbox{green}{\color[gray]{0.75}FO}%
\colorbox{green}{\color[gray]{0.75}FO}%
\colorbox{green}{\color[gray]{0.75}FO}%
\colorbox{green}{\color[gray]{0.75}FO}%
\colorbox{green}{\color[gray]{0.75}FO}%
\colorbox{green}{\color[gray]{0.75}FO}%
\colorbox{green}{\color[gray]{0.75}FO}%
\colorbox{green}{\color[gray]{0.75}FO}%
\colorbox{green}{\color[gray]{0.75}FO}%
\colorbox{green}{\color[gray]{0.75}FO}%
\colorbox{green}{\color[gray]{0.75}FO}%
\colorbox{green}{\color[gray]{0.75}FO}%
\colorbox{green}{\color[gray]{0.75}FO}%
\colorbox{green}{\color[gray]{0.75}FO}%
\colorbox{green}{\color[gray]{0.75}FO}%
\colorbox{green}{\color[gray]{0.75}FO}%
\colorbox{green}{\color[gray]{0.75}FO}%
\colorbox{green}{\color[gray]{0.75}FO}%
\colorbox{green}{\color[gray]{0.75}FO}%
\colorbox{green}{\color[gray]{0.75}FO}%
\colorbox{green}{\color[gray]{0.75}FO}%
\colorbox{green}{\color[gray]{0.75}FO}%
\colorbox{green}{\color[gray]{0.75}FO}%
\colorbox{green}{\color[gray]{0.75}FO}%
\\
\colorbox{green}{\color[gray]{0.75}FO}%
\colorbox{green}{\color[gray]{0.75}FO}%
\colorbox{green}{\color[gray]{0.75}FO}%
\colorbox{green}{\color[gray]{0.75}FO}%
\colorbox{green}{\color[gray]{0.75}FO}%
\colorbox{green}{\color[gray]{0.75}FO}%
\colorbox{green}{\color[gray]{0.75}FO}%
\colorbox{green}{\color[gray]{0.75}FO}%
\colorbox{green}{\color[gray]{0.75}FO}%
\colorbox{green}{\color[gray]{0.75}FO}%
\colorbox{green}{\color[gray]{0.75}FO}%
\colorbox{green}{\color[gray]{0.75}FO}%
\colorbox{green}{\color[gray]{0.75}FO}%
\colorbox{green}{\color[gray]{0.75}FO}%
\colorbox{green}{\color[gray]{0.75}FO}%
\colorbox{green}{\color[gray]{0.75}FO}%
\colorbox{green}{\color[gray]{0.75}FO}%
\colorbox{green}{\color[gray]{0.75}FO}%
\colorbox{green}{\color[gray]{0.75}FO}%
\colorbox{green}{\color[gray]{0.75}FO}%
\colorbox{green}{\color[gray]{0.75}FO}%
\colorbox{green}{\color[gray]{0.75}FO}%
\colorbox{green}{\color[gray]{0.75}FO}%
\colorbox{green}{\color[gray]{0.75}FO}%
\colorbox{green}{\color[gray]{0.75}FO}%
\colorbox{green}{\color[gray]{0.75}FO}%
\colorbox{green}{\color[gray]{0.75}FO}%
\colorbox{green}{\color[gray]{0.75}FO}%
\colorbox{green}{\color[gray]{0.75}FO}%
\colorbox{green}{\color[gray]{0.75}FO}%
\colorbox{green}{\color[gray]{0.75}FO}%
\colorbox{green}{\color[gray]{0.75}FO}%
\colorbox{green}{\color[gray]{0.75}FO}%
\colorbox{green}{\color[gray]{0.75}FO}%
\colorbox{green}{\color[gray]{0.75}FO}%
\colorbox{green}{\color[gray]{0.75}FO}%
\colorbox{green}{\color[gray]{0.75}FO}%
\colorbox{green}{\color[gray]{0.75}FO}%
\colorbox{green}{\color[gray]{0.75}FO}%
\colorbox{green}{\color[gray]{0.75}FO}%
\colorbox{green}{\color[gray]{0.75}FO}%
\colorbox{green}{\color[gray]{0.75}FO}%
\colorbox{green}{\color[gray]{0.75}FO}%
\colorbox{green}{\color[gray]{0.75}FO}%
\colorbox{green}{\color[gray]{0.75}FO}%
\colorbox{green}{\color[gray]{0.75}FO}%
\colorbox{green}{\color[gray]{0.75}FO}%
\colorbox{green}{\color[gray]{0.75}FO}%
\colorbox{green}{\color[gray]{0.75}FO}%
\colorbox{green}{\color[gray]{0.75}FO}%
\colorbox{green}{\color[gray]{0.75}FO}%
\colorbox{green}{\color[gray]{0.75}FO}%
\colorbox{green}{\color[gray]{0.75}FO}%
\colorbox{green}{\color[gray]{0.75}FO}%
\colorbox{green}{\color[gray]{0.75}FO}%
\colorbox{green}{\color[gray]{0.75}FO}%
\colorbox{green}{\color[gray]{0.75}FO}%
\colorbox{green}{\color[gray]{0.75}FO}%
\colorbox{green}{\color[gray]{0.75}FO}%
\colorbox{green}{\color[gray]{0.75}FO}%
\colorbox{green}{\color[gray]{0.75}FO}%
\colorbox{green}{\color[gray]{0.75}FO}%
\colorbox{green}{\color[gray]{0.75}FO}%
\colorbox{green}{\color[gray]{0.75}FO}%
\colorbox{green}{\color[gray]{0.75}FO}%
\colorbox{green}{\color[gray]{0.75}FO}%
\colorbox{green}{\color[gray]{0.75}FO}%
\colorbox{green}{\color[gray]{0.75}FO}%
\colorbox{green}{\color[gray]{0.75}FO}%
\colorbox{green}{\color[gray]{0.75}FO}%
\colorbox{green}{\color[gray]{0.75}FO}%
\colorbox{green}{\color[gray]{0.75}FO}%
\colorbox{green}{\color[gray]{0.75}FO}%
\colorbox{green}{\color[gray]{0.75}FO}%
\colorbox{green}{\color[gray]{0.75}FO}%
\colorbox{green}{\color[gray]{0.75}FO}%
\colorbox{green}{\color[gray]{0.75}FO}%
\colorbox{green}{\color[gray]{0.75}FO}%
\colorbox{green}{\color[gray]{0.75}FO}%
\colorbox{green}{\color[gray]{0.75}FO}%
\colorbox{green}{\color[gray]{0.75}FO}%
\colorbox{green}{\color[gray]{0.75}FO}%
\colorbox{green}{\color[gray]{0.75}FO}%
\colorbox{green}{\color[gray]{0.75}FO}%
\colorbox{green}{\color[gray]{0.75}FO}%
\colorbox{green}{\color[gray]{0.75}FO}%
\colorbox{green}{\color[gray]{0.75}FO}%
\colorbox{green}{\color[gray]{0.75}FO}%
\colorbox{green}{\color[gray]{0.75}FO}%
\colorbox{green}{\color[gray]{0.75}FO}%
\colorbox{green}{\color[gray]{0.75}FO}%
\colorbox{green}{\color[gray]{0.75}FO}%
\colorbox{green}{\color[gray]{0.75}FO}%
\colorbox{green}{\color[gray]{0.75}FO}%
\colorbox{green}{\color[gray]{0.75}FO}%
\colorbox{green}{\color[gray]{0.75}FO}%
\colorbox{green}{\color[gray]{0.75}FO}%
\colorbox{green}{\color[gray]{0.75}FO}%
\colorbox{green}{\color[gray]{0.75}FO}%
\colorbox{green}{\color[gray]{0.75}FO}%
\\
\colorbox{green}{\color[gray]{0.75}FO}%
\colorbox{green}{\color[gray]{0.75}FO}%
\colorbox{green}{\color[gray]{0.75}FO}%
\colorbox{green}{\color[gray]{0.75}FO}%
\colorbox{green}{\color[gray]{0.75}FO}%
\colorbox{green}{\color[gray]{0.75}FO}%
\colorbox{green}{\color[gray]{0.75}FO}%
\colorbox{green}{\color[gray]{0.75}FO}%
\colorbox{green}{\color[gray]{0.75}FO}%
\colorbox{green}{\color[gray]{0.75}FO}%
\colorbox{green}{\color[gray]{0.75}FO}%
\colorbox{green}{\color[gray]{0.75}FO}%
\colorbox{green}{\color[gray]{0.75}FO}%
\colorbox{green}{\color[gray]{0.75}FO}%
\colorbox{green}{\color[gray]{0.75}FO}%
\colorbox{green}{\color[gray]{0.75}FO}%
\colorbox{green}{\color[gray]{0.75}FO}%
\colorbox{green}{\color[gray]{0.75}FO}%
\colorbox{green}{\color[gray]{0.75}FO}%
\colorbox{green}{\color[gray]{0.75}FO}%
\colorbox{green}{\color[gray]{0.75}FO}%
\colorbox{green}{\color[gray]{0.75}FO}%
\colorbox{green}{\color[gray]{0.75}FO}%
\colorbox{green}{\color[gray]{0.75}FO}%
\colorbox{green}{\color[gray]{0.75}FO}%
\colorbox{green}{\color[gray]{0.75}FO}%
\colorbox{green}{\color[gray]{0.75}FO}%
\colorbox{green}{\color[gray]{0.75}FO}%
\colorbox{green}{\color[gray]{0.75}FO}%
\colorbox{green}{\color[gray]{0.75}FO}%
\colorbox{green}{\color[gray]{0.75}FO}%
\colorbox{green}{\color[gray]{0.75}FO}%
\colorbox{green}{\color[gray]{0.75}FO}%
\colorbox{green}{\color[gray]{0.75}FO}%
\colorbox{green}{\color[gray]{0.75}FO}%
\colorbox{green}{\color[gray]{0.75}FO}%
\colorbox{green}{\color[gray]{0.75}FO}%
\colorbox{green}{\color[gray]{0.75}FO}%
\colorbox{green}{\color[gray]{0.75}FO}%
\colorbox{green}{\color[gray]{0.75}FO}%
\colorbox{green}{\color[gray]{0.75}FO}%
\colorbox{green}{\color[gray]{0.75}FO}%
\colorbox{green}{\color[gray]{0.75}FO}%
\colorbox{green}{\color[gray]{0.75}FO}%
\colorbox{green}{\color[gray]{0.75}FO}%
\colorbox{green}{\color[gray]{0.75}FO}%
\colorbox{green}{\color[gray]{0.75}FO}%
\colorbox{green}{\color[gray]{0.75}FO}%
\colorbox{green}{\color[gray]{0.75}FO}%
\colorbox{green}{\color[gray]{0.75}FO}%
\colorbox{green}{\color[gray]{0.75}FO}%
\colorbox{green}{\color[gray]{0.75}FO}%
\colorbox{green}{\color[gray]{0.75}FO}%
\colorbox{green}{\color[gray]{0.75}FO}%
\colorbox{green}{\color[gray]{0.75}FO}%
\colorbox{green}{\color[gray]{0.75}FO}%
\colorbox{green}{\color[gray]{0.75}FO}%
\colorbox{green}{\color[gray]{0.75}FO}%
\colorbox{green}{\color[gray]{0.75}FO}%
\colorbox{green}{\color[gray]{0.75}FO}%
\colorbox{green}{\color[gray]{0.75}FO}%
\colorbox{green}{\color[gray]{0.75}FO}%
\colorbox{green}{\color[gray]{0.75}FO}%
\colorbox{green}{\color[gray]{0.75}FO}%
\colorbox{green}{\color[gray]{0.75}FO}%
\colorbox{green}{\color[gray]{0.75}FO}%
\colorbox{green}{\color[gray]{0.75}FO}%
\colorbox{green}{\color[gray]{0.75}FO}%
\colorbox{green}{\color[gray]{0.75}FO}%
\colorbox{green}{\color[gray]{0.75}FO}%
\colorbox{green}{\color[gray]{0.75}FO}%
\colorbox{green}{\color[gray]{0.75}FO}%
\colorbox{green}{\color[gray]{0.75}FO}%
\colorbox{green}{\color[gray]{0.75}FO}%
\colorbox{green}{\color[gray]{0.75}FO}%
\colorbox{green}{\color[gray]{0.75}FO}%
\colorbox{green}{\color[gray]{0.75}FO}%
\colorbox{green}{\color[gray]{0.75}FO}%
\colorbox{green}{\color[gray]{0.75}FO}%
\colorbox{green}{\color[gray]{0.75}FO}%
\colorbox{green}{\color[gray]{0.75}FO}%
\colorbox{green}{\color[gray]{0.75}FO}%
\colorbox{green}{\color[gray]{0.75}FO}%
\colorbox{green}{\color[gray]{0.75}FO}%
\colorbox{green}{\color[gray]{0.75}FO}%
\colorbox{green}{\color[gray]{0.75}FO}%
\colorbox{green}{\color[gray]{0.75}FO}%
\colorbox{green}{\color[gray]{0.75}FO}%
\colorbox{green}{\color[gray]{0.75}FO}%
\colorbox{green}{\color[gray]{0.75}FO}%
\colorbox{green}{\color[gray]{0.75}FO}%
\colorbox{green}{\color[gray]{0.75}FO}%
\colorbox{green}{\color[gray]{0.75}FO}%
\colorbox{green}{\color[gray]{0.75}FO}%
\colorbox{green}{\color[gray]{0.75}FO}%
\colorbox{green}{\color[gray]{0.75}FO}%
\colorbox{green}{\color[gray]{0.75}FO}%
\colorbox{green}{\color[gray]{0.75}FO}%
\colorbox{green}{\color[gray]{0.75}FO}%
\colorbox{green}{\color[gray]{0.75}FO}%
\\
\colorbox{green}{\color[gray]{0.75}FO}%
\colorbox{green}{\color[gray]{0.75}FO}%
\colorbox{green}{\color[gray]{0.75}FO}%
\colorbox{green}{\color[gray]{0.75}FO}%
\colorbox{green}{\color[gray]{0.75}FO}%
\colorbox{green}{\color[gray]{0.75}FO}%
\colorbox{green}{\color[gray]{0.75}FO}%
\colorbox{green}{\color[gray]{0.75}FO}%
\colorbox{green}{\color[gray]{0.75}FO}%
\colorbox{green}{\color[gray]{0.75}FO}%
\colorbox{green}{\color[gray]{0.75}FO}%
\colorbox{green}{\color[gray]{0.75}FO}%
\colorbox{green}{\color[gray]{0.75}FO}%
\colorbox{green}{\color[gray]{0.75}FO}%
\colorbox{green}{\color[gray]{0.75}FO}%
\colorbox{green}{\color[gray]{0.75}FO}%
\colorbox{green}{\color[gray]{0.75}FO}%
\colorbox{green}{\color[gray]{0.75}FO}%
\colorbox{green}{\color[gray]{0.75}FO}%
\colorbox{green}{\color[gray]{0.75}FO}%
\colorbox{green}{\color[gray]{0.75}FO}%
\colorbox{green}{\color[gray]{0.75}FO}%
\colorbox{green}{\color[gray]{0.75}FO}%
\colorbox{green}{\color[gray]{0.75}FO}%
\colorbox{green}{\color[gray]{0.75}FO}%
\colorbox{green}{\color[gray]{0.75}FO}%
\colorbox{green}{\color[gray]{0.75}FO}%
\colorbox{green}{\color[gray]{0.75}FO}%
\colorbox{green}{\color[gray]{0.75}FO}%
\colorbox{green}{\color[gray]{0.75}FO}%
\colorbox{green}{\color[gray]{0.75}FO}%
\colorbox{green}{\color[gray]{0.75}FO}%
\colorbox{green}{\color[gray]{0.75}FO}%
\colorbox{green}{\color[gray]{0.75}FO}%
\colorbox{green}{\color[gray]{0.75}FO}%
\colorbox{green}{\color[gray]{0.75}FO}%
\colorbox{green}{\color[gray]{0.75}FO}%
\colorbox{green}{\color[gray]{0.75}FO}%
\colorbox{green}{\color[gray]{0.75}FO}%
\colorbox{green}{\color[gray]{0.75}FO}%
\colorbox{green}{\color[gray]{0.75}FO}%
\colorbox{green}{\color[gray]{0.75}FO}%
\colorbox{green}{\color[gray]{0.75}FO}%
\colorbox{green}{\color[gray]{0.75}FO}%
\colorbox{green}{\color[gray]{0.75}FO}%
\colorbox{green}{\color[gray]{0.75}FO}%
\colorbox{green}{\color[gray]{0.75}FO}%
\colorbox{green}{\color[gray]{0.75}FO}%
\colorbox{green}{\color[gray]{0.75}FO}%
\colorbox{green}{\color[gray]{0.75}FO}%
\colorbox{green}{\color[gray]{0.75}FO}%
\colorbox{green}{\color[gray]{0.75}FO}%
\colorbox{green}{\color[gray]{0.75}FO}%
\colorbox{green}{\color[gray]{0.75}FO}%
\colorbox{green}{\color[gray]{0.75}FO}%
\colorbox{green}{\color[gray]{0.75}FO}%
\colorbox{green}{\color[gray]{0.75}FO}%
\colorbox{green}{\color[gray]{0.75}FO}%
\colorbox{green}{\color[gray]{0.75}FO}%
\colorbox{green}{\color[gray]{0.75}FO}%
\colorbox{green}{\color[gray]{0.75}FO}%
\colorbox{green}{\color[gray]{0.75}FO}%
\colorbox{green}{\color[gray]{0.75}FO}%
\colorbox{green}{\color[gray]{0.75}FO}%
\colorbox{green}{\color[gray]{0.75}FO}%
\colorbox{green}{\color[gray]{0.75}FO}%
\colorbox{green}{\color[gray]{0.75}FO}%
\colorbox{green}{\color[gray]{0.75}FO}%
\colorbox{green}{\color[gray]{0.75}FO}%
\colorbox{green}{\color[gray]{0.75}FO}%
\colorbox{green}{\color[gray]{0.75}FO}%
\colorbox{green}{\color[gray]{0.75}FO}%
\colorbox{green}{\color[gray]{0.75}FO}%
\colorbox{green}{\color[gray]{0.75}FO}%
\colorbox{green}{\color[gray]{0.75}FO}%
\colorbox{green}{\color[gray]{0.75}FO}%
\colorbox{green}{\color[gray]{0.75}FO}%
\colorbox{green}{\color[gray]{0.75}FO}%
\colorbox{green}{\color[gray]{0.75}FO}%
\colorbox{green}{\color[gray]{0.75}FO}%
\colorbox{green}{\color[gray]{0.75}FO}%
\colorbox{green}{\color[gray]{0.75}FO}%
\colorbox{green}{\color[gray]{0.75}FO}%
\colorbox{green}{\color[gray]{0.75}FO}%
\colorbox{green}{\color[gray]{0.75}FO}%
\colorbox{green}{\color[gray]{0.75}FO}%
\colorbox{green}{\color[gray]{0.75}FO}%
\colorbox{green}{\color[gray]{0.75}FO}%
\colorbox{green}{\color[gray]{0.75}FO}%
\colorbox{green}{\color[gray]{0.75}FO}%
\colorbox{green}{\color[gray]{0.75}FO}%
\colorbox{green}{\color[gray]{0.75}FO}%
\colorbox{green}{\color[gray]{0.75}FO}%
\colorbox{green}{\color[gray]{0.75}FO}%
\colorbox{green}{\color[gray]{0.75}FO}%
\colorbox{green}{\color[gray]{0.75}FO}%
\colorbox{green}{\color[gray]{0.75}FO}%
\colorbox{green}{\color[gray]{0.75}FO}%
\colorbox{green}{\color[gray]{0.75}FO}%
\colorbox{green}{\color[gray]{0.75}FO}%
\\
\colorbox{green}{\color[gray]{0.75}FO}%
\colorbox{green}{\color[gray]{0.75}FO}%
\colorbox{green}{\color[gray]{0.75}FO}%
\colorbox{green}{\color[gray]{0.75}FO}%
\colorbox{green}{\color[gray]{0.75}FO}%
\colorbox{green}{\color[gray]{0.75}FO}%
\colorbox{green}{\color[gray]{0.75}FO}%
\colorbox{green}{\color[gray]{0.75}FO}%
\colorbox{green}{\color[gray]{0.75}FO}%
\colorbox{green}{\color[gray]{0.75}FO}%
\colorbox{green}{\color[gray]{0.75}FO}%
\colorbox{green}{\color[gray]{0.75}FO}%
\colorbox{green}{\color[gray]{0.75}FO}%
\colorbox{green}{\color[gray]{0.75}FO}%
\colorbox{green}{\color[gray]{0.75}FO}%
\colorbox{green}{\color[gray]{0.75}FO}%
\colorbox{green}{\color[gray]{0.75}FO}%
\colorbox{green}{\color[gray]{0.75}FO}%
\colorbox{green}{\color[gray]{0.75}FO}%
\colorbox{green}{\color[gray]{0.75}FO}%
\colorbox{green}{\color[gray]{0.75}FO}%
\colorbox{green}{\color[gray]{0.75}FO}%
\colorbox{green}{\color[gray]{0.75}FO}%
\colorbox{green}{\color[gray]{0.75}FO}%
\colorbox{green}{\color[gray]{0.75}FO}%
\colorbox{green}{\color[gray]{0.75}FO}%
\colorbox{green}{\color[gray]{0.75}FO}%
\colorbox{green}{\color[gray]{0.75}FO}%
\colorbox{green}{\color[gray]{0.75}FO}%
\colorbox{green}{\color[gray]{0.75}FO}%
\colorbox{green}{\color[gray]{0.75}FO}%
\colorbox{green}{\color[gray]{0.75}FO}%
\colorbox{green}{\color[gray]{0.75}FO}%
\colorbox{green}{\color[gray]{0.75}FO}%
\colorbox{green}{\color[gray]{0.75}FO}%
\colorbox{green}{\color[gray]{0.75}FO}%
\colorbox{green}{\color[gray]{0.75}FO}%
\colorbox{green}{\color[gray]{0.75}FO}%
\colorbox{green}{\color[gray]{0.75}FO}%
\colorbox{green}{\color[gray]{0.75}FO}%
\colorbox{green}{\color[gray]{0.75}FO}%
\colorbox{green}{\color[gray]{0.75}FO}%
\colorbox{green}{\color[gray]{0.75}FO}%
\colorbox{green}{\color[gray]{0.75}FO}%
\colorbox{green}{\color[gray]{0.75}FO}%
\colorbox{green}{\color[gray]{0.75}FO}%
\colorbox{green}{\color[gray]{0.75}FO}%
\colorbox{green}{\color[gray]{0.75}FO}%
\colorbox{green}{\color[gray]{0.75}FO}%
\colorbox{green}{\color[gray]{0.75}FO}%
\colorbox{green}{\color[gray]{0.75}FO}%
\colorbox{green}{\color[gray]{0.75}FO}%
\colorbox{green}{\color[gray]{0.75}FO}%
\colorbox{green}{\color[gray]{0.75}FO}%
\colorbox{green}{\color[gray]{0.75}FO}%
\colorbox{green}{\color[gray]{0.75}FO}%
\colorbox{green}{\color[gray]{0.75}FO}%
\colorbox{green}{\color[gray]{0.75}FO}%
\colorbox{green}{\color[gray]{0.75}FO}%
\colorbox{green}{\color[gray]{0.75}FO}%
\colorbox{green}{\color[gray]{0.75}FO}%
\colorbox{green}{\color[gray]{0.75}FO}%
\colorbox{green}{\color[gray]{0.75}FO}%
\colorbox{green}{\color[gray]{0.75}FO}%
\colorbox{green}{\color[gray]{0.75}FO}%
\colorbox{green}{\color[gray]{0.75}FO}%
\colorbox{green}{\color[gray]{0.75}FO}%
\colorbox{green}{\color[gray]{0.75}FO}%
\colorbox{green}{\color[gray]{0.75}FO}%
\colorbox{green}{\color[gray]{0.75}FO}%
\colorbox{green}{\color[gray]{0.75}FO}%
\colorbox{green}{\color[gray]{0.75}FO}%
\colorbox{green}{\color[gray]{0.75}FO}%
\colorbox{green}{\color[gray]{0.75}FO}%
\colorbox{green}{\color[gray]{0.75}FO}%
\colorbox{green}{\color[gray]{0.75}FO}%
\colorbox{green}{\color[gray]{0.75}FO}%
\colorbox{green}{\color[gray]{0.75}FO}%
\colorbox{green}{\color[gray]{0.75}FO}%
\colorbox{green}{\color[gray]{0.75}FO}%
\colorbox{green}{\color[gray]{0.75}FO}%
\colorbox{green}{\color[gray]{0.75}FO}%
\colorbox{green}{\color[gray]{0.75}FO}%
\colorbox{green}{\color[gray]{0.75}FO}%
\colorbox{green}{\color[gray]{0.75}FO}%
\colorbox{green}{\color[gray]{0.75}FO}%
\colorbox{green}{\color[gray]{0.75}FO}%
\colorbox{green}{\color[gray]{0.75}FO}%
\colorbox{green}{\color[gray]{0.75}FO}%
\colorbox{green}{\color[gray]{0.75}FO}%
\colorbox{green}{\color[gray]{0.75}FO}%
\colorbox{green}{\color[gray]{0.75}FO}%
\colorbox{green}{\color[gray]{0.75}FO}%
\colorbox{green}{\color[gray]{0.75}FO}%
\colorbox{green}{\color[gray]{0.75}FO}%
\colorbox{green}{\color[gray]{0.75}FO}%
\colorbox{green}{\color[gray]{0.75}FO}%
\colorbox{green}{\color[gray]{0.75}FO}%
\colorbox{green}{\color[gray]{0.75}FO}%
\colorbox{green}{\color[gray]{0.75}FO}%
\\
\colorbox{green}{\color[gray]{0.75}FO}%
\colorbox{green}{\color[gray]{0.75}FO}%
\colorbox{green}{\color[gray]{0.75}FO}%
\colorbox{green}{\color[gray]{0.75}FO}%
\colorbox{green}{\color[gray]{0.75}FO}%
\colorbox{green}{\color[gray]{0.75}FO}%
\colorbox{green}{\color[gray]{0.75}FO}%
\colorbox{green}{\color[gray]{0.75}FO}%
\colorbox{green}{\color[gray]{0.75}FO}%
\colorbox{green}{\color[gray]{0.75}FO}%
\colorbox{green}{\color[gray]{0.75}FO}%
\colorbox{green}{\color[gray]{0.75}FO}%
\colorbox{green}{\color[gray]{0.75}FO}%
\colorbox{green}{\color[gray]{0.75}FO}%
\colorbox{green}{\color[gray]{0.75}FO}%
\colorbox{green}{\color[gray]{0.75}FO}%
\colorbox{green}{\color[gray]{0.75}FO}%
\colorbox{green}{\color[gray]{0.75}FO}%
\colorbox{green}{\color[gray]{0.75}FO}%
\colorbox{green}{\color[gray]{0.75}FO}%
\colorbox{green}{\color[gray]{0.75}FO}%
\colorbox{green}{\color[gray]{0.75}FO}%
\colorbox{green}{\color[gray]{0.75}FO}%
\colorbox{green}{\color[gray]{0.75}FO}%
\colorbox{green}{\color[gray]{0.75}FO}%
\colorbox{green}{\color[gray]{0.75}FO}%
\colorbox{green}{\color[gray]{0.75}FO}%
\colorbox{green}{\color[gray]{0.75}FO}%
\colorbox{green}{\color[gray]{0.75}FO}%
\colorbox{green}{\color[gray]{0.75}FO}%
\colorbox{green}{\color[gray]{0.75}FO}%
\colorbox{green}{\color[gray]{0.75}FO}%
\colorbox{green}{\color[gray]{0.75}FO}%
\colorbox{green}{\color[gray]{0.75}FO}%
\colorbox{green}{\color[gray]{0.75}FO}%
\colorbox{green}{\color[gray]{0.75}FO}%
\colorbox{green}{\color[gray]{0.75}FO}%
\colorbox{green}{\color[gray]{0.75}FO}%
\colorbox{green}{\color[gray]{0.75}FO}%
\colorbox{green}{\color[gray]{0.75}FO}%
\colorbox{green}{\color[gray]{0.75}FO}%
\colorbox{green}{\color[gray]{0.75}FO}%
\colorbox{green}{\color[gray]{0.75}FO}%
\colorbox{green}{\color[gray]{0.75}FO}%
\colorbox{green}{\color[gray]{0.75}FO}%
\colorbox{green}{\color[gray]{0.75}FO}%
\colorbox{green}{\color[gray]{0.75}FO}%
\colorbox{green}{\color[gray]{0.75}FO}%
\colorbox{green}{\color[gray]{0.75}FO}%
\colorbox{green}{\color[gray]{0.75}FO}%
\colorbox{green}{\color[gray]{0.75}FO}%
\colorbox{green}{\color[gray]{0.75}FO}%
\colorbox{green}{\color[gray]{0.75}FO}%
\colorbox{green}{\color[gray]{0.75}FO}%
\colorbox{green}{\color[gray]{0.75}FO}%
\colorbox{green}{\color[gray]{0.75}FO}%
\colorbox{green}{\color[gray]{0.75}FO}%
\colorbox{green}{\color[gray]{0.75}FO}%
\colorbox{green}{\color[gray]{0.75}FO}%
\colorbox{green}{\color[gray]{0.75}FO}%
\colorbox{green}{\color[gray]{0.75}FO}%
\colorbox{green}{\color[gray]{0.75}FO}%
\colorbox{green}{\color[gray]{0.75}FO}%
\colorbox{green}{\color[gray]{0.75}FO}%
\colorbox{green}{\color[gray]{0.75}FO}%
\colorbox{green}{\color[gray]{0.75}FO}%
\colorbox{green}{\color[gray]{0.75}FO}%
\colorbox{green}{\color[gray]{0.75}FO}%
\colorbox{green}{\color[gray]{0.75}FO}%
\colorbox{green}{\color[gray]{0.75}FO}%
\colorbox{green}{\color[gray]{0.75}FO}%
\colorbox{green}{\color[gray]{0.75}FO}%
\colorbox{green}{\color[gray]{0.75}FO}%
\colorbox{green}{\color[gray]{0.75}FO}%
\colorbox{green}{\color[gray]{0.75}FO}%
\colorbox{green}{\color[gray]{0.75}FO}%
\colorbox{green}{\color[gray]{0.75}FO}%
\colorbox{green}{\color[gray]{0.75}FO}%
\colorbox{green}{\color[gray]{0.75}FO}%
\colorbox{green}{\color[gray]{0.75}FO}%
\colorbox{green}{\color[gray]{0.75}FO}%
\colorbox{green}{\color[gray]{0.75}FO}%
\colorbox{green}{\color[gray]{0.75}FO}%
\colorbox{green}{\color[gray]{0.75}FO}%
\colorbox{green}{\color[gray]{0.75}FO}%
\colorbox{green}{\color[gray]{0.75}FO}%
\colorbox{green}{\color[gray]{0.75}FO}%
\colorbox{green}{\color[gray]{0.75}FO}%
\colorbox{green}{\color[gray]{0.75}FO}%
\colorbox{green}{\color[gray]{0.75}FO}%
\colorbox{green}{\color[gray]{0.75}FO}%
\colorbox{green}{\color[gray]{0.75}FO}%
\colorbox{green}{\color[gray]{0.75}FO}%
\colorbox{green}{\color[gray]{0.75}FO}%
\colorbox{green}{\color[gray]{0.75}FO}%
\colorbox{green}{\color[gray]{0.75}FO}%
\colorbox{green}{\color[gray]{0.75}FO}%
\colorbox{green}{\color[gray]{0.75}FO}%
\colorbox{green}{\color[gray]{0.75}FO}%
\colorbox{green}{\color[gray]{0.75}FO}%
\\
\colorbox{green}{\color[gray]{0.75}FO}%
\colorbox{green}{\color[gray]{0.75}FO}%
\colorbox{green}{\color[gray]{0.75}FO}%
\colorbox{green}{\color[gray]{0.75}FO}%
\colorbox{green}{\color[gray]{0.75}FO}%
\colorbox{green}{\color[gray]{0.75}FO}%
\colorbox{green}{\color[gray]{0.75}FO}%
\colorbox{green}{\color[gray]{0.75}FO}%
\colorbox{green}{\color[gray]{0.75}FO}%
\colorbox{green}{\color[gray]{0.75}FO}%
\colorbox{green}{\color[gray]{0.75}FO}%
\colorbox{green}{\color[gray]{0.75}FO}%
\colorbox{green}{\color[gray]{0.75}FO}%
\colorbox{green}{\color[gray]{0.75}FO}%
\colorbox{green}{\color[gray]{0.75}FO}%
\colorbox{green}{\color[gray]{0.75}FO}%
\colorbox{green}{\color[gray]{0.75}FO}%
\colorbox{green}{\color[gray]{0.75}FO}%
\colorbox{green}{\color[gray]{0.75}FO}%
\colorbox{green}{\color[gray]{0.75}FO}%
\colorbox{green}{\color[gray]{0.75}FO}%
\colorbox{green}{\color[gray]{0.75}FO}%
\colorbox{green}{\color[gray]{0.75}FO}%
\colorbox{green}{\color[gray]{0.75}FO}%
\colorbox{green}{\color[gray]{0.75}FO}%
\colorbox{green}{\color[gray]{0.75}FO}%
\colorbox{green}{\color[gray]{0.75}FO}%
\colorbox{green}{\color[gray]{0.75}FO}%
\colorbox{green}{\color[gray]{0.75}FO}%
\colorbox{green}{\color[gray]{0.75}FO}%
\colorbox{green}{\color[gray]{0.75}FO}%
\colorbox{green}{\color[gray]{0.75}FO}%
\colorbox{green}{\color[gray]{0.75}FO}%
\colorbox{green}{\color[gray]{0.75}FO}%
\colorbox{green}{\color[gray]{0.75}FO}%
\colorbox{green}{\color[gray]{0.75}FO}%
\colorbox{green}{\color[gray]{0.75}FO}%
\colorbox{green}{\color[gray]{0.75}FO}%
\colorbox{green}{\color[gray]{0.75}FO}%
\colorbox{green}{\color[gray]{0.75}FO}%
\colorbox{green}{\color[gray]{0.75}FO}%
\colorbox{green}{\color[gray]{0.75}FO}%
\colorbox{green}{\color[gray]{0.75}FO}%
\colorbox{green}{\color[gray]{0.75}FO}%
\colorbox{green}{\color[gray]{0.75}FO}%
\colorbox{green}{\color[gray]{0.75}FO}%
\colorbox{green}{\color[gray]{0.75}FO}%
\colorbox{green}{\color[gray]{0.75}FO}%
\colorbox{green}{\color[gray]{0.75}FO}%
\colorbox{green}{\color[gray]{0.75}FO}%
\colorbox{green}{\color[gray]{0.75}FO}%
\colorbox{green}{\color[gray]{0.75}FO}%
\colorbox{green}{\color[gray]{0.75}FO}%
\colorbox{green}{\color[gray]{0.75}FO}%
\colorbox{green}{\color[gray]{0.75}FO}%
\colorbox{green}{\color[gray]{0.75}FO}%
\colorbox{green}{\color[gray]{0.75}FO}%
\colorbox{green}{\color[gray]{0.75}FO}%
\colorbox{green}{\color[gray]{0.75}FO}%
\colorbox{green}{\color[gray]{0.75}FO}%
\colorbox{green}{\color[gray]{0.75}FO}%
\colorbox{green}{\color[gray]{0.75}FO}%
\colorbox{green}{\color[gray]{0.75}FO}%
\colorbox{green}{\color[gray]{0.75}FO}%
\colorbox{green}{\color[gray]{0.75}FO}%
\colorbox{green}{\color[gray]{0.75}FO}%
\colorbox{green}{\color[gray]{0.75}FO}%
\colorbox{green}{\color[gray]{0.75}FO}%
\colorbox{green}{\color[gray]{0.75}FO}%
\colorbox{green}{\color[gray]{0.75}FO}%
\colorbox{green}{\color[gray]{0.75}FO}%
\colorbox{green}{\color[gray]{0.75}FO}%
\colorbox{green}{\color[gray]{0.75}FO}%
\colorbox{green}{\color[gray]{0.75}FO}%
\colorbox{green}{\color[gray]{0.75}FO}%
\colorbox{green}{\color[gray]{0.75}FO}%
\colorbox{green}{\color[gray]{0.75}FO}%
\colorbox{green}{\color[gray]{0.75}FO}%
\colorbox{green}{\color[gray]{0.75}FO}%
\colorbox{green}{\color[gray]{0.75}FO}%
\colorbox{green}{\color[gray]{0.75}FO}%
\colorbox{green}{\color[gray]{0.75}FO}%
\colorbox{green}{\color[gray]{0.75}FO}%
\colorbox{green}{\color[gray]{0.75}FO}%
\colorbox{green}{\color[gray]{0.75}FO}%
\colorbox{green}{\color[gray]{0.75}FO}%
\colorbox{green}{\color[gray]{0.75}FO}%
\colorbox{green}{\color[gray]{0.75}FO}%
\colorbox{green}{\color[gray]{0.75}FO}%
\colorbox{green}{\color[gray]{0.75}FO}%
\colorbox{green}{\color[gray]{0.75}FO}%
\colorbox{green}{\color[gray]{0.75}FO}%
\colorbox{green}{\color[gray]{0.75}FO}%
\colorbox{green}{\color[gray]{0.75}FO}%
\colorbox{green}{\color[gray]{0.75}FO}%
\colorbox{green}{\color[gray]{0.75}FO}%
\colorbox{green}{\color[gray]{0.75}FO}%
\colorbox{green}{\color[gray]{0.75}FO}%
\colorbox{green}{\color[gray]{0.75}FO}%
\colorbox{green}{\color[gray]{0.75}FO}%
\\
\colorbox{green}{\color[gray]{0.75}FO}%
\colorbox{green}{\color[gray]{0.75}FO}%
\colorbox{green}{\color[gray]{0.75}FO}%
\colorbox{green}{\color[gray]{0.75}FO}%
\colorbox{green}{\color[gray]{0.75}FO}%
\colorbox{green}{\color[gray]{0.75}FO}%
\colorbox{green}{\color[gray]{0.75}FO}%
\colorbox{green}{\color[gray]{0.75}FO}%
\colorbox{green}{\color[gray]{0.75}FO}%
\colorbox{green}{\color[gray]{0.75}FO}%
\colorbox{green}{\color[gray]{0.75}FO}%
\colorbox{green}{\color[gray]{0.75}FO}%
\colorbox{green}{\color[gray]{0.75}FO}%
\colorbox{green}{\color[gray]{0.75}FO}%
\colorbox{green}{\color[gray]{0.75}FO}%
\colorbox{green}{\color[gray]{0.75}FO}%
\colorbox{green}{\color[gray]{0.75}FO}%
\colorbox{green}{\color[gray]{0.75}FO}%
\colorbox{green}{\color[gray]{0.75}FO}%
\colorbox{green}{\color[gray]{0.75}FO}%
\colorbox{green}{\color[gray]{0.75}FO}%
\colorbox{green}{\color[gray]{0.75}FO}%
\colorbox{green}{\color[gray]{0.75}FO}%
\colorbox{green}{\color[gray]{0.75}FO}%
\colorbox{green}{\color[gray]{0.75}FO}%
\colorbox{green}{\color[gray]{0.75}FO}%
\colorbox{green}{\color[gray]{0.75}FO}%
\colorbox{green}{\color[gray]{0.75}FO}%
\colorbox{green}{\color[gray]{0.75}FO}%
\colorbox{green}{\color[gray]{0.75}FO}%
\colorbox{green}{\color[gray]{0.75}FO}%
\colorbox{green}{\color[gray]{0.75}FO}%
\colorbox{green}{\color[gray]{0.75}FO}%
\colorbox{green}{\color[gray]{0.75}FO}%
\colorbox{green}{\color[gray]{0.75}FO}%
\colorbox{green}{\color[gray]{0.75}FO}%
\colorbox{green}{\color[gray]{0.75}FO}%
\colorbox{green}{\color[gray]{0.75}FO}%
\colorbox{green}{\color[gray]{0.75}FO}%
\colorbox{green}{\color[gray]{0.75}FO}%
\colorbox{green}{\color[gray]{0.75}FO}%
\colorbox{green}{\color[gray]{0.75}FO}%
\colorbox{green}{\color[gray]{0.75}FO}%
\colorbox{green}{\color[gray]{0.75}FO}%
\colorbox{green}{\color[gray]{0.75}FO}%
\colorbox{green}{\color[gray]{0.75}FO}%
\colorbox{green}{\color[gray]{0.75}FO}%
\colorbox{green}{\color[gray]{0.75}FO}%
\colorbox{green}{\color[gray]{0.75}FO}%
\colorbox{green}{\color[gray]{0.75}FO}%
\colorbox{green}{\color[gray]{0.75}FO}%
\colorbox{green}{\color[gray]{0.75}FO}%
\colorbox{green}{\color[gray]{0.75}FO}%
\colorbox{green}{\color[gray]{0.75}FO}%
\colorbox{green}{\color[gray]{0.75}FO}%
\colorbox{green}{\color[gray]{0.75}FO}%
\colorbox{green}{\color[gray]{0.75}FO}%
\colorbox{green}{\color[gray]{0.75}FO}%
\colorbox{green}{\color[gray]{0.75}FO}%
\colorbox{green}{\color[gray]{0.75}FO}%
\colorbox{green}{\color[gray]{0.75}FO}%
\colorbox{green}{\color[gray]{0.75}FO}%
\colorbox{green}{\color[gray]{0.75}FO}%
\colorbox{green}{\color[gray]{0.75}FO}%
\colorbox{green}{\color[gray]{0.75}FO}%
\colorbox{green}{\color[gray]{0.75}FO}%
\colorbox{green}{\color[gray]{0.75}FO}%
\colorbox{green}{\color[gray]{0.75}FO}%
\colorbox{green}{\color[gray]{0.75}FO}%
\colorbox{green}{\color[gray]{0.75}FO}%
\colorbox{green}{\color[gray]{0.75}FO}%
\colorbox{green}{\color[gray]{0.75}FO}%
\colorbox{green}{\color[gray]{0.75}FO}%
\colorbox{green}{\color[gray]{0.75}FO}%
\colorbox{green}{\color[gray]{0.75}FO}%
\colorbox{green}{\color[gray]{0.75}FO}%
\colorbox{green}{\color[gray]{0.75}FO}%
\colorbox{green}{\color[gray]{0.75}FO}%
\colorbox{green}{\color[gray]{0.75}FO}%
\colorbox{green}{\color[gray]{0.75}FO}%
\colorbox{green}{\color[gray]{0.75}FO}%
\colorbox{green}{\color[gray]{0.75}FO}%
\colorbox{green}{\color[gray]{0.75}FO}%
\colorbox{green}{\color[gray]{0.75}FO}%
\colorbox{green}{\color[gray]{0.75}FO}%
\colorbox{green}{\color[gray]{0.75}FO}%
\colorbox{green}{\color[gray]{0.75}FO}%
\colorbox{green}{\color[gray]{0.75}FO}%
\colorbox{green}{\color[gray]{0.75}FO}%
\colorbox{green}{\color[gray]{0.75}FO}%
\colorbox{green}{\color[gray]{0.75}FO}%
\colorbox{green}{\color[gray]{0.75}FO}%
\colorbox{green}{\color[gray]{0.75}FO}%
\colorbox{green}{\color[gray]{0.75}FO}%
\colorbox{green}{\color[gray]{0.75}FO}%
\colorbox{green}{\color[gray]{0.75}FO}%
\colorbox{green}{\color[gray]{0.75}FO}%
\colorbox{green}{\color[gray]{0.75}FO}%
\colorbox{green}{\color[gray]{0.75}FO}%
\colorbox{green}{\color[gray]{0.75}FO}%
\\
\colorbox{green}{\color[gray]{0.75}FO}%
\colorbox{green}{\color[gray]{0.75}FO}%
\colorbox{green}{\color[gray]{0.75}FO}%
\colorbox{green}{\color[gray]{0.75}FO}%
\colorbox{green}{\color[gray]{0.75}FO}%
\colorbox{green}{\color[gray]{0.75}FO}%
\colorbox{green}{\color[gray]{0.75}FO}%
\colorbox{green}{\color[gray]{0.75}FO}%
\colorbox{green}{\color[gray]{0.75}FO}%
\colorbox{green}{\color[gray]{0.75}FO}%
\colorbox{green}{\color[gray]{0.75}FO}%
\colorbox{green}{\color[gray]{0.75}FO}%
\colorbox{green}{\color[gray]{0.75}FO}%
\colorbox{green}{\color[gray]{0.75}FO}%
\colorbox{green}{\color[gray]{0.75}FO}%
\colorbox{green}{\color[gray]{0.75}FO}%
\colorbox{green}{\color[gray]{0.75}FO}%
\colorbox{green}{\color[gray]{0.75}FO}%
\colorbox{green}{\color[gray]{0.75}FO}%
\colorbox{green}{\color[gray]{0.75}FO}%
\colorbox{green}{\color[gray]{0.75}FO}%
\colorbox{green}{\color[gray]{0.75}FO}%
\colorbox{green}{\color[gray]{0.75}FO}%
\colorbox{green}{\color[gray]{0.75}FO}%
\colorbox{green}{\color[gray]{0.75}FO}%
\colorbox{green}{\color[gray]{0.75}FO}%
\colorbox{green}{\color[gray]{0.75}FO}%
\colorbox{green}{\color[gray]{0.75}FO}%
\colorbox{green}{\color[gray]{0.75}FO}%
\colorbox{green}{\color[gray]{0.75}FO}%
\colorbox{green}{\color[gray]{0.75}FO}%
\colorbox{green}{\color[gray]{0.75}FO}%
\colorbox{green}{\color[gray]{0.75}FO}%
\colorbox{green}{\color[gray]{0.75}FO}%
\colorbox{green}{\color[gray]{0.75}FO}%
\colorbox{green}{\color[gray]{0.75}FO}%
\colorbox{green}{\color[gray]{0.75}FO}%
\colorbox{green}{\color[gray]{0.75}FO}%
\colorbox{green}{\color[gray]{0.75}FO}%
\colorbox{green}{\color[gray]{0.75}FO}%
\colorbox{green}{\color[gray]{0.75}FO}%
\colorbox{green}{\color[gray]{0.75}FO}%
\colorbox{green}{\color[gray]{0.75}FO}%
\colorbox{green}{\color[gray]{0.75}FO}%
\colorbox{green}{\color[gray]{0.75}FO}%
\colorbox{green}{\color[gray]{0.75}FO}%
\colorbox{green}{\color[gray]{0.75}FO}%
\colorbox{green}{\color[gray]{0.75}FO}%
\colorbox{green}{\color[gray]{0.75}FO}%
\colorbox{green}{\color[gray]{0.75}FO}%
\colorbox{green}{\color[gray]{0.75}FO}%
\colorbox{green}{\color[gray]{0.75}FO}%
\colorbox{green}{\color[gray]{0.75}FO}%
\colorbox{green}{\color[gray]{0.75}FO}%
\colorbox{green}{\color[gray]{0.75}FO}%
\colorbox{green}{\color[gray]{0.75}FO}%
\colorbox{green}{\color[gray]{0.75}FO}%
\colorbox{green}{\color[gray]{0.75}FO}%
\colorbox{green}{\color[gray]{0.75}FO}%
\colorbox{green}{\color[gray]{0.75}FO}%
\colorbox{green}{\color[gray]{0.75}FO}%
\colorbox{green}{\color[gray]{0.75}FO}%
\colorbox{green}{\color[gray]{0.75}FO}%
\colorbox{green}{\color[gray]{0.75}FO}%
\colorbox{green}{\color[gray]{0.75}FO}%
\colorbox{green}{\color[gray]{0.75}FO}%
\colorbox{green}{\color[gray]{0.75}FO}%
\colorbox{green}{\color[gray]{0.75}FO}%
\colorbox{green}{\color[gray]{0.75}FO}%
\colorbox{green}{\color[gray]{0.75}FO}%
\colorbox{green}{\color[gray]{0.75}FO}%
\colorbox{green}{\color[gray]{0.75}FO}%
\colorbox{green}{\color[gray]{0.75}FO}%
\colorbox{green}{\color[gray]{0.75}FO}%
\colorbox{green}{\color[gray]{0.75}FO}%
\colorbox{green}{\color[gray]{0.75}FO}%
\colorbox{green}{\color[gray]{0.75}FO}%
\colorbox{green}{\color[gray]{0.75}FO}%
\colorbox{green}{\color[gray]{0.75}FO}%
\colorbox{green}{\color[gray]{0.75}FO}%
\colorbox{green}{\color[gray]{0.75}FO}%
\colorbox{green}{\color[gray]{0.75}FO}%
\colorbox{green}{\color[gray]{0.75}FO}%
\colorbox{green}{\color[gray]{0.75}FO}%
\colorbox{green}{\color[gray]{0.75}FO}%
\colorbox{green}{\color[gray]{0.75}FO}%
\colorbox{green}{\color[gray]{0.75}FO}%
\colorbox{green}{\color[gray]{0.75}FO}%
\colorbox{green}{\color[gray]{0.75}FO}%
\colorbox{green}{\color[gray]{0.75}FO}%
\colorbox{green}{\color[gray]{0.75}FO}%
\colorbox{green}{\color[gray]{0.75}FO}%
\colorbox{green}{\color[gray]{0.75}FO}%
\colorbox{green}{\color[gray]{0.75}FO}%
\colorbox{green}{\color[gray]{0.75}FO}%
\colorbox{green}{\color[gray]{0.75}FO}%
\colorbox{green}{\color[gray]{0.75}FO}%
\colorbox{green}{\color[gray]{0.75}FO}%
\colorbox{green}{\color[gray]{0.75}FO}%
\colorbox{green}{\color[gray]{0.75}FO}%
\\
\colorbox{green}{\color[gray]{0.75}FO}%
\colorbox{green}{\color[gray]{0.75}FO}%
\colorbox{green}{\color[gray]{0.75}FO}%
\colorbox{green}{\color[gray]{0.75}FO}%
\colorbox{green}{\color[gray]{0.75}FO}%
\colorbox{green}{\color[gray]{0.75}FO}%
\colorbox{green}{\color[gray]{0.75}FO}%
\colorbox{green}{\color[gray]{0.75}FO}%
\colorbox{green}{\color[gray]{0.75}FO}%
\colorbox{green}{\color[gray]{0.75}FO}%
\colorbox{green}{\color[gray]{0.75}FO}%
\colorbox{green}{\color[gray]{0.75}FO}%
\colorbox{green}{\color[gray]{0.75}FO}%
\colorbox{green}{\color[gray]{0.75}FO}%
\colorbox{green}{\color[gray]{0.75}FO}%
\colorbox{green}{\color[gray]{0.75}FO}%
\colorbox{green}{\color[gray]{0.75}FO}%
\colorbox{green}{\color[gray]{0.75}FO}%
\colorbox{green}{\color[gray]{0.75}FO}%
\colorbox{green}{\color[gray]{0.75}FO}%
\colorbox{green}{\color[gray]{0.75}FO}%
\colorbox{green}{\color[gray]{0.75}FO}%
\colorbox{green}{\color[gray]{0.75}FO}%
\colorbox{green}{\color[gray]{0.75}FO}%
\colorbox{green}{\color[gray]{0.75}FO}%
\colorbox{green}{\color[gray]{0.75}FO}%
\colorbox{green}{\color[gray]{0.75}FO}%
\colorbox{green}{\color[gray]{0.75}FO}%
\colorbox{green}{\color[gray]{0.75}FO}%
\colorbox{green}{\color[gray]{0.75}FO}%
\colorbox{green}{\color[gray]{0.75}FO}%
\colorbox{green}{\color[gray]{0.75}FO}%
\colorbox{green}{\color[gray]{0.75}FO}%
\colorbox{green}{\color[gray]{0.75}FO}%
\colorbox{green}{\color[gray]{0.75}FO}%
\colorbox{green}{\color[gray]{0.75}FO}%
\colorbox{green}{\color[gray]{0.75}FO}%
\colorbox{green}{\color[gray]{0.75}FO}%
\colorbox{green}{\color[gray]{0.75}FO}%
\colorbox{green}{\color[gray]{0.75}FO}%
\colorbox{green}{\color[gray]{0.75}FO}%
\colorbox{green}{\color[gray]{0.75}FO}%
\colorbox{green}{\color[gray]{0.75}FO}%
\colorbox{green}{\color[gray]{0.75}FO}%
\colorbox{green}{\color[gray]{0.75}FO}%
\colorbox{green}{\color[gray]{0.75}FO}%
\colorbox{green}{\color[gray]{0.75}FO}%
\colorbox{green}{\color[gray]{0.75}FO}%
\colorbox{green}{\color[gray]{0.75}FO}%
\colorbox{green}{\color[gray]{0.75}FO}%
\colorbox{green}{\color[gray]{0.75}FO}%
\colorbox{green}{\color[gray]{0.75}FO}%
\colorbox{green}{\color[gray]{0.75}FO}%
\colorbox{green}{\color[gray]{0.75}FO}%
\colorbox{green}{\color[gray]{0.75}FO}%
\colorbox{green}{\color[gray]{0.75}FO}%
\colorbox{green}{\color[gray]{0.75}FO}%
\colorbox{green}{\color[gray]{0.75}FO}%
\colorbox{green}{\color[gray]{0.75}FO}%
\colorbox{green}{\color[gray]{0.75}FO}%
\colorbox{green}{\color[gray]{0.75}FO}%
\colorbox{green}{\color[gray]{0.75}FO}%
\colorbox{green}{\color[gray]{0.75}FO}%
\colorbox{green}{\color[gray]{0.75}FO}%
\colorbox{green}{\color[gray]{0.75}FO}%
\colorbox{green}{\color[gray]{0.75}FO}%
\colorbox{green}{\color[gray]{0.75}FO}%
\colorbox{green}{\color[gray]{0.75}FO}%
\colorbox{green}{\color[gray]{0.75}FO}%
\colorbox{green}{\color[gray]{0.75}FO}%
\colorbox{green}{\color[gray]{0.75}FO}%
\colorbox{green}{\color[gray]{0.75}FO}%
\colorbox{green}{\color[gray]{0.75}FO}%
\colorbox{green}{\color[gray]{0.75}FO}%
\colorbox{green}{\color[gray]{0.75}FO}%
\colorbox{green}{\color[gray]{0.75}FO}%
\colorbox{green}{\color[gray]{0.75}FO}%
\colorbox{green}{\color[gray]{0.75}FO}%
\colorbox{green}{\color[gray]{0.75}FO}%
\colorbox{green}{\color[gray]{0.75}FO}%
\colorbox{green}{\color[gray]{0.75}FO}%
\colorbox{green}{\color[gray]{0.75}FO}%
\colorbox{green}{\color[gray]{0.75}FO}%
\colorbox{green}{\color[gray]{0.75}FO}%
\colorbox{green}{\color[gray]{0.75}FO}%
\colorbox{green}{\color[gray]{0.75}FO}%
\colorbox{green}{\color[gray]{0.75}FO}%
\colorbox{green}{\color[gray]{0.75}FO}%
\colorbox{green}{\color[gray]{0.75}FO}%
\colorbox{green}{\color[gray]{0.75}FO}%
\colorbox{green}{\color[gray]{0.75}FO}%
\colorbox{green}{\color[gray]{0.75}FO}%
\colorbox{green}{\color[gray]{0.75}FO}%
\colorbox{green}{\color[gray]{0.75}FO}%
\colorbox{green}{\color[gray]{0.75}FO}%
\colorbox{green}{\color[gray]{0.75}FO}%
\colorbox{green}{\color[gray]{0.75}FO}%
\colorbox{green}{\color[gray]{0.75}FO}%
\colorbox{green}{\color[gray]{0.75}FO}%
\colorbox{green}{\color[gray]{0.75}FO}%
\\
\colorbox{green}{\color[gray]{0.75}FO}%
\colorbox{green}{\color[gray]{0.75}FO}%
\colorbox{green}{\color[gray]{0.75}FO}%
\colorbox{green}{\color[gray]{0.75}FO}%
\colorbox{green}{\color[gray]{0.75}FO}%
\colorbox{green}{\color[gray]{0.75}FO}%
\colorbox{green}{\color[gray]{0.75}FO}%
\colorbox{green}{\color[gray]{0.75}FO}%
\colorbox{green}{\color[gray]{0.75}FO}%
\colorbox{green}{\color[gray]{0.75}FO}%
\colorbox{green}{\color[gray]{0.75}FO}%
\colorbox{green}{\color[gray]{0.75}FO}%
\colorbox{green}{\color[gray]{0.75}FO}%
\colorbox{green}{\color[gray]{0.75}FO}%
\colorbox{green}{\color[gray]{0.75}FO}%
\colorbox{green}{\color[gray]{0.75}FO}%
\colorbox{green}{\color[gray]{0.75}FO}%
\colorbox{green}{\color[gray]{0.75}FO}%
\colorbox{green}{\color[gray]{0.75}FO}%
\colorbox{green}{\color[gray]{0.75}FO}%
\colorbox{green}{\color[gray]{0.75}FO}%
\colorbox{green}{\color[gray]{0.75}FO}%
\colorbox{green}{\color[gray]{0.75}FO}%
\colorbox{green}{\color[gray]{0.75}FO}%
\colorbox{green}{\color[gray]{0.75}FO}%
\colorbox{green}{\color[gray]{0.75}FO}%
\colorbox{green}{\color[gray]{0.75}FO}%
\colorbox{green}{\color[gray]{0.75}FO}%
\colorbox{green}{\color[gray]{0.75}FO}%
\colorbox{green}{\color[gray]{0.75}FO}%
\colorbox{green}{\color[gray]{0.75}FO}%
\colorbox{green}{\color[gray]{0.75}FO}%
\colorbox{green}{\color[gray]{0.75}FO}%
\colorbox{green}{\color[gray]{0.75}FO}%
\colorbox{green}{\color[gray]{0.75}FO}%
\colorbox{green}{\color[gray]{0.75}FO}%
\colorbox{green}{\color[gray]{0.75}FO}%
\colorbox{green}{\color[gray]{0.75}FO}%
\colorbox{green}{\color[gray]{0.75}FO}%
\colorbox{green}{\color[gray]{0.75}FO}%
\colorbox{green}{\color[gray]{0.75}FO}%
\colorbox{green}{\color[gray]{0.75}FO}%
\colorbox{green}{\color[gray]{0.75}FO}%
\colorbox{green}{\color[gray]{0.75}FO}%
\colorbox{green}{\color[gray]{0.75}FO}%
\colorbox{green}{\color[gray]{0.75}FO}%
\colorbox{green}{\color[gray]{0.75}FO}%
\colorbox{green}{\color[gray]{0.75}FO}%
\colorbox{green}{\color[gray]{0.75}FO}%
\colorbox{green}{\color[gray]{0.75}FO}%
\colorbox{green}{\color[gray]{0.75}FO}%
\colorbox{green}{\color[gray]{0.75}FO}%
\colorbox{green}{\color[gray]{0.75}FO}%
\colorbox{green}{\color[gray]{0.75}FO}%
\colorbox{green}{\color[gray]{0.75}FO}%
\colorbox{green}{\color[gray]{0.75}FO}%
\colorbox{green}{\color[gray]{0.75}FO}%
\colorbox{green}{\color[gray]{0.75}FO}%
\colorbox{green}{\color[gray]{0.75}FO}%
\colorbox{green}{\color[gray]{0.75}FO}%
\colorbox{green}{\color[gray]{0.75}FO}%
\colorbox{green}{\color[gray]{0.75}FO}%
\colorbox{green}{\color[gray]{0.75}FO}%
\colorbox{green}{\color[gray]{0.75}FO}%
\colorbox{green}{\color[gray]{0.75}FO}%
\colorbox{green}{\color[gray]{0.75}FO}%
\colorbox{green}{\color[gray]{0.75}FO}%
\colorbox{green}{\color[gray]{0.75}FO}%
\colorbox{green}{\color[gray]{0.75}FO}%
\colorbox{green}{\color[gray]{0.75}FO}%
\colorbox{green}{\color[gray]{0.75}FO}%
\colorbox{green}{\color[gray]{0.75}FO}%
\colorbox{green}{\color[gray]{0.75}FO}%
\colorbox{green}{\color[gray]{0.75}FO}%
\colorbox{green}{\color[gray]{0.75}FO}%
\colorbox{green}{\color[gray]{0.75}FO}%
\colorbox{green}{\color[gray]{0.75}FO}%
\colorbox{green}{\color[gray]{0.75}FO}%
\colorbox{green}{\color[gray]{0.75}FO}%
\colorbox{green}{\color[gray]{0.75}FO}%
\colorbox{green}{\color[gray]{0.75}FO}%
\colorbox{green}{\color[gray]{0.75}FO}%
\colorbox{green}{\color[gray]{0.75}FO}%
\colorbox{green}{\color[gray]{0.75}FO}%
\colorbox{green}{\color[gray]{0.75}FO}%
\colorbox{green}{\color[gray]{0.75}FO}%
\colorbox{green}{\color[gray]{0.75}FO}%
\colorbox{green}{\color[gray]{0.75}FO}%
\colorbox{green}{\color[gray]{0.75}FO}%
\colorbox{green}{\color[gray]{0.75}FO}%
\colorbox{green}{\color[gray]{0.75}FO}%
\colorbox{green}{\color[gray]{0.75}FO}%
\colorbox{green}{\color[gray]{0.75}FO}%
\colorbox{green}{\color[gray]{0.75}FO}%
\colorbox{green}{\color[gray]{0.75}FO}%
\colorbox{green}{\color[gray]{0.75}FO}%
\colorbox{green}{\color[gray]{0.75}FO}%
\colorbox{green}{\color[gray]{0.75}FO}%
\colorbox{green}{\color[gray]{0.75}FO}%
\colorbox{green}{\color[gray]{0.75}FO}%
\\
\colorbox{green}{\color[gray]{0.75}FO}%
\colorbox{green}{\color[gray]{0.75}FO}%
\colorbox{green}{\color[gray]{0.75}FO}%
\colorbox{green}{\color[gray]{0.75}FO}%
\colorbox{green}{\color[gray]{0.75}FO}%
\colorbox{green}{\color[gray]{0.75}FO}%
\colorbox{green}{\color[gray]{0.75}FO}%
\colorbox{green}{\color[gray]{0.75}FO}%
\colorbox{green}{\color[gray]{0.75}FO}%
\colorbox{green}{\color[gray]{0.75}FO}%
\colorbox{green}{\color[gray]{0.75}FO}%
\colorbox{green}{\color[gray]{0.75}FO}%
\colorbox{green}{\color[gray]{0.75}FO}%
\colorbox{green}{\color[gray]{0.75}FO}%
\colorbox{green}{\color[gray]{0.75}FO}%
\colorbox{green}{\color[gray]{0.75}FO}%
\colorbox{green}{\color[gray]{0.75}FO}%
\colorbox{green}{\color[gray]{0.75}FO}%
\colorbox{green}{\color[gray]{0.75}FO}%
\colorbox{green}{\color[gray]{0.75}FO}%
\colorbox{green}{\color[gray]{0.75}FO}%
\colorbox{green}{\color[gray]{0.75}FO}%
\colorbox{green}{\color[gray]{0.75}FO}%
\colorbox{green}{\color[gray]{0.75}FO}%
\colorbox{green}{\color[gray]{0.75}FO}%
\colorbox{green}{\color[gray]{0.75}FO}%
\colorbox{green}{\color[gray]{0.75}FO}%
\colorbox{green}{\color[gray]{0.75}FO}%
\colorbox{green}{\color[gray]{0.75}FO}%
\colorbox{green}{\color[gray]{0.75}FO}%
\colorbox{green}{\color[gray]{0.75}FO}%
\colorbox{green}{\color[gray]{0.75}FO}%
\colorbox{green}{\color[gray]{0.75}FO}%
\colorbox{green}{\color[gray]{0.75}FO}%
\colorbox{green}{\color[gray]{0.75}FO}%
\colorbox{green}{\color[gray]{0.75}FO}%
\colorbox{green}{\color[gray]{0.75}FO}%
\colorbox{green}{\color[gray]{0.75}FO}%
\colorbox{green}{\color[gray]{0.75}FO}%
\colorbox{green}{\color[gray]{0.75}FO}%
\colorbox{green}{\color[gray]{0.75}FO}%
\colorbox{green}{\color[gray]{0.75}FO}%
\colorbox{green}{\color[gray]{0.75}FO}%
\colorbox{green}{\color[gray]{0.75}FO}%
\colorbox{green}{\color[gray]{0.75}FO}%
\colorbox{green}{\color[gray]{0.75}FO}%
\colorbox{green}{\color[gray]{0.75}FO}%
\colorbox{green}{\color[gray]{0.75}FO}%
\colorbox{green}{\color[gray]{0.75}FO}%
\colorbox{green}{\color[gray]{0.75}FO}%
\colorbox{green}{\color[gray]{0.75}FO}%
\colorbox{green}{\color[gray]{0.75}FO}%
\colorbox{green}{\color[gray]{0.75}FO}%
\colorbox{green}{\color[gray]{0.75}FO}%
\colorbox{green}{\color[gray]{0.75}FO}%
\colorbox{green}{\color[gray]{0.75}FO}%
\colorbox{green}{\color[gray]{0.75}FO}%
\colorbox{green}{\color[gray]{0.75}FO}%
\colorbox{green}{\color[gray]{0.75}FO}%
\colorbox{green}{\color[gray]{0.75}FO}%
\colorbox{green}{\color[gray]{0.75}FO}%
\colorbox{green}{\color[gray]{0.75}FO}%
\colorbox{green}{\color[gray]{0.75}FO}%
\colorbox{green}{\color[gray]{0.75}FO}%
\colorbox{green}{\color[gray]{0.75}FO}%
\colorbox{green}{\color[gray]{0.75}FO}%
\colorbox{green}{\color[gray]{0.75}FO}%
\colorbox{green}{\color[gray]{0.75}FO}%
\colorbox{green}{\color[gray]{0.75}FO}%
\colorbox{green}{\color[gray]{0.75}FO}%
\colorbox{green}{\color[gray]{0.75}FO}%
\colorbox{green}{\color[gray]{0.75}FO}%
\colorbox{green}{\color[gray]{0.75}FO}%
\colorbox{green}{\color[gray]{0.75}FO}%
\colorbox{green}{\color[gray]{0.75}FO}%
\colorbox{green}{\color[gray]{0.75}FO}%
\colorbox{green}{\color[gray]{0.75}FO}%
\colorbox{green}{\color[gray]{0.75}FO}%
\colorbox{green}{\color[gray]{0.75}FO}%
\colorbox{green}{\color[gray]{0.75}FO}%
\colorbox{green}{\color[gray]{0.75}FO}%
\colorbox{green}{\color[gray]{0.75}FO}%
\colorbox{green}{\color[gray]{0.75}FO}%
\colorbox{green}{\color[gray]{0.75}FO}%
\colorbox{green}{\color[gray]{0.75}FO}%
\colorbox{green}{\color[gray]{0.75}FO}%
\colorbox{green}{\color[gray]{0.75}FO}%
\colorbox{green}{\color[gray]{0.75}FO}%
\colorbox{green}{\color[gray]{0.75}FO}%
\colorbox{green}{\color[gray]{0.75}FO}%
\colorbox{green}{\color[gray]{0.75}FO}%
\colorbox{green}{\color[gray]{0.75}FO}%
\colorbox{green}{\color[gray]{0.75}FO}%
\colorbox{green}{\color[gray]{0.75}FO}%
\colorbox{green}{\color[gray]{0.75}FO}%
\colorbox{green}{\color[gray]{0.75}FO}%
\colorbox{green}{\color[gray]{0.75}FO}%
\colorbox{green}{\color[gray]{0.75}FO}%
\colorbox{green}{\color[gray]{0.75}FO}%
\colorbox{green}{\color[gray]{0.75}FO}%
\\
\colorbox{green}{\color[gray]{0.75}FO}%
\colorbox{green}{\color[gray]{0.75}FO}%
\colorbox{green}{\color[gray]{0.75}FO}%
\colorbox{green}{\color[gray]{0.75}FO}%
\colorbox{green}{\color[gray]{0.75}FO}%
\colorbox{green}{\color[gray]{0.75}FO}%
\colorbox{green}{\color[gray]{0.75}FO}%
\colorbox{green}{\color[gray]{0.75}FO}%
\colorbox{green}{\color[gray]{0.75}FO}%
\colorbox{green}{\color[gray]{0.75}FO}%
\colorbox{green}{\color[gray]{0.75}FO}%
\colorbox{green}{\color[gray]{0.75}FO}%
\colorbox{green}{\color[gray]{0.75}FO}%
\colorbox{green}{\color[gray]{0.75}FO}%
\colorbox{green}{\color[gray]{0.75}FO}%
\colorbox{green}{\color[gray]{0.75}FO}%
\colorbox{green}{\color[gray]{0.75}FO}%
\colorbox{green}{\color[gray]{0.75}FO}%
\colorbox{green}{\color[gray]{0.75}FO}%
\colorbox{green}{\color[gray]{0.75}FO}%
\colorbox{green}{\color[gray]{0.75}FO}%
\colorbox{green}{\color[gray]{0.75}FO}%
\colorbox{green}{\color[gray]{0.75}FO}%
\colorbox{green}{\color[gray]{0.75}FO}%
\colorbox{green}{\color[gray]{0.75}FO}%
\colorbox{green}{\color[gray]{0.75}FO}%
\colorbox{green}{\color[gray]{0.75}FO}%
\colorbox{green}{\color[gray]{0.75}FO}%
\colorbox{green}{\color[gray]{0.75}FO}%
\colorbox{green}{\color[gray]{0.75}FO}%
\colorbox{green}{\color[gray]{0.75}FO}%
\colorbox{green}{\color[gray]{0.75}FO}%
\colorbox{green}{\color[gray]{0.75}FO}%
\colorbox{green}{\color[gray]{0.75}FO}%
\colorbox{green}{\color[gray]{0.75}FO}%
\colorbox{green}{\color[gray]{0.75}FO}%
\colorbox{green}{\color[gray]{0.75}FO}%
\colorbox{green}{\color[gray]{0.75}FO}%
\colorbox{green}{\color[gray]{0.75}FO}%
\colorbox{green}{\color[gray]{0.75}FO}%
\colorbox{green}{\color[gray]{0.75}FO}%
\colorbox{green}{\color[gray]{0.75}FO}%
\colorbox{green}{\color[gray]{0.75}FO}%
\colorbox{green}{\color[gray]{0.75}FO}%
\colorbox{green}{\color[gray]{0.75}FO}%
\colorbox{green}{\color[gray]{0.75}FO}%
\colorbox{green}{\color[gray]{0.75}FO}%
\colorbox{green}{\color[gray]{0.75}FO}%
\colorbox{green}{\color[gray]{0.75}FO}%
\colorbox{green}{\color[gray]{0.75}FO}%
\colorbox{green}{\color[gray]{0.75}FO}%
\colorbox{green}{\color[gray]{0.75}FO}%
\colorbox{green}{\color[gray]{0.75}FO}%
\colorbox{green}{\color[gray]{0.75}FO}%
\colorbox{green}{\color[gray]{0.75}FO}%
\colorbox{green}{\color[gray]{0.75}FO}%
\colorbox{green}{\color[gray]{0.75}FO}%
\colorbox{green}{\color[gray]{0.75}FO}%
\colorbox{green}{\color[gray]{0.75}FO}%
\colorbox{green}{\color[gray]{0.75}FO}%
\colorbox{green}{\color[gray]{0.75}FO}%
\colorbox{green}{\color[gray]{0.75}FO}%
\colorbox{green}{\color[gray]{0.75}FO}%
\colorbox{green}{\color[gray]{0.75}FO}%
\colorbox{green}{\color[gray]{0.75}FO}%
\colorbox{green}{\color[gray]{0.75}FO}%
\colorbox{green}{\color[gray]{0.75}FO}%
\colorbox{green}{\color[gray]{0.75}FO}%
\colorbox{green}{\color[gray]{0.75}FO}%
\colorbox{green}{\color[gray]{0.75}FO}%
\colorbox{green}{\color[gray]{0.75}FO}%
\colorbox{green}{\color[gray]{0.75}FO}%
\colorbox{green}{\color[gray]{0.75}FO}%
\colorbox{green}{\color[gray]{0.75}FO}%
\colorbox{green}{\color[gray]{0.75}FO}%
\colorbox{green}{\color[gray]{0.75}FO}%
\colorbox{green}{\color[gray]{0.75}FO}%
\colorbox{green}{\color[gray]{0.75}FO}%
\colorbox{green}{\color[gray]{0.75}FO}%
\colorbox{green}{\color[gray]{0.75}FO}%
\colorbox{green}{\color[gray]{0.75}FO}%
\colorbox{green}{\color[gray]{0.75}FO}%
\colorbox{green}{\color[gray]{0.75}FO}%
\colorbox{green}{\color[gray]{0.75}FO}%
\colorbox{green}{\color[gray]{0.75}FO}%
\colorbox{green}{\color[gray]{0.75}FO}%
\colorbox{green}{\color[gray]{0.75}FO}%
\colorbox{green}{\color[gray]{0.75}FO}%
\colorbox{green}{\color[gray]{0.75}FO}%
\colorbox{green}{\color[gray]{0.75}FO}%
\colorbox{green}{\color[gray]{0.75}FO}%
\colorbox{green}{\color[gray]{0.75}FO}%
\colorbox{green}{\color[gray]{0.75}FO}%
\colorbox{green}{\color[gray]{0.75}FO}%
\colorbox{green}{\color[gray]{0.75}FO}%
\colorbox{green}{\color[gray]{0.75}FO}%
\colorbox{green}{\color[gray]{0.75}FO}%
\colorbox{green}{\color[gray]{0.75}FO}%
\colorbox{green}{\color[gray]{0.75}FO}%
\colorbox{green}{\color[gray]{0.75}FO}%
\\
\colorbox{green}{\color[gray]{0.75}FO}%
\colorbox{green}{\color[gray]{0.75}FO}%
\colorbox{green}{\color[gray]{0.75}FO}%
\colorbox{green}{\color[gray]{0.75}FO}%
\colorbox{green}{\color[gray]{0.75}FO}%
\colorbox{green}{\color[gray]{0.75}FO}%
\colorbox{green}{\color[gray]{0.75}FO}%
\colorbox{green}{\color[gray]{0.75}FO}%
\colorbox{green}{\color[gray]{0.75}FO}%
\colorbox{green}{\color[gray]{0.75}FO}%
\colorbox{green}{\color[gray]{0.75}FO}%
\colorbox{green}{\color[gray]{0.75}FO}%
\colorbox{green}{\color[gray]{0.75}FO}%
\colorbox{green}{\color[gray]{0.75}FO}%
\colorbox{green}{\color[gray]{0.75}FO}%
\colorbox{green}{\color[gray]{0.75}FO}%
\colorbox{green}{\color[gray]{0.75}FO}%
\colorbox{green}{\color[gray]{0.75}FO}%
\colorbox{green}{\color[gray]{0.75}FO}%
\colorbox{green}{\color[gray]{0.75}FO}%
\colorbox{green}{\color[gray]{0.75}FO}%
\colorbox{green}{\color[gray]{0.75}FO}%
\colorbox{green}{\color[gray]{0.75}FO}%
\colorbox{green}{\color[gray]{0.75}FO}%
\colorbox{green}{\color[gray]{0.75}FO}%
\colorbox{green}{\color[gray]{0.75}FO}%
\colorbox{green}{\color[gray]{0.75}FO}%
\colorbox{green}{\color[gray]{0.75}FO}%
\colorbox{green}{\color[gray]{0.75}FO}%
\colorbox{green}{\color[gray]{0.75}FO}%
\colorbox{green}{\color[gray]{0.75}FO}%
\colorbox{green}{\color[gray]{0.75}FO}%
\colorbox{green}{\color[gray]{0.75}FO}%
\colorbox{green}{\color[gray]{0.75}FO}%
\colorbox{green}{\color[gray]{0.75}FO}%
\colorbox{green}{\color[gray]{0.75}FO}%
\colorbox{green}{\color[gray]{0.75}FO}%
\colorbox{green}{\color[gray]{0.75}FO}%
\colorbox{green}{\color[gray]{0.75}FO}%
\colorbox{green}{\color[gray]{0.75}FO}%
\colorbox{green}{\color[gray]{0.75}FO}%
\colorbox{green}{\color[gray]{0.75}FO}%
\colorbox{green}{\color[gray]{0.75}FO}%
\colorbox{green}{\color[gray]{0.75}FO}%
\colorbox{green}{\color[gray]{0.75}FO}%
\colorbox{green}{\color[gray]{0.75}FO}%
\colorbox{green}{\color[gray]{0.75}FO}%
\colorbox{green}{\color[gray]{0.75}FO}%
\colorbox{green}{\color[gray]{0.75}FO}%
\colorbox{green}{\color[gray]{0.75}FO}%
\colorbox{green}{\color[gray]{0.75}FO}%
\colorbox{green}{\color[gray]{0.75}FO}%
\colorbox{green}{\color[gray]{0.75}FO}%
\colorbox{green}{\color[gray]{0.75}FO}%
\colorbox{green}{\color[gray]{0.75}FO}%
\colorbox{green}{\color[gray]{0.75}FO}%
\colorbox{green}{\color[gray]{0.75}FO}%
\colorbox{green}{\color[gray]{0.75}FO}%
\colorbox{green}{\color[gray]{0.75}FO}%
\colorbox{green}{\color[gray]{0.75}FO}%
\colorbox{green}{\color[gray]{0.75}FO}%
\colorbox{green}{\color[gray]{0.75}FO}%
\colorbox{green}{\color[gray]{0.75}FO}%
\colorbox{green}{\color[gray]{0.75}FO}%
\colorbox{green}{\color[gray]{0.75}FO}%
\colorbox{green}{\color[gray]{0.75}FO}%
\colorbox{green}{\color[gray]{0.75}FO}%
\colorbox{green}{\color[gray]{0.75}FO}%
\colorbox{green}{\color[gray]{0.75}FO}%
\colorbox{green}{\color[gray]{0.75}FO}%
\colorbox{green}{\color[gray]{0.75}FO}%
\colorbox{green}{\color[gray]{0.75}FO}%
\colorbox{green}{\color[gray]{0.75}FO}%
\colorbox{green}{\color[gray]{0.75}FO}%
\colorbox{green}{\color[gray]{0.75}FO}%
\colorbox{green}{\color[gray]{0.75}FO}%
\colorbox{green}{\color[gray]{0.75}FO}%
\colorbox{green}{\color[gray]{0.75}FO}%
\colorbox{green}{\color[gray]{0.75}FO}%
\colorbox{green}{\color[gray]{0.75}FO}%
\colorbox{green}{\color[gray]{0.75}FO}%
\colorbox{green}{\color[gray]{0.75}FO}%
\colorbox{green}{\color[gray]{0.75}FO}%
\colorbox{green}{\color[gray]{0.75}FO}%
\colorbox{green}{\color[gray]{0.75}FO}%
\colorbox{green}{\color[gray]{0.75}FO}%
\colorbox{green}{\color[gray]{0.75}FO}%
\colorbox{green}{\color[gray]{0.75}FO}%
\colorbox{green}{\color[gray]{0.75}FO}%
\colorbox{green}{\color[gray]{0.75}FO}%
\colorbox{green}{\color[gray]{0.75}FO}%
\colorbox{green}{\color[gray]{0.75}FO}%
\colorbox{green}{\color[gray]{0.75}FO}%
\colorbox{green}{\color[gray]{0.75}FO}%
\colorbox{green}{\color[gray]{0.75}FO}%
\colorbox{green}{\color[gray]{0.75}FO}%
\colorbox{green}{\color[gray]{0.75}FO}%
\colorbox{green}{\color[gray]{0.75}FO}%
\colorbox{green}{\color[gray]{0.75}FO}%
\colorbox{green}{\color[gray]{0.75}FO}%
\\
\colorbox{green}{\color[gray]{0.75}FO}%
\colorbox{green}{\color[gray]{0.75}FO}%
\colorbox{green}{\color[gray]{0.75}FO}%
\colorbox{green}{\color[gray]{0.75}FO}%
\colorbox{green}{\color[gray]{0.75}FO}%
\colorbox{green}{\color[gray]{0.75}FO}%
\colorbox{green}{\color[gray]{0.75}FO}%
\colorbox{green}{\color[gray]{0.75}FO}%
\colorbox{green}{\color[gray]{0.75}FO}%
\colorbox{green}{\color[gray]{0.75}FO}%
\colorbox{green}{\color[gray]{0.75}FO}%
\colorbox{green}{\color[gray]{0.75}FO}%
\colorbox{green}{\color[gray]{0.75}FO}%
\colorbox{green}{\color[gray]{0.75}FO}%
\colorbox{green}{\color[gray]{0.75}FO}%
\colorbox{green}{\color[gray]{0.75}FO}%
\colorbox{green}{\color[gray]{0.75}FO}%
\colorbox{green}{\color[gray]{0.75}FO}%
\colorbox{green}{\color[gray]{0.75}FO}%
\colorbox{green}{\color[gray]{0.75}FO}%
\colorbox{green}{\color[gray]{0.75}FO}%
\colorbox{green}{\color[gray]{0.75}FO}%
\colorbox{green}{\color[gray]{0.75}FO}%
\colorbox{green}{\color[gray]{0.75}FO}%
\colorbox{green}{\color[gray]{0.75}FO}%
\colorbox{green}{\color[gray]{0.75}FO}%
\colorbox{green}{\color[gray]{0.75}FO}%
\colorbox{green}{\color[gray]{0.75}FO}%
\colorbox{green}{\color[gray]{0.75}FO}%
\colorbox{green}{\color[gray]{0.75}FO}%
\colorbox{green}{\color[gray]{0.75}FO}%
\colorbox{green}{\color[gray]{0.75}FO}%
\colorbox{green}{\color[gray]{0.75}FO}%
\colorbox{green}{\color[gray]{0.75}FO}%
\colorbox{green}{\color[gray]{0.75}FO}%
\colorbox{green}{\color[gray]{0.75}FO}%
\colorbox{green}{\color[gray]{0.75}FO}%
\colorbox{green}{\color[gray]{0.75}FO}%
\colorbox{green}{\color[gray]{0.75}FO}%
\colorbox{green}{\color[gray]{0.75}FO}%
\colorbox{green}{\color[gray]{0.75}FO}%
\colorbox{green}{\color[gray]{0.75}FO}%
\colorbox{green}{\color[gray]{0.75}FO}%
\colorbox{green}{\color[gray]{0.75}FO}%
\colorbox{green}{\color[gray]{0.75}FO}%
\colorbox{green}{\color[gray]{0.75}FO}%
\colorbox{green}{\color[gray]{0.75}FO}%
\colorbox{green}{\color[gray]{0.75}FO}%
\colorbox{green}{\color[gray]{0.75}FO}%
\colorbox{green}{\color[gray]{0.75}FO}%
\colorbox{green}{\color[gray]{0.75}FO}%
\colorbox{green}{\color[gray]{0.75}FO}%
\colorbox{green}{\color[gray]{0.75}FO}%
\colorbox{green}{\color[gray]{0.75}FO}%
\colorbox{green}{\color[gray]{0.75}FO}%
\colorbox{green}{\color[gray]{0.75}FO}%
\colorbox{green}{\color[gray]{0.75}FO}%
\colorbox{green}{\color[gray]{0.75}FO}%
\colorbox{green}{\color[gray]{0.75}FO}%
\colorbox{green}{\color[gray]{0.75}FO}%
\colorbox{green}{\color[gray]{0.75}FO}%
\colorbox{green}{\color[gray]{0.75}FO}%
\colorbox{green}{\color[gray]{0.75}FO}%
\colorbox{green}{\color[gray]{0.75}FO}%
\colorbox{green}{\color[gray]{0.75}FO}%
\colorbox{green}{\color[gray]{0.75}FO}%
\colorbox{green}{\color[gray]{0.75}FO}%
\colorbox{green}{\color[gray]{0.75}FO}%
\colorbox{green}{\color[gray]{0.75}FO}%
\colorbox{green}{\color[gray]{0.75}FO}%
\colorbox{green}{\color[gray]{0.75}FO}%
\colorbox{green}{\color[gray]{0.75}FO}%
\colorbox{green}{\color[gray]{0.75}FO}%
\colorbox{green}{\color[gray]{0.75}FO}%
\colorbox{green}{\color[gray]{0.75}FO}%
\colorbox{green}{\color[gray]{0.75}FO}%
\colorbox{green}{\color[gray]{0.75}FO}%
\colorbox{green}{\color[gray]{0.75}FO}%
\colorbox{green}{\color[gray]{0.75}FO}%
\colorbox{green}{\color[gray]{0.75}FO}%
\colorbox{green}{\color[gray]{0.75}FO}%
\colorbox{green}{\color[gray]{0.75}FO}%
\colorbox{green}{\color[gray]{0.75}FO}%
\colorbox{green}{\color[gray]{0.75}FO}%
\colorbox{green}{\color[gray]{0.75}FO}%
\colorbox{green}{\color[gray]{0.75}FO}%
\colorbox{green}{\color[gray]{0.75}FO}%
\colorbox{green}{\color[gray]{0.75}FO}%
\colorbox{green}{\color[gray]{0.75}FO}%
\colorbox{green}{\color[gray]{0.75}FO}%
\colorbox{green}{\color[gray]{0.75}FO}%
\colorbox{green}{\color[gray]{0.75}FO}%
\colorbox{green}{\color[gray]{0.75}FO}%
\colorbox{green}{\color[gray]{0.75}FO}%
\colorbox{green}{\color[gray]{0.75}FO}%
\colorbox{green}{\color[gray]{0.75}FO}%
\colorbox{green}{\color[gray]{0.75}FO}%
\colorbox{green}{\color[gray]{0.75}FO}%
\colorbox{green}{\color[gray]{0.75}FO}%
\colorbox{green}{\color[gray]{0.75}FO}%
\\
\colorbox{green}{\color[gray]{0.75}FO}%
\colorbox{green}{\color[gray]{0.75}FO}%
\colorbox{green}{\color[gray]{0.75}FO}%
\colorbox{green}{\color[gray]{0.75}FO}%
\colorbox{green}{\color[gray]{0.75}FO}%
\colorbox{green}{\color[gray]{0.75}FO}%
\colorbox{green}{\color[gray]{0.75}FO}%
\colorbox{green}{\color[gray]{0.75}FO}%
\colorbox{green}{\color[gray]{0.75}FO}%
\colorbox{green}{\color[gray]{0.75}FO}%
\colorbox{green}{\color[gray]{0.75}FO}%
\colorbox{green}{\color[gray]{0.75}FO}%
\colorbox{green}{\color[gray]{0.75}FO}%
\colorbox{green}{\color[gray]{0.75}FO}%
\colorbox{green}{\color[gray]{0.75}FO}%
\colorbox{green}{\color[gray]{0.75}FO}%
\colorbox{green}{\color[gray]{0.75}FO}%
\colorbox{green}{\color[gray]{0.75}FO}%
\colorbox{green}{\color[gray]{0.75}FO}%
\colorbox{green}{\color[gray]{0.75}FO}%
\colorbox{green}{\color[gray]{0.75}FO}%
\colorbox{green}{\color[gray]{0.75}FO}%
\colorbox{green}{\color[gray]{0.75}FO}%
\colorbox{green}{\color[gray]{0.75}FO}%
\colorbox{green}{\color[gray]{0.75}FO}%
\colorbox{green}{\color[gray]{0.75}FO}%
\colorbox{green}{\color[gray]{0.75}FO}%
\colorbox{green}{\color[gray]{0.75}FO}%
\colorbox{green}{\color[gray]{0.75}FO}%
\colorbox{green}{\color[gray]{0.75}FO}%
\colorbox{green}{\color[gray]{0.75}FO}%
\colorbox{green}{\color[gray]{0.75}FO}%
\colorbox{green}{\color[gray]{0.75}FO}%
\colorbox{green}{\color[gray]{0.75}FO}%
\colorbox{green}{\color[gray]{0.75}FO}%
\colorbox{green}{\color[gray]{0.75}FO}%
\colorbox{green}{\color[gray]{0.75}FO}%
\colorbox{green}{\color[gray]{0.75}FO}%
\colorbox{green}{\color[gray]{0.75}FO}%
\colorbox{green}{\color[gray]{0.75}FO}%
\colorbox{green}{\color[gray]{0.75}FO}%
\colorbox{green}{\color[gray]{0.75}FO}%
\colorbox{green}{\color[gray]{0.75}FO}%
\colorbox{green}{\color[gray]{0.75}FO}%
\colorbox{green}{\color[gray]{0.75}FO}%
\colorbox{green}{\color[gray]{0.75}FO}%
\colorbox{green}{\color[gray]{0.75}FO}%
\colorbox{green}{\color[gray]{0.75}FO}%
\colorbox{green}{\color[gray]{0.75}FO}%
\colorbox{green}{\color[gray]{0.75}FO}%
\colorbox{green}{\color[gray]{0.75}FO}%
\colorbox{green}{\color[gray]{0.75}FO}%
\colorbox{green}{\color[gray]{0.75}FO}%
\colorbox{green}{\color[gray]{0.75}FO}%
\colorbox{green}{\color[gray]{0.75}FO}%
\colorbox{green}{\color[gray]{0.75}FO}%
\colorbox{green}{\color[gray]{0.75}FO}%
\colorbox{green}{\color[gray]{0.75}FO}%
\colorbox{green}{\color[gray]{0.75}FO}%
\colorbox{green}{\color[gray]{0.75}FO}%
\colorbox{green}{\color[gray]{0.75}FO}%
\colorbox{green}{\color[gray]{0.75}FO}%
\colorbox{green}{\color[gray]{0.75}FO}%
\colorbox{green}{\color[gray]{0.75}FO}%
\colorbox{green}{\color[gray]{0.75}FO}%
\colorbox{green}{\color[gray]{0.75}FO}%
\colorbox{green}{\color[gray]{0.75}FO}%
\colorbox{green}{\color[gray]{0.75}FO}%
\colorbox{green}{\color[gray]{0.75}FO}%
\colorbox{green}{\color[gray]{0.75}FO}%
\colorbox{green}{\color[gray]{0.75}FO}%
\colorbox{green}{\color[gray]{0.75}FO}%
\colorbox{green}{\color[gray]{0.75}FO}%
\colorbox{green}{\color[gray]{0.75}FO}%
\colorbox{green}{\color[gray]{0.75}FO}%
\colorbox{green}{\color[gray]{0.75}FO}%
\colorbox{green}{\color[gray]{0.75}FO}%
\colorbox{green}{\color[gray]{0.75}FO}%
\colorbox{green}{\color[gray]{0.75}FO}%
\colorbox{green}{\color[gray]{0.75}FO}%
\colorbox{green}{\color[gray]{0.75}FO}%
\colorbox{green}{\color[gray]{0.75}FO}%
\colorbox{green}{\color[gray]{0.75}FO}%
\colorbox{green}{\color[gray]{0.75}FO}%
\colorbox{green}{\color[gray]{0.75}FO}%
\colorbox{green}{\color[gray]{0.75}FO}%
\colorbox{green}{\color[gray]{0.75}FO}%
\colorbox{green}{\color[gray]{0.75}FO}%
\colorbox{green}{\color[gray]{0.75}FO}%
\colorbox{green}{\color[gray]{0.75}FO}%
\colorbox{green}{\color[gray]{0.75}FO}%
\colorbox{green}{\color[gray]{0.75}FO}%
\colorbox{green}{\color[gray]{0.75}FO}%
\colorbox{green}{\color[gray]{0.75}FO}%
\colorbox{green}{\color[gray]{0.75}FO}%
\colorbox{green}{\color[gray]{0.75}FO}%
\colorbox{green}{\color[gray]{0.75}FO}%
\colorbox{green}{\color[gray]{0.75}FO}%
\colorbox{green}{\color[gray]{0.75}FO}%
\colorbox{green}{\color[gray]{0.75}FO}%
\\
\colorbox{green}{\color[gray]{0.75}FO}%
\colorbox{green}{\color[gray]{0.75}FO}%
\colorbox{green}{\color[gray]{0.75}FO}%
\colorbox{green}{\color[gray]{0.75}FO}%
\colorbox{green}{\color[gray]{0.75}FO}%
\colorbox{green}{\color[gray]{0.75}FO}%
\colorbox{green}{\color[gray]{0.75}FO}%
\colorbox{green}{\color[gray]{0.75}FO}%
\colorbox{green}{\color[gray]{0.75}FO}%
\colorbox{green}{\color[gray]{0.75}FO}%
\colorbox{green}{\color[gray]{0.75}FO}%
\colorbox{green}{\color[gray]{0.75}FO}%
\colorbox{green}{\color[gray]{0.75}FO}%
\colorbox{green}{\color[gray]{0.75}FO}%
\colorbox{green}{\color[gray]{0.75}FO}%
\colorbox{green}{\color[gray]{0.75}FO}%
\colorbox{green}{\color[gray]{0.75}FO}%
\colorbox{green}{\color[gray]{0.75}FO}%
\colorbox{green}{\color[gray]{0.75}FO}%
\colorbox{green}{\color[gray]{0.75}FO}%
\colorbox{green}{\color[gray]{0.75}FO}%
\colorbox{green}{\color[gray]{0.75}FO}%
\colorbox{green}{\color[gray]{0.75}FO}%
\colorbox{green}{\color[gray]{0.75}FO}%
\colorbox{green}{\color[gray]{0.75}FO}%
\colorbox{green}{\color[gray]{0.75}FO}%
\colorbox{green}{\color[gray]{0.75}FO}%
\colorbox{green}{\color[gray]{0.75}FO}%
\colorbox{green}{\color[gray]{0.75}FO}%
\colorbox{green}{\color[gray]{0.75}FO}%
\colorbox{green}{\color[gray]{0.75}FO}%
\colorbox{green}{\color[gray]{0.75}FO}%
\colorbox{green}{\color[gray]{0.75}FO}%
\colorbox{green}{\color[gray]{0.75}FO}%
\colorbox{green}{\color[gray]{0.75}FO}%
\colorbox{green}{\color[gray]{0.75}FO}%
\colorbox{green}{\color[gray]{0.75}FO}%
\colorbox{green}{\color[gray]{0.75}FO}%
\colorbox{green}{\color[gray]{0.75}FO}%
\colorbox{green}{\color[gray]{0.75}FO}%
\colorbox{green}{\color[gray]{0.75}FO}%
\colorbox{green}{\color[gray]{0.75}FO}%
\colorbox{green}{\color[gray]{0.75}FO}%
\colorbox{green}{\color[gray]{0.75}FO}%
\colorbox{green}{\color[rgb]{1,0,0}\textbf{BU}}%
\colorbox{green}{\color[gray]{0.75}FO}%
\colorbox{green}{\color[gray]{0.75}FO}%
\colorbox{green}{\color[gray]{0.75}FO}%
\colorbox{green}{\color[gray]{0.75}FO}%
\colorbox{green}{\color[gray]{0.75}FO}%
\colorbox{green}{\color[gray]{0.75}FO}%
\colorbox{green}{\color[gray]{0.75}FO}%
\colorbox{green}{\color[gray]{0.75}FO}%
\colorbox{green}{\color[gray]{0.75}FO}%
\colorbox{green}{\color[gray]{0.75}FO}%
\colorbox{green}{\color[gray]{0.75}FO}%
\colorbox{green}{\color[gray]{0.75}FO}%
\colorbox{green}{\color[gray]{0.75}FO}%
\colorbox{green}{\color[gray]{0.75}FO}%
\colorbox{green}{\color[gray]{0.75}FO}%
\colorbox{green}{\color[gray]{0.75}FO}%
\colorbox{green}{\color[gray]{0.75}FO}%
\colorbox{green}{\color[gray]{0.75}FO}%
\colorbox{green}{\color[gray]{0.75}FO}%
\colorbox{green}{\color[gray]{0.75}FO}%
\colorbox{green}{\color[gray]{0.75}FO}%
\colorbox{green}{\color[gray]{0.75}FO}%
\colorbox{green}{\color[gray]{0.75}FO}%
\colorbox{green}{\color[gray]{0.75}FO}%
\colorbox{green}{\color[gray]{0.75}FO}%
\colorbox{green}{\color[gray]{0.75}FO}%
\colorbox{green}{\color[gray]{0.75}FO}%
\colorbox{green}{\color[gray]{0.75}FO}%
\colorbox{green}{\color[gray]{0.75}FO}%
\colorbox{green}{\color[gray]{0.75}FO}%
\colorbox{green}{\color[gray]{0.75}FO}%
\colorbox{green}{\color[gray]{0.75}FO}%
\colorbox{green}{\color[gray]{0.75}FO}%
\colorbox{green}{\color[gray]{0.75}FO}%
\colorbox{green}{\color[gray]{0.75}FO}%
\colorbox{green}{\color[gray]{0.75}FO}%
\colorbox{green}{\color[gray]{0.75}FO}%
\colorbox{green}{\color[gray]{0.75}FO}%
\colorbox{green}{\color[gray]{0.75}FO}%
\colorbox{green}{\color[gray]{0.75}FO}%
\colorbox{green}{\color[gray]{0.75}FO}%
\colorbox{green}{\color[gray]{0.75}FO}%
\colorbox{green}{\color[gray]{0.75}FO}%
\colorbox{green}{\color[gray]{0.75}FO}%
\colorbox{green}{\color[gray]{0.75}FO}%
\colorbox{green}{\color[gray]{0.75}FO}%
\colorbox{green}{\color[gray]{0.75}FO}%
\colorbox{green}{\color[gray]{0.75}FO}%
\colorbox{green}{\color[gray]{0.75}FO}%
\colorbox{green}{\color[gray]{0.75}FO}%
\colorbox{green}{\color[gray]{0.75}FO}%
\colorbox{green}{\color[gray]{0.75}FO}%
\colorbox{green}{\color[gray]{0.75}FO}%
\colorbox{green}{\color[gray]{0.75}FO}%
\colorbox{green}{\color[gray]{0.75}FO}%
\\
\colorbox{green}{\color[gray]{0.75}FO}%
\colorbox{green}{\color[gray]{0.75}FO}%
\colorbox{green}{\color[gray]{0.75}FO}%
\colorbox{green}{\color[gray]{0.75}FO}%
\colorbox{green}{\color[gray]{0.75}FO}%
\colorbox{green}{\color[gray]{0.75}FO}%
\colorbox{green}{\color[gray]{0.75}FO}%
\colorbox{green}{\color[gray]{0.75}FO}%
\colorbox{green}{\color[gray]{0.75}FO}%
\colorbox{green}{\color[gray]{0.75}FO}%
\colorbox{green}{\color[gray]{0.75}FO}%
\colorbox{green}{\color[gray]{0.75}FO}%
\colorbox{green}{\color[gray]{0.75}FO}%
\colorbox{green}{\color[gray]{0.75}FO}%
\colorbox{green}{\color[gray]{0.75}FO}%
\colorbox{green}{\color[gray]{0.75}FO}%
\colorbox{green}{\color[gray]{0.75}FO}%
\colorbox{green}{\color[gray]{0.75}FO}%
\colorbox{green}{\color[gray]{0.75}FO}%
\colorbox{green}{\color[gray]{0.75}FO}%
\colorbox{green}{\color[gray]{0.75}FO}%
\colorbox{green}{\color[gray]{0.75}FO}%
\colorbox{green}{\color[gray]{0.75}FO}%
\colorbox{green}{\color[gray]{0.75}FO}%
\colorbox{green}{\color[gray]{0.75}FO}%
\colorbox{green}{\color[gray]{0.75}FO}%
\colorbox{green}{\color[gray]{0.75}FO}%
\colorbox{green}{\color[gray]{0.75}FO}%
\colorbox{green}{\color[gray]{0.75}FO}%
\colorbox{green}{\color[gray]{0.75}FO}%
\colorbox{green}{\color[gray]{0.75}FO}%
\colorbox{green}{\color[gray]{0.75}FO}%
\colorbox{green}{\color[gray]{0.75}FO}%
\colorbox{green}{\color[gray]{0.75}FO}%
\colorbox{green}{\color[gray]{0.75}FO}%
\colorbox{green}{\color[gray]{0.75}FO}%
\colorbox{green}{\color[gray]{0.75}FO}%
\colorbox{green}{\color[gray]{0.75}FO}%
\colorbox{green}{\color[gray]{0.75}FO}%
\colorbox{green}{\color[gray]{0.75}FO}%
\colorbox{green}{\color[gray]{0.75}FO}%
\colorbox{green}{\color[gray]{0.75}FO}%
\colorbox{green}{\color[gray]{0.75}FO}%
\colorbox{green}{\color[gray]{0.75}FO}%
\colorbox{green}{\color[gray]{0.75}FO}%
\colorbox{green}{\color[gray]{0.75}FO}%
\colorbox{green}{\color[gray]{0.75}FO}%
\colorbox{green}{\color[gray]{0.75}FO}%
\colorbox{green}{\color[gray]{0.75}FO}%
\colorbox{green}{\color[gray]{0.75}FO}%
\colorbox{green}{\color[gray]{0.75}FO}%
\colorbox{green}{\color[gray]{0.75}FO}%
\colorbox{green}{\color[gray]{0.75}FO}%
\colorbox{green}{\color[gray]{0.75}FO}%
\colorbox{green}{\color[gray]{0.75}FO}%
\colorbox{green}{\color[gray]{0.75}FO}%
\colorbox{green}{\color[gray]{0.75}FO}%
\colorbox{green}{\color[gray]{0.75}FO}%
\colorbox{green}{\color[gray]{0.75}FO}%
\colorbox{green}{\color[gray]{0.75}FO}%
\colorbox{green}{\color[gray]{0.75}FO}%
\colorbox{green}{\color[gray]{0.75}FO}%
\colorbox{green}{\color[gray]{0.75}FO}%
\colorbox{green}{\color[gray]{0.75}FO}%
\colorbox{green}{\color[gray]{0.75}FO}%
\colorbox{green}{\color[gray]{0.75}FO}%
\colorbox{green}{\color[gray]{0.75}FO}%
\colorbox{green}{\color[gray]{0.75}FO}%
\colorbox{green}{\color[gray]{0.75}FO}%
\colorbox{green}{\color[gray]{0.75}FO}%
\colorbox{green}{\color[gray]{0.75}FO}%
\colorbox{green}{\color[gray]{0.75}FO}%
\colorbox{green}{\color[gray]{0.75}FO}%
\colorbox{green}{\color[gray]{0.75}FO}%
\colorbox{green}{\color[gray]{0.75}FO}%
\colorbox{green}{\color[gray]{0.75}FO}%
\colorbox{green}{\color[gray]{0.75}FO}%
\colorbox{green}{\color[gray]{0.75}FO}%
\colorbox{green}{\color[gray]{0.75}FO}%
\colorbox{green}{\color[gray]{0.75}FO}%
\colorbox{green}{\color[gray]{0.75}FO}%
\colorbox{green}{\color[gray]{0.75}FO}%
\colorbox{green}{\color[gray]{0.75}FO}%
\colorbox{green}{\color[gray]{0.75}FO}%
\colorbox{green}{\color[gray]{0.75}FO}%
\colorbox{green}{\color[gray]{0.75}FO}%
\colorbox{green}{\color[gray]{0.75}FO}%
\colorbox{green}{\color[gray]{0.75}FO}%
\colorbox{green}{\color[gray]{0.75}FO}%
\colorbox{green}{\color[gray]{0.75}FO}%
\colorbox{green}{\color[gray]{0.75}FO}%
\colorbox{green}{\color[gray]{0.75}FO}%
\colorbox{green}{\color[gray]{0.75}FO}%
\colorbox{green}{\color[gray]{0.75}FO}%
\colorbox{green}{\color[gray]{0.75}FO}%
\colorbox{green}{\color[gray]{0.75}FO}%
\colorbox{green}{\color[gray]{0.75}FO}%
\colorbox{green}{\color[gray]{0.75}FO}%
\colorbox{green}{\color[gray]{0.75}FO}%
\colorbox{green}{\color[gray]{0.75}FO}%
\\
\colorbox{green}{\color[gray]{0.75}FO}%
\colorbox{green}{\color[gray]{0.75}FO}%
\colorbox{green}{\color[gray]{0.75}FO}%
\colorbox{green}{\color[gray]{0.75}FO}%
\colorbox{green}{\color[gray]{0.75}FO}%
\colorbox{green}{\color[gray]{0.75}FO}%
\colorbox{green}{\color[gray]{0.75}FO}%
\colorbox{green}{\color[gray]{0.75}FO}%
\colorbox{green}{\color[gray]{0.75}FO}%
\colorbox{green}{\color[gray]{0.75}FO}%
\colorbox{green}{\color[gray]{0.75}FO}%
\colorbox{green}{\color[gray]{0.75}FO}%
\colorbox{green}{\color[gray]{0.75}FO}%
\colorbox{green}{\color[gray]{0.75}FO}%
\colorbox{green}{\color[gray]{0.75}FO}%
\colorbox{green}{\color[gray]{0.75}FO}%
\colorbox{green}{\color[gray]{0.75}FO}%
\colorbox{green}{\color[gray]{0.75}FO}%
\colorbox{green}{\color[gray]{0.75}FO}%
\colorbox{green}{\color[gray]{0.75}FO}%
\colorbox{green}{\color[gray]{0.75}FO}%
\colorbox{green}{\color[gray]{0.75}FO}%
\colorbox{green}{\color[gray]{0.75}FO}%
\colorbox{green}{\color[gray]{0.75}FO}%
\colorbox{green}{\color[gray]{0.75}FO}%
\colorbox{green}{\color[gray]{0.75}FO}%
\colorbox{green}{\color[gray]{0.75}FO}%
\colorbox{green}{\color[gray]{0.75}FO}%
\colorbox{green}{\color[gray]{0.75}FO}%
\colorbox{green}{\color[gray]{0.75}FO}%
\colorbox{green}{\color[gray]{0.75}FO}%
\colorbox{green}{\color[gray]{0.75}FO}%
\colorbox{green}{\color[gray]{0.75}FO}%
\colorbox{green}{\color[gray]{0.75}FO}%
\colorbox{green}{\color[gray]{0.75}FO}%
\colorbox{green}{\color[gray]{0.75}FO}%
\colorbox{green}{\color[gray]{0.75}FO}%
\colorbox{green}{\color[gray]{0.75}FO}%
\colorbox{green}{\color[gray]{0.75}FO}%
\colorbox{green}{\color[gray]{0.75}FO}%
\colorbox{green}{\color[gray]{0.75}FO}%
\colorbox{green}{\color[gray]{0.75}FO}%
\colorbox{green}{\color[gray]{0.75}FO}%
\colorbox{green}{\color[gray]{0.75}FO}%
\colorbox{green}{\color[gray]{0.75}FO}%
\colorbox{green}{\color[gray]{0.75}FO}%
\colorbox{green}{\color[gray]{0.75}FO}%
\colorbox{green}{\color[gray]{0.75}FO}%
\colorbox{green}{\color[gray]{0.75}FO}%
\colorbox{green}{\color[gray]{0.75}FO}%
\colorbox{green}{\color[gray]{0.75}FO}%
\colorbox{green}{\color[gray]{0.75}FO}%
\colorbox{green}{\color[gray]{0.75}FO}%
\colorbox{green}{\color[gray]{0.75}FO}%
\colorbox{green}{\color[gray]{0.75}FO}%
\colorbox{green}{\color[gray]{0.75}FO}%
\colorbox{green}{\color[gray]{0.75}FO}%
\colorbox{green}{\color[gray]{0.75}FO}%
\colorbox{green}{\color[gray]{0.75}FO}%
\colorbox{green}{\color[gray]{0.75}FO}%
\colorbox{green}{\color[gray]{0.75}FO}%
\colorbox{green}{\color[gray]{0.75}FO}%
\colorbox{green}{\color[gray]{0.75}FO}%
\colorbox{green}{\color[gray]{0.75}FO}%
\colorbox{green}{\color[gray]{0.75}FO}%
\colorbox{green}{\color[gray]{0.75}FO}%
\colorbox{green}{\color[gray]{0.75}FO}%
\colorbox{green}{\color[gray]{0.75}FO}%
\colorbox{green}{\color[gray]{0.75}FO}%
\colorbox{green}{\color[gray]{0.75}FO}%
\colorbox{green}{\color[gray]{0.75}FO}%
\colorbox{green}{\color[gray]{0.75}FO}%
\colorbox{green}{\color[gray]{0.75}FO}%
\colorbox{green}{\color[gray]{0.75}FO}%
\colorbox{green}{\color[gray]{0.75}FO}%
\colorbox{green}{\color[gray]{0.75}FO}%
\colorbox{green}{\color[gray]{0.75}FO}%
\colorbox{green}{\color[gray]{0.75}FO}%
\colorbox{green}{\color[gray]{0.75}FO}%
\colorbox{green}{\color[gray]{0.75}FO}%
\colorbox{green}{\color[gray]{0.75}FO}%
\colorbox{green}{\color[gray]{0.75}FO}%
\colorbox{green}{\color[gray]{0.75}FO}%
\colorbox{green}{\color[gray]{0.75}FO}%
\colorbox{green}{\color[gray]{0.75}FO}%
\colorbox{green}{\color[gray]{0.75}FO}%
\colorbox{green}{\color[gray]{0.75}FO}%
\colorbox{green}{\color[gray]{0.75}FO}%
\colorbox{green}{\color[gray]{0.75}FO}%
\colorbox{green}{\color[gray]{0.75}FO}%
\colorbox{green}{\color[gray]{0.75}FO}%
\colorbox{green}{\color[gray]{0.75}FO}%
\colorbox{green}{\color[gray]{0.75}FO}%
\colorbox{green}{\color[gray]{0.75}FO}%
\colorbox{green}{\color[gray]{0.75}FO}%
\colorbox{green}{\color[gray]{0.75}FO}%
\colorbox{green}{\color[gray]{0.75}FO}%
\colorbox{green}{\color[gray]{0.75}FO}%
\colorbox{green}{\color[gray]{0.75}FO}%
\colorbox{green}{\color[gray]{0.75}FO}%
\\
\colorbox{green}{\color[gray]{0.75}FO}%
\colorbox{green}{\color[gray]{0.75}FO}%
\colorbox{green}{\color[gray]{0.75}FO}%
\colorbox{green}{\color[gray]{0.75}FO}%
\colorbox{green}{\color[gray]{0.75}FO}%
\colorbox{green}{\color[gray]{0.75}FO}%
\colorbox{green}{\color[gray]{0.75}FO}%
\colorbox{green}{\color[gray]{0.75}FO}%
\colorbox{green}{\color[gray]{0.75}FO}%
\colorbox{green}{\color[gray]{0.75}FO}%
\colorbox{green}{\color[gray]{0.75}FO}%
\colorbox{green}{\color[gray]{0.75}FO}%
\colorbox{green}{\color[gray]{0.75}FO}%
\colorbox{green}{\color[gray]{0.75}FO}%
\colorbox{green}{\color[gray]{0.75}FO}%
\colorbox{green}{\color[gray]{0.75}FO}%
\colorbox{green}{\color[gray]{0.75}FO}%
\colorbox{green}{\color[gray]{0.75}FO}%
\colorbox{green}{\color[gray]{0.75}FO}%
\colorbox{green}{\color[gray]{0.75}FO}%
\colorbox{green}{\color[gray]{0.75}FO}%
\colorbox{green}{\color[gray]{0.75}FO}%
\colorbox{green}{\color[gray]{0.75}FO}%
\colorbox{green}{\color[gray]{0.75}FO}%
\colorbox{green}{\color[gray]{0.75}FO}%
\colorbox{green}{\color[gray]{0.75}FO}%
\colorbox{green}{\color[gray]{0.75}FO}%
\colorbox{green}{\color[gray]{0.75}FO}%
\colorbox{green}{\color[gray]{0.75}FO}%
\colorbox{green}{\color[gray]{0.75}FO}%
\colorbox{green}{\color[gray]{0.75}FO}%
\colorbox{green}{\color[gray]{0.75}FO}%
\colorbox{green}{\color[gray]{0.75}FO}%
\colorbox{green}{\color[gray]{0.75}FO}%
\colorbox{green}{\color[gray]{0.75}FO}%
\colorbox{green}{\color[gray]{0.75}FO}%
\colorbox{green}{\color[gray]{0.75}FO}%
\colorbox{green}{\color[gray]{0.75}FO}%
\colorbox{green}{\color[gray]{0.75}FO}%
\colorbox{green}{\color[gray]{0.75}FO}%
\colorbox{green}{\color[gray]{0.75}FO}%
\colorbox{green}{\color[gray]{0.75}FO}%
\colorbox{green}{\color[gray]{0.75}FO}%
\colorbox{green}{\color[gray]{0.75}FO}%
\colorbox{green}{\color[gray]{0.75}FO}%
\colorbox{green}{\color[gray]{0.75}FO}%
\colorbox{green}{\color[gray]{0.75}FO}%
\colorbox{green}{\color[gray]{0.75}FO}%
\colorbox{green}{\color[gray]{0.75}FO}%
\colorbox{green}{\color[gray]{0.75}FO}%
\colorbox{green}{\color[gray]{0.75}FO}%
\colorbox{green}{\color[gray]{0.75}FO}%
\colorbox{green}{\color[gray]{0.75}FO}%
\colorbox{green}{\color[gray]{0.75}FO}%
\colorbox{green}{\color[gray]{0.75}FO}%
\colorbox{green}{\color[gray]{0.75}FO}%
\colorbox{green}{\color[gray]{0.75}FO}%
\colorbox{green}{\color[gray]{0.75}FO}%
\colorbox{green}{\color[gray]{0.75}FO}%
\colorbox{green}{\color[gray]{0.75}FO}%
\colorbox{green}{\color[gray]{0.75}FO}%
\colorbox{green}{\color[gray]{0.75}FO}%
\colorbox{green}{\color[gray]{0.75}FO}%
\colorbox{green}{\color[gray]{0.75}FO}%
\colorbox{green}{\color[gray]{0.75}FO}%
\colorbox{green}{\color[gray]{0.75}FO}%
\colorbox{green}{\color[gray]{0.75}FO}%
\colorbox{green}{\color[gray]{0.75}FO}%
\colorbox{green}{\color[gray]{0.75}FO}%
\colorbox{green}{\color[gray]{0.75}FO}%
\colorbox{green}{\color[gray]{0.75}FO}%
\colorbox{green}{\color[gray]{0.75}FO}%
\colorbox{green}{\color[gray]{0.75}FO}%
\colorbox{green}{\color[gray]{0.75}FO}%
\colorbox{green}{\color[gray]{0.75}FO}%
\colorbox{green}{\color[gray]{0.75}FO}%
\colorbox{green}{\color[gray]{0.75}FO}%
\colorbox{green}{\color[gray]{0.75}FO}%
\colorbox{green}{\color[gray]{0.75}FO}%
\colorbox{green}{\color[gray]{0.75}FO}%
\colorbox{green}{\color[gray]{0.75}FO}%
\colorbox{green}{\color[gray]{0.75}FO}%
\colorbox{green}{\color[gray]{0.75}FO}%
\colorbox{green}{\color[gray]{0.75}FO}%
\colorbox{green}{\color[gray]{0.75}FO}%
\colorbox{green}{\color[gray]{0.75}FO}%
\colorbox{green}{\color[gray]{0.75}FO}%
\colorbox{green}{\color[gray]{0.75}FO}%
\colorbox{green}{\color[gray]{0.75}FO}%
\colorbox{green}{\color[gray]{0.75}FO}%
\colorbox{green}{\color[gray]{0.75}FO}%
\colorbox{green}{\color[gray]{0.75}FO}%
\colorbox{green}{\color[gray]{0.75}FO}%
\colorbox{green}{\color[gray]{0.75}FO}%
\colorbox{green}{\color[gray]{0.75}FO}%
\colorbox{green}{\color[gray]{0.75}FO}%
\colorbox{green}{\color[gray]{0.75}FO}%
\colorbox{green}{\color[gray]{0.75}FO}%
\colorbox{green}{\color[gray]{0.75}FO}%
\colorbox{green}{\color[gray]{0.75}FO}%
\\
\colorbox{green}{\color[gray]{0.75}FO}%
\colorbox{green}{\color[gray]{0.75}FO}%
\colorbox{green}{\color[gray]{0.75}FO}%
\colorbox{green}{\color[gray]{0.75}FO}%
\colorbox{green}{\color[gray]{0.75}FO}%
\colorbox{green}{\color[gray]{0.75}FO}%
\colorbox{green}{\color[gray]{0.75}FO}%
\colorbox{green}{\color[gray]{0.75}FO}%
\colorbox{green}{\color[gray]{0.75}FO}%
\colorbox{green}{\color[gray]{0.75}FO}%
\colorbox{green}{\color[gray]{0.75}FO}%
\colorbox{green}{\color[gray]{0.75}FO}%
\colorbox{green}{\color[gray]{0.75}FO}%
\colorbox{green}{\color[gray]{0.75}FO}%
\colorbox{green}{\color[gray]{0.75}FO}%
\colorbox{green}{\color[gray]{0.75}FO}%
\colorbox{green}{\color[gray]{0.75}FO}%
\colorbox{green}{\color[gray]{0.75}FO}%
\colorbox{green}{\color[gray]{0.75}FO}%
\colorbox{green}{\color[gray]{0.75}FO}%
\colorbox{green}{\color[gray]{0.75}FO}%
\colorbox{green}{\color[gray]{0.75}FO}%
\colorbox{green}{\color[gray]{0.75}FO}%
\colorbox{green}{\color[gray]{0.75}FO}%
\colorbox{green}{\color[gray]{0.75}FO}%
\colorbox{green}{\color[gray]{0.75}FO}%
\colorbox{green}{\color[gray]{0.75}FO}%
\colorbox{green}{\color[gray]{0.75}FO}%
\colorbox{green}{\color[gray]{0.75}FO}%
\colorbox{green}{\color[gray]{0.75}FO}%
\colorbox{green}{\color[gray]{0.75}FO}%
\colorbox{green}{\color[gray]{0.75}FO}%
\colorbox{green}{\color[gray]{0.75}FO}%
\colorbox{green}{\color[gray]{0.75}FO}%
\colorbox{green}{\color[gray]{0.75}FO}%
\colorbox{green}{\color[gray]{0.75}FO}%
\colorbox{green}{\color[gray]{0.75}FO}%
\colorbox{green}{\color[gray]{0.75}FO}%
\colorbox{green}{\color[gray]{0.75}FO}%
\colorbox{green}{\color[gray]{0.75}FO}%
\colorbox{green}{\color[gray]{0.75}FO}%
\colorbox{green}{\color[gray]{0.75}FO}%
\colorbox{green}{\color[gray]{0.75}FO}%
\colorbox{green}{\color[gray]{0.75}FO}%
\colorbox{green}{\color[gray]{0.75}FO}%
\colorbox{green}{\color[gray]{0.75}FO}%
\colorbox{green}{\color[gray]{0.75}FO}%
\colorbox{green}{\color[gray]{0.75}FO}%
\colorbox{green}{\color[gray]{0.75}FO}%
\colorbox{green}{\color[gray]{0.75}FO}%
\colorbox{green}{\color[gray]{0.75}FO}%
\colorbox{green}{\color[gray]{0.75}FO}%
\colorbox{green}{\color[gray]{0.75}FO}%
\colorbox{green}{\color[gray]{0.75}FO}%
\colorbox{green}{\color[gray]{0.75}FO}%
\colorbox{green}{\color[gray]{0.75}FO}%
\colorbox{green}{\color[gray]{0.75}FO}%
\colorbox{green}{\color[gray]{0.75}FO}%
\colorbox{green}{\color[gray]{0.75}FO}%
\colorbox{green}{\color[gray]{0.75}FO}%
\colorbox{green}{\color[gray]{0.75}FO}%
\colorbox{green}{\color[gray]{0.75}FO}%
\colorbox{green}{\color[gray]{0.75}FO}%
\colorbox{green}{\color[gray]{0.75}FO}%
\colorbox{green}{\color[gray]{0.75}FO}%
\colorbox{green}{\color[gray]{0.75}FO}%
\colorbox{green}{\color[gray]{0.75}FO}%
\colorbox{green}{\color[gray]{0.75}FO}%
\colorbox{green}{\color[gray]{0.75}FO}%
\colorbox{green}{\color[gray]{0.75}FO}%
\colorbox{green}{\color[gray]{0.75}FO}%
\colorbox{green}{\color[gray]{0.75}FO}%
\colorbox{green}{\color[gray]{0.75}FO}%
\colorbox{green}{\color[gray]{0.75}FO}%
\colorbox{green}{\color[gray]{0.75}FO}%
\colorbox{green}{\color[gray]{0.75}FO}%
\colorbox{green}{\color[gray]{0.75}FO}%
\colorbox{green}{\color[gray]{0.75}FO}%
\colorbox{green}{\color[gray]{0.75}FO}%
\colorbox{green}{\color[gray]{0.75}FO}%
\colorbox{green}{\color[gray]{0.75}FO}%
\colorbox{green}{\color[gray]{0.75}FO}%
\colorbox{green}{\color[gray]{0.75}FO}%
\colorbox{green}{\color[gray]{0.75}FO}%
\colorbox{green}{\color[gray]{0.75}FO}%
\colorbox{green}{\color[gray]{0.75}FO}%
\colorbox{green}{\color[gray]{0.75}FO}%
\colorbox{green}{\color[gray]{0.75}FO}%
\colorbox{green}{\color[gray]{0.75}FO}%
\colorbox{green}{\color[gray]{0.75}FO}%
\colorbox{green}{\color[gray]{0.75}FO}%
\colorbox{green}{\color[gray]{0.75}FO}%
\colorbox{green}{\color[gray]{0.75}FO}%
\colorbox{green}{\color[gray]{0.75}FO}%
\colorbox{green}{\color[gray]{0.75}FO}%
\colorbox{green}{\color[gray]{0.75}FO}%
\colorbox{green}{\color[gray]{0.75}FO}%
\colorbox{green}{\color[gray]{0.75}FO}%
\colorbox{green}{\color[gray]{0.75}FO}%
\colorbox{green}{\color[gray]{0.75}FO}%
\\
\colorbox{green}{\color[gray]{0.75}FO}%
\colorbox{green}{\color[gray]{0.75}FO}%
\colorbox{green}{\color[gray]{0.75}FO}%
\colorbox{green}{\color[gray]{0.75}FO}%
\colorbox{green}{\color[gray]{0.75}FO}%
\colorbox{green}{\color[gray]{0.75}FO}%
\colorbox{green}{\color[gray]{0.75}FO}%
\colorbox{green}{\color[gray]{0.75}FO}%
\colorbox{green}{\color[gray]{0.75}FO}%
\colorbox{green}{\color[gray]{0.75}FO}%
\colorbox{green}{\color[gray]{0.75}FO}%
\colorbox{green}{\color[gray]{0.75}FO}%
\colorbox{green}{\color[gray]{0.75}FO}%
\colorbox{green}{\color[gray]{0.75}FO}%
\colorbox{green}{\color[gray]{0.75}FO}%
\colorbox{green}{\color[gray]{0.75}FO}%
\colorbox{green}{\color[gray]{0.75}FO}%
\colorbox{green}{\color[gray]{0.75}FO}%
\colorbox{green}{\color[gray]{0.75}FO}%
\colorbox{green}{\color[gray]{0.75}FO}%
\colorbox{green}{\color[gray]{0.75}FO}%
\colorbox{green}{\color[gray]{0.75}FO}%
\colorbox{green}{\color[gray]{0.75}FO}%
\colorbox{green}{\color[gray]{0.75}FO}%
\colorbox{green}{\color[gray]{0.75}FO}%
\colorbox{green}{\color[gray]{0.75}FO}%
\colorbox{green}{\color[gray]{0.75}FO}%
\colorbox{green}{\color[gray]{0.75}FO}%
\colorbox{green}{\color[gray]{0.75}FO}%
\colorbox{green}{\color[gray]{0.75}FO}%
\colorbox{green}{\color[gray]{0.75}FO}%
\colorbox{green}{\color[gray]{0.75}FO}%
\colorbox{green}{\color[gray]{0.75}FO}%
\colorbox{green}{\color[gray]{0.75}FO}%
\colorbox{green}{\color[gray]{0.75}FO}%
\colorbox{green}{\color[gray]{0.75}FO}%
\colorbox{green}{\color[gray]{0.75}FO}%
\colorbox{green}{\color[gray]{0.75}FO}%
\colorbox{green}{\color[gray]{0.75}FO}%
\colorbox{green}{\color[gray]{0.75}FO}%
\colorbox{green}{\color[gray]{0.75}FO}%
\colorbox{green}{\color[gray]{0.75}FO}%
\colorbox{green}{\color[gray]{0.75}FO}%
\colorbox{green}{\color[gray]{0.75}FO}%
\colorbox{green}{\color[gray]{0.75}FO}%
\colorbox{green}{\color[gray]{0.75}FO}%
\colorbox{green}{\color[gray]{0.75}FO}%
\colorbox{green}{\color[gray]{0.75}FO}%
\colorbox{green}{\color[gray]{0.75}FO}%
\colorbox{green}{\color[gray]{0.75}FO}%
\colorbox{green}{\color[gray]{0.75}FO}%
\colorbox{green}{\color[gray]{0.75}FO}%
\colorbox{green}{\color[gray]{0.75}FO}%
\colorbox{green}{\color[gray]{0.75}FO}%
\colorbox{green}{\color[gray]{0.75}FO}%
\colorbox{green}{\color[gray]{0.75}FO}%
\colorbox{green}{\color[gray]{0.75}FO}%
\colorbox{green}{\color[gray]{0.75}FO}%
\colorbox{green}{\color[gray]{0.75}FO}%
\colorbox{green}{\color[gray]{0.75}FO}%
\colorbox{green}{\color[gray]{0.75}FO}%
\colorbox{green}{\color[gray]{0.75}FO}%
\colorbox{green}{\color[gray]{0.75}FO}%
\colorbox{green}{\color[gray]{0.75}FO}%
\colorbox{green}{\color[gray]{0.75}FO}%
\colorbox{green}{\color[gray]{0.75}FO}%
\colorbox{green}{\color[gray]{0.75}FO}%
\colorbox{green}{\color[gray]{0.75}FO}%
\colorbox{green}{\color[gray]{0.75}FO}%
\colorbox{green}{\color[gray]{0.75}FO}%
\colorbox{green}{\color[gray]{0.75}FO}%
\colorbox{green}{\color[gray]{0.75}FO}%
\colorbox{green}{\color[gray]{0.75}FO}%
\colorbox{green}{\color[gray]{0.75}FO}%
\colorbox{green}{\color[gray]{0.75}FO}%
\colorbox{green}{\color[gray]{0.75}FO}%
\colorbox{green}{\color[gray]{0.75}FO}%
\colorbox{green}{\color[gray]{0.75}FO}%
\colorbox{green}{\color[gray]{0.75}FO}%
\colorbox{green}{\color[gray]{0.75}FO}%
\colorbox{green}{\color[gray]{0.75}FO}%
\colorbox{green}{\color[gray]{0.75}FO}%
\colorbox{green}{\color[gray]{0.75}FO}%
\colorbox{green}{\color[gray]{0.75}FO}%
\colorbox{green}{\color[gray]{0.75}FO}%
\colorbox{green}{\color[gray]{0.75}FO}%
\colorbox{green}{\color[gray]{0.75}FO}%
\colorbox{green}{\color[gray]{0.75}FO}%
\colorbox{green}{\color[gray]{0.75}FO}%
\colorbox{green}{\color[gray]{0.75}FO}%
\colorbox{green}{\color[gray]{0.75}FO}%
\colorbox{green}{\color[gray]{0.75}FO}%
\colorbox{green}{\color[gray]{0.75}FO}%
\colorbox{green}{\color[gray]{0.75}FO}%
\colorbox{green}{\color[gray]{0.75}FO}%
\colorbox{green}{\color[gray]{0.75}FO}%
\colorbox{green}{\color[gray]{0.75}FO}%
\colorbox{green}{\color[gray]{0.75}FO}%
\colorbox{green}{\color[gray]{0.75}FO}%
\colorbox{green}{\color[gray]{0.75}FO}%
\\
\colorbox{green}{\color[gray]{0.75}FO}%
\colorbox{green}{\color[gray]{0.75}FO}%
\colorbox{green}{\color[gray]{0.75}FO}%
\colorbox{green}{\color[gray]{0.75}FO}%
\colorbox{green}{\color[gray]{0.75}FO}%
\colorbox{green}{\color[gray]{0.75}FO}%
\colorbox{green}{\color[gray]{0.75}FO}%
\colorbox{green}{\color[gray]{0.75}FO}%
\colorbox{green}{\color[gray]{0.75}FO}%
\colorbox{green}{\color[gray]{0.75}FO}%
\colorbox{green}{\color[gray]{0.75}FO}%
\colorbox{green}{\color[gray]{0.75}FO}%
\colorbox{green}{\color[gray]{0.75}FO}%
\colorbox{green}{\color[gray]{0.75}FO}%
\colorbox{green}{\color[gray]{0.75}FO}%
\colorbox{green}{\color[gray]{0.75}FO}%
\colorbox{green}{\color[gray]{0.75}FO}%
\colorbox{green}{\color[gray]{0.75}FO}%
\colorbox{green}{\color[gray]{0.75}FO}%
\colorbox{green}{\color[gray]{0.75}FO}%
\colorbox{green}{\color[gray]{0.75}FO}%
\colorbox{green}{\color[gray]{0.75}FO}%
\colorbox{green}{\color[gray]{0.75}FO}%
\colorbox{green}{\color[gray]{0.75}FO}%
\colorbox{green}{\color[gray]{0.75}FO}%
\colorbox{green}{\color[gray]{0.75}FO}%
\colorbox{green}{\color[gray]{0.75}FO}%
\colorbox{green}{\color[gray]{0.75}FO}%
\colorbox{green}{\color[gray]{0.75}FO}%
\colorbox{green}{\color[gray]{0.75}FO}%
\colorbox{green}{\color[gray]{0.75}FO}%
\colorbox{green}{\color[gray]{0.75}FO}%
\colorbox{green}{\color[gray]{0.75}FO}%
\colorbox{green}{\color[gray]{0.75}FO}%
\colorbox{green}{\color[gray]{0.75}FO}%
\colorbox{green}{\color[gray]{0.75}FO}%
\colorbox{green}{\color[gray]{0.75}FO}%
\colorbox{green}{\color[gray]{0.75}FO}%
\colorbox{green}{\color[gray]{0.75}FO}%
\colorbox{green}{\color[gray]{0.75}FO}%
\colorbox{green}{\color[gray]{0.75}FO}%
\colorbox{green}{\color[gray]{0.75}FO}%
\colorbox{green}{\color[gray]{0.75}FO}%
\colorbox{green}{\color[gray]{0.75}FO}%
\colorbox{green}{\color[gray]{0.75}FO}%
\colorbox{green}{\color[gray]{0.75}FO}%
\colorbox{green}{\color[gray]{0.75}FO}%
\colorbox{green}{\color[gray]{0.75}FO}%
\colorbox{green}{\color[gray]{0.75}FO}%
\colorbox{green}{\color[gray]{0.75}FO}%
\colorbox{green}{\color[gray]{0.75}FO}%
\colorbox{green}{\color[gray]{0.75}FO}%
\colorbox{green}{\color[gray]{0.75}FO}%
\colorbox{green}{\color[gray]{0.75}FO}%
\colorbox{green}{\color[gray]{0.75}FO}%
\colorbox{green}{\color[gray]{0.75}FO}%
\colorbox{green}{\color[gray]{0.75}FO}%
\colorbox{green}{\color[gray]{0.75}FO}%
\colorbox{green}{\color[gray]{0.75}FO}%
\colorbox{green}{\color[gray]{0.75}FO}%
\colorbox{green}{\color[gray]{0.75}FO}%
\colorbox{green}{\color[gray]{0.75}FO}%
\colorbox{green}{\color[gray]{0.75}FO}%
\colorbox{green}{\color[gray]{0.75}FO}%
\colorbox{green}{\color[gray]{0.75}FO}%
\colorbox{green}{\color[gray]{0.75}FO}%
\colorbox{green}{\color[gray]{0.75}FO}%
\colorbox{green}{\color[gray]{0.75}FO}%
\colorbox{green}{\color[gray]{0.75}FO}%
\colorbox{green}{\color[gray]{0.75}FO}%
\colorbox{green}{\color[gray]{0.75}FO}%
\colorbox{green}{\color[gray]{0.75}FO}%
\colorbox{green}{\color[gray]{0.75}FO}%
\colorbox{green}{\color[gray]{0.75}FO}%
\colorbox{green}{\color[gray]{0.75}FO}%
\colorbox{green}{\color[gray]{0.75}FO}%
\colorbox{green}{\color[gray]{0.75}FO}%
\colorbox{green}{\color[gray]{0.75}FO}%
\colorbox{green}{\color[gray]{0.75}FO}%
\colorbox{green}{\color[gray]{0.75}FO}%
\colorbox{green}{\color[gray]{0.75}FO}%
\colorbox{green}{\color[gray]{0.75}FO}%
\colorbox{green}{\color[gray]{0.75}FO}%
\colorbox{green}{\color[gray]{0.75}FO}%
\colorbox{green}{\color[gray]{0.75}FO}%
\colorbox{green}{\color[gray]{0.75}FO}%
\colorbox{green}{\color[gray]{0.75}FO}%
\colorbox{green}{\color[gray]{0.75}FO}%
\colorbox{green}{\color[gray]{0.75}FO}%
\colorbox{green}{\color[gray]{0.75}FO}%
\colorbox{green}{\color[gray]{0.75}FO}%
\colorbox{green}{\color[gray]{0.75}FO}%
\colorbox{green}{\color[gray]{0.75}FO}%
\colorbox{green}{\color[gray]{0.75}FO}%
\colorbox{green}{\color[gray]{0.75}FO}%
\colorbox{green}{\color[gray]{0.75}FO}%
\colorbox{green}{\color[gray]{0.75}FO}%
\colorbox{green}{\color[gray]{0.75}FO}%
\colorbox{green}{\color[gray]{0.75}FO}%
\colorbox{green}{\color[gray]{0.75}FO}%
\\
\colorbox{green}{\color[gray]{0.75}FO}%
\colorbox{green}{\color[gray]{0.75}FO}%
\colorbox{green}{\color[gray]{0.75}FO}%
\colorbox{green}{\color[gray]{0.75}FO}%
\colorbox{green}{\color[gray]{0.75}FO}%
\colorbox{green}{\color[gray]{0.75}FO}%
\colorbox{green}{\color[gray]{0.75}FO}%
\colorbox{green}{\color[gray]{0.75}FO}%
\colorbox{green}{\color[gray]{0.75}FO}%
\colorbox{green}{\color[gray]{0.75}FO}%
\colorbox{green}{\color[gray]{0.75}FO}%
\colorbox{green}{\color[gray]{0.75}FO}%
\colorbox{green}{\color[gray]{0.75}FO}%
\colorbox{green}{\color[gray]{0.75}FO}%
\colorbox{green}{\color[gray]{0.75}FO}%
\colorbox{green}{\color[gray]{0.75}FO}%
\colorbox{green}{\color[gray]{0.75}FO}%
\colorbox{green}{\color[gray]{0.75}FO}%
\colorbox{green}{\color[gray]{0.75}FO}%
\colorbox{green}{\color[gray]{0.75}FO}%
\colorbox{green}{\color[gray]{0.75}FO}%
\colorbox{green}{\color[gray]{0.75}FO}%
\colorbox{green}{\color[gray]{0.75}FO}%
\colorbox{green}{\color[gray]{0.75}FO}%
\colorbox{green}{\color[gray]{0.75}FO}%
\colorbox{green}{\color[gray]{0.75}FO}%
\colorbox{green}{\color[gray]{0.75}FO}%
\colorbox{green}{\color[gray]{0.75}FO}%
\colorbox{green}{\color[gray]{0.75}FO}%
\colorbox{green}{\color[gray]{0.75}FO}%
\colorbox{green}{\color[gray]{0.75}FO}%
\colorbox{green}{\color[gray]{0.75}FO}%
\colorbox{green}{\color[gray]{0.75}FO}%
\colorbox{green}{\color[gray]{0.75}FO}%
\colorbox{green}{\color[gray]{0.75}FO}%
\colorbox{green}{\color[gray]{0.75}FO}%
\colorbox{green}{\color[gray]{0.75}FO}%
\colorbox{green}{\color[gray]{0.75}FO}%
\colorbox{green}{\color[gray]{0.75}FO}%
\colorbox{green}{\color[gray]{0.75}FO}%
\colorbox{green}{\color[gray]{0.75}FO}%
\colorbox{green}{\color[gray]{0.75}FO}%
\colorbox{green}{\color[gray]{0.75}FO}%
\colorbox{green}{\color[gray]{0.75}FO}%
\colorbox{green}{\color[gray]{0.75}FO}%
\colorbox{green}{\color[gray]{0.75}FO}%
\colorbox{green}{\color[gray]{0.75}FO}%
\colorbox{green}{\color[gray]{0.75}FO}%
\colorbox{green}{\color[gray]{0.75}FO}%
\colorbox{green}{\color[gray]{0.75}FO}%
\colorbox{green}{\color[gray]{0.75}FO}%
\colorbox{green}{\color[gray]{0.75}FO}%
\colorbox{green}{\color[gray]{0.75}FO}%
\colorbox{green}{\color[gray]{0.75}FO}%
\colorbox{green}{\color[gray]{0.75}FO}%
\colorbox{green}{\color[gray]{0.75}FO}%
\colorbox{green}{\color[gray]{0.75}FO}%
\colorbox{green}{\color[gray]{0.75}FO}%
\colorbox{green}{\color[gray]{0.75}FO}%
\colorbox{green}{\color[gray]{0.75}FO}%
\colorbox{green}{\color[gray]{0.75}FO}%
\colorbox{green}{\color[gray]{0.75}FO}%
\colorbox{green}{\color[gray]{0.75}FO}%
\colorbox{green}{\color[gray]{0.75}FO}%
\colorbox{green}{\color[gray]{0.75}FO}%
\colorbox{green}{\color[gray]{0.75}FO}%
\colorbox{green}{\color[gray]{0.75}FO}%
\colorbox{green}{\color[gray]{0.75}FO}%
\colorbox{green}{\color[gray]{0.75}FO}%
\colorbox{green}{\color[gray]{0.75}FO}%
\colorbox{green}{\color[gray]{0.75}FO}%
\colorbox{green}{\color[gray]{0.75}FO}%
\colorbox{green}{\color[gray]{0.75}FO}%
\colorbox{green}{\color[gray]{0.75}FO}%
\colorbox{green}{\color[gray]{0.75}FO}%
\colorbox{green}{\color[gray]{0.75}FO}%
\colorbox{green}{\color[gray]{0.75}FO}%
\colorbox{green}{\color[gray]{0.75}FO}%
\colorbox{green}{\color[gray]{0.75}FO}%
\colorbox{green}{\color[gray]{0.75}FO}%
\colorbox{green}{\color[gray]{0.75}FO}%
\colorbox{green}{\color[gray]{0.75}FO}%
\colorbox{green}{\color[gray]{0.75}FO}%
\colorbox{green}{\color[gray]{0.75}FO}%
\colorbox{green}{\color[gray]{0.75}FO}%
\colorbox{green}{\color[gray]{0.75}FO}%
\colorbox{green}{\color[gray]{0.75}FO}%
\colorbox{green}{\color[gray]{0.75}FO}%
\colorbox{green}{\color[gray]{0.75}FO}%
\colorbox{green}{\color[gray]{0.75}FO}%
\colorbox{green}{\color[gray]{0.75}FO}%
\colorbox{green}{\color[gray]{0.75}FO}%
\colorbox{green}{\color[gray]{0.75}FO}%
\colorbox{green}{\color[gray]{0.75}FO}%
\colorbox{green}{\color[gray]{0.75}FO}%
\colorbox{green}{\color[gray]{0.75}FO}%
\colorbox{green}{\color[gray]{0.75}FO}%
\colorbox{green}{\color[gray]{0.75}FO}%
\colorbox{green}{\color[gray]{0.75}FO}%
\colorbox{green}{\color[gray]{0.75}FO}%
\\
\colorbox{green}{\color[gray]{0.75}FO}%
\colorbox{green}{\color[gray]{0.75}FO}%
\colorbox{green}{\color[gray]{0.75}FO}%
\colorbox{green}{\color[gray]{0.75}FO}%
\colorbox{green}{\color[gray]{0.75}FO}%
\colorbox{green}{\color[gray]{0.75}FO}%
\colorbox{green}{\color[gray]{0.75}FO}%
\colorbox{green}{\color[gray]{0.75}FO}%
\colorbox{green}{\color[gray]{0.75}FO}%
\colorbox{green}{\color[gray]{0.75}FO}%
\colorbox{green}{\color[gray]{0.75}FO}%
\colorbox{green}{\color[gray]{0.75}FO}%
\colorbox{green}{\color[gray]{0.75}FO}%
\colorbox{green}{\color[gray]{0.75}FO}%
\colorbox{green}{\color[gray]{0.75}FO}%
\colorbox{green}{\color[gray]{0.75}FO}%
\colorbox{green}{\color[gray]{0.75}FO}%
\colorbox{green}{\color[gray]{0.75}FO}%
\colorbox{green}{\color[gray]{0.75}FO}%
\colorbox{green}{\color[gray]{0.75}FO}%
\colorbox{green}{\color[gray]{0.75}FO}%
\colorbox{green}{\color[gray]{0.75}FO}%
\colorbox{green}{\color[gray]{0.75}FO}%
\colorbox{green}{\color[gray]{0.75}FO}%
\colorbox{green}{\color[gray]{0.75}FO}%
\colorbox{green}{\color[gray]{0.75}FO}%
\colorbox{green}{\color[gray]{0.75}FO}%
\colorbox{green}{\color[gray]{0.75}FO}%
\colorbox{green}{\color[gray]{0.75}FO}%
\colorbox{green}{\color[gray]{0.75}FO}%
\colorbox{green}{\color[gray]{0.75}FO}%
\colorbox{green}{\color[gray]{0.75}FO}%
\colorbox{green}{\color[gray]{0.75}FO}%
\colorbox{green}{\color[gray]{0.75}FO}%
\colorbox{green}{\color[gray]{0.75}FO}%
\colorbox{green}{\color[gray]{0.75}FO}%
\colorbox{green}{\color[gray]{0.75}FO}%
\colorbox{green}{\color[gray]{0.75}FO}%
\colorbox{green}{\color[gray]{0.75}FO}%
\colorbox{green}{\color[gray]{0.75}FO}%
\colorbox{green}{\color[gray]{0.75}FO}%
\colorbox{green}{\color[gray]{0.75}FO}%
\colorbox{green}{\color[gray]{0.75}FO}%
\colorbox{green}{\color[gray]{0.75}FO}%
\colorbox{green}{\color[gray]{0.75}FO}%
\colorbox{green}{\color[gray]{0.75}FO}%
\colorbox{green}{\color[gray]{0.75}FO}%
\colorbox{green}{\color[gray]{0.75}FO}%
\colorbox{green}{\color[gray]{0.75}FO}%
\colorbox{green}{\color[gray]{0.75}FO}%
\colorbox{green}{\color[gray]{0.75}FO}%
\colorbox{green}{\color[gray]{0.75}FO}%
\colorbox{green}{\color[gray]{0.75}FO}%
\colorbox{green}{\color[gray]{0.75}FO}%
\colorbox{green}{\color[gray]{0.75}FO}%
\colorbox{green}{\color[gray]{0.75}FO}%
\colorbox{green}{\color[gray]{0.75}FO}%
\colorbox{green}{\color[gray]{0.75}FO}%
\colorbox{green}{\color[gray]{0.75}FO}%
\colorbox{green}{\color[gray]{0.75}FO}%
\colorbox{green}{\color[gray]{0.75}FO}%
\colorbox{green}{\color[gray]{0.75}FO}%
\colorbox{green}{\color[gray]{0.75}FO}%
\colorbox{green}{\color[gray]{0.75}FO}%
\colorbox{green}{\color[gray]{0.75}FO}%
\colorbox{green}{\color[gray]{0.75}FO}%
\colorbox{green}{\color[gray]{0.75}FO}%
\colorbox{green}{\color[gray]{0.75}FO}%
\colorbox{green}{\color[gray]{0.75}FO}%
\colorbox{green}{\color[gray]{0.75}FO}%
\colorbox{green}{\color[gray]{0.75}FO}%
\colorbox{green}{\color[gray]{0.75}FO}%
\colorbox{green}{\color[gray]{0.75}FO}%
\colorbox{green}{\color[gray]{0.75}FO}%
\colorbox{green}{\color[gray]{0.75}FO}%
\colorbox{green}{\color[gray]{0.75}FO}%
\colorbox{green}{\color[gray]{0.75}FO}%
\colorbox{green}{\color[gray]{0.75}FO}%
\colorbox{green}{\color[gray]{0.75}FO}%
\colorbox{green}{\color[gray]{0.75}FO}%
\colorbox{green}{\color[gray]{0.75}FO}%
\colorbox{green}{\color[gray]{0.75}FO}%
\colorbox{green}{\color[gray]{0.75}FO}%
\colorbox{green}{\color[gray]{0.75}FO}%
\colorbox{green}{\color[gray]{0.75}FO}%
\colorbox{green}{\color[gray]{0.75}FO}%
\colorbox{green}{\color[gray]{0.75}FO}%
\colorbox{green}{\color[gray]{0.75}FO}%
\colorbox{green}{\color[gray]{0.75}FO}%
\colorbox{green}{\color[gray]{0.75}FO}%
\colorbox{green}{\color[gray]{0.75}FO}%
\colorbox{green}{\color[gray]{0.75}FO}%
\colorbox{green}{\color[gray]{0.75}FO}%
\colorbox{green}{\color[gray]{0.75}FO}%
\colorbox{green}{\color[gray]{0.75}FO}%
\colorbox{green}{\color[gray]{0.75}FO}%
\colorbox{green}{\color[gray]{0.75}FO}%
\colorbox{green}{\color[gray]{0.75}FO}%
\colorbox{green}{\color[gray]{0.75}FO}%
\colorbox{green}{\color[gray]{0.75}FO}%
\\
\colorbox{green}{\color[gray]{0.75}FO}%
\colorbox{green}{\color[gray]{0.75}FO}%
\colorbox{green}{\color[gray]{0.75}FO}%
\colorbox{green}{\color[gray]{0.75}FO}%
\colorbox{green}{\color[gray]{0.75}FO}%
\colorbox{green}{\color[gray]{0.75}FO}%
\colorbox{green}{\color[gray]{0.75}FO}%
\colorbox{green}{\color[gray]{0.75}FO}%
\colorbox{green}{\color[gray]{0.75}FO}%
\colorbox{green}{\color[gray]{0.75}FO}%
\colorbox{green}{\color[gray]{0.75}FO}%
\colorbox{green}{\color[gray]{0.75}FO}%
\colorbox{green}{\color[gray]{0.75}FO}%
\colorbox{green}{\color[gray]{0.75}FO}%
\colorbox{green}{\color[gray]{0.75}FO}%
\colorbox{green}{\color[gray]{0.75}FO}%
\colorbox{green}{\color[gray]{0.75}FO}%
\colorbox{green}{\color[gray]{0.75}FO}%
\colorbox{green}{\color[gray]{0.75}FO}%
\colorbox{green}{\color[gray]{0.75}FO}%
\colorbox{green}{\color[gray]{0.75}FO}%
\colorbox{green}{\color[gray]{0.75}FO}%
\colorbox{green}{\color[gray]{0.75}FO}%
\colorbox{green}{\color[gray]{0.75}FO}%
\colorbox{green}{\color[gray]{0.75}FO}%
\colorbox{green}{\color[gray]{0.75}FO}%
\colorbox{green}{\color[gray]{0.75}FO}%
\colorbox{green}{\color[gray]{0.75}FO}%
\colorbox{green}{\color[gray]{0.75}FO}%
\colorbox{green}{\color[gray]{0.75}FO}%
\colorbox{green}{\color[gray]{0.75}FO}%
\colorbox{green}{\color[gray]{0.75}FO}%
\colorbox{green}{\color[gray]{0.75}FO}%
\colorbox{green}{\color[gray]{0.75}FO}%
\colorbox{green}{\color[gray]{0.75}FO}%
\colorbox{green}{\color[gray]{0.75}FO}%
\colorbox{green}{\color[gray]{0.75}FO}%
\colorbox{green}{\color[gray]{0.75}FO}%
\colorbox{green}{\color[gray]{0.75}FO}%
\colorbox{green}{\color[gray]{0.75}FO}%
\colorbox{green}{\color[gray]{0.75}FO}%
\colorbox{green}{\color[gray]{0.75}FO}%
\colorbox{green}{\color[gray]{0.75}FO}%
\colorbox{green}{\color[gray]{0.75}FO}%
\colorbox{green}{\color[gray]{0.75}FO}%
\colorbox{green}{\color[gray]{0.75}FO}%
\colorbox{green}{\color[gray]{0.75}FO}%
\colorbox{green}{\color[gray]{0.75}FO}%
\colorbox{green}{\color[gray]{0.75}FO}%
\colorbox{green}{\color[gray]{0.75}FO}%
\colorbox{green}{\color[gray]{0.75}FO}%
\colorbox{green}{\color[gray]{0.75}FO}%
\colorbox{green}{\color[gray]{0.75}FO}%
\colorbox{green}{\color[gray]{0.75}FO}%
\colorbox{green}{\color[gray]{0.75}FO}%
\colorbox{green}{\color[gray]{0.75}FO}%
\colorbox{green}{\color[gray]{0.75}FO}%
\colorbox{green}{\color[gray]{0.75}FO}%
\colorbox{green}{\color[gray]{0.75}FO}%
\colorbox{green}{\color[gray]{0.75}FO}%
\colorbox{green}{\color[gray]{0.75}FO}%
\colorbox{green}{\color[gray]{0.75}FO}%
\colorbox{green}{\color[gray]{0.75}FO}%
\colorbox{green}{\color[gray]{0.75}FO}%
\colorbox{green}{\color[gray]{0.75}FO}%
\colorbox{green}{\color[gray]{0.75}FO}%
\colorbox{green}{\color[gray]{0.75}FO}%
\colorbox{green}{\color[gray]{0.75}FO}%
\colorbox{green}{\color[gray]{0.75}FO}%
\colorbox{green}{\color[gray]{0.75}FO}%
\colorbox{green}{\color[gray]{0.75}FO}%
\colorbox{green}{\color[gray]{0.75}FO}%
\colorbox{green}{\color[gray]{0.75}FO}%
\colorbox{green}{\color[gray]{0.75}FO}%
\colorbox{green}{\color[gray]{0.75}FO}%
\colorbox{green}{\color[gray]{0.75}FO}%
\colorbox{green}{\color[gray]{0.75}FO}%
\colorbox{green}{\color[gray]{0.75}FO}%
\colorbox{green}{\color[gray]{0.75}FO}%
\colorbox{green}{\color[gray]{0.75}FO}%
\colorbox{green}{\color[gray]{0.75}FO}%
\colorbox{green}{\color[gray]{0.75}FO}%
\colorbox{green}{\color[gray]{0.75}FO}%
\colorbox{green}{\color[gray]{0.75}FO}%
\colorbox{green}{\color[gray]{0.75}FO}%
\colorbox{green}{\color[gray]{0.75}FO}%
\colorbox{green}{\color[gray]{0.75}FO}%
\colorbox{green}{\color[gray]{0.75}FO}%
\colorbox{green}{\color[gray]{0.75}FO}%
\colorbox{green}{\color[gray]{0.75}FO}%
\colorbox{green}{\color[gray]{0.75}FO}%
\colorbox{green}{\color[gray]{0.75}FO}%
\colorbox{green}{\color[gray]{0.75}FO}%
\colorbox{green}{\color[gray]{0.75}FO}%
\colorbox{green}{\color[gray]{0.75}FO}%
\colorbox{green}{\color[gray]{0.75}FO}%
\colorbox{green}{\color[gray]{0.75}FO}%
\colorbox{green}{\color[gray]{0.75}FO}%
\colorbox{green}{\color[gray]{0.75}FO}%
\colorbox{green}{\color[gray]{0.75}FO}%
\\
\colorbox{green}{\color[gray]{0.75}FO}%
\colorbox{green}{\color[gray]{0.75}FO}%
\colorbox{green}{\color[gray]{0.75}FO}%
\colorbox{green}{\color[gray]{0.75}FO}%
\colorbox{green}{\color[gray]{0.75}FO}%
\colorbox{green}{\color[gray]{0.75}FO}%
\colorbox{green}{\color[gray]{0.75}FO}%
\colorbox{green}{\color[gray]{0.75}FO}%
\colorbox{green}{\color[gray]{0.75}FO}%
\colorbox{green}{\color[gray]{0.75}FO}%
\colorbox{green}{\color[gray]{0.75}FO}%
\colorbox{green}{\color[gray]{0.75}FO}%
\colorbox{green}{\color[gray]{0.75}FO}%
\colorbox{green}{\color[gray]{0.75}FO}%
\colorbox{green}{\color[gray]{0.75}FO}%
\colorbox{green}{\color[gray]{0.75}FO}%
\colorbox{green}{\color[gray]{0.75}FO}%
\colorbox{green}{\color[gray]{0.75}FO}%
\colorbox{green}{\color[gray]{0.75}FO}%
\colorbox{green}{\color[gray]{0.75}FO}%
\colorbox{green}{\color[gray]{0.75}FO}%
\colorbox{green}{\color[gray]{0.75}FO}%
\colorbox{green}{\color[gray]{0.75}FO}%
\colorbox{green}{\color[gray]{0.75}FO}%
\colorbox{green}{\color[gray]{0.75}FO}%
\colorbox{green}{\color[gray]{0.75}FO}%
\colorbox{green}{\color[gray]{0.75}FO}%
\colorbox{green}{\color[gray]{0.75}FO}%
\colorbox{green}{\color[gray]{0.75}FO}%
\colorbox{green}{\color[gray]{0.75}FO}%
\colorbox{green}{\color[gray]{0.75}FO}%
\colorbox{green}{\color[gray]{0.75}FO}%
\colorbox{green}{\color[gray]{0.75}FO}%
\colorbox{green}{\color[gray]{0.75}FO}%
\colorbox{green}{\color[gray]{0.75}FO}%
\colorbox{green}{\color[gray]{0.75}FO}%
\colorbox{green}{\color[gray]{0.75}FO}%
\colorbox{green}{\color[gray]{0.75}FO}%
\colorbox{green}{\color[gray]{0.75}FO}%
\colorbox{green}{\color[gray]{0.75}FO}%
\colorbox{green}{\color[gray]{0.75}FO}%
\colorbox{green}{\color[gray]{0.75}FO}%
\colorbox{green}{\color[gray]{0.75}FO}%
\colorbox{green}{\color[gray]{0.75}FO}%
\colorbox{green}{\color[gray]{0.75}FO}%
\colorbox{green}{\color[gray]{0.75}FO}%
\colorbox{green}{\color[gray]{0.75}FO}%
\colorbox{green}{\color[gray]{0.75}FO}%
\colorbox{green}{\color[gray]{0.75}FO}%
\colorbox{green}{\color[gray]{0.75}FO}%
\colorbox{green}{\color[gray]{0.75}FO}%
\colorbox{green}{\color[gray]{0.75}FO}%
\colorbox{green}{\color[gray]{0.75}FO}%
\colorbox{green}{\color[gray]{0.75}FO}%
\colorbox{green}{\color[gray]{0.75}FO}%
\colorbox{green}{\color[gray]{0.75}FO}%
\colorbox{green}{\color[gray]{0.75}FO}%
\colorbox{green}{\color[gray]{0.75}FO}%
\colorbox{green}{\color[gray]{0.75}FO}%
\colorbox{green}{\color[gray]{0.75}FO}%
\colorbox{green}{\color[gray]{0.75}FO}%
\colorbox{green}{\color[gray]{0.75}FO}%
\colorbox{green}{\color[gray]{0.75}FO}%
\colorbox{green}{\color[gray]{0.75}FO}%
\colorbox{green}{\color[gray]{0.75}FO}%
\colorbox{green}{\color[gray]{0.75}FO}%
\colorbox{green}{\color[gray]{0.75}FO}%
\colorbox{green}{\color[gray]{0.75}FO}%
\colorbox{green}{\color[gray]{0.75}FO}%
\colorbox{green}{\color[gray]{0.75}FO}%
\colorbox{green}{\color[gray]{0.75}FO}%
\colorbox{green}{\color[gray]{0.75}FO}%
\colorbox{green}{\color[gray]{0.75}FO}%
\colorbox{green}{\color[gray]{0.75}FO}%
\colorbox{green}{\color[gray]{0.75}FO}%
\colorbox{green}{\color[gray]{0.75}FO}%
\colorbox{green}{\color[gray]{0.75}FO}%
\colorbox{green}{\color[gray]{0.75}FO}%
\colorbox{green}{\color[gray]{0.75}FO}%
\colorbox{green}{\color[gray]{0.75}FO}%
\colorbox{green}{\color[gray]{0.75}FO}%
\colorbox{green}{\color[gray]{0.75}FO}%
\colorbox{green}{\color[gray]{0.75}FO}%
\colorbox{green}{\color[gray]{0.75}FO}%
\colorbox{green}{\color[gray]{0.75}FO}%
\colorbox{green}{\color[gray]{0.75}FO}%
\colorbox{green}{\color[gray]{0.75}FO}%
\colorbox{green}{\color[gray]{0.75}FO}%
\colorbox{green}{\color[gray]{0.75}FO}%
\colorbox{green}{\color[gray]{0.75}FO}%
\colorbox{green}{\color[gray]{0.75}FO}%
\colorbox{green}{\color[gray]{0.75}FO}%
\colorbox{green}{\color[gray]{0.75}FO}%
\colorbox{green}{\color[gray]{0.75}FO}%
\colorbox{green}{\color[gray]{0.75}FO}%
\colorbox{green}{\color[gray]{0.75}FO}%
\colorbox{green}{\color[gray]{0.75}FO}%
\colorbox{green}{\color[gray]{0.75}FO}%
\colorbox{green}{\color[gray]{0.75}FO}%
\colorbox{green}{\color[gray]{0.75}FO}%
\\
\colorbox{green}{\color[gray]{0.75}FO}%
\colorbox{green}{\color[gray]{0.75}FO}%
\colorbox{green}{\color[gray]{0.75}FO}%
\colorbox{green}{\color[gray]{0.75}FO}%
\colorbox{green}{\color[gray]{0.75}FO}%
\colorbox{green}{\color[gray]{0.75}FO}%
\colorbox{green}{\color[gray]{0.75}FO}%
\colorbox{green}{\color[gray]{0.75}FO}%
\colorbox{green}{\color[gray]{0.75}FO}%
\colorbox{green}{\color[gray]{0.75}FO}%
\colorbox{green}{\color[gray]{0.75}FO}%
\colorbox{green}{\color[gray]{0.75}FO}%
\colorbox{green}{\color[gray]{0.75}FO}%
\colorbox{green}{\color[gray]{0.75}FO}%
\colorbox{green}{\color[gray]{0.75}FO}%
\colorbox{green}{\color[gray]{0.75}FO}%
\colorbox{green}{\color[gray]{0.75}FO}%
\colorbox{green}{\color[gray]{0.75}FO}%
\colorbox{green}{\color[gray]{0.75}FO}%
\colorbox{green}{\color[gray]{0.75}FO}%
\colorbox{green}{\color[gray]{0.75}FO}%
\colorbox{green}{\color[gray]{0.75}FO}%
\colorbox{green}{\color[gray]{0.75}FO}%
\colorbox{green}{\color[gray]{0.75}FO}%
\colorbox{green}{\color[gray]{0.75}FO}%
\colorbox{green}{\color[gray]{0.75}FO}%
\colorbox{green}{\color[gray]{0.75}FO}%
\colorbox{green}{\color[gray]{0.75}FO}%
\colorbox{green}{\color[gray]{0.75}FO}%
\colorbox{green}{\color[gray]{0.75}FO}%
\colorbox{green}{\color[gray]{0.75}FO}%
\colorbox{green}{\color[gray]{0.75}FO}%
\colorbox{green}{\color[gray]{0.75}FO}%
\colorbox{green}{\color[gray]{0.75}FO}%
\colorbox{green}{\color[gray]{0.75}FO}%
\colorbox{green}{\color[gray]{0.75}FO}%
\colorbox{green}{\color[gray]{0.75}FO}%
\colorbox{green}{\color[gray]{0.75}FO}%
\colorbox{green}{\color[gray]{0.75}FO}%
\colorbox{green}{\color[gray]{0.75}FO}%
\colorbox{green}{\color[gray]{0.75}FO}%
\colorbox{green}{\color[gray]{0.75}FO}%
\colorbox{green}{\color[gray]{0.75}FO}%
\colorbox{green}{\color[gray]{0.75}FO}%
\colorbox{green}{\color[gray]{0.75}FO}%
\colorbox{green}{\color[gray]{0.75}FO}%
\colorbox{green}{\color[gray]{0.75}FO}%
\colorbox{green}{\color[gray]{0.75}FO}%
\colorbox{green}{\color[gray]{0.75}FO}%
\colorbox{green}{\color[gray]{0.75}FO}%
\colorbox{green}{\color[gray]{0.75}FO}%
\colorbox{green}{\color[gray]{0.75}FO}%
\colorbox{green}{\color[gray]{0.75}FO}%
\colorbox{green}{\color[gray]{0.75}FO}%
\colorbox{green}{\color[gray]{0.75}FO}%
\colorbox{green}{\color[gray]{0.75}FO}%
\colorbox{green}{\color[gray]{0.75}FO}%
\colorbox{green}{\color[gray]{0.75}FO}%
\colorbox{green}{\color[gray]{0.75}FO}%
\colorbox{green}{\color[gray]{0.75}FO}%
\colorbox{green}{\color[gray]{0.75}FO}%
\colorbox{green}{\color[gray]{0.75}FO}%
\colorbox{green}{\color[gray]{0.75}FO}%
\colorbox{green}{\color[gray]{0.75}FO}%
\colorbox{green}{\color[gray]{0.75}FO}%
\colorbox{green}{\color[gray]{0.75}FO}%
\colorbox{green}{\color[gray]{0.75}FO}%
\colorbox{green}{\color[gray]{0.75}FO}%
\colorbox{green}{\color[gray]{0.75}FO}%
\colorbox{green}{\color[gray]{0.75}FO}%
\colorbox{green}{\color[gray]{0.75}FO}%
\colorbox{green}{\color[gray]{0.75}FO}%
\colorbox{green}{\color[gray]{0.75}FO}%
\colorbox{green}{\color[gray]{0.75}FO}%
\colorbox{green}{\color[gray]{0.75}FO}%
\colorbox{green}{\color[gray]{0.75}FO}%
\colorbox{green}{\color[gray]{0.75}FO}%
\colorbox{green}{\color[gray]{0.75}FO}%
\colorbox{green}{\color[gray]{0.75}FO}%
\colorbox{green}{\color[gray]{0.75}FO}%
\colorbox{green}{\color[gray]{0.75}FO}%
\colorbox{green}{\color[gray]{0.75}FO}%
\colorbox{green}{\color[gray]{0.75}FO}%
\colorbox{green}{\color[gray]{0.75}FO}%
\colorbox{green}{\color[gray]{0.75}FO}%
\colorbox{green}{\color[gray]{0.75}FO}%
\colorbox{green}{\color[gray]{0.75}FO}%
\colorbox{green}{\color[gray]{0.75}FO}%
\colorbox{green}{\color[gray]{0.75}FO}%
\colorbox{green}{\color[gray]{0.75}FO}%
\colorbox{green}{\color[gray]{0.75}FO}%
\colorbox{green}{\color[gray]{0.75}FO}%
\colorbox{green}{\color[gray]{0.75}FO}%
\colorbox{green}{\color[gray]{0.75}FO}%
\colorbox{green}{\color[gray]{0.75}FO}%
\colorbox{green}{\color[gray]{0.75}FO}%
\colorbox{green}{\color[gray]{0.75}FO}%
\colorbox{green}{\color[gray]{0.75}FO}%
\colorbox{green}{\color[gray]{0.75}FO}%
\colorbox{green}{\color[gray]{0.75}FO}%
\\
\colorbox{green}{\color[gray]{0.75}FO}%
\colorbox{green}{\color[gray]{0.75}FO}%
\colorbox{green}{\color[gray]{0.75}FO}%
\colorbox{green}{\color[gray]{0.75}FO}%
\colorbox{green}{\color[gray]{0.75}FO}%
\colorbox{green}{\color[gray]{0.75}FO}%
\colorbox{green}{\color[gray]{0.75}FO}%
\colorbox{green}{\color[gray]{0.75}FO}%
\colorbox{green}{\color[gray]{0.75}FO}%
\colorbox{green}{\color[gray]{0.75}FO}%
\colorbox{green}{\color[gray]{0.75}FO}%
\colorbox{green}{\color[gray]{0.75}FO}%
\colorbox{green}{\color[gray]{0.75}FO}%
\colorbox{green}{\color[gray]{0.75}FO}%
\colorbox{green}{\color[gray]{0.75}FO}%
\colorbox{green}{\color[gray]{0.75}FO}%
\colorbox{green}{\color[gray]{0.75}FO}%
\colorbox{green}{\color[gray]{0.75}FO}%
\colorbox{green}{\color[gray]{0.75}FO}%
\colorbox{green}{\color[gray]{0.75}FO}%
\colorbox{green}{\color[gray]{0.75}FO}%
\colorbox{green}{\color[gray]{0.75}FO}%
\colorbox{green}{\color[gray]{0.75}FO}%
\colorbox{green}{\color[gray]{0.75}FO}%
\colorbox{green}{\color[gray]{0.75}FO}%
\colorbox{green}{\color[gray]{0.75}FO}%
\colorbox{green}{\color[gray]{0.75}FO}%
\colorbox{green}{\color[gray]{0.75}FO}%
\colorbox{green}{\color[gray]{0.75}FO}%
\colorbox{green}{\color[gray]{0.75}FO}%
\colorbox{green}{\color[gray]{0.75}FO}%
\colorbox{green}{\color[gray]{0.75}FO}%
\colorbox{green}{\color[gray]{0.75}FO}%
\colorbox{green}{\color[gray]{0.75}FO}%
\colorbox{green}{\color[gray]{0.75}FO}%
\colorbox{green}{\color[gray]{0.75}FO}%
\colorbox{green}{\color[gray]{0.75}FO}%
\colorbox{green}{\color[gray]{0.75}FO}%
\colorbox{green}{\color[gray]{0.75}FO}%
\colorbox{green}{\color[gray]{0.75}FO}%
\colorbox{green}{\color[gray]{0.75}FO}%
\colorbox{green}{\color[gray]{0.75}FO}%
\colorbox{green}{\color[gray]{0.75}FO}%
\colorbox{green}{\color[gray]{0.75}FO}%
\colorbox{green}{\color[gray]{0.75}FO}%
\colorbox{green}{\color[gray]{0.75}FO}%
\colorbox{green}{\color[gray]{0.75}FO}%
\colorbox{green}{\color[gray]{0.75}FO}%
\colorbox{green}{\color[gray]{0.75}FO}%
\colorbox{green}{\color[gray]{0.75}FO}%
\colorbox{green}{\color[gray]{0.75}FO}%
\colorbox{green}{\color[gray]{0.75}FO}%
\colorbox{green}{\color[gray]{0.75}FO}%
\colorbox{green}{\color[gray]{0.75}FO}%
\colorbox{green}{\color[gray]{0.75}FO}%
\colorbox{green}{\color[gray]{0.75}FO}%
\colorbox{green}{\color[gray]{0.75}FO}%
\colorbox{green}{\color[gray]{0.75}FO}%
\colorbox{green}{\color[gray]{0.75}FO}%
\colorbox{green}{\color[gray]{0.75}FO}%
\colorbox{green}{\color[gray]{0.75}FO}%
\colorbox{green}{\color[gray]{0.75}FO}%
\colorbox{green}{\color[gray]{0.75}FO}%
\colorbox{green}{\color[gray]{0.75}FO}%
\colorbox{green}{\color[gray]{0.75}FO}%
\colorbox{green}{\color[gray]{0.75}FO}%
\colorbox{green}{\color[gray]{0.75}FO}%
\colorbox{green}{\color[gray]{0.75}FO}%
\colorbox{green}{\color[gray]{0.75}FO}%
\colorbox{green}{\color[gray]{0.75}FO}%
\colorbox{green}{\color[gray]{0.75}FO}%
\colorbox{green}{\color[gray]{0.75}FO}%
\colorbox{green}{\color[gray]{0.75}FO}%
\colorbox{green}{\color[gray]{0.75}FO}%
\colorbox{green}{\color[gray]{0.75}FO}%
\colorbox{green}{\color[gray]{0.75}FO}%
\colorbox{green}{\color[gray]{0.75}FO}%
\colorbox{green}{\color[gray]{0.75}FO}%
\colorbox{green}{\color[gray]{0.75}FO}%
\colorbox{green}{\color[gray]{0.75}FO}%
\colorbox{green}{\color[gray]{0.75}FO}%
\colorbox{green}{\color[gray]{0.75}FO}%
\colorbox{green}{\color[gray]{0.75}FO}%
\colorbox{green}{\color[gray]{0.75}FO}%
\colorbox{green}{\color[gray]{0.75}FO}%
\colorbox{green}{\color[gray]{0.75}FO}%
\colorbox{green}{\color[gray]{0.75}FO}%
\colorbox{green}{\color[gray]{0.75}FO}%
\colorbox{green}{\color[gray]{0.75}FO}%
\colorbox{green}{\color[gray]{0.75}FO}%
\colorbox{green}{\color[gray]{0.75}FO}%
\colorbox{green}{\color[gray]{0.75}FO}%
\colorbox{green}{\color[gray]{0.75}FO}%
\colorbox{green}{\color[gray]{0.75}FO}%
\colorbox{green}{\color[gray]{0.75}FO}%
\colorbox{green}{\color[gray]{0.75}FO}%
\colorbox{green}{\color[gray]{0.75}FO}%
\colorbox{green}{\color[gray]{0.75}FO}%
\colorbox{green}{\color[gray]{0.75}FO}%
\colorbox{green}{\color[gray]{0.75}FO}%
\\
\colorbox{green}{\color[gray]{0.75}FO}%
\colorbox{green}{\color[gray]{0.75}FO}%
\colorbox{green}{\color[gray]{0.75}FO}%
\colorbox{green}{\color[gray]{0.75}FO}%
\colorbox{green}{\color[gray]{0.75}FO}%
\colorbox{green}{\color[gray]{0.75}FO}%
\colorbox{green}{\color[gray]{0.75}FO}%
\colorbox{green}{\color[gray]{0.75}FO}%
\colorbox{green}{\color[gray]{0.75}FO}%
\colorbox{green}{\color[gray]{0.75}FO}%
\colorbox{green}{\color[gray]{0.75}FO}%
\colorbox{green}{\color[gray]{0.75}FO}%
\colorbox{green}{\color[gray]{0.75}FO}%
\colorbox{green}{\color[gray]{0.75}FO}%
\colorbox{green}{\color[gray]{0.75}FO}%
\colorbox{green}{\color[gray]{0.75}FO}%
\colorbox{green}{\color[gray]{0.75}FO}%
\colorbox{green}{\color[gray]{0.75}FO}%
\colorbox{green}{\color[gray]{0.75}FO}%
\colorbox{green}{\color[gray]{0.75}FO}%
\colorbox{green}{\color[gray]{0.75}FO}%
\colorbox{green}{\color[gray]{0.75}FO}%
\colorbox{green}{\color[gray]{0.75}FO}%
\colorbox{green}{\color[gray]{0.75}FO}%
\colorbox{green}{\color[gray]{0.75}FO}%
\colorbox{green}{\color[gray]{0.75}FO}%
\colorbox{green}{\color[gray]{0.75}FO}%
\colorbox{green}{\color[gray]{0.75}FO}%
\colorbox{green}{\color[gray]{0.75}FO}%
\colorbox{green}{\color[gray]{0.75}FO}%
\colorbox{green}{\color[gray]{0.75}FO}%
\colorbox{green}{\color[gray]{0.75}FO}%
\colorbox{green}{\color[gray]{0.75}FO}%
\colorbox{green}{\color[gray]{0.75}FO}%
\colorbox{green}{\color[gray]{0.75}FO}%
\colorbox{green}{\color[gray]{0.75}FO}%
\colorbox{green}{\color[gray]{0.75}FO}%
\colorbox{green}{\color[gray]{0.75}FO}%
\colorbox{green}{\color[gray]{0.75}FO}%
\colorbox{green}{\color[gray]{0.75}FO}%
\colorbox{green}{\color[gray]{0.75}FO}%
\colorbox{green}{\color[gray]{0.75}FO}%
\colorbox{green}{\color[gray]{0.75}FO}%
\colorbox{green}{\color[gray]{0.75}FO}%
\colorbox{green}{\color[gray]{0.75}FO}%
\colorbox{green}{\color[gray]{0.75}FO}%
\colorbox{green}{\color[gray]{0.75}FO}%
\colorbox{green}{\color[gray]{0.75}FO}%
\colorbox{green}{\color[gray]{0.75}FO}%
\colorbox{green}{\color[gray]{0.75}FO}%
\colorbox{green}{\color[gray]{0.75}FO}%
\colorbox{green}{\color[gray]{0.75}FO}%
\colorbox{green}{\color[gray]{0.75}FO}%
\colorbox{green}{\color[gray]{0.75}FO}%
\colorbox{green}{\color[gray]{0.75}FO}%
\colorbox{green}{\color[gray]{0.75}FO}%
\colorbox{green}{\color[gray]{0.75}FO}%
\colorbox{green}{\color[gray]{0.75}FO}%
\colorbox{green}{\color[gray]{0.75}FO}%
\colorbox{green}{\color[gray]{0.75}FO}%
\colorbox{green}{\color[gray]{0.75}FO}%
\colorbox{green}{\color[gray]{0.75}FO}%
\colorbox{green}{\color[gray]{0.75}FO}%
\colorbox{green}{\color[gray]{0.75}FO}%
\colorbox{green}{\color[gray]{0.75}FO}%
\colorbox{green}{\color[gray]{0.75}FO}%
\colorbox{green}{\color[gray]{0.75}FO}%
\colorbox{green}{\color[gray]{0.75}FO}%
\colorbox{green}{\color[gray]{0.75}FO}%
\colorbox{green}{\color[gray]{0.75}FO}%
\colorbox{green}{\color[gray]{0.75}FO}%
\colorbox{green}{\color[gray]{0.75}FO}%
\colorbox{green}{\color[gray]{0.75}FO}%
\colorbox{green}{\color[gray]{0.75}FO}%
\colorbox{green}{\color[gray]{0.75}FO}%
\colorbox{green}{\color[gray]{0.75}FO}%
\colorbox{green}{\color[gray]{0.75}FO}%
\colorbox{green}{\color[gray]{0.75}FO}%
\colorbox{green}{\color[gray]{0.75}FO}%
\colorbox{green}{\color[gray]{0.75}FO}%
\colorbox{green}{\color[gray]{0.75}FO}%
\colorbox{green}{\color[gray]{0.75}FO}%
\colorbox{green}{\color[gray]{0.75}FO}%
\colorbox{green}{\color[gray]{0.75}FO}%
\colorbox{green}{\color[gray]{0.75}FO}%
\colorbox{green}{\color[gray]{0.75}FO}%
\colorbox{green}{\color[gray]{0.75}FO}%
\colorbox{green}{\color[gray]{0.75}FO}%
\colorbox{green}{\color[gray]{0.75}FO}%
\colorbox{green}{\color[gray]{0.75}FO}%
\colorbox{green}{\color[gray]{0.75}FO}%
\colorbox{green}{\color[gray]{0.75}FO}%
\colorbox{green}{\color[gray]{0.75}FO}%
\colorbox{green}{\color[gray]{0.75}FO}%
\colorbox{green}{\color[gray]{0.75}FO}%
\colorbox{green}{\color[gray]{0.75}FO}%
\colorbox{green}{\color[gray]{0.75}FO}%
\colorbox{green}{\color[gray]{0.75}FO}%
\colorbox{green}{\color[gray]{0.75}FO}%
\colorbox{green}{\color[gray]{0.75}FO}%
\\
\colorbox{green}{\color[gray]{0.75}FO}%
\colorbox{green}{\color[gray]{0.75}FO}%
\colorbox{green}{\color[gray]{0.75}FO}%
\colorbox{green}{\color[gray]{0.75}FO}%
\colorbox{green}{\color[gray]{0.75}FO}%
\colorbox{green}{\color[gray]{0.75}FO}%
\colorbox{green}{\color[gray]{0.75}FO}%
\colorbox{green}{\color[gray]{0.75}FO}%
\colorbox{green}{\color[gray]{0.75}FO}%
\colorbox{green}{\color[gray]{0.75}FO}%
\colorbox{green}{\color[gray]{0.75}FO}%
\colorbox{green}{\color[gray]{0.75}FO}%
\colorbox{green}{\color[gray]{0.75}FO}%
\colorbox{green}{\color[gray]{0.75}FO}%
\colorbox{green}{\color[gray]{0.75}FO}%
\colorbox{green}{\color[gray]{0.75}FO}%
\colorbox{green}{\color[gray]{0.75}FO}%
\colorbox{green}{\color[gray]{0.75}FO}%
\colorbox{green}{\color[gray]{0.75}FO}%
\colorbox{green}{\color[gray]{0.75}FO}%
\colorbox{green}{\color[gray]{0.75}FO}%
\colorbox{green}{\color[gray]{0.75}FO}%
\colorbox{green}{\color[gray]{0.75}FO}%
\colorbox{green}{\color[gray]{0.75}FO}%
\colorbox{green}{\color[gray]{0.75}FO}%
\colorbox{green}{\color[gray]{0.75}FO}%
\colorbox{green}{\color[gray]{0.75}FO}%
\colorbox{green}{\color[gray]{0.75}FO}%
\colorbox{green}{\color[gray]{0.75}FO}%
\colorbox{green}{\color[gray]{0.75}FO}%
\colorbox{green}{\color[gray]{0.75}FO}%
\colorbox{green}{\color[gray]{0.75}FO}%
\colorbox{green}{\color[gray]{0.75}FO}%
\colorbox{green}{\color[gray]{0.75}FO}%
\colorbox{green}{\color[gray]{0.75}FO}%
\colorbox{green}{\color[gray]{0.75}FO}%
\colorbox{green}{\color[gray]{0.75}FO}%
\colorbox{green}{\color[gray]{0.75}FO}%
\colorbox{green}{\color[gray]{0.75}FO}%
\colorbox{green}{\color[gray]{0.75}FO}%
\colorbox{green}{\color[gray]{0.75}FO}%
\colorbox{green}{\color[gray]{0.75}FO}%
\colorbox{green}{\color[gray]{0.75}FO}%
\colorbox{green}{\color[gray]{0.75}FO}%
\colorbox{green}{\color[gray]{0.75}FO}%
\colorbox{green}{\color[gray]{0.75}FO}%
\colorbox{green}{\color[gray]{0.75}FO}%
\colorbox{green}{\color[gray]{0.75}FO}%
\colorbox{green}{\color[gray]{0.75}FO}%
\colorbox{green}{\color[gray]{0.75}FO}%
\colorbox{green}{\color[gray]{0.75}FO}%
\colorbox{green}{\color[gray]{0.75}FO}%
\colorbox{green}{\color[gray]{0.75}FO}%
\colorbox{green}{\color[gray]{0.75}FO}%
\colorbox{green}{\color[gray]{0.75}FO}%
\colorbox{green}{\color[gray]{0.75}FO}%
\colorbox{green}{\color[gray]{0.75}FO}%
\colorbox{green}{\color[gray]{0.75}FO}%
\colorbox{green}{\color[gray]{0.75}FO}%
\colorbox{green}{\color[gray]{0.75}FO}%
\colorbox{green}{\color[gray]{0.75}FO}%
\colorbox{green}{\color[gray]{0.75}FO}%
\colorbox{green}{\color[gray]{0.75}FO}%
\colorbox{green}{\color[gray]{0.75}FO}%
\colorbox{green}{\color[gray]{0.75}FO}%
\colorbox{green}{\color[gray]{0.75}FO}%
\colorbox{green}{\color[gray]{0.75}FO}%
\colorbox{green}{\color[gray]{0.75}FO}%
\colorbox{green}{\color[gray]{0.75}FO}%
\colorbox{green}{\color[gray]{0.75}FO}%
\colorbox{green}{\color[gray]{0.75}FO}%
\colorbox{green}{\color[gray]{0.75}FO}%
\colorbox{green}{\color[gray]{0.75}FO}%
\colorbox{green}{\color[gray]{0.75}FO}%
\colorbox{green}{\color[gray]{0.75}FO}%
\colorbox{green}{\color[gray]{0.75}FO}%
\colorbox{green}{\color[gray]{0.75}FO}%
\colorbox{green}{\color[gray]{0.75}FO}%
\colorbox{green}{\color[gray]{0.75}FO}%
\colorbox{green}{\color[gray]{0.75}FO}%
\colorbox{green}{\color[gray]{0.75}FO}%
\colorbox{green}{\color[gray]{0.75}FO}%
\colorbox{green}{\color[gray]{0.75}FO}%
\colorbox{green}{\color[gray]{0.75}FO}%
\colorbox{green}{\color[gray]{0.75}FO}%
\colorbox{green}{\color[gray]{0.75}FO}%
\colorbox{green}{\color[gray]{0.75}FO}%
\colorbox{green}{\color[gray]{0.75}FO}%
\colorbox{green}{\color[gray]{0.75}FO}%
\colorbox{green}{\color[gray]{0.75}FO}%
\colorbox{green}{\color[gray]{0.75}FO}%
\colorbox{green}{\color[gray]{0.75}FO}%
\colorbox{green}{\color[gray]{0.75}FO}%
\colorbox{green}{\color[gray]{0.75}FO}%
\colorbox{green}{\color[gray]{0.75}FO}%
\colorbox{green}{\color[gray]{0.75}FO}%
\colorbox{green}{\color[gray]{0.75}FO}%
\colorbox{green}{\color[gray]{0.75}FO}%
\colorbox{green}{\color[gray]{0.75}FO}%
\colorbox{green}{\color[gray]{0.75}FO}%
\\
\colorbox{green}{\color[gray]{0.75}FO}%
\colorbox{green}{\color[gray]{0.75}FO}%
\colorbox{green}{\color[gray]{0.75}FO}%
\colorbox{green}{\color[gray]{0.75}FO}%
\colorbox{green}{\color[gray]{0.75}FO}%
\colorbox{green}{\color[gray]{0.75}FO}%
\colorbox{green}{\color[gray]{0.75}FO}%
\colorbox{green}{\color[gray]{0.75}FO}%
\colorbox{green}{\color[gray]{0.75}FO}%
\colorbox{green}{\color[gray]{0.75}FO}%
\colorbox{green}{\color[gray]{0.75}FO}%
\colorbox{green}{\color[gray]{0.75}FO}%
\colorbox{green}{\color[gray]{0.75}FO}%
\colorbox{green}{\color[gray]{0.75}FO}%
\colorbox{green}{\color[gray]{0.75}FO}%
\colorbox{green}{\color[gray]{0.75}FO}%
\colorbox{green}{\color[gray]{0.75}FO}%
\colorbox{green}{\color[gray]{0.75}FO}%
\colorbox{green}{\color[gray]{0.75}FO}%
\colorbox{green}{\color[gray]{0.75}FO}%
\colorbox{green}{\color[gray]{0.75}FO}%
\colorbox{green}{\color[gray]{0.75}FO}%
\colorbox{green}{\color[gray]{0.75}FO}%
\colorbox{green}{\color[gray]{0.75}FO}%
\colorbox{green}{\color[gray]{0.75}FO}%
\colorbox{green}{\color[gray]{0.75}FO}%
\colorbox{green}{\color[gray]{0.75}FO}%
\colorbox{green}{\color[gray]{0.75}FO}%
\colorbox{green}{\color[gray]{0.75}FO}%
\colorbox{green}{\color[gray]{0.75}FO}%
\colorbox{green}{\color[gray]{0.75}FO}%
\colorbox{green}{\color[gray]{0.75}FO}%
\colorbox{green}{\color[gray]{0.75}FO}%
\colorbox{green}{\color[gray]{0.75}FO}%
\colorbox{green}{\color[gray]{0.75}FO}%
\colorbox{green}{\color[gray]{0.75}FO}%
\colorbox{green}{\color[gray]{0.75}FO}%
\colorbox{green}{\color[gray]{0.75}FO}%
\colorbox{green}{\color[gray]{0.75}FO}%
\colorbox{green}{\color[gray]{0.75}FO}%
\colorbox{green}{\color[gray]{0.75}FO}%
\colorbox{green}{\color[gray]{0.75}FO}%
\colorbox{green}{\color[gray]{0.75}FO}%
\colorbox{green}{\color[gray]{0.75}FO}%
\colorbox{green}{\color[gray]{0.75}FO}%
\colorbox{green}{\color[gray]{0.75}FO}%
\colorbox{green}{\color[gray]{0.75}FO}%
\colorbox{green}{\color[gray]{0.75}FO}%
\colorbox{green}{\color[gray]{0.75}FO}%
\colorbox{green}{\color[gray]{0.75}FO}%
\colorbox{green}{\color[gray]{0.75}FO}%
\colorbox{green}{\color[gray]{0.75}FO}%
\colorbox{green}{\color[gray]{0.75}FO}%
\colorbox{green}{\color[gray]{0.75}FO}%
\colorbox{green}{\color[gray]{0.75}FO}%
\colorbox{green}{\color[gray]{0.75}FO}%
\colorbox{green}{\color[gray]{0.75}FO}%
\colorbox{green}{\color[gray]{0.75}FO}%
\colorbox{green}{\color[gray]{0.75}FO}%
\colorbox{green}{\color[gray]{0.75}FO}%
\colorbox{green}{\color[gray]{0.75}FO}%
\colorbox{green}{\color[gray]{0.75}FO}%
\colorbox{green}{\color[gray]{0.75}FO}%
\colorbox{green}{\color[gray]{0.75}FO}%
\colorbox{green}{\color[gray]{0.75}FO}%
\colorbox{green}{\color[gray]{0.75}FO}%
\colorbox{green}{\color[gray]{0.75}FO}%
\colorbox{green}{\color[gray]{0.75}FO}%
\colorbox{green}{\color[gray]{0.75}FO}%
\colorbox{green}{\color[gray]{0.75}FO}%
\colorbox{green}{\color[gray]{0.75}FO}%
\colorbox{green}{\color[gray]{0.75}FO}%
\colorbox{green}{\color[gray]{0.75}FO}%
\colorbox{green}{\color[gray]{0.75}FO}%
\colorbox{green}{\color[gray]{0.75}FO}%
\colorbox{green}{\color[gray]{0.75}FO}%
\colorbox{green}{\color[gray]{0.75}FO}%
\colorbox{green}{\color[gray]{0.75}FO}%
\colorbox{green}{\color[gray]{0.75}FO}%
\colorbox{green}{\color[gray]{0.75}FO}%
\colorbox{green}{\color[gray]{0.75}FO}%
\colorbox{green}{\color[gray]{0.75}FO}%
\colorbox{green}{\color[gray]{0.75}FO}%
\colorbox{green}{\color[gray]{0.75}FO}%
\colorbox{green}{\color[gray]{0.75}FO}%
\colorbox{green}{\color[gray]{0.75}FO}%
\colorbox{green}{\color[gray]{0.75}FO}%
\colorbox{green}{\color[gray]{0.75}FO}%
\colorbox{green}{\color[gray]{0.75}FO}%
\colorbox{green}{\color[gray]{0.75}FO}%
\colorbox{green}{\color[gray]{0.75}FO}%
\colorbox{green}{\color[gray]{0.75}FO}%
\colorbox{green}{\color[gray]{0.75}FO}%
\colorbox{green}{\color[gray]{0.75}FO}%
\colorbox{green}{\color[gray]{0.75}FO}%
\colorbox{green}{\color[gray]{0.75}FO}%
\colorbox{green}{\color[gray]{0.75}FO}%
\colorbox{green}{\color[gray]{0.75}FO}%
\colorbox{green}{\color[gray]{0.75}FO}%
\colorbox{green}{\color[gray]{0.75}FO}%
\\
\colorbox{green}{\color[gray]{0.75}FO}%
\colorbox{green}{\color[gray]{0.75}FO}%
\colorbox{green}{\color[gray]{0.75}FO}%
\colorbox{green}{\color[gray]{0.75}FO}%
\colorbox{green}{\color[gray]{0.75}FO}%
\colorbox{green}{\color[gray]{0.75}FO}%
\colorbox{green}{\color[gray]{0.75}FO}%
\colorbox{green}{\color[gray]{0.75}FO}%
\colorbox{green}{\color[gray]{0.75}FO}%
\colorbox{green}{\color[gray]{0.75}FO}%
\colorbox{green}{\color[gray]{0.75}FO}%
\colorbox{green}{\color[gray]{0.75}FO}%
\colorbox{green}{\color[gray]{0.75}FO}%
\colorbox{green}{\color[gray]{0.75}FO}%
\colorbox{green}{\color[gray]{0.75}FO}%
\colorbox{green}{\color[gray]{0.75}FO}%
\colorbox{green}{\color[gray]{0.75}FO}%
\colorbox{green}{\color[gray]{0.75}FO}%
\colorbox{green}{\color[gray]{0.75}FO}%
\colorbox{green}{\color[gray]{0.75}FO}%
\colorbox{green}{\color[gray]{0.75}FO}%
\colorbox{green}{\color[gray]{0.75}FO}%
\colorbox{green}{\color[gray]{0.75}FO}%
\colorbox{green}{\color[gray]{0.75}FO}%
\colorbox{green}{\color[gray]{0.75}FO}%
\colorbox{green}{\color[gray]{0.75}FO}%
\colorbox{green}{\color[gray]{0.75}FO}%
\colorbox{green}{\color[gray]{0.75}FO}%
\colorbox{green}{\color[gray]{0.75}FO}%
\colorbox{green}{\color[gray]{0.75}FO}%
\colorbox{green}{\color[gray]{0.75}FO}%
\colorbox{green}{\color[gray]{0.75}FO}%
\colorbox{green}{\color[gray]{0.75}FO}%
\colorbox{green}{\color[gray]{0.75}FO}%
\colorbox{green}{\color[gray]{0.75}FO}%
\colorbox{green}{\color[gray]{0.75}FO}%
\colorbox{green}{\color[gray]{0.75}FO}%
\colorbox{green}{\color[gray]{0.75}FO}%
\colorbox{green}{\color[gray]{0.75}FO}%
\colorbox{green}{\color[gray]{0.75}FO}%
\colorbox{green}{\color[gray]{0.75}FO}%
\colorbox{green}{\color[gray]{0.75}FO}%
\colorbox{green}{\color[gray]{0.75}FO}%
\colorbox{green}{\color[gray]{0.75}FO}%
\colorbox{green}{\color[gray]{0.75}FO}%
\colorbox{green}{\color[gray]{0.75}FO}%
\colorbox{green}{\color[gray]{0.75}FO}%
\colorbox{green}{\color[gray]{0.75}FO}%
\colorbox{green}{\color[gray]{0.75}FO}%
\colorbox{green}{\color[gray]{0.75}FO}%
\colorbox{green}{\color[gray]{0.75}FO}%
\colorbox{green}{\color[gray]{0.75}FO}%
\colorbox{green}{\color[gray]{0.75}FO}%
\colorbox{green}{\color[gray]{0.75}FO}%
\colorbox{green}{\color[gray]{0.75}FO}%
\colorbox{green}{\color[gray]{0.75}FO}%
\colorbox{green}{\color[gray]{0.75}FO}%
\colorbox{green}{\color[gray]{0.75}FO}%
\colorbox{green}{\color[gray]{0.75}FO}%
\colorbox{green}{\color[gray]{0.75}FO}%
\colorbox{green}{\color[gray]{0.75}FO}%
\colorbox{green}{\color[gray]{0.75}FO}%
\colorbox{green}{\color[gray]{0.75}FO}%
\colorbox{green}{\color[gray]{0.75}FO}%
\colorbox{green}{\color[gray]{0.75}FO}%
\colorbox{green}{\color[gray]{0.75}FO}%
\colorbox{green}{\color[gray]{0.75}FO}%
\colorbox{green}{\color[gray]{0.75}FO}%
\colorbox{green}{\color[gray]{0.75}FO}%
\colorbox{green}{\color[gray]{0.75}FO}%
\colorbox{green}{\color[gray]{0.75}FO}%
\colorbox{green}{\color[gray]{0.75}FO}%
\colorbox{green}{\color[gray]{0.75}FO}%
\colorbox{green}{\color[gray]{0.75}FO}%
\colorbox{green}{\color[gray]{0.75}FO}%
\colorbox{green}{\color[gray]{0.75}FO}%
\colorbox{green}{\color[gray]{0.75}FO}%
\colorbox{green}{\color[gray]{0.75}FO}%
\colorbox{green}{\color[gray]{0.75}FO}%
\colorbox{green}{\color[gray]{0.75}FO}%
\colorbox{green}{\color[gray]{0.75}FO}%
\colorbox{green}{\color[gray]{0.75}FO}%
\colorbox{green}{\color[gray]{0.75}FO}%
\colorbox{green}{\color[gray]{0.75}FO}%
\colorbox{green}{\color[gray]{0.75}FO}%
\colorbox{green}{\color[gray]{0.75}FO}%
\colorbox{green}{\color[gray]{0.75}FO}%
\colorbox{green}{\color[gray]{0.75}FO}%
\colorbox{green}{\color[gray]{0.75}FO}%
\colorbox{green}{\color[gray]{0.75}FO}%
\colorbox{green}{\color[gray]{0.75}FO}%
\colorbox{green}{\color[gray]{0.75}FO}%
\colorbox{green}{\color[gray]{0.75}FO}%
\colorbox{green}{\color[gray]{0.75}FO}%
\colorbox{green}{\color[gray]{0.75}FO}%
\colorbox{green}{\color[gray]{0.75}FO}%
\colorbox{green}{\color[gray]{0.75}FO}%
\colorbox{green}{\color[gray]{0.75}FO}%
\colorbox{green}{\color[gray]{0.75}FO}%
\colorbox{green}{\color[gray]{0.75}FO}%
\\
\colorbox{green}{\color[gray]{0.75}FO}%
\colorbox{green}{\color[gray]{0.75}FO}%
\colorbox{green}{\color[gray]{0.75}FO}%
\colorbox{green}{\color[gray]{0.75}FO}%
\colorbox{green}{\color[gray]{0.75}FO}%
\colorbox{green}{\color[gray]{0.75}FO}%
\colorbox{green}{\color[gray]{0.75}FO}%
\colorbox{green}{\color[gray]{0.75}FO}%
\colorbox{green}{\color[gray]{0.75}FO}%
\colorbox{green}{\color[gray]{0.75}FO}%
\colorbox{green}{\color[gray]{0.75}FO}%
\colorbox{green}{\color[gray]{0.75}FO}%
\colorbox{green}{\color[gray]{0.75}FO}%
\colorbox{green}{\color[gray]{0.75}FO}%
\colorbox{green}{\color[gray]{0.75}FO}%
\colorbox{green}{\color[gray]{0.75}FO}%
\colorbox{green}{\color[gray]{0.75}FO}%
\colorbox{green}{\color[gray]{0.75}FO}%
\colorbox{green}{\color[gray]{0.75}FO}%
\colorbox{green}{\color[gray]{0.75}FO}%
\colorbox{green}{\color[gray]{0.75}FO}%
\colorbox{green}{\color[gray]{0.75}FO}%
\colorbox{green}{\color[gray]{0.75}FO}%
\colorbox{green}{\color[gray]{0.75}FO}%
\colorbox{green}{\color[gray]{0.75}FO}%
\colorbox{green}{\color[gray]{0.75}FO}%
\colorbox{green}{\color[gray]{0.75}FO}%
\colorbox{green}{\color[gray]{0.75}FO}%
\colorbox{green}{\color[gray]{0.75}FO}%
\colorbox{green}{\color[gray]{0.75}FO}%
\colorbox{green}{\color[gray]{0.75}FO}%
\colorbox{green}{\color[gray]{0.75}FO}%
\colorbox{green}{\color[gray]{0.75}FO}%
\colorbox{green}{\color[gray]{0.75}FO}%
\colorbox{green}{\color[gray]{0.75}FO}%
\colorbox{green}{\color[gray]{0.75}FO}%
\colorbox{green}{\color[gray]{0.75}FO}%
\colorbox{green}{\color[gray]{0.75}FO}%
\colorbox{green}{\color[gray]{0.75}FO}%
\colorbox{green}{\color[gray]{0.75}FO}%
\colorbox{green}{\color[gray]{0.75}FO}%
\colorbox{green}{\color[gray]{0.75}FO}%
\colorbox{green}{\color[gray]{0.75}FO}%
\colorbox{green}{\color[gray]{0.75}FO}%
\colorbox{green}{\color[gray]{0.75}FO}%
\colorbox{green}{\color[gray]{0.75}FO}%
\colorbox{green}{\color[gray]{0.75}FO}%
\colorbox{green}{\color[gray]{0.75}FO}%
\colorbox{green}{\color[gray]{0.75}FO}%
\colorbox{green}{\color[gray]{0.75}FO}%
\colorbox{green}{\color[gray]{0.75}FO}%
\colorbox{green}{\color[gray]{0.75}FO}%
\colorbox{green}{\color[gray]{0.75}FO}%
\colorbox{green}{\color[gray]{0.75}FO}%
\colorbox{green}{\color[gray]{0.75}FO}%
\colorbox{green}{\color[gray]{0.75}FO}%
\colorbox{green}{\color[gray]{0.75}FO}%
\colorbox{green}{\color[gray]{0.75}FO}%
\colorbox{green}{\color[gray]{0.75}FO}%
\colorbox{green}{\color[gray]{0.75}FO}%
\colorbox{green}{\color[gray]{0.75}FO}%
\colorbox{green}{\color[gray]{0.75}FO}%
\colorbox{green}{\color[gray]{0.75}FO}%
\colorbox{green}{\color[gray]{0.75}FO}%
\colorbox{green}{\color[gray]{0.75}FO}%
\colorbox{green}{\color[gray]{0.75}FO}%
\colorbox{green}{\color[gray]{0.75}FO}%
\colorbox{green}{\color[gray]{0.75}FO}%
\colorbox{green}{\color[gray]{0.75}FO}%
\colorbox{green}{\color[gray]{0.75}FO}%
\colorbox{green}{\color[gray]{0.75}FO}%
\colorbox{green}{\color[gray]{0.75}FO}%
\colorbox{green}{\color[gray]{0.75}FO}%
\colorbox{green}{\color[gray]{0.75}FO}%
\colorbox{green}{\color[gray]{0.75}FO}%
\colorbox{green}{\color[gray]{0.75}FO}%
\colorbox{green}{\color[gray]{0.75}FO}%
\colorbox{green}{\color[gray]{0.75}FO}%
\colorbox{green}{\color[gray]{0.75}FO}%
\colorbox{green}{\color[gray]{0.75}FO}%
\colorbox{green}{\color[gray]{0.75}FO}%
\colorbox{green}{\color[gray]{0.75}FO}%
\colorbox{green}{\color[gray]{0.75}FO}%
\colorbox{green}{\color[gray]{0.75}FO}%
\colorbox{green}{\color[gray]{0.75}FO}%
\colorbox{green}{\color[gray]{0.75}FO}%
\colorbox{green}{\color[gray]{0.75}FO}%
\colorbox{green}{\color[gray]{0.75}FO}%
\colorbox{green}{\color[gray]{0.75}FO}%
\colorbox{green}{\color[gray]{0.75}FO}%
\colorbox{green}{\color[gray]{0.75}FO}%
\colorbox{green}{\color[gray]{0.75}FO}%
\colorbox{green}{\color[gray]{0.75}FO}%
\colorbox{green}{\color[gray]{0.75}FO}%
\colorbox{green}{\color[gray]{0.75}FO}%
\colorbox{green}{\color[gray]{0.75}FO}%
\colorbox{green}{\color[gray]{0.75}FO}%
\colorbox{green}{\color[gray]{0.75}FO}%
\colorbox{green}{\color[gray]{0.75}FO}%
\colorbox{green}{\color[gray]{0.75}FO}%
\\
\colorbox{green}{\color[gray]{0.75}FO}%
\colorbox{green}{\color[gray]{0.75}FO}%
\colorbox{green}{\color[gray]{0.75}FO}%
\colorbox{green}{\color[gray]{0.75}FO}%
\colorbox{green}{\color[gray]{0.75}FO}%
\colorbox{green}{\color[gray]{0.75}FO}%
\colorbox{green}{\color[gray]{0.75}FO}%
\colorbox{green}{\color[gray]{0.75}FO}%
\colorbox{green}{\color[gray]{0.75}FO}%
\colorbox{green}{\color[gray]{0.75}FO}%
\colorbox{green}{\color[gray]{0.75}FO}%
\colorbox{green}{\color[gray]{0.75}FO}%
\colorbox{green}{\color[gray]{0.75}FO}%
\colorbox{green}{\color[gray]{0.75}FO}%
\colorbox{green}{\color[gray]{0.75}FO}%
\colorbox{green}{\color[gray]{0.75}FO}%
\colorbox{green}{\color[gray]{0.75}FO}%
\colorbox{green}{\color[gray]{0.75}FO}%
\colorbox{green}{\color[gray]{0.75}FO}%
\colorbox{green}{\color[gray]{0.75}FO}%
\colorbox{green}{\color[gray]{0.75}FO}%
\colorbox{green}{\color[gray]{0.75}FO}%
\colorbox{green}{\color[gray]{0.75}FO}%
\colorbox{green}{\color[gray]{0.75}FO}%
\colorbox{green}{\color[gray]{0.75}FO}%
\colorbox{green}{\color[gray]{0.75}FO}%
\colorbox{green}{\color[gray]{0.75}FO}%
\colorbox{green}{\color[gray]{0.75}FO}%
\colorbox{green}{\color[gray]{0.75}FO}%
\colorbox{green}{\color[gray]{0.75}FO}%
\colorbox{green}{\color[gray]{0.75}FO}%
\colorbox{green}{\color[gray]{0.75}FO}%
\colorbox{green}{\color[gray]{0.75}FO}%
\colorbox{green}{\color[gray]{0.75}FO}%
\colorbox{green}{\color[gray]{0.75}FO}%
\colorbox{green}{\color[gray]{0.75}FO}%
\colorbox{green}{\color[gray]{0.75}FO}%
\colorbox{green}{\color[gray]{0.75}FO}%
\colorbox{green}{\color[gray]{0.75}FO}%
\colorbox{green}{\color[gray]{0.75}FO}%
\colorbox{green}{\color[gray]{0.75}FO}%
\colorbox{green}{\color[gray]{0.75}FO}%
\colorbox{green}{\color[gray]{0.75}FO}%
\colorbox{green}{\color[gray]{0.75}FO}%
\colorbox{green}{\color[gray]{0.75}FO}%
\colorbox{green}{\color[gray]{0.75}FO}%
\colorbox{green}{\color[gray]{0.75}FO}%
\colorbox{green}{\color[gray]{0.75}FO}%
\colorbox{green}{\color[gray]{0.75}FO}%
\colorbox{green}{\color[gray]{0.75}FO}%
\colorbox{green}{\color[gray]{0.75}FO}%
\colorbox{green}{\color[gray]{0.75}FO}%
\colorbox{green}{\color[gray]{0.75}FO}%
\colorbox{green}{\color[gray]{0.75}FO}%
\colorbox{green}{\color[gray]{0.75}FO}%
\colorbox{green}{\color[gray]{0.75}FO}%
\colorbox{green}{\color[gray]{0.75}FO}%
\colorbox{green}{\color[gray]{0.75}FO}%
\colorbox{green}{\color[gray]{0.75}FO}%
\colorbox{green}{\color[gray]{0.75}FO}%
\colorbox{green}{\color[gray]{0.75}FO}%
\colorbox{green}{\color[gray]{0.75}FO}%
\colorbox{green}{\color[gray]{0.75}FO}%
\colorbox{green}{\color[gray]{0.75}FO}%
\colorbox{green}{\color[gray]{0.75}FO}%
\colorbox{green}{\color[gray]{0.75}FO}%
\colorbox{green}{\color[gray]{0.75}FO}%
\colorbox{green}{\color[gray]{0.75}FO}%
\colorbox{green}{\color[gray]{0.75}FO}%
\colorbox{green}{\color[gray]{0.75}FO}%
\colorbox{green}{\color[gray]{0.75}FO}%
\colorbox{green}{\color[gray]{0.75}FO}%
\colorbox{green}{\color[gray]{0.75}FO}%
\colorbox{green}{\color[gray]{0.75}FO}%
\colorbox{green}{\color[gray]{0.75}FO}%
\colorbox{green}{\color[gray]{0.75}FO}%
\colorbox{green}{\color[gray]{0.75}FO}%
\colorbox{green}{\color[gray]{0.75}FO}%
\colorbox{green}{\color[gray]{0.75}FO}%
\colorbox{green}{\color[gray]{0.75}FO}%
\colorbox{green}{\color[gray]{0.75}FO}%
\colorbox{green}{\color[gray]{0.75}FO}%
\colorbox{green}{\color[gray]{0.75}FO}%
\colorbox{green}{\color[gray]{0.75}FO}%
\colorbox{green}{\color[gray]{0.75}FO}%
\colorbox{green}{\color[gray]{0.75}FO}%
\colorbox{green}{\color[gray]{0.75}FO}%
\colorbox{green}{\color[gray]{0.75}FO}%
\colorbox{green}{\color[gray]{0.75}FO}%
\colorbox{green}{\color[gray]{0.75}FO}%
\colorbox{green}{\color[gray]{0.75}FO}%
\colorbox{green}{\color[gray]{0.75}FO}%
\colorbox{green}{\color[gray]{0.75}FO}%
\colorbox{green}{\color[gray]{0.75}FO}%
\colorbox{green}{\color[gray]{0.75}FO}%
\colorbox{green}{\color[gray]{0.75}FO}%
\colorbox{green}{\color[gray]{0.75}FO}%
\colorbox{green}{\color[gray]{0.75}FO}%
\colorbox{green}{\color[gray]{0.75}FO}%
\colorbox{green}{\color[gray]{0.75}FO}%
\\
\colorbox{green}{\color[gray]{0.75}FO}%
\colorbox{green}{\color[gray]{0.75}FO}%
\colorbox{green}{\color[gray]{0.75}FO}%
\colorbox{green}{\color[gray]{0.75}FO}%
\colorbox{green}{\color[gray]{0.75}FO}%
\colorbox{green}{\color[gray]{0.75}FO}%
\colorbox{green}{\color[gray]{0.75}FO}%
\colorbox{green}{\color[gray]{0.75}FO}%
\colorbox{green}{\color[gray]{0.75}FO}%
\colorbox{green}{\color[gray]{0.75}FO}%
\colorbox{green}{\color[gray]{0.75}FO}%
\colorbox{green}{\color[gray]{0.75}FO}%
\colorbox{green}{\color[gray]{0.75}FO}%
\colorbox{green}{\color[gray]{0.75}FO}%
\colorbox{green}{\color[gray]{0.75}FO}%
\colorbox{green}{\color[gray]{0.75}FO}%
\colorbox{green}{\color[gray]{0.75}FO}%
\colorbox{green}{\color[gray]{0.75}FO}%
\colorbox{green}{\color[gray]{0.75}FO}%
\colorbox{green}{\color[gray]{0.75}FO}%
\colorbox{green}{\color[gray]{0.75}FO}%
\colorbox{green}{\color[gray]{0.75}FO}%
\colorbox{green}{\color[gray]{0.75}FO}%
\colorbox{green}{\color[gray]{0.75}FO}%
\colorbox{green}{\color[gray]{0.75}FO}%
\colorbox{green}{\color[gray]{0.75}FO}%
\colorbox{green}{\color[gray]{0.75}FO}%
\colorbox{green}{\color[gray]{0.75}FO}%
\colorbox{green}{\color[gray]{0.75}FO}%
\colorbox{green}{\color[gray]{0.75}FO}%
\colorbox{green}{\color[gray]{0.75}FO}%
\colorbox{green}{\color[gray]{0.75}FO}%
\colorbox{green}{\color[gray]{0.75}FO}%
\colorbox{green}{\color[gray]{0.75}FO}%
\colorbox{green}{\color[gray]{0.75}FO}%
\colorbox{green}{\color[gray]{0.75}FO}%
\colorbox{green}{\color[gray]{0.75}FO}%
\colorbox{green}{\color[gray]{0.75}FO}%
\colorbox{green}{\color[gray]{0.75}FO}%
\colorbox{green}{\color[gray]{0.75}FO}%
\colorbox{green}{\color[gray]{0.75}FO}%
\colorbox{green}{\color[gray]{0.75}FO}%
\colorbox{green}{\color[gray]{0.75}FO}%
\colorbox{green}{\color[gray]{0.75}FO}%
\colorbox{green}{\color[gray]{0.75}FO}%
\colorbox{green}{\color[gray]{0.75}FO}%
\colorbox{green}{\color[gray]{0.75}FO}%
\colorbox{green}{\color[gray]{0.75}FO}%
\colorbox{green}{\color[gray]{0.75}FO}%
\colorbox{green}{\color[gray]{0.75}FO}%
\colorbox{green}{\color[gray]{0.75}FO}%
\colorbox{green}{\color[gray]{0.75}FO}%
\colorbox{green}{\color[gray]{0.75}FO}%
\colorbox{green}{\color[gray]{0.75}FO}%
\colorbox{green}{\color[gray]{0.75}FO}%
\colorbox{green}{\color[gray]{0.75}FO}%
\colorbox{green}{\color[gray]{0.75}FO}%
\colorbox{green}{\color[gray]{0.75}FO}%
\colorbox{green}{\color[gray]{0.75}FO}%
\colorbox{green}{\color[gray]{0.75}FO}%
\colorbox{green}{\color[gray]{0.75}FO}%
\colorbox{green}{\color[gray]{0.75}FO}%
\colorbox{green}{\color[gray]{0.75}FO}%
\colorbox{green}{\color[gray]{0.75}FO}%
\colorbox{green}{\color[gray]{0.75}FO}%
\colorbox{green}{\color[gray]{0.75}FO}%
\colorbox{green}{\color[gray]{0.75}FO}%
\colorbox{green}{\color[gray]{0.75}FO}%
\colorbox{green}{\color[gray]{0.75}FO}%
\colorbox{green}{\color[gray]{0.75}FO}%
\colorbox{green}{\color[gray]{0.75}FO}%
\colorbox{green}{\color[gray]{0.75}FO}%
\colorbox{green}{\color[gray]{0.75}FO}%
\colorbox{green}{\color[gray]{0.75}FO}%
\colorbox{green}{\color[gray]{0.75}FO}%
\colorbox{green}{\color[gray]{0.75}FO}%
\colorbox{green}{\color[gray]{0.75}FO}%
\colorbox{green}{\color[gray]{0.75}FO}%
\colorbox{green}{\color[gray]{0.75}FO}%
\colorbox{green}{\color[gray]{0.75}FO}%
\colorbox{green}{\color[gray]{0.75}FO}%
\colorbox{green}{\color[gray]{0.75}FO}%
\colorbox{green}{\color[gray]{0.75}FO}%
\colorbox{green}{\color[gray]{0.75}FO}%
\colorbox{green}{\color[gray]{0.75}FO}%
\colorbox{green}{\color[gray]{0.75}FO}%
\colorbox{green}{\color[gray]{0.75}FO}%
\colorbox{green}{\color[gray]{0.75}FO}%
\colorbox{green}{\color[gray]{0.75}FO}%
\colorbox{green}{\color[gray]{0.75}FO}%
\colorbox{green}{\color[gray]{0.75}FO}%
\colorbox{green}{\color[gray]{0.75}FO}%
\colorbox{green}{\color[gray]{0.75}FO}%
\colorbox{green}{\color[gray]{0.75}FO}%
\colorbox{green}{\color[gray]{0.75}FO}%
\colorbox{green}{\color[gray]{0.75}FO}%
\colorbox{green}{\color[gray]{0.75}FO}%
\colorbox{green}{\color[gray]{0.75}FO}%
\colorbox{green}{\color[gray]{0.75}FO}%
\colorbox{green}{\color[gray]{0.75}FO}%
\\
\colorbox{green}{\color[gray]{0.75}FO}%
\colorbox{green}{\color[gray]{0.75}FO}%
\colorbox{green}{\color[gray]{0.75}FO}%
\colorbox{green}{\color[gray]{0.75}FO}%
\colorbox{green}{\color[gray]{0.75}FO}%
\colorbox{green}{\color[gray]{0.75}FO}%
\colorbox{green}{\color[gray]{0.75}FO}%
\colorbox{green}{\color[gray]{0.75}FO}%
\colorbox{green}{\color[gray]{0.75}FO}%
\colorbox{green}{\color[gray]{0.75}FO}%
\colorbox{green}{\color[gray]{0.75}FO}%
\colorbox{green}{\color[gray]{0.75}FO}%
\colorbox{green}{\color[gray]{0.75}FO}%
\colorbox{green}{\color[gray]{0.75}FO}%
\colorbox{green}{\color[gray]{0.75}FO}%
\colorbox{green}{\color[gray]{0.75}FO}%
\colorbox{green}{\color[gray]{0.75}FO}%
\colorbox{green}{\color[gray]{0.75}FO}%
\colorbox{green}{\color[gray]{0.75}FO}%
\colorbox{green}{\color[gray]{0.75}FO}%
\colorbox{green}{\color[gray]{0.75}FO}%
\colorbox{green}{\color[gray]{0.75}FO}%
\colorbox{green}{\color[gray]{0.75}FO}%
\colorbox{green}{\color[gray]{0.75}FO}%
\colorbox{green}{\color[gray]{0.75}FO}%
\colorbox{green}{\color[gray]{0.75}FO}%
\colorbox{green}{\color[gray]{0.75}FO}%
\colorbox{green}{\color[gray]{0.75}FO}%
\colorbox{green}{\color[gray]{0.75}FO}%
\colorbox{green}{\color[gray]{0.75}FO}%
\colorbox{green}{\color[gray]{0.75}FO}%
\colorbox{green}{\color[gray]{0.75}FO}%
\colorbox{green}{\color[gray]{0.75}FO}%
\colorbox{green}{\color[gray]{0.75}FO}%
\colorbox{green}{\color[gray]{0.75}FO}%
\colorbox{green}{\color[gray]{0.75}FO}%
\colorbox{green}{\color[gray]{0.75}FO}%
\colorbox{green}{\color[gray]{0.75}FO}%
\colorbox{green}{\color[gray]{0.75}FO}%
\colorbox{green}{\color[gray]{0.75}FO}%
\colorbox{green}{\color[gray]{0.75}FO}%
\colorbox{green}{\color[gray]{0.75}FO}%
\colorbox{green}{\color[gray]{0.75}FO}%
\colorbox{green}{\color[gray]{0.75}FO}%
\colorbox{green}{\color[gray]{0.75}FO}%
\colorbox{green}{\color[gray]{0.75}FO}%
\colorbox{green}{\color[gray]{0.75}FO}%
\colorbox{green}{\color[gray]{0.75}FO}%
\colorbox{green}{\color[gray]{0.75}FO}%
\colorbox{green}{\color[gray]{0.75}FO}%
\colorbox{green}{\color[gray]{0.75}FO}%
\colorbox{green}{\color[gray]{0.75}FO}%
\colorbox{green}{\color[gray]{0.75}FO}%
\colorbox{green}{\color[gray]{0.75}FO}%
\colorbox{green}{\color[gray]{0.75}FO}%
\colorbox{green}{\color[gray]{0.75}FO}%
\colorbox{green}{\color[gray]{0.75}FO}%
\colorbox{green}{\color[gray]{0.75}FO}%
\colorbox{green}{\color[gray]{0.75}FO}%
\colorbox{green}{\color[gray]{0.75}FO}%
\colorbox{green}{\color[gray]{0.75}FO}%
\colorbox{green}{\color[gray]{0.75}FO}%
\colorbox{green}{\color[gray]{0.75}FO}%
\colorbox{green}{\color[gray]{0.75}FO}%
\colorbox{green}{\color[gray]{0.75}FO}%
\colorbox{green}{\color[gray]{0.75}FO}%
\colorbox{green}{\color[gray]{0.75}FO}%
\colorbox{green}{\color[gray]{0.75}FO}%
\colorbox{green}{\color[gray]{0.75}FO}%
\colorbox{green}{\color[gray]{0.75}FO}%
\colorbox{green}{\color[gray]{0.75}FO}%
\colorbox{green}{\color[gray]{0.75}FO}%
\colorbox{green}{\color[gray]{0.75}FO}%
\colorbox{green}{\color[gray]{0.75}FO}%
\colorbox{green}{\color[gray]{0.75}FO}%
\colorbox{green}{\color[gray]{0.75}FO}%
\colorbox{green}{\color[gray]{0.75}FO}%
\colorbox{green}{\color[gray]{0.75}FO}%
\colorbox{green}{\color[gray]{0.75}FO}%
\colorbox{green}{\color[gray]{0.75}FO}%
\colorbox{green}{\color[gray]{0.75}FO}%
\colorbox{green}{\color[gray]{0.75}FO}%
\colorbox{green}{\color[gray]{0.75}FO}%
\colorbox{green}{\color[gray]{0.75}FO}%
\colorbox{green}{\color[gray]{0.75}FO}%
\colorbox{green}{\color[gray]{0.75}FO}%
\colorbox{green}{\color[gray]{0.75}FO}%
\colorbox{green}{\color[gray]{0.75}FO}%
\colorbox{green}{\color[gray]{0.75}FO}%
\colorbox{green}{\color[gray]{0.75}FO}%
\colorbox{green}{\color[gray]{0.75}FO}%
\colorbox{green}{\color[gray]{0.75}FO}%
\colorbox{green}{\color[gray]{0.75}FO}%
\colorbox{green}{\color[gray]{0.75}FO}%
\colorbox{green}{\color[gray]{0.75}FO}%
\colorbox{green}{\color[gray]{0.75}FO}%
\colorbox{green}{\color[gray]{0.75}FO}%
\colorbox{green}{\color[gray]{0.75}FO}%
\colorbox{green}{\color[gray]{0.75}FO}%
\colorbox{green}{\color[gray]{0.75}FO}%
\\
\colorbox{green}{\color[gray]{0.75}FO}%
\colorbox{green}{\color[gray]{0.75}FO}%
\colorbox{green}{\color[gray]{0.75}FO}%
\colorbox{green}{\color[gray]{0.75}FO}%
\colorbox{green}{\color[gray]{0.75}FO}%
\colorbox{green}{\color[gray]{0.75}FO}%
\colorbox{green}{\color[gray]{0.75}FO}%
\colorbox{green}{\color[gray]{0.75}FO}%
\colorbox{green}{\color[gray]{0.75}FO}%
\colorbox{green}{\color[gray]{0.75}FO}%
\colorbox{green}{\color[gray]{0.75}FO}%
\colorbox{green}{\color[gray]{0.75}FO}%
\colorbox{green}{\color[gray]{0.75}FO}%
\colorbox{green}{\color[gray]{0.75}FO}%
\colorbox{green}{\color[gray]{0.75}FO}%
\colorbox{green}{\color[gray]{0.75}FO}%
\colorbox{green}{\color[gray]{0.75}FO}%
\colorbox{green}{\color[gray]{0.75}FO}%
\colorbox{green}{\color[gray]{0.75}FO}%
\colorbox{green}{\color[gray]{0.75}FO}%
\colorbox{green}{\color[gray]{0.75}FO}%
\colorbox{green}{\color[gray]{0.75}FO}%
\colorbox{green}{\color[gray]{0.75}FO}%
\colorbox{green}{\color[gray]{0.75}FO}%
\colorbox{green}{\color[gray]{0.75}FO}%
\colorbox{green}{\color[gray]{0.75}FO}%
\colorbox{green}{\color[gray]{0.75}FO}%
\colorbox{green}{\color[gray]{0.75}FO}%
\colorbox{green}{\color[gray]{0.75}FO}%
\colorbox{green}{\color[gray]{0.75}FO}%
\colorbox{green}{\color[gray]{0.75}FO}%
\colorbox{green}{\color[gray]{0.75}FO}%
\colorbox{green}{\color[gray]{0.75}FO}%
\colorbox{green}{\color[gray]{0.75}FO}%
\colorbox{green}{\color[gray]{0.75}FO}%
\colorbox{green}{\color[gray]{0.75}FO}%
\colorbox{green}{\color[gray]{0.75}FO}%
\colorbox{green}{\color[gray]{0.75}FO}%
\colorbox{green}{\color[gray]{0.75}FO}%
\colorbox{green}{\color[gray]{0.75}FO}%
\colorbox{green}{\color[gray]{0.75}FO}%
\colorbox{green}{\color[gray]{0.75}FO}%
\colorbox{green}{\color[gray]{0.75}FO}%
\colorbox{green}{\color[gray]{0.75}FO}%
\colorbox{green}{\color[gray]{0.75}FO}%
\colorbox{green}{\color[gray]{0.75}FO}%
\colorbox{green}{\color[gray]{0.75}FO}%
\colorbox{green}{\color[gray]{0.75}FO}%
\colorbox{green}{\color[gray]{0.75}FO}%
\colorbox{green}{\color[gray]{0.75}FO}%
\colorbox{green}{\color[gray]{0.75}FO}%
\colorbox{green}{\color[gray]{0.75}FO}%
\colorbox{green}{\color[gray]{0.75}FO}%
\colorbox{green}{\color[gray]{0.75}FO}%
\colorbox{green}{\color[gray]{0.75}FO}%
\colorbox{green}{\color[gray]{0.75}FO}%
\colorbox{green}{\color[gray]{0.75}FO}%
\colorbox{green}{\color[gray]{0.75}FO}%
\colorbox{green}{\color[gray]{0.75}FO}%
\colorbox{green}{\color[gray]{0.75}FO}%
\colorbox{green}{\color[gray]{0.75}FO}%
\colorbox{green}{\color[gray]{0.75}FO}%
\colorbox{green}{\color[gray]{0.75}FO}%
\colorbox{green}{\color[gray]{0.75}FO}%
\colorbox{green}{\color[gray]{0.75}FO}%
\colorbox{green}{\color[gray]{0.75}FO}%
\colorbox{green}{\color[gray]{0.75}FO}%
\colorbox{green}{\color[gray]{0.75}FO}%
\colorbox{green}{\color[gray]{0.75}FO}%
\colorbox{green}{\color[gray]{0.75}FO}%
\colorbox{green}{\color[gray]{0.75}FO}%
\colorbox{green}{\color[gray]{0.75}FO}%
\colorbox{green}{\color[gray]{0.75}FO}%
\colorbox{green}{\color[gray]{0.75}FO}%
\colorbox{green}{\color[gray]{0.75}FO}%
\colorbox{green}{\color[gray]{0.75}FO}%
\colorbox{green}{\color[gray]{0.75}FO}%
\colorbox{green}{\color[gray]{0.75}FO}%
\colorbox{green}{\color[gray]{0.75}FO}%
\colorbox{green}{\color[gray]{0.75}FO}%
\colorbox{green}{\color[gray]{0.75}FO}%
\colorbox{green}{\color[gray]{0.75}FO}%
\colorbox{green}{\color[gray]{0.75}FO}%
\colorbox{green}{\color[gray]{0.75}FO}%
\colorbox{green}{\color[gray]{0.75}FO}%
\colorbox{green}{\color[gray]{0.75}FO}%
\colorbox{green}{\color[gray]{0.75}FO}%
\colorbox{green}{\color[gray]{0.75}FO}%
\colorbox{green}{\color[gray]{0.75}FO}%
\colorbox{green}{\color[gray]{0.75}FO}%
\colorbox{green}{\color[gray]{0.75}FO}%
\colorbox{green}{\color[gray]{0.75}FO}%
\colorbox{green}{\color[gray]{0.75}FO}%
\colorbox{green}{\color[gray]{0.75}FO}%
\colorbox{green}{\color[gray]{0.75}FO}%
\colorbox{green}{\color[gray]{0.75}FO}%
\colorbox{green}{\color[gray]{0.75}FO}%
\colorbox{green}{\color[gray]{0.75}FO}%
\colorbox{green}{\color[gray]{0.75}FO}%
\colorbox{green}{\color[gray]{0.75}FO}%
\\
\colorbox{green}{\color[gray]{0.75}FO}%
\colorbox{green}{\color[gray]{0.75}FO}%
\colorbox{green}{\color[gray]{0.75}FO}%
\colorbox{green}{\color[gray]{0.75}FO}%
\colorbox{green}{\color[gray]{0.75}FO}%
\colorbox{green}{\color[gray]{0.75}FO}%
\colorbox{green}{\color[gray]{0.75}FO}%
\colorbox{green}{\color[gray]{0.75}FO}%
\colorbox{green}{\color[gray]{0.75}FO}%
\colorbox{green}{\color[gray]{0.75}FO}%
\colorbox{green}{\color[gray]{0.75}FO}%
\colorbox{green}{\color[gray]{0.75}FO}%
\colorbox{green}{\color[gray]{0.75}FO}%
\colorbox{green}{\color[gray]{0.75}FO}%
\colorbox{green}{\color[gray]{0.75}FO}%
\colorbox{green}{\color[gray]{0.75}FO}%
\colorbox{green}{\color[gray]{0.75}FO}%
\colorbox{green}{\color[gray]{0.75}FO}%
\colorbox{green}{\color[gray]{0.75}FO}%
\colorbox{green}{\color[gray]{0.75}FO}%
\colorbox{green}{\color[gray]{0.75}FO}%
\colorbox{green}{\color[gray]{0.75}FO}%
\colorbox{green}{\color[gray]{0.75}FO}%
\colorbox{green}{\color[gray]{0.75}FO}%
\colorbox{green}{\color[gray]{0.75}FO}%
\colorbox{green}{\color[gray]{0.75}FO}%
\colorbox{green}{\color[gray]{0.75}FO}%
\colorbox{green}{\color[gray]{0.75}FO}%
\colorbox{green}{\color[gray]{0.75}FO}%
\colorbox{green}{\color[gray]{0.75}FO}%
\colorbox{green}{\color[gray]{0.75}FO}%
\colorbox{green}{\color[gray]{0.75}FO}%
\colorbox{green}{\color[gray]{0.75}FO}%
\colorbox{green}{\color[gray]{0.75}FO}%
\colorbox{green}{\color[gray]{0.75}FO}%
\colorbox{green}{\color[gray]{0.75}FO}%
\colorbox{green}{\color[gray]{0.75}FO}%
\colorbox{green}{\color[gray]{0.75}FO}%
\colorbox{green}{\color[gray]{0.75}FO}%
\colorbox{green}{\color[gray]{0.75}FO}%
\colorbox{green}{\color[gray]{0.75}FO}%
\colorbox{green}{\color[gray]{0.75}FO}%
\colorbox{green}{\color[gray]{0.75}FO}%
\colorbox{green}{\color[gray]{0.75}FO}%
\colorbox{green}{\color[gray]{0.75}FO}%
\colorbox{green}{\color[gray]{0.75}FO}%
\colorbox{green}{\color[gray]{0.75}FO}%
\colorbox{green}{\color[gray]{0.75}FO}%
\colorbox{green}{\color[gray]{0.75}FO}%
\colorbox{green}{\color[gray]{0.75}FO}%
\colorbox{green}{\color[gray]{0.75}FO}%
\colorbox{green}{\color[gray]{0.75}FO}%
\colorbox{green}{\color[gray]{0.75}FO}%
\colorbox{green}{\color[gray]{0.75}FO}%
\colorbox{green}{\color[gray]{0.75}FO}%
\colorbox{green}{\color[gray]{0.75}FO}%
\colorbox{green}{\color[gray]{0.75}FO}%
\colorbox{green}{\color[gray]{0.75}FO}%
\colorbox{green}{\color[gray]{0.75}FO}%
\colorbox{green}{\color[gray]{0.75}FO}%
\colorbox{green}{\color[gray]{0.75}FO}%
\colorbox{green}{\color[gray]{0.75}FO}%
\colorbox{green}{\color[gray]{0.75}FO}%
\colorbox{green}{\color[gray]{0.75}FO}%
\colorbox{green}{\color[gray]{0.75}FO}%
\colorbox{green}{\color[gray]{0.75}FO}%
\colorbox{green}{\color[gray]{0.75}FO}%
\colorbox{green}{\color[gray]{0.75}FO}%
\colorbox{green}{\color[gray]{0.75}FO}%
\colorbox{green}{\color[gray]{0.75}FO}%
\colorbox{green}{\color[gray]{0.75}FO}%
\colorbox{green}{\color[gray]{0.75}FO}%
\colorbox{green}{\color[gray]{0.75}FO}%
\colorbox{green}{\color[gray]{0.75}FO}%
\colorbox{green}{\color[gray]{0.75}FO}%
\colorbox{green}{\color[gray]{0.75}FO}%
\colorbox{green}{\color[gray]{0.75}FO}%
\colorbox{green}{\color[gray]{0.75}FO}%
\colorbox{green}{\color[gray]{0.75}FO}%
\colorbox{green}{\color[gray]{0.75}FO}%
\colorbox{green}{\color[gray]{0.75}FO}%
\colorbox{green}{\color[gray]{0.75}FO}%
\colorbox{green}{\color[gray]{0.75}FO}%
\colorbox{green}{\color[gray]{0.75}FO}%
\colorbox{green}{\color[gray]{0.75}FO}%
\colorbox{green}{\color[gray]{0.75}FO}%
\colorbox{green}{\color[gray]{0.75}FO}%
\colorbox{green}{\color[gray]{0.75}FO}%
\colorbox{green}{\color[gray]{0.75}FO}%
\colorbox{green}{\color[gray]{0.75}FO}%
\colorbox{green}{\color[gray]{0.75}FO}%
\colorbox{green}{\color[gray]{0.75}FO}%
\colorbox{green}{\color[gray]{0.75}FO}%
\colorbox{green}{\color[gray]{0.75}FO}%
\colorbox{green}{\color[gray]{0.75}FO}%
\colorbox{green}{\color[gray]{0.75}FO}%
\colorbox{green}{\color[gray]{0.75}FO}%
\colorbox{green}{\color[gray]{0.75}FO}%
\colorbox{green}{\color[gray]{0.75}FO}%
\colorbox{green}{\color[gray]{0.75}FO}%
\\
\colorbox{green}{\color[gray]{0.75}FO}%
\colorbox{green}{\color[gray]{0.75}FO}%
\colorbox{green}{\color[gray]{0.75}FO}%
\colorbox{green}{\color[gray]{0.75}FO}%
\colorbox{green}{\color[gray]{0.75}FO}%
\colorbox{green}{\color[gray]{0.75}FO}%
\colorbox{green}{\color[gray]{0.75}FO}%
\colorbox{green}{\color[gray]{0.75}FO}%
\colorbox{green}{\color[gray]{0.75}FO}%
\colorbox{green}{\color[gray]{0.75}FO}%
\colorbox{green}{\color[gray]{0.75}FO}%
\colorbox{green}{\color[gray]{0.75}FO}%
\colorbox{green}{\color[gray]{0.75}FO}%
\colorbox{green}{\color[gray]{0.75}FO}%
\colorbox{green}{\color[gray]{0.75}FO}%
\colorbox{green}{\color[gray]{0.75}FO}%
\colorbox{green}{\color[gray]{0.75}FO}%
\colorbox{green}{\color[gray]{0.75}FO}%
\colorbox{green}{\color[gray]{0.75}FO}%
\colorbox{green}{\color[gray]{0.75}FO}%
\colorbox{green}{\color[gray]{0.75}FO}%
\colorbox{green}{\color[gray]{0.75}FO}%
\colorbox{green}{\color[gray]{0.75}FO}%
\colorbox{green}{\color[gray]{0.75}FO}%
\colorbox{green}{\color[gray]{0.75}FO}%
\colorbox{green}{\color[gray]{0.75}FO}%
\colorbox{green}{\color[gray]{0.75}FO}%
\colorbox{green}{\color[gray]{0.75}FO}%
\colorbox{green}{\color[gray]{0.75}FO}%
\colorbox{green}{\color[gray]{0.75}FO}%
\colorbox{green}{\color[gray]{0.75}FO}%
\colorbox{green}{\color[gray]{0.75}FO}%
\colorbox{green}{\color[gray]{0.75}FO}%
\colorbox{green}{\color[gray]{0.75}FO}%
\colorbox{green}{\color[gray]{0.75}FO}%
\colorbox{green}{\color[gray]{0.75}FO}%
\colorbox{green}{\color[gray]{0.75}FO}%
\colorbox{green}{\color[gray]{0.75}FO}%
\colorbox{green}{\color[gray]{0.75}FO}%
\colorbox{green}{\color[gray]{0.75}FO}%
\colorbox{green}{\color[gray]{0.75}FO}%
\colorbox{green}{\color[gray]{0.75}FO}%
\colorbox{green}{\color[gray]{0.75}FO}%
\colorbox{green}{\color[gray]{0.75}FO}%
\colorbox{green}{\color[gray]{0.75}FO}%
\colorbox{green}{\color[gray]{0.75}FO}%
\colorbox{green}{\color[gray]{0.75}FO}%
\colorbox{green}{\color[gray]{0.75}FO}%
\colorbox{green}{\color[gray]{0.75}FO}%
\colorbox{green}{\color[gray]{0.75}FO}%
\colorbox{green}{\color[gray]{0.75}FO}%
\colorbox{green}{\color[gray]{0.75}FO}%
\colorbox{green}{\color[gray]{0.75}FO}%
\colorbox{green}{\color[gray]{0.75}FO}%
\colorbox{green}{\color[gray]{0.75}FO}%
\colorbox{green}{\color[gray]{0.75}FO}%
\colorbox{green}{\color[gray]{0.75}FO}%
\colorbox{green}{\color[gray]{0.75}FO}%
\colorbox{green}{\color[gray]{0.75}FO}%
\colorbox{green}{\color[gray]{0.75}FO}%
\colorbox{green}{\color[gray]{0.75}FO}%
\colorbox{green}{\color[gray]{0.75}FO}%
\colorbox{green}{\color[gray]{0.75}FO}%
\colorbox{green}{\color[gray]{0.75}FO}%
\colorbox{green}{\color[gray]{0.75}FO}%
\colorbox{green}{\color[gray]{0.75}FO}%
\colorbox{green}{\color[gray]{0.75}FO}%
\colorbox{green}{\color[gray]{0.75}FO}%
\colorbox{green}{\color[gray]{0.75}FO}%
\colorbox{green}{\color[gray]{0.75}FO}%
\colorbox{green}{\color[gray]{0.75}FO}%
\colorbox{green}{\color[gray]{0.75}FO}%
\colorbox{green}{\color[gray]{0.75}FO}%
\colorbox{green}{\color[gray]{0.75}FO}%
\colorbox{green}{\color[gray]{0.75}FO}%
\colorbox{green}{\color[gray]{0.75}FO}%
\colorbox{green}{\color[gray]{0.75}FO}%
\colorbox{green}{\color[gray]{0.75}FO}%
\colorbox{green}{\color[gray]{0.75}FO}%
\colorbox{green}{\color[gray]{0.75}FO}%
\colorbox{green}{\color[gray]{0.75}FO}%
\colorbox{green}{\color[gray]{0.75}FO}%
\colorbox{green}{\color[gray]{0.75}FO}%
\colorbox{green}{\color[gray]{0.75}FO}%
\colorbox{green}{\color[gray]{0.75}FO}%
\colorbox{green}{\color[gray]{0.75}FO}%
\colorbox{green}{\color[gray]{0.75}FO}%
\colorbox{green}{\color[gray]{0.75}FO}%
\colorbox{green}{\color[gray]{0.75}FO}%
\colorbox{green}{\color[gray]{0.75}FO}%
\colorbox{green}{\color[gray]{0.75}FO}%
\colorbox{green}{\color[gray]{0.75}FO}%
\colorbox{green}{\color[gray]{0.75}FO}%
\colorbox{green}{\color[gray]{0.75}FO}%
\colorbox{green}{\color[gray]{0.75}FO}%
\colorbox{green}{\color[gray]{0.75}FO}%
\colorbox{green}{\color[gray]{0.75}FO}%
\colorbox{green}{\color[gray]{0.75}FO}%
\colorbox{green}{\color[gray]{0.75}FO}%
\colorbox{green}{\color[gray]{0.75}FO}%
\\
\colorbox{green}{\color[gray]{0.75}FO}%
\colorbox{green}{\color[gray]{0.75}FO}%
\colorbox{green}{\color[gray]{0.75}FO}%
\colorbox{green}{\color[gray]{0.75}FO}%
\colorbox{green}{\color[gray]{0.75}FO}%
\colorbox{green}{\color[gray]{0.75}FO}%
\colorbox{green}{\color[gray]{0.75}FO}%
\colorbox{green}{\color[gray]{0.75}FO}%
\colorbox{green}{\color[gray]{0.75}FO}%
\colorbox{green}{\color[gray]{0.75}FO}%
\colorbox{green}{\color[gray]{0.75}FO}%
\colorbox{green}{\color[gray]{0.75}FO}%
\colorbox{green}{\color[gray]{0.75}FO}%
\colorbox{green}{\color[gray]{0.75}FO}%
\colorbox{green}{\color[gray]{0.75}FO}%
\colorbox{green}{\color[gray]{0.75}FO}%
\colorbox{green}{\color[gray]{0.75}FO}%
\colorbox{green}{\color[gray]{0.75}FO}%
\colorbox{green}{\color[gray]{0.75}FO}%
\colorbox{green}{\color[gray]{0.75}FO}%
\colorbox{green}{\color[gray]{0.75}FO}%
\colorbox{green}{\color[gray]{0.75}FO}%
\colorbox{green}{\color[gray]{0.75}FO}%
\colorbox{green}{\color[gray]{0.75}FO}%
\colorbox{green}{\color[gray]{0.75}FO}%
\colorbox{green}{\color[gray]{0.75}FO}%
\colorbox{green}{\color[gray]{0.75}FO}%
\colorbox{green}{\color[gray]{0.75}FO}%
\colorbox{green}{\color[gray]{0.75}FO}%
\colorbox{green}{\color[gray]{0.75}FO}%
\colorbox{green}{\color[gray]{0.75}FO}%
\colorbox{green}{\color[gray]{0.75}FO}%
\colorbox{green}{\color[gray]{0.75}FO}%
\colorbox{green}{\color[gray]{0.75}FO}%
\colorbox{green}{\color[gray]{0.75}FO}%
\colorbox{green}{\color[gray]{0.75}FO}%
\colorbox{green}{\color[gray]{0.75}FO}%
\colorbox{green}{\color[gray]{0.75}FO}%
\colorbox{green}{\color[gray]{0.75}FO}%
\colorbox{green}{\color[gray]{0.75}FO}%
\colorbox{green}{\color[gray]{0.75}FO}%
\colorbox{green}{\color[gray]{0.75}FO}%
\colorbox{green}{\color[gray]{0.75}FO}%
\colorbox{green}{\color[gray]{0.75}FO}%
\colorbox{green}{\color[gray]{0.75}FO}%
\colorbox{green}{\color[gray]{0.75}FO}%
\colorbox{green}{\color[gray]{0.75}FO}%
\colorbox{green}{\color[gray]{0.75}FO}%
\colorbox{green}{\color[gray]{0.75}FO}%
\colorbox{green}{\color[gray]{0.75}FO}%
\colorbox{green}{\color[gray]{0.75}FO}%
\colorbox{green}{\color[gray]{0.75}FO}%
\colorbox{green}{\color[gray]{0.75}FO}%
\colorbox{green}{\color[gray]{0.75}FO}%
\colorbox{green}{\color[gray]{0.75}FO}%
\colorbox{green}{\color[gray]{0.75}FO}%
\colorbox{green}{\color[gray]{0.75}FO}%
\colorbox{green}{\color[gray]{0.75}FO}%
\colorbox{green}{\color[gray]{0.75}FO}%
\colorbox{green}{\color[gray]{0.75}FO}%
\colorbox{green}{\color[gray]{0.75}FO}%
\colorbox{green}{\color[gray]{0.75}FO}%
\colorbox{green}{\color[gray]{0.75}FO}%
\colorbox{green}{\color[gray]{0.75}FO}%
\colorbox{green}{\color[gray]{0.75}FO}%
\colorbox{green}{\color[gray]{0.75}FO}%
\colorbox{green}{\color[gray]{0.75}FO}%
\colorbox{green}{\color[gray]{0.75}FO}%
\colorbox{green}{\color[gray]{0.75}FO}%
\colorbox{green}{\color[gray]{0.75}FO}%
\colorbox{green}{\color[gray]{0.75}FO}%
\colorbox{green}{\color[gray]{0.75}FO}%
\colorbox{green}{\color[gray]{0.75}FO}%
\colorbox{green}{\color[gray]{0.75}FO}%
\colorbox{green}{\color[gray]{0.75}FO}%
\colorbox{green}{\color[gray]{0.75}FO}%
\colorbox{green}{\color[gray]{0.75}FO}%
\colorbox{green}{\color[gray]{0.75}FO}%
\colorbox{green}{\color[gray]{0.75}FO}%
\colorbox{green}{\color[gray]{0.75}FO}%
\colorbox{green}{\color[gray]{0.75}FO}%
\colorbox{green}{\color[gray]{0.75}FO}%
\colorbox{green}{\color[gray]{0.75}FO}%
\colorbox{green}{\color[gray]{0.75}FO}%
\colorbox{green}{\color[gray]{0.75}FO}%
\colorbox{green}{\color[gray]{0.75}FO}%
\colorbox{green}{\color[gray]{0.75}FO}%
\colorbox{green}{\color[gray]{0.75}FO}%
\colorbox{green}{\color[gray]{0.75}FO}%
\colorbox{green}{\color[gray]{0.75}FO}%
\colorbox{green}{\color[gray]{0.75}FO}%
\colorbox{green}{\color[gray]{0.75}FO}%
\colorbox{green}{\color[gray]{0.75}FO}%
\colorbox{green}{\color[gray]{0.75}FO}%
\colorbox{green}{\color[gray]{0.75}FO}%
\colorbox{green}{\color[gray]{0.75}FO}%
\colorbox{green}{\color[gray]{0.75}FO}%
\colorbox{green}{\color[gray]{0.75}FO}%
\colorbox{green}{\color[gray]{0.75}FO}%
\colorbox{green}{\color[gray]{0.75}FO}%
\\
\colorbox{green}{\color[gray]{0.75}FO}%
\colorbox{green}{\color[gray]{0.75}FO}%
\colorbox{green}{\color[gray]{0.75}FO}%
\colorbox{green}{\color[gray]{0.75}FO}%
\colorbox{green}{\color[gray]{0.75}FO}%
\colorbox{green}{\color[gray]{0.75}FO}%
\colorbox{green}{\color[gray]{0.75}FO}%
\colorbox{green}{\color[gray]{0.75}FO}%
\colorbox{green}{\color[gray]{0.75}FO}%
\colorbox{green}{\color[gray]{0.75}FO}%
\colorbox{green}{\color[gray]{0.75}FO}%
\colorbox{green}{\color[gray]{0.75}FO}%
\colorbox{green}{\color[gray]{0.75}FO}%
\colorbox{green}{\color[gray]{0.75}FO}%
\colorbox{green}{\color[gray]{0.75}FO}%
\colorbox{green}{\color[gray]{0.75}FO}%
\colorbox{green}{\color[gray]{0.75}FO}%
\colorbox{green}{\color[gray]{0.75}FO}%
\colorbox{green}{\color[gray]{0.75}FO}%
\colorbox{green}{\color[gray]{0.75}FO}%
\colorbox{green}{\color[gray]{0.75}FO}%
\colorbox{green}{\color[gray]{0.75}FO}%
\colorbox{green}{\color[gray]{0.75}FO}%
\colorbox{green}{\color[gray]{0.75}FO}%
\colorbox{green}{\color[gray]{0.75}FO}%
\colorbox{green}{\color[gray]{0.75}FO}%
\colorbox{green}{\color[gray]{0.75}FO}%
\colorbox{green}{\color[gray]{0.75}FO}%
\colorbox{green}{\color[gray]{0.75}FO}%
\colorbox{green}{\color[gray]{0.75}FO}%
\colorbox{green}{\color[gray]{0.75}FO}%
\colorbox{green}{\color[gray]{0.75}FO}%
\colorbox{green}{\color[gray]{0.75}FO}%
\colorbox{green}{\color[gray]{0.75}FO}%
\colorbox{green}{\color[gray]{0.75}FO}%
\colorbox{green}{\color[gray]{0.75}FO}%
\colorbox{green}{\color[gray]{0.75}FO}%
\colorbox{green}{\color[gray]{0.75}FO}%
\colorbox{green}{\color[gray]{0.75}FO}%
\colorbox{green}{\color[gray]{0.75}FO}%
\colorbox{green}{\color[gray]{0.75}FO}%
\colorbox{green}{\color[gray]{0.75}FO}%
\colorbox{green}{\color[gray]{0.75}FO}%
\colorbox{green}{\color[gray]{0.75}FO}%
\colorbox{green}{\color[gray]{0.75}FO}%
\colorbox{green}{\color[gray]{0.75}FO}%
\colorbox{green}{\color[gray]{0.75}FO}%
\colorbox{green}{\color[gray]{0.75}FO}%
\colorbox{green}{\color[gray]{0.75}FO}%
\colorbox{green}{\color[gray]{0.75}FO}%
\colorbox{green}{\color[gray]{0.75}FO}%
\colorbox{green}{\color[gray]{0.75}FO}%
\colorbox{green}{\color[gray]{0.75}FO}%
\colorbox{green}{\color[gray]{0.75}FO}%
\colorbox{green}{\color[gray]{0.75}FO}%
\colorbox{green}{\color[gray]{0.75}FO}%
\colorbox{green}{\color[gray]{0.75}FO}%
\colorbox{green}{\color[gray]{0.75}FO}%
\colorbox{green}{\color[gray]{0.75}FO}%
\colorbox{green}{\color[gray]{0.75}FO}%
\colorbox{green}{\color[gray]{0.75}FO}%
\colorbox{green}{\color[gray]{0.75}FO}%
\colorbox{green}{\color[gray]{0.75}FO}%
\colorbox{green}{\color[gray]{0.75}FO}%
\colorbox{green}{\color[gray]{0.75}FO}%
\colorbox{green}{\color[gray]{0.75}FO}%
\colorbox{green}{\color[gray]{0.75}FO}%
\colorbox{green}{\color[gray]{0.75}FO}%
\colorbox{green}{\color[gray]{0.75}FO}%
\colorbox{green}{\color[gray]{0.75}FO}%
\colorbox{green}{\color[gray]{0.75}FO}%
\colorbox{green}{\color[gray]{0.75}FO}%
\colorbox{green}{\color[gray]{0.75}FO}%
\colorbox{green}{\color[gray]{0.75}FO}%
\colorbox{green}{\color[gray]{0.75}FO}%
\colorbox{green}{\color[gray]{0.75}FO}%
\colorbox{green}{\color[gray]{0.75}FO}%
\colorbox{green}{\color[gray]{0.75}FO}%
\colorbox{green}{\color[gray]{0.75}FO}%
\colorbox{green}{\color[gray]{0.75}FO}%
\colorbox{green}{\color[gray]{0.75}FO}%
\colorbox{green}{\color[gray]{0.75}FO}%
\colorbox{green}{\color[gray]{0.75}FO}%
\colorbox{green}{\color[gray]{0.75}FO}%
\colorbox{green}{\color[gray]{0.75}FO}%
\colorbox{green}{\color[gray]{0.75}FO}%
\colorbox{green}{\color[gray]{0.75}FO}%
\colorbox{green}{\color[gray]{0.75}FO}%
\colorbox{green}{\color[gray]{0.75}FO}%
\colorbox{green}{\color[gray]{0.75}FO}%
\colorbox{green}{\color[gray]{0.75}FO}%
\colorbox{green}{\color[gray]{0.75}FO}%
\colorbox{green}{\color[gray]{0.75}FO}%
\colorbox{green}{\color[gray]{0.75}FO}%
\colorbox{green}{\color[gray]{0.75}FO}%
\colorbox{green}{\color[gray]{0.75}FO}%
\colorbox{green}{\color[gray]{0.75}FO}%
\colorbox{green}{\color[gray]{0.75}FO}%
\colorbox{green}{\color[gray]{0.75}FO}%
\colorbox{green}{\color[gray]{0.75}FO}%
\\
\colorbox{green}{\color[gray]{0.75}FO}%
\colorbox{green}{\color[gray]{0.75}FO}%
\colorbox{green}{\color[gray]{0.75}FO}%
\colorbox{green}{\color[gray]{0.75}FO}%
\colorbox{green}{\color[gray]{0.75}FO}%
\colorbox{green}{\color[gray]{0.75}FO}%
\colorbox{green}{\color[gray]{0.75}FO}%
\colorbox{green}{\color[gray]{0.75}FO}%
\colorbox{green}{\color[gray]{0.75}FO}%
\colorbox{green}{\color[gray]{0.75}FO}%
\colorbox{green}{\color[gray]{0.75}FO}%
\colorbox{green}{\color[gray]{0.75}FO}%
\colorbox{green}{\color[gray]{0.75}FO}%
\colorbox{green}{\color[gray]{0.75}FO}%
\colorbox{green}{\color[gray]{0.75}FO}%
\colorbox{green}{\color[gray]{0.75}FO}%
\colorbox{green}{\color[gray]{0.75}FO}%
\colorbox{green}{\color[gray]{0.75}FO}%
\colorbox{green}{\color[gray]{0.75}FO}%
\colorbox{green}{\color[gray]{0.75}FO}%
\colorbox{green}{\color[gray]{0.75}FO}%
\colorbox{green}{\color[gray]{0.75}FO}%
\colorbox{green}{\color[gray]{0.75}FO}%
\colorbox{green}{\color[gray]{0.75}FO}%
\colorbox{green}{\color[gray]{0.75}FO}%
\colorbox{green}{\color[gray]{0.75}FO}%
\colorbox{green}{\color[gray]{0.75}FO}%
\colorbox{green}{\color[gray]{0.75}FO}%
\colorbox{green}{\color[gray]{0.75}FO}%
\colorbox{green}{\color[gray]{0.75}FO}%
\colorbox{green}{\color[gray]{0.75}FO}%
\colorbox{green}{\color[gray]{0.75}FO}%
\colorbox{green}{\color[gray]{0.75}FO}%
\colorbox{green}{\color[gray]{0.75}FO}%
\colorbox{green}{\color[gray]{0.75}FO}%
\colorbox{green}{\color[gray]{0.75}FO}%
\colorbox{green}{\color[gray]{0.75}FO}%
\colorbox{green}{\color[gray]{0.75}FO}%
\colorbox{green}{\color[gray]{0.75}FO}%
\colorbox{green}{\color[gray]{0.75}FO}%
\colorbox{green}{\color[gray]{0.75}FO}%
\colorbox{green}{\color[gray]{0.75}FO}%
\colorbox{green}{\color[gray]{0.75}FO}%
\colorbox{green}{\color[gray]{0.75}FO}%
\colorbox{green}{\color[gray]{0.75}FO}%
\colorbox{green}{\color[gray]{0.75}FO}%
\colorbox{green}{\color[gray]{0.75}FO}%
\colorbox{green}{\color[gray]{0.75}FO}%
\colorbox{green}{\color[gray]{0.75}FO}%
\colorbox{green}{\color[gray]{0.75}FO}%
\colorbox{green}{\color[gray]{0.75}FO}%
\colorbox{green}{\color[gray]{0.75}FO}%
\colorbox{green}{\color[gray]{0.75}FO}%
\colorbox{green}{\color[gray]{0.75}FO}%
\colorbox{green}{\color[gray]{0.75}FO}%
\colorbox{green}{\color[gray]{0.75}FO}%
\colorbox{green}{\color[gray]{0.75}FO}%
\colorbox{green}{\color[gray]{0.75}FO}%
\colorbox{green}{\color[gray]{0.75}FO}%
\colorbox{green}{\color[gray]{0.75}FO}%
\colorbox{green}{\color[gray]{0.75}FO}%
\colorbox{green}{\color[gray]{0.75}FO}%
\colorbox{green}{\color[gray]{0.75}FO}%
\colorbox{green}{\color[gray]{0.75}FO}%
\colorbox{green}{\color[gray]{0.75}FO}%
\colorbox{green}{\color[gray]{0.75}FO}%
\colorbox{green}{\color[gray]{0.75}FO}%
\colorbox{green}{\color[gray]{0.75}FO}%
\colorbox{green}{\color[gray]{0.75}FO}%
\colorbox{green}{\color[gray]{0.75}FO}%
\colorbox{green}{\color[gray]{0.75}FO}%
\colorbox{green}{\color[gray]{0.75}FO}%
\colorbox{green}{\color[gray]{0.75}FO}%
\colorbox{green}{\color[gray]{0.75}FO}%
\colorbox{green}{\color[gray]{0.75}FO}%
\colorbox{green}{\color[gray]{0.75}FO}%
\colorbox{green}{\color[gray]{0.75}FO}%
\colorbox{green}{\color[gray]{0.75}FO}%
\colorbox{green}{\color[gray]{0.75}FO}%
\colorbox{green}{\color[gray]{0.75}FO}%
\colorbox{green}{\color[gray]{0.75}FO}%
\colorbox{green}{\color[gray]{0.75}FO}%
\colorbox{green}{\color[gray]{0.75}FO}%
\colorbox{green}{\color[gray]{0.75}FO}%
\colorbox{green}{\color[gray]{0.75}FO}%
\colorbox{green}{\color[gray]{0.75}FO}%
\colorbox{green}{\color[gray]{0.75}FO}%
\colorbox{green}{\color[gray]{0.75}FO}%
\colorbox{green}{\color[gray]{0.75}FO}%
\colorbox{green}{\color[gray]{0.75}FO}%
\colorbox{green}{\color[gray]{0.75}FO}%
\colorbox{green}{\color[gray]{0.75}FO}%
\colorbox{green}{\color[gray]{0.75}FO}%
\colorbox{green}{\color[gray]{0.75}FO}%
\colorbox{green}{\color[gray]{0.75}FO}%
\colorbox{green}{\color[gray]{0.75}FO}%
\colorbox{green}{\color[gray]{0.75}FO}%
\colorbox{green}{\color[gray]{0.75}FO}%
\colorbox{green}{\color[gray]{0.75}FO}%
\colorbox{green}{\color[gray]{0.75}FO}%
\\
\colorbox{green}{\color[gray]{0.75}FO}%
\colorbox{green}{\color[gray]{0.75}FO}%
\colorbox{green}{\color[gray]{0.75}FO}%
\colorbox{green}{\color[gray]{0.75}FO}%
\colorbox{green}{\color[gray]{0.75}FO}%
\colorbox{green}{\color[gray]{0.75}FO}%
\colorbox{green}{\color[gray]{0.75}FO}%
\colorbox{green}{\color[gray]{0.75}FO}%
\colorbox{green}{\color[gray]{0.75}FO}%
\colorbox{green}{\color[gray]{0.75}FO}%
\colorbox{green}{\color[gray]{0.75}FO}%
\colorbox{green}{\color[gray]{0.75}FO}%
\colorbox{green}{\color[gray]{0.75}FO}%
\colorbox{green}{\color[gray]{0.75}FO}%
\colorbox{green}{\color[gray]{0.75}FO}%
\colorbox{green}{\color[gray]{0.75}FO}%
\colorbox{green}{\color[gray]{0.75}FO}%
\colorbox{green}{\color[gray]{0.75}FO}%
\colorbox{green}{\color[gray]{0.75}FO}%
\colorbox{green}{\color[gray]{0.75}FO}%
\colorbox{green}{\color[gray]{0.75}FO}%
\colorbox{green}{\color[gray]{0.75}FO}%
\colorbox{green}{\color[gray]{0.75}FO}%
\colorbox{green}{\color[gray]{0.75}FO}%
\colorbox{green}{\color[gray]{0.75}FO}%
\colorbox{green}{\color[gray]{0.75}FO}%
\colorbox{green}{\color[gray]{0.75}FO}%
\colorbox{green}{\color[gray]{0.75}FO}%
\colorbox{green}{\color[gray]{0.75}FO}%
\colorbox{green}{\color[gray]{0.75}FO}%
\colorbox{green}{\color[gray]{0.75}FO}%
\colorbox{green}{\color[gray]{0.75}FO}%
\colorbox{green}{\color[gray]{0.75}FO}%
\colorbox{green}{\color[gray]{0.75}FO}%
\colorbox{green}{\color[gray]{0.75}FO}%
\colorbox{green}{\color[gray]{0.75}FO}%
\colorbox{green}{\color[gray]{0.75}FO}%
\colorbox{green}{\color[gray]{0.75}FO}%
\colorbox{green}{\color[gray]{0.75}FO}%
\colorbox{green}{\color[gray]{0.75}FO}%
\colorbox{green}{\color[gray]{0.75}FO}%
\colorbox{green}{\color[gray]{0.75}FO}%
\colorbox{green}{\color[gray]{0.75}FO}%
\colorbox{green}{\color[gray]{0.75}FO}%
\colorbox{green}{\color[gray]{0.75}FO}%
\colorbox{green}{\color[gray]{0.75}FO}%
\colorbox{green}{\color[gray]{0.75}FO}%
\colorbox{green}{\color[gray]{0.75}FO}%
\colorbox{green}{\color[gray]{0.75}FO}%
\colorbox{green}{\color[gray]{0.75}FO}%
\colorbox{green}{\color[gray]{0.75}FO}%
\colorbox{green}{\color[gray]{0.75}FO}%
\colorbox{green}{\color[gray]{0.75}FO}%
\colorbox{green}{\color[gray]{0.75}FO}%
\colorbox{green}{\color[gray]{0.75}FO}%
\colorbox{green}{\color[gray]{0.75}FO}%
\colorbox{green}{\color[gray]{0.75}FO}%
\colorbox{green}{\color[gray]{0.75}FO}%
\colorbox{green}{\color[gray]{0.75}FO}%
\colorbox{green}{\color[gray]{0.75}FO}%
\colorbox{green}{\color[gray]{0.75}FO}%
\colorbox{green}{\color[gray]{0.75}FO}%
\colorbox{green}{\color[gray]{0.75}FO}%
\colorbox{green}{\color[gray]{0.75}FO}%
\colorbox{green}{\color[gray]{0.75}FO}%
\colorbox{green}{\color[gray]{0.75}FO}%
\colorbox{green}{\color[gray]{0.75}FO}%
\colorbox{green}{\color[gray]{0.75}FO}%
\colorbox{green}{\color[gray]{0.75}FO}%
\colorbox{green}{\color[gray]{0.75}FO}%
\colorbox{green}{\color[gray]{0.75}FO}%
\colorbox{green}{\color[gray]{0.75}FO}%
\colorbox{green}{\color[gray]{0.75}FO}%
\colorbox{green}{\color[gray]{0.75}FO}%
\colorbox{green}{\color[gray]{0.75}FO}%
\colorbox{green}{\color[gray]{0.75}FO}%
\colorbox{green}{\color[gray]{0.75}FO}%
\colorbox{green}{\color[gray]{0.75}FO}%
\colorbox{green}{\color[gray]{0.75}FO}%
\colorbox{green}{\color[gray]{0.75}FO}%
\colorbox{green}{\color[gray]{0.75}FO}%
\colorbox{green}{\color[gray]{0.75}FO}%
\colorbox{green}{\color[gray]{0.75}FO}%
\colorbox{green}{\color[gray]{0.75}FO}%
\colorbox{green}{\color[gray]{0.75}FO}%
\colorbox{green}{\color[gray]{0.75}FO}%
\colorbox{green}{\color[gray]{0.75}FO}%
\colorbox{green}{\color[gray]{0.75}FO}%
\colorbox{green}{\color[gray]{0.75}FO}%
\colorbox{green}{\color[gray]{0.75}FO}%
\colorbox{green}{\color[gray]{0.75}FO}%
\colorbox{green}{\color[gray]{0.75}FO}%
\colorbox{green}{\color[gray]{0.75}FO}%
\colorbox{green}{\color[gray]{0.75}FO}%
\colorbox{green}{\color[gray]{0.75}FO}%
\colorbox{green}{\color[gray]{0.75}FO}%
\colorbox{green}{\color[gray]{0.75}FO}%
\colorbox{green}{\color[gray]{0.75}FO}%
\colorbox{green}{\color[gray]{0.75}FO}%
\colorbox{green}{\color[gray]{0.75}FO}%
\\
\colorbox{green}{\color[gray]{0.75}FO}%
\colorbox{green}{\color[gray]{0.75}FO}%
\colorbox{green}{\color[gray]{0.75}FO}%
\colorbox{green}{\color[gray]{0.75}FO}%
\colorbox{green}{\color[gray]{0.75}FO}%
\colorbox{green}{\color[gray]{0.75}FO}%
\colorbox{green}{\color[gray]{0.75}FO}%
\colorbox{green}{\color[gray]{0.75}FO}%
\colorbox{green}{\color[gray]{0.75}FO}%
\colorbox{green}{\color[gray]{0.75}FO}%
\colorbox{green}{\color[gray]{0.75}FO}%
\colorbox{green}{\color[gray]{0.75}FO}%
\colorbox{green}{\color[gray]{0.75}FO}%
\colorbox{green}{\color[gray]{0.75}FO}%
\colorbox{green}{\color[gray]{0.75}FO}%
\colorbox{green}{\color[gray]{0.75}FO}%
\colorbox{green}{\color[gray]{0.75}FO}%
\colorbox{green}{\color[gray]{0.75}FO}%
\colorbox{green}{\color[gray]{0.75}FO}%
\colorbox{green}{\color[gray]{0.75}FO}%
\colorbox{green}{\color[gray]{0.75}FO}%
\colorbox{green}{\color[gray]{0.75}FO}%
\colorbox{green}{\color[gray]{0.75}FO}%
\colorbox{green}{\color[gray]{0.75}FO}%
\colorbox{green}{\color[gray]{0.75}FO}%
\colorbox{green}{\color[gray]{0.75}FO}%
\colorbox{green}{\color[gray]{0.75}FO}%
\colorbox{green}{\color[gray]{0.75}FO}%
\colorbox{green}{\color[gray]{0.75}FO}%
\colorbox{green}{\color[gray]{0.75}FO}%
\colorbox{green}{\color[gray]{0.75}FO}%
\colorbox{green}{\color[gray]{0.75}FO}%
\colorbox{green}{\color[gray]{0.75}FO}%
\colorbox{green}{\color[gray]{0.75}FO}%
\colorbox{green}{\color[gray]{0.75}FO}%
\colorbox{green}{\color[gray]{0.75}FO}%
\colorbox{green}{\color[gray]{0.75}FO}%
\colorbox{green}{\color[gray]{0.75}FO}%
\colorbox{green}{\color[gray]{0.75}FO}%
\colorbox{green}{\color[gray]{0.75}FO}%
\colorbox{green}{\color[gray]{0.75}FO}%
\colorbox{green}{\color[gray]{0.75}FO}%
\colorbox{green}{\color[gray]{0.75}FO}%
\colorbox{green}{\color[gray]{0.75}FO}%
\colorbox{green}{\color[gray]{0.75}FO}%
\colorbox{green}{\color[gray]{0.75}FO}%
\colorbox{green}{\color[gray]{0.75}FO}%
\colorbox{green}{\color[gray]{0.75}FO}%
\colorbox{green}{\color[gray]{0.75}FO}%
\colorbox{green}{\color[gray]{0.75}FO}%
\colorbox{green}{\color[gray]{0.75}FO}%
\colorbox{green}{\color[gray]{0.75}FO}%
\colorbox{green}{\color[gray]{0.75}FO}%
\colorbox{green}{\color[gray]{0.75}FO}%
\colorbox{green}{\color[gray]{0.75}FO}%
\colorbox{green}{\color[gray]{0.75}FO}%
\colorbox{green}{\color[gray]{0.75}FO}%
\colorbox{green}{\color[gray]{0.75}FO}%
\colorbox{green}{\color[gray]{0.75}FO}%
\colorbox{green}{\color[gray]{0.75}FO}%
\colorbox{green}{\color[gray]{0.75}FO}%
\colorbox{green}{\color[gray]{0.75}FO}%
\colorbox{green}{\color[gray]{0.75}FO}%
\colorbox{green}{\color[gray]{0.75}FO}%
\colorbox{green}{\color[gray]{0.75}FO}%
\colorbox{green}{\color[gray]{0.75}FO}%
\colorbox{green}{\color[gray]{0.75}FO}%
\colorbox{green}{\color[gray]{0.75}FO}%
\colorbox{green}{\color[gray]{0.75}FO}%
\colorbox{green}{\color[gray]{0.75}FO}%
\colorbox{green}{\color[gray]{0.75}FO}%
\colorbox{green}{\color[gray]{0.75}FO}%
\colorbox{green}{\color[gray]{0.75}FO}%
\colorbox{green}{\color[gray]{0.75}FO}%
\colorbox{green}{\color[gray]{0.75}FO}%
\colorbox{green}{\color[gray]{0.75}FO}%
\colorbox{green}{\color[gray]{0.75}FO}%
\colorbox{green}{\color[gray]{0.75}FO}%
\colorbox{green}{\color[gray]{0.75}FO}%
\colorbox{green}{\color[gray]{0.75}FO}%
\colorbox{green}{\color[gray]{0.75}FO}%
\colorbox{green}{\color[gray]{0.75}FO}%
\colorbox{green}{\color[gray]{0.75}FO}%
\colorbox{green}{\color[gray]{0.75}FO}%
\colorbox{green}{\color[gray]{0.75}FO}%
\colorbox{green}{\color[gray]{0.75}FO}%
\colorbox{green}{\color[gray]{0.75}FO}%
\colorbox{green}{\color[gray]{0.75}FO}%
\colorbox{green}{\color[gray]{0.75}FO}%
\colorbox{green}{\color[gray]{0.75}FO}%
\colorbox{green}{\color[gray]{0.75}FO}%
\colorbox{green}{\color[gray]{0.75}FO}%
\colorbox{green}{\color[gray]{0.75}FO}%
\colorbox{green}{\color[gray]{0.75}FO}%
\colorbox{green}{\color[gray]{0.75}FO}%
\colorbox{green}{\color[gray]{0.75}FO}%
\colorbox{green}{\color[gray]{0.75}FO}%
\colorbox{green}{\color[gray]{0.75}FO}%
\colorbox{green}{\color[gray]{0.75}FO}%
\colorbox{green}{\color[gray]{0.75}FO}%
\\
\colorbox{green}{\color[gray]{0.75}FO}%
\colorbox{green}{\color[gray]{0.75}FO}%
\colorbox{green}{\color[gray]{0.75}FO}%
\colorbox{green}{\color[gray]{0.75}FO}%
\colorbox{green}{\color[gray]{0.75}FO}%
\colorbox{green}{\color[gray]{0.75}FO}%
\colorbox{green}{\color[gray]{0.75}FO}%
\colorbox{green}{\color[gray]{0.75}FO}%
\colorbox{green}{\color[gray]{0.75}FO}%
\colorbox{green}{\color[gray]{0.75}FO}%
\colorbox{green}{\color[gray]{0.75}FO}%
\colorbox{green}{\color[gray]{0.75}FO}%
\colorbox{green}{\color[gray]{0.75}FO}%
\colorbox{green}{\color[gray]{0.75}FO}%
\colorbox{green}{\color[gray]{0.75}FO}%
\colorbox{green}{\color[gray]{0.75}FO}%
\colorbox{green}{\color[gray]{0.75}FO}%
\colorbox{green}{\color[gray]{0.75}FO}%
\colorbox{green}{\color[gray]{0.75}FO}%
\colorbox{green}{\color[gray]{0.75}FO}%
\colorbox{green}{\color[gray]{0.75}FO}%
\colorbox{green}{\color[gray]{0.75}FO}%
\colorbox{green}{\color[gray]{0.75}FO}%
\colorbox{green}{\color[gray]{0.75}FO}%
\colorbox{green}{\color[gray]{0.75}FO}%
\colorbox{green}{\color[gray]{0.75}FO}%
\colorbox{green}{\color[gray]{0.75}FO}%
\colorbox{green}{\color[gray]{0.75}FO}%
\colorbox{green}{\color[gray]{0.75}FO}%
\colorbox{green}{\color[gray]{0.75}FO}%
\colorbox{green}{\color[gray]{0.75}FO}%
\colorbox{green}{\color[gray]{0.75}FO}%
\colorbox{green}{\color[gray]{0.75}FO}%
\colorbox{green}{\color[gray]{0.75}FO}%
\colorbox{green}{\color[gray]{0.75}FO}%
\colorbox{green}{\color[gray]{0.75}FO}%
\colorbox{green}{\color[gray]{0.75}FO}%
\colorbox{green}{\color[gray]{0.75}FO}%
\colorbox{green}{\color[gray]{0.75}FO}%
\colorbox{green}{\color[gray]{0.75}FO}%
\colorbox{green}{\color[gray]{0.75}FO}%
\colorbox{green}{\color[gray]{0.75}FO}%
\colorbox{green}{\color[gray]{0.75}FO}%
\colorbox{green}{\color[gray]{0.75}FO}%
\colorbox{green}{\color[gray]{0.75}FO}%
\colorbox{green}{\color[gray]{0.75}FO}%
\colorbox{green}{\color[gray]{0.75}FO}%
\colorbox{green}{\color[gray]{0.75}FO}%
\colorbox{green}{\color[gray]{0.75}FO}%
\colorbox{green}{\color[gray]{0.75}FO}%
\colorbox{green}{\color[gray]{0.75}FO}%
\colorbox{green}{\color[gray]{0.75}FO}%
\colorbox{green}{\color[gray]{0.75}FO}%
\colorbox{green}{\color[gray]{0.75}FO}%
\colorbox{green}{\color[gray]{0.75}FO}%
\colorbox{green}{\color[gray]{0.75}FO}%
\colorbox{green}{\color[gray]{0.75}FO}%
\colorbox{green}{\color[gray]{0.75}FO}%
\colorbox{green}{\color[gray]{0.75}FO}%
\colorbox{green}{\color[gray]{0.75}FO}%
\colorbox{green}{\color[gray]{0.75}FO}%
\colorbox{green}{\color[gray]{0.75}FO}%
\colorbox{green}{\color[gray]{0.75}FO}%
\colorbox{green}{\color[gray]{0.75}FO}%
\colorbox{green}{\color[gray]{0.75}FO}%
\colorbox{green}{\color[gray]{0.75}FO}%
\colorbox{green}{\color[gray]{0.75}FO}%
\colorbox{green}{\color[gray]{0.75}FO}%
\colorbox{green}{\color[gray]{0.75}FO}%
\colorbox{green}{\color[gray]{0.75}FO}%
\colorbox{green}{\color[gray]{0.75}FO}%
\colorbox{green}{\color[gray]{0.75}FO}%
\colorbox{green}{\color[gray]{0.75}FO}%
\colorbox{green}{\color[gray]{0.75}FO}%
\colorbox{green}{\color[gray]{0.75}FO}%
\colorbox{green}{\color[gray]{0.75}FO}%
\colorbox{green}{\color[gray]{0.75}FO}%
\colorbox{green}{\color[gray]{0.75}FO}%
\colorbox{green}{\color[gray]{0.75}FO}%
\colorbox{green}{\color[gray]{0.75}FO}%
\colorbox{green}{\color[gray]{0.75}FO}%
\colorbox{green}{\color[gray]{0.75}FO}%
\colorbox{green}{\color[gray]{0.75}FO}%
\colorbox{green}{\color[gray]{0.75}FO}%
\colorbox{green}{\color[gray]{0.75}FO}%
\colorbox{green}{\color[gray]{0.75}FO}%
\colorbox{green}{\color[gray]{0.75}FO}%
\colorbox{green}{\color[gray]{0.75}FO}%
\colorbox{green}{\color[gray]{0.75}FO}%
\colorbox{green}{\color[gray]{0.75}FO}%
\colorbox{green}{\color[gray]{0.75}FO}%
\colorbox{green}{\color[gray]{0.75}FO}%
\colorbox{green}{\color[gray]{0.75}FO}%
\colorbox{green}{\color[gray]{0.75}FO}%
\colorbox{green}{\color[gray]{0.75}FO}%
\colorbox{green}{\color[gray]{0.75}FO}%
\colorbox{green}{\color[gray]{0.75}FO}%
\colorbox{green}{\color[gray]{0.75}FO}%
\colorbox{green}{\color[gray]{0.75}FO}%
\colorbox{green}{\color[gray]{0.75}FO}%
\\
\colorbox{green}{\color[gray]{0.75}FO}%
\colorbox{green}{\color[gray]{0.75}FO}%
\colorbox{green}{\color[gray]{0.75}FO}%
\colorbox{green}{\color[gray]{0.75}FO}%
\colorbox{green}{\color[gray]{0.75}FO}%
\colorbox{green}{\color[gray]{0.75}FO}%
\colorbox{green}{\color[gray]{0.75}FO}%
\colorbox{green}{\color[gray]{0.75}FO}%
\colorbox{green}{\color[gray]{0.75}FO}%
\colorbox{green}{\color[gray]{0.75}FO}%
\colorbox{green}{\color[gray]{0.75}FO}%
\colorbox{green}{\color[gray]{0.75}FO}%
\colorbox{green}{\color[gray]{0.75}FO}%
\colorbox{green}{\color[gray]{0.75}FO}%
\colorbox{green}{\color[gray]{0.75}FO}%
\colorbox{green}{\color[gray]{0.75}FO}%
\colorbox{green}{\color[gray]{0.75}FO}%
\colorbox{green}{\color[gray]{0.75}FO}%
\colorbox{green}{\color[gray]{0.75}FO}%
\colorbox{green}{\color[gray]{0.75}FO}%
\colorbox{green}{\color[gray]{0.75}FO}%
\colorbox{green}{\color[gray]{0.75}FO}%
\colorbox{green}{\color[gray]{0.75}FO}%
\colorbox{green}{\color[gray]{0.75}FO}%
\colorbox{green}{\color[gray]{0.75}FO}%
\colorbox{green}{\color[gray]{0.75}FO}%
\colorbox{green}{\color[gray]{0.75}FO}%
\colorbox{green}{\color[gray]{0.75}FO}%
\colorbox{green}{\color[gray]{0.75}FO}%
\colorbox{green}{\color[gray]{0.75}FO}%
\colorbox{green}{\color[gray]{0.75}FO}%
\colorbox{green}{\color[gray]{0.75}FO}%
\colorbox{green}{\color[gray]{0.75}FO}%
\colorbox{green}{\color[gray]{0.75}FO}%
\colorbox{green}{\color[gray]{0.75}FO}%
\colorbox{green}{\color[gray]{0.75}FO}%
\colorbox{green}{\color[gray]{0.75}FO}%
\colorbox{green}{\color[gray]{0.75}FO}%
\colorbox{green}{\color[gray]{0.75}FO}%
\colorbox{green}{\color[gray]{0.75}FO}%
\colorbox{green}{\color[gray]{0.75}FO}%
\colorbox{green}{\color[gray]{0.75}FO}%
\colorbox{green}{\color[gray]{0.75}FO}%
\colorbox{green}{\color[gray]{0.75}FO}%
\colorbox{green}{\color[gray]{0.75}FO}%
\colorbox{green}{\color[gray]{0.75}FO}%
\colorbox{green}{\color[gray]{0.75}FO}%
\colorbox{green}{\color[gray]{0.75}FO}%
\colorbox{green}{\color[gray]{0.75}FO}%
\colorbox{green}{\color[gray]{0.75}FO}%
\colorbox{green}{\color[gray]{0.75}FO}%
\colorbox{green}{\color[gray]{0.75}FO}%
\colorbox{green}{\color[gray]{0.75}FO}%
\colorbox{green}{\color[gray]{0.75}FO}%
\colorbox{green}{\color[gray]{0.75}FO}%
\colorbox{green}{\color[gray]{0.75}FO}%
\colorbox{green}{\color[gray]{0.75}FO}%
\colorbox{green}{\color[gray]{0.75}FO}%
\colorbox{green}{\color[gray]{0.75}FO}%
\colorbox{green}{\color[gray]{0.75}FO}%
\colorbox{green}{\color[gray]{0.75}FO}%
\colorbox{green}{\color[gray]{0.75}FO}%
\colorbox{green}{\color[gray]{0.75}FO}%
\colorbox{green}{\color[gray]{0.75}FO}%
\colorbox{green}{\color[gray]{0.75}FO}%
\colorbox{green}{\color[gray]{0.75}FO}%
\colorbox{green}{\color[gray]{0.75}FO}%
\colorbox{green}{\color[gray]{0.75}FO}%
\colorbox{green}{\color[gray]{0.75}FO}%
\colorbox{green}{\color[gray]{0.75}FO}%
\colorbox{green}{\color[gray]{0.75}FO}%
\colorbox{green}{\color[gray]{0.75}FO}%
\colorbox{green}{\color[gray]{0.75}FO}%
\colorbox{green}{\color[gray]{0.75}FO}%
\colorbox{green}{\color[gray]{0.75}FO}%
\colorbox{green}{\color[gray]{0.75}FO}%
\colorbox{green}{\color[gray]{0.75}FO}%
\colorbox{green}{\color[gray]{0.75}FO}%
\colorbox{green}{\color[gray]{0.75}FO}%
\colorbox{green}{\color[gray]{0.75}FO}%
\colorbox{green}{\color[gray]{0.75}FO}%
\colorbox{green}{\color[gray]{0.75}FO}%
\colorbox{green}{\color[gray]{0.75}FO}%
\colorbox{green}{\color[gray]{0.75}FO}%
\colorbox{green}{\color[gray]{0.75}FO}%
\colorbox{green}{\color[gray]{0.75}FO}%
\colorbox{green}{\color[gray]{0.75}FO}%
\colorbox{green}{\color[gray]{0.75}FO}%
\colorbox{green}{\color[gray]{0.75}FO}%
\colorbox{green}{\color[gray]{0.75}FO}%
\colorbox{green}{\color[gray]{0.75}FO}%
\colorbox{green}{\color[gray]{0.75}FO}%
\colorbox{green}{\color[gray]{0.75}FO}%
\colorbox{green}{\color[gray]{0.75}FO}%
\colorbox{green}{\color[gray]{0.75}FO}%
\colorbox{green}{\color[gray]{0.75}FO}%
\colorbox{green}{\color[gray]{0.75}FO}%
\colorbox{green}{\color[gray]{0.75}FO}%
\colorbox{green}{\color[gray]{0.75}FO}%
\colorbox{green}{\color[gray]{0.75}FO}%
\\
\colorbox{green}{\color[gray]{0.75}FO}%
\colorbox{green}{\color[gray]{0.75}FO}%
\colorbox{green}{\color[gray]{0.75}FO}%
\colorbox{green}{\color[gray]{0.75}FO}%
\colorbox{green}{\color[gray]{0.75}FO}%
\colorbox{green}{\color[gray]{0.75}FO}%
\colorbox{green}{\color[gray]{0.75}FO}%
\colorbox{green}{\color[gray]{0.75}FO}%
\colorbox{green}{\color[gray]{0.75}FO}%
\colorbox{green}{\color[gray]{0.75}FO}%
\colorbox{green}{\color[gray]{0.75}FO}%
\colorbox{green}{\color[gray]{0.75}FO}%
\colorbox{green}{\color[gray]{0.75}FO}%
\colorbox{green}{\color[gray]{0.75}FO}%
\colorbox{green}{\color[gray]{0.75}FO}%
\colorbox{green}{\color[gray]{0.75}FO}%
\colorbox{green}{\color[gray]{0.75}FO}%
\colorbox{green}{\color[gray]{0.75}FO}%
\colorbox{green}{\color[gray]{0.75}FO}%
\colorbox{green}{\color[gray]{0.75}FO}%
\colorbox{green}{\color[gray]{0.75}FO}%
\colorbox{green}{\color[gray]{0.75}FO}%
\colorbox{green}{\color[gray]{0.75}FO}%
\colorbox{green}{\color[gray]{0.75}FO}%
\colorbox{green}{\color[gray]{0.75}FO}%
\colorbox{green}{\color[gray]{0.75}FO}%
\colorbox{green}{\color[gray]{0.75}FO}%
\colorbox{green}{\color[gray]{0.75}FO}%
\colorbox{green}{\color[gray]{0.75}FO}%
\colorbox{green}{\color[gray]{0.75}FO}%
\colorbox{green}{\color[gray]{0.75}FO}%
\colorbox{green}{\color[gray]{0.75}FO}%
\colorbox{green}{\color[gray]{0.75}FO}%
\colorbox{green}{\color[gray]{0.75}FO}%
\colorbox{green}{\color[gray]{0.75}FO}%
\colorbox{green}{\color[gray]{0.75}FO}%
\colorbox{green}{\color[gray]{0.75}FO}%
\colorbox{green}{\color[gray]{0.75}FO}%
\colorbox{green}{\color[gray]{0.75}FO}%
\colorbox{green}{\color[gray]{0.75}FO}%
\colorbox{green}{\color[gray]{0.75}FO}%
\colorbox{green}{\color[gray]{0.75}FO}%
\colorbox{green}{\color[gray]{0.75}FO}%
\colorbox{green}{\color[gray]{0.75}FO}%
\colorbox{green}{\color[gray]{0.75}FO}%
\colorbox{green}{\color[gray]{0.75}FO}%
\colorbox{green}{\color[gray]{0.75}FO}%
\colorbox{green}{\color[gray]{0.75}FO}%
\colorbox{green}{\color[gray]{0.75}FO}%
\colorbox{green}{\color[gray]{0.75}FO}%
\colorbox{green}{\color[gray]{0.75}FO}%
\colorbox{green}{\color[gray]{0.75}FO}%
\colorbox{green}{\color[gray]{0.75}FO}%
\colorbox{green}{\color[gray]{0.75}FO}%
\colorbox{green}{\color[gray]{0.75}FO}%
\colorbox{green}{\color[gray]{0.75}FO}%
\colorbox{green}{\color[gray]{0.75}FO}%
\colorbox{green}{\color[gray]{0.75}FO}%
\colorbox{green}{\color[gray]{0.75}FO}%
\colorbox{green}{\color[gray]{0.75}FO}%
\colorbox{green}{\color[gray]{0.75}FO}%
\colorbox{green}{\color[gray]{0.75}FO}%
\colorbox{green}{\color[gray]{0.75}FO}%
\colorbox{green}{\color[gray]{0.75}FO}%
\colorbox{green}{\color[gray]{0.75}FO}%
\colorbox{green}{\color[gray]{0.75}FO}%
\colorbox{green}{\color[gray]{0.75}FO}%
\colorbox{green}{\color[gray]{0.75}FO}%
\colorbox{green}{\color[gray]{0.75}FO}%
\colorbox{green}{\color[gray]{0.75}FO}%
\colorbox{green}{\color[gray]{0.75}FO}%
\colorbox{green}{\color[gray]{0.75}FO}%
\colorbox{green}{\color[gray]{0.75}FO}%
\colorbox{green}{\color[gray]{0.75}FO}%
\colorbox{green}{\color[gray]{0.75}FO}%
\colorbox{green}{\color[gray]{0.75}FO}%
\colorbox{green}{\color[gray]{0.75}FO}%
\colorbox{green}{\color[gray]{0.75}FO}%
\colorbox{green}{\color[gray]{0.75}FO}%
\colorbox{green}{\color[gray]{0.75}FO}%
\colorbox{green}{\color[gray]{0.75}FO}%
\colorbox{green}{\color[gray]{0.75}FO}%
\colorbox{green}{\color[gray]{0.75}FO}%
\colorbox{green}{\color[gray]{0.75}FO}%
\colorbox{green}{\color[gray]{0.75}FO}%
\colorbox{green}{\color[gray]{0.75}FO}%
\colorbox{green}{\color[gray]{0.75}FO}%
\colorbox{green}{\color[gray]{0.75}FO}%
\colorbox{green}{\color[gray]{0.75}FO}%
\colorbox{green}{\color[gray]{0.75}FO}%
\colorbox{green}{\color[gray]{0.75}FO}%
\colorbox{green}{\color[gray]{0.75}FO}%
\colorbox{green}{\color[gray]{0.75}FO}%
\colorbox{green}{\color[gray]{0.75}FO}%
\colorbox{green}{\color[gray]{0.75}FO}%
\colorbox{green}{\color[gray]{0.75}FO}%
\colorbox{green}{\color[gray]{0.75}FO}%
\colorbox{green}{\color[gray]{0.75}FO}%
\colorbox{green}{\color[gray]{0.75}FO}%
\colorbox{green}{\color[gray]{0.75}FO}%
\\
\colorbox{green}{\color[gray]{0.75}FO}%
\colorbox{green}{\color[gray]{0.75}FO}%
\colorbox{green}{\color[gray]{0.75}FO}%
\colorbox{green}{\color[gray]{0.75}FO}%
\colorbox{green}{\color[gray]{0.75}FO}%
\colorbox{green}{\color[gray]{0.75}FO}%
\colorbox{green}{\color[gray]{0.75}FO}%
\colorbox{green}{\color[gray]{0.75}FO}%
\colorbox{green}{\color[gray]{0.75}FO}%
\colorbox{green}{\color[gray]{0.75}FO}%
\colorbox{green}{\color[gray]{0.75}FO}%
\colorbox{green}{\color[gray]{0.75}FO}%
\colorbox{green}{\color[gray]{0.75}FO}%
\colorbox{green}{\color[gray]{0.75}FO}%
\colorbox{green}{\color[gray]{0.75}FO}%
\colorbox{green}{\color[gray]{0.75}FO}%
\colorbox{green}{\color[gray]{0.75}FO}%
\colorbox{green}{\color[gray]{0.75}FO}%
\colorbox{green}{\color[gray]{0.75}FO}%
\colorbox{green}{\color[gray]{0.75}FO}%
\colorbox{green}{\color[gray]{0.75}FO}%
\colorbox{green}{\color[gray]{0.75}FO}%
\colorbox{green}{\color[gray]{0.75}FO}%
\colorbox{green}{\color[gray]{0.75}FO}%
\colorbox{green}{\color[gray]{0.75}FO}%
\colorbox{green}{\color[gray]{0.75}FO}%
\colorbox{green}{\color[gray]{0.75}FO}%
\colorbox{green}{\color[gray]{0.75}FO}%
\colorbox{green}{\color[gray]{0.75}FO}%
\colorbox{green}{\color[gray]{0.75}FO}%
\colorbox{green}{\color[gray]{0.75}FO}%
\colorbox{green}{\color[gray]{0.75}FO}%
\colorbox{green}{\color[gray]{0.75}FO}%
\colorbox{green}{\color[gray]{0.75}FO}%
\colorbox{green}{\color[gray]{0.75}FO}%
\colorbox{green}{\color[gray]{0.75}FO}%
\colorbox{green}{\color[gray]{0.75}FO}%
\colorbox{green}{\color[gray]{0.75}FO}%
\colorbox{green}{\color[gray]{0.75}FO}%
\colorbox{green}{\color[gray]{0.75}FO}%
\colorbox{green}{\color[gray]{0.75}FO}%
\colorbox{green}{\color[gray]{0.75}FO}%
\colorbox{green}{\color[gray]{0.75}FO}%
\colorbox{green}{\color[gray]{0.75}FO}%
\colorbox{green}{\color[gray]{0.75}FO}%
\colorbox{green}{\color[gray]{0.75}FO}%
\colorbox{green}{\color[gray]{0.75}FO}%
\colorbox{green}{\color[gray]{0.75}FO}%
\colorbox{green}{\color[gray]{0.75}FO}%
\colorbox{green}{\color[gray]{0.75}FO}%
\colorbox{green}{\color[gray]{0.75}FO}%
\colorbox{green}{\color[gray]{0.75}FO}%
\colorbox{green}{\color[gray]{0.75}FO}%
\colorbox{green}{\color[gray]{0.75}FO}%
\colorbox{green}{\color[gray]{0.75}FO}%
\colorbox{green}{\color[gray]{0.75}FO}%
\colorbox{green}{\color[gray]{0.75}FO}%
\colorbox{green}{\color[gray]{0.75}FO}%
\colorbox{green}{\color[gray]{0.75}FO}%
\colorbox{green}{\color[gray]{0.75}FO}%
\colorbox{green}{\color[gray]{0.75}FO}%
\colorbox{green}{\color[gray]{0.75}FO}%
\colorbox{green}{\color[gray]{0.75}FO}%
\colorbox{green}{\color[gray]{0.75}FO}%
\colorbox{green}{\color[gray]{0.75}FO}%
\colorbox{green}{\color[gray]{0.75}FO}%
\colorbox{green}{\color[gray]{0.75}FO}%
\colorbox{green}{\color[gray]{0.75}FO}%
\colorbox{green}{\color[gray]{0.75}FO}%
\colorbox{green}{\color[gray]{0.75}FO}%
\colorbox{green}{\color[gray]{0.75}FO}%
\colorbox{green}{\color[gray]{0.75}FO}%
\colorbox{green}{\color[gray]{0.75}FO}%
\colorbox{green}{\color[gray]{0.75}FO}%
\colorbox{green}{\color[gray]{0.75}FO}%
\colorbox{green}{\color[gray]{0.75}FO}%
\colorbox{green}{\color[gray]{0.75}FO}%
\colorbox{green}{\color[gray]{0.75}FO}%
\colorbox{green}{\color[gray]{0.75}FO}%
\colorbox{green}{\color[gray]{0.75}FO}%
\colorbox{green}{\color[gray]{0.75}FO}%
\colorbox{green}{\color[gray]{0.75}FO}%
\colorbox{green}{\color[gray]{0.75}FO}%
\colorbox{green}{\color[gray]{0.75}FO}%
\colorbox{green}{\color[gray]{0.75}FO}%
\colorbox{green}{\color[gray]{0.75}FO}%
\colorbox{green}{\color[gray]{0.75}FO}%
\colorbox{green}{\color[gray]{0.75}FO}%
\colorbox{green}{\color[gray]{0.75}FO}%
\colorbox{green}{\color[gray]{0.75}FO}%
\colorbox{green}{\color[gray]{0.75}FO}%
\colorbox{green}{\color[gray]{0.75}FO}%
\colorbox{green}{\color[gray]{0.75}FO}%
\colorbox{green}{\color[gray]{0.75}FO}%
\colorbox{green}{\color[gray]{0.75}FO}%
\colorbox{green}{\color[gray]{0.75}FO}%
\colorbox{green}{\color[gray]{0.75}FO}%
\colorbox{green}{\color[gray]{0.75}FO}%
\colorbox{green}{\color[gray]{0.75}FO}%
\colorbox{green}{\color[gray]{0.75}FO}%
\\
\colorbox{green}{\color[gray]{0.75}FO}%
\colorbox{green}{\color[gray]{0.75}FO}%
\colorbox{green}{\color[gray]{0.75}FO}%
\colorbox{green}{\color[gray]{0.75}FO}%
\colorbox{green}{\color[gray]{0.75}FO}%
\colorbox{green}{\color[gray]{0.75}FO}%
\colorbox{green}{\color[gray]{0.75}FO}%
\colorbox{green}{\color[gray]{0.75}FO}%
\colorbox{green}{\color[gray]{0.75}FO}%
\colorbox{green}{\color[gray]{0.75}FO}%
\colorbox{green}{\color[gray]{0.75}FO}%
\colorbox{green}{\color[gray]{0.75}FO}%
\colorbox{green}{\color[gray]{0.75}FO}%
\colorbox{green}{\color[gray]{0.75}FO}%
\colorbox{green}{\color[gray]{0.75}FO}%
\colorbox{green}{\color[gray]{0.75}FO}%
\colorbox{green}{\color[gray]{0.75}FO}%
\colorbox{green}{\color[gray]{0.75}FO}%
\colorbox{green}{\color[gray]{0.75}FO}%
\colorbox{green}{\color[gray]{0.75}FO}%
\colorbox{green}{\color[gray]{0.75}FO}%
\colorbox{green}{\color[gray]{0.75}FO}%
\colorbox{green}{\color[gray]{0.75}FO}%
\colorbox{green}{\color[gray]{0.75}FO}%
\colorbox{green}{\color[gray]{0.75}FO}%
\colorbox{green}{\color[gray]{0.75}FO}%
\colorbox{green}{\color[gray]{0.75}FO}%
\colorbox{green}{\color[gray]{0.75}FO}%
\colorbox{green}{\color[gray]{0.75}FO}%
\colorbox{green}{\color[gray]{0.75}FO}%
\colorbox{green}{\color[gray]{0.75}FO}%
\colorbox{green}{\color[gray]{0.75}FO}%
\colorbox{green}{\color[gray]{0.75}FO}%
\colorbox{green}{\color[gray]{0.75}FO}%
\colorbox{green}{\color[gray]{0.75}FO}%
\colorbox{green}{\color[gray]{0.75}FO}%
\colorbox{green}{\color[gray]{0.75}FO}%
\colorbox{green}{\color[gray]{0.75}FO}%
\colorbox{green}{\color[gray]{0.75}FO}%
\colorbox{green}{\color[gray]{0.75}FO}%
\colorbox{green}{\color[gray]{0.75}FO}%
\colorbox{green}{\color[gray]{0.75}FO}%
\colorbox{green}{\color[gray]{0.75}FO}%
\colorbox{green}{\color[gray]{0.75}FO}%
\colorbox{green}{\color[gray]{0.75}FO}%
\colorbox{green}{\color[gray]{0.75}FO}%
\colorbox{green}{\color[gray]{0.75}FO}%
\colorbox{green}{\color[gray]{0.75}FO}%
\colorbox{green}{\color[gray]{0.75}FO}%
\colorbox{green}{\color[gray]{0.75}FO}%
\colorbox{green}{\color[gray]{0.75}FO}%
\colorbox{green}{\color[gray]{0.75}FO}%
\colorbox{green}{\color[gray]{0.75}FO}%
\colorbox{green}{\color[gray]{0.75}FO}%
\colorbox{green}{\color[gray]{0.75}FO}%
\colorbox{green}{\color[gray]{0.75}FO}%
\colorbox{green}{\color[gray]{0.75}FO}%
\colorbox{green}{\color[gray]{0.75}FO}%
\colorbox{green}{\color[gray]{0.75}FO}%
\colorbox{green}{\color[gray]{0.75}FO}%
\colorbox{green}{\color[gray]{0.75}FO}%
\colorbox{green}{\color[gray]{0.75}FO}%
\colorbox{green}{\color[gray]{0.75}FO}%
\colorbox{green}{\color[gray]{0.75}FO}%
\colorbox{green}{\color[gray]{0.75}FO}%
\colorbox{green}{\color[gray]{0.75}FO}%
\colorbox{green}{\color[gray]{0.75}FO}%
\colorbox{green}{\color[gray]{0.75}FO}%
\colorbox{green}{\color[gray]{0.75}FO}%
\colorbox{green}{\color[gray]{0.75}FO}%
\colorbox{green}{\color[gray]{0.75}FO}%
\colorbox{green}{\color[gray]{0.75}FO}%
\colorbox{green}{\color[gray]{0.75}FO}%
\colorbox{green}{\color[gray]{0.75}FO}%
\colorbox{green}{\color[gray]{0.75}FO}%
\colorbox{green}{\color[gray]{0.75}FO}%
\colorbox{green}{\color[gray]{0.75}FO}%
\colorbox{green}{\color[gray]{0.75}FO}%
\colorbox{green}{\color[gray]{0.75}FO}%
\colorbox{green}{\color[gray]{0.75}FO}%
\colorbox{green}{\color[gray]{0.75}FO}%
\colorbox{green}{\color[gray]{0.75}FO}%
\colorbox{green}{\color[gray]{0.75}FO}%
\colorbox{green}{\color[gray]{0.75}FO}%
\colorbox{green}{\color[gray]{0.75}FO}%
\colorbox{green}{\color[gray]{0.75}FO}%
\colorbox{green}{\color[gray]{0.75}FO}%
\colorbox{green}{\color[gray]{0.75}FO}%
\colorbox{green}{\color[gray]{0.75}FO}%
\colorbox{green}{\color[gray]{0.75}FO}%
\colorbox{green}{\color[gray]{0.75}FO}%
\colorbox{green}{\color[gray]{0.75}FO}%
\colorbox{green}{\color[gray]{0.75}FO}%
\colorbox{green}{\color[gray]{0.75}FO}%
\colorbox{green}{\color[gray]{0.75}FO}%
\colorbox{green}{\color[gray]{0.75}FO}%
\colorbox{green}{\color[gray]{0.75}FO}%
\colorbox{green}{\color[gray]{0.75}FO}%
\colorbox{green}{\color[gray]{0.75}FO}%
\colorbox{green}{\color[gray]{0.75}FO}%
\\
\colorbox{green}{\color[gray]{0.75}FO}%
\colorbox{green}{\color[gray]{0.75}FO}%
\colorbox{green}{\color[gray]{0.75}FO}%
\colorbox{green}{\color[gray]{0.75}FO}%
\colorbox{green}{\color[gray]{0.75}FO}%
\colorbox{green}{\color[gray]{0.75}FO}%
\colorbox{green}{\color[gray]{0.75}FO}%
\colorbox{green}{\color[gray]{0.75}FO}%
\colorbox{green}{\color[gray]{0.75}FO}%
\colorbox{green}{\color[gray]{0.75}FO}%
\colorbox{green}{\color[gray]{0.75}FO}%
\colorbox{green}{\color[gray]{0.75}FO}%
\colorbox{green}{\color[gray]{0.75}FO}%
\colorbox{green}{\color[gray]{0.75}FO}%
\colorbox{green}{\color[gray]{0.75}FO}%
\colorbox{green}{\color[gray]{0.75}FO}%
\colorbox{green}{\color[gray]{0.75}FO}%
\colorbox{green}{\color[gray]{0.75}FO}%
\colorbox{green}{\color[gray]{0.75}FO}%
\colorbox{green}{\color[gray]{0.75}FO}%
\colorbox{green}{\color[gray]{0.75}FO}%
\colorbox{green}{\color[gray]{0.75}FO}%
\colorbox{green}{\color[gray]{0.75}FO}%
\colorbox{green}{\color[gray]{0.75}FO}%
\colorbox{green}{\color[gray]{0.75}FO}%
\colorbox{green}{\color[gray]{0.75}FO}%
\colorbox{green}{\color[gray]{0.75}FO}%
\colorbox{green}{\color[gray]{0.75}FO}%
\colorbox{green}{\color[gray]{0.75}FO}%
\colorbox{green}{\color[gray]{0.75}FO}%
\colorbox{green}{\color[gray]{0.75}FO}%
\colorbox{green}{\color[gray]{0.75}FO}%
\colorbox{green}{\color[gray]{0.75}FO}%
\colorbox{green}{\color[gray]{0.75}FO}%
\colorbox{green}{\color[gray]{0.75}FO}%
\colorbox{green}{\color[gray]{0.75}FO}%
\colorbox{green}{\color[gray]{0.75}FO}%
\colorbox{green}{\color[gray]{0.75}FO}%
\colorbox{green}{\color[gray]{0.75}FO}%
\colorbox{green}{\color[gray]{0.75}FO}%
\colorbox{green}{\color[gray]{0.75}FO}%
\colorbox{green}{\color[gray]{0.75}FO}%
\colorbox{green}{\color[gray]{0.75}FO}%
\colorbox{green}{\color[gray]{0.75}FO}%
\colorbox{green}{\color[gray]{0.75}FO}%
\colorbox{green}{\color[gray]{0.75}FO}%
\colorbox{green}{\color[gray]{0.75}FO}%
\colorbox{green}{\color[gray]{0.75}FO}%
\colorbox{green}{\color[gray]{0.75}FO}%
\colorbox{green}{\color[gray]{0.75}FO}%
\colorbox{green}{\color[gray]{0.75}FO}%
\colorbox{green}{\color[gray]{0.75}FO}%
\colorbox{green}{\color[gray]{0.75}FO}%
\colorbox{green}{\color[gray]{0.75}FO}%
\colorbox{green}{\color[gray]{0.75}FO}%
\colorbox{green}{\color[gray]{0.75}FO}%
\colorbox{green}{\color[gray]{0.75}FO}%
\colorbox{green}{\color[gray]{0.75}FO}%
\colorbox{green}{\color[gray]{0.75}FO}%
\colorbox{green}{\color[gray]{0.75}FO}%
\colorbox{green}{\color[gray]{0.75}FO}%
\colorbox{green}{\color[gray]{0.75}FO}%
\colorbox{green}{\color[gray]{0.75}FO}%
\colorbox{green}{\color[gray]{0.75}FO}%
\colorbox{green}{\color[gray]{0.75}FO}%
\colorbox{green}{\color[gray]{0.75}FO}%
\colorbox{green}{\color[gray]{0.75}FO}%
\colorbox{green}{\color[gray]{0.75}FO}%
\colorbox{green}{\color[gray]{0.75}FO}%
\colorbox{green}{\color[gray]{0.75}FO}%
\colorbox{green}{\color[gray]{0.75}FO}%
\colorbox{green}{\color[gray]{0.75}FO}%
\colorbox{green}{\color[gray]{0.75}FO}%
\colorbox{green}{\color[gray]{0.75}FO}%
\colorbox{green}{\color[gray]{0.75}FO}%
\colorbox{green}{\color[gray]{0.75}FO}%
\colorbox{green}{\color[gray]{0.75}FO}%
\colorbox{green}{\color[gray]{0.75}FO}%
\colorbox{green}{\color[gray]{0.75}FO}%
\colorbox{green}{\color[gray]{0.75}FO}%
\colorbox{green}{\color[gray]{0.75}FO}%
\colorbox{green}{\color[gray]{0.75}FO}%
\colorbox{green}{\color[gray]{0.75}FO}%
\colorbox{green}{\color[gray]{0.75}FO}%
\colorbox{green}{\color[gray]{0.75}FO}%
\colorbox{green}{\color[gray]{0.75}FO}%
\colorbox{green}{\color[gray]{0.75}FO}%
\colorbox{green}{\color[gray]{0.75}FO}%
\colorbox{green}{\color[gray]{0.75}FO}%
\colorbox{green}{\color[gray]{0.75}FO}%
\colorbox{green}{\color[gray]{0.75}FO}%
\colorbox{green}{\color[gray]{0.75}FO}%
\colorbox{green}{\color[gray]{0.75}FO}%
\colorbox{green}{\color[gray]{0.75}FO}%
\colorbox{green}{\color[gray]{0.75}FO}%
\colorbox{green}{\color[gray]{0.75}FO}%
\colorbox{green}{\color[gray]{0.75}FO}%
\colorbox{green}{\color[gray]{0.75}FO}%
\colorbox{green}{\color[gray]{0.75}FO}%
\colorbox{green}{\color[gray]{0.75}FO}%
\\
\colorbox{green}{\color[gray]{0.75}FO}%
\colorbox{green}{\color[gray]{0.75}FO}%
\colorbox{green}{\color[gray]{0.75}FO}%
\colorbox{green}{\color[gray]{0.75}FO}%
\colorbox{green}{\color[gray]{0.75}FO}%
\colorbox{green}{\color[gray]{0.75}FO}%
\colorbox{green}{\color[gray]{0.75}FO}%
\colorbox{green}{\color[gray]{0.75}FO}%
\colorbox{green}{\color[gray]{0.75}FO}%
\colorbox{green}{\color[gray]{0.75}FO}%
\colorbox{green}{\color[gray]{0.75}FO}%
\colorbox{green}{\color[gray]{0.75}FO}%
\colorbox{green}{\color[gray]{0.75}FO}%
\colorbox{green}{\color[gray]{0.75}FO}%
\colorbox{green}{\color[gray]{0.75}FO}%
\colorbox{green}{\color[gray]{0.75}FO}%
\colorbox{green}{\color[gray]{0.75}FO}%
\colorbox{green}{\color[gray]{0.75}FO}%
\colorbox{green}{\color[gray]{0.75}FO}%
\colorbox{green}{\color[gray]{0.75}FO}%
\colorbox{green}{\color[gray]{0.75}FO}%
\colorbox{green}{\color[gray]{0.75}FO}%
\colorbox{green}{\color[gray]{0.75}FO}%
\colorbox{green}{\color[gray]{0.75}FO}%
\colorbox{green}{\color[gray]{0.75}FO}%
\colorbox{green}{\color[gray]{0.75}FO}%
\colorbox{green}{\color[gray]{0.75}FO}%
\colorbox{green}{\color[gray]{0.75}FO}%
\colorbox{green}{\color[gray]{0.75}FO}%
\colorbox{green}{\color[gray]{0.75}FO}%
\colorbox{green}{\color[gray]{0.75}FO}%
\colorbox{green}{\color[gray]{0.75}FO}%
\colorbox{green}{\color[gray]{0.75}FO}%
\colorbox{green}{\color[gray]{0.75}FO}%
\colorbox{green}{\color[gray]{0.75}FO}%
\colorbox{green}{\color[gray]{0.75}FO}%
\colorbox{green}{\color[gray]{0.75}FO}%
\colorbox{green}{\color[gray]{0.75}FO}%
\colorbox{green}{\color[gray]{0.75}FO}%
\colorbox{green}{\color[gray]{0.75}FO}%
\colorbox{green}{\color[gray]{0.75}FO}%
\colorbox{green}{\color[gray]{0.75}FO}%
\colorbox{green}{\color[gray]{0.75}FO}%
\colorbox{green}{\color[gray]{0.75}FO}%
\colorbox{green}{\color[gray]{0.75}FO}%
\colorbox{green}{\color[gray]{0.75}FO}%
\colorbox{green}{\color[gray]{0.75}FO}%
\colorbox{green}{\color[gray]{0.75}FO}%
\colorbox{green}{\color[gray]{0.75}FO}%
\colorbox{green}{\color[gray]{0.75}FO}%
\colorbox{green}{\color[gray]{0.75}FO}%
\colorbox{green}{\color[gray]{0.75}FO}%
\colorbox{green}{\color[gray]{0.75}FO}%
\colorbox{green}{\color[gray]{0.75}FO}%
\colorbox{green}{\color[gray]{0.75}FO}%
\colorbox{green}{\color[gray]{0.75}FO}%
\colorbox{green}{\color[gray]{0.75}FO}%
\colorbox{green}{\color[gray]{0.75}FO}%
\colorbox{green}{\color[gray]{0.75}FO}%
\colorbox{green}{\color[gray]{0.75}FO}%
\colorbox{green}{\color[gray]{0.75}FO}%
\colorbox{green}{\color[gray]{0.75}FO}%
\colorbox{green}{\color[gray]{0.75}FO}%
\colorbox{green}{\color[gray]{0.75}FO}%
\colorbox{green}{\color[gray]{0.75}FO}%
\colorbox{green}{\color[gray]{0.75}FO}%
\colorbox{green}{\color[gray]{0.75}FO}%
\colorbox{green}{\color[gray]{0.75}FO}%
\colorbox{green}{\color[gray]{0.75}FO}%
\colorbox{green}{\color[gray]{0.75}FO}%
\colorbox{green}{\color[gray]{0.75}FO}%
\colorbox{green}{\color[gray]{0.75}FO}%
\colorbox{green}{\color[gray]{0.75}FO}%
\colorbox{green}{\color[gray]{0.75}FO}%
\colorbox{green}{\color[gray]{0.75}FO}%
\colorbox{green}{\color[gray]{0.75}FO}%
\colorbox{green}{\color[gray]{0.75}FO}%
\colorbox{green}{\color[gray]{0.75}FO}%
\colorbox{green}{\color[gray]{0.75}FO}%
\colorbox{green}{\color[gray]{0.75}FO}%
\colorbox{green}{\color[gray]{0.75}FO}%
\colorbox{green}{\color[gray]{0.75}FO}%
\colorbox{green}{\color[gray]{0.75}FO}%
\colorbox{green}{\color[gray]{0.75}FO}%
\colorbox{green}{\color[gray]{0.75}FO}%
\colorbox{green}{\color[gray]{0.75}FO}%
\colorbox{green}{\color[gray]{0.75}FO}%
\colorbox{green}{\color[gray]{0.75}FO}%
\colorbox{green}{\color[gray]{0.75}FO}%
\colorbox{green}{\color[gray]{0.75}FO}%
\colorbox{green}{\color[gray]{0.75}FO}%
\colorbox{green}{\color[gray]{0.75}FO}%
\colorbox{green}{\color[gray]{0.75}FO}%
\colorbox{green}{\color[gray]{0.75}FO}%
\colorbox{green}{\color[gray]{0.75}FO}%
\colorbox{green}{\color[gray]{0.75}FO}%
\colorbox{green}{\color[gray]{0.75}FO}%
\colorbox{green}{\color[gray]{0.75}FO}%
\colorbox{green}{\color[gray]{0.75}FO}%
\colorbox{green}{\color[gray]{0.75}FO}%
\\
\colorbox{green}{\color[gray]{0.75}FO}%
\colorbox{green}{\color[gray]{0.75}FO}%
\colorbox{green}{\color[gray]{0.75}FO}%
\colorbox{green}{\color[gray]{0.75}FO}%
\colorbox{green}{\color[gray]{0.75}FO}%
\colorbox{green}{\color[gray]{0.75}FO}%
\colorbox{green}{\color[gray]{0.75}FO}%
\colorbox{green}{\color[gray]{0.75}FO}%
\colorbox{green}{\color[gray]{0.75}FO}%
\colorbox{green}{\color[gray]{0.75}FO}%
\colorbox{green}{\color[gray]{0.75}FO}%
\colorbox{green}{\color[gray]{0.75}FO}%
\colorbox{green}{\color[gray]{0.75}FO}%
\colorbox{green}{\color[gray]{0.75}FO}%
\colorbox{green}{\color[gray]{0.75}FO}%
\colorbox{green}{\color[gray]{0.75}FO}%
\colorbox{green}{\color[gray]{0.75}FO}%
\colorbox{green}{\color[gray]{0.75}FO}%
\colorbox{green}{\color[gray]{0.75}FO}%
\colorbox{green}{\color[gray]{0.75}FO}%
\colorbox{green}{\color[gray]{0.75}FO}%
\colorbox{green}{\color[gray]{0.75}FO}%
\colorbox{green}{\color[gray]{0.75}FO}%
\colorbox{green}{\color[gray]{0.75}FO}%
\colorbox{green}{\color[gray]{0.75}FO}%
\colorbox{green}{\color[gray]{0.75}FO}%
\colorbox{green}{\color[gray]{0.75}FO}%
\colorbox{green}{\color[gray]{0.75}FO}%
\colorbox{green}{\color[gray]{0.75}FO}%
\colorbox{green}{\color[gray]{0.75}FO}%
\colorbox{green}{\color[gray]{0.75}FO}%
\colorbox{green}{\color[gray]{0.75}FO}%
\colorbox{green}{\color[gray]{0.75}FO}%
\colorbox{green}{\color[gray]{0.75}FO}%
\colorbox{green}{\color[gray]{0.75}FO}%
\colorbox{green}{\color[gray]{0.75}FO}%
\colorbox{green}{\color[gray]{0.75}FO}%
\colorbox{green}{\color[gray]{0.75}FO}%
\colorbox{green}{\color[gray]{0.75}FO}%
\colorbox{green}{\color[gray]{0.75}FO}%
\colorbox{green}{\color[gray]{0.75}FO}%
\colorbox{green}{\color[gray]{0.75}FO}%
\colorbox{green}{\color[gray]{0.75}FO}%
\colorbox{green}{\color[gray]{0.75}FO}%
\colorbox{green}{\color[gray]{0.75}FO}%
\colorbox{green}{\color[gray]{0.75}FO}%
\colorbox{green}{\color[gray]{0.75}FO}%
\colorbox{green}{\color[gray]{0.75}FO}%
\colorbox{green}{\color[gray]{0.75}FO}%
\colorbox{green}{\color[gray]{0.75}FO}%
\colorbox{green}{\color[gray]{0.75}FO}%
\colorbox{green}{\color[gray]{0.75}FO}%
\colorbox{green}{\color[gray]{0.75}FO}%
\colorbox{green}{\color[gray]{0.75}FO}%
\colorbox{green}{\color[gray]{0.75}FO}%
\colorbox{green}{\color[gray]{0.75}FO}%
\colorbox{green}{\color[gray]{0.75}FO}%
\colorbox{green}{\color[gray]{0.75}FO}%
\colorbox{green}{\color[gray]{0.75}FO}%
\colorbox{green}{\color[gray]{0.75}FO}%
\colorbox{green}{\color[gray]{0.75}FO}%
\colorbox{green}{\color[gray]{0.75}FO}%
\colorbox{green}{\color[gray]{0.75}FO}%
\colorbox{green}{\color[gray]{0.75}FO}%
\colorbox{green}{\color[gray]{0.75}FO}%
\colorbox{green}{\color[gray]{0.75}FO}%
\colorbox{green}{\color[gray]{0.75}FO}%
\colorbox{green}{\color[gray]{0.75}FO}%
\colorbox{green}{\color[gray]{0.75}FO}%
\colorbox{green}{\color[gray]{0.75}FO}%
\colorbox{green}{\color[gray]{0.75}FO}%
\colorbox{green}{\color[gray]{0.75}FO}%
\colorbox{green}{\color[gray]{0.75}FO}%
\colorbox{green}{\color[gray]{0.75}FO}%
\colorbox{green}{\color[gray]{0.75}FO}%
\colorbox{green}{\color[gray]{0.75}FO}%
\colorbox{green}{\color[gray]{0.75}FO}%
\colorbox{green}{\color[gray]{0.75}FO}%
\colorbox{green}{\color[gray]{0.75}FO}%
\colorbox{green}{\color[gray]{0.75}FO}%
\colorbox{green}{\color[gray]{0.75}FO}%
\colorbox{green}{\color[gray]{0.75}FO}%
\colorbox{green}{\color[gray]{0.75}FO}%
\colorbox{green}{\color[gray]{0.75}FO}%
\colorbox{green}{\color[gray]{0.75}FO}%
\colorbox{green}{\color[gray]{0.75}FO}%
\colorbox{green}{\color[gray]{0.75}FO}%
\colorbox{green}{\color[gray]{0.75}FO}%
\colorbox{green}{\color[gray]{0.75}FO}%
\colorbox{green}{\color[gray]{0.75}FO}%
\colorbox{green}{\color[gray]{0.75}FO}%
\colorbox{green}{\color[gray]{0.75}FO}%
\colorbox{green}{\color[gray]{0.75}FO}%
\colorbox{green}{\color[gray]{0.75}FO}%
\colorbox{green}{\color[gray]{0.75}FO}%
\colorbox{green}{\color[gray]{0.75}FO}%
\colorbox{green}{\color[gray]{0.75}FO}%
\colorbox{green}{\color[gray]{0.75}FO}%
\colorbox{green}{\color[gray]{0.75}FO}%
\colorbox{green}{\color[gray]{0.75}FO}%
\\
\colorbox{green}{\color[gray]{0.75}FO}%
\colorbox{green}{\color[gray]{0.75}FO}%
\colorbox{green}{\color[gray]{0.75}FO}%
\colorbox{green}{\color[gray]{0.75}FO}%
\colorbox{green}{\color[gray]{0.75}FO}%
\colorbox{green}{\color[gray]{0.75}FO}%
\colorbox{green}{\color[gray]{0.75}FO}%
\colorbox{green}{\color[gray]{0.75}FO}%
\colorbox{green}{\color[gray]{0.75}FO}%
\colorbox{green}{\color[gray]{0.75}FO}%
\colorbox{green}{\color[gray]{0.75}FO}%
\colorbox{green}{\color[gray]{0.75}FO}%
\colorbox{green}{\color[gray]{0.75}FO}%
\colorbox{green}{\color[gray]{0.75}FO}%
\colorbox{green}{\color[gray]{0.75}FO}%
\colorbox{green}{\color[gray]{0.75}FO}%
\colorbox{green}{\color[gray]{0.75}FO}%
\colorbox{green}{\color[gray]{0.75}FO}%
\colorbox{green}{\color[gray]{0.75}FO}%
\colorbox{green}{\color[gray]{0.75}FO}%
\colorbox{green}{\color[gray]{0.75}FO}%
\colorbox{green}{\color[gray]{0.75}FO}%
\colorbox{green}{\color[gray]{0.75}FO}%
\colorbox{green}{\color[gray]{0.75}FO}%
\colorbox{green}{\color[gray]{0.75}FO}%
\colorbox{green}{\color[gray]{0.75}FO}%
\colorbox{green}{\color[gray]{0.75}FO}%
\colorbox{green}{\color[gray]{0.75}FO}%
\colorbox{green}{\color[gray]{0.75}FO}%
\colorbox{green}{\color[gray]{0.75}FO}%
\colorbox{green}{\color[gray]{0.75}FO}%
\colorbox{green}{\color[gray]{0.75}FO}%
\colorbox{green}{\color[gray]{0.75}FO}%
\colorbox{green}{\color[gray]{0.75}FO}%
\colorbox{green}{\color[gray]{0.75}FO}%
\colorbox{green}{\color[gray]{0.75}FO}%
\colorbox{green}{\color[gray]{0.75}FO}%
\colorbox{green}{\color[gray]{0.75}FO}%
\colorbox{green}{\color[gray]{0.75}FO}%
\colorbox{green}{\color[gray]{0.75}FO}%
\colorbox{green}{\color[gray]{0.75}FO}%
\colorbox{green}{\color[gray]{0.75}FO}%
\colorbox{green}{\color[gray]{0.75}FO}%
\colorbox{green}{\color[gray]{0.75}FO}%
\colorbox{green}{\color[gray]{0.75}FO}%
\colorbox{green}{\color[gray]{0.75}FO}%
\colorbox{green}{\color[gray]{0.75}FO}%
\colorbox{green}{\color[gray]{0.75}FO}%
\colorbox{green}{\color[gray]{0.75}FO}%
\colorbox{green}{\color[gray]{0.75}FO}%
\colorbox{green}{\color[gray]{0.75}FO}%
\colorbox{green}{\color[gray]{0.75}FO}%
\colorbox{green}{\color[gray]{0.75}FO}%
\colorbox{green}{\color[gray]{0.75}FO}%
\colorbox{green}{\color[gray]{0.75}FO}%
\colorbox{green}{\color[gray]{0.75}FO}%
\colorbox{green}{\color[gray]{0.75}FO}%
\colorbox{green}{\color[gray]{0.75}FO}%
\colorbox{green}{\color[gray]{0.75}FO}%
\colorbox{green}{\color[gray]{0.75}FO}%
\colorbox{green}{\color[gray]{0.75}FO}%
\colorbox{green}{\color[gray]{0.75}FO}%
\colorbox{green}{\color[gray]{0.75}FO}%
\colorbox{green}{\color[gray]{0.75}FO}%
\colorbox{green}{\color[gray]{0.75}FO}%
\colorbox{green}{\color[gray]{0.75}FO}%
\colorbox{green}{\color[gray]{0.75}FO}%
\colorbox{green}{\color[gray]{0.75}FO}%
\colorbox{green}{\color[gray]{0.75}FO}%
\colorbox{green}{\color[gray]{0.75}FO}%
\colorbox{green}{\color[gray]{0.75}FO}%
\colorbox{green}{\color[gray]{0.75}FO}%
\colorbox{green}{\color[gray]{0.75}FO}%
\colorbox{green}{\color[gray]{0.75}FO}%
\colorbox{green}{\color[gray]{0.75}FO}%
\colorbox{green}{\color[gray]{0.75}FO}%
\colorbox{green}{\color[gray]{0.75}FO}%
\colorbox{green}{\color[gray]{0.75}FO}%
\colorbox{green}{\color[gray]{0.75}FO}%
\colorbox{green}{\color[gray]{0.75}FO}%
\colorbox{green}{\color[gray]{0.75}FO}%
\colorbox{green}{\color[gray]{0.75}FO}%
\colorbox{green}{\color[gray]{0.75}FO}%
\colorbox{green}{\color[gray]{0.75}FO}%
\colorbox{green}{\color[gray]{0.75}FO}%
\colorbox{green}{\color[gray]{0.75}FO}%
\colorbox{green}{\color[gray]{0.75}FO}%
\colorbox{green}{\color[gray]{0.75}FO}%
\colorbox{green}{\color[gray]{0.75}FO}%
\colorbox{green}{\color[gray]{0.75}FO}%
\colorbox{green}{\color[gray]{0.75}FO}%
\colorbox{green}{\color[gray]{0.75}FO}%
\colorbox{green}{\color[gray]{0.75}FO}%
\colorbox{green}{\color[gray]{0.75}FO}%
\colorbox{green}{\color[gray]{0.75}FO}%
\colorbox{green}{\color[gray]{0.75}FO}%
\colorbox{green}{\color[gray]{0.75}FO}%
\colorbox{green}{\color[gray]{0.75}FO}%
\colorbox{green}{\color[gray]{0.75}FO}%
\colorbox{green}{\color[gray]{0.75}FO}%
\\
\colorbox{green}{\color[gray]{0.75}FO}%
\colorbox{green}{\color[gray]{0.75}FO}%
\colorbox{green}{\color[gray]{0.75}FO}%
\colorbox{green}{\color[gray]{0.75}FO}%
\colorbox{green}{\color[gray]{0.75}FO}%
\colorbox{green}{\color[gray]{0.75}FO}%
\colorbox{green}{\color[gray]{0.75}FO}%
\colorbox{green}{\color[gray]{0.75}FO}%
\colorbox{green}{\color[gray]{0.75}FO}%
\colorbox{green}{\color[gray]{0.75}FO}%
\colorbox{green}{\color[gray]{0.75}FO}%
\colorbox{green}{\color[gray]{0.75}FO}%
\colorbox{green}{\color[gray]{0.75}FO}%
\colorbox{green}{\color[gray]{0.75}FO}%
\colorbox{green}{\color[gray]{0.75}FO}%
\colorbox{green}{\color[gray]{0.75}FO}%
\colorbox{green}{\color[gray]{0.75}FO}%
\colorbox{green}{\color[gray]{0.75}FO}%
\colorbox{green}{\color[gray]{0.75}FO}%
\colorbox{green}{\color[gray]{0.75}FO}%
\colorbox{green}{\color[gray]{0.75}FO}%
\colorbox{green}{\color[gray]{0.75}FO}%
\colorbox{green}{\color[gray]{0.75}FO}%
\colorbox{green}{\color[gray]{0.75}FO}%
\colorbox{green}{\color[gray]{0.75}FO}%
\colorbox{green}{\color[gray]{0.75}FO}%
\colorbox{green}{\color[gray]{0.75}FO}%
\colorbox{green}{\color[gray]{0.75}FO}%
\colorbox{green}{\color[gray]{0.75}FO}%
\colorbox{green}{\color[gray]{0.75}FO}%
\colorbox{green}{\color[gray]{0.75}FO}%
\colorbox{green}{\color[gray]{0.75}FO}%
\colorbox{green}{\color[gray]{0.75}FO}%
\colorbox{green}{\color[gray]{0.75}FO}%
\colorbox{green}{\color[gray]{0.75}FO}%
\colorbox{green}{\color[gray]{0.75}FO}%
\colorbox{green}{\color[gray]{0.75}FO}%
\colorbox{green}{\color[gray]{0.75}FO}%
\colorbox{green}{\color[gray]{0.75}FO}%
\colorbox{green}{\color[gray]{0.75}FO}%
\colorbox{green}{\color[gray]{0.75}FO}%
\colorbox{green}{\color[gray]{0.75}FO}%
\colorbox{green}{\color[gray]{0.75}FO}%
\colorbox{green}{\color[gray]{0.75}FO}%
\colorbox{green}{\color[gray]{0.75}FO}%
\colorbox{green}{\color[gray]{0.75}FO}%
\colorbox{green}{\color[gray]{0.75}FO}%
\colorbox{green}{\color[gray]{0.75}FO}%
\colorbox{green}{\color[gray]{0.75}FO}%
\colorbox{green}{\color[gray]{0.75}FO}%
\colorbox{green}{\color[gray]{0.75}FO}%
\colorbox{green}{\color[gray]{0.75}FO}%
\colorbox{green}{\color[gray]{0.75}FO}%
\colorbox{green}{\color[gray]{0.75}FO}%
\colorbox{green}{\color[gray]{0.75}FO}%
\colorbox{green}{\color[gray]{0.75}FO}%
\colorbox{green}{\color[gray]{0.75}FO}%
\colorbox{green}{\color[gray]{0.75}FO}%
\colorbox{green}{\color[gray]{0.75}FO}%
\colorbox{green}{\color[gray]{0.75}FO}%
\colorbox{green}{\color[gray]{0.75}FO}%
\colorbox{green}{\color[gray]{0.75}FO}%
\colorbox{green}{\color[gray]{0.75}FO}%
\colorbox{green}{\color[gray]{0.75}FO}%
\colorbox{green}{\color[gray]{0.75}FO}%
\colorbox{green}{\color[gray]{0.75}FO}%
\colorbox{green}{\color[gray]{0.75}FO}%
\colorbox{green}{\color[gray]{0.75}FO}%
\colorbox{green}{\color[gray]{0.75}FO}%
\colorbox{green}{\color[gray]{0.75}FO}%
\colorbox{green}{\color[gray]{0.75}FO}%
\colorbox{green}{\color[gray]{0.75}FO}%
\colorbox{green}{\color[gray]{0.75}FO}%
\colorbox{green}{\color[gray]{0.75}FO}%
\colorbox{green}{\color[gray]{0.75}FO}%
\colorbox{green}{\color[gray]{0.75}FO}%
\colorbox{green}{\color[gray]{0.75}FO}%
\colorbox{green}{\color[gray]{0.75}FO}%
\colorbox{green}{\color[gray]{0.75}FO}%
\colorbox{green}{\color[gray]{0.75}FO}%
\colorbox{green}{\color[gray]{0.75}FO}%
\colorbox{green}{\color[gray]{0.75}FO}%
\colorbox{green}{\color[gray]{0.75}FO}%
\colorbox{green}{\color[gray]{0.75}FO}%
\colorbox{green}{\color[gray]{0.75}FO}%
\colorbox{green}{\color[gray]{0.75}FO}%
\colorbox{green}{\color[gray]{0.75}FO}%
\colorbox{green}{\color[gray]{0.75}FO}%
\colorbox{green}{\color[gray]{0.75}FO}%
\colorbox{green}{\color[gray]{0.75}FO}%
\colorbox{green}{\color[gray]{0.75}FO}%
\colorbox{green}{\color[gray]{0.75}FO}%
\colorbox{green}{\color[gray]{0.75}FO}%
\colorbox{green}{\color[gray]{0.75}FO}%
\colorbox{green}{\color[gray]{0.75}FO}%
\colorbox{green}{\color[gray]{0.75}FO}%
\colorbox{green}{\color[gray]{0.75}FO}%
\colorbox{green}{\color[gray]{0.75}FO}%
\colorbox{green}{\color[gray]{0.75}FO}%
\colorbox{green}{\color[gray]{0.75}FO}%
\\
\colorbox{green}{\color[gray]{0.75}FO}%
\colorbox{green}{\color[gray]{0.75}FO}%
\colorbox{green}{\color[gray]{0.75}FO}%
\colorbox{green}{\color[gray]{0.75}FO}%
\colorbox{green}{\color[gray]{0.75}FO}%
\colorbox{green}{\color[gray]{0.75}FO}%
\colorbox{green}{\color[gray]{0.75}FO}%
\colorbox{green}{\color[gray]{0.75}FO}%
\colorbox{green}{\color[gray]{0.75}FO}%
\colorbox{green}{\color[gray]{0.75}FO}%
\colorbox{green}{\color[gray]{0.75}FO}%
\colorbox{green}{\color[gray]{0.75}FO}%
\colorbox{green}{\color[gray]{0.75}FO}%
\colorbox{green}{\color[gray]{0.75}FO}%
\colorbox{green}{\color[gray]{0.75}FO}%
\colorbox{green}{\color[gray]{0.75}FO}%
\colorbox{green}{\color[gray]{0.75}FO}%
\colorbox{green}{\color[gray]{0.75}FO}%
\colorbox{green}{\color[gray]{0.75}FO}%
\colorbox{green}{\color[gray]{0.75}FO}%
\colorbox{green}{\color[gray]{0.75}FO}%
\colorbox{green}{\color[gray]{0.75}FO}%
\colorbox{green}{\color[gray]{0.75}FO}%
\colorbox{green}{\color[gray]{0.75}FO}%
\colorbox{green}{\color[gray]{0.75}FO}%
\colorbox{green}{\color[gray]{0.75}FO}%
\colorbox{green}{\color[gray]{0.75}FO}%
\colorbox{green}{\color[gray]{0.75}FO}%
\colorbox{green}{\color[gray]{0.75}FO}%
\colorbox{green}{\color[gray]{0.75}FO}%
\colorbox{green}{\color[gray]{0.75}FO}%
\colorbox{green}{\color[gray]{0.75}FO}%
\colorbox{green}{\color[gray]{0.75}FO}%
\colorbox{green}{\color[gray]{0.75}FO}%
\colorbox{green}{\color[gray]{0.75}FO}%
\colorbox{green}{\color[gray]{0.75}FO}%
\colorbox{green}{\color[gray]{0.75}FO}%
\colorbox{green}{\color[gray]{0.75}FO}%
\colorbox{green}{\color[gray]{0.75}FO}%
\colorbox{green}{\color[gray]{0.75}FO}%
\colorbox{green}{\color[gray]{0.75}FO}%
\colorbox{green}{\color[gray]{0.75}FO}%
\colorbox{green}{\color[gray]{0.75}FO}%
\colorbox{green}{\color[gray]{0.75}FO}%
\colorbox{green}{\color[gray]{0.75}FO}%
\colorbox{green}{\color[gray]{0.75}FO}%
\colorbox{green}{\color[gray]{0.75}FO}%
\colorbox{green}{\color[gray]{0.75}FO}%
\colorbox{green}{\color[gray]{0.75}FO}%
\colorbox{green}{\color[gray]{0.75}FO}%
\colorbox{green}{\color[gray]{0.75}FO}%
\colorbox{green}{\color[gray]{0.75}FO}%
\colorbox{green}{\color[gray]{0.75}FO}%
\colorbox{green}{\color[gray]{0.75}FO}%
\colorbox{green}{\color[gray]{0.75}FO}%
\colorbox{green}{\color[gray]{0.75}FO}%
\colorbox{green}{\color[gray]{0.75}FO}%
\colorbox{green}{\color[gray]{0.75}FO}%
\colorbox{green}{\color[gray]{0.75}FO}%
\colorbox{green}{\color[gray]{0.75}FO}%
\colorbox{green}{\color[gray]{0.75}FO}%
\colorbox{green}{\color[gray]{0.75}FO}%
\colorbox{green}{\color[gray]{0.75}FO}%
\colorbox{green}{\color[gray]{0.75}FO}%
\colorbox{green}{\color[gray]{0.75}FO}%
\colorbox{green}{\color[gray]{0.75}FO}%
\colorbox{green}{\color[gray]{0.75}FO}%
\colorbox{green}{\color[gray]{0.75}FO}%
\colorbox{green}{\color[gray]{0.75}FO}%
\colorbox{green}{\color[gray]{0.75}FO}%
\colorbox{green}{\color[gray]{0.75}FO}%
\colorbox{green}{\color[gray]{0.75}FO}%
\colorbox{green}{\color[gray]{0.75}FO}%
\colorbox{green}{\color[gray]{0.75}FO}%
\colorbox{green}{\color[gray]{0.75}FO}%
\colorbox{green}{\color[gray]{0.75}FO}%
\colorbox{green}{\color[gray]{0.75}FO}%
\colorbox{green}{\color[gray]{0.75}FO}%
\colorbox{green}{\color[gray]{0.75}FO}%
\colorbox{green}{\color[gray]{0.75}FO}%
\colorbox{green}{\color[gray]{0.75}FO}%
\colorbox{green}{\color[gray]{0.75}FO}%
\colorbox{green}{\color[gray]{0.75}FO}%
\colorbox{green}{\color[gray]{0.75}FO}%
\colorbox{green}{\color[gray]{0.75}FO}%
\colorbox{green}{\color[gray]{0.75}FO}%
\colorbox{green}{\color[gray]{0.75}FO}%
\colorbox{green}{\color[gray]{0.75}FO}%
\colorbox{green}{\color[gray]{0.75}FO}%
\colorbox{green}{\color[gray]{0.75}FO}%
\colorbox{green}{\color[gray]{0.75}FO}%
\colorbox{green}{\color[gray]{0.75}FO}%
\colorbox{green}{\color[gray]{0.75}FO}%
\colorbox{green}{\color[gray]{0.75}FO}%
\colorbox{green}{\color[gray]{0.75}FO}%
\colorbox{green}{\color[gray]{0.75}FO}%
\colorbox{green}{\color[gray]{0.75}FO}%
\colorbox{green}{\color[gray]{0.75}FO}%
\colorbox{green}{\color[gray]{0.75}FO}%
\colorbox{green}{\color[gray]{0.75}FO}%
\\
\colorbox{green}{\color[gray]{0.75}FO}%
\colorbox{green}{\color[gray]{0.75}FO}%
\colorbox{green}{\color[gray]{0.75}FO}%
\colorbox{green}{\color[gray]{0.75}FO}%
\colorbox{green}{\color[gray]{0.75}FO}%
\colorbox{green}{\color[gray]{0.75}FO}%
\colorbox{green}{\color[gray]{0.75}FO}%
\colorbox{green}{\color[gray]{0.75}FO}%
\colorbox{green}{\color[gray]{0.75}FO}%
\colorbox{green}{\color[gray]{0.75}FO}%
\colorbox{green}{\color[gray]{0.75}FO}%
\colorbox{green}{\color[gray]{0.75}FO}%
\colorbox{green}{\color[gray]{0.75}FO}%
\colorbox{green}{\color[gray]{0.75}FO}%
\colorbox{green}{\color[gray]{0.75}FO}%
\colorbox{green}{\color[gray]{0.75}FO}%
\colorbox{green}{\color[gray]{0.75}FO}%
\colorbox{green}{\color[gray]{0.75}FO}%
\colorbox{green}{\color[gray]{0.75}FO}%
\colorbox{green}{\color[gray]{0.75}FO}%
\colorbox{green}{\color[gray]{0.75}FO}%
\colorbox{green}{\color[gray]{0.75}FO}%
\colorbox{green}{\color[gray]{0.75}FO}%
\colorbox{green}{\color[gray]{0.75}FO}%
\colorbox{green}{\color[gray]{0.75}FO}%
\colorbox{green}{\color[gray]{0.75}FO}%
\colorbox{green}{\color[gray]{0.75}FO}%
\colorbox{green}{\color[gray]{0.75}FO}%
\colorbox{green}{\color[gray]{0.75}FO}%
\colorbox{green}{\color[gray]{0.75}FO}%
\colorbox{green}{\color[gray]{0.75}FO}%
\colorbox{green}{\color[gray]{0.75}FO}%
\colorbox{green}{\color[gray]{0.75}FO}%
\colorbox{green}{\color[gray]{0.75}FO}%
\colorbox{green}{\color[gray]{0.75}FO}%
\colorbox{green}{\color[gray]{0.75}FO}%
\colorbox{green}{\color[gray]{0.75}FO}%
\colorbox{green}{\color[gray]{0.75}FO}%
\colorbox{green}{\color[gray]{0.75}FO}%
\colorbox{green}{\color[gray]{0.75}FO}%
\colorbox{green}{\color[gray]{0.75}FO}%
\colorbox{green}{\color[gray]{0.75}FO}%
\colorbox{green}{\color[gray]{0.75}FO}%
\colorbox{green}{\color[gray]{0.75}FO}%
\colorbox{green}{\color[gray]{0.75}FO}%
\colorbox{green}{\color[gray]{0.75}FO}%
\colorbox{green}{\color[gray]{0.75}FO}%
\colorbox{green}{\color[gray]{0.75}FO}%
\colorbox{green}{\color[gray]{0.75}FO}%
\colorbox{green}{\color[gray]{0.75}FO}%
\colorbox{green}{\color[gray]{0.75}FO}%
\colorbox{green}{\color[gray]{0.75}FO}%
\colorbox{green}{\color[gray]{0.75}FO}%
\colorbox{green}{\color[gray]{0.75}FO}%
\colorbox{green}{\color[gray]{0.75}FO}%
\colorbox{green}{\color[gray]{0.75}FO}%
\colorbox{green}{\color[gray]{0.75}FO}%
\colorbox{green}{\color[gray]{0.75}FO}%
\colorbox{green}{\color[gray]{0.75}FO}%
\colorbox{green}{\color[gray]{0.75}FO}%
\colorbox{green}{\color[gray]{0.75}FO}%
\colorbox{green}{\color[gray]{0.75}FO}%
\colorbox{green}{\color[gray]{0.75}FO}%
\colorbox{green}{\color[gray]{0.75}FO}%
\colorbox{green}{\color[gray]{0.75}FO}%
\colorbox{green}{\color[gray]{0.75}FO}%
\colorbox{green}{\color[gray]{0.75}FO}%
\colorbox{green}{\color[gray]{0.75}FO}%
\colorbox{green}{\color[gray]{0.75}FO}%
\colorbox{green}{\color[gray]{0.75}FO}%
\colorbox{green}{\color[gray]{0.75}FO}%
\colorbox{green}{\color[gray]{0.75}FO}%
\colorbox{green}{\color[gray]{0.75}FO}%
\colorbox{green}{\color[gray]{0.75}FO}%
\colorbox{green}{\color[gray]{0.75}FO}%
\colorbox{green}{\color[gray]{0.75}FO}%
\colorbox{green}{\color[gray]{0.75}FO}%
\colorbox{green}{\color[gray]{0.75}FO}%
\colorbox{green}{\color[gray]{0.75}FO}%
\colorbox{green}{\color[gray]{0.75}FO}%
\colorbox{green}{\color[gray]{0.75}FO}%
\colorbox{green}{\color[gray]{0.75}FO}%
\colorbox{green}{\color[gray]{0.75}FO}%
\colorbox{green}{\color[gray]{0.75}FO}%
\colorbox{green}{\color[gray]{0.75}FO}%
\colorbox{green}{\color[gray]{0.75}FO}%
\colorbox{green}{\color[gray]{0.75}FO}%
\colorbox{green}{\color[gray]{0.75}FO}%
\colorbox{green}{\color[gray]{0.75}FO}%
\colorbox{green}{\color[gray]{0.75}FO}%
\colorbox{green}{\color[gray]{0.75}FO}%
\colorbox{green}{\color[gray]{0.75}FO}%
\colorbox{green}{\color[gray]{0.75}FO}%
\colorbox{green}{\color[gray]{0.75}FO}%
\colorbox{green}{\color[gray]{0.75}FO}%
\colorbox{green}{\color[gray]{0.75}FO}%
\colorbox{green}{\color[gray]{0.75}FO}%
\colorbox{green}{\color[gray]{0.75}FO}%
\colorbox{green}{\color[gray]{0.75}FO}%
\colorbox{green}{\color[gray]{0.75}FO}%
\\
\colorbox{green}{\color[gray]{0.75}FO}%
\colorbox{green}{\color[gray]{0.75}FO}%
\colorbox{green}{\color[gray]{0.75}FO}%
\colorbox{green}{\color[gray]{0.75}FO}%
\colorbox{green}{\color[gray]{0.75}FO}%
\colorbox{green}{\color[gray]{0.75}FO}%
\colorbox{green}{\color[gray]{0.75}FO}%
\colorbox{green}{\color[gray]{0.75}FO}%
\colorbox{green}{\color[gray]{0.75}FO}%
\colorbox{green}{\color[gray]{0.75}FO}%
\colorbox{green}{\color[gray]{0.75}FO}%
\colorbox{green}{\color[gray]{0.75}FO}%
\colorbox{green}{\color[gray]{0.75}FO}%
\colorbox{green}{\color[gray]{0.75}FO}%
\colorbox{green}{\color[gray]{0.75}FO}%
\colorbox{green}{\color[gray]{0.75}FO}%
\colorbox{green}{\color[gray]{0.75}FO}%
\colorbox{green}{\color[gray]{0.75}FO}%
\colorbox{green}{\color[gray]{0.75}FO}%
\colorbox{green}{\color[gray]{0.75}FO}%
\colorbox{green}{\color[gray]{0.75}FO}%
\colorbox{green}{\color[gray]{0.75}FO}%
\colorbox{green}{\color[gray]{0.75}FO}%
\colorbox{green}{\color[gray]{0.75}FO}%
\colorbox{green}{\color[gray]{0.75}FO}%
\colorbox{green}{\color[gray]{0.75}FO}%
\colorbox{green}{\color[gray]{0.75}FO}%
\colorbox{green}{\color[gray]{0.75}FO}%
\colorbox{green}{\color[gray]{0.75}FO}%
\colorbox{green}{\color[gray]{0.75}FO}%
\colorbox{green}{\color[gray]{0.75}FO}%
\colorbox{green}{\color[gray]{0.75}FO}%
\colorbox{green}{\color[gray]{0.75}FO}%
\colorbox{green}{\color[gray]{0.75}FO}%
\colorbox{green}{\color[gray]{0.75}FO}%
\colorbox{green}{\color[gray]{0.75}FO}%
\colorbox{green}{\color[gray]{0.75}FO}%
\colorbox{green}{\color[gray]{0.75}FO}%
\colorbox{green}{\color[gray]{0.75}FO}%
\colorbox{green}{\color[gray]{0.75}FO}%
\colorbox{green}{\color[gray]{0.75}FO}%
\colorbox{green}{\color[gray]{0.75}FO}%
\colorbox{green}{\color[gray]{0.75}FO}%
\colorbox{green}{\color[gray]{0.75}FO}%
\colorbox{green}{\color[gray]{0.75}FO}%
\colorbox{green}{\color[gray]{0.75}FO}%
\colorbox{green}{\color[gray]{0.75}FO}%
\colorbox{green}{\color[gray]{0.75}FO}%
\colorbox{green}{\color[gray]{0.75}FO}%
\colorbox{green}{\color[gray]{0.75}FO}%
\colorbox{green}{\color[gray]{0.75}FO}%
\colorbox{green}{\color[gray]{0.75}FO}%
\colorbox{green}{\color[gray]{0.75}FO}%
\colorbox{green}{\color[gray]{0.75}FO}%
\colorbox{green}{\color[gray]{0.75}FO}%
\colorbox{green}{\color[gray]{0.75}FO}%
\colorbox{green}{\color[gray]{0.75}FO}%
\colorbox{green}{\color[gray]{0.75}FO}%
\colorbox{green}{\color[gray]{0.75}FO}%
\colorbox{green}{\color[gray]{0.75}FO}%
\colorbox{green}{\color[gray]{0.75}FO}%
\colorbox{green}{\color[gray]{0.75}FO}%
\colorbox{green}{\color[gray]{0.75}FO}%
\colorbox{green}{\color[gray]{0.75}FO}%
\colorbox{green}{\color[gray]{0.75}FO}%
\colorbox{green}{\color[gray]{0.75}FO}%
\colorbox{green}{\color[gray]{0.75}FO}%
\colorbox{green}{\color[gray]{0.75}FO}%
\colorbox{green}{\color[gray]{0.75}FO}%
\colorbox{green}{\color[gray]{0.75}FO}%
\colorbox{green}{\color[gray]{0.75}FO}%
\colorbox{green}{\color[gray]{0.75}FO}%
\colorbox{green}{\color[gray]{0.75}FO}%
\colorbox{green}{\color[gray]{0.75}FO}%
\colorbox{green}{\color[gray]{0.75}FO}%
\colorbox{green}{\color[gray]{0.75}FO}%
\colorbox{green}{\color[gray]{0.75}FO}%
\colorbox{green}{\color[gray]{0.75}FO}%
\colorbox{green}{\color[gray]{0.75}FO}%
\colorbox{green}{\color[gray]{0.75}FO}%
\colorbox{green}{\color[gray]{0.75}FO}%
\colorbox{green}{\color[gray]{0.75}FO}%
\colorbox{green}{\color[gray]{0.75}FO}%
\colorbox{green}{\color[gray]{0.75}FO}%
\colorbox{green}{\color[gray]{0.75}FO}%
\colorbox{green}{\color[gray]{0.75}FO}%
\colorbox{green}{\color[gray]{0.75}FO}%
\colorbox{green}{\color[gray]{0.75}FO}%
\colorbox{green}{\color[gray]{0.75}FO}%
\colorbox{green}{\color[gray]{0.75}FO}%
\colorbox{green}{\color[gray]{0.75}FO}%
\colorbox{green}{\color[gray]{0.75}FO}%
\colorbox{green}{\color[gray]{0.75}FO}%
\colorbox{green}{\color[gray]{0.75}FO}%
\colorbox{green}{\color[gray]{0.75}FO}%
\colorbox{green}{\color[gray]{0.75}FO}%
\colorbox{green}{\color[gray]{0.75}FO}%
\colorbox{green}{\color[gray]{0.75}FO}%
\colorbox{green}{\color[gray]{0.75}FO}%
\colorbox{green}{\color[gray]{0.75}FO}%
\\
\colorbox{green}{\color[gray]{0.75}FO}%
\colorbox{green}{\color[gray]{0.75}FO}%
\colorbox{green}{\color[gray]{0.75}FO}%
\colorbox{green}{\color[gray]{0.75}FO}%
\colorbox{green}{\color[gray]{0.75}FO}%
\colorbox{green}{\color[gray]{0.75}FO}%
\colorbox{green}{\color[gray]{0.75}FO}%
\colorbox{green}{\color[gray]{0.75}FO}%
\colorbox{green}{\color[gray]{0.75}FO}%
\colorbox{green}{\color[gray]{0.75}FO}%
\colorbox{green}{\color[gray]{0.75}FO}%
\colorbox{green}{\color[gray]{0.75}FO}%
\colorbox{green}{\color[gray]{0.75}FO}%
\colorbox{green}{\color[gray]{0.75}FO}%
\colorbox{green}{\color[gray]{0.75}FO}%
\colorbox{green}{\color[gray]{0.75}FO}%
\colorbox{green}{\color[gray]{0.75}FO}%
\colorbox{green}{\color[gray]{0.75}FO}%
\colorbox{green}{\color[gray]{0.75}FO}%
\colorbox{green}{\color[gray]{0.75}FO}%
\colorbox{green}{\color[gray]{0.75}FO}%
\colorbox{green}{\color[gray]{0.75}FO}%
\colorbox{green}{\color[gray]{0.75}FO}%
\colorbox{green}{\color[gray]{0.75}FO}%
\colorbox{green}{\color[gray]{0.75}FO}%
\colorbox{green}{\color[gray]{0.75}FO}%
\colorbox{green}{\color[gray]{0.75}FO}%
\colorbox{green}{\color[gray]{0.75}FO}%
\colorbox{green}{\color[gray]{0.75}FO}%
\colorbox{green}{\color[gray]{0.75}FO}%
\colorbox{green}{\color[gray]{0.75}FO}%
\colorbox{green}{\color[gray]{0.75}FO}%
\colorbox{green}{\color[gray]{0.75}FO}%
\colorbox{green}{\color[gray]{0.75}FO}%
\colorbox{green}{\color[gray]{0.75}FO}%
\colorbox{green}{\color[gray]{0.75}FO}%
\colorbox{green}{\color[gray]{0.75}FO}%
\colorbox{green}{\color[gray]{0.75}FO}%
\colorbox{green}{\color[gray]{0.75}FO}%
\colorbox{green}{\color[gray]{0.75}FO}%
\colorbox{green}{\color[gray]{0.75}FO}%
\colorbox{green}{\color[gray]{0.75}FO}%
\colorbox{green}{\color[gray]{0.75}FO}%
\colorbox{green}{\color[gray]{0.75}FO}%
\colorbox{green}{\color[gray]{0.75}FO}%
\colorbox{green}{\color[gray]{0.75}FO}%
\colorbox{green}{\color[gray]{0.75}FO}%
\colorbox{green}{\color[gray]{0.75}FO}%
\colorbox{green}{\color[gray]{0.75}FO}%
\colorbox{green}{\color[gray]{0.75}FO}%
\colorbox{green}{\color[gray]{0.75}FO}%
\colorbox{green}{\color[gray]{0.75}FO}%
\colorbox{green}{\color[gray]{0.75}FO}%
\colorbox{green}{\color[gray]{0.75}FO}%
\colorbox{green}{\color[gray]{0.75}FO}%
\colorbox{green}{\color[gray]{0.75}FO}%
\colorbox{green}{\color[gray]{0.75}FO}%
\colorbox{green}{\color[gray]{0.75}FO}%
\colorbox{green}{\color[gray]{0.75}FO}%
\colorbox{green}{\color[gray]{0.75}FO}%
\colorbox{green}{\color[gray]{0.75}FO}%
\colorbox{green}{\color[gray]{0.75}FO}%
\colorbox{green}{\color[gray]{0.75}FO}%
\colorbox{green}{\color[gray]{0.75}FO}%
\colorbox{green}{\color[gray]{0.75}FO}%
\colorbox{green}{\color[gray]{0.75}FO}%
\colorbox{green}{\color[gray]{0.75}FO}%
\colorbox{green}{\color[gray]{0.75}FO}%
\colorbox{green}{\color[gray]{0.75}FO}%
\colorbox{green}{\color[gray]{0.75}FO}%
\colorbox{green}{\color[gray]{0.75}FO}%
\colorbox{green}{\color[gray]{0.75}FO}%
\colorbox{green}{\color[gray]{0.75}FO}%
\colorbox{green}{\color[gray]{0.75}FO}%
\colorbox{green}{\color[gray]{0.75}FO}%
\colorbox{green}{\color[gray]{0.75}FO}%
\colorbox{green}{\color[gray]{0.75}FO}%
\colorbox{green}{\color[gray]{0.75}FO}%
\colorbox{green}{\color[gray]{0.75}FO}%
\colorbox{green}{\color[gray]{0.75}FO}%
\colorbox{green}{\color[gray]{0.75}FO}%
\colorbox{green}{\color[gray]{0.75}FO}%
\colorbox{green}{\color[gray]{0.75}FO}%
\colorbox{green}{\color[gray]{0.75}FO}%
\colorbox{green}{\color[gray]{0.75}FO}%
\colorbox{green}{\color[gray]{0.75}FO}%
\colorbox{green}{\color[gray]{0.75}FO}%
\colorbox{green}{\color[gray]{0.75}FO}%
\colorbox{green}{\color[gray]{0.75}FO}%
\colorbox{green}{\color[gray]{0.75}FO}%
\colorbox{green}{\color[gray]{0.75}FO}%
\colorbox{green}{\color[gray]{0.75}FO}%
\colorbox{green}{\color[gray]{0.75}FO}%
\colorbox{green}{\color[gray]{0.75}FO}%
\colorbox{green}{\color[gray]{0.75}FO}%
\colorbox{green}{\color[gray]{0.75}FO}%
\colorbox{green}{\color[gray]{0.75}FO}%
\colorbox{green}{\color[gray]{0.75}FO}%
\colorbox{green}{\color[gray]{0.75}FO}%
\colorbox{green}{\color[gray]{0.75}FO}%
\\
\colorbox{green}{\color[gray]{0.75}FO}%
\colorbox{green}{\color[gray]{0.75}FO}%
\colorbox{green}{\color[gray]{0.75}FO}%
\colorbox{green}{\color[gray]{0.75}FO}%
\colorbox{green}{\color[gray]{0.75}FO}%
\colorbox{green}{\color[gray]{0.75}FO}%
\colorbox{green}{\color[gray]{0.75}FO}%
\colorbox{green}{\color[gray]{0.75}FO}%
\colorbox{green}{\color[gray]{0.75}FO}%
\colorbox{green}{\color[gray]{0.75}FO}%
\colorbox{green}{\color[gray]{0.75}FO}%
\colorbox{green}{\color[gray]{0.75}FO}%
\colorbox{green}{\color[gray]{0.75}FO}%
\colorbox{green}{\color[gray]{0.75}FO}%
\colorbox{green}{\color[gray]{0.75}FO}%
\colorbox{green}{\color[gray]{0.75}FO}%
\colorbox{green}{\color[gray]{0.75}FO}%
\colorbox{green}{\color[gray]{0.75}FO}%
\colorbox{green}{\color[gray]{0.75}FO}%
\colorbox{green}{\color[gray]{0.75}FO}%
\colorbox{green}{\color[gray]{0.75}FO}%
\colorbox{green}{\color[gray]{0.75}FO}%
\colorbox{green}{\color[gray]{0.75}FO}%
\colorbox{green}{\color[gray]{0.75}FO}%
\colorbox{green}{\color[gray]{0.75}FO}%
\colorbox{green}{\color[gray]{0.75}FO}%
\colorbox{green}{\color[gray]{0.75}FO}%
\colorbox{green}{\color[gray]{0.75}FO}%
\colorbox{green}{\color[gray]{0.75}FO}%
\colorbox{green}{\color[gray]{0.75}FO}%
\colorbox{green}{\color[gray]{0.75}FO}%
\colorbox{green}{\color[gray]{0.75}FO}%
\colorbox{green}{\color[gray]{0.75}FO}%
\colorbox{green}{\color[gray]{0.75}FO}%
\colorbox{green}{\color[gray]{0.75}FO}%
\colorbox{green}{\color[gray]{0.75}FO}%
\colorbox{green}{\color[gray]{0.75}FO}%
\colorbox{green}{\color[gray]{0.75}FO}%
\colorbox{green}{\color[gray]{0.75}FO}%
\colorbox{green}{\color[gray]{0.75}FO}%
\colorbox{green}{\color[gray]{0.75}FO}%
\colorbox{green}{\color[gray]{0.75}FO}%
\colorbox{green}{\color[gray]{0.75}FO}%
\colorbox{green}{\color[gray]{0.75}FO}%
\colorbox{green}{\color[gray]{0.75}FO}%
\colorbox{green}{\color[gray]{0.75}FO}%
\colorbox{green}{\color[gray]{0.75}FO}%
\colorbox{green}{\color[gray]{0.75}FO}%
\colorbox{green}{\color[gray]{0.75}FO}%
\colorbox{green}{\color[gray]{0.75}FO}%
\colorbox{green}{\color[gray]{0.75}FO}%
\colorbox{green}{\color[gray]{0.75}FO}%
\colorbox{green}{\color[gray]{0.75}FO}%
\colorbox{green}{\color[gray]{0.75}FO}%
\colorbox{green}{\color[gray]{0.75}FO}%
\colorbox{green}{\color[gray]{0.75}FO}%
\colorbox{green}{\color[gray]{0.75}FO}%
\colorbox{green}{\color[gray]{0.75}FO}%
\colorbox{green}{\color[gray]{0.75}FO}%
\colorbox{green}{\color[gray]{0.75}FO}%
\colorbox{green}{\color[gray]{0.75}FO}%
\colorbox{green}{\color[gray]{0.75}FO}%
\colorbox{green}{\color[gray]{0.75}FO}%
\colorbox{green}{\color[gray]{0.75}FO}%
\colorbox{green}{\color[gray]{0.75}FO}%
\colorbox{green}{\color[gray]{0.75}FO}%
\colorbox{green}{\color[gray]{0.75}FO}%
\colorbox{green}{\color[gray]{0.75}FO}%
\colorbox{green}{\color[gray]{0.75}FO}%
\colorbox{green}{\color[gray]{0.75}FO}%
\colorbox{green}{\color[gray]{0.75}FO}%
\colorbox{green}{\color[gray]{0.75}FO}%
\colorbox{green}{\color[gray]{0.75}FO}%
\colorbox{green}{\color[gray]{0.75}FO}%
\colorbox{green}{\color[gray]{0.75}FO}%
\colorbox{green}{\color[gray]{0.75}FO}%
\colorbox{green}{\color[gray]{0.75}FO}%
\colorbox{green}{\color[gray]{0.75}FO}%
\colorbox{green}{\color[gray]{0.75}FO}%
\colorbox{green}{\color[gray]{0.75}FO}%
\colorbox{green}{\color[gray]{0.75}FO}%
\colorbox{green}{\color[gray]{0.75}FO}%
\colorbox{green}{\color[gray]{0.75}FO}%
\colorbox{green}{\color[gray]{0.75}FO}%
\colorbox{green}{\color[gray]{0.75}FO}%
\colorbox{green}{\color[gray]{0.75}FO}%
\colorbox{green}{\color[gray]{0.75}FO}%
\colorbox{green}{\color[gray]{0.75}FO}%
\colorbox{green}{\color[gray]{0.75}FO}%
\colorbox{green}{\color[gray]{0.75}FO}%
\colorbox{green}{\color[gray]{0.75}FO}%
\colorbox{green}{\color[gray]{0.75}FO}%
\colorbox{green}{\color[gray]{0.75}FO}%
\colorbox{green}{\color[gray]{0.75}FO}%
\colorbox{green}{\color[gray]{0.75}FO}%
\colorbox{green}{\color[gray]{0.75}FO}%
\colorbox{green}{\color[gray]{0.75}FO}%
\colorbox{green}{\color[gray]{0.75}FO}%
\colorbox{green}{\color[gray]{0.75}FO}%
\colorbox{green}{\color[gray]{0.75}FO}%
\\
\colorbox{green}{\color[gray]{0.75}FO}%
\colorbox{green}{\color[gray]{0.75}FO}%
\colorbox{green}{\color[gray]{0.75}FO}%
\colorbox{green}{\color[gray]{0.75}FO}%
\colorbox{green}{\color[gray]{0.75}FO}%
\colorbox{green}{\color[gray]{0.75}FO}%
\colorbox{green}{\color[gray]{0.75}FO}%
\colorbox{green}{\color[gray]{0.75}FO}%
\colorbox{green}{\color[gray]{0.75}FO}%
\colorbox{green}{\color[gray]{0.75}FO}%
\colorbox{green}{\color[gray]{0.75}FO}%
\colorbox{green}{\color[gray]{0.75}FO}%
\colorbox{green}{\color[gray]{0.75}FO}%
\colorbox{green}{\color[gray]{0.75}FO}%
\colorbox{green}{\color[gray]{0.75}FO}%
\colorbox{green}{\color[gray]{0.75}FO}%
\colorbox{green}{\color[gray]{0.75}FO}%
\colorbox{green}{\color[gray]{0.75}FO}%
\colorbox{green}{\color[gray]{0.75}FO}%
\colorbox{green}{\color[gray]{0.75}FO}%
\colorbox{green}{\color[gray]{0.75}FO}%
\colorbox{green}{\color[gray]{0.75}FO}%
\colorbox{green}{\color[gray]{0.75}FO}%
\colorbox{green}{\color[gray]{0.75}FO}%
\colorbox{green}{\color[gray]{0.75}FO}%
\colorbox{green}{\color[gray]{0.75}FO}%
\colorbox{green}{\color[gray]{0.75}FO}%
\colorbox{green}{\color[gray]{0.75}FO}%
\colorbox{green}{\color[gray]{0.75}FO}%
\colorbox{green}{\color[gray]{0.75}FO}%
\colorbox{green}{\color[gray]{0.75}FO}%
\colorbox{green}{\color[gray]{0.75}FO}%
\colorbox{green}{\color[gray]{0.75}FO}%
\colorbox{green}{\color[gray]{0.75}FO}%
\colorbox{green}{\color[gray]{0.75}FO}%
\colorbox{green}{\color[gray]{0.75}FO}%
\colorbox{green}{\color[gray]{0.75}FO}%
\colorbox{green}{\color[gray]{0.75}FO}%
\colorbox{green}{\color[gray]{0.75}FO}%
\colorbox{green}{\color[gray]{0.75}FO}%
\colorbox{green}{\color[gray]{0.75}FO}%
\colorbox{green}{\color[gray]{0.75}FO}%
\colorbox{green}{\color[gray]{0.75}FO}%
\colorbox{green}{\color[gray]{0.75}FO}%
\colorbox{green}{\color[gray]{0.75}FO}%
\colorbox{green}{\color[gray]{0.75}FO}%
\colorbox{green}{\color[gray]{0.75}FO}%
\colorbox{green}{\color[gray]{0.75}FO}%
\colorbox{green}{\color[gray]{0.75}FO}%
\colorbox{green}{\color[gray]{0.75}FO}%
\colorbox{green}{\color[gray]{0.75}FO}%
\colorbox{green}{\color[gray]{0.75}FO}%
\colorbox{green}{\color[gray]{0.75}FO}%
\colorbox{green}{\color[gray]{0.75}FO}%
\colorbox{green}{\color[gray]{0.75}FO}%
\colorbox{green}{\color[gray]{0.75}FO}%
\colorbox{green}{\color[gray]{0.75}FO}%
\colorbox{green}{\color[gray]{0.75}FO}%
\colorbox{green}{\color[gray]{0.75}FO}%
\colorbox{green}{\color[gray]{0.75}FO}%
\colorbox{green}{\color[gray]{0.75}FO}%
\colorbox{green}{\color[gray]{0.75}FO}%
\colorbox{green}{\color[gray]{0.75}FO}%
\colorbox{green}{\color[gray]{0.75}FO}%
\colorbox{green}{\color[gray]{0.75}FO}%
\colorbox{green}{\color[gray]{0.75}FO}%
\colorbox{green}{\color[gray]{0.75}FO}%
\colorbox{green}{\color[gray]{0.75}FO}%
\colorbox{green}{\color[gray]{0.75}FO}%
\colorbox{green}{\color[gray]{0.75}FO}%
\colorbox{green}{\color[gray]{0.75}FO}%
\colorbox{green}{\color[gray]{0.75}FO}%
\colorbox{green}{\color[gray]{0.75}FO}%
\colorbox{green}{\color[gray]{0.75}FO}%
\colorbox{green}{\color[gray]{0.75}FO}%
\colorbox{green}{\color[gray]{0.75}FO}%
\colorbox{green}{\color[gray]{0.75}FO}%
\colorbox{green}{\color[gray]{0.75}FO}%
\colorbox{green}{\color[gray]{0.75}FO}%
\colorbox{green}{\color[gray]{0.75}FO}%
\colorbox{green}{\color[gray]{0.75}FO}%
\colorbox{green}{\color[gray]{0.75}FO}%
\colorbox{green}{\color[gray]{0.75}FO}%
\colorbox{green}{\color[gray]{0.75}FO}%
\colorbox{green}{\color[gray]{0.75}FO}%
\colorbox{green}{\color[gray]{0.75}FO}%
\colorbox{green}{\color[gray]{0.75}FO}%
\colorbox{green}{\color[gray]{0.75}FO}%
\colorbox{green}{\color[gray]{0.75}FO}%
\colorbox{green}{\color[gray]{0.75}FO}%
\colorbox{green}{\color[gray]{0.75}FO}%
\colorbox{green}{\color[gray]{0.75}FO}%
\colorbox{green}{\color[gray]{0.75}FO}%
\colorbox{green}{\color[gray]{0.75}FO}%
\colorbox{green}{\color[gray]{0.75}FO}%
\colorbox{green}{\color[gray]{0.75}FO}%
\colorbox{green}{\color[gray]{0.75}FO}%
\colorbox{green}{\color[gray]{0.75}FO}%
\colorbox{green}{\color[gray]{0.75}FO}%
\colorbox{green}{\color[gray]{0.75}FO}%
\\
\colorbox{green}{\color[gray]{0.75}FO}%
\colorbox{green}{\color[gray]{0.75}FO}%
\colorbox{green}{\color[gray]{0.75}FO}%
\colorbox{green}{\color[gray]{0.75}FO}%
\colorbox{green}{\color[gray]{0.75}FO}%
\colorbox{green}{\color[gray]{0.75}FO}%
\colorbox{green}{\color[gray]{0.75}FO}%
\colorbox{green}{\color[gray]{0.75}FO}%
\colorbox{green}{\color[gray]{0.75}FO}%
\colorbox{green}{\color[gray]{0.75}FO}%
\colorbox{green}{\color[gray]{0.75}FO}%
\colorbox{green}{\color[gray]{0.75}FO}%
\colorbox{green}{\color[gray]{0.75}FO}%
\colorbox{green}{\color[gray]{0.75}FO}%
\colorbox{green}{\color[gray]{0.75}FO}%
\colorbox{green}{\color[gray]{0.75}FO}%
\colorbox{green}{\color[gray]{0.75}FO}%
\colorbox{green}{\color[gray]{0.75}FO}%
\colorbox{green}{\color[gray]{0.75}FO}%
\colorbox{green}{\color[gray]{0.75}FO}%
\colorbox{green}{\color[gray]{0.75}FO}%
\colorbox{green}{\color[gray]{0.75}FO}%
\colorbox{green}{\color[gray]{0.75}FO}%
\colorbox{green}{\color[gray]{0.75}FO}%
\colorbox{green}{\color[gray]{0.75}FO}%
\colorbox{green}{\color[gray]{0.75}FO}%
\colorbox{green}{\color[gray]{0.75}FO}%
\colorbox{green}{\color[gray]{0.75}FO}%
\colorbox{green}{\color[gray]{0.75}FO}%
\colorbox{green}{\color[gray]{0.75}FO}%
\colorbox{green}{\color[gray]{0.75}FO}%
\colorbox{green}{\color[gray]{0.75}FO}%
\colorbox{green}{\color[gray]{0.75}FO}%
\colorbox{green}{\color[gray]{0.75}FO}%
\colorbox{green}{\color[gray]{0.75}FO}%
\colorbox{green}{\color[gray]{0.75}FO}%
\colorbox{green}{\color[gray]{0.75}FO}%
\colorbox{green}{\color[gray]{0.75}FO}%
\colorbox{green}{\color[gray]{0.75}FO}%
\colorbox{green}{\color[gray]{0.75}FO}%
\colorbox{green}{\color[gray]{0.75}FO}%
\colorbox{green}{\color[gray]{0.75}FO}%
\colorbox{green}{\color[gray]{0.75}FO}%
\colorbox{green}{\color[gray]{0.75}FO}%
\colorbox{green}{\color[gray]{0.75}FO}%
\colorbox{green}{\color[gray]{0.75}FO}%
\colorbox{green}{\color[gray]{0.75}FO}%
\colorbox{green}{\color[gray]{0.75}FO}%
\colorbox{green}{\color[gray]{0.75}FO}%
\colorbox{green}{\color[gray]{0.75}FO}%
\colorbox{green}{\color[gray]{0.75}FO}%
\colorbox{green}{\color[gray]{0.75}FO}%
\colorbox{green}{\color[gray]{0.75}FO}%
\colorbox{green}{\color[gray]{0.75}FO}%
\colorbox{green}{\color[gray]{0.75}FO}%
\colorbox{green}{\color[gray]{0.75}FO}%
\colorbox{green}{\color[gray]{0.75}FO}%
\colorbox{green}{\color[gray]{0.75}FO}%
\colorbox{green}{\color[gray]{0.75}FO}%
\colorbox{green}{\color[gray]{0.75}FO}%
\colorbox{green}{\color[gray]{0.75}FO}%
\colorbox{green}{\color[gray]{0.75}FO}%
\colorbox{green}{\color[gray]{0.75}FO}%
\colorbox{green}{\color[gray]{0.75}FO}%
\colorbox{green}{\color[gray]{0.75}FO}%
\colorbox{green}{\color[gray]{0.75}FO}%
\colorbox{green}{\color[gray]{0.75}FO}%
\colorbox{green}{\color[gray]{0.75}FO}%
\colorbox{green}{\color[gray]{0.75}FO}%
\colorbox{green}{\color[gray]{0.75}FO}%
\colorbox{green}{\color[gray]{0.75}FO}%
\colorbox{green}{\color[gray]{0.75}FO}%
\colorbox{green}{\color[gray]{0.75}FO}%
\colorbox{green}{\color[gray]{0.75}FO}%
\colorbox{green}{\color[gray]{0.75}FO}%
\colorbox{green}{\color[gray]{0.75}FO}%
\colorbox{green}{\color[gray]{0.75}FO}%
\colorbox{green}{\color[gray]{0.75}FO}%
\colorbox{green}{\color[gray]{0.75}FO}%
\colorbox{green}{\color[gray]{0.75}FO}%
\colorbox{green}{\color[gray]{0.75}FO}%
\colorbox{green}{\color[gray]{0.75}FO}%
\colorbox{green}{\color[gray]{0.75}FO}%
\colorbox{green}{\color[gray]{0.75}FO}%
\colorbox{green}{\color[gray]{0.75}FO}%
\colorbox{green}{\color[gray]{0.75}FO}%
\colorbox{green}{\color[gray]{0.75}FO}%
\colorbox{green}{\color[gray]{0.75}FO}%
\colorbox{green}{\color[gray]{0.75}FO}%
\colorbox{green}{\color[gray]{0.75}FO}%
\colorbox{green}{\color[gray]{0.75}FO}%
\colorbox{green}{\color[gray]{0.75}FO}%
\colorbox{green}{\color[gray]{0.75}FO}%
\colorbox{green}{\color[gray]{0.75}FO}%
\colorbox{green}{\color[gray]{0.75}FO}%
\colorbox{green}{\color[gray]{0.75}FO}%
\colorbox{green}{\color[gray]{0.75}FO}%
\colorbox{green}{\color[gray]{0.75}FO}%
\colorbox{green}{\color[gray]{0.75}FO}%
\colorbox{green}{\color[gray]{0.75}FO}%
\\
\colorbox{green}{\color[gray]{0.75}FO}%
\colorbox{green}{\color[gray]{0.75}FO}%
\colorbox{green}{\color[gray]{0.75}FO}%
\colorbox{green}{\color[gray]{0.75}FO}%
\colorbox{green}{\color[gray]{0.75}FO}%
\colorbox{green}{\color[gray]{0.75}FO}%
\colorbox{green}{\color[gray]{0.75}FO}%
\colorbox{green}{\color[gray]{0.75}FO}%
\colorbox{green}{\color[gray]{0.75}FO}%
\colorbox{green}{\color[gray]{0.75}FO}%
\colorbox{green}{\color[gray]{0.75}FO}%
\colorbox{green}{\color[gray]{0.75}FO}%
\colorbox{green}{\color[gray]{0.75}FO}%
\colorbox{green}{\color[gray]{0.75}FO}%
\colorbox{green}{\color[gray]{0.75}FO}%
\colorbox{green}{\color[gray]{0.75}FO}%
\colorbox{green}{\color[gray]{0.75}FO}%
\colorbox{green}{\color[gray]{0.75}FO}%
\colorbox{green}{\color[gray]{0.75}FO}%
\colorbox{green}{\color[gray]{0.75}FO}%
\colorbox{green}{\color[gray]{0.75}FO}%
\colorbox{green}{\color[gray]{0.75}FO}%
\colorbox{green}{\color[gray]{0.75}FO}%
\colorbox{green}{\color[gray]{0.75}FO}%
\colorbox{green}{\color[gray]{0.75}FO}%
\colorbox{green}{\color[gray]{0.75}FO}%
\colorbox{green}{\color[gray]{0.75}FO}%
\colorbox{green}{\color[gray]{0.75}FO}%
\colorbox{green}{\color[gray]{0.75}FO}%
\colorbox{green}{\color[gray]{0.75}FO}%
\colorbox{green}{\color[gray]{0.75}FO}%
\colorbox{green}{\color[gray]{0.75}FO}%
\colorbox{green}{\color[gray]{0.75}FO}%
\colorbox{green}{\color[gray]{0.75}FO}%
\colorbox{green}{\color[gray]{0.75}FO}%
\colorbox{green}{\color[gray]{0.75}FO}%
\colorbox{green}{\color[gray]{0.75}FO}%
\colorbox{green}{\color[gray]{0.75}FO}%
\colorbox{green}{\color[gray]{0.75}FO}%
\colorbox{green}{\color[gray]{0.75}FO}%
\colorbox{green}{\color[gray]{0.75}FO}%
\colorbox{green}{\color[gray]{0.75}FO}%
\colorbox{green}{\color[gray]{0.75}FO}%
\colorbox{green}{\color[gray]{0.75}FO}%
\colorbox{green}{\color[gray]{0.75}FO}%
\colorbox{green}{\color[gray]{0.75}FO}%
\colorbox{green}{\color[gray]{0.75}FO}%
\colorbox{green}{\color[gray]{0.75}FO}%
\colorbox{green}{\color[gray]{0.75}FO}%
\colorbox{green}{\color[gray]{0.75}FO}%
\colorbox{green}{\color[gray]{0.75}FO}%
\colorbox{green}{\color[gray]{0.75}FO}%
\colorbox{green}{\color[gray]{0.75}FO}%
\colorbox{green}{\color[gray]{0.75}FO}%
\colorbox{green}{\color[gray]{0.75}FO}%
\colorbox{green}{\color[gray]{0.75}FO}%
\colorbox{green}{\color[gray]{0.75}FO}%
\colorbox{green}{\color[gray]{0.75}FO}%
\colorbox{green}{\color[gray]{0.75}FO}%
\colorbox{green}{\color[gray]{0.75}FO}%
\colorbox{green}{\color[gray]{0.75}FO}%
\colorbox{green}{\color[gray]{0.75}FO}%
\colorbox{green}{\color[gray]{0.75}FO}%
\colorbox{green}{\color[gray]{0.75}FO}%
\colorbox{green}{\color[gray]{0.75}FO}%
\colorbox{green}{\color[gray]{0.75}FO}%
\colorbox{green}{\color[gray]{0.75}FO}%
\colorbox{green}{\color[gray]{0.75}FO}%
\colorbox{green}{\color[gray]{0.75}FO}%
\colorbox{green}{\color[gray]{0.75}FO}%
\colorbox{green}{\color[gray]{0.75}FO}%
\colorbox{green}{\color[gray]{0.75}FO}%
\colorbox{green}{\color[gray]{0.75}FO}%
\colorbox{green}{\color[gray]{0.75}FO}%
\colorbox{green}{\color[gray]{0.75}FO}%
\colorbox{green}{\color[gray]{0.75}FO}%
\colorbox{green}{\color[gray]{0.75}FO}%
\colorbox{green}{\color[gray]{0.75}FO}%
\colorbox{green}{\color[gray]{0.75}FO}%
\colorbox{green}{\color[gray]{0.75}FO}%
\colorbox{green}{\color[gray]{0.75}FO}%
\colorbox{green}{\color[gray]{0.75}FO}%
\colorbox{green}{\color[gray]{0.75}FO}%
\colorbox{green}{\color[gray]{0.75}FO}%
\colorbox{green}{\color[gray]{0.75}FO}%
\colorbox{green}{\color[gray]{0.75}FO}%
\colorbox{green}{\color[gray]{0.75}FO}%
\colorbox{green}{\color[gray]{0.75}FO}%
\colorbox{green}{\color[gray]{0.75}FO}%
\colorbox{green}{\color[gray]{0.75}FO}%
\colorbox{green}{\color[gray]{0.75}FO}%
\colorbox{green}{\color[gray]{0.75}FO}%
\colorbox{green}{\color[gray]{0.75}FO}%
\colorbox{green}{\color[gray]{0.75}FO}%
\colorbox{green}{\color[gray]{0.75}FO}%
\colorbox{green}{\color[gray]{0.75}FO}%
\colorbox{green}{\color[gray]{0.75}FO}%
\colorbox{green}{\color[gray]{0.75}FO}%
\colorbox{green}{\color[gray]{0.75}FO}%
\colorbox{green}{\color[gray]{0.75}FO}%
\\
\colorbox{green}{\color[gray]{0.75}FO}%
\colorbox{green}{\color[gray]{0.75}FO}%
\colorbox{green}{\color[gray]{0.75}FO}%
\colorbox{green}{\color[gray]{0.75}FO}%
\colorbox{green}{\color[gray]{0.75}FO}%
\colorbox{green}{\color[gray]{0.75}FO}%
\colorbox{green}{\color[gray]{0.75}FO}%
\colorbox{green}{\color[gray]{0.75}FO}%
\colorbox{green}{\color[gray]{0.75}FO}%
\colorbox{green}{\color[gray]{0.75}FO}%
\colorbox{green}{\color[gray]{0.75}FO}%
\colorbox{green}{\color[gray]{0.75}FO}%
\colorbox{green}{\color[gray]{0.75}FO}%
\colorbox{green}{\color[gray]{0.75}FO}%
\colorbox{green}{\color[gray]{0.75}FO}%
\colorbox{green}{\color[gray]{0.75}FO}%
\colorbox{green}{\color[gray]{0.75}FO}%
\colorbox{green}{\color[gray]{0.75}FO}%
\colorbox{green}{\color[gray]{0.75}FO}%
\colorbox{green}{\color[gray]{0.75}FO}%
\colorbox{green}{\color[gray]{0.75}FO}%
\colorbox{green}{\color[gray]{0.75}FO}%
\colorbox{green}{\color[gray]{0.75}FO}%
\colorbox{green}{\color[gray]{0.75}FO}%
\colorbox{green}{\color[gray]{0.75}FO}%
\colorbox{green}{\color[gray]{0.75}FO}%
\colorbox{green}{\color[gray]{0.75}FO}%
\colorbox{green}{\color[gray]{0.75}FO}%
\colorbox{green}{\color[gray]{0.75}FO}%
\colorbox{green}{\color[gray]{0.75}FO}%
\colorbox{green}{\color[gray]{0.75}FO}%
\colorbox{green}{\color[gray]{0.75}FO}%
\colorbox{green}{\color[gray]{0.75}FO}%
\colorbox{green}{\color[gray]{0.75}FO}%
\colorbox{green}{\color[gray]{0.75}FO}%
\colorbox{green}{\color[gray]{0.75}FO}%
\colorbox{green}{\color[gray]{0.75}FO}%
\colorbox{green}{\color[gray]{0.75}FO}%
\colorbox{green}{\color[gray]{0.75}FO}%
\colorbox{green}{\color[gray]{0.75}FO}%
\colorbox{green}{\color[gray]{0.75}FO}%
\colorbox{green}{\color[gray]{0.75}FO}%
\colorbox{green}{\color[gray]{0.75}FO}%
\colorbox{green}{\color[gray]{0.75}FO}%
\colorbox{green}{\color[gray]{0.75}FO}%
\colorbox{green}{\color[gray]{0.75}FO}%
\colorbox{green}{\color[gray]{0.75}FO}%
\colorbox{green}{\color[gray]{0.75}FO}%
\colorbox{green}{\color[gray]{0.75}FO}%
\colorbox{green}{\color[gray]{0.75}FO}%
\colorbox{green}{\color[gray]{0.75}FO}%
\colorbox{green}{\color[gray]{0.75}FO}%
\colorbox{green}{\color[gray]{0.75}FO}%
\colorbox{green}{\color[gray]{0.75}FO}%
\colorbox{green}{\color[gray]{0.75}FO}%
\colorbox{green}{\color[gray]{0.75}FO}%
\colorbox{green}{\color[gray]{0.75}FO}%
\colorbox{green}{\color[gray]{0.75}FO}%
\colorbox{green}{\color[gray]{0.75}FO}%
\colorbox{green}{\color[gray]{0.75}FO}%
\colorbox{green}{\color[gray]{0.75}FO}%
\colorbox{green}{\color[gray]{0.75}FO}%
\colorbox{green}{\color[gray]{0.75}FO}%
\colorbox{green}{\color[gray]{0.75}FO}%
\colorbox{green}{\color[gray]{0.75}FO}%
\colorbox{green}{\color[gray]{0.75}FO}%
\colorbox{green}{\color[gray]{0.75}FO}%
\colorbox{green}{\color[gray]{0.75}FO}%
\colorbox{green}{\color[gray]{0.75}FO}%
\colorbox{green}{\color[gray]{0.75}FO}%
\colorbox{green}{\color[gray]{0.75}FO}%
\colorbox{green}{\color[gray]{0.75}FO}%
\colorbox{green}{\color[gray]{0.75}FO}%
\colorbox{green}{\color[gray]{0.75}FO}%
\colorbox{green}{\color[gray]{0.75}FO}%
\colorbox{green}{\color[gray]{0.75}FO}%
\colorbox{green}{\color[gray]{0.75}FO}%
\colorbox{green}{\color[gray]{0.75}FO}%
\colorbox{green}{\color[gray]{0.75}FO}%
\colorbox{green}{\color[gray]{0.75}FO}%
\colorbox{green}{\color[gray]{0.75}FO}%
\colorbox{green}{\color[gray]{0.75}FO}%
\colorbox{green}{\color[gray]{0.75}FO}%
\colorbox{green}{\color[gray]{0.75}FO}%
\colorbox{green}{\color[gray]{0.75}FO}%
\colorbox{green}{\color[gray]{0.75}FO}%
\colorbox{green}{\color[gray]{0.75}FO}%
\colorbox{green}{\color[gray]{0.75}FO}%
\colorbox{green}{\color[gray]{0.75}FO}%
\colorbox{green}{\color[gray]{0.75}FO}%
\colorbox{green}{\color[gray]{0.75}FO}%
\colorbox{green}{\color[gray]{0.75}FO}%
\colorbox{green}{\color[gray]{0.75}FO}%
\colorbox{green}{\color[gray]{0.75}FO}%
\colorbox{green}{\color[gray]{0.75}FO}%
\colorbox{green}{\color[gray]{0.75}FO}%
\colorbox{green}{\color[gray]{0.75}FO}%
\colorbox{green}{\color[gray]{0.75}FO}%
\colorbox{green}{\color[gray]{0.75}FO}%
\colorbox{green}{\color[gray]{0.75}FO}%
\\
\colorbox{green}{\color[gray]{0.75}FO}%
\colorbox{green}{\color[gray]{0.75}FO}%
\colorbox{green}{\color[gray]{0.75}FO}%
\colorbox{green}{\color[gray]{0.75}FO}%
\colorbox{green}{\color[gray]{0.75}FO}%
\colorbox{green}{\color[gray]{0.75}FO}%
\colorbox{green}{\color[gray]{0.75}FO}%
\colorbox{green}{\color[gray]{0.75}FO}%
\colorbox{green}{\color[gray]{0.75}FO}%
\colorbox{green}{\color[gray]{0.75}FO}%
\colorbox{green}{\color[gray]{0.75}FO}%
\colorbox{green}{\color[gray]{0.75}FO}%
\colorbox{green}{\color[gray]{0.75}FO}%
\colorbox{green}{\color[gray]{0.75}FO}%
\colorbox{green}{\color[gray]{0.75}FO}%
\colorbox{green}{\color[gray]{0.75}FO}%
\colorbox{green}{\color[gray]{0.75}FO}%
\colorbox{green}{\color[gray]{0.75}FO}%
\colorbox{green}{\color[gray]{0.75}FO}%
\colorbox{green}{\color[gray]{0.75}FO}%
\colorbox{green}{\color[gray]{0.75}FO}%
\colorbox{green}{\color[gray]{0.75}FO}%
\colorbox{green}{\color[gray]{0.75}FO}%
\colorbox{green}{\color[gray]{0.75}FO}%
\colorbox{green}{\color[gray]{0.75}FO}%
\co
}

\subsection{Quelltext}
{\small
\lstinputlisting{../Aufgabe_1/Buschfeuer.cpp}
}